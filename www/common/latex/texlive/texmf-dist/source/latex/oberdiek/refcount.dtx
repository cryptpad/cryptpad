% \iffalse meta-comment
%
% File: refcount.dtx
% Version: 2011/10/16 v3.4
% Info: Data extraction from label references
%
% Copyright (C) 1998, 2000, 2006, 2008, 2010, 2011 by
%    Heiko Oberdiek <heiko.oberdiek at googlemail.com>
%
% This work may be distributed and/or modified under the
% conditions of the LaTeX Project Public License, either
% version 1.3c of this license or (at your option) any later
% version. This version of this license is in
%    http://www.latex-project.org/lppl/lppl-1-3c.txt
% and the latest version of this license is in
%    http://www.latex-project.org/lppl.txt
% and version 1.3 or later is part of all distributions of
% LaTeX version 2005/12/01 or later.
%
% This work has the LPPL maintenance status "maintained".
%
% This Current Maintainer of this work is Heiko Oberdiek.
%
% This work consists of the main source file refcount.dtx
% and the derived files
%    refcount.sty, refcount.pdf, refcount.ins, refcount.drv,
%    refcount-test1.tex, refcount-test2.tex, refcount-test3.tex,
%    refcount-test4.tex, refcount-test5.tex.
%
% Distribution:
%    CTAN:macros/latex/contrib/oberdiek/refcount.dtx
%    CTAN:macros/latex/contrib/oberdiek/refcount.pdf
%
% Unpacking:
%    (a) If refcount.ins is present:
%           tex refcount.ins
%    (b) Without refcount.ins:
%           tex refcount.dtx
%    (c) If you insist on using LaTeX
%           latex \let\install=y% \iffalse meta-comment
%
% File: refcount.dtx
% Version: 2011/10/16 v3.4
% Info: Data extraction from label references
%
% Copyright (C) 1998, 2000, 2006, 2008, 2010, 2011 by
%    Heiko Oberdiek <heiko.oberdiek at googlemail.com>
%
% This work may be distributed and/or modified under the
% conditions of the LaTeX Project Public License, either
% version 1.3c of this license or (at your option) any later
% version. This version of this license is in
%    http://www.latex-project.org/lppl/lppl-1-3c.txt
% and the latest version of this license is in
%    http://www.latex-project.org/lppl.txt
% and version 1.3 or later is part of all distributions of
% LaTeX version 2005/12/01 or later.
%
% This work has the LPPL maintenance status "maintained".
%
% This Current Maintainer of this work is Heiko Oberdiek.
%
% This work consists of the main source file refcount.dtx
% and the derived files
%    refcount.sty, refcount.pdf, refcount.ins, refcount.drv,
%    refcount-test1.tex, refcount-test2.tex, refcount-test3.tex,
%    refcount-test4.tex, refcount-test5.tex.
%
% Distribution:
%    CTAN:macros/latex/contrib/oberdiek/refcount.dtx
%    CTAN:macros/latex/contrib/oberdiek/refcount.pdf
%
% Unpacking:
%    (a) If refcount.ins is present:
%           tex refcount.ins
%    (b) Without refcount.ins:
%           tex refcount.dtx
%    (c) If you insist on using LaTeX
%           latex \let\install=y% \iffalse meta-comment
%
% File: refcount.dtx
% Version: 2011/10/16 v3.4
% Info: Data extraction from label references
%
% Copyright (C) 1998, 2000, 2006, 2008, 2010, 2011 by
%    Heiko Oberdiek <heiko.oberdiek at googlemail.com>
%
% This work may be distributed and/or modified under the
% conditions of the LaTeX Project Public License, either
% version 1.3c of this license or (at your option) any later
% version. This version of this license is in
%    http://www.latex-project.org/lppl/lppl-1-3c.txt
% and the latest version of this license is in
%    http://www.latex-project.org/lppl.txt
% and version 1.3 or later is part of all distributions of
% LaTeX version 2005/12/01 or later.
%
% This work has the LPPL maintenance status "maintained".
%
% This Current Maintainer of this work is Heiko Oberdiek.
%
% This work consists of the main source file refcount.dtx
% and the derived files
%    refcount.sty, refcount.pdf, refcount.ins, refcount.drv,
%    refcount-test1.tex, refcount-test2.tex, refcount-test3.tex,
%    refcount-test4.tex, refcount-test5.tex.
%
% Distribution:
%    CTAN:macros/latex/contrib/oberdiek/refcount.dtx
%    CTAN:macros/latex/contrib/oberdiek/refcount.pdf
%
% Unpacking:
%    (a) If refcount.ins is present:
%           tex refcount.ins
%    (b) Without refcount.ins:
%           tex refcount.dtx
%    (c) If you insist on using LaTeX
%           latex \let\install=y% \iffalse meta-comment
%
% File: refcount.dtx
% Version: 2011/10/16 v3.4
% Info: Data extraction from label references
%
% Copyright (C) 1998, 2000, 2006, 2008, 2010, 2011 by
%    Heiko Oberdiek <heiko.oberdiek at googlemail.com>
%
% This work may be distributed and/or modified under the
% conditions of the LaTeX Project Public License, either
% version 1.3c of this license or (at your option) any later
% version. This version of this license is in
%    http://www.latex-project.org/lppl/lppl-1-3c.txt
% and the latest version of this license is in
%    http://www.latex-project.org/lppl.txt
% and version 1.3 or later is part of all distributions of
% LaTeX version 2005/12/01 or later.
%
% This work has the LPPL maintenance status "maintained".
%
% This Current Maintainer of this work is Heiko Oberdiek.
%
% This work consists of the main source file refcount.dtx
% and the derived files
%    refcount.sty, refcount.pdf, refcount.ins, refcount.drv,
%    refcount-test1.tex, refcount-test2.tex, refcount-test3.tex,
%    refcount-test4.tex, refcount-test5.tex.
%
% Distribution:
%    CTAN:macros/latex/contrib/oberdiek/refcount.dtx
%    CTAN:macros/latex/contrib/oberdiek/refcount.pdf
%
% Unpacking:
%    (a) If refcount.ins is present:
%           tex refcount.ins
%    (b) Without refcount.ins:
%           tex refcount.dtx
%    (c) If you insist on using LaTeX
%           latex \let\install=y\input{refcount.dtx}
%        (quote the arguments according to the demands of your shell)
%
% Documentation:
%    (a) If refcount.drv is present:
%           latex refcount.drv
%    (b) Without refcount.drv:
%           latex refcount.dtx; ...
%    The class ltxdoc loads the configuration file ltxdoc.cfg
%    if available. Here you can specify further options, e.g.
%    use A4 as paper format:
%       \PassOptionsToClass{a4paper}{article}
%
%    Programm calls to get the documentation (example):
%       pdflatex refcount.dtx
%       makeindex -s gind.ist refcount.idx
%       pdflatex refcount.dtx
%       makeindex -s gind.ist refcount.idx
%       pdflatex refcount.dtx
%
% Installation:
%    TDS:tex/latex/oberdiek/refcount.sty
%    TDS:doc/latex/oberdiek/refcount.pdf
%    TDS:doc/latex/oberdiek/test/refcount-test1.tex
%    TDS:doc/latex/oberdiek/test/refcount-test2.tex
%    TDS:doc/latex/oberdiek/test/refcount-test3.tex
%    TDS:doc/latex/oberdiek/test/refcount-test4.tex
%    TDS:doc/latex/oberdiek/test/refcount-test5.tex
%    TDS:source/latex/oberdiek/refcount.dtx
%
%<*ignore>
\begingroup
  \catcode123=1 %
  \catcode125=2 %
  \def\x{LaTeX2e}%
\expandafter\endgroup
\ifcase 0\ifx\install y1\fi\expandafter
         \ifx\csname processbatchFile\endcsname\relax\else1\fi
         \ifx\fmtname\x\else 1\fi\relax
\else\csname fi\endcsname
%</ignore>
%<*install>
\input docstrip.tex
\Msg{************************************************************************}
\Msg{* Installation}
\Msg{* Package: refcount 2011/10/16 v3.4 Data extraction from label references (HO)}
\Msg{************************************************************************}

\keepsilent
\askforoverwritefalse

\let\MetaPrefix\relax
\preamble

This is a generated file.

Project: refcount
Version: 2011/10/16 v3.4

Copyright (C) 1998, 2000, 2006, 2008, 2010, 2011 by
   Heiko Oberdiek <heiko.oberdiek at googlemail.com>

This work may be distributed and/or modified under the
conditions of the LaTeX Project Public License, either
version 1.3c of this license or (at your option) any later
version. This version of this license is in
   http://www.latex-project.org/lppl/lppl-1-3c.txt
and the latest version of this license is in
   http://www.latex-project.org/lppl.txt
and version 1.3 or later is part of all distributions of
LaTeX version 2005/12/01 or later.

This work has the LPPL maintenance status "maintained".

This Current Maintainer of this work is Heiko Oberdiek.

This work consists of the main source file refcount.dtx
and the derived files
   refcount.sty, refcount.pdf, refcount.ins, refcount.drv,
   refcount-test1.tex, refcount-test2.tex, refcount-test3.tex,
   refcount-test4.tex, refcount-test5.tex.

\endpreamble
\let\MetaPrefix\DoubleperCent

\generate{%
  \file{refcount.ins}{\from{refcount.dtx}{install}}%
  \file{refcount.drv}{\from{refcount.dtx}{driver}}%
  \usedir{tex/latex/oberdiek}%
  \file{refcount.sty}{\from{refcount.dtx}{package}}%
  \usedir{doc/latex/oberdiek/test}%
  \file{refcount-test1.tex}{\from{refcount.dtx}{test1}}%
  \file{refcount-test2.tex}{\from{refcount.dtx}{test2}}%
  \file{refcount-test3.tex}{\from{refcount.dtx}{test3}}%
  \file{refcount-test4.tex}{\from{refcount.dtx}{test3,test4}}%
  \file{refcount-test5.tex}{\from{refcount.dtx}{test5}}%
  \nopreamble
  \nopostamble
  \usedir{source/latex/oberdiek/catalogue}%
  \file{refcount.xml}{\from{refcount.dtx}{catalogue}}%
}

\catcode32=13\relax% active space
\let =\space%
\Msg{************************************************************************}
\Msg{*}
\Msg{* To finish the installation you have to move the following}
\Msg{* file into a directory searched by TeX:}
\Msg{*}
\Msg{*     refcount.sty}
\Msg{*}
\Msg{* To produce the documentation run the file `refcount.drv'}
\Msg{* through LaTeX.}
\Msg{*}
\Msg{* Happy TeXing!}
\Msg{*}
\Msg{************************************************************************}

\endbatchfile
%</install>
%<*ignore>
\fi
%</ignore>
%<*driver>
\NeedsTeXFormat{LaTeX2e}
\ProvidesFile{refcount.drv}%
  [2011/10/16 v3.4 Data extraction from label references (HO)]%
\documentclass{ltxdoc}
\usepackage{holtxdoc}[2011/11/22]
\begin{document}
  \DocInput{refcount.dtx}%
\end{document}
%</driver>
% \fi
%
% \CheckSum{1185}
%
% \CharacterTable
%  {Upper-case    \A\B\C\D\E\F\G\H\I\J\K\L\M\N\O\P\Q\R\S\T\U\V\W\X\Y\Z
%   Lower-case    \a\b\c\d\e\f\g\h\i\j\k\l\m\n\o\p\q\r\s\t\u\v\w\x\y\z
%   Digits        \0\1\2\3\4\5\6\7\8\9
%   Exclamation   \!     Double quote  \"     Hash (number) \#
%   Dollar        \$     Percent       \%     Ampersand     \&
%   Acute accent  \'     Left paren    \(     Right paren   \)
%   Asterisk      \*     Plus          \+     Comma         \,
%   Minus         \-     Point         \.     Solidus       \/
%   Colon         \:     Semicolon     \;     Less than     \<
%   Equals        \=     Greater than  \>     Question mark \?
%   Commercial at \@     Left bracket  \[     Backslash     \\
%   Right bracket \]     Circumflex    \^     Underscore    \_
%   Grave accent  \`     Left brace    \{     Vertical bar  \|
%   Right brace   \}     Tilde         \~}
%
% \GetFileInfo{refcount.drv}
%
% \title{The \xpackage{refcount} package}
% \date{2011/10/16 v3.4}
% \author{Heiko Oberdiek\\\xemail{heiko.oberdiek at googlemail.com}}
%
% \maketitle
%
% \begin{abstract}
% References are not numbers, however they often store numerical
% data such as section or page numbers. \cs{ref} or \cs{pageref}
% cannot be used for counter assignments or calculations because
% they are not expandable, generate warnings, or can even be links.
% The package provides expandable macros to extract the data
% from references. Packages \xpackage{hyperref}, \xpackage{nameref},
% \xpackage{titleref}, and \xpackage{babel} are supported.
% \end{abstract}
%
% \tableofcontents
%
% \section{Usage}
%
% \subsection{Setting counters}
%
% The following commands are similar to \LaTeX's
% \cs{setcounter} and \cs{addtocounter},
% but they extract the number value from a reference:
% \begin{quote}
%   \cs{setcounterref}, \cs{addtocounterref}\\
%   \cs{setcounterpageref}, \cs{addtocounterpageref}
% \end{quote}
% They take two arguments:
% \begin{quote}
%    \cs{...counter...ref} |{|\meta{\LaTeX\ counter}|}|
%    |{|\meta{reference}|}|
% \end{quote}
% An undefined references produces the usual LaTeX warning
% and its value is assumed to be zero.
% Example:
% \begin{quote}
%\begin{verbatim}
%\newcounter{ctrA}
%\newcounter{ctrB}
%\refstepcounter{ctrA}\label{ref:A}
%\setcounterref{ctrB}{ref:A}
%\addtocounterpageref{ctrB}{ref:A}
%\end{verbatim}
% \end{quote}
%
% \subsection{Expandable commands}
%
% These commands that can be used in expandible contexts
% (inside calculations, \cs{edef}, \cs{csname}, \cs{write}, \dots):
% \begin{quote}
%   \cs{getrefnumber}, \cs{getpagerefnumber}
% \end{quote}
% They take one argument, the reference:
% \begin{quote}
%   \cs{get...refnumber} |{|\meta{reference}|}|
% \end{quote}
% The default for undefined references can be changed
% with macro \cs{setrefcountdefault}, for example this
% package calls:
% \begin{quote}
%   \cs{setrefcountdefault}|{0}|
% \end{quote}
%
% Since version 2.0 of this package there is a new
% command:
% \begin{quote}
%   \cs{getrefbykeydefault} |{|\meta{reference}|}|
%   |{|\meta{key}|}| |{|\meta{default}|}|
% \end{quote}
% This generalized version allows the extraction
% of further properties of a reference than the
% two standard ones. Thus the following properties
% are supported, if they are available:
% \begin{quote}
% \begin{tabular}{@{}l|l|l@{}}
%    Key & Description & Package\\
% \hline
%   \meta{empty} & same as \cs{ref} & \LaTeX\\
%   |page| & same as \cs{pageref} & \LaTeX\\
%   |title| & section and caption titles & \xpackage{titleref}\\
%   |name| & section and caption titles & \xpackage{nameref}\\
%   |anchor| & anchor name & \xpackage{hyperref}\\
%   |url| & url/file & \xpackage{hyperref}/\xpackage{xr}
% \end{tabular}
% \end{quote}
%
% Since version 3.2 the expandable macros described before
% in this section
% are expandable in exact two expansion steps.
%
% \subsection{Undefined references}
%
% Because warnings and assignments cannot be used in
% expandible contexts, undefined references do not
% produce a warning, their values are assumed to be zero.
% Example:
% \begin{quote}
%\begin{verbatim}
%\label{ref:here}% somewhere
%\refused{ref:here}% see below
%\ifodd\getpagerefnumber{ref:here}%
%  reference is on an odd page
%\else
%  reference is on an even page
%\fi
%\end{verbatim}
% \end{quote}
%
% In case of undefined references the user usually want's
% to be informed. Also \LaTeX\ prints a warning at
% the end of the \LaTeX\ run. To notify \LaTeX\ and
% get a normal warning, just use
% \begin{quote}
%   \cs{refused} |{|\meta{reference}|}|
% \end{quote}
% outside the expanding context. Example, see above.
%
% \subsubsection{Check for undefined references}
%
% In version 3.2 macros were added, that test, whether references
% are defined.
% \begin{declcs}{IfRefUndefinedExpandable} \M{refname} \M{then} \M{else}\\
%   \cs{IfRefUndefinedBabel} \M{refname} \M{then} \M{else}
% \end{declcs}
% If the reference is not available and therefore undefined, then
% argument \meta{then} is executed, otherwise argument \meta{else}
% is called. Macro \cs{IfRefUndefinedExpandable} is expandable,
% but \meta{refname} must not contain babel shorthand characters.
% Macro \cs{IfRefUndefinedBabel} supports shorthand characters of
% babel, but it is not expandable.
%
% \subsection{Notes}
%
% \begin{itemize}
% \item
%   The method of extracting the number in this
%   package also works in cases, where the
%   reference cannot be used directly, because
%   a package such as \xpackage{hyperref} has added
%   extra stuff (hyper link), so that the reference cannot
%   be used as number any more.
% \item
%   If the reference does not contain a number,
%   assignments to a counter will fail of course.
% \end{itemize}
%
%
% \StopEventually{
% }
%
% \section{Implementation}
%
%    \begin{macrocode}
%<*package>
%    \end{macrocode}
%    Reload check, especially if the package is not used with \LaTeX.
%    \begin{macrocode}
\begingroup\catcode61\catcode48\catcode32=10\relax%
  \catcode13=5 % ^^M
  \endlinechar=13 %
  \catcode35=6 % #
  \catcode39=12 % '
  \catcode44=12 % ,
  \catcode45=12 % -
  \catcode46=12 % .
  \catcode58=12 % :
  \catcode64=11 % @
  \catcode123=1 % {
  \catcode125=2 % }
  \expandafter\let\expandafter\x\csname ver@refcount.sty\endcsname
  \ifx\x\relax % plain-TeX, first loading
  \else
    \def\empty{}%
    \ifx\x\empty % LaTeX, first loading,
      % variable is initialized, but \ProvidesPackage not yet seen
    \else
      \expandafter\ifx\csname PackageInfo\endcsname\relax
        \def\x#1#2{%
          \immediate\write-1{Package #1 Info: #2.}%
        }%
      \else
        \def\x#1#2{\PackageInfo{#1}{#2, stopped}}%
      \fi
      \x{refcount}{The package is already loaded}%
      \aftergroup\endinput
    \fi
  \fi
\endgroup%
%    \end{macrocode}
%    Package identification:
%    \begin{macrocode}
\begingroup\catcode61\catcode48\catcode32=10\relax%
  \catcode13=5 % ^^M
  \endlinechar=13 %
  \catcode35=6 % #
  \catcode39=12 % '
  \catcode40=12 % (
  \catcode41=12 % )
  \catcode44=12 % ,
  \catcode45=12 % -
  \catcode46=12 % .
  \catcode47=12 % /
  \catcode58=12 % :
  \catcode64=11 % @
  \catcode91=12 % [
  \catcode93=12 % ]
  \catcode123=1 % {
  \catcode125=2 % }
  \expandafter\ifx\csname ProvidesPackage\endcsname\relax
    \def\x#1#2#3[#4]{\endgroup
      \immediate\write-1{Package: #3 #4}%
      \xdef#1{#4}%
    }%
  \else
    \def\x#1#2[#3]{\endgroup
      #2[{#3}]%
      \ifx#1\@undefined
        \xdef#1{#3}%
      \fi
      \ifx#1\relax
        \xdef#1{#3}%
      \fi
    }%
  \fi
\expandafter\x\csname ver@refcount.sty\endcsname
\ProvidesPackage{refcount}%
  [2011/10/16 v3.4 Data extraction from label references (HO)]%
%    \end{macrocode}
%
%    \begin{macrocode}
\begingroup\catcode61\catcode48\catcode32=10\relax%
  \catcode13=5 % ^^M
  \endlinechar=13 %
  \catcode123=1 % {
  \catcode125=2 % }
  \catcode64=11 % @
  \def\x{\endgroup
    \expandafter\edef\csname rc@AtEnd\endcsname{%
      \endlinechar=\the\endlinechar\relax
      \catcode13=\the\catcode13\relax
      \catcode32=\the\catcode32\relax
      \catcode35=\the\catcode35\relax
      \catcode61=\the\catcode61\relax
      \catcode64=\the\catcode64\relax
      \catcode123=\the\catcode123\relax
      \catcode125=\the\catcode125\relax
    }%
  }%
\x\catcode61\catcode48\catcode32=10\relax%
\catcode13=5 % ^^M
\endlinechar=13 %
\catcode35=6 % #
\catcode64=11 % @
\catcode123=1 % {
\catcode125=2 % }
\def\TMP@EnsureCode#1#2{%
  \edef\rc@AtEnd{%
    \rc@AtEnd
    \catcode#1=\the\catcode#1\relax
  }%
  \catcode#1=#2\relax
}
\TMP@EnsureCode{33}{12}% !
\TMP@EnsureCode{39}{12}% '
\TMP@EnsureCode{42}{12}% *
\TMP@EnsureCode{45}{12}% -
\TMP@EnsureCode{46}{12}% .
\TMP@EnsureCode{47}{12}% /
\TMP@EnsureCode{91}{12}% [
\TMP@EnsureCode{93}{12}% ]
\TMP@EnsureCode{96}{12}% `
\edef\rc@AtEnd{\rc@AtEnd\noexpand\endinput}
%    \end{macrocode}
%
% \subsection{Loading packages}
%
%    \begin{macrocode}
\begingroup\expandafter\expandafter\expandafter\endgroup
\expandafter\ifx\csname RequirePackage\endcsname\relax
  \input ltxcmds.sty\relax
  \input infwarerr.sty\relax
\else
  \RequirePackage{ltxcmds}[2011/11/09]%
  \RequirePackage{infwarerr}[2010/04/08]%
\fi
%    \end{macrocode}
%
% \subsection{Defining commands}
%
%    \begin{macro}{\rc@IfDefinable}
%    \begin{macrocode}
\ltx@IfUndefined{@ifdefinable}{%
  \def\rc@IfDefinable#1{%
    \ifx#1\ltx@undefined
      \expandafter\ltx@firstofone
    \else
      \ifx#1\relax
        \expandafter\expandafter\expandafter\ltx@firstofone
      \else
        \@PackageError{refcount}{%
          Command \string#1 is already defined.\MessageBreak
          It will not redefined by this package%
        }\@ehc
        \expandafter\expandafter\expandafter\ltx@gobble
      \fi
    \fi
  }%
}{%
  \let\rc@IfDefinable\@ifdefinable
}
%    \end{macrocode}
%    \end{macro}
%
%    \begin{macro}{\rc@RobustDefOne}
%    \begin{macro}{\rc@RobustDefZero}
%    \begin{macrocode}
\ltx@IfUndefined{protected}{%
  \ltx@IfUndefined{DeclareRobustCommand}{%
    \def\rc@RobustDefOne#1#2#3#4{%
      \rc@IfDefinable#3{%
        #1\def#3##1{#4}%
      }%
    }%
    \def\rc@RobustDefZero#1#2{%
      \rc@IfDefinable#1{%
        \def#1{#2}%
      }%
    }%
  }{%
    \def\rc@RobustDefOne#1#2#3#4{%
      \rc@IfDefinable#3{%
        \DeclareRobustCommand#2#3[1]{#4}%
      }%
    }%
    \def\rc@RobustDefZero#1#2{%
      \rc@IfDefinable#1{%
        \DeclareRobustCommand#1{#2}%
      }%
    }%
  }%
}{%
  \def\rc@RobustDefOne#1#2#3#4{%
    \rc@IfDefinable#3{%
      \protected#1\def#3##1{#4}%
    }%
  }%
  \def\rc@RobustDefZero#1#2{%
    \rc@IfDefinable#1{%
      \protected\def#1{#2}%
    }%
  }%
}
%    \end{macrocode}
%    \end{macro}
%    \end{macro}
%
%    \begin{macro}{\rc@newcommand}
%    \begin{macrocode}
\ltx@IfUndefined{newcommand}{%
  \def\rc@newcommand*#1[#2]#3{% hash-ok
    \rc@IfDefinable#1{%
      \ifcase#2 %
        \def#1{#3}%
      \or
        \def#1##1{#3}%
      \or
        \def#1##1##2{#3}%
      \else
        \rc@InternalError
      \fi
    }%
  }%
}{%
  \let\rc@newcommand\newcommand
}
%    \end{macrocode}
%    \end{macro}
%
% \subsection{\cs{setrefcountdefault}}
%
%    \begin{macro}{\setrefcountdefault}
%    \begin{macrocode}
\rc@RobustDefOne\long{}\setrefcountdefault{%
  \def\rc@default{#1}%
}
%    \end{macrocode}
%    \end{macro}
%    \begin{macrocode}
\setrefcountdefault{0}
%    \end{macrocode}
%
% \subsection{\cs{refused}}
%
%    \begin{macro}{\refused}
%    \begin{macrocode}
\ltx@IfUndefined{G@refundefinedtrue}{%
  \rc@RobustDefOne{}{*}\refused{%
    \begingroup
      \csname @safe@activestrue\endcsname
      \ltx@IfUndefined{r@#1}{%
        \protect\G@refundefinedtrue
        \rc@WarningUndefined{#1}%
      }{}%
    \endgroup
  }%
}{%
  \rc@RobustDefOne{}{*}\refused{%
    \begingroup
      \csname @safe@activestrue\endcsname
      \ltx@IfUndefined{r@#1}{%
        \csname protect\expandafter\endcsname
        \csname G@refundefinedtrue\endcsname
        \rc@WarningUndefined{#1}%
      }{}%
    \endgroup
  }%
}
%    \end{macrocode}
%    \end{macro}
%    \begin{macro}{\rc@WarningUndefined}
%    \begin{macrocode}
\ltx@IfUndefined{@latex@warning}{%
  \def\rc@WarningUndefined#1{%
    \ltx@ifundefined{thepage}{%
      \def\thepage{\number\count0 }%
    }{}%
    \@PackageWarning{refcount}{%
      Reference `#1' on page \thepage\space undefined%
    }%
  }%
}{%
  \def\rc@WarningUndefined#1{%
    \@latex@warning{%
      Reference `#1' on page \thepage\space undefined%
    }%
  }%
}
%    \end{macrocode}
%    \end{macro}
%
% \subsection{Setting counters by reference data}
%
% \subsubsection{Generic setting}
%
%    \begin{macro}{\rc@set}
% Generic command for
% |\|$\{$|set|$,$|addto|$\}$|counter|$\{$|page|$,\}$|ref|:
%\begin{quote}
%\begin{tabular}[t]{@{}l@{: }l@{}}
% |#1|& \cs{setcounter}, \cs{addtocounter}\\
% |#2|& \cs{ltx@car} (for \cs{ref}), \cs{ltx@cartwo} (for \cs{pageref})\\
% |#3|& \hologo{LaTeX} counter\\
% |#4|& reference\\
%\end{tabular}
%\end{quote}
%    \begin{macrocode}
\def\rc@set#1#2#3#4{%
  \begingroup
    \csname @safe@activestrue\endcsname
    \refused{#4}%
    \expandafter\rc@@set\csname r@#4\endcsname{#1}{#2}{#3}%
  \endgroup
}
%    \end{macrocode}
%    \end{macro}
%    \begin{macro}{\rc@@set}
%\begin{quote}
%\begin{tabular}[t]{@{}l@{: }l@{}}
% |#1|& \cs{r@<...>}\\
% |#2|& \cs{setcounter}, \cs{addtocounter}\\
% |#3|& \cs{ltx@car} (for \cs{ref}), \cs{ltx@carsecond} (for \cs{pageref})\\
% |#4|& \hologo{LaTeX} counter\\
%\end{tabular}
%\end{quote}
%    \begin{macrocode}
\def\rc@@set#1#2#3#4{%
  \ifx#1\relax
    #2{#4}{\rc@default}%
  \else
    #2{#4}{%
      \expandafter#3#1\rc@default\rc@default\@nil
    }%
  \fi
}
%    \end{macrocode}
%    \end{macro}
%
% \subsubsection{User commands}
%
%    \begin{macro}{\setcounterref}
%    \begin{macrocode}
\rc@RobustDefZero\setcounterref{%
  \rc@set\setcounter\ltx@car
}
%    \end{macrocode}
%    \end{macro}
%    \begin{macro}{\addtocounterref}
%    \begin{macrocode}
\rc@RobustDefZero\addtocounterref{%
  \rc@set\addtocounter\ltx@car
}
%    \end{macrocode}
%    \end{macro}
%    \begin{macro}{\setcounterpageref}
%    \begin{macrocode}
\rc@RobustDefZero\setcounterpageref{%
  \rc@set\setcounter\ltx@carsecond
}
%    \end{macrocode}
%    \end{macro}
%    \begin{macro}{\addtocounterpageref}
%    \begin{macrocode}
\rc@RobustDefZero\addtocounterpageref{%
  \rc@set\addtocounter\ltx@carsecond
}
%    \end{macrocode}
%    \end{macro}
%
% \subsection{Extracting references}
%
%    \begin{macro}{\getrefnumber}
%    \begin{macrocode}
\rc@newcommand*{\getrefnumber}[1]{%
  \romannumeral
  \ltx@ifundefined{r@#1}{%
    \expandafter\ltx@zero
    \rc@default
  }{%
    \expandafter\expandafter\expandafter\rc@extract@
    \expandafter\expandafter\expandafter!%
    \csname r@#1\expandafter\endcsname
    \expandafter{\rc@default}\@nil
  }%
}
%    \end{macrocode}
%    \end{macro}
%    \begin{macro}{\getpagerefnumber}
%    \begin{macrocode}
\rc@newcommand*{\getpagerefnumber}[1]{%
  \romannumeral
  \ltx@ifundefined{r@#1}{%
    \expandafter\ltx@zero
    \rc@default
  }{%
    \expandafter\expandafter\expandafter\rc@extract@page
    \expandafter\expandafter\expandafter!%
    \csname r@#1\expandafter\expandafter\expandafter\endcsname
    \expandafter\expandafter\expandafter{%
      \expandafter\rc@default
    \expandafter}\expandafter{\rc@default}\@nil
  }%
}
%    \end{macrocode}
%    \end{macro}
%    \begin{macro}{\getrefbykeydefault}
%    \begin{macrocode}
\rc@newcommand*{\getrefbykeydefault}[2]{%
  \romannumeral
  \expandafter\rc@getrefbykeydefault
    \csname r@#1\expandafter\endcsname
    \csname rc@extract@#2\endcsname
}
%    \end{macrocode}
%    \end{macro}
%    \begin{macro}{\rc@getrefbykeydefault}
%\begin{quote}
%\begin{tabular}[t]{@{}l@{: }l@{}}
% |#1|& \cs{r@<...>}\\
% |#2|& \cs{rc@extract@<...>}\\
% |#3|& default\\
%\end{tabular}
%\end{quote}
%    \begin{macrocode}
\long\def\rc@getrefbykeydefault#1#2#3{%
  \ifx#1\relax
    % reference is undefined
    \ltx@ReturnAfterElseFi{%
      \ltx@zero
      #3%
    }%
  \else
    \ltx@ReturnAfterFi{%
      \ifx#2\relax
        % extract method is missing
        \ltx@ReturnAfterElseFi{%
          \ltx@zero
          #3%
        }%
      \else
        \ltx@ReturnAfterFi{%
          \expandafter
          \rc@generic#1{#3}{#3}{#3}{#3}{#3}\@nil#2{#3}%
        }%
      \fi
    }%
  \fi
}
%    \end{macrocode}
%    \end{macro}
%    \begin{macro}{\rc@generic}
%\begin{quote}
%\begin{tabular}[t]{@{}l@{: }l@{}}
% |#1|& first item in \cs{r@<...>}\\
% |#2|& remaining items in \cs{r@<...>}\\
% |#3|& \cs{rc@extract@<...>}\\
% |#4|& default\\
%\end{tabular}
%\end{quote}
%    \begin{macrocode}
\long\def\rc@generic#1#2\@nil#3#4{%
  #3{#1\TR@TitleReference\@empty{#4}\@nil}{#1}#2\@nil
}
%    \end{macrocode}
%    \end{macro}
%    \begin{macro}{\rc@extract@}
%    \begin{macrocode}
\long\def\rc@extract@#1#2#3\@nil{%
  \ltx@zero
  #2%
}
%    \end{macrocode}
%    \end{macro}
%    \begin{macro}{\rc@extract@page}
%    \begin{macrocode}
\long\def\rc@extract@page#1#2#3#4\@nil{%
  \ltx@zero
  #3%
}
%    \end{macrocode}
%    \end{macro}
%    \begin{macro}{\rc@extract@name}
%    \begin{macrocode}
\long\def\rc@extract@name#1#2#3#4#5\@nil{%
  \ltx@zero
  #4%
}
%    \end{macrocode}
%    \end{macro}
%    \begin{macro}{\rc@extract@anchor}
%    \begin{macrocode}
\long\def\rc@extract@anchor#1#2#3#4#5#6\@nil{%
  \ltx@zero
  #5%
}
%    \end{macrocode}
%    \end{macro}
%    \begin{macro}{\rc@extract@url}
%    \begin{macrocode}
\long\def\rc@extract@url#1#2#3#4#5#6#7\@nil{%
  \ltx@zero
  #6%
}
%    \end{macrocode}
%    \end{macro}
%    \begin{macro}{\rc@extract@title}
%    \begin{macrocode}
\long\def\rc@extract@title#1#2\@nil{%
  \rc@@extract@title#1%
}
%    \end{macrocode}
%    \end{macro}
%    \begin{macro}{\rc@@extract@title}
%    \begin{macrocode}
\long\def\rc@@extract@title#1\TR@TitleReference#2#3#4\@nil{%
  \ltx@zero
  #3%
}
%    \end{macrocode}
%    \end{macro}
%
% \subsection{Macros for checking undefined references}
%
%    \begin{macro}{\IfRefUndefinedExpandable}
%    \begin{macrocode}
\rc@newcommand*{\IfRefUndefinedExpandable}[1]{%
  \ltx@ifundefined{r@#1}\ltx@firstoftwo\ltx@secondoftwo
}
%    \end{macrocode}
%    \end{macro}
%    \begin{macro}{\IfRefUndefinedBabel}
%    \begin{macrocode}
\rc@RobustDefOne{}*\IfRefUndefinedBabel{%
  \begingroup
    \csname safe@actives@true\endcsname
  \expandafter\expandafter\expandafter\endgroup
  \expandafter\ifx\csname r@#1\endcsname\relax
    \expandafter\ltx@firstoftwo
  \else
    \expandafter\ltx@secondoftwo
  \fi
}
%    \end{macrocode}
%    \end{macro}
%    \begin{macrocode}
\rc@AtEnd%
%</package>
%    \end{macrocode}
%
% \section{Test}
%
% \subsection{Catcode checks for loading}
%
%    \begin{macrocode}
%<*test1>
%    \end{macrocode}
%    \begin{macrocode}
\catcode`\{=1 %
\catcode`\}=2 %
\catcode`\#=6 %
\catcode`\@=11 %
\expandafter\ifx\csname count@\endcsname\relax
  \countdef\count@=255 %
\fi
\expandafter\ifx\csname @gobble\endcsname\relax
  \long\def\@gobble#1{}%
\fi
\expandafter\ifx\csname @firstofone\endcsname\relax
  \long\def\@firstofone#1{#1}%
\fi
\expandafter\ifx\csname loop\endcsname\relax
  \expandafter\@firstofone
\else
  \expandafter\@gobble
\fi
{%
  \def\loop#1\repeat{%
    \def\body{#1}%
    \iterate
  }%
  \def\iterate{%
    \body
      \let\next\iterate
    \else
      \let\next\relax
    \fi
    \next
  }%
  \let\repeat=\fi
}%
\def\RestoreCatcodes{}
\count@=0 %
\loop
  \edef\RestoreCatcodes{%
    \RestoreCatcodes
    \catcode\the\count@=\the\catcode\count@\relax
  }%
\ifnum\count@<255 %
  \advance\count@ 1 %
\repeat

\def\RangeCatcodeInvalid#1#2{%
  \count@=#1\relax
  \loop
    \catcode\count@=15 %
  \ifnum\count@<#2\relax
    \advance\count@ 1 %
  \repeat
}
\def\RangeCatcodeCheck#1#2#3{%
  \count@=#1\relax
  \loop
    \ifnum#3=\catcode\count@
    \else
      \errmessage{%
        Character \the\count@\space
        with wrong catcode \the\catcode\count@\space
        instead of \number#3%
      }%
    \fi
  \ifnum\count@<#2\relax
    \advance\count@ 1 %
  \repeat
}
\def\space{ }
\expandafter\ifx\csname LoadCommand\endcsname\relax
  \def\LoadCommand{\input refcount.sty\relax}%
\fi
\def\Test{%
  \RangeCatcodeInvalid{0}{47}%
  \RangeCatcodeInvalid{58}{64}%
  \RangeCatcodeInvalid{91}{96}%
  \RangeCatcodeInvalid{123}{255}%
  \catcode`\@=12 %
  \catcode`\\=0 %
  \catcode`\%=14 %
  \LoadCommand
  \RangeCatcodeCheck{0}{36}{15}%
  \RangeCatcodeCheck{37}{37}{14}%
  \RangeCatcodeCheck{38}{47}{15}%
  \RangeCatcodeCheck{48}{57}{12}%
  \RangeCatcodeCheck{58}{63}{15}%
  \RangeCatcodeCheck{64}{64}{12}%
  \RangeCatcodeCheck{65}{90}{11}%
  \RangeCatcodeCheck{91}{91}{15}%
  \RangeCatcodeCheck{92}{92}{0}%
  \RangeCatcodeCheck{93}{96}{15}%
  \RangeCatcodeCheck{97}{122}{11}%
  \RangeCatcodeCheck{123}{255}{15}%
  \RestoreCatcodes
}
\Test
\csname @@end\endcsname
\end
%    \end{macrocode}
%    \begin{macrocode}
%</test1>
%    \end{macrocode}
%
% \subsection{Macro tests}
%
%    \begin{macrocode}
%<*test2>
\errorcontextlines=10000 %
\showboxbreadth=10000 %
\showboxdepth=10000 %
\begingroup\expandafter\expandafter\expandafter\endgroup
\expandafter\ifx\csname RequirePackage\endcsname\relax
  \input refcount.sty\relax
\else
  \RequirePackage{refcount}[2011/10/16]%
\fi
\catcode`\@=11 %
\begingroup\expandafter\expandafter\expandafter\endgroup
\expandafter\ifx\csname @onelevel@sanitize\endcsname\relax
  \begingroup\expandafter\expandafter\expandafter\endgroup
  \expandafter\ifx\csname detokenize\endcsname\relax
    \def\strip@prefix#1->{}%
    \def\@onelevel@sanitize#1{%
      \edef#1{%
        \expandafter\strip@prefix\meaning#1%
      }%
    }%
  \else
    \def\@onelevel@sanitize#1{%
      \edef#1{%
        \detokenize\expandafter{#1}%
      }%
    }%
  \fi
\fi
\def\msg#{\immediate\write16}
\def\empty{}
\def\space{ }
%    \end{macrocode}
%    \begin{macrocode}
\def\r@foo{{\empty 1}{\empty 2}}
\long\def\test#1#2{%
  \begingroup
    \setbox0=\hbox{%
      \def\TestTask{#1}%
      \@onelevel@sanitize\TestTask
      \msg{* \TestTask}%
      \expandafter\expandafter\expandafter\def
      \expandafter\expandafter\expandafter\TestResult
      \expandafter\expandafter\expandafter{%
        #1%
      }%
      \def\TestExpected{#2}%
      \ifx\TestResult\TestExpected
        \msg{ \space ok.}%
      \else
        \@onelevel@sanitize\TestResult
        \@onelevel@sanitize\TestExpected
        \msg{ \space Result: \space\space[\TestResult]}%
        \msg{ \space Expected: [\TestExpected]}%
        \errmessage{Test failed!}%
      \fi
    }%
    \ifdim\wd0=0pt %
    \else
      \showbox0 %
    \fi
  \endgroup
}
\test{\getrefnumber{foo}}{\empty 1}
\test{\getpagerefnumber{foo}}{\empty 2}
\test{\getrefbykeydefault{foo}{}{\empty default}}{\empty 1}
\test{\getrefbykeydefault{foo}{page}{\empty default}}{\empty 2}
\test{\getrefbykeydefault{foo}{name}{\empty default}}{\empty default}
\test{\getrefbykeydefault{foo}{anchor}{\empty default}}{\empty default}
\test{\getrefbykeydefault{foo}{url}{\empty default}}{\empty default}
\test{\getrefbykeydefault{foo}{title}{\empty default}}{\empty default}
\msg{}
\def\r@foo{{}{}{}{}{}{}{}{}{}{}}
\def\Test#1#2\\{%
  \test{#1{foo}#2}{}%
}
\def\TestGroup{%
  \Test\getrefnumber\\%
  \Test\getpagerefnumber\\%
  \Test\getrefbykeydefault{}{}\\%
  \Test\getrefbykeydefault{page}{}\\%
  \Test\getrefbykeydefault{anchor}{}\\%
  \Test\getrefbykeydefault{name}{}\\%
  \Test\getrefbykeydefault{url}{}\\%
}
\TestGroup
\Test\getrefbykeydefault{title}{}\\%
\msg{}
\def\r@foo{\par\par\par\par\par\par\par\par}
\long\def\Test#1#2\\{%
  \test{#1{foo}#2}{\par}%
}
\TestGroup
\test{\getrefbykeydefault{title}{}{}}{}
\msg{}
\def\r@foo{{ }{ }{ }{ }{ }}
\def\Test#1#2\\{%
  \test{#1{foo}#2}{ }%
}
\TestGroup
\msg{}
\long\def\TestDefault#1{%
  \begingroup
    \setrefcountdefault{#1}%
    \test{\getrefnumber{foo}}{#1}%
    \test{\getpagerefnumber{foo}}{#1}%
  \endgroup
}
\def\TestDefaultX{%
  \TestDefault{}%
  \TestDefault{\par}%
  \TestDefault{ }%
  \TestDefault{\space}%
}
\let\r@foo\@undefined
\TestDefaultX
\let\r@foo\relax
\TestDefaultX
\def\r@foo{}
\TestDefaultX
%    \end{macrocode}
%    \begin{macrocode}
\msg{}
\long\def\Test#1#2#3#4{%
  \begingroup
    \def\TestTask{#1}%
    \@onelevel@sanitize\TestTask
    \msg{* [\TestTask]}%
    \edef\TestResultA{\IfRefUndefinedExpandable{#1}{#2}{#3}}%
    \IfRefUndefinedBabel{#1}{%
      \def\TestResultB{#2}%
    }{%
      \def\TestResultB{#3}%
    }%
    \def\TestExpected{#4}%
    \ifx\TestResultA\TestExpected
      \msg{ \space ok.}%
    \else
      \begingroup
        \@onelevel@sanitize\TestResultA
        \@onelevel@sanitize\TestExpected
        \msg{ \space Result: \space\space[\TestResultA]}%
        \msg{ \space Expected: [\TestExpected]}%
        \errmessage{Test failed!}%
      \endgroup
    \fi
    \ifx\TestResultB\TestExpected
      \msg{ \space ok.}%
    \else
      \begingroup
        \@onelevel@sanitize\TestResultB
        \@onelevel@sanitize\TestExpected
        \msg{ \space Result: \space\space[\TestResultB]}%
        \msg{ \space Expected: [\TestExpected]}%
        \errmessage{Test failed!}%
      \endgroup
    \fi
  \endgroup
}
\begingroup
  \def\r@foo{{}{}}%
  \let\r@bar\@undefined
  \let\r@xyz\relax
  \Test{foo}{true}{false}{false}%
  \Test{bar}{true}{false}{true}%
  \Test{xyz}{true}{false}{true}%
\endgroup
%    \end{macrocode}
%    \begin{macrocode}
\csname @@end\endcsname\end
%</test2>
%    \end{macrocode}
% \subsection{Test with package \xpackage{titleref}}
%
%    \begin{macrocode}
%<*test3>
\NeedsTeXFormat{LaTeX2e}
\documentclass{article}
\usepackage{refcount}[2011/10/16]
%<test4>\usepackage{nameref}
\usepackage{titleref}
\begin{document}
\section{Hello World}
\label{sec:hello}
\section{\hbox{xy}}
\label{sec:foo}
%
\makeatletter
\@ifundefined{r@sec:hello}{%
  \typeout{==> Compile twice!}%
}{%
  \def\test#1#2{%
    \begingroup
      \def\TestTask{#1}%
      \@onelevel@sanitize\TestTask
      \typeout{* \TestTask}%
      \expandafter\expandafter\expandafter\def
      \expandafter\expandafter\expandafter\TestResult
      \expandafter\expandafter\expandafter{%
        #1%
      }%
      \def\TestExpected{#2}%
      \ifx\TestResult\TestExpected
        \typeout{ \space ok.}%
      \else
        \@onelevel@sanitize\TestResult
        \@onelevel@sanitize\TestExpected
        \typeout{ \space Result: \space\space[\TestResult]}%
        \typeout{ \space Expected: [\TestExpected]}%
        \errmessage{Test failed!}%
      \fi
    \endgroup
  }%
  \test{\getrefbykeydefault{sec:hello}{title}{}}{Hello World}%
  \test{\getrefbykeydefault{sec:foo}{title}{}}{\hbox{xy}}%
  \begingroup
    \def\hbox#1{[#1]}% hash-ok
    \test{\getrefbykeydefault{sec:foo}{title}{}}{\hbox{xy}}%
  \endgroup
}
\makeatother
%    \end{macrocode}
%    \begin{macrocode}
\end{document}
%</test3>
%    \end{macrocode}
%    \begin{macrocode}
%<*test5>
\NeedsTeXFormat{LaTeX2e}
\documentclass{book}
\usepackage{refcount}[2011/10/16]
\usepackage{zref-runs}
\newcounter{test}
\begin{document}
\ifnum\zruns>1 %
  \makeatletter
  \def\Test#1#2#3{%
    \begingroup
      \setcounter{test}{10}%
      \sbox0{%
        #1{test}{#2}%
        \ifnum#3=\value{test}%
        \else
          \PackageError{test}{\string#1{#2} <> #3 (\the\value{test})}%
        \fi
      }%
      \ifdim\wd0=0pt %
      \else
        \PackageError{test}{Non-empty box}\@ehc
      \fi
    \endgroup
  }%
  \makeatother
  \Test\setcounterpageref{ch:two}{1}%
  \Test\setcounterpageref{ch:three}{3}%
  \Test\setcounterpageref{ch:four}{5}%
  \Test\setcounterpageref{ch:five}{7}%
  \Test\setcounterpageref{ch:six}{9}%
  \Test\setcounterpageref{ch:seven}{13}%
  \Test\addtocounterpageref{ch:two}{11}%
  \Test\addtocounterpageref{ch:three}{13}%
  \Test\addtocounterpageref{ch:four}{15}%
  \Test\addtocounterpageref{ch:five}{17}%
  \Test\addtocounterpageref{ch:six}{19}%
  \Test\addtocounterpageref{ch:seven}{23}%
  \Test\setcounterref{ch:two}{1}%
  \Test\setcounterref{ch:three}{2}%
  \Test\setcounterref{ch:four}{11}%
  \Test\addtocounterref{ch:two}{11}%
  \Test\addtocounterref{ch:three}{12}%
  \Test\addtocounterref{ch:four}{21}%
\fi
\frontmatter
\chapter{Chapter one}\label{ch:one}
\cleardoublepage
\mainmatter
\chapter{Chapter two}\label{ch:two}
\cleardoublepage
\chapter{Chapter three}\label{ch:three}
\cleardoublepage
\setcounter{chapter}{10}
\chapter{Chapter four}\label{ch:four}
\cleardoublepage
\appendix
\chapter{Chapter five}\label{ch:five}
\cleardoublepage
\chapter{Chapter six}\label{ch:six}
\cleardoublepage
\null
\cleardoublepage
\chapter{Chapter seven}\label{ch:seven}
\end{document}
%</test5>
%    \end{macrocode}
%
% \section{Installation}
%
% \subsection{Download}
%
% \paragraph{Package.} This package is available on
% CTAN\footnote{\url{ftp://ftp.ctan.org/tex-archive/}}:
% \begin{description}
% \item[\CTAN{macros/latex/contrib/oberdiek/refcount.dtx}] The source file.
% \item[\CTAN{macros/latex/contrib/oberdiek/refcount.pdf}] Documentation.
% \end{description}
%
%
% \paragraph{Bundle.} All the packages of the bundle `oberdiek'
% are also available in a TDS compliant ZIP archive. There
% the packages are already unpacked and the documentation files
% are generated. The files and directories obey the TDS standard.
% \begin{description}
% \item[\CTAN{install/macros/latex/contrib/oberdiek.tds.zip}]
% \end{description}
% \emph{TDS} refers to the standard ``A Directory Structure
% for \TeX\ Files'' (\CTAN{tds/tds.pdf}). Directories
% with \xfile{texmf} in their name are usually organized this way.
%
% \subsection{Bundle installation}
%
% \paragraph{Unpacking.} Unpack the \xfile{oberdiek.tds.zip} in the
% TDS tree (also known as \xfile{texmf} tree) of your choice.
% Example (linux):
% \begin{quote}
%   |unzip oberdiek.tds.zip -d ~/texmf|
% \end{quote}
%
% \paragraph{Script installation.}
% Check the directory \xfile{TDS:scripts/oberdiek/} for
% scripts that need further installation steps.
% Package \xpackage{attachfile2} comes with the Perl script
% \xfile{pdfatfi.pl} that should be installed in such a way
% that it can be called as \texttt{pdfatfi}.
% Example (linux):
% \begin{quote}
%   |chmod +x scripts/oberdiek/pdfatfi.pl|\\
%   |cp scripts/oberdiek/pdfatfi.pl /usr/local/bin/|
% \end{quote}
%
% \subsection{Package installation}
%
% \paragraph{Unpacking.} The \xfile{.dtx} file is a self-extracting
% \docstrip\ archive. The files are extracted by running the
% \xfile{.dtx} through \plainTeX:
% \begin{quote}
%   \verb|tex refcount.dtx|
% \end{quote}
%
% \paragraph{TDS.} Now the different files must be moved into
% the different directories in your installation TDS tree
% (also known as \xfile{texmf} tree):
% \begin{quote}
% \def\t{^^A
% \begin{tabular}{@{}>{\ttfamily}l@{ $\rightarrow$ }>{\ttfamily}l@{}}
%   refcount.sty & tex/latex/oberdiek/refcount.sty\\
%   refcount.pdf & doc/latex/oberdiek/refcount.pdf\\
%   test/refcount-test1.tex & doc/latex/oberdiek/test/refcount-test1.tex\\
%   test/refcount-test2.tex & doc/latex/oberdiek/test/refcount-test2.tex\\
%   test/refcount-test3.tex & doc/latex/oberdiek/test/refcount-test3.tex\\
%   test/refcount-test4.tex & doc/latex/oberdiek/test/refcount-test4.tex\\
%   test/refcount-test5.tex & doc/latex/oberdiek/test/refcount-test5.tex\\
%   refcount.dtx & source/latex/oberdiek/refcount.dtx\\
% \end{tabular}^^A
% }^^A
% \sbox0{\t}^^A
% \ifdim\wd0>\linewidth
%   \begingroup
%     \advance\linewidth by\leftmargin
%     \advance\linewidth by\rightmargin
%   \edef\x{\endgroup
%     \def\noexpand\lw{\the\linewidth}^^A
%   }\x
%   \def\lwbox{^^A
%     \leavevmode
%     \hbox to \linewidth{^^A
%       \kern-\leftmargin\relax
%       \hss
%       \usebox0
%       \hss
%       \kern-\rightmargin\relax
%     }^^A
%   }^^A
%   \ifdim\wd0>\lw
%     \sbox0{\small\t}^^A
%     \ifdim\wd0>\linewidth
%       \ifdim\wd0>\lw
%         \sbox0{\footnotesize\t}^^A
%         \ifdim\wd0>\linewidth
%           \ifdim\wd0>\lw
%             \sbox0{\scriptsize\t}^^A
%             \ifdim\wd0>\linewidth
%               \ifdim\wd0>\lw
%                 \sbox0{\tiny\t}^^A
%                 \ifdim\wd0>\linewidth
%                   \lwbox
%                 \else
%                   \usebox0
%                 \fi
%               \else
%                 \lwbox
%               \fi
%             \else
%               \usebox0
%             \fi
%           \else
%             \lwbox
%           \fi
%         \else
%           \usebox0
%         \fi
%       \else
%         \lwbox
%       \fi
%     \else
%       \usebox0
%     \fi
%   \else
%     \lwbox
%   \fi
% \else
%   \usebox0
% \fi
% \end{quote}
% If you have a \xfile{docstrip.cfg} that configures and enables \docstrip's
% TDS installing feature, then some files can already be in the right
% place, see the documentation of \docstrip.
%
% \subsection{Refresh file name databases}
%
% If your \TeX~distribution
% (\teTeX, \mikTeX, \dots) relies on file name databases, you must refresh
% these. For example, \teTeX\ users run \verb|texhash| or
% \verb|mktexlsr|.
%
% \subsection{Some details for the interested}
%
% \paragraph{Attached source.}
%
% The PDF documentation on CTAN also includes the
% \xfile{.dtx} source file. It can be extracted by
% AcrobatReader 6 or higher. Another option is \textsf{pdftk},
% e.g. unpack the file into the current directory:
% \begin{quote}
%   \verb|pdftk refcount.pdf unpack_files output .|
% \end{quote}
%
% \paragraph{Unpacking with \LaTeX.}
% The \xfile{.dtx} chooses its action depending on the format:
% \begin{description}
% \item[\plainTeX:] Run \docstrip\ and extract the files.
% \item[\LaTeX:] Generate the documentation.
% \end{description}
% If you insist on using \LaTeX\ for \docstrip\ (really,
% \docstrip\ does not need \LaTeX), then inform the autodetect routine
% about your intention:
% \begin{quote}
%   \verb|latex \let\install=y\input{refcount.dtx}|
% \end{quote}
% Do not forget to quote the argument according to the demands
% of your shell.
%
% \paragraph{Generating the documentation.}
% You can use both the \xfile{.dtx} or the \xfile{.drv} to generate
% the documentation. The process can be configured by the
% configuration file \xfile{ltxdoc.cfg}. For instance, put this
% line into this file, if you want to have A4 as paper format:
% \begin{quote}
%   \verb|\PassOptionsToClass{a4paper}{article}|
% \end{quote}
% An example follows how to generate the
% documentation with pdf\LaTeX:
% \begin{quote}
%\begin{verbatim}
%pdflatex refcount.dtx
%makeindex -s gind.ist refcount.idx
%pdflatex refcount.dtx
%makeindex -s gind.ist refcount.idx
%pdflatex refcount.dtx
%\end{verbatim}
% \end{quote}
%
% \section{Catalogue}
%
% The following XML file can be used as source for the
% \href{http://mirror.ctan.org/help/Catalogue/catalogue.html}{\TeX\ Catalogue}.
% The elements \texttt{caption} and \texttt{description} are imported
% from the original XML file from the Catalogue.
% The name of the XML file in the Catalogue is \xfile{refcount.xml}.
%    \begin{macrocode}
%<*catalogue>
<?xml version='1.0' encoding='us-ascii'?>
<!DOCTYPE entry SYSTEM 'catalogue.dtd'>
<entry datestamp='$Date$' modifier='$Author$' id='refcount'>
  <name>refcount</name>
  <caption>Counter operations with label references.</caption>
  <authorref id='auth:oberdiek'/>
  <copyright owner='Heiko Oberdiek' year='1998,2000,2006,2008,2010,2011'/>
  <license type='lppl1.3'/>
  <version number='3.4'/>
  <description>
    Provides commands <tt>\setcounterref</tt> and
    <tt>\addtocounterref</tt> which use the section (or whatever)
    number from the reference as the value to put into the counter, as
    in:

    <pre>
    ...\label{sec:foo}
    ...
    \setcounterref{foonum}{sec:foo}
    </pre>
    Commands <tt>\setcounterpageref</tt> and
    <tt>\addtocounterpageref</tt> do the corresponding thing with the
    page reference of the label.
    <p/>
    No <tt>.ins</tt> file is distributed; process the
    <tt>.dtx</tt> with plain TeX to create one.
    <p/>
    The package is part of the <xref refid='oberdiek'>oberdiek</xref>
    bundle.
  </description>
  <documentation details='Package documentation'
      href='ctan:/macros/latex/contrib/oberdiek/refcount.pdf'/>
  <ctan file='true' path='/macros/latex/contrib/oberdiek/refcount.dtx'/>
  <miktex location='oberdiek'/>
  <texlive location='oberdiek'/>
  <install path='/macros/latex/contrib/oberdiek/oberdiek.tds.zip'/>
</entry>
%</catalogue>
%    \end{macrocode}
%
% \begin{History}
%   \begin{Version}{1998/04/08 v1.0}
%   \item
%     First public release, written as answer in the
%     newsgroup \xnewsgroup{comp.text.tex}:
%     \URL{``\link{Re: Adding a \cs{ref} to a counter?}''}^^A
%     {http://groups.google.com/group/comp.text.tex/msg/c3f2a135ef5ee528}
%   \end{Version}
%   \begin{Version}{2000/09/07 v2.0}
%   \item
%     Documentation added.
%   \item
%     LPPL 1.2
%   \item
%     Package rewritten, new commands added.
%   \end{Version}
%   \begin{Version}{2006/02/20 v3.0}
%   \item
%     Support for \xpackage{hyperref} and \xpackage{nameref} improved.
%   \item
%     Support for \xpackage{titleref} and \xpackage{babel}'s shorthands added.
%   \item
%     New: \cs{refused}, \cs{getrefbykeydefault}
%   \end{Version}
%   \begin{Version}{2008/08/11 v3.1}
%   \item
%     Code is not changed.
%   \item
%     URLs updated.
%   \end{Version}
%   \begin{Version}{2010/12/01 v3.2}
%   \item
%     \cs{IfRefUndefinedExpandable} and \cs{IfRefUndefinedBabel} added.
%   \item
%     \cs{getrefnumber}, \cs{getpagerefnumber}, \cs{getrefbykeydefault}
%     are expandable in exact two expansion steps.
%   \item
%     Non-expandable macros are made robust.
%   \item
%     Test files added.
%   \end{Version}
%   \begin{Version}{2011/06/22 v3.3}
%   \item
%     Bug fix: \cs{rc@refused} is undefined for \cs{setcounterpageref}
%     and similar macros. (Bug found by Marc van Dongen.)
%   \end{Version}
%   \begin{Version}{2011/10/16 v3.4}
%   \item
%     Bug fix: \cs{setcounterpageref} and \cs{addtocounterpageref} fixed.
%     (Bug found by Staz.)
%   \item
%     Macros \cs{(set|addto)counter(page|)ref} are made robust.
%   \end{Version}
% \end{History}
%
% \PrintIndex
%
% \Finale
\endinput

%        (quote the arguments according to the demands of your shell)
%
% Documentation:
%    (a) If refcount.drv is present:
%           latex refcount.drv
%    (b) Without refcount.drv:
%           latex refcount.dtx; ...
%    The class ltxdoc loads the configuration file ltxdoc.cfg
%    if available. Here you can specify further options, e.g.
%    use A4 as paper format:
%       \PassOptionsToClass{a4paper}{article}
%
%    Programm calls to get the documentation (example):
%       pdflatex refcount.dtx
%       makeindex -s gind.ist refcount.idx
%       pdflatex refcount.dtx
%       makeindex -s gind.ist refcount.idx
%       pdflatex refcount.dtx
%
% Installation:
%    TDS:tex/latex/oberdiek/refcount.sty
%    TDS:doc/latex/oberdiek/refcount.pdf
%    TDS:doc/latex/oberdiek/test/refcount-test1.tex
%    TDS:doc/latex/oberdiek/test/refcount-test2.tex
%    TDS:doc/latex/oberdiek/test/refcount-test3.tex
%    TDS:doc/latex/oberdiek/test/refcount-test4.tex
%    TDS:doc/latex/oberdiek/test/refcount-test5.tex
%    TDS:source/latex/oberdiek/refcount.dtx
%
%<*ignore>
\begingroup
  \catcode123=1 %
  \catcode125=2 %
  \def\x{LaTeX2e}%
\expandafter\endgroup
\ifcase 0\ifx\install y1\fi\expandafter
         \ifx\csname processbatchFile\endcsname\relax\else1\fi
         \ifx\fmtname\x\else 1\fi\relax
\else\csname fi\endcsname
%</ignore>
%<*install>
\input docstrip.tex
\Msg{************************************************************************}
\Msg{* Installation}
\Msg{* Package: refcount 2011/10/16 v3.4 Data extraction from label references (HO)}
\Msg{************************************************************************}

\keepsilent
\askforoverwritefalse

\let\MetaPrefix\relax
\preamble

This is a generated file.

Project: refcount
Version: 2011/10/16 v3.4

Copyright (C) 1998, 2000, 2006, 2008, 2010, 2011 by
   Heiko Oberdiek <heiko.oberdiek at googlemail.com>

This work may be distributed and/or modified under the
conditions of the LaTeX Project Public License, either
version 1.3c of this license or (at your option) any later
version. This version of this license is in
   http://www.latex-project.org/lppl/lppl-1-3c.txt
and the latest version of this license is in
   http://www.latex-project.org/lppl.txt
and version 1.3 or later is part of all distributions of
LaTeX version 2005/12/01 or later.

This work has the LPPL maintenance status "maintained".

This Current Maintainer of this work is Heiko Oberdiek.

This work consists of the main source file refcount.dtx
and the derived files
   refcount.sty, refcount.pdf, refcount.ins, refcount.drv,
   refcount-test1.tex, refcount-test2.tex, refcount-test3.tex,
   refcount-test4.tex, refcount-test5.tex.

\endpreamble
\let\MetaPrefix\DoubleperCent

\generate{%
  \file{refcount.ins}{\from{refcount.dtx}{install}}%
  \file{refcount.drv}{\from{refcount.dtx}{driver}}%
  \usedir{tex/latex/oberdiek}%
  \file{refcount.sty}{\from{refcount.dtx}{package}}%
  \usedir{doc/latex/oberdiek/test}%
  \file{refcount-test1.tex}{\from{refcount.dtx}{test1}}%
  \file{refcount-test2.tex}{\from{refcount.dtx}{test2}}%
  \file{refcount-test3.tex}{\from{refcount.dtx}{test3}}%
  \file{refcount-test4.tex}{\from{refcount.dtx}{test3,test4}}%
  \file{refcount-test5.tex}{\from{refcount.dtx}{test5}}%
  \nopreamble
  \nopostamble
  \usedir{source/latex/oberdiek/catalogue}%
  \file{refcount.xml}{\from{refcount.dtx}{catalogue}}%
}

\catcode32=13\relax% active space
\let =\space%
\Msg{************************************************************************}
\Msg{*}
\Msg{* To finish the installation you have to move the following}
\Msg{* file into a directory searched by TeX:}
\Msg{*}
\Msg{*     refcount.sty}
\Msg{*}
\Msg{* To produce the documentation run the file `refcount.drv'}
\Msg{* through LaTeX.}
\Msg{*}
\Msg{* Happy TeXing!}
\Msg{*}
\Msg{************************************************************************}

\endbatchfile
%</install>
%<*ignore>
\fi
%</ignore>
%<*driver>
\NeedsTeXFormat{LaTeX2e}
\ProvidesFile{refcount.drv}%
  [2011/10/16 v3.4 Data extraction from label references (HO)]%
\documentclass{ltxdoc}
\usepackage{holtxdoc}[2011/11/22]
\begin{document}
  \DocInput{refcount.dtx}%
\end{document}
%</driver>
% \fi
%
% \CheckSum{1185}
%
% \CharacterTable
%  {Upper-case    \A\B\C\D\E\F\G\H\I\J\K\L\M\N\O\P\Q\R\S\T\U\V\W\X\Y\Z
%   Lower-case    \a\b\c\d\e\f\g\h\i\j\k\l\m\n\o\p\q\r\s\t\u\v\w\x\y\z
%   Digits        \0\1\2\3\4\5\6\7\8\9
%   Exclamation   \!     Double quote  \"     Hash (number) \#
%   Dollar        \$     Percent       \%     Ampersand     \&
%   Acute accent  \'     Left paren    \(     Right paren   \)
%   Asterisk      \*     Plus          \+     Comma         \,
%   Minus         \-     Point         \.     Solidus       \/
%   Colon         \:     Semicolon     \;     Less than     \<
%   Equals        \=     Greater than  \>     Question mark \?
%   Commercial at \@     Left bracket  \[     Backslash     \\
%   Right bracket \]     Circumflex    \^     Underscore    \_
%   Grave accent  \`     Left brace    \{     Vertical bar  \|
%   Right brace   \}     Tilde         \~}
%
% \GetFileInfo{refcount.drv}
%
% \title{The \xpackage{refcount} package}
% \date{2011/10/16 v3.4}
% \author{Heiko Oberdiek\\\xemail{heiko.oberdiek at googlemail.com}}
%
% \maketitle
%
% \begin{abstract}
% References are not numbers, however they often store numerical
% data such as section or page numbers. \cs{ref} or \cs{pageref}
% cannot be used for counter assignments or calculations because
% they are not expandable, generate warnings, or can even be links.
% The package provides expandable macros to extract the data
% from references. Packages \xpackage{hyperref}, \xpackage{nameref},
% \xpackage{titleref}, and \xpackage{babel} are supported.
% \end{abstract}
%
% \tableofcontents
%
% \section{Usage}
%
% \subsection{Setting counters}
%
% The following commands are similar to \LaTeX's
% \cs{setcounter} and \cs{addtocounter},
% but they extract the number value from a reference:
% \begin{quote}
%   \cs{setcounterref}, \cs{addtocounterref}\\
%   \cs{setcounterpageref}, \cs{addtocounterpageref}
% \end{quote}
% They take two arguments:
% \begin{quote}
%    \cs{...counter...ref} |{|\meta{\LaTeX\ counter}|}|
%    |{|\meta{reference}|}|
% \end{quote}
% An undefined references produces the usual LaTeX warning
% and its value is assumed to be zero.
% Example:
% \begin{quote}
%\begin{verbatim}
%\newcounter{ctrA}
%\newcounter{ctrB}
%\refstepcounter{ctrA}\label{ref:A}
%\setcounterref{ctrB}{ref:A}
%\addtocounterpageref{ctrB}{ref:A}
%\end{verbatim}
% \end{quote}
%
% \subsection{Expandable commands}
%
% These commands that can be used in expandible contexts
% (inside calculations, \cs{edef}, \cs{csname}, \cs{write}, \dots):
% \begin{quote}
%   \cs{getrefnumber}, \cs{getpagerefnumber}
% \end{quote}
% They take one argument, the reference:
% \begin{quote}
%   \cs{get...refnumber} |{|\meta{reference}|}|
% \end{quote}
% The default for undefined references can be changed
% with macro \cs{setrefcountdefault}, for example this
% package calls:
% \begin{quote}
%   \cs{setrefcountdefault}|{0}|
% \end{quote}
%
% Since version 2.0 of this package there is a new
% command:
% \begin{quote}
%   \cs{getrefbykeydefault} |{|\meta{reference}|}|
%   |{|\meta{key}|}| |{|\meta{default}|}|
% \end{quote}
% This generalized version allows the extraction
% of further properties of a reference than the
% two standard ones. Thus the following properties
% are supported, if they are available:
% \begin{quote}
% \begin{tabular}{@{}l|l|l@{}}
%    Key & Description & Package\\
% \hline
%   \meta{empty} & same as \cs{ref} & \LaTeX\\
%   |page| & same as \cs{pageref} & \LaTeX\\
%   |title| & section and caption titles & \xpackage{titleref}\\
%   |name| & section and caption titles & \xpackage{nameref}\\
%   |anchor| & anchor name & \xpackage{hyperref}\\
%   |url| & url/file & \xpackage{hyperref}/\xpackage{xr}
% \end{tabular}
% \end{quote}
%
% Since version 3.2 the expandable macros described before
% in this section
% are expandable in exact two expansion steps.
%
% \subsection{Undefined references}
%
% Because warnings and assignments cannot be used in
% expandible contexts, undefined references do not
% produce a warning, their values are assumed to be zero.
% Example:
% \begin{quote}
%\begin{verbatim}
%\label{ref:here}% somewhere
%\refused{ref:here}% see below
%\ifodd\getpagerefnumber{ref:here}%
%  reference is on an odd page
%\else
%  reference is on an even page
%\fi
%\end{verbatim}
% \end{quote}
%
% In case of undefined references the user usually want's
% to be informed. Also \LaTeX\ prints a warning at
% the end of the \LaTeX\ run. To notify \LaTeX\ and
% get a normal warning, just use
% \begin{quote}
%   \cs{refused} |{|\meta{reference}|}|
% \end{quote}
% outside the expanding context. Example, see above.
%
% \subsubsection{Check for undefined references}
%
% In version 3.2 macros were added, that test, whether references
% are defined.
% \begin{declcs}{IfRefUndefinedExpandable} \M{refname} \M{then} \M{else}\\
%   \cs{IfRefUndefinedBabel} \M{refname} \M{then} \M{else}
% \end{declcs}
% If the reference is not available and therefore undefined, then
% argument \meta{then} is executed, otherwise argument \meta{else}
% is called. Macro \cs{IfRefUndefinedExpandable} is expandable,
% but \meta{refname} must not contain babel shorthand characters.
% Macro \cs{IfRefUndefinedBabel} supports shorthand characters of
% babel, but it is not expandable.
%
% \subsection{Notes}
%
% \begin{itemize}
% \item
%   The method of extracting the number in this
%   package also works in cases, where the
%   reference cannot be used directly, because
%   a package such as \xpackage{hyperref} has added
%   extra stuff (hyper link), so that the reference cannot
%   be used as number any more.
% \item
%   If the reference does not contain a number,
%   assignments to a counter will fail of course.
% \end{itemize}
%
%
% \StopEventually{
% }
%
% \section{Implementation}
%
%    \begin{macrocode}
%<*package>
%    \end{macrocode}
%    Reload check, especially if the package is not used with \LaTeX.
%    \begin{macrocode}
\begingroup\catcode61\catcode48\catcode32=10\relax%
  \catcode13=5 % ^^M
  \endlinechar=13 %
  \catcode35=6 % #
  \catcode39=12 % '
  \catcode44=12 % ,
  \catcode45=12 % -
  \catcode46=12 % .
  \catcode58=12 % :
  \catcode64=11 % @
  \catcode123=1 % {
  \catcode125=2 % }
  \expandafter\let\expandafter\x\csname ver@refcount.sty\endcsname
  \ifx\x\relax % plain-TeX, first loading
  \else
    \def\empty{}%
    \ifx\x\empty % LaTeX, first loading,
      % variable is initialized, but \ProvidesPackage not yet seen
    \else
      \expandafter\ifx\csname PackageInfo\endcsname\relax
        \def\x#1#2{%
          \immediate\write-1{Package #1 Info: #2.}%
        }%
      \else
        \def\x#1#2{\PackageInfo{#1}{#2, stopped}}%
      \fi
      \x{refcount}{The package is already loaded}%
      \aftergroup\endinput
    \fi
  \fi
\endgroup%
%    \end{macrocode}
%    Package identification:
%    \begin{macrocode}
\begingroup\catcode61\catcode48\catcode32=10\relax%
  \catcode13=5 % ^^M
  \endlinechar=13 %
  \catcode35=6 % #
  \catcode39=12 % '
  \catcode40=12 % (
  \catcode41=12 % )
  \catcode44=12 % ,
  \catcode45=12 % -
  \catcode46=12 % .
  \catcode47=12 % /
  \catcode58=12 % :
  \catcode64=11 % @
  \catcode91=12 % [
  \catcode93=12 % ]
  \catcode123=1 % {
  \catcode125=2 % }
  \expandafter\ifx\csname ProvidesPackage\endcsname\relax
    \def\x#1#2#3[#4]{\endgroup
      \immediate\write-1{Package: #3 #4}%
      \xdef#1{#4}%
    }%
  \else
    \def\x#1#2[#3]{\endgroup
      #2[{#3}]%
      \ifx#1\@undefined
        \xdef#1{#3}%
      \fi
      \ifx#1\relax
        \xdef#1{#3}%
      \fi
    }%
  \fi
\expandafter\x\csname ver@refcount.sty\endcsname
\ProvidesPackage{refcount}%
  [2011/10/16 v3.4 Data extraction from label references (HO)]%
%    \end{macrocode}
%
%    \begin{macrocode}
\begingroup\catcode61\catcode48\catcode32=10\relax%
  \catcode13=5 % ^^M
  \endlinechar=13 %
  \catcode123=1 % {
  \catcode125=2 % }
  \catcode64=11 % @
  \def\x{\endgroup
    \expandafter\edef\csname rc@AtEnd\endcsname{%
      \endlinechar=\the\endlinechar\relax
      \catcode13=\the\catcode13\relax
      \catcode32=\the\catcode32\relax
      \catcode35=\the\catcode35\relax
      \catcode61=\the\catcode61\relax
      \catcode64=\the\catcode64\relax
      \catcode123=\the\catcode123\relax
      \catcode125=\the\catcode125\relax
    }%
  }%
\x\catcode61\catcode48\catcode32=10\relax%
\catcode13=5 % ^^M
\endlinechar=13 %
\catcode35=6 % #
\catcode64=11 % @
\catcode123=1 % {
\catcode125=2 % }
\def\TMP@EnsureCode#1#2{%
  \edef\rc@AtEnd{%
    \rc@AtEnd
    \catcode#1=\the\catcode#1\relax
  }%
  \catcode#1=#2\relax
}
\TMP@EnsureCode{33}{12}% !
\TMP@EnsureCode{39}{12}% '
\TMP@EnsureCode{42}{12}% *
\TMP@EnsureCode{45}{12}% -
\TMP@EnsureCode{46}{12}% .
\TMP@EnsureCode{47}{12}% /
\TMP@EnsureCode{91}{12}% [
\TMP@EnsureCode{93}{12}% ]
\TMP@EnsureCode{96}{12}% `
\edef\rc@AtEnd{\rc@AtEnd\noexpand\endinput}
%    \end{macrocode}
%
% \subsection{Loading packages}
%
%    \begin{macrocode}
\begingroup\expandafter\expandafter\expandafter\endgroup
\expandafter\ifx\csname RequirePackage\endcsname\relax
  \input ltxcmds.sty\relax
  \input infwarerr.sty\relax
\else
  \RequirePackage{ltxcmds}[2011/11/09]%
  \RequirePackage{infwarerr}[2010/04/08]%
\fi
%    \end{macrocode}
%
% \subsection{Defining commands}
%
%    \begin{macro}{\rc@IfDefinable}
%    \begin{macrocode}
\ltx@IfUndefined{@ifdefinable}{%
  \def\rc@IfDefinable#1{%
    \ifx#1\ltx@undefined
      \expandafter\ltx@firstofone
    \else
      \ifx#1\relax
        \expandafter\expandafter\expandafter\ltx@firstofone
      \else
        \@PackageError{refcount}{%
          Command \string#1 is already defined.\MessageBreak
          It will not redefined by this package%
        }\@ehc
        \expandafter\expandafter\expandafter\ltx@gobble
      \fi
    \fi
  }%
}{%
  \let\rc@IfDefinable\@ifdefinable
}
%    \end{macrocode}
%    \end{macro}
%
%    \begin{macro}{\rc@RobustDefOne}
%    \begin{macro}{\rc@RobustDefZero}
%    \begin{macrocode}
\ltx@IfUndefined{protected}{%
  \ltx@IfUndefined{DeclareRobustCommand}{%
    \def\rc@RobustDefOne#1#2#3#4{%
      \rc@IfDefinable#3{%
        #1\def#3##1{#4}%
      }%
    }%
    \def\rc@RobustDefZero#1#2{%
      \rc@IfDefinable#1{%
        \def#1{#2}%
      }%
    }%
  }{%
    \def\rc@RobustDefOne#1#2#3#4{%
      \rc@IfDefinable#3{%
        \DeclareRobustCommand#2#3[1]{#4}%
      }%
    }%
    \def\rc@RobustDefZero#1#2{%
      \rc@IfDefinable#1{%
        \DeclareRobustCommand#1{#2}%
      }%
    }%
  }%
}{%
  \def\rc@RobustDefOne#1#2#3#4{%
    \rc@IfDefinable#3{%
      \protected#1\def#3##1{#4}%
    }%
  }%
  \def\rc@RobustDefZero#1#2{%
    \rc@IfDefinable#1{%
      \protected\def#1{#2}%
    }%
  }%
}
%    \end{macrocode}
%    \end{macro}
%    \end{macro}
%
%    \begin{macro}{\rc@newcommand}
%    \begin{macrocode}
\ltx@IfUndefined{newcommand}{%
  \def\rc@newcommand*#1[#2]#3{% hash-ok
    \rc@IfDefinable#1{%
      \ifcase#2 %
        \def#1{#3}%
      \or
        \def#1##1{#3}%
      \or
        \def#1##1##2{#3}%
      \else
        \rc@InternalError
      \fi
    }%
  }%
}{%
  \let\rc@newcommand\newcommand
}
%    \end{macrocode}
%    \end{macro}
%
% \subsection{\cs{setrefcountdefault}}
%
%    \begin{macro}{\setrefcountdefault}
%    \begin{macrocode}
\rc@RobustDefOne\long{}\setrefcountdefault{%
  \def\rc@default{#1}%
}
%    \end{macrocode}
%    \end{macro}
%    \begin{macrocode}
\setrefcountdefault{0}
%    \end{macrocode}
%
% \subsection{\cs{refused}}
%
%    \begin{macro}{\refused}
%    \begin{macrocode}
\ltx@IfUndefined{G@refundefinedtrue}{%
  \rc@RobustDefOne{}{*}\refused{%
    \begingroup
      \csname @safe@activestrue\endcsname
      \ltx@IfUndefined{r@#1}{%
        \protect\G@refundefinedtrue
        \rc@WarningUndefined{#1}%
      }{}%
    \endgroup
  }%
}{%
  \rc@RobustDefOne{}{*}\refused{%
    \begingroup
      \csname @safe@activestrue\endcsname
      \ltx@IfUndefined{r@#1}{%
        \csname protect\expandafter\endcsname
        \csname G@refundefinedtrue\endcsname
        \rc@WarningUndefined{#1}%
      }{}%
    \endgroup
  }%
}
%    \end{macrocode}
%    \end{macro}
%    \begin{macro}{\rc@WarningUndefined}
%    \begin{macrocode}
\ltx@IfUndefined{@latex@warning}{%
  \def\rc@WarningUndefined#1{%
    \ltx@ifundefined{thepage}{%
      \def\thepage{\number\count0 }%
    }{}%
    \@PackageWarning{refcount}{%
      Reference `#1' on page \thepage\space undefined%
    }%
  }%
}{%
  \def\rc@WarningUndefined#1{%
    \@latex@warning{%
      Reference `#1' on page \thepage\space undefined%
    }%
  }%
}
%    \end{macrocode}
%    \end{macro}
%
% \subsection{Setting counters by reference data}
%
% \subsubsection{Generic setting}
%
%    \begin{macro}{\rc@set}
% Generic command for
% |\|$\{$|set|$,$|addto|$\}$|counter|$\{$|page|$,\}$|ref|:
%\begin{quote}
%\begin{tabular}[t]{@{}l@{: }l@{}}
% |#1|& \cs{setcounter}, \cs{addtocounter}\\
% |#2|& \cs{ltx@car} (for \cs{ref}), \cs{ltx@cartwo} (for \cs{pageref})\\
% |#3|& \hologo{LaTeX} counter\\
% |#4|& reference\\
%\end{tabular}
%\end{quote}
%    \begin{macrocode}
\def\rc@set#1#2#3#4{%
  \begingroup
    \csname @safe@activestrue\endcsname
    \refused{#4}%
    \expandafter\rc@@set\csname r@#4\endcsname{#1}{#2}{#3}%
  \endgroup
}
%    \end{macrocode}
%    \end{macro}
%    \begin{macro}{\rc@@set}
%\begin{quote}
%\begin{tabular}[t]{@{}l@{: }l@{}}
% |#1|& \cs{r@<...>}\\
% |#2|& \cs{setcounter}, \cs{addtocounter}\\
% |#3|& \cs{ltx@car} (for \cs{ref}), \cs{ltx@carsecond} (for \cs{pageref})\\
% |#4|& \hologo{LaTeX} counter\\
%\end{tabular}
%\end{quote}
%    \begin{macrocode}
\def\rc@@set#1#2#3#4{%
  \ifx#1\relax
    #2{#4}{\rc@default}%
  \else
    #2{#4}{%
      \expandafter#3#1\rc@default\rc@default\@nil
    }%
  \fi
}
%    \end{macrocode}
%    \end{macro}
%
% \subsubsection{User commands}
%
%    \begin{macro}{\setcounterref}
%    \begin{macrocode}
\rc@RobustDefZero\setcounterref{%
  \rc@set\setcounter\ltx@car
}
%    \end{macrocode}
%    \end{macro}
%    \begin{macro}{\addtocounterref}
%    \begin{macrocode}
\rc@RobustDefZero\addtocounterref{%
  \rc@set\addtocounter\ltx@car
}
%    \end{macrocode}
%    \end{macro}
%    \begin{macro}{\setcounterpageref}
%    \begin{macrocode}
\rc@RobustDefZero\setcounterpageref{%
  \rc@set\setcounter\ltx@carsecond
}
%    \end{macrocode}
%    \end{macro}
%    \begin{macro}{\addtocounterpageref}
%    \begin{macrocode}
\rc@RobustDefZero\addtocounterpageref{%
  \rc@set\addtocounter\ltx@carsecond
}
%    \end{macrocode}
%    \end{macro}
%
% \subsection{Extracting references}
%
%    \begin{macro}{\getrefnumber}
%    \begin{macrocode}
\rc@newcommand*{\getrefnumber}[1]{%
  \romannumeral
  \ltx@ifundefined{r@#1}{%
    \expandafter\ltx@zero
    \rc@default
  }{%
    \expandafter\expandafter\expandafter\rc@extract@
    \expandafter\expandafter\expandafter!%
    \csname r@#1\expandafter\endcsname
    \expandafter{\rc@default}\@nil
  }%
}
%    \end{macrocode}
%    \end{macro}
%    \begin{macro}{\getpagerefnumber}
%    \begin{macrocode}
\rc@newcommand*{\getpagerefnumber}[1]{%
  \romannumeral
  \ltx@ifundefined{r@#1}{%
    \expandafter\ltx@zero
    \rc@default
  }{%
    \expandafter\expandafter\expandafter\rc@extract@page
    \expandafter\expandafter\expandafter!%
    \csname r@#1\expandafter\expandafter\expandafter\endcsname
    \expandafter\expandafter\expandafter{%
      \expandafter\rc@default
    \expandafter}\expandafter{\rc@default}\@nil
  }%
}
%    \end{macrocode}
%    \end{macro}
%    \begin{macro}{\getrefbykeydefault}
%    \begin{macrocode}
\rc@newcommand*{\getrefbykeydefault}[2]{%
  \romannumeral
  \expandafter\rc@getrefbykeydefault
    \csname r@#1\expandafter\endcsname
    \csname rc@extract@#2\endcsname
}
%    \end{macrocode}
%    \end{macro}
%    \begin{macro}{\rc@getrefbykeydefault}
%\begin{quote}
%\begin{tabular}[t]{@{}l@{: }l@{}}
% |#1|& \cs{r@<...>}\\
% |#2|& \cs{rc@extract@<...>}\\
% |#3|& default\\
%\end{tabular}
%\end{quote}
%    \begin{macrocode}
\long\def\rc@getrefbykeydefault#1#2#3{%
  \ifx#1\relax
    % reference is undefined
    \ltx@ReturnAfterElseFi{%
      \ltx@zero
      #3%
    }%
  \else
    \ltx@ReturnAfterFi{%
      \ifx#2\relax
        % extract method is missing
        \ltx@ReturnAfterElseFi{%
          \ltx@zero
          #3%
        }%
      \else
        \ltx@ReturnAfterFi{%
          \expandafter
          \rc@generic#1{#3}{#3}{#3}{#3}{#3}\@nil#2{#3}%
        }%
      \fi
    }%
  \fi
}
%    \end{macrocode}
%    \end{macro}
%    \begin{macro}{\rc@generic}
%\begin{quote}
%\begin{tabular}[t]{@{}l@{: }l@{}}
% |#1|& first item in \cs{r@<...>}\\
% |#2|& remaining items in \cs{r@<...>}\\
% |#3|& \cs{rc@extract@<...>}\\
% |#4|& default\\
%\end{tabular}
%\end{quote}
%    \begin{macrocode}
\long\def\rc@generic#1#2\@nil#3#4{%
  #3{#1\TR@TitleReference\@empty{#4}\@nil}{#1}#2\@nil
}
%    \end{macrocode}
%    \end{macro}
%    \begin{macro}{\rc@extract@}
%    \begin{macrocode}
\long\def\rc@extract@#1#2#3\@nil{%
  \ltx@zero
  #2%
}
%    \end{macrocode}
%    \end{macro}
%    \begin{macro}{\rc@extract@page}
%    \begin{macrocode}
\long\def\rc@extract@page#1#2#3#4\@nil{%
  \ltx@zero
  #3%
}
%    \end{macrocode}
%    \end{macro}
%    \begin{macro}{\rc@extract@name}
%    \begin{macrocode}
\long\def\rc@extract@name#1#2#3#4#5\@nil{%
  \ltx@zero
  #4%
}
%    \end{macrocode}
%    \end{macro}
%    \begin{macro}{\rc@extract@anchor}
%    \begin{macrocode}
\long\def\rc@extract@anchor#1#2#3#4#5#6\@nil{%
  \ltx@zero
  #5%
}
%    \end{macrocode}
%    \end{macro}
%    \begin{macro}{\rc@extract@url}
%    \begin{macrocode}
\long\def\rc@extract@url#1#2#3#4#5#6#7\@nil{%
  \ltx@zero
  #6%
}
%    \end{macrocode}
%    \end{macro}
%    \begin{macro}{\rc@extract@title}
%    \begin{macrocode}
\long\def\rc@extract@title#1#2\@nil{%
  \rc@@extract@title#1%
}
%    \end{macrocode}
%    \end{macro}
%    \begin{macro}{\rc@@extract@title}
%    \begin{macrocode}
\long\def\rc@@extract@title#1\TR@TitleReference#2#3#4\@nil{%
  \ltx@zero
  #3%
}
%    \end{macrocode}
%    \end{macro}
%
% \subsection{Macros for checking undefined references}
%
%    \begin{macro}{\IfRefUndefinedExpandable}
%    \begin{macrocode}
\rc@newcommand*{\IfRefUndefinedExpandable}[1]{%
  \ltx@ifundefined{r@#1}\ltx@firstoftwo\ltx@secondoftwo
}
%    \end{macrocode}
%    \end{macro}
%    \begin{macro}{\IfRefUndefinedBabel}
%    \begin{macrocode}
\rc@RobustDefOne{}*\IfRefUndefinedBabel{%
  \begingroup
    \csname safe@actives@true\endcsname
  \expandafter\expandafter\expandafter\endgroup
  \expandafter\ifx\csname r@#1\endcsname\relax
    \expandafter\ltx@firstoftwo
  \else
    \expandafter\ltx@secondoftwo
  \fi
}
%    \end{macrocode}
%    \end{macro}
%    \begin{macrocode}
\rc@AtEnd%
%</package>
%    \end{macrocode}
%
% \section{Test}
%
% \subsection{Catcode checks for loading}
%
%    \begin{macrocode}
%<*test1>
%    \end{macrocode}
%    \begin{macrocode}
\catcode`\{=1 %
\catcode`\}=2 %
\catcode`\#=6 %
\catcode`\@=11 %
\expandafter\ifx\csname count@\endcsname\relax
  \countdef\count@=255 %
\fi
\expandafter\ifx\csname @gobble\endcsname\relax
  \long\def\@gobble#1{}%
\fi
\expandafter\ifx\csname @firstofone\endcsname\relax
  \long\def\@firstofone#1{#1}%
\fi
\expandafter\ifx\csname loop\endcsname\relax
  \expandafter\@firstofone
\else
  \expandafter\@gobble
\fi
{%
  \def\loop#1\repeat{%
    \def\body{#1}%
    \iterate
  }%
  \def\iterate{%
    \body
      \let\next\iterate
    \else
      \let\next\relax
    \fi
    \next
  }%
  \let\repeat=\fi
}%
\def\RestoreCatcodes{}
\count@=0 %
\loop
  \edef\RestoreCatcodes{%
    \RestoreCatcodes
    \catcode\the\count@=\the\catcode\count@\relax
  }%
\ifnum\count@<255 %
  \advance\count@ 1 %
\repeat

\def\RangeCatcodeInvalid#1#2{%
  \count@=#1\relax
  \loop
    \catcode\count@=15 %
  \ifnum\count@<#2\relax
    \advance\count@ 1 %
  \repeat
}
\def\RangeCatcodeCheck#1#2#3{%
  \count@=#1\relax
  \loop
    \ifnum#3=\catcode\count@
    \else
      \errmessage{%
        Character \the\count@\space
        with wrong catcode \the\catcode\count@\space
        instead of \number#3%
      }%
    \fi
  \ifnum\count@<#2\relax
    \advance\count@ 1 %
  \repeat
}
\def\space{ }
\expandafter\ifx\csname LoadCommand\endcsname\relax
  \def\LoadCommand{\input refcount.sty\relax}%
\fi
\def\Test{%
  \RangeCatcodeInvalid{0}{47}%
  \RangeCatcodeInvalid{58}{64}%
  \RangeCatcodeInvalid{91}{96}%
  \RangeCatcodeInvalid{123}{255}%
  \catcode`\@=12 %
  \catcode`\\=0 %
  \catcode`\%=14 %
  \LoadCommand
  \RangeCatcodeCheck{0}{36}{15}%
  \RangeCatcodeCheck{37}{37}{14}%
  \RangeCatcodeCheck{38}{47}{15}%
  \RangeCatcodeCheck{48}{57}{12}%
  \RangeCatcodeCheck{58}{63}{15}%
  \RangeCatcodeCheck{64}{64}{12}%
  \RangeCatcodeCheck{65}{90}{11}%
  \RangeCatcodeCheck{91}{91}{15}%
  \RangeCatcodeCheck{92}{92}{0}%
  \RangeCatcodeCheck{93}{96}{15}%
  \RangeCatcodeCheck{97}{122}{11}%
  \RangeCatcodeCheck{123}{255}{15}%
  \RestoreCatcodes
}
\Test
\csname @@end\endcsname
\end
%    \end{macrocode}
%    \begin{macrocode}
%</test1>
%    \end{macrocode}
%
% \subsection{Macro tests}
%
%    \begin{macrocode}
%<*test2>
\errorcontextlines=10000 %
\showboxbreadth=10000 %
\showboxdepth=10000 %
\begingroup\expandafter\expandafter\expandafter\endgroup
\expandafter\ifx\csname RequirePackage\endcsname\relax
  \input refcount.sty\relax
\else
  \RequirePackage{refcount}[2011/10/16]%
\fi
\catcode`\@=11 %
\begingroup\expandafter\expandafter\expandafter\endgroup
\expandafter\ifx\csname @onelevel@sanitize\endcsname\relax
  \begingroup\expandafter\expandafter\expandafter\endgroup
  \expandafter\ifx\csname detokenize\endcsname\relax
    \def\strip@prefix#1->{}%
    \def\@onelevel@sanitize#1{%
      \edef#1{%
        \expandafter\strip@prefix\meaning#1%
      }%
    }%
  \else
    \def\@onelevel@sanitize#1{%
      \edef#1{%
        \detokenize\expandafter{#1}%
      }%
    }%
  \fi
\fi
\def\msg#{\immediate\write16}
\def\empty{}
\def\space{ }
%    \end{macrocode}
%    \begin{macrocode}
\def\r@foo{{\empty 1}{\empty 2}}
\long\def\test#1#2{%
  \begingroup
    \setbox0=\hbox{%
      \def\TestTask{#1}%
      \@onelevel@sanitize\TestTask
      \msg{* \TestTask}%
      \expandafter\expandafter\expandafter\def
      \expandafter\expandafter\expandafter\TestResult
      \expandafter\expandafter\expandafter{%
        #1%
      }%
      \def\TestExpected{#2}%
      \ifx\TestResult\TestExpected
        \msg{ \space ok.}%
      \else
        \@onelevel@sanitize\TestResult
        \@onelevel@sanitize\TestExpected
        \msg{ \space Result: \space\space[\TestResult]}%
        \msg{ \space Expected: [\TestExpected]}%
        \errmessage{Test failed!}%
      \fi
    }%
    \ifdim\wd0=0pt %
    \else
      \showbox0 %
    \fi
  \endgroup
}
\test{\getrefnumber{foo}}{\empty 1}
\test{\getpagerefnumber{foo}}{\empty 2}
\test{\getrefbykeydefault{foo}{}{\empty default}}{\empty 1}
\test{\getrefbykeydefault{foo}{page}{\empty default}}{\empty 2}
\test{\getrefbykeydefault{foo}{name}{\empty default}}{\empty default}
\test{\getrefbykeydefault{foo}{anchor}{\empty default}}{\empty default}
\test{\getrefbykeydefault{foo}{url}{\empty default}}{\empty default}
\test{\getrefbykeydefault{foo}{title}{\empty default}}{\empty default}
\msg{}
\def\r@foo{{}{}{}{}{}{}{}{}{}{}}
\def\Test#1#2\\{%
  \test{#1{foo}#2}{}%
}
\def\TestGroup{%
  \Test\getrefnumber\\%
  \Test\getpagerefnumber\\%
  \Test\getrefbykeydefault{}{}\\%
  \Test\getrefbykeydefault{page}{}\\%
  \Test\getrefbykeydefault{anchor}{}\\%
  \Test\getrefbykeydefault{name}{}\\%
  \Test\getrefbykeydefault{url}{}\\%
}
\TestGroup
\Test\getrefbykeydefault{title}{}\\%
\msg{}
\def\r@foo{\par\par\par\par\par\par\par\par}
\long\def\Test#1#2\\{%
  \test{#1{foo}#2}{\par}%
}
\TestGroup
\test{\getrefbykeydefault{title}{}{}}{}
\msg{}
\def\r@foo{{ }{ }{ }{ }{ }}
\def\Test#1#2\\{%
  \test{#1{foo}#2}{ }%
}
\TestGroup
\msg{}
\long\def\TestDefault#1{%
  \begingroup
    \setrefcountdefault{#1}%
    \test{\getrefnumber{foo}}{#1}%
    \test{\getpagerefnumber{foo}}{#1}%
  \endgroup
}
\def\TestDefaultX{%
  \TestDefault{}%
  \TestDefault{\par}%
  \TestDefault{ }%
  \TestDefault{\space}%
}
\let\r@foo\@undefined
\TestDefaultX
\let\r@foo\relax
\TestDefaultX
\def\r@foo{}
\TestDefaultX
%    \end{macrocode}
%    \begin{macrocode}
\msg{}
\long\def\Test#1#2#3#4{%
  \begingroup
    \def\TestTask{#1}%
    \@onelevel@sanitize\TestTask
    \msg{* [\TestTask]}%
    \edef\TestResultA{\IfRefUndefinedExpandable{#1}{#2}{#3}}%
    \IfRefUndefinedBabel{#1}{%
      \def\TestResultB{#2}%
    }{%
      \def\TestResultB{#3}%
    }%
    \def\TestExpected{#4}%
    \ifx\TestResultA\TestExpected
      \msg{ \space ok.}%
    \else
      \begingroup
        \@onelevel@sanitize\TestResultA
        \@onelevel@sanitize\TestExpected
        \msg{ \space Result: \space\space[\TestResultA]}%
        \msg{ \space Expected: [\TestExpected]}%
        \errmessage{Test failed!}%
      \endgroup
    \fi
    \ifx\TestResultB\TestExpected
      \msg{ \space ok.}%
    \else
      \begingroup
        \@onelevel@sanitize\TestResultB
        \@onelevel@sanitize\TestExpected
        \msg{ \space Result: \space\space[\TestResultB]}%
        \msg{ \space Expected: [\TestExpected]}%
        \errmessage{Test failed!}%
      \endgroup
    \fi
  \endgroup
}
\begingroup
  \def\r@foo{{}{}}%
  \let\r@bar\@undefined
  \let\r@xyz\relax
  \Test{foo}{true}{false}{false}%
  \Test{bar}{true}{false}{true}%
  \Test{xyz}{true}{false}{true}%
\endgroup
%    \end{macrocode}
%    \begin{macrocode}
\csname @@end\endcsname\end
%</test2>
%    \end{macrocode}
% \subsection{Test with package \xpackage{titleref}}
%
%    \begin{macrocode}
%<*test3>
\NeedsTeXFormat{LaTeX2e}
\documentclass{article}
\usepackage{refcount}[2011/10/16]
%<test4>\usepackage{nameref}
\usepackage{titleref}
\begin{document}
\section{Hello World}
\label{sec:hello}
\section{\hbox{xy}}
\label{sec:foo}
%
\makeatletter
\@ifundefined{r@sec:hello}{%
  \typeout{==> Compile twice!}%
}{%
  \def\test#1#2{%
    \begingroup
      \def\TestTask{#1}%
      \@onelevel@sanitize\TestTask
      \typeout{* \TestTask}%
      \expandafter\expandafter\expandafter\def
      \expandafter\expandafter\expandafter\TestResult
      \expandafter\expandafter\expandafter{%
        #1%
      }%
      \def\TestExpected{#2}%
      \ifx\TestResult\TestExpected
        \typeout{ \space ok.}%
      \else
        \@onelevel@sanitize\TestResult
        \@onelevel@sanitize\TestExpected
        \typeout{ \space Result: \space\space[\TestResult]}%
        \typeout{ \space Expected: [\TestExpected]}%
        \errmessage{Test failed!}%
      \fi
    \endgroup
  }%
  \test{\getrefbykeydefault{sec:hello}{title}{}}{Hello World}%
  \test{\getrefbykeydefault{sec:foo}{title}{}}{\hbox{xy}}%
  \begingroup
    \def\hbox#1{[#1]}% hash-ok
    \test{\getrefbykeydefault{sec:foo}{title}{}}{\hbox{xy}}%
  \endgroup
}
\makeatother
%    \end{macrocode}
%    \begin{macrocode}
\end{document}
%</test3>
%    \end{macrocode}
%    \begin{macrocode}
%<*test5>
\NeedsTeXFormat{LaTeX2e}
\documentclass{book}
\usepackage{refcount}[2011/10/16]
\usepackage{zref-runs}
\newcounter{test}
\begin{document}
\ifnum\zruns>1 %
  \makeatletter
  \def\Test#1#2#3{%
    \begingroup
      \setcounter{test}{10}%
      \sbox0{%
        #1{test}{#2}%
        \ifnum#3=\value{test}%
        \else
          \PackageError{test}{\string#1{#2} <> #3 (\the\value{test})}%
        \fi
      }%
      \ifdim\wd0=0pt %
      \else
        \PackageError{test}{Non-empty box}\@ehc
      \fi
    \endgroup
  }%
  \makeatother
  \Test\setcounterpageref{ch:two}{1}%
  \Test\setcounterpageref{ch:three}{3}%
  \Test\setcounterpageref{ch:four}{5}%
  \Test\setcounterpageref{ch:five}{7}%
  \Test\setcounterpageref{ch:six}{9}%
  \Test\setcounterpageref{ch:seven}{13}%
  \Test\addtocounterpageref{ch:two}{11}%
  \Test\addtocounterpageref{ch:three}{13}%
  \Test\addtocounterpageref{ch:four}{15}%
  \Test\addtocounterpageref{ch:five}{17}%
  \Test\addtocounterpageref{ch:six}{19}%
  \Test\addtocounterpageref{ch:seven}{23}%
  \Test\setcounterref{ch:two}{1}%
  \Test\setcounterref{ch:three}{2}%
  \Test\setcounterref{ch:four}{11}%
  \Test\addtocounterref{ch:two}{11}%
  \Test\addtocounterref{ch:three}{12}%
  \Test\addtocounterref{ch:four}{21}%
\fi
\frontmatter
\chapter{Chapter one}\label{ch:one}
\cleardoublepage
\mainmatter
\chapter{Chapter two}\label{ch:two}
\cleardoublepage
\chapter{Chapter three}\label{ch:three}
\cleardoublepage
\setcounter{chapter}{10}
\chapter{Chapter four}\label{ch:four}
\cleardoublepage
\appendix
\chapter{Chapter five}\label{ch:five}
\cleardoublepage
\chapter{Chapter six}\label{ch:six}
\cleardoublepage
\null
\cleardoublepage
\chapter{Chapter seven}\label{ch:seven}
\end{document}
%</test5>
%    \end{macrocode}
%
% \section{Installation}
%
% \subsection{Download}
%
% \paragraph{Package.} This package is available on
% CTAN\footnote{\url{ftp://ftp.ctan.org/tex-archive/}}:
% \begin{description}
% \item[\CTAN{macros/latex/contrib/oberdiek/refcount.dtx}] The source file.
% \item[\CTAN{macros/latex/contrib/oberdiek/refcount.pdf}] Documentation.
% \end{description}
%
%
% \paragraph{Bundle.} All the packages of the bundle `oberdiek'
% are also available in a TDS compliant ZIP archive. There
% the packages are already unpacked and the documentation files
% are generated. The files and directories obey the TDS standard.
% \begin{description}
% \item[\CTAN{install/macros/latex/contrib/oberdiek.tds.zip}]
% \end{description}
% \emph{TDS} refers to the standard ``A Directory Structure
% for \TeX\ Files'' (\CTAN{tds/tds.pdf}). Directories
% with \xfile{texmf} in their name are usually organized this way.
%
% \subsection{Bundle installation}
%
% \paragraph{Unpacking.} Unpack the \xfile{oberdiek.tds.zip} in the
% TDS tree (also known as \xfile{texmf} tree) of your choice.
% Example (linux):
% \begin{quote}
%   |unzip oberdiek.tds.zip -d ~/texmf|
% \end{quote}
%
% \paragraph{Script installation.}
% Check the directory \xfile{TDS:scripts/oberdiek/} for
% scripts that need further installation steps.
% Package \xpackage{attachfile2} comes with the Perl script
% \xfile{pdfatfi.pl} that should be installed in such a way
% that it can be called as \texttt{pdfatfi}.
% Example (linux):
% \begin{quote}
%   |chmod +x scripts/oberdiek/pdfatfi.pl|\\
%   |cp scripts/oberdiek/pdfatfi.pl /usr/local/bin/|
% \end{quote}
%
% \subsection{Package installation}
%
% \paragraph{Unpacking.} The \xfile{.dtx} file is a self-extracting
% \docstrip\ archive. The files are extracted by running the
% \xfile{.dtx} through \plainTeX:
% \begin{quote}
%   \verb|tex refcount.dtx|
% \end{quote}
%
% \paragraph{TDS.} Now the different files must be moved into
% the different directories in your installation TDS tree
% (also known as \xfile{texmf} tree):
% \begin{quote}
% \def\t{^^A
% \begin{tabular}{@{}>{\ttfamily}l@{ $\rightarrow$ }>{\ttfamily}l@{}}
%   refcount.sty & tex/latex/oberdiek/refcount.sty\\
%   refcount.pdf & doc/latex/oberdiek/refcount.pdf\\
%   test/refcount-test1.tex & doc/latex/oberdiek/test/refcount-test1.tex\\
%   test/refcount-test2.tex & doc/latex/oberdiek/test/refcount-test2.tex\\
%   test/refcount-test3.tex & doc/latex/oberdiek/test/refcount-test3.tex\\
%   test/refcount-test4.tex & doc/latex/oberdiek/test/refcount-test4.tex\\
%   test/refcount-test5.tex & doc/latex/oberdiek/test/refcount-test5.tex\\
%   refcount.dtx & source/latex/oberdiek/refcount.dtx\\
% \end{tabular}^^A
% }^^A
% \sbox0{\t}^^A
% \ifdim\wd0>\linewidth
%   \begingroup
%     \advance\linewidth by\leftmargin
%     \advance\linewidth by\rightmargin
%   \edef\x{\endgroup
%     \def\noexpand\lw{\the\linewidth}^^A
%   }\x
%   \def\lwbox{^^A
%     \leavevmode
%     \hbox to \linewidth{^^A
%       \kern-\leftmargin\relax
%       \hss
%       \usebox0
%       \hss
%       \kern-\rightmargin\relax
%     }^^A
%   }^^A
%   \ifdim\wd0>\lw
%     \sbox0{\small\t}^^A
%     \ifdim\wd0>\linewidth
%       \ifdim\wd0>\lw
%         \sbox0{\footnotesize\t}^^A
%         \ifdim\wd0>\linewidth
%           \ifdim\wd0>\lw
%             \sbox0{\scriptsize\t}^^A
%             \ifdim\wd0>\linewidth
%               \ifdim\wd0>\lw
%                 \sbox0{\tiny\t}^^A
%                 \ifdim\wd0>\linewidth
%                   \lwbox
%                 \else
%                   \usebox0
%                 \fi
%               \else
%                 \lwbox
%               \fi
%             \else
%               \usebox0
%             \fi
%           \else
%             \lwbox
%           \fi
%         \else
%           \usebox0
%         \fi
%       \else
%         \lwbox
%       \fi
%     \else
%       \usebox0
%     \fi
%   \else
%     \lwbox
%   \fi
% \else
%   \usebox0
% \fi
% \end{quote}
% If you have a \xfile{docstrip.cfg} that configures and enables \docstrip's
% TDS installing feature, then some files can already be in the right
% place, see the documentation of \docstrip.
%
% \subsection{Refresh file name databases}
%
% If your \TeX~distribution
% (\teTeX, \mikTeX, \dots) relies on file name databases, you must refresh
% these. For example, \teTeX\ users run \verb|texhash| or
% \verb|mktexlsr|.
%
% \subsection{Some details for the interested}
%
% \paragraph{Attached source.}
%
% The PDF documentation on CTAN also includes the
% \xfile{.dtx} source file. It can be extracted by
% AcrobatReader 6 or higher. Another option is \textsf{pdftk},
% e.g. unpack the file into the current directory:
% \begin{quote}
%   \verb|pdftk refcount.pdf unpack_files output .|
% \end{quote}
%
% \paragraph{Unpacking with \LaTeX.}
% The \xfile{.dtx} chooses its action depending on the format:
% \begin{description}
% \item[\plainTeX:] Run \docstrip\ and extract the files.
% \item[\LaTeX:] Generate the documentation.
% \end{description}
% If you insist on using \LaTeX\ for \docstrip\ (really,
% \docstrip\ does not need \LaTeX), then inform the autodetect routine
% about your intention:
% \begin{quote}
%   \verb|latex \let\install=y% \iffalse meta-comment
%
% File: refcount.dtx
% Version: 2011/10/16 v3.4
% Info: Data extraction from label references
%
% Copyright (C) 1998, 2000, 2006, 2008, 2010, 2011 by
%    Heiko Oberdiek <heiko.oberdiek at googlemail.com>
%
% This work may be distributed and/or modified under the
% conditions of the LaTeX Project Public License, either
% version 1.3c of this license or (at your option) any later
% version. This version of this license is in
%    http://www.latex-project.org/lppl/lppl-1-3c.txt
% and the latest version of this license is in
%    http://www.latex-project.org/lppl.txt
% and version 1.3 or later is part of all distributions of
% LaTeX version 2005/12/01 or later.
%
% This work has the LPPL maintenance status "maintained".
%
% This Current Maintainer of this work is Heiko Oberdiek.
%
% This work consists of the main source file refcount.dtx
% and the derived files
%    refcount.sty, refcount.pdf, refcount.ins, refcount.drv,
%    refcount-test1.tex, refcount-test2.tex, refcount-test3.tex,
%    refcount-test4.tex, refcount-test5.tex.
%
% Distribution:
%    CTAN:macros/latex/contrib/oberdiek/refcount.dtx
%    CTAN:macros/latex/contrib/oberdiek/refcount.pdf
%
% Unpacking:
%    (a) If refcount.ins is present:
%           tex refcount.ins
%    (b) Without refcount.ins:
%           tex refcount.dtx
%    (c) If you insist on using LaTeX
%           latex \let\install=y\input{refcount.dtx}
%        (quote the arguments according to the demands of your shell)
%
% Documentation:
%    (a) If refcount.drv is present:
%           latex refcount.drv
%    (b) Without refcount.drv:
%           latex refcount.dtx; ...
%    The class ltxdoc loads the configuration file ltxdoc.cfg
%    if available. Here you can specify further options, e.g.
%    use A4 as paper format:
%       \PassOptionsToClass{a4paper}{article}
%
%    Programm calls to get the documentation (example):
%       pdflatex refcount.dtx
%       makeindex -s gind.ist refcount.idx
%       pdflatex refcount.dtx
%       makeindex -s gind.ist refcount.idx
%       pdflatex refcount.dtx
%
% Installation:
%    TDS:tex/latex/oberdiek/refcount.sty
%    TDS:doc/latex/oberdiek/refcount.pdf
%    TDS:doc/latex/oberdiek/test/refcount-test1.tex
%    TDS:doc/latex/oberdiek/test/refcount-test2.tex
%    TDS:doc/latex/oberdiek/test/refcount-test3.tex
%    TDS:doc/latex/oberdiek/test/refcount-test4.tex
%    TDS:doc/latex/oberdiek/test/refcount-test5.tex
%    TDS:source/latex/oberdiek/refcount.dtx
%
%<*ignore>
\begingroup
  \catcode123=1 %
  \catcode125=2 %
  \def\x{LaTeX2e}%
\expandafter\endgroup
\ifcase 0\ifx\install y1\fi\expandafter
         \ifx\csname processbatchFile\endcsname\relax\else1\fi
         \ifx\fmtname\x\else 1\fi\relax
\else\csname fi\endcsname
%</ignore>
%<*install>
\input docstrip.tex
\Msg{************************************************************************}
\Msg{* Installation}
\Msg{* Package: refcount 2011/10/16 v3.4 Data extraction from label references (HO)}
\Msg{************************************************************************}

\keepsilent
\askforoverwritefalse

\let\MetaPrefix\relax
\preamble

This is a generated file.

Project: refcount
Version: 2011/10/16 v3.4

Copyright (C) 1998, 2000, 2006, 2008, 2010, 2011 by
   Heiko Oberdiek <heiko.oberdiek at googlemail.com>

This work may be distributed and/or modified under the
conditions of the LaTeX Project Public License, either
version 1.3c of this license or (at your option) any later
version. This version of this license is in
   http://www.latex-project.org/lppl/lppl-1-3c.txt
and the latest version of this license is in
   http://www.latex-project.org/lppl.txt
and version 1.3 or later is part of all distributions of
LaTeX version 2005/12/01 or later.

This work has the LPPL maintenance status "maintained".

This Current Maintainer of this work is Heiko Oberdiek.

This work consists of the main source file refcount.dtx
and the derived files
   refcount.sty, refcount.pdf, refcount.ins, refcount.drv,
   refcount-test1.tex, refcount-test2.tex, refcount-test3.tex,
   refcount-test4.tex, refcount-test5.tex.

\endpreamble
\let\MetaPrefix\DoubleperCent

\generate{%
  \file{refcount.ins}{\from{refcount.dtx}{install}}%
  \file{refcount.drv}{\from{refcount.dtx}{driver}}%
  \usedir{tex/latex/oberdiek}%
  \file{refcount.sty}{\from{refcount.dtx}{package}}%
  \usedir{doc/latex/oberdiek/test}%
  \file{refcount-test1.tex}{\from{refcount.dtx}{test1}}%
  \file{refcount-test2.tex}{\from{refcount.dtx}{test2}}%
  \file{refcount-test3.tex}{\from{refcount.dtx}{test3}}%
  \file{refcount-test4.tex}{\from{refcount.dtx}{test3,test4}}%
  \file{refcount-test5.tex}{\from{refcount.dtx}{test5}}%
  \nopreamble
  \nopostamble
  \usedir{source/latex/oberdiek/catalogue}%
  \file{refcount.xml}{\from{refcount.dtx}{catalogue}}%
}

\catcode32=13\relax% active space
\let =\space%
\Msg{************************************************************************}
\Msg{*}
\Msg{* To finish the installation you have to move the following}
\Msg{* file into a directory searched by TeX:}
\Msg{*}
\Msg{*     refcount.sty}
\Msg{*}
\Msg{* To produce the documentation run the file `refcount.drv'}
\Msg{* through LaTeX.}
\Msg{*}
\Msg{* Happy TeXing!}
\Msg{*}
\Msg{************************************************************************}

\endbatchfile
%</install>
%<*ignore>
\fi
%</ignore>
%<*driver>
\NeedsTeXFormat{LaTeX2e}
\ProvidesFile{refcount.drv}%
  [2011/10/16 v3.4 Data extraction from label references (HO)]%
\documentclass{ltxdoc}
\usepackage{holtxdoc}[2011/11/22]
\begin{document}
  \DocInput{refcount.dtx}%
\end{document}
%</driver>
% \fi
%
% \CheckSum{1185}
%
% \CharacterTable
%  {Upper-case    \A\B\C\D\E\F\G\H\I\J\K\L\M\N\O\P\Q\R\S\T\U\V\W\X\Y\Z
%   Lower-case    \a\b\c\d\e\f\g\h\i\j\k\l\m\n\o\p\q\r\s\t\u\v\w\x\y\z
%   Digits        \0\1\2\3\4\5\6\7\8\9
%   Exclamation   \!     Double quote  \"     Hash (number) \#
%   Dollar        \$     Percent       \%     Ampersand     \&
%   Acute accent  \'     Left paren    \(     Right paren   \)
%   Asterisk      \*     Plus          \+     Comma         \,
%   Minus         \-     Point         \.     Solidus       \/
%   Colon         \:     Semicolon     \;     Less than     \<
%   Equals        \=     Greater than  \>     Question mark \?
%   Commercial at \@     Left bracket  \[     Backslash     \\
%   Right bracket \]     Circumflex    \^     Underscore    \_
%   Grave accent  \`     Left brace    \{     Vertical bar  \|
%   Right brace   \}     Tilde         \~}
%
% \GetFileInfo{refcount.drv}
%
% \title{The \xpackage{refcount} package}
% \date{2011/10/16 v3.4}
% \author{Heiko Oberdiek\\\xemail{heiko.oberdiek at googlemail.com}}
%
% \maketitle
%
% \begin{abstract}
% References are not numbers, however they often store numerical
% data such as section or page numbers. \cs{ref} or \cs{pageref}
% cannot be used for counter assignments or calculations because
% they are not expandable, generate warnings, or can even be links.
% The package provides expandable macros to extract the data
% from references. Packages \xpackage{hyperref}, \xpackage{nameref},
% \xpackage{titleref}, and \xpackage{babel} are supported.
% \end{abstract}
%
% \tableofcontents
%
% \section{Usage}
%
% \subsection{Setting counters}
%
% The following commands are similar to \LaTeX's
% \cs{setcounter} and \cs{addtocounter},
% but they extract the number value from a reference:
% \begin{quote}
%   \cs{setcounterref}, \cs{addtocounterref}\\
%   \cs{setcounterpageref}, \cs{addtocounterpageref}
% \end{quote}
% They take two arguments:
% \begin{quote}
%    \cs{...counter...ref} |{|\meta{\LaTeX\ counter}|}|
%    |{|\meta{reference}|}|
% \end{quote}
% An undefined references produces the usual LaTeX warning
% and its value is assumed to be zero.
% Example:
% \begin{quote}
%\begin{verbatim}
%\newcounter{ctrA}
%\newcounter{ctrB}
%\refstepcounter{ctrA}\label{ref:A}
%\setcounterref{ctrB}{ref:A}
%\addtocounterpageref{ctrB}{ref:A}
%\end{verbatim}
% \end{quote}
%
% \subsection{Expandable commands}
%
% These commands that can be used in expandible contexts
% (inside calculations, \cs{edef}, \cs{csname}, \cs{write}, \dots):
% \begin{quote}
%   \cs{getrefnumber}, \cs{getpagerefnumber}
% \end{quote}
% They take one argument, the reference:
% \begin{quote}
%   \cs{get...refnumber} |{|\meta{reference}|}|
% \end{quote}
% The default for undefined references can be changed
% with macro \cs{setrefcountdefault}, for example this
% package calls:
% \begin{quote}
%   \cs{setrefcountdefault}|{0}|
% \end{quote}
%
% Since version 2.0 of this package there is a new
% command:
% \begin{quote}
%   \cs{getrefbykeydefault} |{|\meta{reference}|}|
%   |{|\meta{key}|}| |{|\meta{default}|}|
% \end{quote}
% This generalized version allows the extraction
% of further properties of a reference than the
% two standard ones. Thus the following properties
% are supported, if they are available:
% \begin{quote}
% \begin{tabular}{@{}l|l|l@{}}
%    Key & Description & Package\\
% \hline
%   \meta{empty} & same as \cs{ref} & \LaTeX\\
%   |page| & same as \cs{pageref} & \LaTeX\\
%   |title| & section and caption titles & \xpackage{titleref}\\
%   |name| & section and caption titles & \xpackage{nameref}\\
%   |anchor| & anchor name & \xpackage{hyperref}\\
%   |url| & url/file & \xpackage{hyperref}/\xpackage{xr}
% \end{tabular}
% \end{quote}
%
% Since version 3.2 the expandable macros described before
% in this section
% are expandable in exact two expansion steps.
%
% \subsection{Undefined references}
%
% Because warnings and assignments cannot be used in
% expandible contexts, undefined references do not
% produce a warning, their values are assumed to be zero.
% Example:
% \begin{quote}
%\begin{verbatim}
%\label{ref:here}% somewhere
%\refused{ref:here}% see below
%\ifodd\getpagerefnumber{ref:here}%
%  reference is on an odd page
%\else
%  reference is on an even page
%\fi
%\end{verbatim}
% \end{quote}
%
% In case of undefined references the user usually want's
% to be informed. Also \LaTeX\ prints a warning at
% the end of the \LaTeX\ run. To notify \LaTeX\ and
% get a normal warning, just use
% \begin{quote}
%   \cs{refused} |{|\meta{reference}|}|
% \end{quote}
% outside the expanding context. Example, see above.
%
% \subsubsection{Check for undefined references}
%
% In version 3.2 macros were added, that test, whether references
% are defined.
% \begin{declcs}{IfRefUndefinedExpandable} \M{refname} \M{then} \M{else}\\
%   \cs{IfRefUndefinedBabel} \M{refname} \M{then} \M{else}
% \end{declcs}
% If the reference is not available and therefore undefined, then
% argument \meta{then} is executed, otherwise argument \meta{else}
% is called. Macro \cs{IfRefUndefinedExpandable} is expandable,
% but \meta{refname} must not contain babel shorthand characters.
% Macro \cs{IfRefUndefinedBabel} supports shorthand characters of
% babel, but it is not expandable.
%
% \subsection{Notes}
%
% \begin{itemize}
% \item
%   The method of extracting the number in this
%   package also works in cases, where the
%   reference cannot be used directly, because
%   a package such as \xpackage{hyperref} has added
%   extra stuff (hyper link), so that the reference cannot
%   be used as number any more.
% \item
%   If the reference does not contain a number,
%   assignments to a counter will fail of course.
% \end{itemize}
%
%
% \StopEventually{
% }
%
% \section{Implementation}
%
%    \begin{macrocode}
%<*package>
%    \end{macrocode}
%    Reload check, especially if the package is not used with \LaTeX.
%    \begin{macrocode}
\begingroup\catcode61\catcode48\catcode32=10\relax%
  \catcode13=5 % ^^M
  \endlinechar=13 %
  \catcode35=6 % #
  \catcode39=12 % '
  \catcode44=12 % ,
  \catcode45=12 % -
  \catcode46=12 % .
  \catcode58=12 % :
  \catcode64=11 % @
  \catcode123=1 % {
  \catcode125=2 % }
  \expandafter\let\expandafter\x\csname ver@refcount.sty\endcsname
  \ifx\x\relax % plain-TeX, first loading
  \else
    \def\empty{}%
    \ifx\x\empty % LaTeX, first loading,
      % variable is initialized, but \ProvidesPackage not yet seen
    \else
      \expandafter\ifx\csname PackageInfo\endcsname\relax
        \def\x#1#2{%
          \immediate\write-1{Package #1 Info: #2.}%
        }%
      \else
        \def\x#1#2{\PackageInfo{#1}{#2, stopped}}%
      \fi
      \x{refcount}{The package is already loaded}%
      \aftergroup\endinput
    \fi
  \fi
\endgroup%
%    \end{macrocode}
%    Package identification:
%    \begin{macrocode}
\begingroup\catcode61\catcode48\catcode32=10\relax%
  \catcode13=5 % ^^M
  \endlinechar=13 %
  \catcode35=6 % #
  \catcode39=12 % '
  \catcode40=12 % (
  \catcode41=12 % )
  \catcode44=12 % ,
  \catcode45=12 % -
  \catcode46=12 % .
  \catcode47=12 % /
  \catcode58=12 % :
  \catcode64=11 % @
  \catcode91=12 % [
  \catcode93=12 % ]
  \catcode123=1 % {
  \catcode125=2 % }
  \expandafter\ifx\csname ProvidesPackage\endcsname\relax
    \def\x#1#2#3[#4]{\endgroup
      \immediate\write-1{Package: #3 #4}%
      \xdef#1{#4}%
    }%
  \else
    \def\x#1#2[#3]{\endgroup
      #2[{#3}]%
      \ifx#1\@undefined
        \xdef#1{#3}%
      \fi
      \ifx#1\relax
        \xdef#1{#3}%
      \fi
    }%
  \fi
\expandafter\x\csname ver@refcount.sty\endcsname
\ProvidesPackage{refcount}%
  [2011/10/16 v3.4 Data extraction from label references (HO)]%
%    \end{macrocode}
%
%    \begin{macrocode}
\begingroup\catcode61\catcode48\catcode32=10\relax%
  \catcode13=5 % ^^M
  \endlinechar=13 %
  \catcode123=1 % {
  \catcode125=2 % }
  \catcode64=11 % @
  \def\x{\endgroup
    \expandafter\edef\csname rc@AtEnd\endcsname{%
      \endlinechar=\the\endlinechar\relax
      \catcode13=\the\catcode13\relax
      \catcode32=\the\catcode32\relax
      \catcode35=\the\catcode35\relax
      \catcode61=\the\catcode61\relax
      \catcode64=\the\catcode64\relax
      \catcode123=\the\catcode123\relax
      \catcode125=\the\catcode125\relax
    }%
  }%
\x\catcode61\catcode48\catcode32=10\relax%
\catcode13=5 % ^^M
\endlinechar=13 %
\catcode35=6 % #
\catcode64=11 % @
\catcode123=1 % {
\catcode125=2 % }
\def\TMP@EnsureCode#1#2{%
  \edef\rc@AtEnd{%
    \rc@AtEnd
    \catcode#1=\the\catcode#1\relax
  }%
  \catcode#1=#2\relax
}
\TMP@EnsureCode{33}{12}% !
\TMP@EnsureCode{39}{12}% '
\TMP@EnsureCode{42}{12}% *
\TMP@EnsureCode{45}{12}% -
\TMP@EnsureCode{46}{12}% .
\TMP@EnsureCode{47}{12}% /
\TMP@EnsureCode{91}{12}% [
\TMP@EnsureCode{93}{12}% ]
\TMP@EnsureCode{96}{12}% `
\edef\rc@AtEnd{\rc@AtEnd\noexpand\endinput}
%    \end{macrocode}
%
% \subsection{Loading packages}
%
%    \begin{macrocode}
\begingroup\expandafter\expandafter\expandafter\endgroup
\expandafter\ifx\csname RequirePackage\endcsname\relax
  \input ltxcmds.sty\relax
  \input infwarerr.sty\relax
\else
  \RequirePackage{ltxcmds}[2011/11/09]%
  \RequirePackage{infwarerr}[2010/04/08]%
\fi
%    \end{macrocode}
%
% \subsection{Defining commands}
%
%    \begin{macro}{\rc@IfDefinable}
%    \begin{macrocode}
\ltx@IfUndefined{@ifdefinable}{%
  \def\rc@IfDefinable#1{%
    \ifx#1\ltx@undefined
      \expandafter\ltx@firstofone
    \else
      \ifx#1\relax
        \expandafter\expandafter\expandafter\ltx@firstofone
      \else
        \@PackageError{refcount}{%
          Command \string#1 is already defined.\MessageBreak
          It will not redefined by this package%
        }\@ehc
        \expandafter\expandafter\expandafter\ltx@gobble
      \fi
    \fi
  }%
}{%
  \let\rc@IfDefinable\@ifdefinable
}
%    \end{macrocode}
%    \end{macro}
%
%    \begin{macro}{\rc@RobustDefOne}
%    \begin{macro}{\rc@RobustDefZero}
%    \begin{macrocode}
\ltx@IfUndefined{protected}{%
  \ltx@IfUndefined{DeclareRobustCommand}{%
    \def\rc@RobustDefOne#1#2#3#4{%
      \rc@IfDefinable#3{%
        #1\def#3##1{#4}%
      }%
    }%
    \def\rc@RobustDefZero#1#2{%
      \rc@IfDefinable#1{%
        \def#1{#2}%
      }%
    }%
  }{%
    \def\rc@RobustDefOne#1#2#3#4{%
      \rc@IfDefinable#3{%
        \DeclareRobustCommand#2#3[1]{#4}%
      }%
    }%
    \def\rc@RobustDefZero#1#2{%
      \rc@IfDefinable#1{%
        \DeclareRobustCommand#1{#2}%
      }%
    }%
  }%
}{%
  \def\rc@RobustDefOne#1#2#3#4{%
    \rc@IfDefinable#3{%
      \protected#1\def#3##1{#4}%
    }%
  }%
  \def\rc@RobustDefZero#1#2{%
    \rc@IfDefinable#1{%
      \protected\def#1{#2}%
    }%
  }%
}
%    \end{macrocode}
%    \end{macro}
%    \end{macro}
%
%    \begin{macro}{\rc@newcommand}
%    \begin{macrocode}
\ltx@IfUndefined{newcommand}{%
  \def\rc@newcommand*#1[#2]#3{% hash-ok
    \rc@IfDefinable#1{%
      \ifcase#2 %
        \def#1{#3}%
      \or
        \def#1##1{#3}%
      \or
        \def#1##1##2{#3}%
      \else
        \rc@InternalError
      \fi
    }%
  }%
}{%
  \let\rc@newcommand\newcommand
}
%    \end{macrocode}
%    \end{macro}
%
% \subsection{\cs{setrefcountdefault}}
%
%    \begin{macro}{\setrefcountdefault}
%    \begin{macrocode}
\rc@RobustDefOne\long{}\setrefcountdefault{%
  \def\rc@default{#1}%
}
%    \end{macrocode}
%    \end{macro}
%    \begin{macrocode}
\setrefcountdefault{0}
%    \end{macrocode}
%
% \subsection{\cs{refused}}
%
%    \begin{macro}{\refused}
%    \begin{macrocode}
\ltx@IfUndefined{G@refundefinedtrue}{%
  \rc@RobustDefOne{}{*}\refused{%
    \begingroup
      \csname @safe@activestrue\endcsname
      \ltx@IfUndefined{r@#1}{%
        \protect\G@refundefinedtrue
        \rc@WarningUndefined{#1}%
      }{}%
    \endgroup
  }%
}{%
  \rc@RobustDefOne{}{*}\refused{%
    \begingroup
      \csname @safe@activestrue\endcsname
      \ltx@IfUndefined{r@#1}{%
        \csname protect\expandafter\endcsname
        \csname G@refundefinedtrue\endcsname
        \rc@WarningUndefined{#1}%
      }{}%
    \endgroup
  }%
}
%    \end{macrocode}
%    \end{macro}
%    \begin{macro}{\rc@WarningUndefined}
%    \begin{macrocode}
\ltx@IfUndefined{@latex@warning}{%
  \def\rc@WarningUndefined#1{%
    \ltx@ifundefined{thepage}{%
      \def\thepage{\number\count0 }%
    }{}%
    \@PackageWarning{refcount}{%
      Reference `#1' on page \thepage\space undefined%
    }%
  }%
}{%
  \def\rc@WarningUndefined#1{%
    \@latex@warning{%
      Reference `#1' on page \thepage\space undefined%
    }%
  }%
}
%    \end{macrocode}
%    \end{macro}
%
% \subsection{Setting counters by reference data}
%
% \subsubsection{Generic setting}
%
%    \begin{macro}{\rc@set}
% Generic command for
% |\|$\{$|set|$,$|addto|$\}$|counter|$\{$|page|$,\}$|ref|:
%\begin{quote}
%\begin{tabular}[t]{@{}l@{: }l@{}}
% |#1|& \cs{setcounter}, \cs{addtocounter}\\
% |#2|& \cs{ltx@car} (for \cs{ref}), \cs{ltx@cartwo} (for \cs{pageref})\\
% |#3|& \hologo{LaTeX} counter\\
% |#4|& reference\\
%\end{tabular}
%\end{quote}
%    \begin{macrocode}
\def\rc@set#1#2#3#4{%
  \begingroup
    \csname @safe@activestrue\endcsname
    \refused{#4}%
    \expandafter\rc@@set\csname r@#4\endcsname{#1}{#2}{#3}%
  \endgroup
}
%    \end{macrocode}
%    \end{macro}
%    \begin{macro}{\rc@@set}
%\begin{quote}
%\begin{tabular}[t]{@{}l@{: }l@{}}
% |#1|& \cs{r@<...>}\\
% |#2|& \cs{setcounter}, \cs{addtocounter}\\
% |#3|& \cs{ltx@car} (for \cs{ref}), \cs{ltx@carsecond} (for \cs{pageref})\\
% |#4|& \hologo{LaTeX} counter\\
%\end{tabular}
%\end{quote}
%    \begin{macrocode}
\def\rc@@set#1#2#3#4{%
  \ifx#1\relax
    #2{#4}{\rc@default}%
  \else
    #2{#4}{%
      \expandafter#3#1\rc@default\rc@default\@nil
    }%
  \fi
}
%    \end{macrocode}
%    \end{macro}
%
% \subsubsection{User commands}
%
%    \begin{macro}{\setcounterref}
%    \begin{macrocode}
\rc@RobustDefZero\setcounterref{%
  \rc@set\setcounter\ltx@car
}
%    \end{macrocode}
%    \end{macro}
%    \begin{macro}{\addtocounterref}
%    \begin{macrocode}
\rc@RobustDefZero\addtocounterref{%
  \rc@set\addtocounter\ltx@car
}
%    \end{macrocode}
%    \end{macro}
%    \begin{macro}{\setcounterpageref}
%    \begin{macrocode}
\rc@RobustDefZero\setcounterpageref{%
  \rc@set\setcounter\ltx@carsecond
}
%    \end{macrocode}
%    \end{macro}
%    \begin{macro}{\addtocounterpageref}
%    \begin{macrocode}
\rc@RobustDefZero\addtocounterpageref{%
  \rc@set\addtocounter\ltx@carsecond
}
%    \end{macrocode}
%    \end{macro}
%
% \subsection{Extracting references}
%
%    \begin{macro}{\getrefnumber}
%    \begin{macrocode}
\rc@newcommand*{\getrefnumber}[1]{%
  \romannumeral
  \ltx@ifundefined{r@#1}{%
    \expandafter\ltx@zero
    \rc@default
  }{%
    \expandafter\expandafter\expandafter\rc@extract@
    \expandafter\expandafter\expandafter!%
    \csname r@#1\expandafter\endcsname
    \expandafter{\rc@default}\@nil
  }%
}
%    \end{macrocode}
%    \end{macro}
%    \begin{macro}{\getpagerefnumber}
%    \begin{macrocode}
\rc@newcommand*{\getpagerefnumber}[1]{%
  \romannumeral
  \ltx@ifundefined{r@#1}{%
    \expandafter\ltx@zero
    \rc@default
  }{%
    \expandafter\expandafter\expandafter\rc@extract@page
    \expandafter\expandafter\expandafter!%
    \csname r@#1\expandafter\expandafter\expandafter\endcsname
    \expandafter\expandafter\expandafter{%
      \expandafter\rc@default
    \expandafter}\expandafter{\rc@default}\@nil
  }%
}
%    \end{macrocode}
%    \end{macro}
%    \begin{macro}{\getrefbykeydefault}
%    \begin{macrocode}
\rc@newcommand*{\getrefbykeydefault}[2]{%
  \romannumeral
  \expandafter\rc@getrefbykeydefault
    \csname r@#1\expandafter\endcsname
    \csname rc@extract@#2\endcsname
}
%    \end{macrocode}
%    \end{macro}
%    \begin{macro}{\rc@getrefbykeydefault}
%\begin{quote}
%\begin{tabular}[t]{@{}l@{: }l@{}}
% |#1|& \cs{r@<...>}\\
% |#2|& \cs{rc@extract@<...>}\\
% |#3|& default\\
%\end{tabular}
%\end{quote}
%    \begin{macrocode}
\long\def\rc@getrefbykeydefault#1#2#3{%
  \ifx#1\relax
    % reference is undefined
    \ltx@ReturnAfterElseFi{%
      \ltx@zero
      #3%
    }%
  \else
    \ltx@ReturnAfterFi{%
      \ifx#2\relax
        % extract method is missing
        \ltx@ReturnAfterElseFi{%
          \ltx@zero
          #3%
        }%
      \else
        \ltx@ReturnAfterFi{%
          \expandafter
          \rc@generic#1{#3}{#3}{#3}{#3}{#3}\@nil#2{#3}%
        }%
      \fi
    }%
  \fi
}
%    \end{macrocode}
%    \end{macro}
%    \begin{macro}{\rc@generic}
%\begin{quote}
%\begin{tabular}[t]{@{}l@{: }l@{}}
% |#1|& first item in \cs{r@<...>}\\
% |#2|& remaining items in \cs{r@<...>}\\
% |#3|& \cs{rc@extract@<...>}\\
% |#4|& default\\
%\end{tabular}
%\end{quote}
%    \begin{macrocode}
\long\def\rc@generic#1#2\@nil#3#4{%
  #3{#1\TR@TitleReference\@empty{#4}\@nil}{#1}#2\@nil
}
%    \end{macrocode}
%    \end{macro}
%    \begin{macro}{\rc@extract@}
%    \begin{macrocode}
\long\def\rc@extract@#1#2#3\@nil{%
  \ltx@zero
  #2%
}
%    \end{macrocode}
%    \end{macro}
%    \begin{macro}{\rc@extract@page}
%    \begin{macrocode}
\long\def\rc@extract@page#1#2#3#4\@nil{%
  \ltx@zero
  #3%
}
%    \end{macrocode}
%    \end{macro}
%    \begin{macro}{\rc@extract@name}
%    \begin{macrocode}
\long\def\rc@extract@name#1#2#3#4#5\@nil{%
  \ltx@zero
  #4%
}
%    \end{macrocode}
%    \end{macro}
%    \begin{macro}{\rc@extract@anchor}
%    \begin{macrocode}
\long\def\rc@extract@anchor#1#2#3#4#5#6\@nil{%
  \ltx@zero
  #5%
}
%    \end{macrocode}
%    \end{macro}
%    \begin{macro}{\rc@extract@url}
%    \begin{macrocode}
\long\def\rc@extract@url#1#2#3#4#5#6#7\@nil{%
  \ltx@zero
  #6%
}
%    \end{macrocode}
%    \end{macro}
%    \begin{macro}{\rc@extract@title}
%    \begin{macrocode}
\long\def\rc@extract@title#1#2\@nil{%
  \rc@@extract@title#1%
}
%    \end{macrocode}
%    \end{macro}
%    \begin{macro}{\rc@@extract@title}
%    \begin{macrocode}
\long\def\rc@@extract@title#1\TR@TitleReference#2#3#4\@nil{%
  \ltx@zero
  #3%
}
%    \end{macrocode}
%    \end{macro}
%
% \subsection{Macros for checking undefined references}
%
%    \begin{macro}{\IfRefUndefinedExpandable}
%    \begin{macrocode}
\rc@newcommand*{\IfRefUndefinedExpandable}[1]{%
  \ltx@ifundefined{r@#1}\ltx@firstoftwo\ltx@secondoftwo
}
%    \end{macrocode}
%    \end{macro}
%    \begin{macro}{\IfRefUndefinedBabel}
%    \begin{macrocode}
\rc@RobustDefOne{}*\IfRefUndefinedBabel{%
  \begingroup
    \csname safe@actives@true\endcsname
  \expandafter\expandafter\expandafter\endgroup
  \expandafter\ifx\csname r@#1\endcsname\relax
    \expandafter\ltx@firstoftwo
  \else
    \expandafter\ltx@secondoftwo
  \fi
}
%    \end{macrocode}
%    \end{macro}
%    \begin{macrocode}
\rc@AtEnd%
%</package>
%    \end{macrocode}
%
% \section{Test}
%
% \subsection{Catcode checks for loading}
%
%    \begin{macrocode}
%<*test1>
%    \end{macrocode}
%    \begin{macrocode}
\catcode`\{=1 %
\catcode`\}=2 %
\catcode`\#=6 %
\catcode`\@=11 %
\expandafter\ifx\csname count@\endcsname\relax
  \countdef\count@=255 %
\fi
\expandafter\ifx\csname @gobble\endcsname\relax
  \long\def\@gobble#1{}%
\fi
\expandafter\ifx\csname @firstofone\endcsname\relax
  \long\def\@firstofone#1{#1}%
\fi
\expandafter\ifx\csname loop\endcsname\relax
  \expandafter\@firstofone
\else
  \expandafter\@gobble
\fi
{%
  \def\loop#1\repeat{%
    \def\body{#1}%
    \iterate
  }%
  \def\iterate{%
    \body
      \let\next\iterate
    \else
      \let\next\relax
    \fi
    \next
  }%
  \let\repeat=\fi
}%
\def\RestoreCatcodes{}
\count@=0 %
\loop
  \edef\RestoreCatcodes{%
    \RestoreCatcodes
    \catcode\the\count@=\the\catcode\count@\relax
  }%
\ifnum\count@<255 %
  \advance\count@ 1 %
\repeat

\def\RangeCatcodeInvalid#1#2{%
  \count@=#1\relax
  \loop
    \catcode\count@=15 %
  \ifnum\count@<#2\relax
    \advance\count@ 1 %
  \repeat
}
\def\RangeCatcodeCheck#1#2#3{%
  \count@=#1\relax
  \loop
    \ifnum#3=\catcode\count@
    \else
      \errmessage{%
        Character \the\count@\space
        with wrong catcode \the\catcode\count@\space
        instead of \number#3%
      }%
    \fi
  \ifnum\count@<#2\relax
    \advance\count@ 1 %
  \repeat
}
\def\space{ }
\expandafter\ifx\csname LoadCommand\endcsname\relax
  \def\LoadCommand{\input refcount.sty\relax}%
\fi
\def\Test{%
  \RangeCatcodeInvalid{0}{47}%
  \RangeCatcodeInvalid{58}{64}%
  \RangeCatcodeInvalid{91}{96}%
  \RangeCatcodeInvalid{123}{255}%
  \catcode`\@=12 %
  \catcode`\\=0 %
  \catcode`\%=14 %
  \LoadCommand
  \RangeCatcodeCheck{0}{36}{15}%
  \RangeCatcodeCheck{37}{37}{14}%
  \RangeCatcodeCheck{38}{47}{15}%
  \RangeCatcodeCheck{48}{57}{12}%
  \RangeCatcodeCheck{58}{63}{15}%
  \RangeCatcodeCheck{64}{64}{12}%
  \RangeCatcodeCheck{65}{90}{11}%
  \RangeCatcodeCheck{91}{91}{15}%
  \RangeCatcodeCheck{92}{92}{0}%
  \RangeCatcodeCheck{93}{96}{15}%
  \RangeCatcodeCheck{97}{122}{11}%
  \RangeCatcodeCheck{123}{255}{15}%
  \RestoreCatcodes
}
\Test
\csname @@end\endcsname
\end
%    \end{macrocode}
%    \begin{macrocode}
%</test1>
%    \end{macrocode}
%
% \subsection{Macro tests}
%
%    \begin{macrocode}
%<*test2>
\errorcontextlines=10000 %
\showboxbreadth=10000 %
\showboxdepth=10000 %
\begingroup\expandafter\expandafter\expandafter\endgroup
\expandafter\ifx\csname RequirePackage\endcsname\relax
  \input refcount.sty\relax
\else
  \RequirePackage{refcount}[2011/10/16]%
\fi
\catcode`\@=11 %
\begingroup\expandafter\expandafter\expandafter\endgroup
\expandafter\ifx\csname @onelevel@sanitize\endcsname\relax
  \begingroup\expandafter\expandafter\expandafter\endgroup
  \expandafter\ifx\csname detokenize\endcsname\relax
    \def\strip@prefix#1->{}%
    \def\@onelevel@sanitize#1{%
      \edef#1{%
        \expandafter\strip@prefix\meaning#1%
      }%
    }%
  \else
    \def\@onelevel@sanitize#1{%
      \edef#1{%
        \detokenize\expandafter{#1}%
      }%
    }%
  \fi
\fi
\def\msg#{\immediate\write16}
\def\empty{}
\def\space{ }
%    \end{macrocode}
%    \begin{macrocode}
\def\r@foo{{\empty 1}{\empty 2}}
\long\def\test#1#2{%
  \begingroup
    \setbox0=\hbox{%
      \def\TestTask{#1}%
      \@onelevel@sanitize\TestTask
      \msg{* \TestTask}%
      \expandafter\expandafter\expandafter\def
      \expandafter\expandafter\expandafter\TestResult
      \expandafter\expandafter\expandafter{%
        #1%
      }%
      \def\TestExpected{#2}%
      \ifx\TestResult\TestExpected
        \msg{ \space ok.}%
      \else
        \@onelevel@sanitize\TestResult
        \@onelevel@sanitize\TestExpected
        \msg{ \space Result: \space\space[\TestResult]}%
        \msg{ \space Expected: [\TestExpected]}%
        \errmessage{Test failed!}%
      \fi
    }%
    \ifdim\wd0=0pt %
    \else
      \showbox0 %
    \fi
  \endgroup
}
\test{\getrefnumber{foo}}{\empty 1}
\test{\getpagerefnumber{foo}}{\empty 2}
\test{\getrefbykeydefault{foo}{}{\empty default}}{\empty 1}
\test{\getrefbykeydefault{foo}{page}{\empty default}}{\empty 2}
\test{\getrefbykeydefault{foo}{name}{\empty default}}{\empty default}
\test{\getrefbykeydefault{foo}{anchor}{\empty default}}{\empty default}
\test{\getrefbykeydefault{foo}{url}{\empty default}}{\empty default}
\test{\getrefbykeydefault{foo}{title}{\empty default}}{\empty default}
\msg{}
\def\r@foo{{}{}{}{}{}{}{}{}{}{}}
\def\Test#1#2\\{%
  \test{#1{foo}#2}{}%
}
\def\TestGroup{%
  \Test\getrefnumber\\%
  \Test\getpagerefnumber\\%
  \Test\getrefbykeydefault{}{}\\%
  \Test\getrefbykeydefault{page}{}\\%
  \Test\getrefbykeydefault{anchor}{}\\%
  \Test\getrefbykeydefault{name}{}\\%
  \Test\getrefbykeydefault{url}{}\\%
}
\TestGroup
\Test\getrefbykeydefault{title}{}\\%
\msg{}
\def\r@foo{\par\par\par\par\par\par\par\par}
\long\def\Test#1#2\\{%
  \test{#1{foo}#2}{\par}%
}
\TestGroup
\test{\getrefbykeydefault{title}{}{}}{}
\msg{}
\def\r@foo{{ }{ }{ }{ }{ }}
\def\Test#1#2\\{%
  \test{#1{foo}#2}{ }%
}
\TestGroup
\msg{}
\long\def\TestDefault#1{%
  \begingroup
    \setrefcountdefault{#1}%
    \test{\getrefnumber{foo}}{#1}%
    \test{\getpagerefnumber{foo}}{#1}%
  \endgroup
}
\def\TestDefaultX{%
  \TestDefault{}%
  \TestDefault{\par}%
  \TestDefault{ }%
  \TestDefault{\space}%
}
\let\r@foo\@undefined
\TestDefaultX
\let\r@foo\relax
\TestDefaultX
\def\r@foo{}
\TestDefaultX
%    \end{macrocode}
%    \begin{macrocode}
\msg{}
\long\def\Test#1#2#3#4{%
  \begingroup
    \def\TestTask{#1}%
    \@onelevel@sanitize\TestTask
    \msg{* [\TestTask]}%
    \edef\TestResultA{\IfRefUndefinedExpandable{#1}{#2}{#3}}%
    \IfRefUndefinedBabel{#1}{%
      \def\TestResultB{#2}%
    }{%
      \def\TestResultB{#3}%
    }%
    \def\TestExpected{#4}%
    \ifx\TestResultA\TestExpected
      \msg{ \space ok.}%
    \else
      \begingroup
        \@onelevel@sanitize\TestResultA
        \@onelevel@sanitize\TestExpected
        \msg{ \space Result: \space\space[\TestResultA]}%
        \msg{ \space Expected: [\TestExpected]}%
        \errmessage{Test failed!}%
      \endgroup
    \fi
    \ifx\TestResultB\TestExpected
      \msg{ \space ok.}%
    \else
      \begingroup
        \@onelevel@sanitize\TestResultB
        \@onelevel@sanitize\TestExpected
        \msg{ \space Result: \space\space[\TestResultB]}%
        \msg{ \space Expected: [\TestExpected]}%
        \errmessage{Test failed!}%
      \endgroup
    \fi
  \endgroup
}
\begingroup
  \def\r@foo{{}{}}%
  \let\r@bar\@undefined
  \let\r@xyz\relax
  \Test{foo}{true}{false}{false}%
  \Test{bar}{true}{false}{true}%
  \Test{xyz}{true}{false}{true}%
\endgroup
%    \end{macrocode}
%    \begin{macrocode}
\csname @@end\endcsname\end
%</test2>
%    \end{macrocode}
% \subsection{Test with package \xpackage{titleref}}
%
%    \begin{macrocode}
%<*test3>
\NeedsTeXFormat{LaTeX2e}
\documentclass{article}
\usepackage{refcount}[2011/10/16]
%<test4>\usepackage{nameref}
\usepackage{titleref}
\begin{document}
\section{Hello World}
\label{sec:hello}
\section{\hbox{xy}}
\label{sec:foo}
%
\makeatletter
\@ifundefined{r@sec:hello}{%
  \typeout{==> Compile twice!}%
}{%
  \def\test#1#2{%
    \begingroup
      \def\TestTask{#1}%
      \@onelevel@sanitize\TestTask
      \typeout{* \TestTask}%
      \expandafter\expandafter\expandafter\def
      \expandafter\expandafter\expandafter\TestResult
      \expandafter\expandafter\expandafter{%
        #1%
      }%
      \def\TestExpected{#2}%
      \ifx\TestResult\TestExpected
        \typeout{ \space ok.}%
      \else
        \@onelevel@sanitize\TestResult
        \@onelevel@sanitize\TestExpected
        \typeout{ \space Result: \space\space[\TestResult]}%
        \typeout{ \space Expected: [\TestExpected]}%
        \errmessage{Test failed!}%
      \fi
    \endgroup
  }%
  \test{\getrefbykeydefault{sec:hello}{title}{}}{Hello World}%
  \test{\getrefbykeydefault{sec:foo}{title}{}}{\hbox{xy}}%
  \begingroup
    \def\hbox#1{[#1]}% hash-ok
    \test{\getrefbykeydefault{sec:foo}{title}{}}{\hbox{xy}}%
  \endgroup
}
\makeatother
%    \end{macrocode}
%    \begin{macrocode}
\end{document}
%</test3>
%    \end{macrocode}
%    \begin{macrocode}
%<*test5>
\NeedsTeXFormat{LaTeX2e}
\documentclass{book}
\usepackage{refcount}[2011/10/16]
\usepackage{zref-runs}
\newcounter{test}
\begin{document}
\ifnum\zruns>1 %
  \makeatletter
  \def\Test#1#2#3{%
    \begingroup
      \setcounter{test}{10}%
      \sbox0{%
        #1{test}{#2}%
        \ifnum#3=\value{test}%
        \else
          \PackageError{test}{\string#1{#2} <> #3 (\the\value{test})}%
        \fi
      }%
      \ifdim\wd0=0pt %
      \else
        \PackageError{test}{Non-empty box}\@ehc
      \fi
    \endgroup
  }%
  \makeatother
  \Test\setcounterpageref{ch:two}{1}%
  \Test\setcounterpageref{ch:three}{3}%
  \Test\setcounterpageref{ch:four}{5}%
  \Test\setcounterpageref{ch:five}{7}%
  \Test\setcounterpageref{ch:six}{9}%
  \Test\setcounterpageref{ch:seven}{13}%
  \Test\addtocounterpageref{ch:two}{11}%
  \Test\addtocounterpageref{ch:three}{13}%
  \Test\addtocounterpageref{ch:four}{15}%
  \Test\addtocounterpageref{ch:five}{17}%
  \Test\addtocounterpageref{ch:six}{19}%
  \Test\addtocounterpageref{ch:seven}{23}%
  \Test\setcounterref{ch:two}{1}%
  \Test\setcounterref{ch:three}{2}%
  \Test\setcounterref{ch:four}{11}%
  \Test\addtocounterref{ch:two}{11}%
  \Test\addtocounterref{ch:three}{12}%
  \Test\addtocounterref{ch:four}{21}%
\fi
\frontmatter
\chapter{Chapter one}\label{ch:one}
\cleardoublepage
\mainmatter
\chapter{Chapter two}\label{ch:two}
\cleardoublepage
\chapter{Chapter three}\label{ch:three}
\cleardoublepage
\setcounter{chapter}{10}
\chapter{Chapter four}\label{ch:four}
\cleardoublepage
\appendix
\chapter{Chapter five}\label{ch:five}
\cleardoublepage
\chapter{Chapter six}\label{ch:six}
\cleardoublepage
\null
\cleardoublepage
\chapter{Chapter seven}\label{ch:seven}
\end{document}
%</test5>
%    \end{macrocode}
%
% \section{Installation}
%
% \subsection{Download}
%
% \paragraph{Package.} This package is available on
% CTAN\footnote{\url{ftp://ftp.ctan.org/tex-archive/}}:
% \begin{description}
% \item[\CTAN{macros/latex/contrib/oberdiek/refcount.dtx}] The source file.
% \item[\CTAN{macros/latex/contrib/oberdiek/refcount.pdf}] Documentation.
% \end{description}
%
%
% \paragraph{Bundle.} All the packages of the bundle `oberdiek'
% are also available in a TDS compliant ZIP archive. There
% the packages are already unpacked and the documentation files
% are generated. The files and directories obey the TDS standard.
% \begin{description}
% \item[\CTAN{install/macros/latex/contrib/oberdiek.tds.zip}]
% \end{description}
% \emph{TDS} refers to the standard ``A Directory Structure
% for \TeX\ Files'' (\CTAN{tds/tds.pdf}). Directories
% with \xfile{texmf} in their name are usually organized this way.
%
% \subsection{Bundle installation}
%
% \paragraph{Unpacking.} Unpack the \xfile{oberdiek.tds.zip} in the
% TDS tree (also known as \xfile{texmf} tree) of your choice.
% Example (linux):
% \begin{quote}
%   |unzip oberdiek.tds.zip -d ~/texmf|
% \end{quote}
%
% \paragraph{Script installation.}
% Check the directory \xfile{TDS:scripts/oberdiek/} for
% scripts that need further installation steps.
% Package \xpackage{attachfile2} comes with the Perl script
% \xfile{pdfatfi.pl} that should be installed in such a way
% that it can be called as \texttt{pdfatfi}.
% Example (linux):
% \begin{quote}
%   |chmod +x scripts/oberdiek/pdfatfi.pl|\\
%   |cp scripts/oberdiek/pdfatfi.pl /usr/local/bin/|
% \end{quote}
%
% \subsection{Package installation}
%
% \paragraph{Unpacking.} The \xfile{.dtx} file is a self-extracting
% \docstrip\ archive. The files are extracted by running the
% \xfile{.dtx} through \plainTeX:
% \begin{quote}
%   \verb|tex refcount.dtx|
% \end{quote}
%
% \paragraph{TDS.} Now the different files must be moved into
% the different directories in your installation TDS tree
% (also known as \xfile{texmf} tree):
% \begin{quote}
% \def\t{^^A
% \begin{tabular}{@{}>{\ttfamily}l@{ $\rightarrow$ }>{\ttfamily}l@{}}
%   refcount.sty & tex/latex/oberdiek/refcount.sty\\
%   refcount.pdf & doc/latex/oberdiek/refcount.pdf\\
%   test/refcount-test1.tex & doc/latex/oberdiek/test/refcount-test1.tex\\
%   test/refcount-test2.tex & doc/latex/oberdiek/test/refcount-test2.tex\\
%   test/refcount-test3.tex & doc/latex/oberdiek/test/refcount-test3.tex\\
%   test/refcount-test4.tex & doc/latex/oberdiek/test/refcount-test4.tex\\
%   test/refcount-test5.tex & doc/latex/oberdiek/test/refcount-test5.tex\\
%   refcount.dtx & source/latex/oberdiek/refcount.dtx\\
% \end{tabular}^^A
% }^^A
% \sbox0{\t}^^A
% \ifdim\wd0>\linewidth
%   \begingroup
%     \advance\linewidth by\leftmargin
%     \advance\linewidth by\rightmargin
%   \edef\x{\endgroup
%     \def\noexpand\lw{\the\linewidth}^^A
%   }\x
%   \def\lwbox{^^A
%     \leavevmode
%     \hbox to \linewidth{^^A
%       \kern-\leftmargin\relax
%       \hss
%       \usebox0
%       \hss
%       \kern-\rightmargin\relax
%     }^^A
%   }^^A
%   \ifdim\wd0>\lw
%     \sbox0{\small\t}^^A
%     \ifdim\wd0>\linewidth
%       \ifdim\wd0>\lw
%         \sbox0{\footnotesize\t}^^A
%         \ifdim\wd0>\linewidth
%           \ifdim\wd0>\lw
%             \sbox0{\scriptsize\t}^^A
%             \ifdim\wd0>\linewidth
%               \ifdim\wd0>\lw
%                 \sbox0{\tiny\t}^^A
%                 \ifdim\wd0>\linewidth
%                   \lwbox
%                 \else
%                   \usebox0
%                 \fi
%               \else
%                 \lwbox
%               \fi
%             \else
%               \usebox0
%             \fi
%           \else
%             \lwbox
%           \fi
%         \else
%           \usebox0
%         \fi
%       \else
%         \lwbox
%       \fi
%     \else
%       \usebox0
%     \fi
%   \else
%     \lwbox
%   \fi
% \else
%   \usebox0
% \fi
% \end{quote}
% If you have a \xfile{docstrip.cfg} that configures and enables \docstrip's
% TDS installing feature, then some files can already be in the right
% place, see the documentation of \docstrip.
%
% \subsection{Refresh file name databases}
%
% If your \TeX~distribution
% (\teTeX, \mikTeX, \dots) relies on file name databases, you must refresh
% these. For example, \teTeX\ users run \verb|texhash| or
% \verb|mktexlsr|.
%
% \subsection{Some details for the interested}
%
% \paragraph{Attached source.}
%
% The PDF documentation on CTAN also includes the
% \xfile{.dtx} source file. It can be extracted by
% AcrobatReader 6 or higher. Another option is \textsf{pdftk},
% e.g. unpack the file into the current directory:
% \begin{quote}
%   \verb|pdftk refcount.pdf unpack_files output .|
% \end{quote}
%
% \paragraph{Unpacking with \LaTeX.}
% The \xfile{.dtx} chooses its action depending on the format:
% \begin{description}
% \item[\plainTeX:] Run \docstrip\ and extract the files.
% \item[\LaTeX:] Generate the documentation.
% \end{description}
% If you insist on using \LaTeX\ for \docstrip\ (really,
% \docstrip\ does not need \LaTeX), then inform the autodetect routine
% about your intention:
% \begin{quote}
%   \verb|latex \let\install=y\input{refcount.dtx}|
% \end{quote}
% Do not forget to quote the argument according to the demands
% of your shell.
%
% \paragraph{Generating the documentation.}
% You can use both the \xfile{.dtx} or the \xfile{.drv} to generate
% the documentation. The process can be configured by the
% configuration file \xfile{ltxdoc.cfg}. For instance, put this
% line into this file, if you want to have A4 as paper format:
% \begin{quote}
%   \verb|\PassOptionsToClass{a4paper}{article}|
% \end{quote}
% An example follows how to generate the
% documentation with pdf\LaTeX:
% \begin{quote}
%\begin{verbatim}
%pdflatex refcount.dtx
%makeindex -s gind.ist refcount.idx
%pdflatex refcount.dtx
%makeindex -s gind.ist refcount.idx
%pdflatex refcount.dtx
%\end{verbatim}
% \end{quote}
%
% \section{Catalogue}
%
% The following XML file can be used as source for the
% \href{http://mirror.ctan.org/help/Catalogue/catalogue.html}{\TeX\ Catalogue}.
% The elements \texttt{caption} and \texttt{description} are imported
% from the original XML file from the Catalogue.
% The name of the XML file in the Catalogue is \xfile{refcount.xml}.
%    \begin{macrocode}
%<*catalogue>
<?xml version='1.0' encoding='us-ascii'?>
<!DOCTYPE entry SYSTEM 'catalogue.dtd'>
<entry datestamp='$Date$' modifier='$Author$' id='refcount'>
  <name>refcount</name>
  <caption>Counter operations with label references.</caption>
  <authorref id='auth:oberdiek'/>
  <copyright owner='Heiko Oberdiek' year='1998,2000,2006,2008,2010,2011'/>
  <license type='lppl1.3'/>
  <version number='3.4'/>
  <description>
    Provides commands <tt>\setcounterref</tt> and
    <tt>\addtocounterref</tt> which use the section (or whatever)
    number from the reference as the value to put into the counter, as
    in:

    <pre>
    ...\label{sec:foo}
    ...
    \setcounterref{foonum}{sec:foo}
    </pre>
    Commands <tt>\setcounterpageref</tt> and
    <tt>\addtocounterpageref</tt> do the corresponding thing with the
    page reference of the label.
    <p/>
    No <tt>.ins</tt> file is distributed; process the
    <tt>.dtx</tt> with plain TeX to create one.
    <p/>
    The package is part of the <xref refid='oberdiek'>oberdiek</xref>
    bundle.
  </description>
  <documentation details='Package documentation'
      href='ctan:/macros/latex/contrib/oberdiek/refcount.pdf'/>
  <ctan file='true' path='/macros/latex/contrib/oberdiek/refcount.dtx'/>
  <miktex location='oberdiek'/>
  <texlive location='oberdiek'/>
  <install path='/macros/latex/contrib/oberdiek/oberdiek.tds.zip'/>
</entry>
%</catalogue>
%    \end{macrocode}
%
% \begin{History}
%   \begin{Version}{1998/04/08 v1.0}
%   \item
%     First public release, written as answer in the
%     newsgroup \xnewsgroup{comp.text.tex}:
%     \URL{``\link{Re: Adding a \cs{ref} to a counter?}''}^^A
%     {http://groups.google.com/group/comp.text.tex/msg/c3f2a135ef5ee528}
%   \end{Version}
%   \begin{Version}{2000/09/07 v2.0}
%   \item
%     Documentation added.
%   \item
%     LPPL 1.2
%   \item
%     Package rewritten, new commands added.
%   \end{Version}
%   \begin{Version}{2006/02/20 v3.0}
%   \item
%     Support for \xpackage{hyperref} and \xpackage{nameref} improved.
%   \item
%     Support for \xpackage{titleref} and \xpackage{babel}'s shorthands added.
%   \item
%     New: \cs{refused}, \cs{getrefbykeydefault}
%   \end{Version}
%   \begin{Version}{2008/08/11 v3.1}
%   \item
%     Code is not changed.
%   \item
%     URLs updated.
%   \end{Version}
%   \begin{Version}{2010/12/01 v3.2}
%   \item
%     \cs{IfRefUndefinedExpandable} and \cs{IfRefUndefinedBabel} added.
%   \item
%     \cs{getrefnumber}, \cs{getpagerefnumber}, \cs{getrefbykeydefault}
%     are expandable in exact two expansion steps.
%   \item
%     Non-expandable macros are made robust.
%   \item
%     Test files added.
%   \end{Version}
%   \begin{Version}{2011/06/22 v3.3}
%   \item
%     Bug fix: \cs{rc@refused} is undefined for \cs{setcounterpageref}
%     and similar macros. (Bug found by Marc van Dongen.)
%   \end{Version}
%   \begin{Version}{2011/10/16 v3.4}
%   \item
%     Bug fix: \cs{setcounterpageref} and \cs{addtocounterpageref} fixed.
%     (Bug found by Staz.)
%   \item
%     Macros \cs{(set|addto)counter(page|)ref} are made robust.
%   \end{Version}
% \end{History}
%
% \PrintIndex
%
% \Finale
\endinput
|
% \end{quote}
% Do not forget to quote the argument according to the demands
% of your shell.
%
% \paragraph{Generating the documentation.}
% You can use both the \xfile{.dtx} or the \xfile{.drv} to generate
% the documentation. The process can be configured by the
% configuration file \xfile{ltxdoc.cfg}. For instance, put this
% line into this file, if you want to have A4 as paper format:
% \begin{quote}
%   \verb|\PassOptionsToClass{a4paper}{article}|
% \end{quote}
% An example follows how to generate the
% documentation with pdf\LaTeX:
% \begin{quote}
%\begin{verbatim}
%pdflatex refcount.dtx
%makeindex -s gind.ist refcount.idx
%pdflatex refcount.dtx
%makeindex -s gind.ist refcount.idx
%pdflatex refcount.dtx
%\end{verbatim}
% \end{quote}
%
% \section{Catalogue}
%
% The following XML file can be used as source for the
% \href{http://mirror.ctan.org/help/Catalogue/catalogue.html}{\TeX\ Catalogue}.
% The elements \texttt{caption} and \texttt{description} are imported
% from the original XML file from the Catalogue.
% The name of the XML file in the Catalogue is \xfile{refcount.xml}.
%    \begin{macrocode}
%<*catalogue>
<?xml version='1.0' encoding='us-ascii'?>
<!DOCTYPE entry SYSTEM 'catalogue.dtd'>
<entry datestamp='$Date$' modifier='$Author$' id='refcount'>
  <name>refcount</name>
  <caption>Counter operations with label references.</caption>
  <authorref id='auth:oberdiek'/>
  <copyright owner='Heiko Oberdiek' year='1998,2000,2006,2008,2010,2011'/>
  <license type='lppl1.3'/>
  <version number='3.4'/>
  <description>
    Provides commands <tt>\setcounterref</tt> and
    <tt>\addtocounterref</tt> which use the section (or whatever)
    number from the reference as the value to put into the counter, as
    in:

    <pre>
    ...\label{sec:foo}
    ...
    \setcounterref{foonum}{sec:foo}
    </pre>
    Commands <tt>\setcounterpageref</tt> and
    <tt>\addtocounterpageref</tt> do the corresponding thing with the
    page reference of the label.
    <p/>
    No <tt>.ins</tt> file is distributed; process the
    <tt>.dtx</tt> with plain TeX to create one.
    <p/>
    The package is part of the <xref refid='oberdiek'>oberdiek</xref>
    bundle.
  </description>
  <documentation details='Package documentation'
      href='ctan:/macros/latex/contrib/oberdiek/refcount.pdf'/>
  <ctan file='true' path='/macros/latex/contrib/oberdiek/refcount.dtx'/>
  <miktex location='oberdiek'/>
  <texlive location='oberdiek'/>
  <install path='/macros/latex/contrib/oberdiek/oberdiek.tds.zip'/>
</entry>
%</catalogue>
%    \end{macrocode}
%
% \begin{History}
%   \begin{Version}{1998/04/08 v1.0}
%   \item
%     First public release, written as answer in the
%     newsgroup \xnewsgroup{comp.text.tex}:
%     \URL{``\link{Re: Adding a \cs{ref} to a counter?}''}^^A
%     {http://groups.google.com/group/comp.text.tex/msg/c3f2a135ef5ee528}
%   \end{Version}
%   \begin{Version}{2000/09/07 v2.0}
%   \item
%     Documentation added.
%   \item
%     LPPL 1.2
%   \item
%     Package rewritten, new commands added.
%   \end{Version}
%   \begin{Version}{2006/02/20 v3.0}
%   \item
%     Support for \xpackage{hyperref} and \xpackage{nameref} improved.
%   \item
%     Support for \xpackage{titleref} and \xpackage{babel}'s shorthands added.
%   \item
%     New: \cs{refused}, \cs{getrefbykeydefault}
%   \end{Version}
%   \begin{Version}{2008/08/11 v3.1}
%   \item
%     Code is not changed.
%   \item
%     URLs updated.
%   \end{Version}
%   \begin{Version}{2010/12/01 v3.2}
%   \item
%     \cs{IfRefUndefinedExpandable} and \cs{IfRefUndefinedBabel} added.
%   \item
%     \cs{getrefnumber}, \cs{getpagerefnumber}, \cs{getrefbykeydefault}
%     are expandable in exact two expansion steps.
%   \item
%     Non-expandable macros are made robust.
%   \item
%     Test files added.
%   \end{Version}
%   \begin{Version}{2011/06/22 v3.3}
%   \item
%     Bug fix: \cs{rc@refused} is undefined for \cs{setcounterpageref}
%     and similar macros. (Bug found by Marc van Dongen.)
%   \end{Version}
%   \begin{Version}{2011/10/16 v3.4}
%   \item
%     Bug fix: \cs{setcounterpageref} and \cs{addtocounterpageref} fixed.
%     (Bug found by Staz.)
%   \item
%     Macros \cs{(set|addto)counter(page|)ref} are made robust.
%   \end{Version}
% \end{History}
%
% \PrintIndex
%
% \Finale
\endinput

%        (quote the arguments according to the demands of your shell)
%
% Documentation:
%    (a) If refcount.drv is present:
%           latex refcount.drv
%    (b) Without refcount.drv:
%           latex refcount.dtx; ...
%    The class ltxdoc loads the configuration file ltxdoc.cfg
%    if available. Here you can specify further options, e.g.
%    use A4 as paper format:
%       \PassOptionsToClass{a4paper}{article}
%
%    Programm calls to get the documentation (example):
%       pdflatex refcount.dtx
%       makeindex -s gind.ist refcount.idx
%       pdflatex refcount.dtx
%       makeindex -s gind.ist refcount.idx
%       pdflatex refcount.dtx
%
% Installation:
%    TDS:tex/latex/oberdiek/refcount.sty
%    TDS:doc/latex/oberdiek/refcount.pdf
%    TDS:doc/latex/oberdiek/test/refcount-test1.tex
%    TDS:doc/latex/oberdiek/test/refcount-test2.tex
%    TDS:doc/latex/oberdiek/test/refcount-test3.tex
%    TDS:doc/latex/oberdiek/test/refcount-test4.tex
%    TDS:doc/latex/oberdiek/test/refcount-test5.tex
%    TDS:source/latex/oberdiek/refcount.dtx
%
%<*ignore>
\begingroup
  \catcode123=1 %
  \catcode125=2 %
  \def\x{LaTeX2e}%
\expandafter\endgroup
\ifcase 0\ifx\install y1\fi\expandafter
         \ifx\csname processbatchFile\endcsname\relax\else1\fi
         \ifx\fmtname\x\else 1\fi\relax
\else\csname fi\endcsname
%</ignore>
%<*install>
\input docstrip.tex
\Msg{************************************************************************}
\Msg{* Installation}
\Msg{* Package: refcount 2011/10/16 v3.4 Data extraction from label references (HO)}
\Msg{************************************************************************}

\keepsilent
\askforoverwritefalse

\let\MetaPrefix\relax
\preamble

This is a generated file.

Project: refcount
Version: 2011/10/16 v3.4

Copyright (C) 1998, 2000, 2006, 2008, 2010, 2011 by
   Heiko Oberdiek <heiko.oberdiek at googlemail.com>

This work may be distributed and/or modified under the
conditions of the LaTeX Project Public License, either
version 1.3c of this license or (at your option) any later
version. This version of this license is in
   http://www.latex-project.org/lppl/lppl-1-3c.txt
and the latest version of this license is in
   http://www.latex-project.org/lppl.txt
and version 1.3 or later is part of all distributions of
LaTeX version 2005/12/01 or later.

This work has the LPPL maintenance status "maintained".

This Current Maintainer of this work is Heiko Oberdiek.

This work consists of the main source file refcount.dtx
and the derived files
   refcount.sty, refcount.pdf, refcount.ins, refcount.drv,
   refcount-test1.tex, refcount-test2.tex, refcount-test3.tex,
   refcount-test4.tex, refcount-test5.tex.

\endpreamble
\let\MetaPrefix\DoubleperCent

\generate{%
  \file{refcount.ins}{\from{refcount.dtx}{install}}%
  \file{refcount.drv}{\from{refcount.dtx}{driver}}%
  \usedir{tex/latex/oberdiek}%
  \file{refcount.sty}{\from{refcount.dtx}{package}}%
  \usedir{doc/latex/oberdiek/test}%
  \file{refcount-test1.tex}{\from{refcount.dtx}{test1}}%
  \file{refcount-test2.tex}{\from{refcount.dtx}{test2}}%
  \file{refcount-test3.tex}{\from{refcount.dtx}{test3}}%
  \file{refcount-test4.tex}{\from{refcount.dtx}{test3,test4}}%
  \file{refcount-test5.tex}{\from{refcount.dtx}{test5}}%
  \nopreamble
  \nopostamble
  \usedir{source/latex/oberdiek/catalogue}%
  \file{refcount.xml}{\from{refcount.dtx}{catalogue}}%
}

\catcode32=13\relax% active space
\let =\space%
\Msg{************************************************************************}
\Msg{*}
\Msg{* To finish the installation you have to move the following}
\Msg{* file into a directory searched by TeX:}
\Msg{*}
\Msg{*     refcount.sty}
\Msg{*}
\Msg{* To produce the documentation run the file `refcount.drv'}
\Msg{* through LaTeX.}
\Msg{*}
\Msg{* Happy TeXing!}
\Msg{*}
\Msg{************************************************************************}

\endbatchfile
%</install>
%<*ignore>
\fi
%</ignore>
%<*driver>
\NeedsTeXFormat{LaTeX2e}
\ProvidesFile{refcount.drv}%
  [2011/10/16 v3.4 Data extraction from label references (HO)]%
\documentclass{ltxdoc}
\usepackage{holtxdoc}[2011/11/22]
\begin{document}
  \DocInput{refcount.dtx}%
\end{document}
%</driver>
% \fi
%
% \CheckSum{1185}
%
% \CharacterTable
%  {Upper-case    \A\B\C\D\E\F\G\H\I\J\K\L\M\N\O\P\Q\R\S\T\U\V\W\X\Y\Z
%   Lower-case    \a\b\c\d\e\f\g\h\i\j\k\l\m\n\o\p\q\r\s\t\u\v\w\x\y\z
%   Digits        \0\1\2\3\4\5\6\7\8\9
%   Exclamation   \!     Double quote  \"     Hash (number) \#
%   Dollar        \$     Percent       \%     Ampersand     \&
%   Acute accent  \'     Left paren    \(     Right paren   \)
%   Asterisk      \*     Plus          \+     Comma         \,
%   Minus         \-     Point         \.     Solidus       \/
%   Colon         \:     Semicolon     \;     Less than     \<
%   Equals        \=     Greater than  \>     Question mark \?
%   Commercial at \@     Left bracket  \[     Backslash     \\
%   Right bracket \]     Circumflex    \^     Underscore    \_
%   Grave accent  \`     Left brace    \{     Vertical bar  \|
%   Right brace   \}     Tilde         \~}
%
% \GetFileInfo{refcount.drv}
%
% \title{The \xpackage{refcount} package}
% \date{2011/10/16 v3.4}
% \author{Heiko Oberdiek\\\xemail{heiko.oberdiek at googlemail.com}}
%
% \maketitle
%
% \begin{abstract}
% References are not numbers, however they often store numerical
% data such as section or page numbers. \cs{ref} or \cs{pageref}
% cannot be used for counter assignments or calculations because
% they are not expandable, generate warnings, or can even be links.
% The package provides expandable macros to extract the data
% from references. Packages \xpackage{hyperref}, \xpackage{nameref},
% \xpackage{titleref}, and \xpackage{babel} are supported.
% \end{abstract}
%
% \tableofcontents
%
% \section{Usage}
%
% \subsection{Setting counters}
%
% The following commands are similar to \LaTeX's
% \cs{setcounter} and \cs{addtocounter},
% but they extract the number value from a reference:
% \begin{quote}
%   \cs{setcounterref}, \cs{addtocounterref}\\
%   \cs{setcounterpageref}, \cs{addtocounterpageref}
% \end{quote}
% They take two arguments:
% \begin{quote}
%    \cs{...counter...ref} |{|\meta{\LaTeX\ counter}|}|
%    |{|\meta{reference}|}|
% \end{quote}
% An undefined references produces the usual LaTeX warning
% and its value is assumed to be zero.
% Example:
% \begin{quote}
%\begin{verbatim}
%\newcounter{ctrA}
%\newcounter{ctrB}
%\refstepcounter{ctrA}\label{ref:A}
%\setcounterref{ctrB}{ref:A}
%\addtocounterpageref{ctrB}{ref:A}
%\end{verbatim}
% \end{quote}
%
% \subsection{Expandable commands}
%
% These commands that can be used in expandible contexts
% (inside calculations, \cs{edef}, \cs{csname}, \cs{write}, \dots):
% \begin{quote}
%   \cs{getrefnumber}, \cs{getpagerefnumber}
% \end{quote}
% They take one argument, the reference:
% \begin{quote}
%   \cs{get...refnumber} |{|\meta{reference}|}|
% \end{quote}
% The default for undefined references can be changed
% with macro \cs{setrefcountdefault}, for example this
% package calls:
% \begin{quote}
%   \cs{setrefcountdefault}|{0}|
% \end{quote}
%
% Since version 2.0 of this package there is a new
% command:
% \begin{quote}
%   \cs{getrefbykeydefault} |{|\meta{reference}|}|
%   |{|\meta{key}|}| |{|\meta{default}|}|
% \end{quote}
% This generalized version allows the extraction
% of further properties of a reference than the
% two standard ones. Thus the following properties
% are supported, if they are available:
% \begin{quote}
% \begin{tabular}{@{}l|l|l@{}}
%    Key & Description & Package\\
% \hline
%   \meta{empty} & same as \cs{ref} & \LaTeX\\
%   |page| & same as \cs{pageref} & \LaTeX\\
%   |title| & section and caption titles & \xpackage{titleref}\\
%   |name| & section and caption titles & \xpackage{nameref}\\
%   |anchor| & anchor name & \xpackage{hyperref}\\
%   |url| & url/file & \xpackage{hyperref}/\xpackage{xr}
% \end{tabular}
% \end{quote}
%
% Since version 3.2 the expandable macros described before
% in this section
% are expandable in exact two expansion steps.
%
% \subsection{Undefined references}
%
% Because warnings and assignments cannot be used in
% expandible contexts, undefined references do not
% produce a warning, their values are assumed to be zero.
% Example:
% \begin{quote}
%\begin{verbatim}
%\label{ref:here}% somewhere
%\refused{ref:here}% see below
%\ifodd\getpagerefnumber{ref:here}%
%  reference is on an odd page
%\else
%  reference is on an even page
%\fi
%\end{verbatim}
% \end{quote}
%
% In case of undefined references the user usually want's
% to be informed. Also \LaTeX\ prints a warning at
% the end of the \LaTeX\ run. To notify \LaTeX\ and
% get a normal warning, just use
% \begin{quote}
%   \cs{refused} |{|\meta{reference}|}|
% \end{quote}
% outside the expanding context. Example, see above.
%
% \subsubsection{Check for undefined references}
%
% In version 3.2 macros were added, that test, whether references
% are defined.
% \begin{declcs}{IfRefUndefinedExpandable} \M{refname} \M{then} \M{else}\\
%   \cs{IfRefUndefinedBabel} \M{refname} \M{then} \M{else}
% \end{declcs}
% If the reference is not available and therefore undefined, then
% argument \meta{then} is executed, otherwise argument \meta{else}
% is called. Macro \cs{IfRefUndefinedExpandable} is expandable,
% but \meta{refname} must not contain babel shorthand characters.
% Macro \cs{IfRefUndefinedBabel} supports shorthand characters of
% babel, but it is not expandable.
%
% \subsection{Notes}
%
% \begin{itemize}
% \item
%   The method of extracting the number in this
%   package also works in cases, where the
%   reference cannot be used directly, because
%   a package such as \xpackage{hyperref} has added
%   extra stuff (hyper link), so that the reference cannot
%   be used as number any more.
% \item
%   If the reference does not contain a number,
%   assignments to a counter will fail of course.
% \end{itemize}
%
%
% \StopEventually{
% }
%
% \section{Implementation}
%
%    \begin{macrocode}
%<*package>
%    \end{macrocode}
%    Reload check, especially if the package is not used with \LaTeX.
%    \begin{macrocode}
\begingroup\catcode61\catcode48\catcode32=10\relax%
  \catcode13=5 % ^^M
  \endlinechar=13 %
  \catcode35=6 % #
  \catcode39=12 % '
  \catcode44=12 % ,
  \catcode45=12 % -
  \catcode46=12 % .
  \catcode58=12 % :
  \catcode64=11 % @
  \catcode123=1 % {
  \catcode125=2 % }
  \expandafter\let\expandafter\x\csname ver@refcount.sty\endcsname
  \ifx\x\relax % plain-TeX, first loading
  \else
    \def\empty{}%
    \ifx\x\empty % LaTeX, first loading,
      % variable is initialized, but \ProvidesPackage not yet seen
    \else
      \expandafter\ifx\csname PackageInfo\endcsname\relax
        \def\x#1#2{%
          \immediate\write-1{Package #1 Info: #2.}%
        }%
      \else
        \def\x#1#2{\PackageInfo{#1}{#2, stopped}}%
      \fi
      \x{refcount}{The package is already loaded}%
      \aftergroup\endinput
    \fi
  \fi
\endgroup%
%    \end{macrocode}
%    Package identification:
%    \begin{macrocode}
\begingroup\catcode61\catcode48\catcode32=10\relax%
  \catcode13=5 % ^^M
  \endlinechar=13 %
  \catcode35=6 % #
  \catcode39=12 % '
  \catcode40=12 % (
  \catcode41=12 % )
  \catcode44=12 % ,
  \catcode45=12 % -
  \catcode46=12 % .
  \catcode47=12 % /
  \catcode58=12 % :
  \catcode64=11 % @
  \catcode91=12 % [
  \catcode93=12 % ]
  \catcode123=1 % {
  \catcode125=2 % }
  \expandafter\ifx\csname ProvidesPackage\endcsname\relax
    \def\x#1#2#3[#4]{\endgroup
      \immediate\write-1{Package: #3 #4}%
      \xdef#1{#4}%
    }%
  \else
    \def\x#1#2[#3]{\endgroup
      #2[{#3}]%
      \ifx#1\@undefined
        \xdef#1{#3}%
      \fi
      \ifx#1\relax
        \xdef#1{#3}%
      \fi
    }%
  \fi
\expandafter\x\csname ver@refcount.sty\endcsname
\ProvidesPackage{refcount}%
  [2011/10/16 v3.4 Data extraction from label references (HO)]%
%    \end{macrocode}
%
%    \begin{macrocode}
\begingroup\catcode61\catcode48\catcode32=10\relax%
  \catcode13=5 % ^^M
  \endlinechar=13 %
  \catcode123=1 % {
  \catcode125=2 % }
  \catcode64=11 % @
  \def\x{\endgroup
    \expandafter\edef\csname rc@AtEnd\endcsname{%
      \endlinechar=\the\endlinechar\relax
      \catcode13=\the\catcode13\relax
      \catcode32=\the\catcode32\relax
      \catcode35=\the\catcode35\relax
      \catcode61=\the\catcode61\relax
      \catcode64=\the\catcode64\relax
      \catcode123=\the\catcode123\relax
      \catcode125=\the\catcode125\relax
    }%
  }%
\x\catcode61\catcode48\catcode32=10\relax%
\catcode13=5 % ^^M
\endlinechar=13 %
\catcode35=6 % #
\catcode64=11 % @
\catcode123=1 % {
\catcode125=2 % }
\def\TMP@EnsureCode#1#2{%
  \edef\rc@AtEnd{%
    \rc@AtEnd
    \catcode#1=\the\catcode#1\relax
  }%
  \catcode#1=#2\relax
}
\TMP@EnsureCode{33}{12}% !
\TMP@EnsureCode{39}{12}% '
\TMP@EnsureCode{42}{12}% *
\TMP@EnsureCode{45}{12}% -
\TMP@EnsureCode{46}{12}% .
\TMP@EnsureCode{47}{12}% /
\TMP@EnsureCode{91}{12}% [
\TMP@EnsureCode{93}{12}% ]
\TMP@EnsureCode{96}{12}% `
\edef\rc@AtEnd{\rc@AtEnd\noexpand\endinput}
%    \end{macrocode}
%
% \subsection{Loading packages}
%
%    \begin{macrocode}
\begingroup\expandafter\expandafter\expandafter\endgroup
\expandafter\ifx\csname RequirePackage\endcsname\relax
  \input ltxcmds.sty\relax
  \input infwarerr.sty\relax
\else
  \RequirePackage{ltxcmds}[2011/11/09]%
  \RequirePackage{infwarerr}[2010/04/08]%
\fi
%    \end{macrocode}
%
% \subsection{Defining commands}
%
%    \begin{macro}{\rc@IfDefinable}
%    \begin{macrocode}
\ltx@IfUndefined{@ifdefinable}{%
  \def\rc@IfDefinable#1{%
    \ifx#1\ltx@undefined
      \expandafter\ltx@firstofone
    \else
      \ifx#1\relax
        \expandafter\expandafter\expandafter\ltx@firstofone
      \else
        \@PackageError{refcount}{%
          Command \string#1 is already defined.\MessageBreak
          It will not redefined by this package%
        }\@ehc
        \expandafter\expandafter\expandafter\ltx@gobble
      \fi
    \fi
  }%
}{%
  \let\rc@IfDefinable\@ifdefinable
}
%    \end{macrocode}
%    \end{macro}
%
%    \begin{macro}{\rc@RobustDefOne}
%    \begin{macro}{\rc@RobustDefZero}
%    \begin{macrocode}
\ltx@IfUndefined{protected}{%
  \ltx@IfUndefined{DeclareRobustCommand}{%
    \def\rc@RobustDefOne#1#2#3#4{%
      \rc@IfDefinable#3{%
        #1\def#3##1{#4}%
      }%
    }%
    \def\rc@RobustDefZero#1#2{%
      \rc@IfDefinable#1{%
        \def#1{#2}%
      }%
    }%
  }{%
    \def\rc@RobustDefOne#1#2#3#4{%
      \rc@IfDefinable#3{%
        \DeclareRobustCommand#2#3[1]{#4}%
      }%
    }%
    \def\rc@RobustDefZero#1#2{%
      \rc@IfDefinable#1{%
        \DeclareRobustCommand#1{#2}%
      }%
    }%
  }%
}{%
  \def\rc@RobustDefOne#1#2#3#4{%
    \rc@IfDefinable#3{%
      \protected#1\def#3##1{#4}%
    }%
  }%
  \def\rc@RobustDefZero#1#2{%
    \rc@IfDefinable#1{%
      \protected\def#1{#2}%
    }%
  }%
}
%    \end{macrocode}
%    \end{macro}
%    \end{macro}
%
%    \begin{macro}{\rc@newcommand}
%    \begin{macrocode}
\ltx@IfUndefined{newcommand}{%
  \def\rc@newcommand*#1[#2]#3{% hash-ok
    \rc@IfDefinable#1{%
      \ifcase#2 %
        \def#1{#3}%
      \or
        \def#1##1{#3}%
      \or
        \def#1##1##2{#3}%
      \else
        \rc@InternalError
      \fi
    }%
  }%
}{%
  \let\rc@newcommand\newcommand
}
%    \end{macrocode}
%    \end{macro}
%
% \subsection{\cs{setrefcountdefault}}
%
%    \begin{macro}{\setrefcountdefault}
%    \begin{macrocode}
\rc@RobustDefOne\long{}\setrefcountdefault{%
  \def\rc@default{#1}%
}
%    \end{macrocode}
%    \end{macro}
%    \begin{macrocode}
\setrefcountdefault{0}
%    \end{macrocode}
%
% \subsection{\cs{refused}}
%
%    \begin{macro}{\refused}
%    \begin{macrocode}
\ltx@IfUndefined{G@refundefinedtrue}{%
  \rc@RobustDefOne{}{*}\refused{%
    \begingroup
      \csname @safe@activestrue\endcsname
      \ltx@IfUndefined{r@#1}{%
        \protect\G@refundefinedtrue
        \rc@WarningUndefined{#1}%
      }{}%
    \endgroup
  }%
}{%
  \rc@RobustDefOne{}{*}\refused{%
    \begingroup
      \csname @safe@activestrue\endcsname
      \ltx@IfUndefined{r@#1}{%
        \csname protect\expandafter\endcsname
        \csname G@refundefinedtrue\endcsname
        \rc@WarningUndefined{#1}%
      }{}%
    \endgroup
  }%
}
%    \end{macrocode}
%    \end{macro}
%    \begin{macro}{\rc@WarningUndefined}
%    \begin{macrocode}
\ltx@IfUndefined{@latex@warning}{%
  \def\rc@WarningUndefined#1{%
    \ltx@ifundefined{thepage}{%
      \def\thepage{\number\count0 }%
    }{}%
    \@PackageWarning{refcount}{%
      Reference `#1' on page \thepage\space undefined%
    }%
  }%
}{%
  \def\rc@WarningUndefined#1{%
    \@latex@warning{%
      Reference `#1' on page \thepage\space undefined%
    }%
  }%
}
%    \end{macrocode}
%    \end{macro}
%
% \subsection{Setting counters by reference data}
%
% \subsubsection{Generic setting}
%
%    \begin{macro}{\rc@set}
% Generic command for
% |\|$\{$|set|$,$|addto|$\}$|counter|$\{$|page|$,\}$|ref|:
%\begin{quote}
%\begin{tabular}[t]{@{}l@{: }l@{}}
% |#1|& \cs{setcounter}, \cs{addtocounter}\\
% |#2|& \cs{ltx@car} (for \cs{ref}), \cs{ltx@cartwo} (for \cs{pageref})\\
% |#3|& \hologo{LaTeX} counter\\
% |#4|& reference\\
%\end{tabular}
%\end{quote}
%    \begin{macrocode}
\def\rc@set#1#2#3#4{%
  \begingroup
    \csname @safe@activestrue\endcsname
    \refused{#4}%
    \expandafter\rc@@set\csname r@#4\endcsname{#1}{#2}{#3}%
  \endgroup
}
%    \end{macrocode}
%    \end{macro}
%    \begin{macro}{\rc@@set}
%\begin{quote}
%\begin{tabular}[t]{@{}l@{: }l@{}}
% |#1|& \cs{r@<...>}\\
% |#2|& \cs{setcounter}, \cs{addtocounter}\\
% |#3|& \cs{ltx@car} (for \cs{ref}), \cs{ltx@carsecond} (for \cs{pageref})\\
% |#4|& \hologo{LaTeX} counter\\
%\end{tabular}
%\end{quote}
%    \begin{macrocode}
\def\rc@@set#1#2#3#4{%
  \ifx#1\relax
    #2{#4}{\rc@default}%
  \else
    #2{#4}{%
      \expandafter#3#1\rc@default\rc@default\@nil
    }%
  \fi
}
%    \end{macrocode}
%    \end{macro}
%
% \subsubsection{User commands}
%
%    \begin{macro}{\setcounterref}
%    \begin{macrocode}
\rc@RobustDefZero\setcounterref{%
  \rc@set\setcounter\ltx@car
}
%    \end{macrocode}
%    \end{macro}
%    \begin{macro}{\addtocounterref}
%    \begin{macrocode}
\rc@RobustDefZero\addtocounterref{%
  \rc@set\addtocounter\ltx@car
}
%    \end{macrocode}
%    \end{macro}
%    \begin{macro}{\setcounterpageref}
%    \begin{macrocode}
\rc@RobustDefZero\setcounterpageref{%
  \rc@set\setcounter\ltx@carsecond
}
%    \end{macrocode}
%    \end{macro}
%    \begin{macro}{\addtocounterpageref}
%    \begin{macrocode}
\rc@RobustDefZero\addtocounterpageref{%
  \rc@set\addtocounter\ltx@carsecond
}
%    \end{macrocode}
%    \end{macro}
%
% \subsection{Extracting references}
%
%    \begin{macro}{\getrefnumber}
%    \begin{macrocode}
\rc@newcommand*{\getrefnumber}[1]{%
  \romannumeral
  \ltx@ifundefined{r@#1}{%
    \expandafter\ltx@zero
    \rc@default
  }{%
    \expandafter\expandafter\expandafter\rc@extract@
    \expandafter\expandafter\expandafter!%
    \csname r@#1\expandafter\endcsname
    \expandafter{\rc@default}\@nil
  }%
}
%    \end{macrocode}
%    \end{macro}
%    \begin{macro}{\getpagerefnumber}
%    \begin{macrocode}
\rc@newcommand*{\getpagerefnumber}[1]{%
  \romannumeral
  \ltx@ifundefined{r@#1}{%
    \expandafter\ltx@zero
    \rc@default
  }{%
    \expandafter\expandafter\expandafter\rc@extract@page
    \expandafter\expandafter\expandafter!%
    \csname r@#1\expandafter\expandafter\expandafter\endcsname
    \expandafter\expandafter\expandafter{%
      \expandafter\rc@default
    \expandafter}\expandafter{\rc@default}\@nil
  }%
}
%    \end{macrocode}
%    \end{macro}
%    \begin{macro}{\getrefbykeydefault}
%    \begin{macrocode}
\rc@newcommand*{\getrefbykeydefault}[2]{%
  \romannumeral
  \expandafter\rc@getrefbykeydefault
    \csname r@#1\expandafter\endcsname
    \csname rc@extract@#2\endcsname
}
%    \end{macrocode}
%    \end{macro}
%    \begin{macro}{\rc@getrefbykeydefault}
%\begin{quote}
%\begin{tabular}[t]{@{}l@{: }l@{}}
% |#1|& \cs{r@<...>}\\
% |#2|& \cs{rc@extract@<...>}\\
% |#3|& default\\
%\end{tabular}
%\end{quote}
%    \begin{macrocode}
\long\def\rc@getrefbykeydefault#1#2#3{%
  \ifx#1\relax
    % reference is undefined
    \ltx@ReturnAfterElseFi{%
      \ltx@zero
      #3%
    }%
  \else
    \ltx@ReturnAfterFi{%
      \ifx#2\relax
        % extract method is missing
        \ltx@ReturnAfterElseFi{%
          \ltx@zero
          #3%
        }%
      \else
        \ltx@ReturnAfterFi{%
          \expandafter
          \rc@generic#1{#3}{#3}{#3}{#3}{#3}\@nil#2{#3}%
        }%
      \fi
    }%
  \fi
}
%    \end{macrocode}
%    \end{macro}
%    \begin{macro}{\rc@generic}
%\begin{quote}
%\begin{tabular}[t]{@{}l@{: }l@{}}
% |#1|& first item in \cs{r@<...>}\\
% |#2|& remaining items in \cs{r@<...>}\\
% |#3|& \cs{rc@extract@<...>}\\
% |#4|& default\\
%\end{tabular}
%\end{quote}
%    \begin{macrocode}
\long\def\rc@generic#1#2\@nil#3#4{%
  #3{#1\TR@TitleReference\@empty{#4}\@nil}{#1}#2\@nil
}
%    \end{macrocode}
%    \end{macro}
%    \begin{macro}{\rc@extract@}
%    \begin{macrocode}
\long\def\rc@extract@#1#2#3\@nil{%
  \ltx@zero
  #2%
}
%    \end{macrocode}
%    \end{macro}
%    \begin{macro}{\rc@extract@page}
%    \begin{macrocode}
\long\def\rc@extract@page#1#2#3#4\@nil{%
  \ltx@zero
  #3%
}
%    \end{macrocode}
%    \end{macro}
%    \begin{macro}{\rc@extract@name}
%    \begin{macrocode}
\long\def\rc@extract@name#1#2#3#4#5\@nil{%
  \ltx@zero
  #4%
}
%    \end{macrocode}
%    \end{macro}
%    \begin{macro}{\rc@extract@anchor}
%    \begin{macrocode}
\long\def\rc@extract@anchor#1#2#3#4#5#6\@nil{%
  \ltx@zero
  #5%
}
%    \end{macrocode}
%    \end{macro}
%    \begin{macro}{\rc@extract@url}
%    \begin{macrocode}
\long\def\rc@extract@url#1#2#3#4#5#6#7\@nil{%
  \ltx@zero
  #6%
}
%    \end{macrocode}
%    \end{macro}
%    \begin{macro}{\rc@extract@title}
%    \begin{macrocode}
\long\def\rc@extract@title#1#2\@nil{%
  \rc@@extract@title#1%
}
%    \end{macrocode}
%    \end{macro}
%    \begin{macro}{\rc@@extract@title}
%    \begin{macrocode}
\long\def\rc@@extract@title#1\TR@TitleReference#2#3#4\@nil{%
  \ltx@zero
  #3%
}
%    \end{macrocode}
%    \end{macro}
%
% \subsection{Macros for checking undefined references}
%
%    \begin{macro}{\IfRefUndefinedExpandable}
%    \begin{macrocode}
\rc@newcommand*{\IfRefUndefinedExpandable}[1]{%
  \ltx@ifundefined{r@#1}\ltx@firstoftwo\ltx@secondoftwo
}
%    \end{macrocode}
%    \end{macro}
%    \begin{macro}{\IfRefUndefinedBabel}
%    \begin{macrocode}
\rc@RobustDefOne{}*\IfRefUndefinedBabel{%
  \begingroup
    \csname safe@actives@true\endcsname
  \expandafter\expandafter\expandafter\endgroup
  \expandafter\ifx\csname r@#1\endcsname\relax
    \expandafter\ltx@firstoftwo
  \else
    \expandafter\ltx@secondoftwo
  \fi
}
%    \end{macrocode}
%    \end{macro}
%    \begin{macrocode}
\rc@AtEnd%
%</package>
%    \end{macrocode}
%
% \section{Test}
%
% \subsection{Catcode checks for loading}
%
%    \begin{macrocode}
%<*test1>
%    \end{macrocode}
%    \begin{macrocode}
\catcode`\{=1 %
\catcode`\}=2 %
\catcode`\#=6 %
\catcode`\@=11 %
\expandafter\ifx\csname count@\endcsname\relax
  \countdef\count@=255 %
\fi
\expandafter\ifx\csname @gobble\endcsname\relax
  \long\def\@gobble#1{}%
\fi
\expandafter\ifx\csname @firstofone\endcsname\relax
  \long\def\@firstofone#1{#1}%
\fi
\expandafter\ifx\csname loop\endcsname\relax
  \expandafter\@firstofone
\else
  \expandafter\@gobble
\fi
{%
  \def\loop#1\repeat{%
    \def\body{#1}%
    \iterate
  }%
  \def\iterate{%
    \body
      \let\next\iterate
    \else
      \let\next\relax
    \fi
    \next
  }%
  \let\repeat=\fi
}%
\def\RestoreCatcodes{}
\count@=0 %
\loop
  \edef\RestoreCatcodes{%
    \RestoreCatcodes
    \catcode\the\count@=\the\catcode\count@\relax
  }%
\ifnum\count@<255 %
  \advance\count@ 1 %
\repeat

\def\RangeCatcodeInvalid#1#2{%
  \count@=#1\relax
  \loop
    \catcode\count@=15 %
  \ifnum\count@<#2\relax
    \advance\count@ 1 %
  \repeat
}
\def\RangeCatcodeCheck#1#2#3{%
  \count@=#1\relax
  \loop
    \ifnum#3=\catcode\count@
    \else
      \errmessage{%
        Character \the\count@\space
        with wrong catcode \the\catcode\count@\space
        instead of \number#3%
      }%
    \fi
  \ifnum\count@<#2\relax
    \advance\count@ 1 %
  \repeat
}
\def\space{ }
\expandafter\ifx\csname LoadCommand\endcsname\relax
  \def\LoadCommand{\input refcount.sty\relax}%
\fi
\def\Test{%
  \RangeCatcodeInvalid{0}{47}%
  \RangeCatcodeInvalid{58}{64}%
  \RangeCatcodeInvalid{91}{96}%
  \RangeCatcodeInvalid{123}{255}%
  \catcode`\@=12 %
  \catcode`\\=0 %
  \catcode`\%=14 %
  \LoadCommand
  \RangeCatcodeCheck{0}{36}{15}%
  \RangeCatcodeCheck{37}{37}{14}%
  \RangeCatcodeCheck{38}{47}{15}%
  \RangeCatcodeCheck{48}{57}{12}%
  \RangeCatcodeCheck{58}{63}{15}%
  \RangeCatcodeCheck{64}{64}{12}%
  \RangeCatcodeCheck{65}{90}{11}%
  \RangeCatcodeCheck{91}{91}{15}%
  \RangeCatcodeCheck{92}{92}{0}%
  \RangeCatcodeCheck{93}{96}{15}%
  \RangeCatcodeCheck{97}{122}{11}%
  \RangeCatcodeCheck{123}{255}{15}%
  \RestoreCatcodes
}
\Test
\csname @@end\endcsname
\end
%    \end{macrocode}
%    \begin{macrocode}
%</test1>
%    \end{macrocode}
%
% \subsection{Macro tests}
%
%    \begin{macrocode}
%<*test2>
\errorcontextlines=10000 %
\showboxbreadth=10000 %
\showboxdepth=10000 %
\begingroup\expandafter\expandafter\expandafter\endgroup
\expandafter\ifx\csname RequirePackage\endcsname\relax
  \input refcount.sty\relax
\else
  \RequirePackage{refcount}[2011/10/16]%
\fi
\catcode`\@=11 %
\begingroup\expandafter\expandafter\expandafter\endgroup
\expandafter\ifx\csname @onelevel@sanitize\endcsname\relax
  \begingroup\expandafter\expandafter\expandafter\endgroup
  \expandafter\ifx\csname detokenize\endcsname\relax
    \def\strip@prefix#1->{}%
    \def\@onelevel@sanitize#1{%
      \edef#1{%
        \expandafter\strip@prefix\meaning#1%
      }%
    }%
  \else
    \def\@onelevel@sanitize#1{%
      \edef#1{%
        \detokenize\expandafter{#1}%
      }%
    }%
  \fi
\fi
\def\msg#{\immediate\write16}
\def\empty{}
\def\space{ }
%    \end{macrocode}
%    \begin{macrocode}
\def\r@foo{{\empty 1}{\empty 2}}
\long\def\test#1#2{%
  \begingroup
    \setbox0=\hbox{%
      \def\TestTask{#1}%
      \@onelevel@sanitize\TestTask
      \msg{* \TestTask}%
      \expandafter\expandafter\expandafter\def
      \expandafter\expandafter\expandafter\TestResult
      \expandafter\expandafter\expandafter{%
        #1%
      }%
      \def\TestExpected{#2}%
      \ifx\TestResult\TestExpected
        \msg{ \space ok.}%
      \else
        \@onelevel@sanitize\TestResult
        \@onelevel@sanitize\TestExpected
        \msg{ \space Result: \space\space[\TestResult]}%
        \msg{ \space Expected: [\TestExpected]}%
        \errmessage{Test failed!}%
      \fi
    }%
    \ifdim\wd0=0pt %
    \else
      \showbox0 %
    \fi
  \endgroup
}
\test{\getrefnumber{foo}}{\empty 1}
\test{\getpagerefnumber{foo}}{\empty 2}
\test{\getrefbykeydefault{foo}{}{\empty default}}{\empty 1}
\test{\getrefbykeydefault{foo}{page}{\empty default}}{\empty 2}
\test{\getrefbykeydefault{foo}{name}{\empty default}}{\empty default}
\test{\getrefbykeydefault{foo}{anchor}{\empty default}}{\empty default}
\test{\getrefbykeydefault{foo}{url}{\empty default}}{\empty default}
\test{\getrefbykeydefault{foo}{title}{\empty default}}{\empty default}
\msg{}
\def\r@foo{{}{}{}{}{}{}{}{}{}{}}
\def\Test#1#2\\{%
  \test{#1{foo}#2}{}%
}
\def\TestGroup{%
  \Test\getrefnumber\\%
  \Test\getpagerefnumber\\%
  \Test\getrefbykeydefault{}{}\\%
  \Test\getrefbykeydefault{page}{}\\%
  \Test\getrefbykeydefault{anchor}{}\\%
  \Test\getrefbykeydefault{name}{}\\%
  \Test\getrefbykeydefault{url}{}\\%
}
\TestGroup
\Test\getrefbykeydefault{title}{}\\%
\msg{}
\def\r@foo{\par\par\par\par\par\par\par\par}
\long\def\Test#1#2\\{%
  \test{#1{foo}#2}{\par}%
}
\TestGroup
\test{\getrefbykeydefault{title}{}{}}{}
\msg{}
\def\r@foo{{ }{ }{ }{ }{ }}
\def\Test#1#2\\{%
  \test{#1{foo}#2}{ }%
}
\TestGroup
\msg{}
\long\def\TestDefault#1{%
  \begingroup
    \setrefcountdefault{#1}%
    \test{\getrefnumber{foo}}{#1}%
    \test{\getpagerefnumber{foo}}{#1}%
  \endgroup
}
\def\TestDefaultX{%
  \TestDefault{}%
  \TestDefault{\par}%
  \TestDefault{ }%
  \TestDefault{\space}%
}
\let\r@foo\@undefined
\TestDefaultX
\let\r@foo\relax
\TestDefaultX
\def\r@foo{}
\TestDefaultX
%    \end{macrocode}
%    \begin{macrocode}
\msg{}
\long\def\Test#1#2#3#4{%
  \begingroup
    \def\TestTask{#1}%
    \@onelevel@sanitize\TestTask
    \msg{* [\TestTask]}%
    \edef\TestResultA{\IfRefUndefinedExpandable{#1}{#2}{#3}}%
    \IfRefUndefinedBabel{#1}{%
      \def\TestResultB{#2}%
    }{%
      \def\TestResultB{#3}%
    }%
    \def\TestExpected{#4}%
    \ifx\TestResultA\TestExpected
      \msg{ \space ok.}%
    \else
      \begingroup
        \@onelevel@sanitize\TestResultA
        \@onelevel@sanitize\TestExpected
        \msg{ \space Result: \space\space[\TestResultA]}%
        \msg{ \space Expected: [\TestExpected]}%
        \errmessage{Test failed!}%
      \endgroup
    \fi
    \ifx\TestResultB\TestExpected
      \msg{ \space ok.}%
    \else
      \begingroup
        \@onelevel@sanitize\TestResultB
        \@onelevel@sanitize\TestExpected
        \msg{ \space Result: \space\space[\TestResultB]}%
        \msg{ \space Expected: [\TestExpected]}%
        \errmessage{Test failed!}%
      \endgroup
    \fi
  \endgroup
}
\begingroup
  \def\r@foo{{}{}}%
  \let\r@bar\@undefined
  \let\r@xyz\relax
  \Test{foo}{true}{false}{false}%
  \Test{bar}{true}{false}{true}%
  \Test{xyz}{true}{false}{true}%
\endgroup
%    \end{macrocode}
%    \begin{macrocode}
\csname @@end\endcsname\end
%</test2>
%    \end{macrocode}
% \subsection{Test with package \xpackage{titleref}}
%
%    \begin{macrocode}
%<*test3>
\NeedsTeXFormat{LaTeX2e}
\documentclass{article}
\usepackage{refcount}[2011/10/16]
%<test4>\usepackage{nameref}
\usepackage{titleref}
\begin{document}
\section{Hello World}
\label{sec:hello}
\section{\hbox{xy}}
\label{sec:foo}
%
\makeatletter
\@ifundefined{r@sec:hello}{%
  \typeout{==> Compile twice!}%
}{%
  \def\test#1#2{%
    \begingroup
      \def\TestTask{#1}%
      \@onelevel@sanitize\TestTask
      \typeout{* \TestTask}%
      \expandafter\expandafter\expandafter\def
      \expandafter\expandafter\expandafter\TestResult
      \expandafter\expandafter\expandafter{%
        #1%
      }%
      \def\TestExpected{#2}%
      \ifx\TestResult\TestExpected
        \typeout{ \space ok.}%
      \else
        \@onelevel@sanitize\TestResult
        \@onelevel@sanitize\TestExpected
        \typeout{ \space Result: \space\space[\TestResult]}%
        \typeout{ \space Expected: [\TestExpected]}%
        \errmessage{Test failed!}%
      \fi
    \endgroup
  }%
  \test{\getrefbykeydefault{sec:hello}{title}{}}{Hello World}%
  \test{\getrefbykeydefault{sec:foo}{title}{}}{\hbox{xy}}%
  \begingroup
    \def\hbox#1{[#1]}% hash-ok
    \test{\getrefbykeydefault{sec:foo}{title}{}}{\hbox{xy}}%
  \endgroup
}
\makeatother
%    \end{macrocode}
%    \begin{macrocode}
\end{document}
%</test3>
%    \end{macrocode}
%    \begin{macrocode}
%<*test5>
\NeedsTeXFormat{LaTeX2e}
\documentclass{book}
\usepackage{refcount}[2011/10/16]
\usepackage{zref-runs}
\newcounter{test}
\begin{document}
\ifnum\zruns>1 %
  \makeatletter
  \def\Test#1#2#3{%
    \begingroup
      \setcounter{test}{10}%
      \sbox0{%
        #1{test}{#2}%
        \ifnum#3=\value{test}%
        \else
          \PackageError{test}{\string#1{#2} <> #3 (\the\value{test})}%
        \fi
      }%
      \ifdim\wd0=0pt %
      \else
        \PackageError{test}{Non-empty box}\@ehc
      \fi
    \endgroup
  }%
  \makeatother
  \Test\setcounterpageref{ch:two}{1}%
  \Test\setcounterpageref{ch:three}{3}%
  \Test\setcounterpageref{ch:four}{5}%
  \Test\setcounterpageref{ch:five}{7}%
  \Test\setcounterpageref{ch:six}{9}%
  \Test\setcounterpageref{ch:seven}{13}%
  \Test\addtocounterpageref{ch:two}{11}%
  \Test\addtocounterpageref{ch:three}{13}%
  \Test\addtocounterpageref{ch:four}{15}%
  \Test\addtocounterpageref{ch:five}{17}%
  \Test\addtocounterpageref{ch:six}{19}%
  \Test\addtocounterpageref{ch:seven}{23}%
  \Test\setcounterref{ch:two}{1}%
  \Test\setcounterref{ch:three}{2}%
  \Test\setcounterref{ch:four}{11}%
  \Test\addtocounterref{ch:two}{11}%
  \Test\addtocounterref{ch:three}{12}%
  \Test\addtocounterref{ch:four}{21}%
\fi
\frontmatter
\chapter{Chapter one}\label{ch:one}
\cleardoublepage
\mainmatter
\chapter{Chapter two}\label{ch:two}
\cleardoublepage
\chapter{Chapter three}\label{ch:three}
\cleardoublepage
\setcounter{chapter}{10}
\chapter{Chapter four}\label{ch:four}
\cleardoublepage
\appendix
\chapter{Chapter five}\label{ch:five}
\cleardoublepage
\chapter{Chapter six}\label{ch:six}
\cleardoublepage
\null
\cleardoublepage
\chapter{Chapter seven}\label{ch:seven}
\end{document}
%</test5>
%    \end{macrocode}
%
% \section{Installation}
%
% \subsection{Download}
%
% \paragraph{Package.} This package is available on
% CTAN\footnote{\url{ftp://ftp.ctan.org/tex-archive/}}:
% \begin{description}
% \item[\CTAN{macros/latex/contrib/oberdiek/refcount.dtx}] The source file.
% \item[\CTAN{macros/latex/contrib/oberdiek/refcount.pdf}] Documentation.
% \end{description}
%
%
% \paragraph{Bundle.} All the packages of the bundle `oberdiek'
% are also available in a TDS compliant ZIP archive. There
% the packages are already unpacked and the documentation files
% are generated. The files and directories obey the TDS standard.
% \begin{description}
% \item[\CTAN{install/macros/latex/contrib/oberdiek.tds.zip}]
% \end{description}
% \emph{TDS} refers to the standard ``A Directory Structure
% for \TeX\ Files'' (\CTAN{tds/tds.pdf}). Directories
% with \xfile{texmf} in their name are usually organized this way.
%
% \subsection{Bundle installation}
%
% \paragraph{Unpacking.} Unpack the \xfile{oberdiek.tds.zip} in the
% TDS tree (also known as \xfile{texmf} tree) of your choice.
% Example (linux):
% \begin{quote}
%   |unzip oberdiek.tds.zip -d ~/texmf|
% \end{quote}
%
% \paragraph{Script installation.}
% Check the directory \xfile{TDS:scripts/oberdiek/} for
% scripts that need further installation steps.
% Package \xpackage{attachfile2} comes with the Perl script
% \xfile{pdfatfi.pl} that should be installed in such a way
% that it can be called as \texttt{pdfatfi}.
% Example (linux):
% \begin{quote}
%   |chmod +x scripts/oberdiek/pdfatfi.pl|\\
%   |cp scripts/oberdiek/pdfatfi.pl /usr/local/bin/|
% \end{quote}
%
% \subsection{Package installation}
%
% \paragraph{Unpacking.} The \xfile{.dtx} file is a self-extracting
% \docstrip\ archive. The files are extracted by running the
% \xfile{.dtx} through \plainTeX:
% \begin{quote}
%   \verb|tex refcount.dtx|
% \end{quote}
%
% \paragraph{TDS.} Now the different files must be moved into
% the different directories in your installation TDS tree
% (also known as \xfile{texmf} tree):
% \begin{quote}
% \def\t{^^A
% \begin{tabular}{@{}>{\ttfamily}l@{ $\rightarrow$ }>{\ttfamily}l@{}}
%   refcount.sty & tex/latex/oberdiek/refcount.sty\\
%   refcount.pdf & doc/latex/oberdiek/refcount.pdf\\
%   test/refcount-test1.tex & doc/latex/oberdiek/test/refcount-test1.tex\\
%   test/refcount-test2.tex & doc/latex/oberdiek/test/refcount-test2.tex\\
%   test/refcount-test3.tex & doc/latex/oberdiek/test/refcount-test3.tex\\
%   test/refcount-test4.tex & doc/latex/oberdiek/test/refcount-test4.tex\\
%   test/refcount-test5.tex & doc/latex/oberdiek/test/refcount-test5.tex\\
%   refcount.dtx & source/latex/oberdiek/refcount.dtx\\
% \end{tabular}^^A
% }^^A
% \sbox0{\t}^^A
% \ifdim\wd0>\linewidth
%   \begingroup
%     \advance\linewidth by\leftmargin
%     \advance\linewidth by\rightmargin
%   \edef\x{\endgroup
%     \def\noexpand\lw{\the\linewidth}^^A
%   }\x
%   \def\lwbox{^^A
%     \leavevmode
%     \hbox to \linewidth{^^A
%       \kern-\leftmargin\relax
%       \hss
%       \usebox0
%       \hss
%       \kern-\rightmargin\relax
%     }^^A
%   }^^A
%   \ifdim\wd0>\lw
%     \sbox0{\small\t}^^A
%     \ifdim\wd0>\linewidth
%       \ifdim\wd0>\lw
%         \sbox0{\footnotesize\t}^^A
%         \ifdim\wd0>\linewidth
%           \ifdim\wd0>\lw
%             \sbox0{\scriptsize\t}^^A
%             \ifdim\wd0>\linewidth
%               \ifdim\wd0>\lw
%                 \sbox0{\tiny\t}^^A
%                 \ifdim\wd0>\linewidth
%                   \lwbox
%                 \else
%                   \usebox0
%                 \fi
%               \else
%                 \lwbox
%               \fi
%             \else
%               \usebox0
%             \fi
%           \else
%             \lwbox
%           \fi
%         \else
%           \usebox0
%         \fi
%       \else
%         \lwbox
%       \fi
%     \else
%       \usebox0
%     \fi
%   \else
%     \lwbox
%   \fi
% \else
%   \usebox0
% \fi
% \end{quote}
% If you have a \xfile{docstrip.cfg} that configures and enables \docstrip's
% TDS installing feature, then some files can already be in the right
% place, see the documentation of \docstrip.
%
% \subsection{Refresh file name databases}
%
% If your \TeX~distribution
% (\teTeX, \mikTeX, \dots) relies on file name databases, you must refresh
% these. For example, \teTeX\ users run \verb|texhash| or
% \verb|mktexlsr|.
%
% \subsection{Some details for the interested}
%
% \paragraph{Attached source.}
%
% The PDF documentation on CTAN also includes the
% \xfile{.dtx} source file. It can be extracted by
% AcrobatReader 6 or higher. Another option is \textsf{pdftk},
% e.g. unpack the file into the current directory:
% \begin{quote}
%   \verb|pdftk refcount.pdf unpack_files output .|
% \end{quote}
%
% \paragraph{Unpacking with \LaTeX.}
% The \xfile{.dtx} chooses its action depending on the format:
% \begin{description}
% \item[\plainTeX:] Run \docstrip\ and extract the files.
% \item[\LaTeX:] Generate the documentation.
% \end{description}
% If you insist on using \LaTeX\ for \docstrip\ (really,
% \docstrip\ does not need \LaTeX), then inform the autodetect routine
% about your intention:
% \begin{quote}
%   \verb|latex \let\install=y% \iffalse meta-comment
%
% File: refcount.dtx
% Version: 2011/10/16 v3.4
% Info: Data extraction from label references
%
% Copyright (C) 1998, 2000, 2006, 2008, 2010, 2011 by
%    Heiko Oberdiek <heiko.oberdiek at googlemail.com>
%
% This work may be distributed and/or modified under the
% conditions of the LaTeX Project Public License, either
% version 1.3c of this license or (at your option) any later
% version. This version of this license is in
%    http://www.latex-project.org/lppl/lppl-1-3c.txt
% and the latest version of this license is in
%    http://www.latex-project.org/lppl.txt
% and version 1.3 or later is part of all distributions of
% LaTeX version 2005/12/01 or later.
%
% This work has the LPPL maintenance status "maintained".
%
% This Current Maintainer of this work is Heiko Oberdiek.
%
% This work consists of the main source file refcount.dtx
% and the derived files
%    refcount.sty, refcount.pdf, refcount.ins, refcount.drv,
%    refcount-test1.tex, refcount-test2.tex, refcount-test3.tex,
%    refcount-test4.tex, refcount-test5.tex.
%
% Distribution:
%    CTAN:macros/latex/contrib/oberdiek/refcount.dtx
%    CTAN:macros/latex/contrib/oberdiek/refcount.pdf
%
% Unpacking:
%    (a) If refcount.ins is present:
%           tex refcount.ins
%    (b) Without refcount.ins:
%           tex refcount.dtx
%    (c) If you insist on using LaTeX
%           latex \let\install=y% \iffalse meta-comment
%
% File: refcount.dtx
% Version: 2011/10/16 v3.4
% Info: Data extraction from label references
%
% Copyright (C) 1998, 2000, 2006, 2008, 2010, 2011 by
%    Heiko Oberdiek <heiko.oberdiek at googlemail.com>
%
% This work may be distributed and/or modified under the
% conditions of the LaTeX Project Public License, either
% version 1.3c of this license or (at your option) any later
% version. This version of this license is in
%    http://www.latex-project.org/lppl/lppl-1-3c.txt
% and the latest version of this license is in
%    http://www.latex-project.org/lppl.txt
% and version 1.3 or later is part of all distributions of
% LaTeX version 2005/12/01 or later.
%
% This work has the LPPL maintenance status "maintained".
%
% This Current Maintainer of this work is Heiko Oberdiek.
%
% This work consists of the main source file refcount.dtx
% and the derived files
%    refcount.sty, refcount.pdf, refcount.ins, refcount.drv,
%    refcount-test1.tex, refcount-test2.tex, refcount-test3.tex,
%    refcount-test4.tex, refcount-test5.tex.
%
% Distribution:
%    CTAN:macros/latex/contrib/oberdiek/refcount.dtx
%    CTAN:macros/latex/contrib/oberdiek/refcount.pdf
%
% Unpacking:
%    (a) If refcount.ins is present:
%           tex refcount.ins
%    (b) Without refcount.ins:
%           tex refcount.dtx
%    (c) If you insist on using LaTeX
%           latex \let\install=y\input{refcount.dtx}
%        (quote the arguments according to the demands of your shell)
%
% Documentation:
%    (a) If refcount.drv is present:
%           latex refcount.drv
%    (b) Without refcount.drv:
%           latex refcount.dtx; ...
%    The class ltxdoc loads the configuration file ltxdoc.cfg
%    if available. Here you can specify further options, e.g.
%    use A4 as paper format:
%       \PassOptionsToClass{a4paper}{article}
%
%    Programm calls to get the documentation (example):
%       pdflatex refcount.dtx
%       makeindex -s gind.ist refcount.idx
%       pdflatex refcount.dtx
%       makeindex -s gind.ist refcount.idx
%       pdflatex refcount.dtx
%
% Installation:
%    TDS:tex/latex/oberdiek/refcount.sty
%    TDS:doc/latex/oberdiek/refcount.pdf
%    TDS:doc/latex/oberdiek/test/refcount-test1.tex
%    TDS:doc/latex/oberdiek/test/refcount-test2.tex
%    TDS:doc/latex/oberdiek/test/refcount-test3.tex
%    TDS:doc/latex/oberdiek/test/refcount-test4.tex
%    TDS:doc/latex/oberdiek/test/refcount-test5.tex
%    TDS:source/latex/oberdiek/refcount.dtx
%
%<*ignore>
\begingroup
  \catcode123=1 %
  \catcode125=2 %
  \def\x{LaTeX2e}%
\expandafter\endgroup
\ifcase 0\ifx\install y1\fi\expandafter
         \ifx\csname processbatchFile\endcsname\relax\else1\fi
         \ifx\fmtname\x\else 1\fi\relax
\else\csname fi\endcsname
%</ignore>
%<*install>
\input docstrip.tex
\Msg{************************************************************************}
\Msg{* Installation}
\Msg{* Package: refcount 2011/10/16 v3.4 Data extraction from label references (HO)}
\Msg{************************************************************************}

\keepsilent
\askforoverwritefalse

\let\MetaPrefix\relax
\preamble

This is a generated file.

Project: refcount
Version: 2011/10/16 v3.4

Copyright (C) 1998, 2000, 2006, 2008, 2010, 2011 by
   Heiko Oberdiek <heiko.oberdiek at googlemail.com>

This work may be distributed and/or modified under the
conditions of the LaTeX Project Public License, either
version 1.3c of this license or (at your option) any later
version. This version of this license is in
   http://www.latex-project.org/lppl/lppl-1-3c.txt
and the latest version of this license is in
   http://www.latex-project.org/lppl.txt
and version 1.3 or later is part of all distributions of
LaTeX version 2005/12/01 or later.

This work has the LPPL maintenance status "maintained".

This Current Maintainer of this work is Heiko Oberdiek.

This work consists of the main source file refcount.dtx
and the derived files
   refcount.sty, refcount.pdf, refcount.ins, refcount.drv,
   refcount-test1.tex, refcount-test2.tex, refcount-test3.tex,
   refcount-test4.tex, refcount-test5.tex.

\endpreamble
\let\MetaPrefix\DoubleperCent

\generate{%
  \file{refcount.ins}{\from{refcount.dtx}{install}}%
  \file{refcount.drv}{\from{refcount.dtx}{driver}}%
  \usedir{tex/latex/oberdiek}%
  \file{refcount.sty}{\from{refcount.dtx}{package}}%
  \usedir{doc/latex/oberdiek/test}%
  \file{refcount-test1.tex}{\from{refcount.dtx}{test1}}%
  \file{refcount-test2.tex}{\from{refcount.dtx}{test2}}%
  \file{refcount-test3.tex}{\from{refcount.dtx}{test3}}%
  \file{refcount-test4.tex}{\from{refcount.dtx}{test3,test4}}%
  \file{refcount-test5.tex}{\from{refcount.dtx}{test5}}%
  \nopreamble
  \nopostamble
  \usedir{source/latex/oberdiek/catalogue}%
  \file{refcount.xml}{\from{refcount.dtx}{catalogue}}%
}

\catcode32=13\relax% active space
\let =\space%
\Msg{************************************************************************}
\Msg{*}
\Msg{* To finish the installation you have to move the following}
\Msg{* file into a directory searched by TeX:}
\Msg{*}
\Msg{*     refcount.sty}
\Msg{*}
\Msg{* To produce the documentation run the file `refcount.drv'}
\Msg{* through LaTeX.}
\Msg{*}
\Msg{* Happy TeXing!}
\Msg{*}
\Msg{************************************************************************}

\endbatchfile
%</install>
%<*ignore>
\fi
%</ignore>
%<*driver>
\NeedsTeXFormat{LaTeX2e}
\ProvidesFile{refcount.drv}%
  [2011/10/16 v3.4 Data extraction from label references (HO)]%
\documentclass{ltxdoc}
\usepackage{holtxdoc}[2011/11/22]
\begin{document}
  \DocInput{refcount.dtx}%
\end{document}
%</driver>
% \fi
%
% \CheckSum{1185}
%
% \CharacterTable
%  {Upper-case    \A\B\C\D\E\F\G\H\I\J\K\L\M\N\O\P\Q\R\S\T\U\V\W\X\Y\Z
%   Lower-case    \a\b\c\d\e\f\g\h\i\j\k\l\m\n\o\p\q\r\s\t\u\v\w\x\y\z
%   Digits        \0\1\2\3\4\5\6\7\8\9
%   Exclamation   \!     Double quote  \"     Hash (number) \#
%   Dollar        \$     Percent       \%     Ampersand     \&
%   Acute accent  \'     Left paren    \(     Right paren   \)
%   Asterisk      \*     Plus          \+     Comma         \,
%   Minus         \-     Point         \.     Solidus       \/
%   Colon         \:     Semicolon     \;     Less than     \<
%   Equals        \=     Greater than  \>     Question mark \?
%   Commercial at \@     Left bracket  \[     Backslash     \\
%   Right bracket \]     Circumflex    \^     Underscore    \_
%   Grave accent  \`     Left brace    \{     Vertical bar  \|
%   Right brace   \}     Tilde         \~}
%
% \GetFileInfo{refcount.drv}
%
% \title{The \xpackage{refcount} package}
% \date{2011/10/16 v3.4}
% \author{Heiko Oberdiek\\\xemail{heiko.oberdiek at googlemail.com}}
%
% \maketitle
%
% \begin{abstract}
% References are not numbers, however they often store numerical
% data such as section or page numbers. \cs{ref} or \cs{pageref}
% cannot be used for counter assignments or calculations because
% they are not expandable, generate warnings, or can even be links.
% The package provides expandable macros to extract the data
% from references. Packages \xpackage{hyperref}, \xpackage{nameref},
% \xpackage{titleref}, and \xpackage{babel} are supported.
% \end{abstract}
%
% \tableofcontents
%
% \section{Usage}
%
% \subsection{Setting counters}
%
% The following commands are similar to \LaTeX's
% \cs{setcounter} and \cs{addtocounter},
% but they extract the number value from a reference:
% \begin{quote}
%   \cs{setcounterref}, \cs{addtocounterref}\\
%   \cs{setcounterpageref}, \cs{addtocounterpageref}
% \end{quote}
% They take two arguments:
% \begin{quote}
%    \cs{...counter...ref} |{|\meta{\LaTeX\ counter}|}|
%    |{|\meta{reference}|}|
% \end{quote}
% An undefined references produces the usual LaTeX warning
% and its value is assumed to be zero.
% Example:
% \begin{quote}
%\begin{verbatim}
%\newcounter{ctrA}
%\newcounter{ctrB}
%\refstepcounter{ctrA}\label{ref:A}
%\setcounterref{ctrB}{ref:A}
%\addtocounterpageref{ctrB}{ref:A}
%\end{verbatim}
% \end{quote}
%
% \subsection{Expandable commands}
%
% These commands that can be used in expandible contexts
% (inside calculations, \cs{edef}, \cs{csname}, \cs{write}, \dots):
% \begin{quote}
%   \cs{getrefnumber}, \cs{getpagerefnumber}
% \end{quote}
% They take one argument, the reference:
% \begin{quote}
%   \cs{get...refnumber} |{|\meta{reference}|}|
% \end{quote}
% The default for undefined references can be changed
% with macro \cs{setrefcountdefault}, for example this
% package calls:
% \begin{quote}
%   \cs{setrefcountdefault}|{0}|
% \end{quote}
%
% Since version 2.0 of this package there is a new
% command:
% \begin{quote}
%   \cs{getrefbykeydefault} |{|\meta{reference}|}|
%   |{|\meta{key}|}| |{|\meta{default}|}|
% \end{quote}
% This generalized version allows the extraction
% of further properties of a reference than the
% two standard ones. Thus the following properties
% are supported, if they are available:
% \begin{quote}
% \begin{tabular}{@{}l|l|l@{}}
%    Key & Description & Package\\
% \hline
%   \meta{empty} & same as \cs{ref} & \LaTeX\\
%   |page| & same as \cs{pageref} & \LaTeX\\
%   |title| & section and caption titles & \xpackage{titleref}\\
%   |name| & section and caption titles & \xpackage{nameref}\\
%   |anchor| & anchor name & \xpackage{hyperref}\\
%   |url| & url/file & \xpackage{hyperref}/\xpackage{xr}
% \end{tabular}
% \end{quote}
%
% Since version 3.2 the expandable macros described before
% in this section
% are expandable in exact two expansion steps.
%
% \subsection{Undefined references}
%
% Because warnings and assignments cannot be used in
% expandible contexts, undefined references do not
% produce a warning, their values are assumed to be zero.
% Example:
% \begin{quote}
%\begin{verbatim}
%\label{ref:here}% somewhere
%\refused{ref:here}% see below
%\ifodd\getpagerefnumber{ref:here}%
%  reference is on an odd page
%\else
%  reference is on an even page
%\fi
%\end{verbatim}
% \end{quote}
%
% In case of undefined references the user usually want's
% to be informed. Also \LaTeX\ prints a warning at
% the end of the \LaTeX\ run. To notify \LaTeX\ and
% get a normal warning, just use
% \begin{quote}
%   \cs{refused} |{|\meta{reference}|}|
% \end{quote}
% outside the expanding context. Example, see above.
%
% \subsubsection{Check for undefined references}
%
% In version 3.2 macros were added, that test, whether references
% are defined.
% \begin{declcs}{IfRefUndefinedExpandable} \M{refname} \M{then} \M{else}\\
%   \cs{IfRefUndefinedBabel} \M{refname} \M{then} \M{else}
% \end{declcs}
% If the reference is not available and therefore undefined, then
% argument \meta{then} is executed, otherwise argument \meta{else}
% is called. Macro \cs{IfRefUndefinedExpandable} is expandable,
% but \meta{refname} must not contain babel shorthand characters.
% Macro \cs{IfRefUndefinedBabel} supports shorthand characters of
% babel, but it is not expandable.
%
% \subsection{Notes}
%
% \begin{itemize}
% \item
%   The method of extracting the number in this
%   package also works in cases, where the
%   reference cannot be used directly, because
%   a package such as \xpackage{hyperref} has added
%   extra stuff (hyper link), so that the reference cannot
%   be used as number any more.
% \item
%   If the reference does not contain a number,
%   assignments to a counter will fail of course.
% \end{itemize}
%
%
% \StopEventually{
% }
%
% \section{Implementation}
%
%    \begin{macrocode}
%<*package>
%    \end{macrocode}
%    Reload check, especially if the package is not used with \LaTeX.
%    \begin{macrocode}
\begingroup\catcode61\catcode48\catcode32=10\relax%
  \catcode13=5 % ^^M
  \endlinechar=13 %
  \catcode35=6 % #
  \catcode39=12 % '
  \catcode44=12 % ,
  \catcode45=12 % -
  \catcode46=12 % .
  \catcode58=12 % :
  \catcode64=11 % @
  \catcode123=1 % {
  \catcode125=2 % }
  \expandafter\let\expandafter\x\csname ver@refcount.sty\endcsname
  \ifx\x\relax % plain-TeX, first loading
  \else
    \def\empty{}%
    \ifx\x\empty % LaTeX, first loading,
      % variable is initialized, but \ProvidesPackage not yet seen
    \else
      \expandafter\ifx\csname PackageInfo\endcsname\relax
        \def\x#1#2{%
          \immediate\write-1{Package #1 Info: #2.}%
        }%
      \else
        \def\x#1#2{\PackageInfo{#1}{#2, stopped}}%
      \fi
      \x{refcount}{The package is already loaded}%
      \aftergroup\endinput
    \fi
  \fi
\endgroup%
%    \end{macrocode}
%    Package identification:
%    \begin{macrocode}
\begingroup\catcode61\catcode48\catcode32=10\relax%
  \catcode13=5 % ^^M
  \endlinechar=13 %
  \catcode35=6 % #
  \catcode39=12 % '
  \catcode40=12 % (
  \catcode41=12 % )
  \catcode44=12 % ,
  \catcode45=12 % -
  \catcode46=12 % .
  \catcode47=12 % /
  \catcode58=12 % :
  \catcode64=11 % @
  \catcode91=12 % [
  \catcode93=12 % ]
  \catcode123=1 % {
  \catcode125=2 % }
  \expandafter\ifx\csname ProvidesPackage\endcsname\relax
    \def\x#1#2#3[#4]{\endgroup
      \immediate\write-1{Package: #3 #4}%
      \xdef#1{#4}%
    }%
  \else
    \def\x#1#2[#3]{\endgroup
      #2[{#3}]%
      \ifx#1\@undefined
        \xdef#1{#3}%
      \fi
      \ifx#1\relax
        \xdef#1{#3}%
      \fi
    }%
  \fi
\expandafter\x\csname ver@refcount.sty\endcsname
\ProvidesPackage{refcount}%
  [2011/10/16 v3.4 Data extraction from label references (HO)]%
%    \end{macrocode}
%
%    \begin{macrocode}
\begingroup\catcode61\catcode48\catcode32=10\relax%
  \catcode13=5 % ^^M
  \endlinechar=13 %
  \catcode123=1 % {
  \catcode125=2 % }
  \catcode64=11 % @
  \def\x{\endgroup
    \expandafter\edef\csname rc@AtEnd\endcsname{%
      \endlinechar=\the\endlinechar\relax
      \catcode13=\the\catcode13\relax
      \catcode32=\the\catcode32\relax
      \catcode35=\the\catcode35\relax
      \catcode61=\the\catcode61\relax
      \catcode64=\the\catcode64\relax
      \catcode123=\the\catcode123\relax
      \catcode125=\the\catcode125\relax
    }%
  }%
\x\catcode61\catcode48\catcode32=10\relax%
\catcode13=5 % ^^M
\endlinechar=13 %
\catcode35=6 % #
\catcode64=11 % @
\catcode123=1 % {
\catcode125=2 % }
\def\TMP@EnsureCode#1#2{%
  \edef\rc@AtEnd{%
    \rc@AtEnd
    \catcode#1=\the\catcode#1\relax
  }%
  \catcode#1=#2\relax
}
\TMP@EnsureCode{33}{12}% !
\TMP@EnsureCode{39}{12}% '
\TMP@EnsureCode{42}{12}% *
\TMP@EnsureCode{45}{12}% -
\TMP@EnsureCode{46}{12}% .
\TMP@EnsureCode{47}{12}% /
\TMP@EnsureCode{91}{12}% [
\TMP@EnsureCode{93}{12}% ]
\TMP@EnsureCode{96}{12}% `
\edef\rc@AtEnd{\rc@AtEnd\noexpand\endinput}
%    \end{macrocode}
%
% \subsection{Loading packages}
%
%    \begin{macrocode}
\begingroup\expandafter\expandafter\expandafter\endgroup
\expandafter\ifx\csname RequirePackage\endcsname\relax
  \input ltxcmds.sty\relax
  \input infwarerr.sty\relax
\else
  \RequirePackage{ltxcmds}[2011/11/09]%
  \RequirePackage{infwarerr}[2010/04/08]%
\fi
%    \end{macrocode}
%
% \subsection{Defining commands}
%
%    \begin{macro}{\rc@IfDefinable}
%    \begin{macrocode}
\ltx@IfUndefined{@ifdefinable}{%
  \def\rc@IfDefinable#1{%
    \ifx#1\ltx@undefined
      \expandafter\ltx@firstofone
    \else
      \ifx#1\relax
        \expandafter\expandafter\expandafter\ltx@firstofone
      \else
        \@PackageError{refcount}{%
          Command \string#1 is already defined.\MessageBreak
          It will not redefined by this package%
        }\@ehc
        \expandafter\expandafter\expandafter\ltx@gobble
      \fi
    \fi
  }%
}{%
  \let\rc@IfDefinable\@ifdefinable
}
%    \end{macrocode}
%    \end{macro}
%
%    \begin{macro}{\rc@RobustDefOne}
%    \begin{macro}{\rc@RobustDefZero}
%    \begin{macrocode}
\ltx@IfUndefined{protected}{%
  \ltx@IfUndefined{DeclareRobustCommand}{%
    \def\rc@RobustDefOne#1#2#3#4{%
      \rc@IfDefinable#3{%
        #1\def#3##1{#4}%
      }%
    }%
    \def\rc@RobustDefZero#1#2{%
      \rc@IfDefinable#1{%
        \def#1{#2}%
      }%
    }%
  }{%
    \def\rc@RobustDefOne#1#2#3#4{%
      \rc@IfDefinable#3{%
        \DeclareRobustCommand#2#3[1]{#4}%
      }%
    }%
    \def\rc@RobustDefZero#1#2{%
      \rc@IfDefinable#1{%
        \DeclareRobustCommand#1{#2}%
      }%
    }%
  }%
}{%
  \def\rc@RobustDefOne#1#2#3#4{%
    \rc@IfDefinable#3{%
      \protected#1\def#3##1{#4}%
    }%
  }%
  \def\rc@RobustDefZero#1#2{%
    \rc@IfDefinable#1{%
      \protected\def#1{#2}%
    }%
  }%
}
%    \end{macrocode}
%    \end{macro}
%    \end{macro}
%
%    \begin{macro}{\rc@newcommand}
%    \begin{macrocode}
\ltx@IfUndefined{newcommand}{%
  \def\rc@newcommand*#1[#2]#3{% hash-ok
    \rc@IfDefinable#1{%
      \ifcase#2 %
        \def#1{#3}%
      \or
        \def#1##1{#3}%
      \or
        \def#1##1##2{#3}%
      \else
        \rc@InternalError
      \fi
    }%
  }%
}{%
  \let\rc@newcommand\newcommand
}
%    \end{macrocode}
%    \end{macro}
%
% \subsection{\cs{setrefcountdefault}}
%
%    \begin{macro}{\setrefcountdefault}
%    \begin{macrocode}
\rc@RobustDefOne\long{}\setrefcountdefault{%
  \def\rc@default{#1}%
}
%    \end{macrocode}
%    \end{macro}
%    \begin{macrocode}
\setrefcountdefault{0}
%    \end{macrocode}
%
% \subsection{\cs{refused}}
%
%    \begin{macro}{\refused}
%    \begin{macrocode}
\ltx@IfUndefined{G@refundefinedtrue}{%
  \rc@RobustDefOne{}{*}\refused{%
    \begingroup
      \csname @safe@activestrue\endcsname
      \ltx@IfUndefined{r@#1}{%
        \protect\G@refundefinedtrue
        \rc@WarningUndefined{#1}%
      }{}%
    \endgroup
  }%
}{%
  \rc@RobustDefOne{}{*}\refused{%
    \begingroup
      \csname @safe@activestrue\endcsname
      \ltx@IfUndefined{r@#1}{%
        \csname protect\expandafter\endcsname
        \csname G@refundefinedtrue\endcsname
        \rc@WarningUndefined{#1}%
      }{}%
    \endgroup
  }%
}
%    \end{macrocode}
%    \end{macro}
%    \begin{macro}{\rc@WarningUndefined}
%    \begin{macrocode}
\ltx@IfUndefined{@latex@warning}{%
  \def\rc@WarningUndefined#1{%
    \ltx@ifundefined{thepage}{%
      \def\thepage{\number\count0 }%
    }{}%
    \@PackageWarning{refcount}{%
      Reference `#1' on page \thepage\space undefined%
    }%
  }%
}{%
  \def\rc@WarningUndefined#1{%
    \@latex@warning{%
      Reference `#1' on page \thepage\space undefined%
    }%
  }%
}
%    \end{macrocode}
%    \end{macro}
%
% \subsection{Setting counters by reference data}
%
% \subsubsection{Generic setting}
%
%    \begin{macro}{\rc@set}
% Generic command for
% |\|$\{$|set|$,$|addto|$\}$|counter|$\{$|page|$,\}$|ref|:
%\begin{quote}
%\begin{tabular}[t]{@{}l@{: }l@{}}
% |#1|& \cs{setcounter}, \cs{addtocounter}\\
% |#2|& \cs{ltx@car} (for \cs{ref}), \cs{ltx@cartwo} (for \cs{pageref})\\
% |#3|& \hologo{LaTeX} counter\\
% |#4|& reference\\
%\end{tabular}
%\end{quote}
%    \begin{macrocode}
\def\rc@set#1#2#3#4{%
  \begingroup
    \csname @safe@activestrue\endcsname
    \refused{#4}%
    \expandafter\rc@@set\csname r@#4\endcsname{#1}{#2}{#3}%
  \endgroup
}
%    \end{macrocode}
%    \end{macro}
%    \begin{macro}{\rc@@set}
%\begin{quote}
%\begin{tabular}[t]{@{}l@{: }l@{}}
% |#1|& \cs{r@<...>}\\
% |#2|& \cs{setcounter}, \cs{addtocounter}\\
% |#3|& \cs{ltx@car} (for \cs{ref}), \cs{ltx@carsecond} (for \cs{pageref})\\
% |#4|& \hologo{LaTeX} counter\\
%\end{tabular}
%\end{quote}
%    \begin{macrocode}
\def\rc@@set#1#2#3#4{%
  \ifx#1\relax
    #2{#4}{\rc@default}%
  \else
    #2{#4}{%
      \expandafter#3#1\rc@default\rc@default\@nil
    }%
  \fi
}
%    \end{macrocode}
%    \end{macro}
%
% \subsubsection{User commands}
%
%    \begin{macro}{\setcounterref}
%    \begin{macrocode}
\rc@RobustDefZero\setcounterref{%
  \rc@set\setcounter\ltx@car
}
%    \end{macrocode}
%    \end{macro}
%    \begin{macro}{\addtocounterref}
%    \begin{macrocode}
\rc@RobustDefZero\addtocounterref{%
  \rc@set\addtocounter\ltx@car
}
%    \end{macrocode}
%    \end{macro}
%    \begin{macro}{\setcounterpageref}
%    \begin{macrocode}
\rc@RobustDefZero\setcounterpageref{%
  \rc@set\setcounter\ltx@carsecond
}
%    \end{macrocode}
%    \end{macro}
%    \begin{macro}{\addtocounterpageref}
%    \begin{macrocode}
\rc@RobustDefZero\addtocounterpageref{%
  \rc@set\addtocounter\ltx@carsecond
}
%    \end{macrocode}
%    \end{macro}
%
% \subsection{Extracting references}
%
%    \begin{macro}{\getrefnumber}
%    \begin{macrocode}
\rc@newcommand*{\getrefnumber}[1]{%
  \romannumeral
  \ltx@ifundefined{r@#1}{%
    \expandafter\ltx@zero
    \rc@default
  }{%
    \expandafter\expandafter\expandafter\rc@extract@
    \expandafter\expandafter\expandafter!%
    \csname r@#1\expandafter\endcsname
    \expandafter{\rc@default}\@nil
  }%
}
%    \end{macrocode}
%    \end{macro}
%    \begin{macro}{\getpagerefnumber}
%    \begin{macrocode}
\rc@newcommand*{\getpagerefnumber}[1]{%
  \romannumeral
  \ltx@ifundefined{r@#1}{%
    \expandafter\ltx@zero
    \rc@default
  }{%
    \expandafter\expandafter\expandafter\rc@extract@page
    \expandafter\expandafter\expandafter!%
    \csname r@#1\expandafter\expandafter\expandafter\endcsname
    \expandafter\expandafter\expandafter{%
      \expandafter\rc@default
    \expandafter}\expandafter{\rc@default}\@nil
  }%
}
%    \end{macrocode}
%    \end{macro}
%    \begin{macro}{\getrefbykeydefault}
%    \begin{macrocode}
\rc@newcommand*{\getrefbykeydefault}[2]{%
  \romannumeral
  \expandafter\rc@getrefbykeydefault
    \csname r@#1\expandafter\endcsname
    \csname rc@extract@#2\endcsname
}
%    \end{macrocode}
%    \end{macro}
%    \begin{macro}{\rc@getrefbykeydefault}
%\begin{quote}
%\begin{tabular}[t]{@{}l@{: }l@{}}
% |#1|& \cs{r@<...>}\\
% |#2|& \cs{rc@extract@<...>}\\
% |#3|& default\\
%\end{tabular}
%\end{quote}
%    \begin{macrocode}
\long\def\rc@getrefbykeydefault#1#2#3{%
  \ifx#1\relax
    % reference is undefined
    \ltx@ReturnAfterElseFi{%
      \ltx@zero
      #3%
    }%
  \else
    \ltx@ReturnAfterFi{%
      \ifx#2\relax
        % extract method is missing
        \ltx@ReturnAfterElseFi{%
          \ltx@zero
          #3%
        }%
      \else
        \ltx@ReturnAfterFi{%
          \expandafter
          \rc@generic#1{#3}{#3}{#3}{#3}{#3}\@nil#2{#3}%
        }%
      \fi
    }%
  \fi
}
%    \end{macrocode}
%    \end{macro}
%    \begin{macro}{\rc@generic}
%\begin{quote}
%\begin{tabular}[t]{@{}l@{: }l@{}}
% |#1|& first item in \cs{r@<...>}\\
% |#2|& remaining items in \cs{r@<...>}\\
% |#3|& \cs{rc@extract@<...>}\\
% |#4|& default\\
%\end{tabular}
%\end{quote}
%    \begin{macrocode}
\long\def\rc@generic#1#2\@nil#3#4{%
  #3{#1\TR@TitleReference\@empty{#4}\@nil}{#1}#2\@nil
}
%    \end{macrocode}
%    \end{macro}
%    \begin{macro}{\rc@extract@}
%    \begin{macrocode}
\long\def\rc@extract@#1#2#3\@nil{%
  \ltx@zero
  #2%
}
%    \end{macrocode}
%    \end{macro}
%    \begin{macro}{\rc@extract@page}
%    \begin{macrocode}
\long\def\rc@extract@page#1#2#3#4\@nil{%
  \ltx@zero
  #3%
}
%    \end{macrocode}
%    \end{macro}
%    \begin{macro}{\rc@extract@name}
%    \begin{macrocode}
\long\def\rc@extract@name#1#2#3#4#5\@nil{%
  \ltx@zero
  #4%
}
%    \end{macrocode}
%    \end{macro}
%    \begin{macro}{\rc@extract@anchor}
%    \begin{macrocode}
\long\def\rc@extract@anchor#1#2#3#4#5#6\@nil{%
  \ltx@zero
  #5%
}
%    \end{macrocode}
%    \end{macro}
%    \begin{macro}{\rc@extract@url}
%    \begin{macrocode}
\long\def\rc@extract@url#1#2#3#4#5#6#7\@nil{%
  \ltx@zero
  #6%
}
%    \end{macrocode}
%    \end{macro}
%    \begin{macro}{\rc@extract@title}
%    \begin{macrocode}
\long\def\rc@extract@title#1#2\@nil{%
  \rc@@extract@title#1%
}
%    \end{macrocode}
%    \end{macro}
%    \begin{macro}{\rc@@extract@title}
%    \begin{macrocode}
\long\def\rc@@extract@title#1\TR@TitleReference#2#3#4\@nil{%
  \ltx@zero
  #3%
}
%    \end{macrocode}
%    \end{macro}
%
% \subsection{Macros for checking undefined references}
%
%    \begin{macro}{\IfRefUndefinedExpandable}
%    \begin{macrocode}
\rc@newcommand*{\IfRefUndefinedExpandable}[1]{%
  \ltx@ifundefined{r@#1}\ltx@firstoftwo\ltx@secondoftwo
}
%    \end{macrocode}
%    \end{macro}
%    \begin{macro}{\IfRefUndefinedBabel}
%    \begin{macrocode}
\rc@RobustDefOne{}*\IfRefUndefinedBabel{%
  \begingroup
    \csname safe@actives@true\endcsname
  \expandafter\expandafter\expandafter\endgroup
  \expandafter\ifx\csname r@#1\endcsname\relax
    \expandafter\ltx@firstoftwo
  \else
    \expandafter\ltx@secondoftwo
  \fi
}
%    \end{macrocode}
%    \end{macro}
%    \begin{macrocode}
\rc@AtEnd%
%</package>
%    \end{macrocode}
%
% \section{Test}
%
% \subsection{Catcode checks for loading}
%
%    \begin{macrocode}
%<*test1>
%    \end{macrocode}
%    \begin{macrocode}
\catcode`\{=1 %
\catcode`\}=2 %
\catcode`\#=6 %
\catcode`\@=11 %
\expandafter\ifx\csname count@\endcsname\relax
  \countdef\count@=255 %
\fi
\expandafter\ifx\csname @gobble\endcsname\relax
  \long\def\@gobble#1{}%
\fi
\expandafter\ifx\csname @firstofone\endcsname\relax
  \long\def\@firstofone#1{#1}%
\fi
\expandafter\ifx\csname loop\endcsname\relax
  \expandafter\@firstofone
\else
  \expandafter\@gobble
\fi
{%
  \def\loop#1\repeat{%
    \def\body{#1}%
    \iterate
  }%
  \def\iterate{%
    \body
      \let\next\iterate
    \else
      \let\next\relax
    \fi
    \next
  }%
  \let\repeat=\fi
}%
\def\RestoreCatcodes{}
\count@=0 %
\loop
  \edef\RestoreCatcodes{%
    \RestoreCatcodes
    \catcode\the\count@=\the\catcode\count@\relax
  }%
\ifnum\count@<255 %
  \advance\count@ 1 %
\repeat

\def\RangeCatcodeInvalid#1#2{%
  \count@=#1\relax
  \loop
    \catcode\count@=15 %
  \ifnum\count@<#2\relax
    \advance\count@ 1 %
  \repeat
}
\def\RangeCatcodeCheck#1#2#3{%
  \count@=#1\relax
  \loop
    \ifnum#3=\catcode\count@
    \else
      \errmessage{%
        Character \the\count@\space
        with wrong catcode \the\catcode\count@\space
        instead of \number#3%
      }%
    \fi
  \ifnum\count@<#2\relax
    \advance\count@ 1 %
  \repeat
}
\def\space{ }
\expandafter\ifx\csname LoadCommand\endcsname\relax
  \def\LoadCommand{\input refcount.sty\relax}%
\fi
\def\Test{%
  \RangeCatcodeInvalid{0}{47}%
  \RangeCatcodeInvalid{58}{64}%
  \RangeCatcodeInvalid{91}{96}%
  \RangeCatcodeInvalid{123}{255}%
  \catcode`\@=12 %
  \catcode`\\=0 %
  \catcode`\%=14 %
  \LoadCommand
  \RangeCatcodeCheck{0}{36}{15}%
  \RangeCatcodeCheck{37}{37}{14}%
  \RangeCatcodeCheck{38}{47}{15}%
  \RangeCatcodeCheck{48}{57}{12}%
  \RangeCatcodeCheck{58}{63}{15}%
  \RangeCatcodeCheck{64}{64}{12}%
  \RangeCatcodeCheck{65}{90}{11}%
  \RangeCatcodeCheck{91}{91}{15}%
  \RangeCatcodeCheck{92}{92}{0}%
  \RangeCatcodeCheck{93}{96}{15}%
  \RangeCatcodeCheck{97}{122}{11}%
  \RangeCatcodeCheck{123}{255}{15}%
  \RestoreCatcodes
}
\Test
\csname @@end\endcsname
\end
%    \end{macrocode}
%    \begin{macrocode}
%</test1>
%    \end{macrocode}
%
% \subsection{Macro tests}
%
%    \begin{macrocode}
%<*test2>
\errorcontextlines=10000 %
\showboxbreadth=10000 %
\showboxdepth=10000 %
\begingroup\expandafter\expandafter\expandafter\endgroup
\expandafter\ifx\csname RequirePackage\endcsname\relax
  \input refcount.sty\relax
\else
  \RequirePackage{refcount}[2011/10/16]%
\fi
\catcode`\@=11 %
\begingroup\expandafter\expandafter\expandafter\endgroup
\expandafter\ifx\csname @onelevel@sanitize\endcsname\relax
  \begingroup\expandafter\expandafter\expandafter\endgroup
  \expandafter\ifx\csname detokenize\endcsname\relax
    \def\strip@prefix#1->{}%
    \def\@onelevel@sanitize#1{%
      \edef#1{%
        \expandafter\strip@prefix\meaning#1%
      }%
    }%
  \else
    \def\@onelevel@sanitize#1{%
      \edef#1{%
        \detokenize\expandafter{#1}%
      }%
    }%
  \fi
\fi
\def\msg#{\immediate\write16}
\def\empty{}
\def\space{ }
%    \end{macrocode}
%    \begin{macrocode}
\def\r@foo{{\empty 1}{\empty 2}}
\long\def\test#1#2{%
  \begingroup
    \setbox0=\hbox{%
      \def\TestTask{#1}%
      \@onelevel@sanitize\TestTask
      \msg{* \TestTask}%
      \expandafter\expandafter\expandafter\def
      \expandafter\expandafter\expandafter\TestResult
      \expandafter\expandafter\expandafter{%
        #1%
      }%
      \def\TestExpected{#2}%
      \ifx\TestResult\TestExpected
        \msg{ \space ok.}%
      \else
        \@onelevel@sanitize\TestResult
        \@onelevel@sanitize\TestExpected
        \msg{ \space Result: \space\space[\TestResult]}%
        \msg{ \space Expected: [\TestExpected]}%
        \errmessage{Test failed!}%
      \fi
    }%
    \ifdim\wd0=0pt %
    \else
      \showbox0 %
    \fi
  \endgroup
}
\test{\getrefnumber{foo}}{\empty 1}
\test{\getpagerefnumber{foo}}{\empty 2}
\test{\getrefbykeydefault{foo}{}{\empty default}}{\empty 1}
\test{\getrefbykeydefault{foo}{page}{\empty default}}{\empty 2}
\test{\getrefbykeydefault{foo}{name}{\empty default}}{\empty default}
\test{\getrefbykeydefault{foo}{anchor}{\empty default}}{\empty default}
\test{\getrefbykeydefault{foo}{url}{\empty default}}{\empty default}
\test{\getrefbykeydefault{foo}{title}{\empty default}}{\empty default}
\msg{}
\def\r@foo{{}{}{}{}{}{}{}{}{}{}}
\def\Test#1#2\\{%
  \test{#1{foo}#2}{}%
}
\def\TestGroup{%
  \Test\getrefnumber\\%
  \Test\getpagerefnumber\\%
  \Test\getrefbykeydefault{}{}\\%
  \Test\getrefbykeydefault{page}{}\\%
  \Test\getrefbykeydefault{anchor}{}\\%
  \Test\getrefbykeydefault{name}{}\\%
  \Test\getrefbykeydefault{url}{}\\%
}
\TestGroup
\Test\getrefbykeydefault{title}{}\\%
\msg{}
\def\r@foo{\par\par\par\par\par\par\par\par}
\long\def\Test#1#2\\{%
  \test{#1{foo}#2}{\par}%
}
\TestGroup
\test{\getrefbykeydefault{title}{}{}}{}
\msg{}
\def\r@foo{{ }{ }{ }{ }{ }}
\def\Test#1#2\\{%
  \test{#1{foo}#2}{ }%
}
\TestGroup
\msg{}
\long\def\TestDefault#1{%
  \begingroup
    \setrefcountdefault{#1}%
    \test{\getrefnumber{foo}}{#1}%
    \test{\getpagerefnumber{foo}}{#1}%
  \endgroup
}
\def\TestDefaultX{%
  \TestDefault{}%
  \TestDefault{\par}%
  \TestDefault{ }%
  \TestDefault{\space}%
}
\let\r@foo\@undefined
\TestDefaultX
\let\r@foo\relax
\TestDefaultX
\def\r@foo{}
\TestDefaultX
%    \end{macrocode}
%    \begin{macrocode}
\msg{}
\long\def\Test#1#2#3#4{%
  \begingroup
    \def\TestTask{#1}%
    \@onelevel@sanitize\TestTask
    \msg{* [\TestTask]}%
    \edef\TestResultA{\IfRefUndefinedExpandable{#1}{#2}{#3}}%
    \IfRefUndefinedBabel{#1}{%
      \def\TestResultB{#2}%
    }{%
      \def\TestResultB{#3}%
    }%
    \def\TestExpected{#4}%
    \ifx\TestResultA\TestExpected
      \msg{ \space ok.}%
    \else
      \begingroup
        \@onelevel@sanitize\TestResultA
        \@onelevel@sanitize\TestExpected
        \msg{ \space Result: \space\space[\TestResultA]}%
        \msg{ \space Expected: [\TestExpected]}%
        \errmessage{Test failed!}%
      \endgroup
    \fi
    \ifx\TestResultB\TestExpected
      \msg{ \space ok.}%
    \else
      \begingroup
        \@onelevel@sanitize\TestResultB
        \@onelevel@sanitize\TestExpected
        \msg{ \space Result: \space\space[\TestResultB]}%
        \msg{ \space Expected: [\TestExpected]}%
        \errmessage{Test failed!}%
      \endgroup
    \fi
  \endgroup
}
\begingroup
  \def\r@foo{{}{}}%
  \let\r@bar\@undefined
  \let\r@xyz\relax
  \Test{foo}{true}{false}{false}%
  \Test{bar}{true}{false}{true}%
  \Test{xyz}{true}{false}{true}%
\endgroup
%    \end{macrocode}
%    \begin{macrocode}
\csname @@end\endcsname\end
%</test2>
%    \end{macrocode}
% \subsection{Test with package \xpackage{titleref}}
%
%    \begin{macrocode}
%<*test3>
\NeedsTeXFormat{LaTeX2e}
\documentclass{article}
\usepackage{refcount}[2011/10/16]
%<test4>\usepackage{nameref}
\usepackage{titleref}
\begin{document}
\section{Hello World}
\label{sec:hello}
\section{\hbox{xy}}
\label{sec:foo}
%
\makeatletter
\@ifundefined{r@sec:hello}{%
  \typeout{==> Compile twice!}%
}{%
  \def\test#1#2{%
    \begingroup
      \def\TestTask{#1}%
      \@onelevel@sanitize\TestTask
      \typeout{* \TestTask}%
      \expandafter\expandafter\expandafter\def
      \expandafter\expandafter\expandafter\TestResult
      \expandafter\expandafter\expandafter{%
        #1%
      }%
      \def\TestExpected{#2}%
      \ifx\TestResult\TestExpected
        \typeout{ \space ok.}%
      \else
        \@onelevel@sanitize\TestResult
        \@onelevel@sanitize\TestExpected
        \typeout{ \space Result: \space\space[\TestResult]}%
        \typeout{ \space Expected: [\TestExpected]}%
        \errmessage{Test failed!}%
      \fi
    \endgroup
  }%
  \test{\getrefbykeydefault{sec:hello}{title}{}}{Hello World}%
  \test{\getrefbykeydefault{sec:foo}{title}{}}{\hbox{xy}}%
  \begingroup
    \def\hbox#1{[#1]}% hash-ok
    \test{\getrefbykeydefault{sec:foo}{title}{}}{\hbox{xy}}%
  \endgroup
}
\makeatother
%    \end{macrocode}
%    \begin{macrocode}
\end{document}
%</test3>
%    \end{macrocode}
%    \begin{macrocode}
%<*test5>
\NeedsTeXFormat{LaTeX2e}
\documentclass{book}
\usepackage{refcount}[2011/10/16]
\usepackage{zref-runs}
\newcounter{test}
\begin{document}
\ifnum\zruns>1 %
  \makeatletter
  \def\Test#1#2#3{%
    \begingroup
      \setcounter{test}{10}%
      \sbox0{%
        #1{test}{#2}%
        \ifnum#3=\value{test}%
        \else
          \PackageError{test}{\string#1{#2} <> #3 (\the\value{test})}%
        \fi
      }%
      \ifdim\wd0=0pt %
      \else
        \PackageError{test}{Non-empty box}\@ehc
      \fi
    \endgroup
  }%
  \makeatother
  \Test\setcounterpageref{ch:two}{1}%
  \Test\setcounterpageref{ch:three}{3}%
  \Test\setcounterpageref{ch:four}{5}%
  \Test\setcounterpageref{ch:five}{7}%
  \Test\setcounterpageref{ch:six}{9}%
  \Test\setcounterpageref{ch:seven}{13}%
  \Test\addtocounterpageref{ch:two}{11}%
  \Test\addtocounterpageref{ch:three}{13}%
  \Test\addtocounterpageref{ch:four}{15}%
  \Test\addtocounterpageref{ch:five}{17}%
  \Test\addtocounterpageref{ch:six}{19}%
  \Test\addtocounterpageref{ch:seven}{23}%
  \Test\setcounterref{ch:two}{1}%
  \Test\setcounterref{ch:three}{2}%
  \Test\setcounterref{ch:four}{11}%
  \Test\addtocounterref{ch:two}{11}%
  \Test\addtocounterref{ch:three}{12}%
  \Test\addtocounterref{ch:four}{21}%
\fi
\frontmatter
\chapter{Chapter one}\label{ch:one}
\cleardoublepage
\mainmatter
\chapter{Chapter two}\label{ch:two}
\cleardoublepage
\chapter{Chapter three}\label{ch:three}
\cleardoublepage
\setcounter{chapter}{10}
\chapter{Chapter four}\label{ch:four}
\cleardoublepage
\appendix
\chapter{Chapter five}\label{ch:five}
\cleardoublepage
\chapter{Chapter six}\label{ch:six}
\cleardoublepage
\null
\cleardoublepage
\chapter{Chapter seven}\label{ch:seven}
\end{document}
%</test5>
%    \end{macrocode}
%
% \section{Installation}
%
% \subsection{Download}
%
% \paragraph{Package.} This package is available on
% CTAN\footnote{\url{ftp://ftp.ctan.org/tex-archive/}}:
% \begin{description}
% \item[\CTAN{macros/latex/contrib/oberdiek/refcount.dtx}] The source file.
% \item[\CTAN{macros/latex/contrib/oberdiek/refcount.pdf}] Documentation.
% \end{description}
%
%
% \paragraph{Bundle.} All the packages of the bundle `oberdiek'
% are also available in a TDS compliant ZIP archive. There
% the packages are already unpacked and the documentation files
% are generated. The files and directories obey the TDS standard.
% \begin{description}
% \item[\CTAN{install/macros/latex/contrib/oberdiek.tds.zip}]
% \end{description}
% \emph{TDS} refers to the standard ``A Directory Structure
% for \TeX\ Files'' (\CTAN{tds/tds.pdf}). Directories
% with \xfile{texmf} in their name are usually organized this way.
%
% \subsection{Bundle installation}
%
% \paragraph{Unpacking.} Unpack the \xfile{oberdiek.tds.zip} in the
% TDS tree (also known as \xfile{texmf} tree) of your choice.
% Example (linux):
% \begin{quote}
%   |unzip oberdiek.tds.zip -d ~/texmf|
% \end{quote}
%
% \paragraph{Script installation.}
% Check the directory \xfile{TDS:scripts/oberdiek/} for
% scripts that need further installation steps.
% Package \xpackage{attachfile2} comes with the Perl script
% \xfile{pdfatfi.pl} that should be installed in such a way
% that it can be called as \texttt{pdfatfi}.
% Example (linux):
% \begin{quote}
%   |chmod +x scripts/oberdiek/pdfatfi.pl|\\
%   |cp scripts/oberdiek/pdfatfi.pl /usr/local/bin/|
% \end{quote}
%
% \subsection{Package installation}
%
% \paragraph{Unpacking.} The \xfile{.dtx} file is a self-extracting
% \docstrip\ archive. The files are extracted by running the
% \xfile{.dtx} through \plainTeX:
% \begin{quote}
%   \verb|tex refcount.dtx|
% \end{quote}
%
% \paragraph{TDS.} Now the different files must be moved into
% the different directories in your installation TDS tree
% (also known as \xfile{texmf} tree):
% \begin{quote}
% \def\t{^^A
% \begin{tabular}{@{}>{\ttfamily}l@{ $\rightarrow$ }>{\ttfamily}l@{}}
%   refcount.sty & tex/latex/oberdiek/refcount.sty\\
%   refcount.pdf & doc/latex/oberdiek/refcount.pdf\\
%   test/refcount-test1.tex & doc/latex/oberdiek/test/refcount-test1.tex\\
%   test/refcount-test2.tex & doc/latex/oberdiek/test/refcount-test2.tex\\
%   test/refcount-test3.tex & doc/latex/oberdiek/test/refcount-test3.tex\\
%   test/refcount-test4.tex & doc/latex/oberdiek/test/refcount-test4.tex\\
%   test/refcount-test5.tex & doc/latex/oberdiek/test/refcount-test5.tex\\
%   refcount.dtx & source/latex/oberdiek/refcount.dtx\\
% \end{tabular}^^A
% }^^A
% \sbox0{\t}^^A
% \ifdim\wd0>\linewidth
%   \begingroup
%     \advance\linewidth by\leftmargin
%     \advance\linewidth by\rightmargin
%   \edef\x{\endgroup
%     \def\noexpand\lw{\the\linewidth}^^A
%   }\x
%   \def\lwbox{^^A
%     \leavevmode
%     \hbox to \linewidth{^^A
%       \kern-\leftmargin\relax
%       \hss
%       \usebox0
%       \hss
%       \kern-\rightmargin\relax
%     }^^A
%   }^^A
%   \ifdim\wd0>\lw
%     \sbox0{\small\t}^^A
%     \ifdim\wd0>\linewidth
%       \ifdim\wd0>\lw
%         \sbox0{\footnotesize\t}^^A
%         \ifdim\wd0>\linewidth
%           \ifdim\wd0>\lw
%             \sbox0{\scriptsize\t}^^A
%             \ifdim\wd0>\linewidth
%               \ifdim\wd0>\lw
%                 \sbox0{\tiny\t}^^A
%                 \ifdim\wd0>\linewidth
%                   \lwbox
%                 \else
%                   \usebox0
%                 \fi
%               \else
%                 \lwbox
%               \fi
%             \else
%               \usebox0
%             \fi
%           \else
%             \lwbox
%           \fi
%         \else
%           \usebox0
%         \fi
%       \else
%         \lwbox
%       \fi
%     \else
%       \usebox0
%     \fi
%   \else
%     \lwbox
%   \fi
% \else
%   \usebox0
% \fi
% \end{quote}
% If you have a \xfile{docstrip.cfg} that configures and enables \docstrip's
% TDS installing feature, then some files can already be in the right
% place, see the documentation of \docstrip.
%
% \subsection{Refresh file name databases}
%
% If your \TeX~distribution
% (\teTeX, \mikTeX, \dots) relies on file name databases, you must refresh
% these. For example, \teTeX\ users run \verb|texhash| or
% \verb|mktexlsr|.
%
% \subsection{Some details for the interested}
%
% \paragraph{Attached source.}
%
% The PDF documentation on CTAN also includes the
% \xfile{.dtx} source file. It can be extracted by
% AcrobatReader 6 or higher. Another option is \textsf{pdftk},
% e.g. unpack the file into the current directory:
% \begin{quote}
%   \verb|pdftk refcount.pdf unpack_files output .|
% \end{quote}
%
% \paragraph{Unpacking with \LaTeX.}
% The \xfile{.dtx} chooses its action depending on the format:
% \begin{description}
% \item[\plainTeX:] Run \docstrip\ and extract the files.
% \item[\LaTeX:] Generate the documentation.
% \end{description}
% If you insist on using \LaTeX\ for \docstrip\ (really,
% \docstrip\ does not need \LaTeX), then inform the autodetect routine
% about your intention:
% \begin{quote}
%   \verb|latex \let\install=y\input{refcount.dtx}|
% \end{quote}
% Do not forget to quote the argument according to the demands
% of your shell.
%
% \paragraph{Generating the documentation.}
% You can use both the \xfile{.dtx} or the \xfile{.drv} to generate
% the documentation. The process can be configured by the
% configuration file \xfile{ltxdoc.cfg}. For instance, put this
% line into this file, if you want to have A4 as paper format:
% \begin{quote}
%   \verb|\PassOptionsToClass{a4paper}{article}|
% \end{quote}
% An example follows how to generate the
% documentation with pdf\LaTeX:
% \begin{quote}
%\begin{verbatim}
%pdflatex refcount.dtx
%makeindex -s gind.ist refcount.idx
%pdflatex refcount.dtx
%makeindex -s gind.ist refcount.idx
%pdflatex refcount.dtx
%\end{verbatim}
% \end{quote}
%
% \section{Catalogue}
%
% The following XML file can be used as source for the
% \href{http://mirror.ctan.org/help/Catalogue/catalogue.html}{\TeX\ Catalogue}.
% The elements \texttt{caption} and \texttt{description} are imported
% from the original XML file from the Catalogue.
% The name of the XML file in the Catalogue is \xfile{refcount.xml}.
%    \begin{macrocode}
%<*catalogue>
<?xml version='1.0' encoding='us-ascii'?>
<!DOCTYPE entry SYSTEM 'catalogue.dtd'>
<entry datestamp='$Date$' modifier='$Author$' id='refcount'>
  <name>refcount</name>
  <caption>Counter operations with label references.</caption>
  <authorref id='auth:oberdiek'/>
  <copyright owner='Heiko Oberdiek' year='1998,2000,2006,2008,2010,2011'/>
  <license type='lppl1.3'/>
  <version number='3.4'/>
  <description>
    Provides commands <tt>\setcounterref</tt> and
    <tt>\addtocounterref</tt> which use the section (or whatever)
    number from the reference as the value to put into the counter, as
    in:

    <pre>
    ...\label{sec:foo}
    ...
    \setcounterref{foonum}{sec:foo}
    </pre>
    Commands <tt>\setcounterpageref</tt> and
    <tt>\addtocounterpageref</tt> do the corresponding thing with the
    page reference of the label.
    <p/>
    No <tt>.ins</tt> file is distributed; process the
    <tt>.dtx</tt> with plain TeX to create one.
    <p/>
    The package is part of the <xref refid='oberdiek'>oberdiek</xref>
    bundle.
  </description>
  <documentation details='Package documentation'
      href='ctan:/macros/latex/contrib/oberdiek/refcount.pdf'/>
  <ctan file='true' path='/macros/latex/contrib/oberdiek/refcount.dtx'/>
  <miktex location='oberdiek'/>
  <texlive location='oberdiek'/>
  <install path='/macros/latex/contrib/oberdiek/oberdiek.tds.zip'/>
</entry>
%</catalogue>
%    \end{macrocode}
%
% \begin{History}
%   \begin{Version}{1998/04/08 v1.0}
%   \item
%     First public release, written as answer in the
%     newsgroup \xnewsgroup{comp.text.tex}:
%     \URL{``\link{Re: Adding a \cs{ref} to a counter?}''}^^A
%     {http://groups.google.com/group/comp.text.tex/msg/c3f2a135ef5ee528}
%   \end{Version}
%   \begin{Version}{2000/09/07 v2.0}
%   \item
%     Documentation added.
%   \item
%     LPPL 1.2
%   \item
%     Package rewritten, new commands added.
%   \end{Version}
%   \begin{Version}{2006/02/20 v3.0}
%   \item
%     Support for \xpackage{hyperref} and \xpackage{nameref} improved.
%   \item
%     Support for \xpackage{titleref} and \xpackage{babel}'s shorthands added.
%   \item
%     New: \cs{refused}, \cs{getrefbykeydefault}
%   \end{Version}
%   \begin{Version}{2008/08/11 v3.1}
%   \item
%     Code is not changed.
%   \item
%     URLs updated.
%   \end{Version}
%   \begin{Version}{2010/12/01 v3.2}
%   \item
%     \cs{IfRefUndefinedExpandable} and \cs{IfRefUndefinedBabel} added.
%   \item
%     \cs{getrefnumber}, \cs{getpagerefnumber}, \cs{getrefbykeydefault}
%     are expandable in exact two expansion steps.
%   \item
%     Non-expandable macros are made robust.
%   \item
%     Test files added.
%   \end{Version}
%   \begin{Version}{2011/06/22 v3.3}
%   \item
%     Bug fix: \cs{rc@refused} is undefined for \cs{setcounterpageref}
%     and similar macros. (Bug found by Marc van Dongen.)
%   \end{Version}
%   \begin{Version}{2011/10/16 v3.4}
%   \item
%     Bug fix: \cs{setcounterpageref} and \cs{addtocounterpageref} fixed.
%     (Bug found by Staz.)
%   \item
%     Macros \cs{(set|addto)counter(page|)ref} are made robust.
%   \end{Version}
% \end{History}
%
% \PrintIndex
%
% \Finale
\endinput

%        (quote the arguments according to the demands of your shell)
%
% Documentation:
%    (a) If refcount.drv is present:
%           latex refcount.drv
%    (b) Without refcount.drv:
%           latex refcount.dtx; ...
%    The class ltxdoc loads the configuration file ltxdoc.cfg
%    if available. Here you can specify further options, e.g.
%    use A4 as paper format:
%       \PassOptionsToClass{a4paper}{article}
%
%    Programm calls to get the documentation (example):
%       pdflatex refcount.dtx
%       makeindex -s gind.ist refcount.idx
%       pdflatex refcount.dtx
%       makeindex -s gind.ist refcount.idx
%       pdflatex refcount.dtx
%
% Installation:
%    TDS:tex/latex/oberdiek/refcount.sty
%    TDS:doc/latex/oberdiek/refcount.pdf
%    TDS:doc/latex/oberdiek/test/refcount-test1.tex
%    TDS:doc/latex/oberdiek/test/refcount-test2.tex
%    TDS:doc/latex/oberdiek/test/refcount-test3.tex
%    TDS:doc/latex/oberdiek/test/refcount-test4.tex
%    TDS:doc/latex/oberdiek/test/refcount-test5.tex
%    TDS:source/latex/oberdiek/refcount.dtx
%
%<*ignore>
\begingroup
  \catcode123=1 %
  \catcode125=2 %
  \def\x{LaTeX2e}%
\expandafter\endgroup
\ifcase 0\ifx\install y1\fi\expandafter
         \ifx\csname processbatchFile\endcsname\relax\else1\fi
         \ifx\fmtname\x\else 1\fi\relax
\else\csname fi\endcsname
%</ignore>
%<*install>
\input docstrip.tex
\Msg{************************************************************************}
\Msg{* Installation}
\Msg{* Package: refcount 2011/10/16 v3.4 Data extraction from label references (HO)}
\Msg{************************************************************************}

\keepsilent
\askforoverwritefalse

\let\MetaPrefix\relax
\preamble

This is a generated file.

Project: refcount
Version: 2011/10/16 v3.4

Copyright (C) 1998, 2000, 2006, 2008, 2010, 2011 by
   Heiko Oberdiek <heiko.oberdiek at googlemail.com>

This work may be distributed and/or modified under the
conditions of the LaTeX Project Public License, either
version 1.3c of this license or (at your option) any later
version. This version of this license is in
   http://www.latex-project.org/lppl/lppl-1-3c.txt
and the latest version of this license is in
   http://www.latex-project.org/lppl.txt
and version 1.3 or later is part of all distributions of
LaTeX version 2005/12/01 or later.

This work has the LPPL maintenance status "maintained".

This Current Maintainer of this work is Heiko Oberdiek.

This work consists of the main source file refcount.dtx
and the derived files
   refcount.sty, refcount.pdf, refcount.ins, refcount.drv,
   refcount-test1.tex, refcount-test2.tex, refcount-test3.tex,
   refcount-test4.tex, refcount-test5.tex.

\endpreamble
\let\MetaPrefix\DoubleperCent

\generate{%
  \file{refcount.ins}{\from{refcount.dtx}{install}}%
  \file{refcount.drv}{\from{refcount.dtx}{driver}}%
  \usedir{tex/latex/oberdiek}%
  \file{refcount.sty}{\from{refcount.dtx}{package}}%
  \usedir{doc/latex/oberdiek/test}%
  \file{refcount-test1.tex}{\from{refcount.dtx}{test1}}%
  \file{refcount-test2.tex}{\from{refcount.dtx}{test2}}%
  \file{refcount-test3.tex}{\from{refcount.dtx}{test3}}%
  \file{refcount-test4.tex}{\from{refcount.dtx}{test3,test4}}%
  \file{refcount-test5.tex}{\from{refcount.dtx}{test5}}%
  \nopreamble
  \nopostamble
  \usedir{source/latex/oberdiek/catalogue}%
  \file{refcount.xml}{\from{refcount.dtx}{catalogue}}%
}

\catcode32=13\relax% active space
\let =\space%
\Msg{************************************************************************}
\Msg{*}
\Msg{* To finish the installation you have to move the following}
\Msg{* file into a directory searched by TeX:}
\Msg{*}
\Msg{*     refcount.sty}
\Msg{*}
\Msg{* To produce the documentation run the file `refcount.drv'}
\Msg{* through LaTeX.}
\Msg{*}
\Msg{* Happy TeXing!}
\Msg{*}
\Msg{************************************************************************}

\endbatchfile
%</install>
%<*ignore>
\fi
%</ignore>
%<*driver>
\NeedsTeXFormat{LaTeX2e}
\ProvidesFile{refcount.drv}%
  [2011/10/16 v3.4 Data extraction from label references (HO)]%
\documentclass{ltxdoc}
\usepackage{holtxdoc}[2011/11/22]
\begin{document}
  \DocInput{refcount.dtx}%
\end{document}
%</driver>
% \fi
%
% \CheckSum{1185}
%
% \CharacterTable
%  {Upper-case    \A\B\C\D\E\F\G\H\I\J\K\L\M\N\O\P\Q\R\S\T\U\V\W\X\Y\Z
%   Lower-case    \a\b\c\d\e\f\g\h\i\j\k\l\m\n\o\p\q\r\s\t\u\v\w\x\y\z
%   Digits        \0\1\2\3\4\5\6\7\8\9
%   Exclamation   \!     Double quote  \"     Hash (number) \#
%   Dollar        \$     Percent       \%     Ampersand     \&
%   Acute accent  \'     Left paren    \(     Right paren   \)
%   Asterisk      \*     Plus          \+     Comma         \,
%   Minus         \-     Point         \.     Solidus       \/
%   Colon         \:     Semicolon     \;     Less than     \<
%   Equals        \=     Greater than  \>     Question mark \?
%   Commercial at \@     Left bracket  \[     Backslash     \\
%   Right bracket \]     Circumflex    \^     Underscore    \_
%   Grave accent  \`     Left brace    \{     Vertical bar  \|
%   Right brace   \}     Tilde         \~}
%
% \GetFileInfo{refcount.drv}
%
% \title{The \xpackage{refcount} package}
% \date{2011/10/16 v3.4}
% \author{Heiko Oberdiek\\\xemail{heiko.oberdiek at googlemail.com}}
%
% \maketitle
%
% \begin{abstract}
% References are not numbers, however they often store numerical
% data such as section or page numbers. \cs{ref} or \cs{pageref}
% cannot be used for counter assignments or calculations because
% they are not expandable, generate warnings, or can even be links.
% The package provides expandable macros to extract the data
% from references. Packages \xpackage{hyperref}, \xpackage{nameref},
% \xpackage{titleref}, and \xpackage{babel} are supported.
% \end{abstract}
%
% \tableofcontents
%
% \section{Usage}
%
% \subsection{Setting counters}
%
% The following commands are similar to \LaTeX's
% \cs{setcounter} and \cs{addtocounter},
% but they extract the number value from a reference:
% \begin{quote}
%   \cs{setcounterref}, \cs{addtocounterref}\\
%   \cs{setcounterpageref}, \cs{addtocounterpageref}
% \end{quote}
% They take two arguments:
% \begin{quote}
%    \cs{...counter...ref} |{|\meta{\LaTeX\ counter}|}|
%    |{|\meta{reference}|}|
% \end{quote}
% An undefined references produces the usual LaTeX warning
% and its value is assumed to be zero.
% Example:
% \begin{quote}
%\begin{verbatim}
%\newcounter{ctrA}
%\newcounter{ctrB}
%\refstepcounter{ctrA}\label{ref:A}
%\setcounterref{ctrB}{ref:A}
%\addtocounterpageref{ctrB}{ref:A}
%\end{verbatim}
% \end{quote}
%
% \subsection{Expandable commands}
%
% These commands that can be used in expandible contexts
% (inside calculations, \cs{edef}, \cs{csname}, \cs{write}, \dots):
% \begin{quote}
%   \cs{getrefnumber}, \cs{getpagerefnumber}
% \end{quote}
% They take one argument, the reference:
% \begin{quote}
%   \cs{get...refnumber} |{|\meta{reference}|}|
% \end{quote}
% The default for undefined references can be changed
% with macro \cs{setrefcountdefault}, for example this
% package calls:
% \begin{quote}
%   \cs{setrefcountdefault}|{0}|
% \end{quote}
%
% Since version 2.0 of this package there is a new
% command:
% \begin{quote}
%   \cs{getrefbykeydefault} |{|\meta{reference}|}|
%   |{|\meta{key}|}| |{|\meta{default}|}|
% \end{quote}
% This generalized version allows the extraction
% of further properties of a reference than the
% two standard ones. Thus the following properties
% are supported, if they are available:
% \begin{quote}
% \begin{tabular}{@{}l|l|l@{}}
%    Key & Description & Package\\
% \hline
%   \meta{empty} & same as \cs{ref} & \LaTeX\\
%   |page| & same as \cs{pageref} & \LaTeX\\
%   |title| & section and caption titles & \xpackage{titleref}\\
%   |name| & section and caption titles & \xpackage{nameref}\\
%   |anchor| & anchor name & \xpackage{hyperref}\\
%   |url| & url/file & \xpackage{hyperref}/\xpackage{xr}
% \end{tabular}
% \end{quote}
%
% Since version 3.2 the expandable macros described before
% in this section
% are expandable in exact two expansion steps.
%
% \subsection{Undefined references}
%
% Because warnings and assignments cannot be used in
% expandible contexts, undefined references do not
% produce a warning, their values are assumed to be zero.
% Example:
% \begin{quote}
%\begin{verbatim}
%\label{ref:here}% somewhere
%\refused{ref:here}% see below
%\ifodd\getpagerefnumber{ref:here}%
%  reference is on an odd page
%\else
%  reference is on an even page
%\fi
%\end{verbatim}
% \end{quote}
%
% In case of undefined references the user usually want's
% to be informed. Also \LaTeX\ prints a warning at
% the end of the \LaTeX\ run. To notify \LaTeX\ and
% get a normal warning, just use
% \begin{quote}
%   \cs{refused} |{|\meta{reference}|}|
% \end{quote}
% outside the expanding context. Example, see above.
%
% \subsubsection{Check for undefined references}
%
% In version 3.2 macros were added, that test, whether references
% are defined.
% \begin{declcs}{IfRefUndefinedExpandable} \M{refname} \M{then} \M{else}\\
%   \cs{IfRefUndefinedBabel} \M{refname} \M{then} \M{else}
% \end{declcs}
% If the reference is not available and therefore undefined, then
% argument \meta{then} is executed, otherwise argument \meta{else}
% is called. Macro \cs{IfRefUndefinedExpandable} is expandable,
% but \meta{refname} must not contain babel shorthand characters.
% Macro \cs{IfRefUndefinedBabel} supports shorthand characters of
% babel, but it is not expandable.
%
% \subsection{Notes}
%
% \begin{itemize}
% \item
%   The method of extracting the number in this
%   package also works in cases, where the
%   reference cannot be used directly, because
%   a package such as \xpackage{hyperref} has added
%   extra stuff (hyper link), so that the reference cannot
%   be used as number any more.
% \item
%   If the reference does not contain a number,
%   assignments to a counter will fail of course.
% \end{itemize}
%
%
% \StopEventually{
% }
%
% \section{Implementation}
%
%    \begin{macrocode}
%<*package>
%    \end{macrocode}
%    Reload check, especially if the package is not used with \LaTeX.
%    \begin{macrocode}
\begingroup\catcode61\catcode48\catcode32=10\relax%
  \catcode13=5 % ^^M
  \endlinechar=13 %
  \catcode35=6 % #
  \catcode39=12 % '
  \catcode44=12 % ,
  \catcode45=12 % -
  \catcode46=12 % .
  \catcode58=12 % :
  \catcode64=11 % @
  \catcode123=1 % {
  \catcode125=2 % }
  \expandafter\let\expandafter\x\csname ver@refcount.sty\endcsname
  \ifx\x\relax % plain-TeX, first loading
  \else
    \def\empty{}%
    \ifx\x\empty % LaTeX, first loading,
      % variable is initialized, but \ProvidesPackage not yet seen
    \else
      \expandafter\ifx\csname PackageInfo\endcsname\relax
        \def\x#1#2{%
          \immediate\write-1{Package #1 Info: #2.}%
        }%
      \else
        \def\x#1#2{\PackageInfo{#1}{#2, stopped}}%
      \fi
      \x{refcount}{The package is already loaded}%
      \aftergroup\endinput
    \fi
  \fi
\endgroup%
%    \end{macrocode}
%    Package identification:
%    \begin{macrocode}
\begingroup\catcode61\catcode48\catcode32=10\relax%
  \catcode13=5 % ^^M
  \endlinechar=13 %
  \catcode35=6 % #
  \catcode39=12 % '
  \catcode40=12 % (
  \catcode41=12 % )
  \catcode44=12 % ,
  \catcode45=12 % -
  \catcode46=12 % .
  \catcode47=12 % /
  \catcode58=12 % :
  \catcode64=11 % @
  \catcode91=12 % [
  \catcode93=12 % ]
  \catcode123=1 % {
  \catcode125=2 % }
  \expandafter\ifx\csname ProvidesPackage\endcsname\relax
    \def\x#1#2#3[#4]{\endgroup
      \immediate\write-1{Package: #3 #4}%
      \xdef#1{#4}%
    }%
  \else
    \def\x#1#2[#3]{\endgroup
      #2[{#3}]%
      \ifx#1\@undefined
        \xdef#1{#3}%
      \fi
      \ifx#1\relax
        \xdef#1{#3}%
      \fi
    }%
  \fi
\expandafter\x\csname ver@refcount.sty\endcsname
\ProvidesPackage{refcount}%
  [2011/10/16 v3.4 Data extraction from label references (HO)]%
%    \end{macrocode}
%
%    \begin{macrocode}
\begingroup\catcode61\catcode48\catcode32=10\relax%
  \catcode13=5 % ^^M
  \endlinechar=13 %
  \catcode123=1 % {
  \catcode125=2 % }
  \catcode64=11 % @
  \def\x{\endgroup
    \expandafter\edef\csname rc@AtEnd\endcsname{%
      \endlinechar=\the\endlinechar\relax
      \catcode13=\the\catcode13\relax
      \catcode32=\the\catcode32\relax
      \catcode35=\the\catcode35\relax
      \catcode61=\the\catcode61\relax
      \catcode64=\the\catcode64\relax
      \catcode123=\the\catcode123\relax
      \catcode125=\the\catcode125\relax
    }%
  }%
\x\catcode61\catcode48\catcode32=10\relax%
\catcode13=5 % ^^M
\endlinechar=13 %
\catcode35=6 % #
\catcode64=11 % @
\catcode123=1 % {
\catcode125=2 % }
\def\TMP@EnsureCode#1#2{%
  \edef\rc@AtEnd{%
    \rc@AtEnd
    \catcode#1=\the\catcode#1\relax
  }%
  \catcode#1=#2\relax
}
\TMP@EnsureCode{33}{12}% !
\TMP@EnsureCode{39}{12}% '
\TMP@EnsureCode{42}{12}% *
\TMP@EnsureCode{45}{12}% -
\TMP@EnsureCode{46}{12}% .
\TMP@EnsureCode{47}{12}% /
\TMP@EnsureCode{91}{12}% [
\TMP@EnsureCode{93}{12}% ]
\TMP@EnsureCode{96}{12}% `
\edef\rc@AtEnd{\rc@AtEnd\noexpand\endinput}
%    \end{macrocode}
%
% \subsection{Loading packages}
%
%    \begin{macrocode}
\begingroup\expandafter\expandafter\expandafter\endgroup
\expandafter\ifx\csname RequirePackage\endcsname\relax
  \input ltxcmds.sty\relax
  \input infwarerr.sty\relax
\else
  \RequirePackage{ltxcmds}[2011/11/09]%
  \RequirePackage{infwarerr}[2010/04/08]%
\fi
%    \end{macrocode}
%
% \subsection{Defining commands}
%
%    \begin{macro}{\rc@IfDefinable}
%    \begin{macrocode}
\ltx@IfUndefined{@ifdefinable}{%
  \def\rc@IfDefinable#1{%
    \ifx#1\ltx@undefined
      \expandafter\ltx@firstofone
    \else
      \ifx#1\relax
        \expandafter\expandafter\expandafter\ltx@firstofone
      \else
        \@PackageError{refcount}{%
          Command \string#1 is already defined.\MessageBreak
          It will not redefined by this package%
        }\@ehc
        \expandafter\expandafter\expandafter\ltx@gobble
      \fi
    \fi
  }%
}{%
  \let\rc@IfDefinable\@ifdefinable
}
%    \end{macrocode}
%    \end{macro}
%
%    \begin{macro}{\rc@RobustDefOne}
%    \begin{macro}{\rc@RobustDefZero}
%    \begin{macrocode}
\ltx@IfUndefined{protected}{%
  \ltx@IfUndefined{DeclareRobustCommand}{%
    \def\rc@RobustDefOne#1#2#3#4{%
      \rc@IfDefinable#3{%
        #1\def#3##1{#4}%
      }%
    }%
    \def\rc@RobustDefZero#1#2{%
      \rc@IfDefinable#1{%
        \def#1{#2}%
      }%
    }%
  }{%
    \def\rc@RobustDefOne#1#2#3#4{%
      \rc@IfDefinable#3{%
        \DeclareRobustCommand#2#3[1]{#4}%
      }%
    }%
    \def\rc@RobustDefZero#1#2{%
      \rc@IfDefinable#1{%
        \DeclareRobustCommand#1{#2}%
      }%
    }%
  }%
}{%
  \def\rc@RobustDefOne#1#2#3#4{%
    \rc@IfDefinable#3{%
      \protected#1\def#3##1{#4}%
    }%
  }%
  \def\rc@RobustDefZero#1#2{%
    \rc@IfDefinable#1{%
      \protected\def#1{#2}%
    }%
  }%
}
%    \end{macrocode}
%    \end{macro}
%    \end{macro}
%
%    \begin{macro}{\rc@newcommand}
%    \begin{macrocode}
\ltx@IfUndefined{newcommand}{%
  \def\rc@newcommand*#1[#2]#3{% hash-ok
    \rc@IfDefinable#1{%
      \ifcase#2 %
        \def#1{#3}%
      \or
        \def#1##1{#3}%
      \or
        \def#1##1##2{#3}%
      \else
        \rc@InternalError
      \fi
    }%
  }%
}{%
  \let\rc@newcommand\newcommand
}
%    \end{macrocode}
%    \end{macro}
%
% \subsection{\cs{setrefcountdefault}}
%
%    \begin{macro}{\setrefcountdefault}
%    \begin{macrocode}
\rc@RobustDefOne\long{}\setrefcountdefault{%
  \def\rc@default{#1}%
}
%    \end{macrocode}
%    \end{macro}
%    \begin{macrocode}
\setrefcountdefault{0}
%    \end{macrocode}
%
% \subsection{\cs{refused}}
%
%    \begin{macro}{\refused}
%    \begin{macrocode}
\ltx@IfUndefined{G@refundefinedtrue}{%
  \rc@RobustDefOne{}{*}\refused{%
    \begingroup
      \csname @safe@activestrue\endcsname
      \ltx@IfUndefined{r@#1}{%
        \protect\G@refundefinedtrue
        \rc@WarningUndefined{#1}%
      }{}%
    \endgroup
  }%
}{%
  \rc@RobustDefOne{}{*}\refused{%
    \begingroup
      \csname @safe@activestrue\endcsname
      \ltx@IfUndefined{r@#1}{%
        \csname protect\expandafter\endcsname
        \csname G@refundefinedtrue\endcsname
        \rc@WarningUndefined{#1}%
      }{}%
    \endgroup
  }%
}
%    \end{macrocode}
%    \end{macro}
%    \begin{macro}{\rc@WarningUndefined}
%    \begin{macrocode}
\ltx@IfUndefined{@latex@warning}{%
  \def\rc@WarningUndefined#1{%
    \ltx@ifundefined{thepage}{%
      \def\thepage{\number\count0 }%
    }{}%
    \@PackageWarning{refcount}{%
      Reference `#1' on page \thepage\space undefined%
    }%
  }%
}{%
  \def\rc@WarningUndefined#1{%
    \@latex@warning{%
      Reference `#1' on page \thepage\space undefined%
    }%
  }%
}
%    \end{macrocode}
%    \end{macro}
%
% \subsection{Setting counters by reference data}
%
% \subsubsection{Generic setting}
%
%    \begin{macro}{\rc@set}
% Generic command for
% |\|$\{$|set|$,$|addto|$\}$|counter|$\{$|page|$,\}$|ref|:
%\begin{quote}
%\begin{tabular}[t]{@{}l@{: }l@{}}
% |#1|& \cs{setcounter}, \cs{addtocounter}\\
% |#2|& \cs{ltx@car} (for \cs{ref}), \cs{ltx@cartwo} (for \cs{pageref})\\
% |#3|& \hologo{LaTeX} counter\\
% |#4|& reference\\
%\end{tabular}
%\end{quote}
%    \begin{macrocode}
\def\rc@set#1#2#3#4{%
  \begingroup
    \csname @safe@activestrue\endcsname
    \refused{#4}%
    \expandafter\rc@@set\csname r@#4\endcsname{#1}{#2}{#3}%
  \endgroup
}
%    \end{macrocode}
%    \end{macro}
%    \begin{macro}{\rc@@set}
%\begin{quote}
%\begin{tabular}[t]{@{}l@{: }l@{}}
% |#1|& \cs{r@<...>}\\
% |#2|& \cs{setcounter}, \cs{addtocounter}\\
% |#3|& \cs{ltx@car} (for \cs{ref}), \cs{ltx@carsecond} (for \cs{pageref})\\
% |#4|& \hologo{LaTeX} counter\\
%\end{tabular}
%\end{quote}
%    \begin{macrocode}
\def\rc@@set#1#2#3#4{%
  \ifx#1\relax
    #2{#4}{\rc@default}%
  \else
    #2{#4}{%
      \expandafter#3#1\rc@default\rc@default\@nil
    }%
  \fi
}
%    \end{macrocode}
%    \end{macro}
%
% \subsubsection{User commands}
%
%    \begin{macro}{\setcounterref}
%    \begin{macrocode}
\rc@RobustDefZero\setcounterref{%
  \rc@set\setcounter\ltx@car
}
%    \end{macrocode}
%    \end{macro}
%    \begin{macro}{\addtocounterref}
%    \begin{macrocode}
\rc@RobustDefZero\addtocounterref{%
  \rc@set\addtocounter\ltx@car
}
%    \end{macrocode}
%    \end{macro}
%    \begin{macro}{\setcounterpageref}
%    \begin{macrocode}
\rc@RobustDefZero\setcounterpageref{%
  \rc@set\setcounter\ltx@carsecond
}
%    \end{macrocode}
%    \end{macro}
%    \begin{macro}{\addtocounterpageref}
%    \begin{macrocode}
\rc@RobustDefZero\addtocounterpageref{%
  \rc@set\addtocounter\ltx@carsecond
}
%    \end{macrocode}
%    \end{macro}
%
% \subsection{Extracting references}
%
%    \begin{macro}{\getrefnumber}
%    \begin{macrocode}
\rc@newcommand*{\getrefnumber}[1]{%
  \romannumeral
  \ltx@ifundefined{r@#1}{%
    \expandafter\ltx@zero
    \rc@default
  }{%
    \expandafter\expandafter\expandafter\rc@extract@
    \expandafter\expandafter\expandafter!%
    \csname r@#1\expandafter\endcsname
    \expandafter{\rc@default}\@nil
  }%
}
%    \end{macrocode}
%    \end{macro}
%    \begin{macro}{\getpagerefnumber}
%    \begin{macrocode}
\rc@newcommand*{\getpagerefnumber}[1]{%
  \romannumeral
  \ltx@ifundefined{r@#1}{%
    \expandafter\ltx@zero
    \rc@default
  }{%
    \expandafter\expandafter\expandafter\rc@extract@page
    \expandafter\expandafter\expandafter!%
    \csname r@#1\expandafter\expandafter\expandafter\endcsname
    \expandafter\expandafter\expandafter{%
      \expandafter\rc@default
    \expandafter}\expandafter{\rc@default}\@nil
  }%
}
%    \end{macrocode}
%    \end{macro}
%    \begin{macro}{\getrefbykeydefault}
%    \begin{macrocode}
\rc@newcommand*{\getrefbykeydefault}[2]{%
  \romannumeral
  \expandafter\rc@getrefbykeydefault
    \csname r@#1\expandafter\endcsname
    \csname rc@extract@#2\endcsname
}
%    \end{macrocode}
%    \end{macro}
%    \begin{macro}{\rc@getrefbykeydefault}
%\begin{quote}
%\begin{tabular}[t]{@{}l@{: }l@{}}
% |#1|& \cs{r@<...>}\\
% |#2|& \cs{rc@extract@<...>}\\
% |#3|& default\\
%\end{tabular}
%\end{quote}
%    \begin{macrocode}
\long\def\rc@getrefbykeydefault#1#2#3{%
  \ifx#1\relax
    % reference is undefined
    \ltx@ReturnAfterElseFi{%
      \ltx@zero
      #3%
    }%
  \else
    \ltx@ReturnAfterFi{%
      \ifx#2\relax
        % extract method is missing
        \ltx@ReturnAfterElseFi{%
          \ltx@zero
          #3%
        }%
      \else
        \ltx@ReturnAfterFi{%
          \expandafter
          \rc@generic#1{#3}{#3}{#3}{#3}{#3}\@nil#2{#3}%
        }%
      \fi
    }%
  \fi
}
%    \end{macrocode}
%    \end{macro}
%    \begin{macro}{\rc@generic}
%\begin{quote}
%\begin{tabular}[t]{@{}l@{: }l@{}}
% |#1|& first item in \cs{r@<...>}\\
% |#2|& remaining items in \cs{r@<...>}\\
% |#3|& \cs{rc@extract@<...>}\\
% |#4|& default\\
%\end{tabular}
%\end{quote}
%    \begin{macrocode}
\long\def\rc@generic#1#2\@nil#3#4{%
  #3{#1\TR@TitleReference\@empty{#4}\@nil}{#1}#2\@nil
}
%    \end{macrocode}
%    \end{macro}
%    \begin{macro}{\rc@extract@}
%    \begin{macrocode}
\long\def\rc@extract@#1#2#3\@nil{%
  \ltx@zero
  #2%
}
%    \end{macrocode}
%    \end{macro}
%    \begin{macro}{\rc@extract@page}
%    \begin{macrocode}
\long\def\rc@extract@page#1#2#3#4\@nil{%
  \ltx@zero
  #3%
}
%    \end{macrocode}
%    \end{macro}
%    \begin{macro}{\rc@extract@name}
%    \begin{macrocode}
\long\def\rc@extract@name#1#2#3#4#5\@nil{%
  \ltx@zero
  #4%
}
%    \end{macrocode}
%    \end{macro}
%    \begin{macro}{\rc@extract@anchor}
%    \begin{macrocode}
\long\def\rc@extract@anchor#1#2#3#4#5#6\@nil{%
  \ltx@zero
  #5%
}
%    \end{macrocode}
%    \end{macro}
%    \begin{macro}{\rc@extract@url}
%    \begin{macrocode}
\long\def\rc@extract@url#1#2#3#4#5#6#7\@nil{%
  \ltx@zero
  #6%
}
%    \end{macrocode}
%    \end{macro}
%    \begin{macro}{\rc@extract@title}
%    \begin{macrocode}
\long\def\rc@extract@title#1#2\@nil{%
  \rc@@extract@title#1%
}
%    \end{macrocode}
%    \end{macro}
%    \begin{macro}{\rc@@extract@title}
%    \begin{macrocode}
\long\def\rc@@extract@title#1\TR@TitleReference#2#3#4\@nil{%
  \ltx@zero
  #3%
}
%    \end{macrocode}
%    \end{macro}
%
% \subsection{Macros for checking undefined references}
%
%    \begin{macro}{\IfRefUndefinedExpandable}
%    \begin{macrocode}
\rc@newcommand*{\IfRefUndefinedExpandable}[1]{%
  \ltx@ifundefined{r@#1}\ltx@firstoftwo\ltx@secondoftwo
}
%    \end{macrocode}
%    \end{macro}
%    \begin{macro}{\IfRefUndefinedBabel}
%    \begin{macrocode}
\rc@RobustDefOne{}*\IfRefUndefinedBabel{%
  \begingroup
    \csname safe@actives@true\endcsname
  \expandafter\expandafter\expandafter\endgroup
  \expandafter\ifx\csname r@#1\endcsname\relax
    \expandafter\ltx@firstoftwo
  \else
    \expandafter\ltx@secondoftwo
  \fi
}
%    \end{macrocode}
%    \end{macro}
%    \begin{macrocode}
\rc@AtEnd%
%</package>
%    \end{macrocode}
%
% \section{Test}
%
% \subsection{Catcode checks for loading}
%
%    \begin{macrocode}
%<*test1>
%    \end{macrocode}
%    \begin{macrocode}
\catcode`\{=1 %
\catcode`\}=2 %
\catcode`\#=6 %
\catcode`\@=11 %
\expandafter\ifx\csname count@\endcsname\relax
  \countdef\count@=255 %
\fi
\expandafter\ifx\csname @gobble\endcsname\relax
  \long\def\@gobble#1{}%
\fi
\expandafter\ifx\csname @firstofone\endcsname\relax
  \long\def\@firstofone#1{#1}%
\fi
\expandafter\ifx\csname loop\endcsname\relax
  \expandafter\@firstofone
\else
  \expandafter\@gobble
\fi
{%
  \def\loop#1\repeat{%
    \def\body{#1}%
    \iterate
  }%
  \def\iterate{%
    \body
      \let\next\iterate
    \else
      \let\next\relax
    \fi
    \next
  }%
  \let\repeat=\fi
}%
\def\RestoreCatcodes{}
\count@=0 %
\loop
  \edef\RestoreCatcodes{%
    \RestoreCatcodes
    \catcode\the\count@=\the\catcode\count@\relax
  }%
\ifnum\count@<255 %
  \advance\count@ 1 %
\repeat

\def\RangeCatcodeInvalid#1#2{%
  \count@=#1\relax
  \loop
    \catcode\count@=15 %
  \ifnum\count@<#2\relax
    \advance\count@ 1 %
  \repeat
}
\def\RangeCatcodeCheck#1#2#3{%
  \count@=#1\relax
  \loop
    \ifnum#3=\catcode\count@
    \else
      \errmessage{%
        Character \the\count@\space
        with wrong catcode \the\catcode\count@\space
        instead of \number#3%
      }%
    \fi
  \ifnum\count@<#2\relax
    \advance\count@ 1 %
  \repeat
}
\def\space{ }
\expandafter\ifx\csname LoadCommand\endcsname\relax
  \def\LoadCommand{\input refcount.sty\relax}%
\fi
\def\Test{%
  \RangeCatcodeInvalid{0}{47}%
  \RangeCatcodeInvalid{58}{64}%
  \RangeCatcodeInvalid{91}{96}%
  \RangeCatcodeInvalid{123}{255}%
  \catcode`\@=12 %
  \catcode`\\=0 %
  \catcode`\%=14 %
  \LoadCommand
  \RangeCatcodeCheck{0}{36}{15}%
  \RangeCatcodeCheck{37}{37}{14}%
  \RangeCatcodeCheck{38}{47}{15}%
  \RangeCatcodeCheck{48}{57}{12}%
  \RangeCatcodeCheck{58}{63}{15}%
  \RangeCatcodeCheck{64}{64}{12}%
  \RangeCatcodeCheck{65}{90}{11}%
  \RangeCatcodeCheck{91}{91}{15}%
  \RangeCatcodeCheck{92}{92}{0}%
  \RangeCatcodeCheck{93}{96}{15}%
  \RangeCatcodeCheck{97}{122}{11}%
  \RangeCatcodeCheck{123}{255}{15}%
  \RestoreCatcodes
}
\Test
\csname @@end\endcsname
\end
%    \end{macrocode}
%    \begin{macrocode}
%</test1>
%    \end{macrocode}
%
% \subsection{Macro tests}
%
%    \begin{macrocode}
%<*test2>
\errorcontextlines=10000 %
\showboxbreadth=10000 %
\showboxdepth=10000 %
\begingroup\expandafter\expandafter\expandafter\endgroup
\expandafter\ifx\csname RequirePackage\endcsname\relax
  \input refcount.sty\relax
\else
  \RequirePackage{refcount}[2011/10/16]%
\fi
\catcode`\@=11 %
\begingroup\expandafter\expandafter\expandafter\endgroup
\expandafter\ifx\csname @onelevel@sanitize\endcsname\relax
  \begingroup\expandafter\expandafter\expandafter\endgroup
  \expandafter\ifx\csname detokenize\endcsname\relax
    \def\strip@prefix#1->{}%
    \def\@onelevel@sanitize#1{%
      \edef#1{%
        \expandafter\strip@prefix\meaning#1%
      }%
    }%
  \else
    \def\@onelevel@sanitize#1{%
      \edef#1{%
        \detokenize\expandafter{#1}%
      }%
    }%
  \fi
\fi
\def\msg#{\immediate\write16}
\def\empty{}
\def\space{ }
%    \end{macrocode}
%    \begin{macrocode}
\def\r@foo{{\empty 1}{\empty 2}}
\long\def\test#1#2{%
  \begingroup
    \setbox0=\hbox{%
      \def\TestTask{#1}%
      \@onelevel@sanitize\TestTask
      \msg{* \TestTask}%
      \expandafter\expandafter\expandafter\def
      \expandafter\expandafter\expandafter\TestResult
      \expandafter\expandafter\expandafter{%
        #1%
      }%
      \def\TestExpected{#2}%
      \ifx\TestResult\TestExpected
        \msg{ \space ok.}%
      \else
        \@onelevel@sanitize\TestResult
        \@onelevel@sanitize\TestExpected
        \msg{ \space Result: \space\space[\TestResult]}%
        \msg{ \space Expected: [\TestExpected]}%
        \errmessage{Test failed!}%
      \fi
    }%
    \ifdim\wd0=0pt %
    \else
      \showbox0 %
    \fi
  \endgroup
}
\test{\getrefnumber{foo}}{\empty 1}
\test{\getpagerefnumber{foo}}{\empty 2}
\test{\getrefbykeydefault{foo}{}{\empty default}}{\empty 1}
\test{\getrefbykeydefault{foo}{page}{\empty default}}{\empty 2}
\test{\getrefbykeydefault{foo}{name}{\empty default}}{\empty default}
\test{\getrefbykeydefault{foo}{anchor}{\empty default}}{\empty default}
\test{\getrefbykeydefault{foo}{url}{\empty default}}{\empty default}
\test{\getrefbykeydefault{foo}{title}{\empty default}}{\empty default}
\msg{}
\def\r@foo{{}{}{}{}{}{}{}{}{}{}}
\def\Test#1#2\\{%
  \test{#1{foo}#2}{}%
}
\def\TestGroup{%
  \Test\getrefnumber\\%
  \Test\getpagerefnumber\\%
  \Test\getrefbykeydefault{}{}\\%
  \Test\getrefbykeydefault{page}{}\\%
  \Test\getrefbykeydefault{anchor}{}\\%
  \Test\getrefbykeydefault{name}{}\\%
  \Test\getrefbykeydefault{url}{}\\%
}
\TestGroup
\Test\getrefbykeydefault{title}{}\\%
\msg{}
\def\r@foo{\par\par\par\par\par\par\par\par}
\long\def\Test#1#2\\{%
  \test{#1{foo}#2}{\par}%
}
\TestGroup
\test{\getrefbykeydefault{title}{}{}}{}
\msg{}
\def\r@foo{{ }{ }{ }{ }{ }}
\def\Test#1#2\\{%
  \test{#1{foo}#2}{ }%
}
\TestGroup
\msg{}
\long\def\TestDefault#1{%
  \begingroup
    \setrefcountdefault{#1}%
    \test{\getrefnumber{foo}}{#1}%
    \test{\getpagerefnumber{foo}}{#1}%
  \endgroup
}
\def\TestDefaultX{%
  \TestDefault{}%
  \TestDefault{\par}%
  \TestDefault{ }%
  \TestDefault{\space}%
}
\let\r@foo\@undefined
\TestDefaultX
\let\r@foo\relax
\TestDefaultX
\def\r@foo{}
\TestDefaultX
%    \end{macrocode}
%    \begin{macrocode}
\msg{}
\long\def\Test#1#2#3#4{%
  \begingroup
    \def\TestTask{#1}%
    \@onelevel@sanitize\TestTask
    \msg{* [\TestTask]}%
    \edef\TestResultA{\IfRefUndefinedExpandable{#1}{#2}{#3}}%
    \IfRefUndefinedBabel{#1}{%
      \def\TestResultB{#2}%
    }{%
      \def\TestResultB{#3}%
    }%
    \def\TestExpected{#4}%
    \ifx\TestResultA\TestExpected
      \msg{ \space ok.}%
    \else
      \begingroup
        \@onelevel@sanitize\TestResultA
        \@onelevel@sanitize\TestExpected
        \msg{ \space Result: \space\space[\TestResultA]}%
        \msg{ \space Expected: [\TestExpected]}%
        \errmessage{Test failed!}%
      \endgroup
    \fi
    \ifx\TestResultB\TestExpected
      \msg{ \space ok.}%
    \else
      \begingroup
        \@onelevel@sanitize\TestResultB
        \@onelevel@sanitize\TestExpected
        \msg{ \space Result: \space\space[\TestResultB]}%
        \msg{ \space Expected: [\TestExpected]}%
        \errmessage{Test failed!}%
      \endgroup
    \fi
  \endgroup
}
\begingroup
  \def\r@foo{{}{}}%
  \let\r@bar\@undefined
  \let\r@xyz\relax
  \Test{foo}{true}{false}{false}%
  \Test{bar}{true}{false}{true}%
  \Test{xyz}{true}{false}{true}%
\endgroup
%    \end{macrocode}
%    \begin{macrocode}
\csname @@end\endcsname\end
%</test2>
%    \end{macrocode}
% \subsection{Test with package \xpackage{titleref}}
%
%    \begin{macrocode}
%<*test3>
\NeedsTeXFormat{LaTeX2e}
\documentclass{article}
\usepackage{refcount}[2011/10/16]
%<test4>\usepackage{nameref}
\usepackage{titleref}
\begin{document}
\section{Hello World}
\label{sec:hello}
\section{\hbox{xy}}
\label{sec:foo}
%
\makeatletter
\@ifundefined{r@sec:hello}{%
  \typeout{==> Compile twice!}%
}{%
  \def\test#1#2{%
    \begingroup
      \def\TestTask{#1}%
      \@onelevel@sanitize\TestTask
      \typeout{* \TestTask}%
      \expandafter\expandafter\expandafter\def
      \expandafter\expandafter\expandafter\TestResult
      \expandafter\expandafter\expandafter{%
        #1%
      }%
      \def\TestExpected{#2}%
      \ifx\TestResult\TestExpected
        \typeout{ \space ok.}%
      \else
        \@onelevel@sanitize\TestResult
        \@onelevel@sanitize\TestExpected
        \typeout{ \space Result: \space\space[\TestResult]}%
        \typeout{ \space Expected: [\TestExpected]}%
        \errmessage{Test failed!}%
      \fi
    \endgroup
  }%
  \test{\getrefbykeydefault{sec:hello}{title}{}}{Hello World}%
  \test{\getrefbykeydefault{sec:foo}{title}{}}{\hbox{xy}}%
  \begingroup
    \def\hbox#1{[#1]}% hash-ok
    \test{\getrefbykeydefault{sec:foo}{title}{}}{\hbox{xy}}%
  \endgroup
}
\makeatother
%    \end{macrocode}
%    \begin{macrocode}
\end{document}
%</test3>
%    \end{macrocode}
%    \begin{macrocode}
%<*test5>
\NeedsTeXFormat{LaTeX2e}
\documentclass{book}
\usepackage{refcount}[2011/10/16]
\usepackage{zref-runs}
\newcounter{test}
\begin{document}
\ifnum\zruns>1 %
  \makeatletter
  \def\Test#1#2#3{%
    \begingroup
      \setcounter{test}{10}%
      \sbox0{%
        #1{test}{#2}%
        \ifnum#3=\value{test}%
        \else
          \PackageError{test}{\string#1{#2} <> #3 (\the\value{test})}%
        \fi
      }%
      \ifdim\wd0=0pt %
      \else
        \PackageError{test}{Non-empty box}\@ehc
      \fi
    \endgroup
  }%
  \makeatother
  \Test\setcounterpageref{ch:two}{1}%
  \Test\setcounterpageref{ch:three}{3}%
  \Test\setcounterpageref{ch:four}{5}%
  \Test\setcounterpageref{ch:five}{7}%
  \Test\setcounterpageref{ch:six}{9}%
  \Test\setcounterpageref{ch:seven}{13}%
  \Test\addtocounterpageref{ch:two}{11}%
  \Test\addtocounterpageref{ch:three}{13}%
  \Test\addtocounterpageref{ch:four}{15}%
  \Test\addtocounterpageref{ch:five}{17}%
  \Test\addtocounterpageref{ch:six}{19}%
  \Test\addtocounterpageref{ch:seven}{23}%
  \Test\setcounterref{ch:two}{1}%
  \Test\setcounterref{ch:three}{2}%
  \Test\setcounterref{ch:four}{11}%
  \Test\addtocounterref{ch:two}{11}%
  \Test\addtocounterref{ch:three}{12}%
  \Test\addtocounterref{ch:four}{21}%
\fi
\frontmatter
\chapter{Chapter one}\label{ch:one}
\cleardoublepage
\mainmatter
\chapter{Chapter two}\label{ch:two}
\cleardoublepage
\chapter{Chapter three}\label{ch:three}
\cleardoublepage
\setcounter{chapter}{10}
\chapter{Chapter four}\label{ch:four}
\cleardoublepage
\appendix
\chapter{Chapter five}\label{ch:five}
\cleardoublepage
\chapter{Chapter six}\label{ch:six}
\cleardoublepage
\null
\cleardoublepage
\chapter{Chapter seven}\label{ch:seven}
\end{document}
%</test5>
%    \end{macrocode}
%
% \section{Installation}
%
% \subsection{Download}
%
% \paragraph{Package.} This package is available on
% CTAN\footnote{\url{ftp://ftp.ctan.org/tex-archive/}}:
% \begin{description}
% \item[\CTAN{macros/latex/contrib/oberdiek/refcount.dtx}] The source file.
% \item[\CTAN{macros/latex/contrib/oberdiek/refcount.pdf}] Documentation.
% \end{description}
%
%
% \paragraph{Bundle.} All the packages of the bundle `oberdiek'
% are also available in a TDS compliant ZIP archive. There
% the packages are already unpacked and the documentation files
% are generated. The files and directories obey the TDS standard.
% \begin{description}
% \item[\CTAN{install/macros/latex/contrib/oberdiek.tds.zip}]
% \end{description}
% \emph{TDS} refers to the standard ``A Directory Structure
% for \TeX\ Files'' (\CTAN{tds/tds.pdf}). Directories
% with \xfile{texmf} in their name are usually organized this way.
%
% \subsection{Bundle installation}
%
% \paragraph{Unpacking.} Unpack the \xfile{oberdiek.tds.zip} in the
% TDS tree (also known as \xfile{texmf} tree) of your choice.
% Example (linux):
% \begin{quote}
%   |unzip oberdiek.tds.zip -d ~/texmf|
% \end{quote}
%
% \paragraph{Script installation.}
% Check the directory \xfile{TDS:scripts/oberdiek/} for
% scripts that need further installation steps.
% Package \xpackage{attachfile2} comes with the Perl script
% \xfile{pdfatfi.pl} that should be installed in such a way
% that it can be called as \texttt{pdfatfi}.
% Example (linux):
% \begin{quote}
%   |chmod +x scripts/oberdiek/pdfatfi.pl|\\
%   |cp scripts/oberdiek/pdfatfi.pl /usr/local/bin/|
% \end{quote}
%
% \subsection{Package installation}
%
% \paragraph{Unpacking.} The \xfile{.dtx} file is a self-extracting
% \docstrip\ archive. The files are extracted by running the
% \xfile{.dtx} through \plainTeX:
% \begin{quote}
%   \verb|tex refcount.dtx|
% \end{quote}
%
% \paragraph{TDS.} Now the different files must be moved into
% the different directories in your installation TDS tree
% (also known as \xfile{texmf} tree):
% \begin{quote}
% \def\t{^^A
% \begin{tabular}{@{}>{\ttfamily}l@{ $\rightarrow$ }>{\ttfamily}l@{}}
%   refcount.sty & tex/latex/oberdiek/refcount.sty\\
%   refcount.pdf & doc/latex/oberdiek/refcount.pdf\\
%   test/refcount-test1.tex & doc/latex/oberdiek/test/refcount-test1.tex\\
%   test/refcount-test2.tex & doc/latex/oberdiek/test/refcount-test2.tex\\
%   test/refcount-test3.tex & doc/latex/oberdiek/test/refcount-test3.tex\\
%   test/refcount-test4.tex & doc/latex/oberdiek/test/refcount-test4.tex\\
%   test/refcount-test5.tex & doc/latex/oberdiek/test/refcount-test5.tex\\
%   refcount.dtx & source/latex/oberdiek/refcount.dtx\\
% \end{tabular}^^A
% }^^A
% \sbox0{\t}^^A
% \ifdim\wd0>\linewidth
%   \begingroup
%     \advance\linewidth by\leftmargin
%     \advance\linewidth by\rightmargin
%   \edef\x{\endgroup
%     \def\noexpand\lw{\the\linewidth}^^A
%   }\x
%   \def\lwbox{^^A
%     \leavevmode
%     \hbox to \linewidth{^^A
%       \kern-\leftmargin\relax
%       \hss
%       \usebox0
%       \hss
%       \kern-\rightmargin\relax
%     }^^A
%   }^^A
%   \ifdim\wd0>\lw
%     \sbox0{\small\t}^^A
%     \ifdim\wd0>\linewidth
%       \ifdim\wd0>\lw
%         \sbox0{\footnotesize\t}^^A
%         \ifdim\wd0>\linewidth
%           \ifdim\wd0>\lw
%             \sbox0{\scriptsize\t}^^A
%             \ifdim\wd0>\linewidth
%               \ifdim\wd0>\lw
%                 \sbox0{\tiny\t}^^A
%                 \ifdim\wd0>\linewidth
%                   \lwbox
%                 \else
%                   \usebox0
%                 \fi
%               \else
%                 \lwbox
%               \fi
%             \else
%               \usebox0
%             \fi
%           \else
%             \lwbox
%           \fi
%         \else
%           \usebox0
%         \fi
%       \else
%         \lwbox
%       \fi
%     \else
%       \usebox0
%     \fi
%   \else
%     \lwbox
%   \fi
% \else
%   \usebox0
% \fi
% \end{quote}
% If you have a \xfile{docstrip.cfg} that configures and enables \docstrip's
% TDS installing feature, then some files can already be in the right
% place, see the documentation of \docstrip.
%
% \subsection{Refresh file name databases}
%
% If your \TeX~distribution
% (\teTeX, \mikTeX, \dots) relies on file name databases, you must refresh
% these. For example, \teTeX\ users run \verb|texhash| or
% \verb|mktexlsr|.
%
% \subsection{Some details for the interested}
%
% \paragraph{Attached source.}
%
% The PDF documentation on CTAN also includes the
% \xfile{.dtx} source file. It can be extracted by
% AcrobatReader 6 or higher. Another option is \textsf{pdftk},
% e.g. unpack the file into the current directory:
% \begin{quote}
%   \verb|pdftk refcount.pdf unpack_files output .|
% \end{quote}
%
% \paragraph{Unpacking with \LaTeX.}
% The \xfile{.dtx} chooses its action depending on the format:
% \begin{description}
% \item[\plainTeX:] Run \docstrip\ and extract the files.
% \item[\LaTeX:] Generate the documentation.
% \end{description}
% If you insist on using \LaTeX\ for \docstrip\ (really,
% \docstrip\ does not need \LaTeX), then inform the autodetect routine
% about your intention:
% \begin{quote}
%   \verb|latex \let\install=y% \iffalse meta-comment
%
% File: refcount.dtx
% Version: 2011/10/16 v3.4
% Info: Data extraction from label references
%
% Copyright (C) 1998, 2000, 2006, 2008, 2010, 2011 by
%    Heiko Oberdiek <heiko.oberdiek at googlemail.com>
%
% This work may be distributed and/or modified under the
% conditions of the LaTeX Project Public License, either
% version 1.3c of this license or (at your option) any later
% version. This version of this license is in
%    http://www.latex-project.org/lppl/lppl-1-3c.txt
% and the latest version of this license is in
%    http://www.latex-project.org/lppl.txt
% and version 1.3 or later is part of all distributions of
% LaTeX version 2005/12/01 or later.
%
% This work has the LPPL maintenance status "maintained".
%
% This Current Maintainer of this work is Heiko Oberdiek.
%
% This work consists of the main source file refcount.dtx
% and the derived files
%    refcount.sty, refcount.pdf, refcount.ins, refcount.drv,
%    refcount-test1.tex, refcount-test2.tex, refcount-test3.tex,
%    refcount-test4.tex, refcount-test5.tex.
%
% Distribution:
%    CTAN:macros/latex/contrib/oberdiek/refcount.dtx
%    CTAN:macros/latex/contrib/oberdiek/refcount.pdf
%
% Unpacking:
%    (a) If refcount.ins is present:
%           tex refcount.ins
%    (b) Without refcount.ins:
%           tex refcount.dtx
%    (c) If you insist on using LaTeX
%           latex \let\install=y\input{refcount.dtx}
%        (quote the arguments according to the demands of your shell)
%
% Documentation:
%    (a) If refcount.drv is present:
%           latex refcount.drv
%    (b) Without refcount.drv:
%           latex refcount.dtx; ...
%    The class ltxdoc loads the configuration file ltxdoc.cfg
%    if available. Here you can specify further options, e.g.
%    use A4 as paper format:
%       \PassOptionsToClass{a4paper}{article}
%
%    Programm calls to get the documentation (example):
%       pdflatex refcount.dtx
%       makeindex -s gind.ist refcount.idx
%       pdflatex refcount.dtx
%       makeindex -s gind.ist refcount.idx
%       pdflatex refcount.dtx
%
% Installation:
%    TDS:tex/latex/oberdiek/refcount.sty
%    TDS:doc/latex/oberdiek/refcount.pdf
%    TDS:doc/latex/oberdiek/test/refcount-test1.tex
%    TDS:doc/latex/oberdiek/test/refcount-test2.tex
%    TDS:doc/latex/oberdiek/test/refcount-test3.tex
%    TDS:doc/latex/oberdiek/test/refcount-test4.tex
%    TDS:doc/latex/oberdiek/test/refcount-test5.tex
%    TDS:source/latex/oberdiek/refcount.dtx
%
%<*ignore>
\begingroup
  \catcode123=1 %
  \catcode125=2 %
  \def\x{LaTeX2e}%
\expandafter\endgroup
\ifcase 0\ifx\install y1\fi\expandafter
         \ifx\csname processbatchFile\endcsname\relax\else1\fi
         \ifx\fmtname\x\else 1\fi\relax
\else\csname fi\endcsname
%</ignore>
%<*install>
\input docstrip.tex
\Msg{************************************************************************}
\Msg{* Installation}
\Msg{* Package: refcount 2011/10/16 v3.4 Data extraction from label references (HO)}
\Msg{************************************************************************}

\keepsilent
\askforoverwritefalse

\let\MetaPrefix\relax
\preamble

This is a generated file.

Project: refcount
Version: 2011/10/16 v3.4

Copyright (C) 1998, 2000, 2006, 2008, 2010, 2011 by
   Heiko Oberdiek <heiko.oberdiek at googlemail.com>

This work may be distributed and/or modified under the
conditions of the LaTeX Project Public License, either
version 1.3c of this license or (at your option) any later
version. This version of this license is in
   http://www.latex-project.org/lppl/lppl-1-3c.txt
and the latest version of this license is in
   http://www.latex-project.org/lppl.txt
and version 1.3 or later is part of all distributions of
LaTeX version 2005/12/01 or later.

This work has the LPPL maintenance status "maintained".

This Current Maintainer of this work is Heiko Oberdiek.

This work consists of the main source file refcount.dtx
and the derived files
   refcount.sty, refcount.pdf, refcount.ins, refcount.drv,
   refcount-test1.tex, refcount-test2.tex, refcount-test3.tex,
   refcount-test4.tex, refcount-test5.tex.

\endpreamble
\let\MetaPrefix\DoubleperCent

\generate{%
  \file{refcount.ins}{\from{refcount.dtx}{install}}%
  \file{refcount.drv}{\from{refcount.dtx}{driver}}%
  \usedir{tex/latex/oberdiek}%
  \file{refcount.sty}{\from{refcount.dtx}{package}}%
  \usedir{doc/latex/oberdiek/test}%
  \file{refcount-test1.tex}{\from{refcount.dtx}{test1}}%
  \file{refcount-test2.tex}{\from{refcount.dtx}{test2}}%
  \file{refcount-test3.tex}{\from{refcount.dtx}{test3}}%
  \file{refcount-test4.tex}{\from{refcount.dtx}{test3,test4}}%
  \file{refcount-test5.tex}{\from{refcount.dtx}{test5}}%
  \nopreamble
  \nopostamble
  \usedir{source/latex/oberdiek/catalogue}%
  \file{refcount.xml}{\from{refcount.dtx}{catalogue}}%
}

\catcode32=13\relax% active space
\let =\space%
\Msg{************************************************************************}
\Msg{*}
\Msg{* To finish the installation you have to move the following}
\Msg{* file into a directory searched by TeX:}
\Msg{*}
\Msg{*     refcount.sty}
\Msg{*}
\Msg{* To produce the documentation run the file `refcount.drv'}
\Msg{* through LaTeX.}
\Msg{*}
\Msg{* Happy TeXing!}
\Msg{*}
\Msg{************************************************************************}

\endbatchfile
%</install>
%<*ignore>
\fi
%</ignore>
%<*driver>
\NeedsTeXFormat{LaTeX2e}
\ProvidesFile{refcount.drv}%
  [2011/10/16 v3.4 Data extraction from label references (HO)]%
\documentclass{ltxdoc}
\usepackage{holtxdoc}[2011/11/22]
\begin{document}
  \DocInput{refcount.dtx}%
\end{document}
%</driver>
% \fi
%
% \CheckSum{1185}
%
% \CharacterTable
%  {Upper-case    \A\B\C\D\E\F\G\H\I\J\K\L\M\N\O\P\Q\R\S\T\U\V\W\X\Y\Z
%   Lower-case    \a\b\c\d\e\f\g\h\i\j\k\l\m\n\o\p\q\r\s\t\u\v\w\x\y\z
%   Digits        \0\1\2\3\4\5\6\7\8\9
%   Exclamation   \!     Double quote  \"     Hash (number) \#
%   Dollar        \$     Percent       \%     Ampersand     \&
%   Acute accent  \'     Left paren    \(     Right paren   \)
%   Asterisk      \*     Plus          \+     Comma         \,
%   Minus         \-     Point         \.     Solidus       \/
%   Colon         \:     Semicolon     \;     Less than     \<
%   Equals        \=     Greater than  \>     Question mark \?
%   Commercial at \@     Left bracket  \[     Backslash     \\
%   Right bracket \]     Circumflex    \^     Underscore    \_
%   Grave accent  \`     Left brace    \{     Vertical bar  \|
%   Right brace   \}     Tilde         \~}
%
% \GetFileInfo{refcount.drv}
%
% \title{The \xpackage{refcount} package}
% \date{2011/10/16 v3.4}
% \author{Heiko Oberdiek\\\xemail{heiko.oberdiek at googlemail.com}}
%
% \maketitle
%
% \begin{abstract}
% References are not numbers, however they often store numerical
% data such as section or page numbers. \cs{ref} or \cs{pageref}
% cannot be used for counter assignments or calculations because
% they are not expandable, generate warnings, or can even be links.
% The package provides expandable macros to extract the data
% from references. Packages \xpackage{hyperref}, \xpackage{nameref},
% \xpackage{titleref}, and \xpackage{babel} are supported.
% \end{abstract}
%
% \tableofcontents
%
% \section{Usage}
%
% \subsection{Setting counters}
%
% The following commands are similar to \LaTeX's
% \cs{setcounter} and \cs{addtocounter},
% but they extract the number value from a reference:
% \begin{quote}
%   \cs{setcounterref}, \cs{addtocounterref}\\
%   \cs{setcounterpageref}, \cs{addtocounterpageref}
% \end{quote}
% They take two arguments:
% \begin{quote}
%    \cs{...counter...ref} |{|\meta{\LaTeX\ counter}|}|
%    |{|\meta{reference}|}|
% \end{quote}
% An undefined references produces the usual LaTeX warning
% and its value is assumed to be zero.
% Example:
% \begin{quote}
%\begin{verbatim}
%\newcounter{ctrA}
%\newcounter{ctrB}
%\refstepcounter{ctrA}\label{ref:A}
%\setcounterref{ctrB}{ref:A}
%\addtocounterpageref{ctrB}{ref:A}
%\end{verbatim}
% \end{quote}
%
% \subsection{Expandable commands}
%
% These commands that can be used in expandible contexts
% (inside calculations, \cs{edef}, \cs{csname}, \cs{write}, \dots):
% \begin{quote}
%   \cs{getrefnumber}, \cs{getpagerefnumber}
% \end{quote}
% They take one argument, the reference:
% \begin{quote}
%   \cs{get...refnumber} |{|\meta{reference}|}|
% \end{quote}
% The default for undefined references can be changed
% with macro \cs{setrefcountdefault}, for example this
% package calls:
% \begin{quote}
%   \cs{setrefcountdefault}|{0}|
% \end{quote}
%
% Since version 2.0 of this package there is a new
% command:
% \begin{quote}
%   \cs{getrefbykeydefault} |{|\meta{reference}|}|
%   |{|\meta{key}|}| |{|\meta{default}|}|
% \end{quote}
% This generalized version allows the extraction
% of further properties of a reference than the
% two standard ones. Thus the following properties
% are supported, if they are available:
% \begin{quote}
% \begin{tabular}{@{}l|l|l@{}}
%    Key & Description & Package\\
% \hline
%   \meta{empty} & same as \cs{ref} & \LaTeX\\
%   |page| & same as \cs{pageref} & \LaTeX\\
%   |title| & section and caption titles & \xpackage{titleref}\\
%   |name| & section and caption titles & \xpackage{nameref}\\
%   |anchor| & anchor name & \xpackage{hyperref}\\
%   |url| & url/file & \xpackage{hyperref}/\xpackage{xr}
% \end{tabular}
% \end{quote}
%
% Since version 3.2 the expandable macros described before
% in this section
% are expandable in exact two expansion steps.
%
% \subsection{Undefined references}
%
% Because warnings and assignments cannot be used in
% expandible contexts, undefined references do not
% produce a warning, their values are assumed to be zero.
% Example:
% \begin{quote}
%\begin{verbatim}
%\label{ref:here}% somewhere
%\refused{ref:here}% see below
%\ifodd\getpagerefnumber{ref:here}%
%  reference is on an odd page
%\else
%  reference is on an even page
%\fi
%\end{verbatim}
% \end{quote}
%
% In case of undefined references the user usually want's
% to be informed. Also \LaTeX\ prints a warning at
% the end of the \LaTeX\ run. To notify \LaTeX\ and
% get a normal warning, just use
% \begin{quote}
%   \cs{refused} |{|\meta{reference}|}|
% \end{quote}
% outside the expanding context. Example, see above.
%
% \subsubsection{Check for undefined references}
%
% In version 3.2 macros were added, that test, whether references
% are defined.
% \begin{declcs}{IfRefUndefinedExpandable} \M{refname} \M{then} \M{else}\\
%   \cs{IfRefUndefinedBabel} \M{refname} \M{then} \M{else}
% \end{declcs}
% If the reference is not available and therefore undefined, then
% argument \meta{then} is executed, otherwise argument \meta{else}
% is called. Macro \cs{IfRefUndefinedExpandable} is expandable,
% but \meta{refname} must not contain babel shorthand characters.
% Macro \cs{IfRefUndefinedBabel} supports shorthand characters of
% babel, but it is not expandable.
%
% \subsection{Notes}
%
% \begin{itemize}
% \item
%   The method of extracting the number in this
%   package also works in cases, where the
%   reference cannot be used directly, because
%   a package such as \xpackage{hyperref} has added
%   extra stuff (hyper link), so that the reference cannot
%   be used as number any more.
% \item
%   If the reference does not contain a number,
%   assignments to a counter will fail of course.
% \end{itemize}
%
%
% \StopEventually{
% }
%
% \section{Implementation}
%
%    \begin{macrocode}
%<*package>
%    \end{macrocode}
%    Reload check, especially if the package is not used with \LaTeX.
%    \begin{macrocode}
\begingroup\catcode61\catcode48\catcode32=10\relax%
  \catcode13=5 % ^^M
  \endlinechar=13 %
  \catcode35=6 % #
  \catcode39=12 % '
  \catcode44=12 % ,
  \catcode45=12 % -
  \catcode46=12 % .
  \catcode58=12 % :
  \catcode64=11 % @
  \catcode123=1 % {
  \catcode125=2 % }
  \expandafter\let\expandafter\x\csname ver@refcount.sty\endcsname
  \ifx\x\relax % plain-TeX, first loading
  \else
    \def\empty{}%
    \ifx\x\empty % LaTeX, first loading,
      % variable is initialized, but \ProvidesPackage not yet seen
    \else
      \expandafter\ifx\csname PackageInfo\endcsname\relax
        \def\x#1#2{%
          \immediate\write-1{Package #1 Info: #2.}%
        }%
      \else
        \def\x#1#2{\PackageInfo{#1}{#2, stopped}}%
      \fi
      \x{refcount}{The package is already loaded}%
      \aftergroup\endinput
    \fi
  \fi
\endgroup%
%    \end{macrocode}
%    Package identification:
%    \begin{macrocode}
\begingroup\catcode61\catcode48\catcode32=10\relax%
  \catcode13=5 % ^^M
  \endlinechar=13 %
  \catcode35=6 % #
  \catcode39=12 % '
  \catcode40=12 % (
  \catcode41=12 % )
  \catcode44=12 % ,
  \catcode45=12 % -
  \catcode46=12 % .
  \catcode47=12 % /
  \catcode58=12 % :
  \catcode64=11 % @
  \catcode91=12 % [
  \catcode93=12 % ]
  \catcode123=1 % {
  \catcode125=2 % }
  \expandafter\ifx\csname ProvidesPackage\endcsname\relax
    \def\x#1#2#3[#4]{\endgroup
      \immediate\write-1{Package: #3 #4}%
      \xdef#1{#4}%
    }%
  \else
    \def\x#1#2[#3]{\endgroup
      #2[{#3}]%
      \ifx#1\@undefined
        \xdef#1{#3}%
      \fi
      \ifx#1\relax
        \xdef#1{#3}%
      \fi
    }%
  \fi
\expandafter\x\csname ver@refcount.sty\endcsname
\ProvidesPackage{refcount}%
  [2011/10/16 v3.4 Data extraction from label references (HO)]%
%    \end{macrocode}
%
%    \begin{macrocode}
\begingroup\catcode61\catcode48\catcode32=10\relax%
  \catcode13=5 % ^^M
  \endlinechar=13 %
  \catcode123=1 % {
  \catcode125=2 % }
  \catcode64=11 % @
  \def\x{\endgroup
    \expandafter\edef\csname rc@AtEnd\endcsname{%
      \endlinechar=\the\endlinechar\relax
      \catcode13=\the\catcode13\relax
      \catcode32=\the\catcode32\relax
      \catcode35=\the\catcode35\relax
      \catcode61=\the\catcode61\relax
      \catcode64=\the\catcode64\relax
      \catcode123=\the\catcode123\relax
      \catcode125=\the\catcode125\relax
    }%
  }%
\x\catcode61\catcode48\catcode32=10\relax%
\catcode13=5 % ^^M
\endlinechar=13 %
\catcode35=6 % #
\catcode64=11 % @
\catcode123=1 % {
\catcode125=2 % }
\def\TMP@EnsureCode#1#2{%
  \edef\rc@AtEnd{%
    \rc@AtEnd
    \catcode#1=\the\catcode#1\relax
  }%
  \catcode#1=#2\relax
}
\TMP@EnsureCode{33}{12}% !
\TMP@EnsureCode{39}{12}% '
\TMP@EnsureCode{42}{12}% *
\TMP@EnsureCode{45}{12}% -
\TMP@EnsureCode{46}{12}% .
\TMP@EnsureCode{47}{12}% /
\TMP@EnsureCode{91}{12}% [
\TMP@EnsureCode{93}{12}% ]
\TMP@EnsureCode{96}{12}% `
\edef\rc@AtEnd{\rc@AtEnd\noexpand\endinput}
%    \end{macrocode}
%
% \subsection{Loading packages}
%
%    \begin{macrocode}
\begingroup\expandafter\expandafter\expandafter\endgroup
\expandafter\ifx\csname RequirePackage\endcsname\relax
  \input ltxcmds.sty\relax
  \input infwarerr.sty\relax
\else
  \RequirePackage{ltxcmds}[2011/11/09]%
  \RequirePackage{infwarerr}[2010/04/08]%
\fi
%    \end{macrocode}
%
% \subsection{Defining commands}
%
%    \begin{macro}{\rc@IfDefinable}
%    \begin{macrocode}
\ltx@IfUndefined{@ifdefinable}{%
  \def\rc@IfDefinable#1{%
    \ifx#1\ltx@undefined
      \expandafter\ltx@firstofone
    \else
      \ifx#1\relax
        \expandafter\expandafter\expandafter\ltx@firstofone
      \else
        \@PackageError{refcount}{%
          Command \string#1 is already defined.\MessageBreak
          It will not redefined by this package%
        }\@ehc
        \expandafter\expandafter\expandafter\ltx@gobble
      \fi
    \fi
  }%
}{%
  \let\rc@IfDefinable\@ifdefinable
}
%    \end{macrocode}
%    \end{macro}
%
%    \begin{macro}{\rc@RobustDefOne}
%    \begin{macro}{\rc@RobustDefZero}
%    \begin{macrocode}
\ltx@IfUndefined{protected}{%
  \ltx@IfUndefined{DeclareRobustCommand}{%
    \def\rc@RobustDefOne#1#2#3#4{%
      \rc@IfDefinable#3{%
        #1\def#3##1{#4}%
      }%
    }%
    \def\rc@RobustDefZero#1#2{%
      \rc@IfDefinable#1{%
        \def#1{#2}%
      }%
    }%
  }{%
    \def\rc@RobustDefOne#1#2#3#4{%
      \rc@IfDefinable#3{%
        \DeclareRobustCommand#2#3[1]{#4}%
      }%
    }%
    \def\rc@RobustDefZero#1#2{%
      \rc@IfDefinable#1{%
        \DeclareRobustCommand#1{#2}%
      }%
    }%
  }%
}{%
  \def\rc@RobustDefOne#1#2#3#4{%
    \rc@IfDefinable#3{%
      \protected#1\def#3##1{#4}%
    }%
  }%
  \def\rc@RobustDefZero#1#2{%
    \rc@IfDefinable#1{%
      \protected\def#1{#2}%
    }%
  }%
}
%    \end{macrocode}
%    \end{macro}
%    \end{macro}
%
%    \begin{macro}{\rc@newcommand}
%    \begin{macrocode}
\ltx@IfUndefined{newcommand}{%
  \def\rc@newcommand*#1[#2]#3{% hash-ok
    \rc@IfDefinable#1{%
      \ifcase#2 %
        \def#1{#3}%
      \or
        \def#1##1{#3}%
      \or
        \def#1##1##2{#3}%
      \else
        \rc@InternalError
      \fi
    }%
  }%
}{%
  \let\rc@newcommand\newcommand
}
%    \end{macrocode}
%    \end{macro}
%
% \subsection{\cs{setrefcountdefault}}
%
%    \begin{macro}{\setrefcountdefault}
%    \begin{macrocode}
\rc@RobustDefOne\long{}\setrefcountdefault{%
  \def\rc@default{#1}%
}
%    \end{macrocode}
%    \end{macro}
%    \begin{macrocode}
\setrefcountdefault{0}
%    \end{macrocode}
%
% \subsection{\cs{refused}}
%
%    \begin{macro}{\refused}
%    \begin{macrocode}
\ltx@IfUndefined{G@refundefinedtrue}{%
  \rc@RobustDefOne{}{*}\refused{%
    \begingroup
      \csname @safe@activestrue\endcsname
      \ltx@IfUndefined{r@#1}{%
        \protect\G@refundefinedtrue
        \rc@WarningUndefined{#1}%
      }{}%
    \endgroup
  }%
}{%
  \rc@RobustDefOne{}{*}\refused{%
    \begingroup
      \csname @safe@activestrue\endcsname
      \ltx@IfUndefined{r@#1}{%
        \csname protect\expandafter\endcsname
        \csname G@refundefinedtrue\endcsname
        \rc@WarningUndefined{#1}%
      }{}%
    \endgroup
  }%
}
%    \end{macrocode}
%    \end{macro}
%    \begin{macro}{\rc@WarningUndefined}
%    \begin{macrocode}
\ltx@IfUndefined{@latex@warning}{%
  \def\rc@WarningUndefined#1{%
    \ltx@ifundefined{thepage}{%
      \def\thepage{\number\count0 }%
    }{}%
    \@PackageWarning{refcount}{%
      Reference `#1' on page \thepage\space undefined%
    }%
  }%
}{%
  \def\rc@WarningUndefined#1{%
    \@latex@warning{%
      Reference `#1' on page \thepage\space undefined%
    }%
  }%
}
%    \end{macrocode}
%    \end{macro}
%
% \subsection{Setting counters by reference data}
%
% \subsubsection{Generic setting}
%
%    \begin{macro}{\rc@set}
% Generic command for
% |\|$\{$|set|$,$|addto|$\}$|counter|$\{$|page|$,\}$|ref|:
%\begin{quote}
%\begin{tabular}[t]{@{}l@{: }l@{}}
% |#1|& \cs{setcounter}, \cs{addtocounter}\\
% |#2|& \cs{ltx@car} (for \cs{ref}), \cs{ltx@cartwo} (for \cs{pageref})\\
% |#3|& \hologo{LaTeX} counter\\
% |#4|& reference\\
%\end{tabular}
%\end{quote}
%    \begin{macrocode}
\def\rc@set#1#2#3#4{%
  \begingroup
    \csname @safe@activestrue\endcsname
    \refused{#4}%
    \expandafter\rc@@set\csname r@#4\endcsname{#1}{#2}{#3}%
  \endgroup
}
%    \end{macrocode}
%    \end{macro}
%    \begin{macro}{\rc@@set}
%\begin{quote}
%\begin{tabular}[t]{@{}l@{: }l@{}}
% |#1|& \cs{r@<...>}\\
% |#2|& \cs{setcounter}, \cs{addtocounter}\\
% |#3|& \cs{ltx@car} (for \cs{ref}), \cs{ltx@carsecond} (for \cs{pageref})\\
% |#4|& \hologo{LaTeX} counter\\
%\end{tabular}
%\end{quote}
%    \begin{macrocode}
\def\rc@@set#1#2#3#4{%
  \ifx#1\relax
    #2{#4}{\rc@default}%
  \else
    #2{#4}{%
      \expandafter#3#1\rc@default\rc@default\@nil
    }%
  \fi
}
%    \end{macrocode}
%    \end{macro}
%
% \subsubsection{User commands}
%
%    \begin{macro}{\setcounterref}
%    \begin{macrocode}
\rc@RobustDefZero\setcounterref{%
  \rc@set\setcounter\ltx@car
}
%    \end{macrocode}
%    \end{macro}
%    \begin{macro}{\addtocounterref}
%    \begin{macrocode}
\rc@RobustDefZero\addtocounterref{%
  \rc@set\addtocounter\ltx@car
}
%    \end{macrocode}
%    \end{macro}
%    \begin{macro}{\setcounterpageref}
%    \begin{macrocode}
\rc@RobustDefZero\setcounterpageref{%
  \rc@set\setcounter\ltx@carsecond
}
%    \end{macrocode}
%    \end{macro}
%    \begin{macro}{\addtocounterpageref}
%    \begin{macrocode}
\rc@RobustDefZero\addtocounterpageref{%
  \rc@set\addtocounter\ltx@carsecond
}
%    \end{macrocode}
%    \end{macro}
%
% \subsection{Extracting references}
%
%    \begin{macro}{\getrefnumber}
%    \begin{macrocode}
\rc@newcommand*{\getrefnumber}[1]{%
  \romannumeral
  \ltx@ifundefined{r@#1}{%
    \expandafter\ltx@zero
    \rc@default
  }{%
    \expandafter\expandafter\expandafter\rc@extract@
    \expandafter\expandafter\expandafter!%
    \csname r@#1\expandafter\endcsname
    \expandafter{\rc@default}\@nil
  }%
}
%    \end{macrocode}
%    \end{macro}
%    \begin{macro}{\getpagerefnumber}
%    \begin{macrocode}
\rc@newcommand*{\getpagerefnumber}[1]{%
  \romannumeral
  \ltx@ifundefined{r@#1}{%
    \expandafter\ltx@zero
    \rc@default
  }{%
    \expandafter\expandafter\expandafter\rc@extract@page
    \expandafter\expandafter\expandafter!%
    \csname r@#1\expandafter\expandafter\expandafter\endcsname
    \expandafter\expandafter\expandafter{%
      \expandafter\rc@default
    \expandafter}\expandafter{\rc@default}\@nil
  }%
}
%    \end{macrocode}
%    \end{macro}
%    \begin{macro}{\getrefbykeydefault}
%    \begin{macrocode}
\rc@newcommand*{\getrefbykeydefault}[2]{%
  \romannumeral
  \expandafter\rc@getrefbykeydefault
    \csname r@#1\expandafter\endcsname
    \csname rc@extract@#2\endcsname
}
%    \end{macrocode}
%    \end{macro}
%    \begin{macro}{\rc@getrefbykeydefault}
%\begin{quote}
%\begin{tabular}[t]{@{}l@{: }l@{}}
% |#1|& \cs{r@<...>}\\
% |#2|& \cs{rc@extract@<...>}\\
% |#3|& default\\
%\end{tabular}
%\end{quote}
%    \begin{macrocode}
\long\def\rc@getrefbykeydefault#1#2#3{%
  \ifx#1\relax
    % reference is undefined
    \ltx@ReturnAfterElseFi{%
      \ltx@zero
      #3%
    }%
  \else
    \ltx@ReturnAfterFi{%
      \ifx#2\relax
        % extract method is missing
        \ltx@ReturnAfterElseFi{%
          \ltx@zero
          #3%
        }%
      \else
        \ltx@ReturnAfterFi{%
          \expandafter
          \rc@generic#1{#3}{#3}{#3}{#3}{#3}\@nil#2{#3}%
        }%
      \fi
    }%
  \fi
}
%    \end{macrocode}
%    \end{macro}
%    \begin{macro}{\rc@generic}
%\begin{quote}
%\begin{tabular}[t]{@{}l@{: }l@{}}
% |#1|& first item in \cs{r@<...>}\\
% |#2|& remaining items in \cs{r@<...>}\\
% |#3|& \cs{rc@extract@<...>}\\
% |#4|& default\\
%\end{tabular}
%\end{quote}
%    \begin{macrocode}
\long\def\rc@generic#1#2\@nil#3#4{%
  #3{#1\TR@TitleReference\@empty{#4}\@nil}{#1}#2\@nil
}
%    \end{macrocode}
%    \end{macro}
%    \begin{macro}{\rc@extract@}
%    \begin{macrocode}
\long\def\rc@extract@#1#2#3\@nil{%
  \ltx@zero
  #2%
}
%    \end{macrocode}
%    \end{macro}
%    \begin{macro}{\rc@extract@page}
%    \begin{macrocode}
\long\def\rc@extract@page#1#2#3#4\@nil{%
  \ltx@zero
  #3%
}
%    \end{macrocode}
%    \end{macro}
%    \begin{macro}{\rc@extract@name}
%    \begin{macrocode}
\long\def\rc@extract@name#1#2#3#4#5\@nil{%
  \ltx@zero
  #4%
}
%    \end{macrocode}
%    \end{macro}
%    \begin{macro}{\rc@extract@anchor}
%    \begin{macrocode}
\long\def\rc@extract@anchor#1#2#3#4#5#6\@nil{%
  \ltx@zero
  #5%
}
%    \end{macrocode}
%    \end{macro}
%    \begin{macro}{\rc@extract@url}
%    \begin{macrocode}
\long\def\rc@extract@url#1#2#3#4#5#6#7\@nil{%
  \ltx@zero
  #6%
}
%    \end{macrocode}
%    \end{macro}
%    \begin{macro}{\rc@extract@title}
%    \begin{macrocode}
\long\def\rc@extract@title#1#2\@nil{%
  \rc@@extract@title#1%
}
%    \end{macrocode}
%    \end{macro}
%    \begin{macro}{\rc@@extract@title}
%    \begin{macrocode}
\long\def\rc@@extract@title#1\TR@TitleReference#2#3#4\@nil{%
  \ltx@zero
  #3%
}
%    \end{macrocode}
%    \end{macro}
%
% \subsection{Macros for checking undefined references}
%
%    \begin{macro}{\IfRefUndefinedExpandable}
%    \begin{macrocode}
\rc@newcommand*{\IfRefUndefinedExpandable}[1]{%
  \ltx@ifundefined{r@#1}\ltx@firstoftwo\ltx@secondoftwo
}
%    \end{macrocode}
%    \end{macro}
%    \begin{macro}{\IfRefUndefinedBabel}
%    \begin{macrocode}
\rc@RobustDefOne{}*\IfRefUndefinedBabel{%
  \begingroup
    \csname safe@actives@true\endcsname
  \expandafter\expandafter\expandafter\endgroup
  \expandafter\ifx\csname r@#1\endcsname\relax
    \expandafter\ltx@firstoftwo
  \else
    \expandafter\ltx@secondoftwo
  \fi
}
%    \end{macrocode}
%    \end{macro}
%    \begin{macrocode}
\rc@AtEnd%
%</package>
%    \end{macrocode}
%
% \section{Test}
%
% \subsection{Catcode checks for loading}
%
%    \begin{macrocode}
%<*test1>
%    \end{macrocode}
%    \begin{macrocode}
\catcode`\{=1 %
\catcode`\}=2 %
\catcode`\#=6 %
\catcode`\@=11 %
\expandafter\ifx\csname count@\endcsname\relax
  \countdef\count@=255 %
\fi
\expandafter\ifx\csname @gobble\endcsname\relax
  \long\def\@gobble#1{}%
\fi
\expandafter\ifx\csname @firstofone\endcsname\relax
  \long\def\@firstofone#1{#1}%
\fi
\expandafter\ifx\csname loop\endcsname\relax
  \expandafter\@firstofone
\else
  \expandafter\@gobble
\fi
{%
  \def\loop#1\repeat{%
    \def\body{#1}%
    \iterate
  }%
  \def\iterate{%
    \body
      \let\next\iterate
    \else
      \let\next\relax
    \fi
    \next
  }%
  \let\repeat=\fi
}%
\def\RestoreCatcodes{}
\count@=0 %
\loop
  \edef\RestoreCatcodes{%
    \RestoreCatcodes
    \catcode\the\count@=\the\catcode\count@\relax
  }%
\ifnum\count@<255 %
  \advance\count@ 1 %
\repeat

\def\RangeCatcodeInvalid#1#2{%
  \count@=#1\relax
  \loop
    \catcode\count@=15 %
  \ifnum\count@<#2\relax
    \advance\count@ 1 %
  \repeat
}
\def\RangeCatcodeCheck#1#2#3{%
  \count@=#1\relax
  \loop
    \ifnum#3=\catcode\count@
    \else
      \errmessage{%
        Character \the\count@\space
        with wrong catcode \the\catcode\count@\space
        instead of \number#3%
      }%
    \fi
  \ifnum\count@<#2\relax
    \advance\count@ 1 %
  \repeat
}
\def\space{ }
\expandafter\ifx\csname LoadCommand\endcsname\relax
  \def\LoadCommand{\input refcount.sty\relax}%
\fi
\def\Test{%
  \RangeCatcodeInvalid{0}{47}%
  \RangeCatcodeInvalid{58}{64}%
  \RangeCatcodeInvalid{91}{96}%
  \RangeCatcodeInvalid{123}{255}%
  \catcode`\@=12 %
  \catcode`\\=0 %
  \catcode`\%=14 %
  \LoadCommand
  \RangeCatcodeCheck{0}{36}{15}%
  \RangeCatcodeCheck{37}{37}{14}%
  \RangeCatcodeCheck{38}{47}{15}%
  \RangeCatcodeCheck{48}{57}{12}%
  \RangeCatcodeCheck{58}{63}{15}%
  \RangeCatcodeCheck{64}{64}{12}%
  \RangeCatcodeCheck{65}{90}{11}%
  \RangeCatcodeCheck{91}{91}{15}%
  \RangeCatcodeCheck{92}{92}{0}%
  \RangeCatcodeCheck{93}{96}{15}%
  \RangeCatcodeCheck{97}{122}{11}%
  \RangeCatcodeCheck{123}{255}{15}%
  \RestoreCatcodes
}
\Test
\csname @@end\endcsname
\end
%    \end{macrocode}
%    \begin{macrocode}
%</test1>
%    \end{macrocode}
%
% \subsection{Macro tests}
%
%    \begin{macrocode}
%<*test2>
\errorcontextlines=10000 %
\showboxbreadth=10000 %
\showboxdepth=10000 %
\begingroup\expandafter\expandafter\expandafter\endgroup
\expandafter\ifx\csname RequirePackage\endcsname\relax
  \input refcount.sty\relax
\else
  \RequirePackage{refcount}[2011/10/16]%
\fi
\catcode`\@=11 %
\begingroup\expandafter\expandafter\expandafter\endgroup
\expandafter\ifx\csname @onelevel@sanitize\endcsname\relax
  \begingroup\expandafter\expandafter\expandafter\endgroup
  \expandafter\ifx\csname detokenize\endcsname\relax
    \def\strip@prefix#1->{}%
    \def\@onelevel@sanitize#1{%
      \edef#1{%
        \expandafter\strip@prefix\meaning#1%
      }%
    }%
  \else
    \def\@onelevel@sanitize#1{%
      \edef#1{%
        \detokenize\expandafter{#1}%
      }%
    }%
  \fi
\fi
\def\msg#{\immediate\write16}
\def\empty{}
\def\space{ }
%    \end{macrocode}
%    \begin{macrocode}
\def\r@foo{{\empty 1}{\empty 2}}
\long\def\test#1#2{%
  \begingroup
    \setbox0=\hbox{%
      \def\TestTask{#1}%
      \@onelevel@sanitize\TestTask
      \msg{* \TestTask}%
      \expandafter\expandafter\expandafter\def
      \expandafter\expandafter\expandafter\TestResult
      \expandafter\expandafter\expandafter{%
        #1%
      }%
      \def\TestExpected{#2}%
      \ifx\TestResult\TestExpected
        \msg{ \space ok.}%
      \else
        \@onelevel@sanitize\TestResult
        \@onelevel@sanitize\TestExpected
        \msg{ \space Result: \space\space[\TestResult]}%
        \msg{ \space Expected: [\TestExpected]}%
        \errmessage{Test failed!}%
      \fi
    }%
    \ifdim\wd0=0pt %
    \else
      \showbox0 %
    \fi
  \endgroup
}
\test{\getrefnumber{foo}}{\empty 1}
\test{\getpagerefnumber{foo}}{\empty 2}
\test{\getrefbykeydefault{foo}{}{\empty default}}{\empty 1}
\test{\getrefbykeydefault{foo}{page}{\empty default}}{\empty 2}
\test{\getrefbykeydefault{foo}{name}{\empty default}}{\empty default}
\test{\getrefbykeydefault{foo}{anchor}{\empty default}}{\empty default}
\test{\getrefbykeydefault{foo}{url}{\empty default}}{\empty default}
\test{\getrefbykeydefault{foo}{title}{\empty default}}{\empty default}
\msg{}
\def\r@foo{{}{}{}{}{}{}{}{}{}{}}
\def\Test#1#2\\{%
  \test{#1{foo}#2}{}%
}
\def\TestGroup{%
  \Test\getrefnumber\\%
  \Test\getpagerefnumber\\%
  \Test\getrefbykeydefault{}{}\\%
  \Test\getrefbykeydefault{page}{}\\%
  \Test\getrefbykeydefault{anchor}{}\\%
  \Test\getrefbykeydefault{name}{}\\%
  \Test\getrefbykeydefault{url}{}\\%
}
\TestGroup
\Test\getrefbykeydefault{title}{}\\%
\msg{}
\def\r@foo{\par\par\par\par\par\par\par\par}
\long\def\Test#1#2\\{%
  \test{#1{foo}#2}{\par}%
}
\TestGroup
\test{\getrefbykeydefault{title}{}{}}{}
\msg{}
\def\r@foo{{ }{ }{ }{ }{ }}
\def\Test#1#2\\{%
  \test{#1{foo}#2}{ }%
}
\TestGroup
\msg{}
\long\def\TestDefault#1{%
  \begingroup
    \setrefcountdefault{#1}%
    \test{\getrefnumber{foo}}{#1}%
    \test{\getpagerefnumber{foo}}{#1}%
  \endgroup
}
\def\TestDefaultX{%
  \TestDefault{}%
  \TestDefault{\par}%
  \TestDefault{ }%
  \TestDefault{\space}%
}
\let\r@foo\@undefined
\TestDefaultX
\let\r@foo\relax
\TestDefaultX
\def\r@foo{}
\TestDefaultX
%    \end{macrocode}
%    \begin{macrocode}
\msg{}
\long\def\Test#1#2#3#4{%
  \begingroup
    \def\TestTask{#1}%
    \@onelevel@sanitize\TestTask
    \msg{* [\TestTask]}%
    \edef\TestResultA{\IfRefUndefinedExpandable{#1}{#2}{#3}}%
    \IfRefUndefinedBabel{#1}{%
      \def\TestResultB{#2}%
    }{%
      \def\TestResultB{#3}%
    }%
    \def\TestExpected{#4}%
    \ifx\TestResultA\TestExpected
      \msg{ \space ok.}%
    \else
      \begingroup
        \@onelevel@sanitize\TestResultA
        \@onelevel@sanitize\TestExpected
        \msg{ \space Result: \space\space[\TestResultA]}%
        \msg{ \space Expected: [\TestExpected]}%
        \errmessage{Test failed!}%
      \endgroup
    \fi
    \ifx\TestResultB\TestExpected
      \msg{ \space ok.}%
    \else
      \begingroup
        \@onelevel@sanitize\TestResultB
        \@onelevel@sanitize\TestExpected
        \msg{ \space Result: \space\space[\TestResultB]}%
        \msg{ \space Expected: [\TestExpected]}%
        \errmessage{Test failed!}%
      \endgroup
    \fi
  \endgroup
}
\begingroup
  \def\r@foo{{}{}}%
  \let\r@bar\@undefined
  \let\r@xyz\relax
  \Test{foo}{true}{false}{false}%
  \Test{bar}{true}{false}{true}%
  \Test{xyz}{true}{false}{true}%
\endgroup
%    \end{macrocode}
%    \begin{macrocode}
\csname @@end\endcsname\end
%</test2>
%    \end{macrocode}
% \subsection{Test with package \xpackage{titleref}}
%
%    \begin{macrocode}
%<*test3>
\NeedsTeXFormat{LaTeX2e}
\documentclass{article}
\usepackage{refcount}[2011/10/16]
%<test4>\usepackage{nameref}
\usepackage{titleref}
\begin{document}
\section{Hello World}
\label{sec:hello}
\section{\hbox{xy}}
\label{sec:foo}
%
\makeatletter
\@ifundefined{r@sec:hello}{%
  \typeout{==> Compile twice!}%
}{%
  \def\test#1#2{%
    \begingroup
      \def\TestTask{#1}%
      \@onelevel@sanitize\TestTask
      \typeout{* \TestTask}%
      \expandafter\expandafter\expandafter\def
      \expandafter\expandafter\expandafter\TestResult
      \expandafter\expandafter\expandafter{%
        #1%
      }%
      \def\TestExpected{#2}%
      \ifx\TestResult\TestExpected
        \typeout{ \space ok.}%
      \else
        \@onelevel@sanitize\TestResult
        \@onelevel@sanitize\TestExpected
        \typeout{ \space Result: \space\space[\TestResult]}%
        \typeout{ \space Expected: [\TestExpected]}%
        \errmessage{Test failed!}%
      \fi
    \endgroup
  }%
  \test{\getrefbykeydefault{sec:hello}{title}{}}{Hello World}%
  \test{\getrefbykeydefault{sec:foo}{title}{}}{\hbox{xy}}%
  \begingroup
    \def\hbox#1{[#1]}% hash-ok
    \test{\getrefbykeydefault{sec:foo}{title}{}}{\hbox{xy}}%
  \endgroup
}
\makeatother
%    \end{macrocode}
%    \begin{macrocode}
\end{document}
%</test3>
%    \end{macrocode}
%    \begin{macrocode}
%<*test5>
\NeedsTeXFormat{LaTeX2e}
\documentclass{book}
\usepackage{refcount}[2011/10/16]
\usepackage{zref-runs}
\newcounter{test}
\begin{document}
\ifnum\zruns>1 %
  \makeatletter
  \def\Test#1#2#3{%
    \begingroup
      \setcounter{test}{10}%
      \sbox0{%
        #1{test}{#2}%
        \ifnum#3=\value{test}%
        \else
          \PackageError{test}{\string#1{#2} <> #3 (\the\value{test})}%
        \fi
      }%
      \ifdim\wd0=0pt %
      \else
        \PackageError{test}{Non-empty box}\@ehc
      \fi
    \endgroup
  }%
  \makeatother
  \Test\setcounterpageref{ch:two}{1}%
  \Test\setcounterpageref{ch:three}{3}%
  \Test\setcounterpageref{ch:four}{5}%
  \Test\setcounterpageref{ch:five}{7}%
  \Test\setcounterpageref{ch:six}{9}%
  \Test\setcounterpageref{ch:seven}{13}%
  \Test\addtocounterpageref{ch:two}{11}%
  \Test\addtocounterpageref{ch:three}{13}%
  \Test\addtocounterpageref{ch:four}{15}%
  \Test\addtocounterpageref{ch:five}{17}%
  \Test\addtocounterpageref{ch:six}{19}%
  \Test\addtocounterpageref{ch:seven}{23}%
  \Test\setcounterref{ch:two}{1}%
  \Test\setcounterref{ch:three}{2}%
  \Test\setcounterref{ch:four}{11}%
  \Test\addtocounterref{ch:two}{11}%
  \Test\addtocounterref{ch:three}{12}%
  \Test\addtocounterref{ch:four}{21}%
\fi
\frontmatter
\chapter{Chapter one}\label{ch:one}
\cleardoublepage
\mainmatter
\chapter{Chapter two}\label{ch:two}
\cleardoublepage
\chapter{Chapter three}\label{ch:three}
\cleardoublepage
\setcounter{chapter}{10}
\chapter{Chapter four}\label{ch:four}
\cleardoublepage
\appendix
\chapter{Chapter five}\label{ch:five}
\cleardoublepage
\chapter{Chapter six}\label{ch:six}
\cleardoublepage
\null
\cleardoublepage
\chapter{Chapter seven}\label{ch:seven}
\end{document}
%</test5>
%    \end{macrocode}
%
% \section{Installation}
%
% \subsection{Download}
%
% \paragraph{Package.} This package is available on
% CTAN\footnote{\url{ftp://ftp.ctan.org/tex-archive/}}:
% \begin{description}
% \item[\CTAN{macros/latex/contrib/oberdiek/refcount.dtx}] The source file.
% \item[\CTAN{macros/latex/contrib/oberdiek/refcount.pdf}] Documentation.
% \end{description}
%
%
% \paragraph{Bundle.} All the packages of the bundle `oberdiek'
% are also available in a TDS compliant ZIP archive. There
% the packages are already unpacked and the documentation files
% are generated. The files and directories obey the TDS standard.
% \begin{description}
% \item[\CTAN{install/macros/latex/contrib/oberdiek.tds.zip}]
% \end{description}
% \emph{TDS} refers to the standard ``A Directory Structure
% for \TeX\ Files'' (\CTAN{tds/tds.pdf}). Directories
% with \xfile{texmf} in their name are usually organized this way.
%
% \subsection{Bundle installation}
%
% \paragraph{Unpacking.} Unpack the \xfile{oberdiek.tds.zip} in the
% TDS tree (also known as \xfile{texmf} tree) of your choice.
% Example (linux):
% \begin{quote}
%   |unzip oberdiek.tds.zip -d ~/texmf|
% \end{quote}
%
% \paragraph{Script installation.}
% Check the directory \xfile{TDS:scripts/oberdiek/} for
% scripts that need further installation steps.
% Package \xpackage{attachfile2} comes with the Perl script
% \xfile{pdfatfi.pl} that should be installed in such a way
% that it can be called as \texttt{pdfatfi}.
% Example (linux):
% \begin{quote}
%   |chmod +x scripts/oberdiek/pdfatfi.pl|\\
%   |cp scripts/oberdiek/pdfatfi.pl /usr/local/bin/|
% \end{quote}
%
% \subsection{Package installation}
%
% \paragraph{Unpacking.} The \xfile{.dtx} file is a self-extracting
% \docstrip\ archive. The files are extracted by running the
% \xfile{.dtx} through \plainTeX:
% \begin{quote}
%   \verb|tex refcount.dtx|
% \end{quote}
%
% \paragraph{TDS.} Now the different files must be moved into
% the different directories in your installation TDS tree
% (also known as \xfile{texmf} tree):
% \begin{quote}
% \def\t{^^A
% \begin{tabular}{@{}>{\ttfamily}l@{ $\rightarrow$ }>{\ttfamily}l@{}}
%   refcount.sty & tex/latex/oberdiek/refcount.sty\\
%   refcount.pdf & doc/latex/oberdiek/refcount.pdf\\
%   test/refcount-test1.tex & doc/latex/oberdiek/test/refcount-test1.tex\\
%   test/refcount-test2.tex & doc/latex/oberdiek/test/refcount-test2.tex\\
%   test/refcount-test3.tex & doc/latex/oberdiek/test/refcount-test3.tex\\
%   test/refcount-test4.tex & doc/latex/oberdiek/test/refcount-test4.tex\\
%   test/refcount-test5.tex & doc/latex/oberdiek/test/refcount-test5.tex\\
%   refcount.dtx & source/latex/oberdiek/refcount.dtx\\
% \end{tabular}^^A
% }^^A
% \sbox0{\t}^^A
% \ifdim\wd0>\linewidth
%   \begingroup
%     \advance\linewidth by\leftmargin
%     \advance\linewidth by\rightmargin
%   \edef\x{\endgroup
%     \def\noexpand\lw{\the\linewidth}^^A
%   }\x
%   \def\lwbox{^^A
%     \leavevmode
%     \hbox to \linewidth{^^A
%       \kern-\leftmargin\relax
%       \hss
%       \usebox0
%       \hss
%       \kern-\rightmargin\relax
%     }^^A
%   }^^A
%   \ifdim\wd0>\lw
%     \sbox0{\small\t}^^A
%     \ifdim\wd0>\linewidth
%       \ifdim\wd0>\lw
%         \sbox0{\footnotesize\t}^^A
%         \ifdim\wd0>\linewidth
%           \ifdim\wd0>\lw
%             \sbox0{\scriptsize\t}^^A
%             \ifdim\wd0>\linewidth
%               \ifdim\wd0>\lw
%                 \sbox0{\tiny\t}^^A
%                 \ifdim\wd0>\linewidth
%                   \lwbox
%                 \else
%                   \usebox0
%                 \fi
%               \else
%                 \lwbox
%               \fi
%             \else
%               \usebox0
%             \fi
%           \else
%             \lwbox
%           \fi
%         \else
%           \usebox0
%         \fi
%       \else
%         \lwbox
%       \fi
%     \else
%       \usebox0
%     \fi
%   \else
%     \lwbox
%   \fi
% \else
%   \usebox0
% \fi
% \end{quote}
% If you have a \xfile{docstrip.cfg} that configures and enables \docstrip's
% TDS installing feature, then some files can already be in the right
% place, see the documentation of \docstrip.
%
% \subsection{Refresh file name databases}
%
% If your \TeX~distribution
% (\teTeX, \mikTeX, \dots) relies on file name databases, you must refresh
% these. For example, \teTeX\ users run \verb|texhash| or
% \verb|mktexlsr|.
%
% \subsection{Some details for the interested}
%
% \paragraph{Attached source.}
%
% The PDF documentation on CTAN also includes the
% \xfile{.dtx} source file. It can be extracted by
% AcrobatReader 6 or higher. Another option is \textsf{pdftk},
% e.g. unpack the file into the current directory:
% \begin{quote}
%   \verb|pdftk refcount.pdf unpack_files output .|
% \end{quote}
%
% \paragraph{Unpacking with \LaTeX.}
% The \xfile{.dtx} chooses its action depending on the format:
% \begin{description}
% \item[\plainTeX:] Run \docstrip\ and extract the files.
% \item[\LaTeX:] Generate the documentation.
% \end{description}
% If you insist on using \LaTeX\ for \docstrip\ (really,
% \docstrip\ does not need \LaTeX), then inform the autodetect routine
% about your intention:
% \begin{quote}
%   \verb|latex \let\install=y\input{refcount.dtx}|
% \end{quote}
% Do not forget to quote the argument according to the demands
% of your shell.
%
% \paragraph{Generating the documentation.}
% You can use both the \xfile{.dtx} or the \xfile{.drv} to generate
% the documentation. The process can be configured by the
% configuration file \xfile{ltxdoc.cfg}. For instance, put this
% line into this file, if you want to have A4 as paper format:
% \begin{quote}
%   \verb|\PassOptionsToClass{a4paper}{article}|
% \end{quote}
% An example follows how to generate the
% documentation with pdf\LaTeX:
% \begin{quote}
%\begin{verbatim}
%pdflatex refcount.dtx
%makeindex -s gind.ist refcount.idx
%pdflatex refcount.dtx
%makeindex -s gind.ist refcount.idx
%pdflatex refcount.dtx
%\end{verbatim}
% \end{quote}
%
% \section{Catalogue}
%
% The following XML file can be used as source for the
% \href{http://mirror.ctan.org/help/Catalogue/catalogue.html}{\TeX\ Catalogue}.
% The elements \texttt{caption} and \texttt{description} are imported
% from the original XML file from the Catalogue.
% The name of the XML file in the Catalogue is \xfile{refcount.xml}.
%    \begin{macrocode}
%<*catalogue>
<?xml version='1.0' encoding='us-ascii'?>
<!DOCTYPE entry SYSTEM 'catalogue.dtd'>
<entry datestamp='$Date$' modifier='$Author$' id='refcount'>
  <name>refcount</name>
  <caption>Counter operations with label references.</caption>
  <authorref id='auth:oberdiek'/>
  <copyright owner='Heiko Oberdiek' year='1998,2000,2006,2008,2010,2011'/>
  <license type='lppl1.3'/>
  <version number='3.4'/>
  <description>
    Provides commands <tt>\setcounterref</tt> and
    <tt>\addtocounterref</tt> which use the section (or whatever)
    number from the reference as the value to put into the counter, as
    in:

    <pre>
    ...\label{sec:foo}
    ...
    \setcounterref{foonum}{sec:foo}
    </pre>
    Commands <tt>\setcounterpageref</tt> and
    <tt>\addtocounterpageref</tt> do the corresponding thing with the
    page reference of the label.
    <p/>
    No <tt>.ins</tt> file is distributed; process the
    <tt>.dtx</tt> with plain TeX to create one.
    <p/>
    The package is part of the <xref refid='oberdiek'>oberdiek</xref>
    bundle.
  </description>
  <documentation details='Package documentation'
      href='ctan:/macros/latex/contrib/oberdiek/refcount.pdf'/>
  <ctan file='true' path='/macros/latex/contrib/oberdiek/refcount.dtx'/>
  <miktex location='oberdiek'/>
  <texlive location='oberdiek'/>
  <install path='/macros/latex/contrib/oberdiek/oberdiek.tds.zip'/>
</entry>
%</catalogue>
%    \end{macrocode}
%
% \begin{History}
%   \begin{Version}{1998/04/08 v1.0}
%   \item
%     First public release, written as answer in the
%     newsgroup \xnewsgroup{comp.text.tex}:
%     \URL{``\link{Re: Adding a \cs{ref} to a counter?}''}^^A
%     {http://groups.google.com/group/comp.text.tex/msg/c3f2a135ef5ee528}
%   \end{Version}
%   \begin{Version}{2000/09/07 v2.0}
%   \item
%     Documentation added.
%   \item
%     LPPL 1.2
%   \item
%     Package rewritten, new commands added.
%   \end{Version}
%   \begin{Version}{2006/02/20 v3.0}
%   \item
%     Support for \xpackage{hyperref} and \xpackage{nameref} improved.
%   \item
%     Support for \xpackage{titleref} and \xpackage{babel}'s shorthands added.
%   \item
%     New: \cs{refused}, \cs{getrefbykeydefault}
%   \end{Version}
%   \begin{Version}{2008/08/11 v3.1}
%   \item
%     Code is not changed.
%   \item
%     URLs updated.
%   \end{Version}
%   \begin{Version}{2010/12/01 v3.2}
%   \item
%     \cs{IfRefUndefinedExpandable} and \cs{IfRefUndefinedBabel} added.
%   \item
%     \cs{getrefnumber}, \cs{getpagerefnumber}, \cs{getrefbykeydefault}
%     are expandable in exact two expansion steps.
%   \item
%     Non-expandable macros are made robust.
%   \item
%     Test files added.
%   \end{Version}
%   \begin{Version}{2011/06/22 v3.3}
%   \item
%     Bug fix: \cs{rc@refused} is undefined for \cs{setcounterpageref}
%     and similar macros. (Bug found by Marc van Dongen.)
%   \end{Version}
%   \begin{Version}{2011/10/16 v3.4}
%   \item
%     Bug fix: \cs{setcounterpageref} and \cs{addtocounterpageref} fixed.
%     (Bug found by Staz.)
%   \item
%     Macros \cs{(set|addto)counter(page|)ref} are made robust.
%   \end{Version}
% \end{History}
%
% \PrintIndex
%
% \Finale
\endinput
|
% \end{quote}
% Do not forget to quote the argument according to the demands
% of your shell.
%
% \paragraph{Generating the documentation.}
% You can use both the \xfile{.dtx} or the \xfile{.drv} to generate
% the documentation. The process can be configured by the
% configuration file \xfile{ltxdoc.cfg}. For instance, put this
% line into this file, if you want to have A4 as paper format:
% \begin{quote}
%   \verb|\PassOptionsToClass{a4paper}{article}|
% \end{quote}
% An example follows how to generate the
% documentation with pdf\LaTeX:
% \begin{quote}
%\begin{verbatim}
%pdflatex refcount.dtx
%makeindex -s gind.ist refcount.idx
%pdflatex refcount.dtx
%makeindex -s gind.ist refcount.idx
%pdflatex refcount.dtx
%\end{verbatim}
% \end{quote}
%
% \section{Catalogue}
%
% The following XML file can be used as source for the
% \href{http://mirror.ctan.org/help/Catalogue/catalogue.html}{\TeX\ Catalogue}.
% The elements \texttt{caption} and \texttt{description} are imported
% from the original XML file from the Catalogue.
% The name of the XML file in the Catalogue is \xfile{refcount.xml}.
%    \begin{macrocode}
%<*catalogue>
<?xml version='1.0' encoding='us-ascii'?>
<!DOCTYPE entry SYSTEM 'catalogue.dtd'>
<entry datestamp='$Date$' modifier='$Author$' id='refcount'>
  <name>refcount</name>
  <caption>Counter operations with label references.</caption>
  <authorref id='auth:oberdiek'/>
  <copyright owner='Heiko Oberdiek' year='1998,2000,2006,2008,2010,2011'/>
  <license type='lppl1.3'/>
  <version number='3.4'/>
  <description>
    Provides commands <tt>\setcounterref</tt> and
    <tt>\addtocounterref</tt> which use the section (or whatever)
    number from the reference as the value to put into the counter, as
    in:

    <pre>
    ...\label{sec:foo}
    ...
    \setcounterref{foonum}{sec:foo}
    </pre>
    Commands <tt>\setcounterpageref</tt> and
    <tt>\addtocounterpageref</tt> do the corresponding thing with the
    page reference of the label.
    <p/>
    No <tt>.ins</tt> file is distributed; process the
    <tt>.dtx</tt> with plain TeX to create one.
    <p/>
    The package is part of the <xref refid='oberdiek'>oberdiek</xref>
    bundle.
  </description>
  <documentation details='Package documentation'
      href='ctan:/macros/latex/contrib/oberdiek/refcount.pdf'/>
  <ctan file='true' path='/macros/latex/contrib/oberdiek/refcount.dtx'/>
  <miktex location='oberdiek'/>
  <texlive location='oberdiek'/>
  <install path='/macros/latex/contrib/oberdiek/oberdiek.tds.zip'/>
</entry>
%</catalogue>
%    \end{macrocode}
%
% \begin{History}
%   \begin{Version}{1998/04/08 v1.0}
%   \item
%     First public release, written as answer in the
%     newsgroup \xnewsgroup{comp.text.tex}:
%     \URL{``\link{Re: Adding a \cs{ref} to a counter?}''}^^A
%     {http://groups.google.com/group/comp.text.tex/msg/c3f2a135ef5ee528}
%   \end{Version}
%   \begin{Version}{2000/09/07 v2.0}
%   \item
%     Documentation added.
%   \item
%     LPPL 1.2
%   \item
%     Package rewritten, new commands added.
%   \end{Version}
%   \begin{Version}{2006/02/20 v3.0}
%   \item
%     Support for \xpackage{hyperref} and \xpackage{nameref} improved.
%   \item
%     Support for \xpackage{titleref} and \xpackage{babel}'s shorthands added.
%   \item
%     New: \cs{refused}, \cs{getrefbykeydefault}
%   \end{Version}
%   \begin{Version}{2008/08/11 v3.1}
%   \item
%     Code is not changed.
%   \item
%     URLs updated.
%   \end{Version}
%   \begin{Version}{2010/12/01 v3.2}
%   \item
%     \cs{IfRefUndefinedExpandable} and \cs{IfRefUndefinedBabel} added.
%   \item
%     \cs{getrefnumber}, \cs{getpagerefnumber}, \cs{getrefbykeydefault}
%     are expandable in exact two expansion steps.
%   \item
%     Non-expandable macros are made robust.
%   \item
%     Test files added.
%   \end{Version}
%   \begin{Version}{2011/06/22 v3.3}
%   \item
%     Bug fix: \cs{rc@refused} is undefined for \cs{setcounterpageref}
%     and similar macros. (Bug found by Marc van Dongen.)
%   \end{Version}
%   \begin{Version}{2011/10/16 v3.4}
%   \item
%     Bug fix: \cs{setcounterpageref} and \cs{addtocounterpageref} fixed.
%     (Bug found by Staz.)
%   \item
%     Macros \cs{(set|addto)counter(page|)ref} are made robust.
%   \end{Version}
% \end{History}
%
% \PrintIndex
%
% \Finale
\endinput
|
% \end{quote}
% Do not forget to quote the argument according to the demands
% of your shell.
%
% \paragraph{Generating the documentation.}
% You can use both the \xfile{.dtx} or the \xfile{.drv} to generate
% the documentation. The process can be configured by the
% configuration file \xfile{ltxdoc.cfg}. For instance, put this
% line into this file, if you want to have A4 as paper format:
% \begin{quote}
%   \verb|\PassOptionsToClass{a4paper}{article}|
% \end{quote}
% An example follows how to generate the
% documentation with pdf\LaTeX:
% \begin{quote}
%\begin{verbatim}
%pdflatex refcount.dtx
%makeindex -s gind.ist refcount.idx
%pdflatex refcount.dtx
%makeindex -s gind.ist refcount.idx
%pdflatex refcount.dtx
%\end{verbatim}
% \end{quote}
%
% \section{Catalogue}
%
% The following XML file can be used as source for the
% \href{http://mirror.ctan.org/help/Catalogue/catalogue.html}{\TeX\ Catalogue}.
% The elements \texttt{caption} and \texttt{description} are imported
% from the original XML file from the Catalogue.
% The name of the XML file in the Catalogue is \xfile{refcount.xml}.
%    \begin{macrocode}
%<*catalogue>
<?xml version='1.0' encoding='us-ascii'?>
<!DOCTYPE entry SYSTEM 'catalogue.dtd'>
<entry datestamp='$Date$' modifier='$Author$' id='refcount'>
  <name>refcount</name>
  <caption>Counter operations with label references.</caption>
  <authorref id='auth:oberdiek'/>
  <copyright owner='Heiko Oberdiek' year='1998,2000,2006,2008,2010,2011'/>
  <license type='lppl1.3'/>
  <version number='3.4'/>
  <description>
    Provides commands <tt>\setcounterref</tt> and
    <tt>\addtocounterref</tt> which use the section (or whatever)
    number from the reference as the value to put into the counter, as
    in:

    <pre>
    ...\label{sec:foo}
    ...
    \setcounterref{foonum}{sec:foo}
    </pre>
    Commands <tt>\setcounterpageref</tt> and
    <tt>\addtocounterpageref</tt> do the corresponding thing with the
    page reference of the label.
    <p/>
    No <tt>.ins</tt> file is distributed; process the
    <tt>.dtx</tt> with plain TeX to create one.
    <p/>
    The package is part of the <xref refid='oberdiek'>oberdiek</xref>
    bundle.
  </description>
  <documentation details='Package documentation'
      href='ctan:/macros/latex/contrib/oberdiek/refcount.pdf'/>
  <ctan file='true' path='/macros/latex/contrib/oberdiek/refcount.dtx'/>
  <miktex location='oberdiek'/>
  <texlive location='oberdiek'/>
  <install path='/macros/latex/contrib/oberdiek/oberdiek.tds.zip'/>
</entry>
%</catalogue>
%    \end{macrocode}
%
% \begin{History}
%   \begin{Version}{1998/04/08 v1.0}
%   \item
%     First public release, written as answer in the
%     newsgroup \xnewsgroup{comp.text.tex}:
%     \URL{``\link{Re: Adding a \cs{ref} to a counter?}''}^^A
%     {http://groups.google.com/group/comp.text.tex/msg/c3f2a135ef5ee528}
%   \end{Version}
%   \begin{Version}{2000/09/07 v2.0}
%   \item
%     Documentation added.
%   \item
%     LPPL 1.2
%   \item
%     Package rewritten, new commands added.
%   \end{Version}
%   \begin{Version}{2006/02/20 v3.0}
%   \item
%     Support for \xpackage{hyperref} and \xpackage{nameref} improved.
%   \item
%     Support for \xpackage{titleref} and \xpackage{babel}'s shorthands added.
%   \item
%     New: \cs{refused}, \cs{getrefbykeydefault}
%   \end{Version}
%   \begin{Version}{2008/08/11 v3.1}
%   \item
%     Code is not changed.
%   \item
%     URLs updated.
%   \end{Version}
%   \begin{Version}{2010/12/01 v3.2}
%   \item
%     \cs{IfRefUndefinedExpandable} and \cs{IfRefUndefinedBabel} added.
%   \item
%     \cs{getrefnumber}, \cs{getpagerefnumber}, \cs{getrefbykeydefault}
%     are expandable in exact two expansion steps.
%   \item
%     Non-expandable macros are made robust.
%   \item
%     Test files added.
%   \end{Version}
%   \begin{Version}{2011/06/22 v3.3}
%   \item
%     Bug fix: \cs{rc@refused} is undefined for \cs{setcounterpageref}
%     and similar macros. (Bug found by Marc van Dongen.)
%   \end{Version}
%   \begin{Version}{2011/10/16 v3.4}
%   \item
%     Bug fix: \cs{setcounterpageref} and \cs{addtocounterpageref} fixed.
%     (Bug found by Staz.)
%   \item
%     Macros \cs{(set|addto)counter(page|)ref} are made robust.
%   \end{Version}
% \end{History}
%
% \PrintIndex
%
% \Finale
\endinput

%        (quote the arguments according to the demands of your shell)
%
% Documentation:
%    (a) If refcount.drv is present:
%           latex refcount.drv
%    (b) Without refcount.drv:
%           latex refcount.dtx; ...
%    The class ltxdoc loads the configuration file ltxdoc.cfg
%    if available. Here you can specify further options, e.g.
%    use A4 as paper format:
%       \PassOptionsToClass{a4paper}{article}
%
%    Programm calls to get the documentation (example):
%       pdflatex refcount.dtx
%       makeindex -s gind.ist refcount.idx
%       pdflatex refcount.dtx
%       makeindex -s gind.ist refcount.idx
%       pdflatex refcount.dtx
%
% Installation:
%    TDS:tex/latex/oberdiek/refcount.sty
%    TDS:doc/latex/oberdiek/refcount.pdf
%    TDS:doc/latex/oberdiek/test/refcount-test1.tex
%    TDS:doc/latex/oberdiek/test/refcount-test2.tex
%    TDS:doc/latex/oberdiek/test/refcount-test3.tex
%    TDS:doc/latex/oberdiek/test/refcount-test4.tex
%    TDS:doc/latex/oberdiek/test/refcount-test5.tex
%    TDS:source/latex/oberdiek/refcount.dtx
%
%<*ignore>
\begingroup
  \catcode123=1 %
  \catcode125=2 %
  \def\x{LaTeX2e}%
\expandafter\endgroup
\ifcase 0\ifx\install y1\fi\expandafter
         \ifx\csname processbatchFile\endcsname\relax\else1\fi
         \ifx\fmtname\x\else 1\fi\relax
\else\csname fi\endcsname
%</ignore>
%<*install>
\input docstrip.tex
\Msg{************************************************************************}
\Msg{* Installation}
\Msg{* Package: refcount 2011/10/16 v3.4 Data extraction from label references (HO)}
\Msg{************************************************************************}

\keepsilent
\askforoverwritefalse

\let\MetaPrefix\relax
\preamble

This is a generated file.

Project: refcount
Version: 2011/10/16 v3.4

Copyright (C) 1998, 2000, 2006, 2008, 2010, 2011 by
   Heiko Oberdiek <heiko.oberdiek at googlemail.com>

This work may be distributed and/or modified under the
conditions of the LaTeX Project Public License, either
version 1.3c of this license or (at your option) any later
version. This version of this license is in
   http://www.latex-project.org/lppl/lppl-1-3c.txt
and the latest version of this license is in
   http://www.latex-project.org/lppl.txt
and version 1.3 or later is part of all distributions of
LaTeX version 2005/12/01 or later.

This work has the LPPL maintenance status "maintained".

This Current Maintainer of this work is Heiko Oberdiek.

This work consists of the main source file refcount.dtx
and the derived files
   refcount.sty, refcount.pdf, refcount.ins, refcount.drv,
   refcount-test1.tex, refcount-test2.tex, refcount-test3.tex,
   refcount-test4.tex, refcount-test5.tex.

\endpreamble
\let\MetaPrefix\DoubleperCent

\generate{%
  \file{refcount.ins}{\from{refcount.dtx}{install}}%
  \file{refcount.drv}{\from{refcount.dtx}{driver}}%
  \usedir{tex/latex/oberdiek}%
  \file{refcount.sty}{\from{refcount.dtx}{package}}%
  \usedir{doc/latex/oberdiek/test}%
  \file{refcount-test1.tex}{\from{refcount.dtx}{test1}}%
  \file{refcount-test2.tex}{\from{refcount.dtx}{test2}}%
  \file{refcount-test3.tex}{\from{refcount.dtx}{test3}}%
  \file{refcount-test4.tex}{\from{refcount.dtx}{test3,test4}}%
  \file{refcount-test5.tex}{\from{refcount.dtx}{test5}}%
  \nopreamble
  \nopostamble
  \usedir{source/latex/oberdiek/catalogue}%
  \file{refcount.xml}{\from{refcount.dtx}{catalogue}}%
}

\catcode32=13\relax% active space
\let =\space%
\Msg{************************************************************************}
\Msg{*}
\Msg{* To finish the installation you have to move the following}
\Msg{* file into a directory searched by TeX:}
\Msg{*}
\Msg{*     refcount.sty}
\Msg{*}
\Msg{* To produce the documentation run the file `refcount.drv'}
\Msg{* through LaTeX.}
\Msg{*}
\Msg{* Happy TeXing!}
\Msg{*}
\Msg{************************************************************************}

\endbatchfile
%</install>
%<*ignore>
\fi
%</ignore>
%<*driver>
\NeedsTeXFormat{LaTeX2e}
\ProvidesFile{refcount.drv}%
  [2011/10/16 v3.4 Data extraction from label references (HO)]%
\documentclass{ltxdoc}
\usepackage{holtxdoc}[2011/11/22]
\begin{document}
  \DocInput{refcount.dtx}%
\end{document}
%</driver>
% \fi
%
% \CheckSum{1185}
%
% \CharacterTable
%  {Upper-case    \A\B\C\D\E\F\G\H\I\J\K\L\M\N\O\P\Q\R\S\T\U\V\W\X\Y\Z
%   Lower-case    \a\b\c\d\e\f\g\h\i\j\k\l\m\n\o\p\q\r\s\t\u\v\w\x\y\z
%   Digits        \0\1\2\3\4\5\6\7\8\9
%   Exclamation   \!     Double quote  \"     Hash (number) \#
%   Dollar        \$     Percent       \%     Ampersand     \&
%   Acute accent  \'     Left paren    \(     Right paren   \)
%   Asterisk      \*     Plus          \+     Comma         \,
%   Minus         \-     Point         \.     Solidus       \/
%   Colon         \:     Semicolon     \;     Less than     \<
%   Equals        \=     Greater than  \>     Question mark \?
%   Commercial at \@     Left bracket  \[     Backslash     \\
%   Right bracket \]     Circumflex    \^     Underscore    \_
%   Grave accent  \`     Left brace    \{     Vertical bar  \|
%   Right brace   \}     Tilde         \~}
%
% \GetFileInfo{refcount.drv}
%
% \title{The \xpackage{refcount} package}
% \date{2011/10/16 v3.4}
% \author{Heiko Oberdiek\\\xemail{heiko.oberdiek at googlemail.com}}
%
% \maketitle
%
% \begin{abstract}
% References are not numbers, however they often store numerical
% data such as section or page numbers. \cs{ref} or \cs{pageref}
% cannot be used for counter assignments or calculations because
% they are not expandable, generate warnings, or can even be links.
% The package provides expandable macros to extract the data
% from references. Packages \xpackage{hyperref}, \xpackage{nameref},
% \xpackage{titleref}, and \xpackage{babel} are supported.
% \end{abstract}
%
% \tableofcontents
%
% \section{Usage}
%
% \subsection{Setting counters}
%
% The following commands are similar to \LaTeX's
% \cs{setcounter} and \cs{addtocounter},
% but they extract the number value from a reference:
% \begin{quote}
%   \cs{setcounterref}, \cs{addtocounterref}\\
%   \cs{setcounterpageref}, \cs{addtocounterpageref}
% \end{quote}
% They take two arguments:
% \begin{quote}
%    \cs{...counter...ref} |{|\meta{\LaTeX\ counter}|}|
%    |{|\meta{reference}|}|
% \end{quote}
% An undefined references produces the usual LaTeX warning
% and its value is assumed to be zero.
% Example:
% \begin{quote}
%\begin{verbatim}
%\newcounter{ctrA}
%\newcounter{ctrB}
%\refstepcounter{ctrA}\label{ref:A}
%\setcounterref{ctrB}{ref:A}
%\addtocounterpageref{ctrB}{ref:A}
%\end{verbatim}
% \end{quote}
%
% \subsection{Expandable commands}
%
% These commands that can be used in expandible contexts
% (inside calculations, \cs{edef}, \cs{csname}, \cs{write}, \dots):
% \begin{quote}
%   \cs{getrefnumber}, \cs{getpagerefnumber}
% \end{quote}
% They take one argument, the reference:
% \begin{quote}
%   \cs{get...refnumber} |{|\meta{reference}|}|
% \end{quote}
% The default for undefined references can be changed
% with macro \cs{setrefcountdefault}, for example this
% package calls:
% \begin{quote}
%   \cs{setrefcountdefault}|{0}|
% \end{quote}
%
% Since version 2.0 of this package there is a new
% command:
% \begin{quote}
%   \cs{getrefbykeydefault} |{|\meta{reference}|}|
%   |{|\meta{key}|}| |{|\meta{default}|}|
% \end{quote}
% This generalized version allows the extraction
% of further properties of a reference than the
% two standard ones. Thus the following properties
% are supported, if they are available:
% \begin{quote}
% \begin{tabular}{@{}l|l|l@{}}
%    Key & Description & Package\\
% \hline
%   \meta{empty} & same as \cs{ref} & \LaTeX\\
%   |page| & same as \cs{pageref} & \LaTeX\\
%   |title| & section and caption titles & \xpackage{titleref}\\
%   |name| & section and caption titles & \xpackage{nameref}\\
%   |anchor| & anchor name & \xpackage{hyperref}\\
%   |url| & url/file & \xpackage{hyperref}/\xpackage{xr}
% \end{tabular}
% \end{quote}
%
% Since version 3.2 the expandable macros described before
% in this section
% are expandable in exact two expansion steps.
%
% \subsection{Undefined references}
%
% Because warnings and assignments cannot be used in
% expandible contexts, undefined references do not
% produce a warning, their values are assumed to be zero.
% Example:
% \begin{quote}
%\begin{verbatim}
%\label{ref:here}% somewhere
%\refused{ref:here}% see below
%\ifodd\getpagerefnumber{ref:here}%
%  reference is on an odd page
%\else
%  reference is on an even page
%\fi
%\end{verbatim}
% \end{quote}
%
% In case of undefined references the user usually want's
% to be informed. Also \LaTeX\ prints a warning at
% the end of the \LaTeX\ run. To notify \LaTeX\ and
% get a normal warning, just use
% \begin{quote}
%   \cs{refused} |{|\meta{reference}|}|
% \end{quote}
% outside the expanding context. Example, see above.
%
% \subsubsection{Check for undefined references}
%
% In version 3.2 macros were added, that test, whether references
% are defined.
% \begin{declcs}{IfRefUndefinedExpandable} \M{refname} \M{then} \M{else}\\
%   \cs{IfRefUndefinedBabel} \M{refname} \M{then} \M{else}
% \end{declcs}
% If the reference is not available and therefore undefined, then
% argument \meta{then} is executed, otherwise argument \meta{else}
% is called. Macro \cs{IfRefUndefinedExpandable} is expandable,
% but \meta{refname} must not contain babel shorthand characters.
% Macro \cs{IfRefUndefinedBabel} supports shorthand characters of
% babel, but it is not expandable.
%
% \subsection{Notes}
%
% \begin{itemize}
% \item
%   The method of extracting the number in this
%   package also works in cases, where the
%   reference cannot be used directly, because
%   a package such as \xpackage{hyperref} has added
%   extra stuff (hyper link), so that the reference cannot
%   be used as number any more.
% \item
%   If the reference does not contain a number,
%   assignments to a counter will fail of course.
% \end{itemize}
%
%
% \StopEventually{
% }
%
% \section{Implementation}
%
%    \begin{macrocode}
%<*package>
%    \end{macrocode}
%    Reload check, especially if the package is not used with \LaTeX.
%    \begin{macrocode}
\begingroup\catcode61\catcode48\catcode32=10\relax%
  \catcode13=5 % ^^M
  \endlinechar=13 %
  \catcode35=6 % #
  \catcode39=12 % '
  \catcode44=12 % ,
  \catcode45=12 % -
  \catcode46=12 % .
  \catcode58=12 % :
  \catcode64=11 % @
  \catcode123=1 % {
  \catcode125=2 % }
  \expandafter\let\expandafter\x\csname ver@refcount.sty\endcsname
  \ifx\x\relax % plain-TeX, first loading
  \else
    \def\empty{}%
    \ifx\x\empty % LaTeX, first loading,
      % variable is initialized, but \ProvidesPackage not yet seen
    \else
      \expandafter\ifx\csname PackageInfo\endcsname\relax
        \def\x#1#2{%
          \immediate\write-1{Package #1 Info: #2.}%
        }%
      \else
        \def\x#1#2{\PackageInfo{#1}{#2, stopped}}%
      \fi
      \x{refcount}{The package is already loaded}%
      \aftergroup\endinput
    \fi
  \fi
\endgroup%
%    \end{macrocode}
%    Package identification:
%    \begin{macrocode}
\begingroup\catcode61\catcode48\catcode32=10\relax%
  \catcode13=5 % ^^M
  \endlinechar=13 %
  \catcode35=6 % #
  \catcode39=12 % '
  \catcode40=12 % (
  \catcode41=12 % )
  \catcode44=12 % ,
  \catcode45=12 % -
  \catcode46=12 % .
  \catcode47=12 % /
  \catcode58=12 % :
  \catcode64=11 % @
  \catcode91=12 % [
  \catcode93=12 % ]
  \catcode123=1 % {
  \catcode125=2 % }
  \expandafter\ifx\csname ProvidesPackage\endcsname\relax
    \def\x#1#2#3[#4]{\endgroup
      \immediate\write-1{Package: #3 #4}%
      \xdef#1{#4}%
    }%
  \else
    \def\x#1#2[#3]{\endgroup
      #2[{#3}]%
      \ifx#1\@undefined
        \xdef#1{#3}%
      \fi
      \ifx#1\relax
        \xdef#1{#3}%
      \fi
    }%
  \fi
\expandafter\x\csname ver@refcount.sty\endcsname
\ProvidesPackage{refcount}%
  [2011/10/16 v3.4 Data extraction from label references (HO)]%
%    \end{macrocode}
%
%    \begin{macrocode}
\begingroup\catcode61\catcode48\catcode32=10\relax%
  \catcode13=5 % ^^M
  \endlinechar=13 %
  \catcode123=1 % {
  \catcode125=2 % }
  \catcode64=11 % @
  \def\x{\endgroup
    \expandafter\edef\csname rc@AtEnd\endcsname{%
      \endlinechar=\the\endlinechar\relax
      \catcode13=\the\catcode13\relax
      \catcode32=\the\catcode32\relax
      \catcode35=\the\catcode35\relax
      \catcode61=\the\catcode61\relax
      \catcode64=\the\catcode64\relax
      \catcode123=\the\catcode123\relax
      \catcode125=\the\catcode125\relax
    }%
  }%
\x\catcode61\catcode48\catcode32=10\relax%
\catcode13=5 % ^^M
\endlinechar=13 %
\catcode35=6 % #
\catcode64=11 % @
\catcode123=1 % {
\catcode125=2 % }
\def\TMP@EnsureCode#1#2{%
  \edef\rc@AtEnd{%
    \rc@AtEnd
    \catcode#1=\the\catcode#1\relax
  }%
  \catcode#1=#2\relax
}
\TMP@EnsureCode{33}{12}% !
\TMP@EnsureCode{39}{12}% '
\TMP@EnsureCode{42}{12}% *
\TMP@EnsureCode{45}{12}% -
\TMP@EnsureCode{46}{12}% .
\TMP@EnsureCode{47}{12}% /
\TMP@EnsureCode{91}{12}% [
\TMP@EnsureCode{93}{12}% ]
\TMP@EnsureCode{96}{12}% `
\edef\rc@AtEnd{\rc@AtEnd\noexpand\endinput}
%    \end{macrocode}
%
% \subsection{Loading packages}
%
%    \begin{macrocode}
\begingroup\expandafter\expandafter\expandafter\endgroup
\expandafter\ifx\csname RequirePackage\endcsname\relax
  \input ltxcmds.sty\relax
  \input infwarerr.sty\relax
\else
  \RequirePackage{ltxcmds}[2011/11/09]%
  \RequirePackage{infwarerr}[2010/04/08]%
\fi
%    \end{macrocode}
%
% \subsection{Defining commands}
%
%    \begin{macro}{\rc@IfDefinable}
%    \begin{macrocode}
\ltx@IfUndefined{@ifdefinable}{%
  \def\rc@IfDefinable#1{%
    \ifx#1\ltx@undefined
      \expandafter\ltx@firstofone
    \else
      \ifx#1\relax
        \expandafter\expandafter\expandafter\ltx@firstofone
      \else
        \@PackageError{refcount}{%
          Command \string#1 is already defined.\MessageBreak
          It will not redefined by this package%
        }\@ehc
        \expandafter\expandafter\expandafter\ltx@gobble
      \fi
    \fi
  }%
}{%
  \let\rc@IfDefinable\@ifdefinable
}
%    \end{macrocode}
%    \end{macro}
%
%    \begin{macro}{\rc@RobustDefOne}
%    \begin{macro}{\rc@RobustDefZero}
%    \begin{macrocode}
\ltx@IfUndefined{protected}{%
  \ltx@IfUndefined{DeclareRobustCommand}{%
    \def\rc@RobustDefOne#1#2#3#4{%
      \rc@IfDefinable#3{%
        #1\def#3##1{#4}%
      }%
    }%
    \def\rc@RobustDefZero#1#2{%
      \rc@IfDefinable#1{%
        \def#1{#2}%
      }%
    }%
  }{%
    \def\rc@RobustDefOne#1#2#3#4{%
      \rc@IfDefinable#3{%
        \DeclareRobustCommand#2#3[1]{#4}%
      }%
    }%
    \def\rc@RobustDefZero#1#2{%
      \rc@IfDefinable#1{%
        \DeclareRobustCommand#1{#2}%
      }%
    }%
  }%
}{%
  \def\rc@RobustDefOne#1#2#3#4{%
    \rc@IfDefinable#3{%
      \protected#1\def#3##1{#4}%
    }%
  }%
  \def\rc@RobustDefZero#1#2{%
    \rc@IfDefinable#1{%
      \protected\def#1{#2}%
    }%
  }%
}
%    \end{macrocode}
%    \end{macro}
%    \end{macro}
%
%    \begin{macro}{\rc@newcommand}
%    \begin{macrocode}
\ltx@IfUndefined{newcommand}{%
  \def\rc@newcommand*#1[#2]#3{% hash-ok
    \rc@IfDefinable#1{%
      \ifcase#2 %
        \def#1{#3}%
      \or
        \def#1##1{#3}%
      \or
        \def#1##1##2{#3}%
      \else
        \rc@InternalError
      \fi
    }%
  }%
}{%
  \let\rc@newcommand\newcommand
}
%    \end{macrocode}
%    \end{macro}
%
% \subsection{\cs{setrefcountdefault}}
%
%    \begin{macro}{\setrefcountdefault}
%    \begin{macrocode}
\rc@RobustDefOne\long{}\setrefcountdefault{%
  \def\rc@default{#1}%
}
%    \end{macrocode}
%    \end{macro}
%    \begin{macrocode}
\setrefcountdefault{0}
%    \end{macrocode}
%
% \subsection{\cs{refused}}
%
%    \begin{macro}{\refused}
%    \begin{macrocode}
\ltx@IfUndefined{G@refundefinedtrue}{%
  \rc@RobustDefOne{}{*}\refused{%
    \begingroup
      \csname @safe@activestrue\endcsname
      \ltx@IfUndefined{r@#1}{%
        \protect\G@refundefinedtrue
        \rc@WarningUndefined{#1}%
      }{}%
    \endgroup
  }%
}{%
  \rc@RobustDefOne{}{*}\refused{%
    \begingroup
      \csname @safe@activestrue\endcsname
      \ltx@IfUndefined{r@#1}{%
        \csname protect\expandafter\endcsname
        \csname G@refundefinedtrue\endcsname
        \rc@WarningUndefined{#1}%
      }{}%
    \endgroup
  }%
}
%    \end{macrocode}
%    \end{macro}
%    \begin{macro}{\rc@WarningUndefined}
%    \begin{macrocode}
\ltx@IfUndefined{@latex@warning}{%
  \def\rc@WarningUndefined#1{%
    \ltx@ifundefined{thepage}{%
      \def\thepage{\number\count0 }%
    }{}%
    \@PackageWarning{refcount}{%
      Reference `#1' on page \thepage\space undefined%
    }%
  }%
}{%
  \def\rc@WarningUndefined#1{%
    \@latex@warning{%
      Reference `#1' on page \thepage\space undefined%
    }%
  }%
}
%    \end{macrocode}
%    \end{macro}
%
% \subsection{Setting counters by reference data}
%
% \subsubsection{Generic setting}
%
%    \begin{macro}{\rc@set}
% Generic command for
% |\|$\{$|set|$,$|addto|$\}$|counter|$\{$|page|$,\}$|ref|:
%\begin{quote}
%\begin{tabular}[t]{@{}l@{: }l@{}}
% |#1|& \cs{setcounter}, \cs{addtocounter}\\
% |#2|& \cs{ltx@car} (for \cs{ref}), \cs{ltx@cartwo} (for \cs{pageref})\\
% |#3|& \hologo{LaTeX} counter\\
% |#4|& reference\\
%\end{tabular}
%\end{quote}
%    \begin{macrocode}
\def\rc@set#1#2#3#4{%
  \begingroup
    \csname @safe@activestrue\endcsname
    \refused{#4}%
    \expandafter\rc@@set\csname r@#4\endcsname{#1}{#2}{#3}%
  \endgroup
}
%    \end{macrocode}
%    \end{macro}
%    \begin{macro}{\rc@@set}
%\begin{quote}
%\begin{tabular}[t]{@{}l@{: }l@{}}
% |#1|& \cs{r@<...>}\\
% |#2|& \cs{setcounter}, \cs{addtocounter}\\
% |#3|& \cs{ltx@car} (for \cs{ref}), \cs{ltx@carsecond} (for \cs{pageref})\\
% |#4|& \hologo{LaTeX} counter\\
%\end{tabular}
%\end{quote}
%    \begin{macrocode}
\def\rc@@set#1#2#3#4{%
  \ifx#1\relax
    #2{#4}{\rc@default}%
  \else
    #2{#4}{%
      \expandafter#3#1\rc@default\rc@default\@nil
    }%
  \fi
}
%    \end{macrocode}
%    \end{macro}
%
% \subsubsection{User commands}
%
%    \begin{macro}{\setcounterref}
%    \begin{macrocode}
\rc@RobustDefZero\setcounterref{%
  \rc@set\setcounter\ltx@car
}
%    \end{macrocode}
%    \end{macro}
%    \begin{macro}{\addtocounterref}
%    \begin{macrocode}
\rc@RobustDefZero\addtocounterref{%
  \rc@set\addtocounter\ltx@car
}
%    \end{macrocode}
%    \end{macro}
%    \begin{macro}{\setcounterpageref}
%    \begin{macrocode}
\rc@RobustDefZero\setcounterpageref{%
  \rc@set\setcounter\ltx@carsecond
}
%    \end{macrocode}
%    \end{macro}
%    \begin{macro}{\addtocounterpageref}
%    \begin{macrocode}
\rc@RobustDefZero\addtocounterpageref{%
  \rc@set\addtocounter\ltx@carsecond
}
%    \end{macrocode}
%    \end{macro}
%
% \subsection{Extracting references}
%
%    \begin{macro}{\getrefnumber}
%    \begin{macrocode}
\rc@newcommand*{\getrefnumber}[1]{%
  \romannumeral
  \ltx@ifundefined{r@#1}{%
    \expandafter\ltx@zero
    \rc@default
  }{%
    \expandafter\expandafter\expandafter\rc@extract@
    \expandafter\expandafter\expandafter!%
    \csname r@#1\expandafter\endcsname
    \expandafter{\rc@default}\@nil
  }%
}
%    \end{macrocode}
%    \end{macro}
%    \begin{macro}{\getpagerefnumber}
%    \begin{macrocode}
\rc@newcommand*{\getpagerefnumber}[1]{%
  \romannumeral
  \ltx@ifundefined{r@#1}{%
    \expandafter\ltx@zero
    \rc@default
  }{%
    \expandafter\expandafter\expandafter\rc@extract@page
    \expandafter\expandafter\expandafter!%
    \csname r@#1\expandafter\expandafter\expandafter\endcsname
    \expandafter\expandafter\expandafter{%
      \expandafter\rc@default
    \expandafter}\expandafter{\rc@default}\@nil
  }%
}
%    \end{macrocode}
%    \end{macro}
%    \begin{macro}{\getrefbykeydefault}
%    \begin{macrocode}
\rc@newcommand*{\getrefbykeydefault}[2]{%
  \romannumeral
  \expandafter\rc@getrefbykeydefault
    \csname r@#1\expandafter\endcsname
    \csname rc@extract@#2\endcsname
}
%    \end{macrocode}
%    \end{macro}
%    \begin{macro}{\rc@getrefbykeydefault}
%\begin{quote}
%\begin{tabular}[t]{@{}l@{: }l@{}}
% |#1|& \cs{r@<...>}\\
% |#2|& \cs{rc@extract@<...>}\\
% |#3|& default\\
%\end{tabular}
%\end{quote}
%    \begin{macrocode}
\long\def\rc@getrefbykeydefault#1#2#3{%
  \ifx#1\relax
    % reference is undefined
    \ltx@ReturnAfterElseFi{%
      \ltx@zero
      #3%
    }%
  \else
    \ltx@ReturnAfterFi{%
      \ifx#2\relax
        % extract method is missing
        \ltx@ReturnAfterElseFi{%
          \ltx@zero
          #3%
        }%
      \else
        \ltx@ReturnAfterFi{%
          \expandafter
          \rc@generic#1{#3}{#3}{#3}{#3}{#3}\@nil#2{#3}%
        }%
      \fi
    }%
  \fi
}
%    \end{macrocode}
%    \end{macro}
%    \begin{macro}{\rc@generic}
%\begin{quote}
%\begin{tabular}[t]{@{}l@{: }l@{}}
% |#1|& first item in \cs{r@<...>}\\
% |#2|& remaining items in \cs{r@<...>}\\
% |#3|& \cs{rc@extract@<...>}\\
% |#4|& default\\
%\end{tabular}
%\end{quote}
%    \begin{macrocode}
\long\def\rc@generic#1#2\@nil#3#4{%
  #3{#1\TR@TitleReference\@empty{#4}\@nil}{#1}#2\@nil
}
%    \end{macrocode}
%    \end{macro}
%    \begin{macro}{\rc@extract@}
%    \begin{macrocode}
\long\def\rc@extract@#1#2#3\@nil{%
  \ltx@zero
  #2%
}
%    \end{macrocode}
%    \end{macro}
%    \begin{macro}{\rc@extract@page}
%    \begin{macrocode}
\long\def\rc@extract@page#1#2#3#4\@nil{%
  \ltx@zero
  #3%
}
%    \end{macrocode}
%    \end{macro}
%    \begin{macro}{\rc@extract@name}
%    \begin{macrocode}
\long\def\rc@extract@name#1#2#3#4#5\@nil{%
  \ltx@zero
  #4%
}
%    \end{macrocode}
%    \end{macro}
%    \begin{macro}{\rc@extract@anchor}
%    \begin{macrocode}
\long\def\rc@extract@anchor#1#2#3#4#5#6\@nil{%
  \ltx@zero
  #5%
}
%    \end{macrocode}
%    \end{macro}
%    \begin{macro}{\rc@extract@url}
%    \begin{macrocode}
\long\def\rc@extract@url#1#2#3#4#5#6#7\@nil{%
  \ltx@zero
  #6%
}
%    \end{macrocode}
%    \end{macro}
%    \begin{macro}{\rc@extract@title}
%    \begin{macrocode}
\long\def\rc@extract@title#1#2\@nil{%
  \rc@@extract@title#1%
}
%    \end{macrocode}
%    \end{macro}
%    \begin{macro}{\rc@@extract@title}
%    \begin{macrocode}
\long\def\rc@@extract@title#1\TR@TitleReference#2#3#4\@nil{%
  \ltx@zero
  #3%
}
%    \end{macrocode}
%    \end{macro}
%
% \subsection{Macros for checking undefined references}
%
%    \begin{macro}{\IfRefUndefinedExpandable}
%    \begin{macrocode}
\rc@newcommand*{\IfRefUndefinedExpandable}[1]{%
  \ltx@ifundefined{r@#1}\ltx@firstoftwo\ltx@secondoftwo
}
%    \end{macrocode}
%    \end{macro}
%    \begin{macro}{\IfRefUndefinedBabel}
%    \begin{macrocode}
\rc@RobustDefOne{}*\IfRefUndefinedBabel{%
  \begingroup
    \csname safe@actives@true\endcsname
  \expandafter\expandafter\expandafter\endgroup
  \expandafter\ifx\csname r@#1\endcsname\relax
    \expandafter\ltx@firstoftwo
  \else
    \expandafter\ltx@secondoftwo
  \fi
}
%    \end{macrocode}
%    \end{macro}
%    \begin{macrocode}
\rc@AtEnd%
%</package>
%    \end{macrocode}
%
% \section{Test}
%
% \subsection{Catcode checks for loading}
%
%    \begin{macrocode}
%<*test1>
%    \end{macrocode}
%    \begin{macrocode}
\catcode`\{=1 %
\catcode`\}=2 %
\catcode`\#=6 %
\catcode`\@=11 %
\expandafter\ifx\csname count@\endcsname\relax
  \countdef\count@=255 %
\fi
\expandafter\ifx\csname @gobble\endcsname\relax
  \long\def\@gobble#1{}%
\fi
\expandafter\ifx\csname @firstofone\endcsname\relax
  \long\def\@firstofone#1{#1}%
\fi
\expandafter\ifx\csname loop\endcsname\relax
  \expandafter\@firstofone
\else
  \expandafter\@gobble
\fi
{%
  \def\loop#1\repeat{%
    \def\body{#1}%
    \iterate
  }%
  \def\iterate{%
    \body
      \let\next\iterate
    \else
      \let\next\relax
    \fi
    \next
  }%
  \let\repeat=\fi
}%
\def\RestoreCatcodes{}
\count@=0 %
\loop
  \edef\RestoreCatcodes{%
    \RestoreCatcodes
    \catcode\the\count@=\the\catcode\count@\relax
  }%
\ifnum\count@<255 %
  \advance\count@ 1 %
\repeat

\def\RangeCatcodeInvalid#1#2{%
  \count@=#1\relax
  \loop
    \catcode\count@=15 %
  \ifnum\count@<#2\relax
    \advance\count@ 1 %
  \repeat
}
\def\RangeCatcodeCheck#1#2#3{%
  \count@=#1\relax
  \loop
    \ifnum#3=\catcode\count@
    \else
      \errmessage{%
        Character \the\count@\space
        with wrong catcode \the\catcode\count@\space
        instead of \number#3%
      }%
    \fi
  \ifnum\count@<#2\relax
    \advance\count@ 1 %
  \repeat
}
\def\space{ }
\expandafter\ifx\csname LoadCommand\endcsname\relax
  \def\LoadCommand{\input refcount.sty\relax}%
\fi
\def\Test{%
  \RangeCatcodeInvalid{0}{47}%
  \RangeCatcodeInvalid{58}{64}%
  \RangeCatcodeInvalid{91}{96}%
  \RangeCatcodeInvalid{123}{255}%
  \catcode`\@=12 %
  \catcode`\\=0 %
  \catcode`\%=14 %
  \LoadCommand
  \RangeCatcodeCheck{0}{36}{15}%
  \RangeCatcodeCheck{37}{37}{14}%
  \RangeCatcodeCheck{38}{47}{15}%
  \RangeCatcodeCheck{48}{57}{12}%
  \RangeCatcodeCheck{58}{63}{15}%
  \RangeCatcodeCheck{64}{64}{12}%
  \RangeCatcodeCheck{65}{90}{11}%
  \RangeCatcodeCheck{91}{91}{15}%
  \RangeCatcodeCheck{92}{92}{0}%
  \RangeCatcodeCheck{93}{96}{15}%
  \RangeCatcodeCheck{97}{122}{11}%
  \RangeCatcodeCheck{123}{255}{15}%
  \RestoreCatcodes
}
\Test
\csname @@end\endcsname
\end
%    \end{macrocode}
%    \begin{macrocode}
%</test1>
%    \end{macrocode}
%
% \subsection{Macro tests}
%
%    \begin{macrocode}
%<*test2>
\errorcontextlines=10000 %
\showboxbreadth=10000 %
\showboxdepth=10000 %
\begingroup\expandafter\expandafter\expandafter\endgroup
\expandafter\ifx\csname RequirePackage\endcsname\relax
  \input refcount.sty\relax
\else
  \RequirePackage{refcount}[2011/10/16]%
\fi
\catcode`\@=11 %
\begingroup\expandafter\expandafter\expandafter\endgroup
\expandafter\ifx\csname @onelevel@sanitize\endcsname\relax
  \begingroup\expandafter\expandafter\expandafter\endgroup
  \expandafter\ifx\csname detokenize\endcsname\relax
    \def\strip@prefix#1->{}%
    \def\@onelevel@sanitize#1{%
      \edef#1{%
        \expandafter\strip@prefix\meaning#1%
      }%
    }%
  \else
    \def\@onelevel@sanitize#1{%
      \edef#1{%
        \detokenize\expandafter{#1}%
      }%
    }%
  \fi
\fi
\def\msg#{\immediate\write16}
\def\empty{}
\def\space{ }
%    \end{macrocode}
%    \begin{macrocode}
\def\r@foo{{\empty 1}{\empty 2}}
\long\def\test#1#2{%
  \begingroup
    \setbox0=\hbox{%
      \def\TestTask{#1}%
      \@onelevel@sanitize\TestTask
      \msg{* \TestTask}%
      \expandafter\expandafter\expandafter\def
      \expandafter\expandafter\expandafter\TestResult
      \expandafter\expandafter\expandafter{%
        #1%
      }%
      \def\TestExpected{#2}%
      \ifx\TestResult\TestExpected
        \msg{ \space ok.}%
      \else
        \@onelevel@sanitize\TestResult
        \@onelevel@sanitize\TestExpected
        \msg{ \space Result: \space\space[\TestResult]}%
        \msg{ \space Expected: [\TestExpected]}%
        \errmessage{Test failed!}%
      \fi
    }%
    \ifdim\wd0=0pt %
    \else
      \showbox0 %
    \fi
  \endgroup
}
\test{\getrefnumber{foo}}{\empty 1}
\test{\getpagerefnumber{foo}}{\empty 2}
\test{\getrefbykeydefault{foo}{}{\empty default}}{\empty 1}
\test{\getrefbykeydefault{foo}{page}{\empty default}}{\empty 2}
\test{\getrefbykeydefault{foo}{name}{\empty default}}{\empty default}
\test{\getrefbykeydefault{foo}{anchor}{\empty default}}{\empty default}
\test{\getrefbykeydefault{foo}{url}{\empty default}}{\empty default}
\test{\getrefbykeydefault{foo}{title}{\empty default}}{\empty default}
\msg{}
\def\r@foo{{}{}{}{}{}{}{}{}{}{}}
\def\Test#1#2\\{%
  \test{#1{foo}#2}{}%
}
\def\TestGroup{%
  \Test\getrefnumber\\%
  \Test\getpagerefnumber\\%
  \Test\getrefbykeydefault{}{}\\%
  \Test\getrefbykeydefault{page}{}\\%
  \Test\getrefbykeydefault{anchor}{}\\%
  \Test\getrefbykeydefault{name}{}\\%
  \Test\getrefbykeydefault{url}{}\\%
}
\TestGroup
\Test\getrefbykeydefault{title}{}\\%
\msg{}
\def\r@foo{\par\par\par\par\par\par\par\par}
\long\def\Test#1#2\\{%
  \test{#1{foo}#2}{\par}%
}
\TestGroup
\test{\getrefbykeydefault{title}{}{}}{}
\msg{}
\def\r@foo{{ }{ }{ }{ }{ }}
\def\Test#1#2\\{%
  \test{#1{foo}#2}{ }%
}
\TestGroup
\msg{}
\long\def\TestDefault#1{%
  \begingroup
    \setrefcountdefault{#1}%
    \test{\getrefnumber{foo}}{#1}%
    \test{\getpagerefnumber{foo}}{#1}%
  \endgroup
}
\def\TestDefaultX{%
  \TestDefault{}%
  \TestDefault{\par}%
  \TestDefault{ }%
  \TestDefault{\space}%
}
\let\r@foo\@undefined
\TestDefaultX
\let\r@foo\relax
\TestDefaultX
\def\r@foo{}
\TestDefaultX
%    \end{macrocode}
%    \begin{macrocode}
\msg{}
\long\def\Test#1#2#3#4{%
  \begingroup
    \def\TestTask{#1}%
    \@onelevel@sanitize\TestTask
    \msg{* [\TestTask]}%
    \edef\TestResultA{\IfRefUndefinedExpandable{#1}{#2}{#3}}%
    \IfRefUndefinedBabel{#1}{%
      \def\TestResultB{#2}%
    }{%
      \def\TestResultB{#3}%
    }%
    \def\TestExpected{#4}%
    \ifx\TestResultA\TestExpected
      \msg{ \space ok.}%
    \else
      \begingroup
        \@onelevel@sanitize\TestResultA
        \@onelevel@sanitize\TestExpected
        \msg{ \space Result: \space\space[\TestResultA]}%
        \msg{ \space Expected: [\TestExpected]}%
        \errmessage{Test failed!}%
      \endgroup
    \fi
    \ifx\TestResultB\TestExpected
      \msg{ \space ok.}%
    \else
      \begingroup
        \@onelevel@sanitize\TestResultB
        \@onelevel@sanitize\TestExpected
        \msg{ \space Result: \space\space[\TestResultB]}%
        \msg{ \space Expected: [\TestExpected]}%
        \errmessage{Test failed!}%
      \endgroup
    \fi
  \endgroup
}
\begingroup
  \def\r@foo{{}{}}%
  \let\r@bar\@undefined
  \let\r@xyz\relax
  \Test{foo}{true}{false}{false}%
  \Test{bar}{true}{false}{true}%
  \Test{xyz}{true}{false}{true}%
\endgroup
%    \end{macrocode}
%    \begin{macrocode}
\csname @@end\endcsname\end
%</test2>
%    \end{macrocode}
% \subsection{Test with package \xpackage{titleref}}
%
%    \begin{macrocode}
%<*test3>
\NeedsTeXFormat{LaTeX2e}
\documentclass{article}
\usepackage{refcount}[2011/10/16]
%<test4>\usepackage{nameref}
\usepackage{titleref}
\begin{document}
\section{Hello World}
\label{sec:hello}
\section{\hbox{xy}}
\label{sec:foo}
%
\makeatletter
\@ifundefined{r@sec:hello}{%
  \typeout{==> Compile twice!}%
}{%
  \def\test#1#2{%
    \begingroup
      \def\TestTask{#1}%
      \@onelevel@sanitize\TestTask
      \typeout{* \TestTask}%
      \expandafter\expandafter\expandafter\def
      \expandafter\expandafter\expandafter\TestResult
      \expandafter\expandafter\expandafter{%
        #1%
      }%
      \def\TestExpected{#2}%
      \ifx\TestResult\TestExpected
        \typeout{ \space ok.}%
      \else
        \@onelevel@sanitize\TestResult
        \@onelevel@sanitize\TestExpected
        \typeout{ \space Result: \space\space[\TestResult]}%
        \typeout{ \space Expected: [\TestExpected]}%
        \errmessage{Test failed!}%
      \fi
    \endgroup
  }%
  \test{\getrefbykeydefault{sec:hello}{title}{}}{Hello World}%
  \test{\getrefbykeydefault{sec:foo}{title}{}}{\hbox{xy}}%
  \begingroup
    \def\hbox#1{[#1]}% hash-ok
    \test{\getrefbykeydefault{sec:foo}{title}{}}{\hbox{xy}}%
  \endgroup
}
\makeatother
%    \end{macrocode}
%    \begin{macrocode}
\end{document}
%</test3>
%    \end{macrocode}
%    \begin{macrocode}
%<*test5>
\NeedsTeXFormat{LaTeX2e}
\documentclass{book}
\usepackage{refcount}[2011/10/16]
\usepackage{zref-runs}
\newcounter{test}
\begin{document}
\ifnum\zruns>1 %
  \makeatletter
  \def\Test#1#2#3{%
    \begingroup
      \setcounter{test}{10}%
      \sbox0{%
        #1{test}{#2}%
        \ifnum#3=\value{test}%
        \else
          \PackageError{test}{\string#1{#2} <> #3 (\the\value{test})}%
        \fi
      }%
      \ifdim\wd0=0pt %
      \else
        \PackageError{test}{Non-empty box}\@ehc
      \fi
    \endgroup
  }%
  \makeatother
  \Test\setcounterpageref{ch:two}{1}%
  \Test\setcounterpageref{ch:three}{3}%
  \Test\setcounterpageref{ch:four}{5}%
  \Test\setcounterpageref{ch:five}{7}%
  \Test\setcounterpageref{ch:six}{9}%
  \Test\setcounterpageref{ch:seven}{13}%
  \Test\addtocounterpageref{ch:two}{11}%
  \Test\addtocounterpageref{ch:three}{13}%
  \Test\addtocounterpageref{ch:four}{15}%
  \Test\addtocounterpageref{ch:five}{17}%
  \Test\addtocounterpageref{ch:six}{19}%
  \Test\addtocounterpageref{ch:seven}{23}%
  \Test\setcounterref{ch:two}{1}%
  \Test\setcounterref{ch:three}{2}%
  \Test\setcounterref{ch:four}{11}%
  \Test\addtocounterref{ch:two}{11}%
  \Test\addtocounterref{ch:three}{12}%
  \Test\addtocounterref{ch:four}{21}%
\fi
\frontmatter
\chapter{Chapter one}\label{ch:one}
\cleardoublepage
\mainmatter
\chapter{Chapter two}\label{ch:two}
\cleardoublepage
\chapter{Chapter three}\label{ch:three}
\cleardoublepage
\setcounter{chapter}{10}
\chapter{Chapter four}\label{ch:four}
\cleardoublepage
\appendix
\chapter{Chapter five}\label{ch:five}
\cleardoublepage
\chapter{Chapter six}\label{ch:six}
\cleardoublepage
\null
\cleardoublepage
\chapter{Chapter seven}\label{ch:seven}
\end{document}
%</test5>
%    \end{macrocode}
%
% \section{Installation}
%
% \subsection{Download}
%
% \paragraph{Package.} This package is available on
% CTAN\footnote{\url{ftp://ftp.ctan.org/tex-archive/}}:
% \begin{description}
% \item[\CTAN{macros/latex/contrib/oberdiek/refcount.dtx}] The source file.
% \item[\CTAN{macros/latex/contrib/oberdiek/refcount.pdf}] Documentation.
% \end{description}
%
%
% \paragraph{Bundle.} All the packages of the bundle `oberdiek'
% are also available in a TDS compliant ZIP archive. There
% the packages are already unpacked and the documentation files
% are generated. The files and directories obey the TDS standard.
% \begin{description}
% \item[\CTAN{install/macros/latex/contrib/oberdiek.tds.zip}]
% \end{description}
% \emph{TDS} refers to the standard ``A Directory Structure
% for \TeX\ Files'' (\CTAN{tds/tds.pdf}). Directories
% with \xfile{texmf} in their name are usually organized this way.
%
% \subsection{Bundle installation}
%
% \paragraph{Unpacking.} Unpack the \xfile{oberdiek.tds.zip} in the
% TDS tree (also known as \xfile{texmf} tree) of your choice.
% Example (linux):
% \begin{quote}
%   |unzip oberdiek.tds.zip -d ~/texmf|
% \end{quote}
%
% \paragraph{Script installation.}
% Check the directory \xfile{TDS:scripts/oberdiek/} for
% scripts that need further installation steps.
% Package \xpackage{attachfile2} comes with the Perl script
% \xfile{pdfatfi.pl} that should be installed in such a way
% that it can be called as \texttt{pdfatfi}.
% Example (linux):
% \begin{quote}
%   |chmod +x scripts/oberdiek/pdfatfi.pl|\\
%   |cp scripts/oberdiek/pdfatfi.pl /usr/local/bin/|
% \end{quote}
%
% \subsection{Package installation}
%
% \paragraph{Unpacking.} The \xfile{.dtx} file is a self-extracting
% \docstrip\ archive. The files are extracted by running the
% \xfile{.dtx} through \plainTeX:
% \begin{quote}
%   \verb|tex refcount.dtx|
% \end{quote}
%
% \paragraph{TDS.} Now the different files must be moved into
% the different directories in your installation TDS tree
% (also known as \xfile{texmf} tree):
% \begin{quote}
% \def\t{^^A
% \begin{tabular}{@{}>{\ttfamily}l@{ $\rightarrow$ }>{\ttfamily}l@{}}
%   refcount.sty & tex/latex/oberdiek/refcount.sty\\
%   refcount.pdf & doc/latex/oberdiek/refcount.pdf\\
%   test/refcount-test1.tex & doc/latex/oberdiek/test/refcount-test1.tex\\
%   test/refcount-test2.tex & doc/latex/oberdiek/test/refcount-test2.tex\\
%   test/refcount-test3.tex & doc/latex/oberdiek/test/refcount-test3.tex\\
%   test/refcount-test4.tex & doc/latex/oberdiek/test/refcount-test4.tex\\
%   test/refcount-test5.tex & doc/latex/oberdiek/test/refcount-test5.tex\\
%   refcount.dtx & source/latex/oberdiek/refcount.dtx\\
% \end{tabular}^^A
% }^^A
% \sbox0{\t}^^A
% \ifdim\wd0>\linewidth
%   \begingroup
%     \advance\linewidth by\leftmargin
%     \advance\linewidth by\rightmargin
%   \edef\x{\endgroup
%     \def\noexpand\lw{\the\linewidth}^^A
%   }\x
%   \def\lwbox{^^A
%     \leavevmode
%     \hbox to \linewidth{^^A
%       \kern-\leftmargin\relax
%       \hss
%       \usebox0
%       \hss
%       \kern-\rightmargin\relax
%     }^^A
%   }^^A
%   \ifdim\wd0>\lw
%     \sbox0{\small\t}^^A
%     \ifdim\wd0>\linewidth
%       \ifdim\wd0>\lw
%         \sbox0{\footnotesize\t}^^A
%         \ifdim\wd0>\linewidth
%           \ifdim\wd0>\lw
%             \sbox0{\scriptsize\t}^^A
%             \ifdim\wd0>\linewidth
%               \ifdim\wd0>\lw
%                 \sbox0{\tiny\t}^^A
%                 \ifdim\wd0>\linewidth
%                   \lwbox
%                 \else
%                   \usebox0
%                 \fi
%               \else
%                 \lwbox
%               \fi
%             \else
%               \usebox0
%             \fi
%           \else
%             \lwbox
%           \fi
%         \else
%           \usebox0
%         \fi
%       \else
%         \lwbox
%       \fi
%     \else
%       \usebox0
%     \fi
%   \else
%     \lwbox
%   \fi
% \else
%   \usebox0
% \fi
% \end{quote}
% If you have a \xfile{docstrip.cfg} that configures and enables \docstrip's
% TDS installing feature, then some files can already be in the right
% place, see the documentation of \docstrip.
%
% \subsection{Refresh file name databases}
%
% If your \TeX~distribution
% (\teTeX, \mikTeX, \dots) relies on file name databases, you must refresh
% these. For example, \teTeX\ users run \verb|texhash| or
% \verb|mktexlsr|.
%
% \subsection{Some details for the interested}
%
% \paragraph{Attached source.}
%
% The PDF documentation on CTAN also includes the
% \xfile{.dtx} source file. It can be extracted by
% AcrobatReader 6 or higher. Another option is \textsf{pdftk},
% e.g. unpack the file into the current directory:
% \begin{quote}
%   \verb|pdftk refcount.pdf unpack_files output .|
% \end{quote}
%
% \paragraph{Unpacking with \LaTeX.}
% The \xfile{.dtx} chooses its action depending on the format:
% \begin{description}
% \item[\plainTeX:] Run \docstrip\ and extract the files.
% \item[\LaTeX:] Generate the documentation.
% \end{description}
% If you insist on using \LaTeX\ for \docstrip\ (really,
% \docstrip\ does not need \LaTeX), then inform the autodetect routine
% about your intention:
% \begin{quote}
%   \verb|latex \let\install=y% \iffalse meta-comment
%
% File: refcount.dtx
% Version: 2011/10/16 v3.4
% Info: Data extraction from label references
%
% Copyright (C) 1998, 2000, 2006, 2008, 2010, 2011 by
%    Heiko Oberdiek <heiko.oberdiek at googlemail.com>
%
% This work may be distributed and/or modified under the
% conditions of the LaTeX Project Public License, either
% version 1.3c of this license or (at your option) any later
% version. This version of this license is in
%    http://www.latex-project.org/lppl/lppl-1-3c.txt
% and the latest version of this license is in
%    http://www.latex-project.org/lppl.txt
% and version 1.3 or later is part of all distributions of
% LaTeX version 2005/12/01 or later.
%
% This work has the LPPL maintenance status "maintained".
%
% This Current Maintainer of this work is Heiko Oberdiek.
%
% This work consists of the main source file refcount.dtx
% and the derived files
%    refcount.sty, refcount.pdf, refcount.ins, refcount.drv,
%    refcount-test1.tex, refcount-test2.tex, refcount-test3.tex,
%    refcount-test4.tex, refcount-test5.tex.
%
% Distribution:
%    CTAN:macros/latex/contrib/oberdiek/refcount.dtx
%    CTAN:macros/latex/contrib/oberdiek/refcount.pdf
%
% Unpacking:
%    (a) If refcount.ins is present:
%           tex refcount.ins
%    (b) Without refcount.ins:
%           tex refcount.dtx
%    (c) If you insist on using LaTeX
%           latex \let\install=y% \iffalse meta-comment
%
% File: refcount.dtx
% Version: 2011/10/16 v3.4
% Info: Data extraction from label references
%
% Copyright (C) 1998, 2000, 2006, 2008, 2010, 2011 by
%    Heiko Oberdiek <heiko.oberdiek at googlemail.com>
%
% This work may be distributed and/or modified under the
% conditions of the LaTeX Project Public License, either
% version 1.3c of this license or (at your option) any later
% version. This version of this license is in
%    http://www.latex-project.org/lppl/lppl-1-3c.txt
% and the latest version of this license is in
%    http://www.latex-project.org/lppl.txt
% and version 1.3 or later is part of all distributions of
% LaTeX version 2005/12/01 or later.
%
% This work has the LPPL maintenance status "maintained".
%
% This Current Maintainer of this work is Heiko Oberdiek.
%
% This work consists of the main source file refcount.dtx
% and the derived files
%    refcount.sty, refcount.pdf, refcount.ins, refcount.drv,
%    refcount-test1.tex, refcount-test2.tex, refcount-test3.tex,
%    refcount-test4.tex, refcount-test5.tex.
%
% Distribution:
%    CTAN:macros/latex/contrib/oberdiek/refcount.dtx
%    CTAN:macros/latex/contrib/oberdiek/refcount.pdf
%
% Unpacking:
%    (a) If refcount.ins is present:
%           tex refcount.ins
%    (b) Without refcount.ins:
%           tex refcount.dtx
%    (c) If you insist on using LaTeX
%           latex \let\install=y% \iffalse meta-comment
%
% File: refcount.dtx
% Version: 2011/10/16 v3.4
% Info: Data extraction from label references
%
% Copyright (C) 1998, 2000, 2006, 2008, 2010, 2011 by
%    Heiko Oberdiek <heiko.oberdiek at googlemail.com>
%
% This work may be distributed and/or modified under the
% conditions of the LaTeX Project Public License, either
% version 1.3c of this license or (at your option) any later
% version. This version of this license is in
%    http://www.latex-project.org/lppl/lppl-1-3c.txt
% and the latest version of this license is in
%    http://www.latex-project.org/lppl.txt
% and version 1.3 or later is part of all distributions of
% LaTeX version 2005/12/01 or later.
%
% This work has the LPPL maintenance status "maintained".
%
% This Current Maintainer of this work is Heiko Oberdiek.
%
% This work consists of the main source file refcount.dtx
% and the derived files
%    refcount.sty, refcount.pdf, refcount.ins, refcount.drv,
%    refcount-test1.tex, refcount-test2.tex, refcount-test3.tex,
%    refcount-test4.tex, refcount-test5.tex.
%
% Distribution:
%    CTAN:macros/latex/contrib/oberdiek/refcount.dtx
%    CTAN:macros/latex/contrib/oberdiek/refcount.pdf
%
% Unpacking:
%    (a) If refcount.ins is present:
%           tex refcount.ins
%    (b) Without refcount.ins:
%           tex refcount.dtx
%    (c) If you insist on using LaTeX
%           latex \let\install=y\input{refcount.dtx}
%        (quote the arguments according to the demands of your shell)
%
% Documentation:
%    (a) If refcount.drv is present:
%           latex refcount.drv
%    (b) Without refcount.drv:
%           latex refcount.dtx; ...
%    The class ltxdoc loads the configuration file ltxdoc.cfg
%    if available. Here you can specify further options, e.g.
%    use A4 as paper format:
%       \PassOptionsToClass{a4paper}{article}
%
%    Programm calls to get the documentation (example):
%       pdflatex refcount.dtx
%       makeindex -s gind.ist refcount.idx
%       pdflatex refcount.dtx
%       makeindex -s gind.ist refcount.idx
%       pdflatex refcount.dtx
%
% Installation:
%    TDS:tex/latex/oberdiek/refcount.sty
%    TDS:doc/latex/oberdiek/refcount.pdf
%    TDS:doc/latex/oberdiek/test/refcount-test1.tex
%    TDS:doc/latex/oberdiek/test/refcount-test2.tex
%    TDS:doc/latex/oberdiek/test/refcount-test3.tex
%    TDS:doc/latex/oberdiek/test/refcount-test4.tex
%    TDS:doc/latex/oberdiek/test/refcount-test5.tex
%    TDS:source/latex/oberdiek/refcount.dtx
%
%<*ignore>
\begingroup
  \catcode123=1 %
  \catcode125=2 %
  \def\x{LaTeX2e}%
\expandafter\endgroup
\ifcase 0\ifx\install y1\fi\expandafter
         \ifx\csname processbatchFile\endcsname\relax\else1\fi
         \ifx\fmtname\x\else 1\fi\relax
\else\csname fi\endcsname
%</ignore>
%<*install>
\input docstrip.tex
\Msg{************************************************************************}
\Msg{* Installation}
\Msg{* Package: refcount 2011/10/16 v3.4 Data extraction from label references (HO)}
\Msg{************************************************************************}

\keepsilent
\askforoverwritefalse

\let\MetaPrefix\relax
\preamble

This is a generated file.

Project: refcount
Version: 2011/10/16 v3.4

Copyright (C) 1998, 2000, 2006, 2008, 2010, 2011 by
   Heiko Oberdiek <heiko.oberdiek at googlemail.com>

This work may be distributed and/or modified under the
conditions of the LaTeX Project Public License, either
version 1.3c of this license or (at your option) any later
version. This version of this license is in
   http://www.latex-project.org/lppl/lppl-1-3c.txt
and the latest version of this license is in
   http://www.latex-project.org/lppl.txt
and version 1.3 or later is part of all distributions of
LaTeX version 2005/12/01 or later.

This work has the LPPL maintenance status "maintained".

This Current Maintainer of this work is Heiko Oberdiek.

This work consists of the main source file refcount.dtx
and the derived files
   refcount.sty, refcount.pdf, refcount.ins, refcount.drv,
   refcount-test1.tex, refcount-test2.tex, refcount-test3.tex,
   refcount-test4.tex, refcount-test5.tex.

\endpreamble
\let\MetaPrefix\DoubleperCent

\generate{%
  \file{refcount.ins}{\from{refcount.dtx}{install}}%
  \file{refcount.drv}{\from{refcount.dtx}{driver}}%
  \usedir{tex/latex/oberdiek}%
  \file{refcount.sty}{\from{refcount.dtx}{package}}%
  \usedir{doc/latex/oberdiek/test}%
  \file{refcount-test1.tex}{\from{refcount.dtx}{test1}}%
  \file{refcount-test2.tex}{\from{refcount.dtx}{test2}}%
  \file{refcount-test3.tex}{\from{refcount.dtx}{test3}}%
  \file{refcount-test4.tex}{\from{refcount.dtx}{test3,test4}}%
  \file{refcount-test5.tex}{\from{refcount.dtx}{test5}}%
  \nopreamble
  \nopostamble
  \usedir{source/latex/oberdiek/catalogue}%
  \file{refcount.xml}{\from{refcount.dtx}{catalogue}}%
}

\catcode32=13\relax% active space
\let =\space%
\Msg{************************************************************************}
\Msg{*}
\Msg{* To finish the installation you have to move the following}
\Msg{* file into a directory searched by TeX:}
\Msg{*}
\Msg{*     refcount.sty}
\Msg{*}
\Msg{* To produce the documentation run the file `refcount.drv'}
\Msg{* through LaTeX.}
\Msg{*}
\Msg{* Happy TeXing!}
\Msg{*}
\Msg{************************************************************************}

\endbatchfile
%</install>
%<*ignore>
\fi
%</ignore>
%<*driver>
\NeedsTeXFormat{LaTeX2e}
\ProvidesFile{refcount.drv}%
  [2011/10/16 v3.4 Data extraction from label references (HO)]%
\documentclass{ltxdoc}
\usepackage{holtxdoc}[2011/11/22]
\begin{document}
  \DocInput{refcount.dtx}%
\end{document}
%</driver>
% \fi
%
% \CheckSum{1185}
%
% \CharacterTable
%  {Upper-case    \A\B\C\D\E\F\G\H\I\J\K\L\M\N\O\P\Q\R\S\T\U\V\W\X\Y\Z
%   Lower-case    \a\b\c\d\e\f\g\h\i\j\k\l\m\n\o\p\q\r\s\t\u\v\w\x\y\z
%   Digits        \0\1\2\3\4\5\6\7\8\9
%   Exclamation   \!     Double quote  \"     Hash (number) \#
%   Dollar        \$     Percent       \%     Ampersand     \&
%   Acute accent  \'     Left paren    \(     Right paren   \)
%   Asterisk      \*     Plus          \+     Comma         \,
%   Minus         \-     Point         \.     Solidus       \/
%   Colon         \:     Semicolon     \;     Less than     \<
%   Equals        \=     Greater than  \>     Question mark \?
%   Commercial at \@     Left bracket  \[     Backslash     \\
%   Right bracket \]     Circumflex    \^     Underscore    \_
%   Grave accent  \`     Left brace    \{     Vertical bar  \|
%   Right brace   \}     Tilde         \~}
%
% \GetFileInfo{refcount.drv}
%
% \title{The \xpackage{refcount} package}
% \date{2011/10/16 v3.4}
% \author{Heiko Oberdiek\\\xemail{heiko.oberdiek at googlemail.com}}
%
% \maketitle
%
% \begin{abstract}
% References are not numbers, however they often store numerical
% data such as section or page numbers. \cs{ref} or \cs{pageref}
% cannot be used for counter assignments or calculations because
% they are not expandable, generate warnings, or can even be links.
% The package provides expandable macros to extract the data
% from references. Packages \xpackage{hyperref}, \xpackage{nameref},
% \xpackage{titleref}, and \xpackage{babel} are supported.
% \end{abstract}
%
% \tableofcontents
%
% \section{Usage}
%
% \subsection{Setting counters}
%
% The following commands are similar to \LaTeX's
% \cs{setcounter} and \cs{addtocounter},
% but they extract the number value from a reference:
% \begin{quote}
%   \cs{setcounterref}, \cs{addtocounterref}\\
%   \cs{setcounterpageref}, \cs{addtocounterpageref}
% \end{quote}
% They take two arguments:
% \begin{quote}
%    \cs{...counter...ref} |{|\meta{\LaTeX\ counter}|}|
%    |{|\meta{reference}|}|
% \end{quote}
% An undefined references produces the usual LaTeX warning
% and its value is assumed to be zero.
% Example:
% \begin{quote}
%\begin{verbatim}
%\newcounter{ctrA}
%\newcounter{ctrB}
%\refstepcounter{ctrA}\label{ref:A}
%\setcounterref{ctrB}{ref:A}
%\addtocounterpageref{ctrB}{ref:A}
%\end{verbatim}
% \end{quote}
%
% \subsection{Expandable commands}
%
% These commands that can be used in expandible contexts
% (inside calculations, \cs{edef}, \cs{csname}, \cs{write}, \dots):
% \begin{quote}
%   \cs{getrefnumber}, \cs{getpagerefnumber}
% \end{quote}
% They take one argument, the reference:
% \begin{quote}
%   \cs{get...refnumber} |{|\meta{reference}|}|
% \end{quote}
% The default for undefined references can be changed
% with macro \cs{setrefcountdefault}, for example this
% package calls:
% \begin{quote}
%   \cs{setrefcountdefault}|{0}|
% \end{quote}
%
% Since version 2.0 of this package there is a new
% command:
% \begin{quote}
%   \cs{getrefbykeydefault} |{|\meta{reference}|}|
%   |{|\meta{key}|}| |{|\meta{default}|}|
% \end{quote}
% This generalized version allows the extraction
% of further properties of a reference than the
% two standard ones. Thus the following properties
% are supported, if they are available:
% \begin{quote}
% \begin{tabular}{@{}l|l|l@{}}
%    Key & Description & Package\\
% \hline
%   \meta{empty} & same as \cs{ref} & \LaTeX\\
%   |page| & same as \cs{pageref} & \LaTeX\\
%   |title| & section and caption titles & \xpackage{titleref}\\
%   |name| & section and caption titles & \xpackage{nameref}\\
%   |anchor| & anchor name & \xpackage{hyperref}\\
%   |url| & url/file & \xpackage{hyperref}/\xpackage{xr}
% \end{tabular}
% \end{quote}
%
% Since version 3.2 the expandable macros described before
% in this section
% are expandable in exact two expansion steps.
%
% \subsection{Undefined references}
%
% Because warnings and assignments cannot be used in
% expandible contexts, undefined references do not
% produce a warning, their values are assumed to be zero.
% Example:
% \begin{quote}
%\begin{verbatim}
%\label{ref:here}% somewhere
%\refused{ref:here}% see below
%\ifodd\getpagerefnumber{ref:here}%
%  reference is on an odd page
%\else
%  reference is on an even page
%\fi
%\end{verbatim}
% \end{quote}
%
% In case of undefined references the user usually want's
% to be informed. Also \LaTeX\ prints a warning at
% the end of the \LaTeX\ run. To notify \LaTeX\ and
% get a normal warning, just use
% \begin{quote}
%   \cs{refused} |{|\meta{reference}|}|
% \end{quote}
% outside the expanding context. Example, see above.
%
% \subsubsection{Check for undefined references}
%
% In version 3.2 macros were added, that test, whether references
% are defined.
% \begin{declcs}{IfRefUndefinedExpandable} \M{refname} \M{then} \M{else}\\
%   \cs{IfRefUndefinedBabel} \M{refname} \M{then} \M{else}
% \end{declcs}
% If the reference is not available and therefore undefined, then
% argument \meta{then} is executed, otherwise argument \meta{else}
% is called. Macro \cs{IfRefUndefinedExpandable} is expandable,
% but \meta{refname} must not contain babel shorthand characters.
% Macro \cs{IfRefUndefinedBabel} supports shorthand characters of
% babel, but it is not expandable.
%
% \subsection{Notes}
%
% \begin{itemize}
% \item
%   The method of extracting the number in this
%   package also works in cases, where the
%   reference cannot be used directly, because
%   a package such as \xpackage{hyperref} has added
%   extra stuff (hyper link), so that the reference cannot
%   be used as number any more.
% \item
%   If the reference does not contain a number,
%   assignments to a counter will fail of course.
% \end{itemize}
%
%
% \StopEventually{
% }
%
% \section{Implementation}
%
%    \begin{macrocode}
%<*package>
%    \end{macrocode}
%    Reload check, especially if the package is not used with \LaTeX.
%    \begin{macrocode}
\begingroup\catcode61\catcode48\catcode32=10\relax%
  \catcode13=5 % ^^M
  \endlinechar=13 %
  \catcode35=6 % #
  \catcode39=12 % '
  \catcode44=12 % ,
  \catcode45=12 % -
  \catcode46=12 % .
  \catcode58=12 % :
  \catcode64=11 % @
  \catcode123=1 % {
  \catcode125=2 % }
  \expandafter\let\expandafter\x\csname ver@refcount.sty\endcsname
  \ifx\x\relax % plain-TeX, first loading
  \else
    \def\empty{}%
    \ifx\x\empty % LaTeX, first loading,
      % variable is initialized, but \ProvidesPackage not yet seen
    \else
      \expandafter\ifx\csname PackageInfo\endcsname\relax
        \def\x#1#2{%
          \immediate\write-1{Package #1 Info: #2.}%
        }%
      \else
        \def\x#1#2{\PackageInfo{#1}{#2, stopped}}%
      \fi
      \x{refcount}{The package is already loaded}%
      \aftergroup\endinput
    \fi
  \fi
\endgroup%
%    \end{macrocode}
%    Package identification:
%    \begin{macrocode}
\begingroup\catcode61\catcode48\catcode32=10\relax%
  \catcode13=5 % ^^M
  \endlinechar=13 %
  \catcode35=6 % #
  \catcode39=12 % '
  \catcode40=12 % (
  \catcode41=12 % )
  \catcode44=12 % ,
  \catcode45=12 % -
  \catcode46=12 % .
  \catcode47=12 % /
  \catcode58=12 % :
  \catcode64=11 % @
  \catcode91=12 % [
  \catcode93=12 % ]
  \catcode123=1 % {
  \catcode125=2 % }
  \expandafter\ifx\csname ProvidesPackage\endcsname\relax
    \def\x#1#2#3[#4]{\endgroup
      \immediate\write-1{Package: #3 #4}%
      \xdef#1{#4}%
    }%
  \else
    \def\x#1#2[#3]{\endgroup
      #2[{#3}]%
      \ifx#1\@undefined
        \xdef#1{#3}%
      \fi
      \ifx#1\relax
        \xdef#1{#3}%
      \fi
    }%
  \fi
\expandafter\x\csname ver@refcount.sty\endcsname
\ProvidesPackage{refcount}%
  [2011/10/16 v3.4 Data extraction from label references (HO)]%
%    \end{macrocode}
%
%    \begin{macrocode}
\begingroup\catcode61\catcode48\catcode32=10\relax%
  \catcode13=5 % ^^M
  \endlinechar=13 %
  \catcode123=1 % {
  \catcode125=2 % }
  \catcode64=11 % @
  \def\x{\endgroup
    \expandafter\edef\csname rc@AtEnd\endcsname{%
      \endlinechar=\the\endlinechar\relax
      \catcode13=\the\catcode13\relax
      \catcode32=\the\catcode32\relax
      \catcode35=\the\catcode35\relax
      \catcode61=\the\catcode61\relax
      \catcode64=\the\catcode64\relax
      \catcode123=\the\catcode123\relax
      \catcode125=\the\catcode125\relax
    }%
  }%
\x\catcode61\catcode48\catcode32=10\relax%
\catcode13=5 % ^^M
\endlinechar=13 %
\catcode35=6 % #
\catcode64=11 % @
\catcode123=1 % {
\catcode125=2 % }
\def\TMP@EnsureCode#1#2{%
  \edef\rc@AtEnd{%
    \rc@AtEnd
    \catcode#1=\the\catcode#1\relax
  }%
  \catcode#1=#2\relax
}
\TMP@EnsureCode{33}{12}% !
\TMP@EnsureCode{39}{12}% '
\TMP@EnsureCode{42}{12}% *
\TMP@EnsureCode{45}{12}% -
\TMP@EnsureCode{46}{12}% .
\TMP@EnsureCode{47}{12}% /
\TMP@EnsureCode{91}{12}% [
\TMP@EnsureCode{93}{12}% ]
\TMP@EnsureCode{96}{12}% `
\edef\rc@AtEnd{\rc@AtEnd\noexpand\endinput}
%    \end{macrocode}
%
% \subsection{Loading packages}
%
%    \begin{macrocode}
\begingroup\expandafter\expandafter\expandafter\endgroup
\expandafter\ifx\csname RequirePackage\endcsname\relax
  \input ltxcmds.sty\relax
  \input infwarerr.sty\relax
\else
  \RequirePackage{ltxcmds}[2011/11/09]%
  \RequirePackage{infwarerr}[2010/04/08]%
\fi
%    \end{macrocode}
%
% \subsection{Defining commands}
%
%    \begin{macro}{\rc@IfDefinable}
%    \begin{macrocode}
\ltx@IfUndefined{@ifdefinable}{%
  \def\rc@IfDefinable#1{%
    \ifx#1\ltx@undefined
      \expandafter\ltx@firstofone
    \else
      \ifx#1\relax
        \expandafter\expandafter\expandafter\ltx@firstofone
      \else
        \@PackageError{refcount}{%
          Command \string#1 is already defined.\MessageBreak
          It will not redefined by this package%
        }\@ehc
        \expandafter\expandafter\expandafter\ltx@gobble
      \fi
    \fi
  }%
}{%
  \let\rc@IfDefinable\@ifdefinable
}
%    \end{macrocode}
%    \end{macro}
%
%    \begin{macro}{\rc@RobustDefOne}
%    \begin{macro}{\rc@RobustDefZero}
%    \begin{macrocode}
\ltx@IfUndefined{protected}{%
  \ltx@IfUndefined{DeclareRobustCommand}{%
    \def\rc@RobustDefOne#1#2#3#4{%
      \rc@IfDefinable#3{%
        #1\def#3##1{#4}%
      }%
    }%
    \def\rc@RobustDefZero#1#2{%
      \rc@IfDefinable#1{%
        \def#1{#2}%
      }%
    }%
  }{%
    \def\rc@RobustDefOne#1#2#3#4{%
      \rc@IfDefinable#3{%
        \DeclareRobustCommand#2#3[1]{#4}%
      }%
    }%
    \def\rc@RobustDefZero#1#2{%
      \rc@IfDefinable#1{%
        \DeclareRobustCommand#1{#2}%
      }%
    }%
  }%
}{%
  \def\rc@RobustDefOne#1#2#3#4{%
    \rc@IfDefinable#3{%
      \protected#1\def#3##1{#4}%
    }%
  }%
  \def\rc@RobustDefZero#1#2{%
    \rc@IfDefinable#1{%
      \protected\def#1{#2}%
    }%
  }%
}
%    \end{macrocode}
%    \end{macro}
%    \end{macro}
%
%    \begin{macro}{\rc@newcommand}
%    \begin{macrocode}
\ltx@IfUndefined{newcommand}{%
  \def\rc@newcommand*#1[#2]#3{% hash-ok
    \rc@IfDefinable#1{%
      \ifcase#2 %
        \def#1{#3}%
      \or
        \def#1##1{#3}%
      \or
        \def#1##1##2{#3}%
      \else
        \rc@InternalError
      \fi
    }%
  }%
}{%
  \let\rc@newcommand\newcommand
}
%    \end{macrocode}
%    \end{macro}
%
% \subsection{\cs{setrefcountdefault}}
%
%    \begin{macro}{\setrefcountdefault}
%    \begin{macrocode}
\rc@RobustDefOne\long{}\setrefcountdefault{%
  \def\rc@default{#1}%
}
%    \end{macrocode}
%    \end{macro}
%    \begin{macrocode}
\setrefcountdefault{0}
%    \end{macrocode}
%
% \subsection{\cs{refused}}
%
%    \begin{macro}{\refused}
%    \begin{macrocode}
\ltx@IfUndefined{G@refundefinedtrue}{%
  \rc@RobustDefOne{}{*}\refused{%
    \begingroup
      \csname @safe@activestrue\endcsname
      \ltx@IfUndefined{r@#1}{%
        \protect\G@refundefinedtrue
        \rc@WarningUndefined{#1}%
      }{}%
    \endgroup
  }%
}{%
  \rc@RobustDefOne{}{*}\refused{%
    \begingroup
      \csname @safe@activestrue\endcsname
      \ltx@IfUndefined{r@#1}{%
        \csname protect\expandafter\endcsname
        \csname G@refundefinedtrue\endcsname
        \rc@WarningUndefined{#1}%
      }{}%
    \endgroup
  }%
}
%    \end{macrocode}
%    \end{macro}
%    \begin{macro}{\rc@WarningUndefined}
%    \begin{macrocode}
\ltx@IfUndefined{@latex@warning}{%
  \def\rc@WarningUndefined#1{%
    \ltx@ifundefined{thepage}{%
      \def\thepage{\number\count0 }%
    }{}%
    \@PackageWarning{refcount}{%
      Reference `#1' on page \thepage\space undefined%
    }%
  }%
}{%
  \def\rc@WarningUndefined#1{%
    \@latex@warning{%
      Reference `#1' on page \thepage\space undefined%
    }%
  }%
}
%    \end{macrocode}
%    \end{macro}
%
% \subsection{Setting counters by reference data}
%
% \subsubsection{Generic setting}
%
%    \begin{macro}{\rc@set}
% Generic command for
% |\|$\{$|set|$,$|addto|$\}$|counter|$\{$|page|$,\}$|ref|:
%\begin{quote}
%\begin{tabular}[t]{@{}l@{: }l@{}}
% |#1|& \cs{setcounter}, \cs{addtocounter}\\
% |#2|& \cs{ltx@car} (for \cs{ref}), \cs{ltx@cartwo} (for \cs{pageref})\\
% |#3|& \hologo{LaTeX} counter\\
% |#4|& reference\\
%\end{tabular}
%\end{quote}
%    \begin{macrocode}
\def\rc@set#1#2#3#4{%
  \begingroup
    \csname @safe@activestrue\endcsname
    \refused{#4}%
    \expandafter\rc@@set\csname r@#4\endcsname{#1}{#2}{#3}%
  \endgroup
}
%    \end{macrocode}
%    \end{macro}
%    \begin{macro}{\rc@@set}
%\begin{quote}
%\begin{tabular}[t]{@{}l@{: }l@{}}
% |#1|& \cs{r@<...>}\\
% |#2|& \cs{setcounter}, \cs{addtocounter}\\
% |#3|& \cs{ltx@car} (for \cs{ref}), \cs{ltx@carsecond} (for \cs{pageref})\\
% |#4|& \hologo{LaTeX} counter\\
%\end{tabular}
%\end{quote}
%    \begin{macrocode}
\def\rc@@set#1#2#3#4{%
  \ifx#1\relax
    #2{#4}{\rc@default}%
  \else
    #2{#4}{%
      \expandafter#3#1\rc@default\rc@default\@nil
    }%
  \fi
}
%    \end{macrocode}
%    \end{macro}
%
% \subsubsection{User commands}
%
%    \begin{macro}{\setcounterref}
%    \begin{macrocode}
\rc@RobustDefZero\setcounterref{%
  \rc@set\setcounter\ltx@car
}
%    \end{macrocode}
%    \end{macro}
%    \begin{macro}{\addtocounterref}
%    \begin{macrocode}
\rc@RobustDefZero\addtocounterref{%
  \rc@set\addtocounter\ltx@car
}
%    \end{macrocode}
%    \end{macro}
%    \begin{macro}{\setcounterpageref}
%    \begin{macrocode}
\rc@RobustDefZero\setcounterpageref{%
  \rc@set\setcounter\ltx@carsecond
}
%    \end{macrocode}
%    \end{macro}
%    \begin{macro}{\addtocounterpageref}
%    \begin{macrocode}
\rc@RobustDefZero\addtocounterpageref{%
  \rc@set\addtocounter\ltx@carsecond
}
%    \end{macrocode}
%    \end{macro}
%
% \subsection{Extracting references}
%
%    \begin{macro}{\getrefnumber}
%    \begin{macrocode}
\rc@newcommand*{\getrefnumber}[1]{%
  \romannumeral
  \ltx@ifundefined{r@#1}{%
    \expandafter\ltx@zero
    \rc@default
  }{%
    \expandafter\expandafter\expandafter\rc@extract@
    \expandafter\expandafter\expandafter!%
    \csname r@#1\expandafter\endcsname
    \expandafter{\rc@default}\@nil
  }%
}
%    \end{macrocode}
%    \end{macro}
%    \begin{macro}{\getpagerefnumber}
%    \begin{macrocode}
\rc@newcommand*{\getpagerefnumber}[1]{%
  \romannumeral
  \ltx@ifundefined{r@#1}{%
    \expandafter\ltx@zero
    \rc@default
  }{%
    \expandafter\expandafter\expandafter\rc@extract@page
    \expandafter\expandafter\expandafter!%
    \csname r@#1\expandafter\expandafter\expandafter\endcsname
    \expandafter\expandafter\expandafter{%
      \expandafter\rc@default
    \expandafter}\expandafter{\rc@default}\@nil
  }%
}
%    \end{macrocode}
%    \end{macro}
%    \begin{macro}{\getrefbykeydefault}
%    \begin{macrocode}
\rc@newcommand*{\getrefbykeydefault}[2]{%
  \romannumeral
  \expandafter\rc@getrefbykeydefault
    \csname r@#1\expandafter\endcsname
    \csname rc@extract@#2\endcsname
}
%    \end{macrocode}
%    \end{macro}
%    \begin{macro}{\rc@getrefbykeydefault}
%\begin{quote}
%\begin{tabular}[t]{@{}l@{: }l@{}}
% |#1|& \cs{r@<...>}\\
% |#2|& \cs{rc@extract@<...>}\\
% |#3|& default\\
%\end{tabular}
%\end{quote}
%    \begin{macrocode}
\long\def\rc@getrefbykeydefault#1#2#3{%
  \ifx#1\relax
    % reference is undefined
    \ltx@ReturnAfterElseFi{%
      \ltx@zero
      #3%
    }%
  \else
    \ltx@ReturnAfterFi{%
      \ifx#2\relax
        % extract method is missing
        \ltx@ReturnAfterElseFi{%
          \ltx@zero
          #3%
        }%
      \else
        \ltx@ReturnAfterFi{%
          \expandafter
          \rc@generic#1{#3}{#3}{#3}{#3}{#3}\@nil#2{#3}%
        }%
      \fi
    }%
  \fi
}
%    \end{macrocode}
%    \end{macro}
%    \begin{macro}{\rc@generic}
%\begin{quote}
%\begin{tabular}[t]{@{}l@{: }l@{}}
% |#1|& first item in \cs{r@<...>}\\
% |#2|& remaining items in \cs{r@<...>}\\
% |#3|& \cs{rc@extract@<...>}\\
% |#4|& default\\
%\end{tabular}
%\end{quote}
%    \begin{macrocode}
\long\def\rc@generic#1#2\@nil#3#4{%
  #3{#1\TR@TitleReference\@empty{#4}\@nil}{#1}#2\@nil
}
%    \end{macrocode}
%    \end{macro}
%    \begin{macro}{\rc@extract@}
%    \begin{macrocode}
\long\def\rc@extract@#1#2#3\@nil{%
  \ltx@zero
  #2%
}
%    \end{macrocode}
%    \end{macro}
%    \begin{macro}{\rc@extract@page}
%    \begin{macrocode}
\long\def\rc@extract@page#1#2#3#4\@nil{%
  \ltx@zero
  #3%
}
%    \end{macrocode}
%    \end{macro}
%    \begin{macro}{\rc@extract@name}
%    \begin{macrocode}
\long\def\rc@extract@name#1#2#3#4#5\@nil{%
  \ltx@zero
  #4%
}
%    \end{macrocode}
%    \end{macro}
%    \begin{macro}{\rc@extract@anchor}
%    \begin{macrocode}
\long\def\rc@extract@anchor#1#2#3#4#5#6\@nil{%
  \ltx@zero
  #5%
}
%    \end{macrocode}
%    \end{macro}
%    \begin{macro}{\rc@extract@url}
%    \begin{macrocode}
\long\def\rc@extract@url#1#2#3#4#5#6#7\@nil{%
  \ltx@zero
  #6%
}
%    \end{macrocode}
%    \end{macro}
%    \begin{macro}{\rc@extract@title}
%    \begin{macrocode}
\long\def\rc@extract@title#1#2\@nil{%
  \rc@@extract@title#1%
}
%    \end{macrocode}
%    \end{macro}
%    \begin{macro}{\rc@@extract@title}
%    \begin{macrocode}
\long\def\rc@@extract@title#1\TR@TitleReference#2#3#4\@nil{%
  \ltx@zero
  #3%
}
%    \end{macrocode}
%    \end{macro}
%
% \subsection{Macros for checking undefined references}
%
%    \begin{macro}{\IfRefUndefinedExpandable}
%    \begin{macrocode}
\rc@newcommand*{\IfRefUndefinedExpandable}[1]{%
  \ltx@ifundefined{r@#1}\ltx@firstoftwo\ltx@secondoftwo
}
%    \end{macrocode}
%    \end{macro}
%    \begin{macro}{\IfRefUndefinedBabel}
%    \begin{macrocode}
\rc@RobustDefOne{}*\IfRefUndefinedBabel{%
  \begingroup
    \csname safe@actives@true\endcsname
  \expandafter\expandafter\expandafter\endgroup
  \expandafter\ifx\csname r@#1\endcsname\relax
    \expandafter\ltx@firstoftwo
  \else
    \expandafter\ltx@secondoftwo
  \fi
}
%    \end{macrocode}
%    \end{macro}
%    \begin{macrocode}
\rc@AtEnd%
%</package>
%    \end{macrocode}
%
% \section{Test}
%
% \subsection{Catcode checks for loading}
%
%    \begin{macrocode}
%<*test1>
%    \end{macrocode}
%    \begin{macrocode}
\catcode`\{=1 %
\catcode`\}=2 %
\catcode`\#=6 %
\catcode`\@=11 %
\expandafter\ifx\csname count@\endcsname\relax
  \countdef\count@=255 %
\fi
\expandafter\ifx\csname @gobble\endcsname\relax
  \long\def\@gobble#1{}%
\fi
\expandafter\ifx\csname @firstofone\endcsname\relax
  \long\def\@firstofone#1{#1}%
\fi
\expandafter\ifx\csname loop\endcsname\relax
  \expandafter\@firstofone
\else
  \expandafter\@gobble
\fi
{%
  \def\loop#1\repeat{%
    \def\body{#1}%
    \iterate
  }%
  \def\iterate{%
    \body
      \let\next\iterate
    \else
      \let\next\relax
    \fi
    \next
  }%
  \let\repeat=\fi
}%
\def\RestoreCatcodes{}
\count@=0 %
\loop
  \edef\RestoreCatcodes{%
    \RestoreCatcodes
    \catcode\the\count@=\the\catcode\count@\relax
  }%
\ifnum\count@<255 %
  \advance\count@ 1 %
\repeat

\def\RangeCatcodeInvalid#1#2{%
  \count@=#1\relax
  \loop
    \catcode\count@=15 %
  \ifnum\count@<#2\relax
    \advance\count@ 1 %
  \repeat
}
\def\RangeCatcodeCheck#1#2#3{%
  \count@=#1\relax
  \loop
    \ifnum#3=\catcode\count@
    \else
      \errmessage{%
        Character \the\count@\space
        with wrong catcode \the\catcode\count@\space
        instead of \number#3%
      }%
    \fi
  \ifnum\count@<#2\relax
    \advance\count@ 1 %
  \repeat
}
\def\space{ }
\expandafter\ifx\csname LoadCommand\endcsname\relax
  \def\LoadCommand{\input refcount.sty\relax}%
\fi
\def\Test{%
  \RangeCatcodeInvalid{0}{47}%
  \RangeCatcodeInvalid{58}{64}%
  \RangeCatcodeInvalid{91}{96}%
  \RangeCatcodeInvalid{123}{255}%
  \catcode`\@=12 %
  \catcode`\\=0 %
  \catcode`\%=14 %
  \LoadCommand
  \RangeCatcodeCheck{0}{36}{15}%
  \RangeCatcodeCheck{37}{37}{14}%
  \RangeCatcodeCheck{38}{47}{15}%
  \RangeCatcodeCheck{48}{57}{12}%
  \RangeCatcodeCheck{58}{63}{15}%
  \RangeCatcodeCheck{64}{64}{12}%
  \RangeCatcodeCheck{65}{90}{11}%
  \RangeCatcodeCheck{91}{91}{15}%
  \RangeCatcodeCheck{92}{92}{0}%
  \RangeCatcodeCheck{93}{96}{15}%
  \RangeCatcodeCheck{97}{122}{11}%
  \RangeCatcodeCheck{123}{255}{15}%
  \RestoreCatcodes
}
\Test
\csname @@end\endcsname
\end
%    \end{macrocode}
%    \begin{macrocode}
%</test1>
%    \end{macrocode}
%
% \subsection{Macro tests}
%
%    \begin{macrocode}
%<*test2>
\errorcontextlines=10000 %
\showboxbreadth=10000 %
\showboxdepth=10000 %
\begingroup\expandafter\expandafter\expandafter\endgroup
\expandafter\ifx\csname RequirePackage\endcsname\relax
  \input refcount.sty\relax
\else
  \RequirePackage{refcount}[2011/10/16]%
\fi
\catcode`\@=11 %
\begingroup\expandafter\expandafter\expandafter\endgroup
\expandafter\ifx\csname @onelevel@sanitize\endcsname\relax
  \begingroup\expandafter\expandafter\expandafter\endgroup
  \expandafter\ifx\csname detokenize\endcsname\relax
    \def\strip@prefix#1->{}%
    \def\@onelevel@sanitize#1{%
      \edef#1{%
        \expandafter\strip@prefix\meaning#1%
      }%
    }%
  \else
    \def\@onelevel@sanitize#1{%
      \edef#1{%
        \detokenize\expandafter{#1}%
      }%
    }%
  \fi
\fi
\def\msg#{\immediate\write16}
\def\empty{}
\def\space{ }
%    \end{macrocode}
%    \begin{macrocode}
\def\r@foo{{\empty 1}{\empty 2}}
\long\def\test#1#2{%
  \begingroup
    \setbox0=\hbox{%
      \def\TestTask{#1}%
      \@onelevel@sanitize\TestTask
      \msg{* \TestTask}%
      \expandafter\expandafter\expandafter\def
      \expandafter\expandafter\expandafter\TestResult
      \expandafter\expandafter\expandafter{%
        #1%
      }%
      \def\TestExpected{#2}%
      \ifx\TestResult\TestExpected
        \msg{ \space ok.}%
      \else
        \@onelevel@sanitize\TestResult
        \@onelevel@sanitize\TestExpected
        \msg{ \space Result: \space\space[\TestResult]}%
        \msg{ \space Expected: [\TestExpected]}%
        \errmessage{Test failed!}%
      \fi
    }%
    \ifdim\wd0=0pt %
    \else
      \showbox0 %
    \fi
  \endgroup
}
\test{\getrefnumber{foo}}{\empty 1}
\test{\getpagerefnumber{foo}}{\empty 2}
\test{\getrefbykeydefault{foo}{}{\empty default}}{\empty 1}
\test{\getrefbykeydefault{foo}{page}{\empty default}}{\empty 2}
\test{\getrefbykeydefault{foo}{name}{\empty default}}{\empty default}
\test{\getrefbykeydefault{foo}{anchor}{\empty default}}{\empty default}
\test{\getrefbykeydefault{foo}{url}{\empty default}}{\empty default}
\test{\getrefbykeydefault{foo}{title}{\empty default}}{\empty default}
\msg{}
\def\r@foo{{}{}{}{}{}{}{}{}{}{}}
\def\Test#1#2\\{%
  \test{#1{foo}#2}{}%
}
\def\TestGroup{%
  \Test\getrefnumber\\%
  \Test\getpagerefnumber\\%
  \Test\getrefbykeydefault{}{}\\%
  \Test\getrefbykeydefault{page}{}\\%
  \Test\getrefbykeydefault{anchor}{}\\%
  \Test\getrefbykeydefault{name}{}\\%
  \Test\getrefbykeydefault{url}{}\\%
}
\TestGroup
\Test\getrefbykeydefault{title}{}\\%
\msg{}
\def\r@foo{\par\par\par\par\par\par\par\par}
\long\def\Test#1#2\\{%
  \test{#1{foo}#2}{\par}%
}
\TestGroup
\test{\getrefbykeydefault{title}{}{}}{}
\msg{}
\def\r@foo{{ }{ }{ }{ }{ }}
\def\Test#1#2\\{%
  \test{#1{foo}#2}{ }%
}
\TestGroup
\msg{}
\long\def\TestDefault#1{%
  \begingroup
    \setrefcountdefault{#1}%
    \test{\getrefnumber{foo}}{#1}%
    \test{\getpagerefnumber{foo}}{#1}%
  \endgroup
}
\def\TestDefaultX{%
  \TestDefault{}%
  \TestDefault{\par}%
  \TestDefault{ }%
  \TestDefault{\space}%
}
\let\r@foo\@undefined
\TestDefaultX
\let\r@foo\relax
\TestDefaultX
\def\r@foo{}
\TestDefaultX
%    \end{macrocode}
%    \begin{macrocode}
\msg{}
\long\def\Test#1#2#3#4{%
  \begingroup
    \def\TestTask{#1}%
    \@onelevel@sanitize\TestTask
    \msg{* [\TestTask]}%
    \edef\TestResultA{\IfRefUndefinedExpandable{#1}{#2}{#3}}%
    \IfRefUndefinedBabel{#1}{%
      \def\TestResultB{#2}%
    }{%
      \def\TestResultB{#3}%
    }%
    \def\TestExpected{#4}%
    \ifx\TestResultA\TestExpected
      \msg{ \space ok.}%
    \else
      \begingroup
        \@onelevel@sanitize\TestResultA
        \@onelevel@sanitize\TestExpected
        \msg{ \space Result: \space\space[\TestResultA]}%
        \msg{ \space Expected: [\TestExpected]}%
        \errmessage{Test failed!}%
      \endgroup
    \fi
    \ifx\TestResultB\TestExpected
      \msg{ \space ok.}%
    \else
      \begingroup
        \@onelevel@sanitize\TestResultB
        \@onelevel@sanitize\TestExpected
        \msg{ \space Result: \space\space[\TestResultB]}%
        \msg{ \space Expected: [\TestExpected]}%
        \errmessage{Test failed!}%
      \endgroup
    \fi
  \endgroup
}
\begingroup
  \def\r@foo{{}{}}%
  \let\r@bar\@undefined
  \let\r@xyz\relax
  \Test{foo}{true}{false}{false}%
  \Test{bar}{true}{false}{true}%
  \Test{xyz}{true}{false}{true}%
\endgroup
%    \end{macrocode}
%    \begin{macrocode}
\csname @@end\endcsname\end
%</test2>
%    \end{macrocode}
% \subsection{Test with package \xpackage{titleref}}
%
%    \begin{macrocode}
%<*test3>
\NeedsTeXFormat{LaTeX2e}
\documentclass{article}
\usepackage{refcount}[2011/10/16]
%<test4>\usepackage{nameref}
\usepackage{titleref}
\begin{document}
\section{Hello World}
\label{sec:hello}
\section{\hbox{xy}}
\label{sec:foo}
%
\makeatletter
\@ifundefined{r@sec:hello}{%
  \typeout{==> Compile twice!}%
}{%
  \def\test#1#2{%
    \begingroup
      \def\TestTask{#1}%
      \@onelevel@sanitize\TestTask
      \typeout{* \TestTask}%
      \expandafter\expandafter\expandafter\def
      \expandafter\expandafter\expandafter\TestResult
      \expandafter\expandafter\expandafter{%
        #1%
      }%
      \def\TestExpected{#2}%
      \ifx\TestResult\TestExpected
        \typeout{ \space ok.}%
      \else
        \@onelevel@sanitize\TestResult
        \@onelevel@sanitize\TestExpected
        \typeout{ \space Result: \space\space[\TestResult]}%
        \typeout{ \space Expected: [\TestExpected]}%
        \errmessage{Test failed!}%
      \fi
    \endgroup
  }%
  \test{\getrefbykeydefault{sec:hello}{title}{}}{Hello World}%
  \test{\getrefbykeydefault{sec:foo}{title}{}}{\hbox{xy}}%
  \begingroup
    \def\hbox#1{[#1]}% hash-ok
    \test{\getrefbykeydefault{sec:foo}{title}{}}{\hbox{xy}}%
  \endgroup
}
\makeatother
%    \end{macrocode}
%    \begin{macrocode}
\end{document}
%</test3>
%    \end{macrocode}
%    \begin{macrocode}
%<*test5>
\NeedsTeXFormat{LaTeX2e}
\documentclass{book}
\usepackage{refcount}[2011/10/16]
\usepackage{zref-runs}
\newcounter{test}
\begin{document}
\ifnum\zruns>1 %
  \makeatletter
  \def\Test#1#2#3{%
    \begingroup
      \setcounter{test}{10}%
      \sbox0{%
        #1{test}{#2}%
        \ifnum#3=\value{test}%
        \else
          \PackageError{test}{\string#1{#2} <> #3 (\the\value{test})}%
        \fi
      }%
      \ifdim\wd0=0pt %
      \else
        \PackageError{test}{Non-empty box}\@ehc
      \fi
    \endgroup
  }%
  \makeatother
  \Test\setcounterpageref{ch:two}{1}%
  \Test\setcounterpageref{ch:three}{3}%
  \Test\setcounterpageref{ch:four}{5}%
  \Test\setcounterpageref{ch:five}{7}%
  \Test\setcounterpageref{ch:six}{9}%
  \Test\setcounterpageref{ch:seven}{13}%
  \Test\addtocounterpageref{ch:two}{11}%
  \Test\addtocounterpageref{ch:three}{13}%
  \Test\addtocounterpageref{ch:four}{15}%
  \Test\addtocounterpageref{ch:five}{17}%
  \Test\addtocounterpageref{ch:six}{19}%
  \Test\addtocounterpageref{ch:seven}{23}%
  \Test\setcounterref{ch:two}{1}%
  \Test\setcounterref{ch:three}{2}%
  \Test\setcounterref{ch:four}{11}%
  \Test\addtocounterref{ch:two}{11}%
  \Test\addtocounterref{ch:three}{12}%
  \Test\addtocounterref{ch:four}{21}%
\fi
\frontmatter
\chapter{Chapter one}\label{ch:one}
\cleardoublepage
\mainmatter
\chapter{Chapter two}\label{ch:two}
\cleardoublepage
\chapter{Chapter three}\label{ch:three}
\cleardoublepage
\setcounter{chapter}{10}
\chapter{Chapter four}\label{ch:four}
\cleardoublepage
\appendix
\chapter{Chapter five}\label{ch:five}
\cleardoublepage
\chapter{Chapter six}\label{ch:six}
\cleardoublepage
\null
\cleardoublepage
\chapter{Chapter seven}\label{ch:seven}
\end{document}
%</test5>
%    \end{macrocode}
%
% \section{Installation}
%
% \subsection{Download}
%
% \paragraph{Package.} This package is available on
% CTAN\footnote{\url{ftp://ftp.ctan.org/tex-archive/}}:
% \begin{description}
% \item[\CTAN{macros/latex/contrib/oberdiek/refcount.dtx}] The source file.
% \item[\CTAN{macros/latex/contrib/oberdiek/refcount.pdf}] Documentation.
% \end{description}
%
%
% \paragraph{Bundle.} All the packages of the bundle `oberdiek'
% are also available in a TDS compliant ZIP archive. There
% the packages are already unpacked and the documentation files
% are generated. The files and directories obey the TDS standard.
% \begin{description}
% \item[\CTAN{install/macros/latex/contrib/oberdiek.tds.zip}]
% \end{description}
% \emph{TDS} refers to the standard ``A Directory Structure
% for \TeX\ Files'' (\CTAN{tds/tds.pdf}). Directories
% with \xfile{texmf} in their name are usually organized this way.
%
% \subsection{Bundle installation}
%
% \paragraph{Unpacking.} Unpack the \xfile{oberdiek.tds.zip} in the
% TDS tree (also known as \xfile{texmf} tree) of your choice.
% Example (linux):
% \begin{quote}
%   |unzip oberdiek.tds.zip -d ~/texmf|
% \end{quote}
%
% \paragraph{Script installation.}
% Check the directory \xfile{TDS:scripts/oberdiek/} for
% scripts that need further installation steps.
% Package \xpackage{attachfile2} comes with the Perl script
% \xfile{pdfatfi.pl} that should be installed in such a way
% that it can be called as \texttt{pdfatfi}.
% Example (linux):
% \begin{quote}
%   |chmod +x scripts/oberdiek/pdfatfi.pl|\\
%   |cp scripts/oberdiek/pdfatfi.pl /usr/local/bin/|
% \end{quote}
%
% \subsection{Package installation}
%
% \paragraph{Unpacking.} The \xfile{.dtx} file is a self-extracting
% \docstrip\ archive. The files are extracted by running the
% \xfile{.dtx} through \plainTeX:
% \begin{quote}
%   \verb|tex refcount.dtx|
% \end{quote}
%
% \paragraph{TDS.} Now the different files must be moved into
% the different directories in your installation TDS tree
% (also known as \xfile{texmf} tree):
% \begin{quote}
% \def\t{^^A
% \begin{tabular}{@{}>{\ttfamily}l@{ $\rightarrow$ }>{\ttfamily}l@{}}
%   refcount.sty & tex/latex/oberdiek/refcount.sty\\
%   refcount.pdf & doc/latex/oberdiek/refcount.pdf\\
%   test/refcount-test1.tex & doc/latex/oberdiek/test/refcount-test1.tex\\
%   test/refcount-test2.tex & doc/latex/oberdiek/test/refcount-test2.tex\\
%   test/refcount-test3.tex & doc/latex/oberdiek/test/refcount-test3.tex\\
%   test/refcount-test4.tex & doc/latex/oberdiek/test/refcount-test4.tex\\
%   test/refcount-test5.tex & doc/latex/oberdiek/test/refcount-test5.tex\\
%   refcount.dtx & source/latex/oberdiek/refcount.dtx\\
% \end{tabular}^^A
% }^^A
% \sbox0{\t}^^A
% \ifdim\wd0>\linewidth
%   \begingroup
%     \advance\linewidth by\leftmargin
%     \advance\linewidth by\rightmargin
%   \edef\x{\endgroup
%     \def\noexpand\lw{\the\linewidth}^^A
%   }\x
%   \def\lwbox{^^A
%     \leavevmode
%     \hbox to \linewidth{^^A
%       \kern-\leftmargin\relax
%       \hss
%       \usebox0
%       \hss
%       \kern-\rightmargin\relax
%     }^^A
%   }^^A
%   \ifdim\wd0>\lw
%     \sbox0{\small\t}^^A
%     \ifdim\wd0>\linewidth
%       \ifdim\wd0>\lw
%         \sbox0{\footnotesize\t}^^A
%         \ifdim\wd0>\linewidth
%           \ifdim\wd0>\lw
%             \sbox0{\scriptsize\t}^^A
%             \ifdim\wd0>\linewidth
%               \ifdim\wd0>\lw
%                 \sbox0{\tiny\t}^^A
%                 \ifdim\wd0>\linewidth
%                   \lwbox
%                 \else
%                   \usebox0
%                 \fi
%               \else
%                 \lwbox
%               \fi
%             \else
%               \usebox0
%             \fi
%           \else
%             \lwbox
%           \fi
%         \else
%           \usebox0
%         \fi
%       \else
%         \lwbox
%       \fi
%     \else
%       \usebox0
%     \fi
%   \else
%     \lwbox
%   \fi
% \else
%   \usebox0
% \fi
% \end{quote}
% If you have a \xfile{docstrip.cfg} that configures and enables \docstrip's
% TDS installing feature, then some files can already be in the right
% place, see the documentation of \docstrip.
%
% \subsection{Refresh file name databases}
%
% If your \TeX~distribution
% (\teTeX, \mikTeX, \dots) relies on file name databases, you must refresh
% these. For example, \teTeX\ users run \verb|texhash| or
% \verb|mktexlsr|.
%
% \subsection{Some details for the interested}
%
% \paragraph{Attached source.}
%
% The PDF documentation on CTAN also includes the
% \xfile{.dtx} source file. It can be extracted by
% AcrobatReader 6 or higher. Another option is \textsf{pdftk},
% e.g. unpack the file into the current directory:
% \begin{quote}
%   \verb|pdftk refcount.pdf unpack_files output .|
% \end{quote}
%
% \paragraph{Unpacking with \LaTeX.}
% The \xfile{.dtx} chooses its action depending on the format:
% \begin{description}
% \item[\plainTeX:] Run \docstrip\ and extract the files.
% \item[\LaTeX:] Generate the documentation.
% \end{description}
% If you insist on using \LaTeX\ for \docstrip\ (really,
% \docstrip\ does not need \LaTeX), then inform the autodetect routine
% about your intention:
% \begin{quote}
%   \verb|latex \let\install=y\input{refcount.dtx}|
% \end{quote}
% Do not forget to quote the argument according to the demands
% of your shell.
%
% \paragraph{Generating the documentation.}
% You can use both the \xfile{.dtx} or the \xfile{.drv} to generate
% the documentation. The process can be configured by the
% configuration file \xfile{ltxdoc.cfg}. For instance, put this
% line into this file, if you want to have A4 as paper format:
% \begin{quote}
%   \verb|\PassOptionsToClass{a4paper}{article}|
% \end{quote}
% An example follows how to generate the
% documentation with pdf\LaTeX:
% \begin{quote}
%\begin{verbatim}
%pdflatex refcount.dtx
%makeindex -s gind.ist refcount.idx
%pdflatex refcount.dtx
%makeindex -s gind.ist refcount.idx
%pdflatex refcount.dtx
%\end{verbatim}
% \end{quote}
%
% \section{Catalogue}
%
% The following XML file can be used as source for the
% \href{http://mirror.ctan.org/help/Catalogue/catalogue.html}{\TeX\ Catalogue}.
% The elements \texttt{caption} and \texttt{description} are imported
% from the original XML file from the Catalogue.
% The name of the XML file in the Catalogue is \xfile{refcount.xml}.
%    \begin{macrocode}
%<*catalogue>
<?xml version='1.0' encoding='us-ascii'?>
<!DOCTYPE entry SYSTEM 'catalogue.dtd'>
<entry datestamp='$Date$' modifier='$Author$' id='refcount'>
  <name>refcount</name>
  <caption>Counter operations with label references.</caption>
  <authorref id='auth:oberdiek'/>
  <copyright owner='Heiko Oberdiek' year='1998,2000,2006,2008,2010,2011'/>
  <license type='lppl1.3'/>
  <version number='3.4'/>
  <description>
    Provides commands <tt>\setcounterref</tt> and
    <tt>\addtocounterref</tt> which use the section (or whatever)
    number from the reference as the value to put into the counter, as
    in:

    <pre>
    ...\label{sec:foo}
    ...
    \setcounterref{foonum}{sec:foo}
    </pre>
    Commands <tt>\setcounterpageref</tt> and
    <tt>\addtocounterpageref</tt> do the corresponding thing with the
    page reference of the label.
    <p/>
    No <tt>.ins</tt> file is distributed; process the
    <tt>.dtx</tt> with plain TeX to create one.
    <p/>
    The package is part of the <xref refid='oberdiek'>oberdiek</xref>
    bundle.
  </description>
  <documentation details='Package documentation'
      href='ctan:/macros/latex/contrib/oberdiek/refcount.pdf'/>
  <ctan file='true' path='/macros/latex/contrib/oberdiek/refcount.dtx'/>
  <miktex location='oberdiek'/>
  <texlive location='oberdiek'/>
  <install path='/macros/latex/contrib/oberdiek/oberdiek.tds.zip'/>
</entry>
%</catalogue>
%    \end{macrocode}
%
% \begin{History}
%   \begin{Version}{1998/04/08 v1.0}
%   \item
%     First public release, written as answer in the
%     newsgroup \xnewsgroup{comp.text.tex}:
%     \URL{``\link{Re: Adding a \cs{ref} to a counter?}''}^^A
%     {http://groups.google.com/group/comp.text.tex/msg/c3f2a135ef5ee528}
%   \end{Version}
%   \begin{Version}{2000/09/07 v2.0}
%   \item
%     Documentation added.
%   \item
%     LPPL 1.2
%   \item
%     Package rewritten, new commands added.
%   \end{Version}
%   \begin{Version}{2006/02/20 v3.0}
%   \item
%     Support for \xpackage{hyperref} and \xpackage{nameref} improved.
%   \item
%     Support for \xpackage{titleref} and \xpackage{babel}'s shorthands added.
%   \item
%     New: \cs{refused}, \cs{getrefbykeydefault}
%   \end{Version}
%   \begin{Version}{2008/08/11 v3.1}
%   \item
%     Code is not changed.
%   \item
%     URLs updated.
%   \end{Version}
%   \begin{Version}{2010/12/01 v3.2}
%   \item
%     \cs{IfRefUndefinedExpandable} and \cs{IfRefUndefinedBabel} added.
%   \item
%     \cs{getrefnumber}, \cs{getpagerefnumber}, \cs{getrefbykeydefault}
%     are expandable in exact two expansion steps.
%   \item
%     Non-expandable macros are made robust.
%   \item
%     Test files added.
%   \end{Version}
%   \begin{Version}{2011/06/22 v3.3}
%   \item
%     Bug fix: \cs{rc@refused} is undefined for \cs{setcounterpageref}
%     and similar macros. (Bug found by Marc van Dongen.)
%   \end{Version}
%   \begin{Version}{2011/10/16 v3.4}
%   \item
%     Bug fix: \cs{setcounterpageref} and \cs{addtocounterpageref} fixed.
%     (Bug found by Staz.)
%   \item
%     Macros \cs{(set|addto)counter(page|)ref} are made robust.
%   \end{Version}
% \end{History}
%
% \PrintIndex
%
% \Finale
\endinput

%        (quote the arguments according to the demands of your shell)
%
% Documentation:
%    (a) If refcount.drv is present:
%           latex refcount.drv
%    (b) Without refcount.drv:
%           latex refcount.dtx; ...
%    The class ltxdoc loads the configuration file ltxdoc.cfg
%    if available. Here you can specify further options, e.g.
%    use A4 as paper format:
%       \PassOptionsToClass{a4paper}{article}
%
%    Programm calls to get the documentation (example):
%       pdflatex refcount.dtx
%       makeindex -s gind.ist refcount.idx
%       pdflatex refcount.dtx
%       makeindex -s gind.ist refcount.idx
%       pdflatex refcount.dtx
%
% Installation:
%    TDS:tex/latex/oberdiek/refcount.sty
%    TDS:doc/latex/oberdiek/refcount.pdf
%    TDS:doc/latex/oberdiek/test/refcount-test1.tex
%    TDS:doc/latex/oberdiek/test/refcount-test2.tex
%    TDS:doc/latex/oberdiek/test/refcount-test3.tex
%    TDS:doc/latex/oberdiek/test/refcount-test4.tex
%    TDS:doc/latex/oberdiek/test/refcount-test5.tex
%    TDS:source/latex/oberdiek/refcount.dtx
%
%<*ignore>
\begingroup
  \catcode123=1 %
  \catcode125=2 %
  \def\x{LaTeX2e}%
\expandafter\endgroup
\ifcase 0\ifx\install y1\fi\expandafter
         \ifx\csname processbatchFile\endcsname\relax\else1\fi
         \ifx\fmtname\x\else 1\fi\relax
\else\csname fi\endcsname
%</ignore>
%<*install>
\input docstrip.tex
\Msg{************************************************************************}
\Msg{* Installation}
\Msg{* Package: refcount 2011/10/16 v3.4 Data extraction from label references (HO)}
\Msg{************************************************************************}

\keepsilent
\askforoverwritefalse

\let\MetaPrefix\relax
\preamble

This is a generated file.

Project: refcount
Version: 2011/10/16 v3.4

Copyright (C) 1998, 2000, 2006, 2008, 2010, 2011 by
   Heiko Oberdiek <heiko.oberdiek at googlemail.com>

This work may be distributed and/or modified under the
conditions of the LaTeX Project Public License, either
version 1.3c of this license or (at your option) any later
version. This version of this license is in
   http://www.latex-project.org/lppl/lppl-1-3c.txt
and the latest version of this license is in
   http://www.latex-project.org/lppl.txt
and version 1.3 or later is part of all distributions of
LaTeX version 2005/12/01 or later.

This work has the LPPL maintenance status "maintained".

This Current Maintainer of this work is Heiko Oberdiek.

This work consists of the main source file refcount.dtx
and the derived files
   refcount.sty, refcount.pdf, refcount.ins, refcount.drv,
   refcount-test1.tex, refcount-test2.tex, refcount-test3.tex,
   refcount-test4.tex, refcount-test5.tex.

\endpreamble
\let\MetaPrefix\DoubleperCent

\generate{%
  \file{refcount.ins}{\from{refcount.dtx}{install}}%
  \file{refcount.drv}{\from{refcount.dtx}{driver}}%
  \usedir{tex/latex/oberdiek}%
  \file{refcount.sty}{\from{refcount.dtx}{package}}%
  \usedir{doc/latex/oberdiek/test}%
  \file{refcount-test1.tex}{\from{refcount.dtx}{test1}}%
  \file{refcount-test2.tex}{\from{refcount.dtx}{test2}}%
  \file{refcount-test3.tex}{\from{refcount.dtx}{test3}}%
  \file{refcount-test4.tex}{\from{refcount.dtx}{test3,test4}}%
  \file{refcount-test5.tex}{\from{refcount.dtx}{test5}}%
  \nopreamble
  \nopostamble
  \usedir{source/latex/oberdiek/catalogue}%
  \file{refcount.xml}{\from{refcount.dtx}{catalogue}}%
}

\catcode32=13\relax% active space
\let =\space%
\Msg{************************************************************************}
\Msg{*}
\Msg{* To finish the installation you have to move the following}
\Msg{* file into a directory searched by TeX:}
\Msg{*}
\Msg{*     refcount.sty}
\Msg{*}
\Msg{* To produce the documentation run the file `refcount.drv'}
\Msg{* through LaTeX.}
\Msg{*}
\Msg{* Happy TeXing!}
\Msg{*}
\Msg{************************************************************************}

\endbatchfile
%</install>
%<*ignore>
\fi
%</ignore>
%<*driver>
\NeedsTeXFormat{LaTeX2e}
\ProvidesFile{refcount.drv}%
  [2011/10/16 v3.4 Data extraction from label references (HO)]%
\documentclass{ltxdoc}
\usepackage{holtxdoc}[2011/11/22]
\begin{document}
  \DocInput{refcount.dtx}%
\end{document}
%</driver>
% \fi
%
% \CheckSum{1185}
%
% \CharacterTable
%  {Upper-case    \A\B\C\D\E\F\G\H\I\J\K\L\M\N\O\P\Q\R\S\T\U\V\W\X\Y\Z
%   Lower-case    \a\b\c\d\e\f\g\h\i\j\k\l\m\n\o\p\q\r\s\t\u\v\w\x\y\z
%   Digits        \0\1\2\3\4\5\6\7\8\9
%   Exclamation   \!     Double quote  \"     Hash (number) \#
%   Dollar        \$     Percent       \%     Ampersand     \&
%   Acute accent  \'     Left paren    \(     Right paren   \)
%   Asterisk      \*     Plus          \+     Comma         \,
%   Minus         \-     Point         \.     Solidus       \/
%   Colon         \:     Semicolon     \;     Less than     \<
%   Equals        \=     Greater than  \>     Question mark \?
%   Commercial at \@     Left bracket  \[     Backslash     \\
%   Right bracket \]     Circumflex    \^     Underscore    \_
%   Grave accent  \`     Left brace    \{     Vertical bar  \|
%   Right brace   \}     Tilde         \~}
%
% \GetFileInfo{refcount.drv}
%
% \title{The \xpackage{refcount} package}
% \date{2011/10/16 v3.4}
% \author{Heiko Oberdiek\\\xemail{heiko.oberdiek at googlemail.com}}
%
% \maketitle
%
% \begin{abstract}
% References are not numbers, however they often store numerical
% data such as section or page numbers. \cs{ref} or \cs{pageref}
% cannot be used for counter assignments or calculations because
% they are not expandable, generate warnings, or can even be links.
% The package provides expandable macros to extract the data
% from references. Packages \xpackage{hyperref}, \xpackage{nameref},
% \xpackage{titleref}, and \xpackage{babel} are supported.
% \end{abstract}
%
% \tableofcontents
%
% \section{Usage}
%
% \subsection{Setting counters}
%
% The following commands are similar to \LaTeX's
% \cs{setcounter} and \cs{addtocounter},
% but they extract the number value from a reference:
% \begin{quote}
%   \cs{setcounterref}, \cs{addtocounterref}\\
%   \cs{setcounterpageref}, \cs{addtocounterpageref}
% \end{quote}
% They take two arguments:
% \begin{quote}
%    \cs{...counter...ref} |{|\meta{\LaTeX\ counter}|}|
%    |{|\meta{reference}|}|
% \end{quote}
% An undefined references produces the usual LaTeX warning
% and its value is assumed to be zero.
% Example:
% \begin{quote}
%\begin{verbatim}
%\newcounter{ctrA}
%\newcounter{ctrB}
%\refstepcounter{ctrA}\label{ref:A}
%\setcounterref{ctrB}{ref:A}
%\addtocounterpageref{ctrB}{ref:A}
%\end{verbatim}
% \end{quote}
%
% \subsection{Expandable commands}
%
% These commands that can be used in expandible contexts
% (inside calculations, \cs{edef}, \cs{csname}, \cs{write}, \dots):
% \begin{quote}
%   \cs{getrefnumber}, \cs{getpagerefnumber}
% \end{quote}
% They take one argument, the reference:
% \begin{quote}
%   \cs{get...refnumber} |{|\meta{reference}|}|
% \end{quote}
% The default for undefined references can be changed
% with macro \cs{setrefcountdefault}, for example this
% package calls:
% \begin{quote}
%   \cs{setrefcountdefault}|{0}|
% \end{quote}
%
% Since version 2.0 of this package there is a new
% command:
% \begin{quote}
%   \cs{getrefbykeydefault} |{|\meta{reference}|}|
%   |{|\meta{key}|}| |{|\meta{default}|}|
% \end{quote}
% This generalized version allows the extraction
% of further properties of a reference than the
% two standard ones. Thus the following properties
% are supported, if they are available:
% \begin{quote}
% \begin{tabular}{@{}l|l|l@{}}
%    Key & Description & Package\\
% \hline
%   \meta{empty} & same as \cs{ref} & \LaTeX\\
%   |page| & same as \cs{pageref} & \LaTeX\\
%   |title| & section and caption titles & \xpackage{titleref}\\
%   |name| & section and caption titles & \xpackage{nameref}\\
%   |anchor| & anchor name & \xpackage{hyperref}\\
%   |url| & url/file & \xpackage{hyperref}/\xpackage{xr}
% \end{tabular}
% \end{quote}
%
% Since version 3.2 the expandable macros described before
% in this section
% are expandable in exact two expansion steps.
%
% \subsection{Undefined references}
%
% Because warnings and assignments cannot be used in
% expandible contexts, undefined references do not
% produce a warning, their values are assumed to be zero.
% Example:
% \begin{quote}
%\begin{verbatim}
%\label{ref:here}% somewhere
%\refused{ref:here}% see below
%\ifodd\getpagerefnumber{ref:here}%
%  reference is on an odd page
%\else
%  reference is on an even page
%\fi
%\end{verbatim}
% \end{quote}
%
% In case of undefined references the user usually want's
% to be informed. Also \LaTeX\ prints a warning at
% the end of the \LaTeX\ run. To notify \LaTeX\ and
% get a normal warning, just use
% \begin{quote}
%   \cs{refused} |{|\meta{reference}|}|
% \end{quote}
% outside the expanding context. Example, see above.
%
% \subsubsection{Check for undefined references}
%
% In version 3.2 macros were added, that test, whether references
% are defined.
% \begin{declcs}{IfRefUndefinedExpandable} \M{refname} \M{then} \M{else}\\
%   \cs{IfRefUndefinedBabel} \M{refname} \M{then} \M{else}
% \end{declcs}
% If the reference is not available and therefore undefined, then
% argument \meta{then} is executed, otherwise argument \meta{else}
% is called. Macro \cs{IfRefUndefinedExpandable} is expandable,
% but \meta{refname} must not contain babel shorthand characters.
% Macro \cs{IfRefUndefinedBabel} supports shorthand characters of
% babel, but it is not expandable.
%
% \subsection{Notes}
%
% \begin{itemize}
% \item
%   The method of extracting the number in this
%   package also works in cases, where the
%   reference cannot be used directly, because
%   a package such as \xpackage{hyperref} has added
%   extra stuff (hyper link), so that the reference cannot
%   be used as number any more.
% \item
%   If the reference does not contain a number,
%   assignments to a counter will fail of course.
% \end{itemize}
%
%
% \StopEventually{
% }
%
% \section{Implementation}
%
%    \begin{macrocode}
%<*package>
%    \end{macrocode}
%    Reload check, especially if the package is not used with \LaTeX.
%    \begin{macrocode}
\begingroup\catcode61\catcode48\catcode32=10\relax%
  \catcode13=5 % ^^M
  \endlinechar=13 %
  \catcode35=6 % #
  \catcode39=12 % '
  \catcode44=12 % ,
  \catcode45=12 % -
  \catcode46=12 % .
  \catcode58=12 % :
  \catcode64=11 % @
  \catcode123=1 % {
  \catcode125=2 % }
  \expandafter\let\expandafter\x\csname ver@refcount.sty\endcsname
  \ifx\x\relax % plain-TeX, first loading
  \else
    \def\empty{}%
    \ifx\x\empty % LaTeX, first loading,
      % variable is initialized, but \ProvidesPackage not yet seen
    \else
      \expandafter\ifx\csname PackageInfo\endcsname\relax
        \def\x#1#2{%
          \immediate\write-1{Package #1 Info: #2.}%
        }%
      \else
        \def\x#1#2{\PackageInfo{#1}{#2, stopped}}%
      \fi
      \x{refcount}{The package is already loaded}%
      \aftergroup\endinput
    \fi
  \fi
\endgroup%
%    \end{macrocode}
%    Package identification:
%    \begin{macrocode}
\begingroup\catcode61\catcode48\catcode32=10\relax%
  \catcode13=5 % ^^M
  \endlinechar=13 %
  \catcode35=6 % #
  \catcode39=12 % '
  \catcode40=12 % (
  \catcode41=12 % )
  \catcode44=12 % ,
  \catcode45=12 % -
  \catcode46=12 % .
  \catcode47=12 % /
  \catcode58=12 % :
  \catcode64=11 % @
  \catcode91=12 % [
  \catcode93=12 % ]
  \catcode123=1 % {
  \catcode125=2 % }
  \expandafter\ifx\csname ProvidesPackage\endcsname\relax
    \def\x#1#2#3[#4]{\endgroup
      \immediate\write-1{Package: #3 #4}%
      \xdef#1{#4}%
    }%
  \else
    \def\x#1#2[#3]{\endgroup
      #2[{#3}]%
      \ifx#1\@undefined
        \xdef#1{#3}%
      \fi
      \ifx#1\relax
        \xdef#1{#3}%
      \fi
    }%
  \fi
\expandafter\x\csname ver@refcount.sty\endcsname
\ProvidesPackage{refcount}%
  [2011/10/16 v3.4 Data extraction from label references (HO)]%
%    \end{macrocode}
%
%    \begin{macrocode}
\begingroup\catcode61\catcode48\catcode32=10\relax%
  \catcode13=5 % ^^M
  \endlinechar=13 %
  \catcode123=1 % {
  \catcode125=2 % }
  \catcode64=11 % @
  \def\x{\endgroup
    \expandafter\edef\csname rc@AtEnd\endcsname{%
      \endlinechar=\the\endlinechar\relax
      \catcode13=\the\catcode13\relax
      \catcode32=\the\catcode32\relax
      \catcode35=\the\catcode35\relax
      \catcode61=\the\catcode61\relax
      \catcode64=\the\catcode64\relax
      \catcode123=\the\catcode123\relax
      \catcode125=\the\catcode125\relax
    }%
  }%
\x\catcode61\catcode48\catcode32=10\relax%
\catcode13=5 % ^^M
\endlinechar=13 %
\catcode35=6 % #
\catcode64=11 % @
\catcode123=1 % {
\catcode125=2 % }
\def\TMP@EnsureCode#1#2{%
  \edef\rc@AtEnd{%
    \rc@AtEnd
    \catcode#1=\the\catcode#1\relax
  }%
  \catcode#1=#2\relax
}
\TMP@EnsureCode{33}{12}% !
\TMP@EnsureCode{39}{12}% '
\TMP@EnsureCode{42}{12}% *
\TMP@EnsureCode{45}{12}% -
\TMP@EnsureCode{46}{12}% .
\TMP@EnsureCode{47}{12}% /
\TMP@EnsureCode{91}{12}% [
\TMP@EnsureCode{93}{12}% ]
\TMP@EnsureCode{96}{12}% `
\edef\rc@AtEnd{\rc@AtEnd\noexpand\endinput}
%    \end{macrocode}
%
% \subsection{Loading packages}
%
%    \begin{macrocode}
\begingroup\expandafter\expandafter\expandafter\endgroup
\expandafter\ifx\csname RequirePackage\endcsname\relax
  \input ltxcmds.sty\relax
  \input infwarerr.sty\relax
\else
  \RequirePackage{ltxcmds}[2011/11/09]%
  \RequirePackage{infwarerr}[2010/04/08]%
\fi
%    \end{macrocode}
%
% \subsection{Defining commands}
%
%    \begin{macro}{\rc@IfDefinable}
%    \begin{macrocode}
\ltx@IfUndefined{@ifdefinable}{%
  \def\rc@IfDefinable#1{%
    \ifx#1\ltx@undefined
      \expandafter\ltx@firstofone
    \else
      \ifx#1\relax
        \expandafter\expandafter\expandafter\ltx@firstofone
      \else
        \@PackageError{refcount}{%
          Command \string#1 is already defined.\MessageBreak
          It will not redefined by this package%
        }\@ehc
        \expandafter\expandafter\expandafter\ltx@gobble
      \fi
    \fi
  }%
}{%
  \let\rc@IfDefinable\@ifdefinable
}
%    \end{macrocode}
%    \end{macro}
%
%    \begin{macro}{\rc@RobustDefOne}
%    \begin{macro}{\rc@RobustDefZero}
%    \begin{macrocode}
\ltx@IfUndefined{protected}{%
  \ltx@IfUndefined{DeclareRobustCommand}{%
    \def\rc@RobustDefOne#1#2#3#4{%
      \rc@IfDefinable#3{%
        #1\def#3##1{#4}%
      }%
    }%
    \def\rc@RobustDefZero#1#2{%
      \rc@IfDefinable#1{%
        \def#1{#2}%
      }%
    }%
  }{%
    \def\rc@RobustDefOne#1#2#3#4{%
      \rc@IfDefinable#3{%
        \DeclareRobustCommand#2#3[1]{#4}%
      }%
    }%
    \def\rc@RobustDefZero#1#2{%
      \rc@IfDefinable#1{%
        \DeclareRobustCommand#1{#2}%
      }%
    }%
  }%
}{%
  \def\rc@RobustDefOne#1#2#3#4{%
    \rc@IfDefinable#3{%
      \protected#1\def#3##1{#4}%
    }%
  }%
  \def\rc@RobustDefZero#1#2{%
    \rc@IfDefinable#1{%
      \protected\def#1{#2}%
    }%
  }%
}
%    \end{macrocode}
%    \end{macro}
%    \end{macro}
%
%    \begin{macro}{\rc@newcommand}
%    \begin{macrocode}
\ltx@IfUndefined{newcommand}{%
  \def\rc@newcommand*#1[#2]#3{% hash-ok
    \rc@IfDefinable#1{%
      \ifcase#2 %
        \def#1{#3}%
      \or
        \def#1##1{#3}%
      \or
        \def#1##1##2{#3}%
      \else
        \rc@InternalError
      \fi
    }%
  }%
}{%
  \let\rc@newcommand\newcommand
}
%    \end{macrocode}
%    \end{macro}
%
% \subsection{\cs{setrefcountdefault}}
%
%    \begin{macro}{\setrefcountdefault}
%    \begin{macrocode}
\rc@RobustDefOne\long{}\setrefcountdefault{%
  \def\rc@default{#1}%
}
%    \end{macrocode}
%    \end{macro}
%    \begin{macrocode}
\setrefcountdefault{0}
%    \end{macrocode}
%
% \subsection{\cs{refused}}
%
%    \begin{macro}{\refused}
%    \begin{macrocode}
\ltx@IfUndefined{G@refundefinedtrue}{%
  \rc@RobustDefOne{}{*}\refused{%
    \begingroup
      \csname @safe@activestrue\endcsname
      \ltx@IfUndefined{r@#1}{%
        \protect\G@refundefinedtrue
        \rc@WarningUndefined{#1}%
      }{}%
    \endgroup
  }%
}{%
  \rc@RobustDefOne{}{*}\refused{%
    \begingroup
      \csname @safe@activestrue\endcsname
      \ltx@IfUndefined{r@#1}{%
        \csname protect\expandafter\endcsname
        \csname G@refundefinedtrue\endcsname
        \rc@WarningUndefined{#1}%
      }{}%
    \endgroup
  }%
}
%    \end{macrocode}
%    \end{macro}
%    \begin{macro}{\rc@WarningUndefined}
%    \begin{macrocode}
\ltx@IfUndefined{@latex@warning}{%
  \def\rc@WarningUndefined#1{%
    \ltx@ifundefined{thepage}{%
      \def\thepage{\number\count0 }%
    }{}%
    \@PackageWarning{refcount}{%
      Reference `#1' on page \thepage\space undefined%
    }%
  }%
}{%
  \def\rc@WarningUndefined#1{%
    \@latex@warning{%
      Reference `#1' on page \thepage\space undefined%
    }%
  }%
}
%    \end{macrocode}
%    \end{macro}
%
% \subsection{Setting counters by reference data}
%
% \subsubsection{Generic setting}
%
%    \begin{macro}{\rc@set}
% Generic command for
% |\|$\{$|set|$,$|addto|$\}$|counter|$\{$|page|$,\}$|ref|:
%\begin{quote}
%\begin{tabular}[t]{@{}l@{: }l@{}}
% |#1|& \cs{setcounter}, \cs{addtocounter}\\
% |#2|& \cs{ltx@car} (for \cs{ref}), \cs{ltx@cartwo} (for \cs{pageref})\\
% |#3|& \hologo{LaTeX} counter\\
% |#4|& reference\\
%\end{tabular}
%\end{quote}
%    \begin{macrocode}
\def\rc@set#1#2#3#4{%
  \begingroup
    \csname @safe@activestrue\endcsname
    \refused{#4}%
    \expandafter\rc@@set\csname r@#4\endcsname{#1}{#2}{#3}%
  \endgroup
}
%    \end{macrocode}
%    \end{macro}
%    \begin{macro}{\rc@@set}
%\begin{quote}
%\begin{tabular}[t]{@{}l@{: }l@{}}
% |#1|& \cs{r@<...>}\\
% |#2|& \cs{setcounter}, \cs{addtocounter}\\
% |#3|& \cs{ltx@car} (for \cs{ref}), \cs{ltx@carsecond} (for \cs{pageref})\\
% |#4|& \hologo{LaTeX} counter\\
%\end{tabular}
%\end{quote}
%    \begin{macrocode}
\def\rc@@set#1#2#3#4{%
  \ifx#1\relax
    #2{#4}{\rc@default}%
  \else
    #2{#4}{%
      \expandafter#3#1\rc@default\rc@default\@nil
    }%
  \fi
}
%    \end{macrocode}
%    \end{macro}
%
% \subsubsection{User commands}
%
%    \begin{macro}{\setcounterref}
%    \begin{macrocode}
\rc@RobustDefZero\setcounterref{%
  \rc@set\setcounter\ltx@car
}
%    \end{macrocode}
%    \end{macro}
%    \begin{macro}{\addtocounterref}
%    \begin{macrocode}
\rc@RobustDefZero\addtocounterref{%
  \rc@set\addtocounter\ltx@car
}
%    \end{macrocode}
%    \end{macro}
%    \begin{macro}{\setcounterpageref}
%    \begin{macrocode}
\rc@RobustDefZero\setcounterpageref{%
  \rc@set\setcounter\ltx@carsecond
}
%    \end{macrocode}
%    \end{macro}
%    \begin{macro}{\addtocounterpageref}
%    \begin{macrocode}
\rc@RobustDefZero\addtocounterpageref{%
  \rc@set\addtocounter\ltx@carsecond
}
%    \end{macrocode}
%    \end{macro}
%
% \subsection{Extracting references}
%
%    \begin{macro}{\getrefnumber}
%    \begin{macrocode}
\rc@newcommand*{\getrefnumber}[1]{%
  \romannumeral
  \ltx@ifundefined{r@#1}{%
    \expandafter\ltx@zero
    \rc@default
  }{%
    \expandafter\expandafter\expandafter\rc@extract@
    \expandafter\expandafter\expandafter!%
    \csname r@#1\expandafter\endcsname
    \expandafter{\rc@default}\@nil
  }%
}
%    \end{macrocode}
%    \end{macro}
%    \begin{macro}{\getpagerefnumber}
%    \begin{macrocode}
\rc@newcommand*{\getpagerefnumber}[1]{%
  \romannumeral
  \ltx@ifundefined{r@#1}{%
    \expandafter\ltx@zero
    \rc@default
  }{%
    \expandafter\expandafter\expandafter\rc@extract@page
    \expandafter\expandafter\expandafter!%
    \csname r@#1\expandafter\expandafter\expandafter\endcsname
    \expandafter\expandafter\expandafter{%
      \expandafter\rc@default
    \expandafter}\expandafter{\rc@default}\@nil
  }%
}
%    \end{macrocode}
%    \end{macro}
%    \begin{macro}{\getrefbykeydefault}
%    \begin{macrocode}
\rc@newcommand*{\getrefbykeydefault}[2]{%
  \romannumeral
  \expandafter\rc@getrefbykeydefault
    \csname r@#1\expandafter\endcsname
    \csname rc@extract@#2\endcsname
}
%    \end{macrocode}
%    \end{macro}
%    \begin{macro}{\rc@getrefbykeydefault}
%\begin{quote}
%\begin{tabular}[t]{@{}l@{: }l@{}}
% |#1|& \cs{r@<...>}\\
% |#2|& \cs{rc@extract@<...>}\\
% |#3|& default\\
%\end{tabular}
%\end{quote}
%    \begin{macrocode}
\long\def\rc@getrefbykeydefault#1#2#3{%
  \ifx#1\relax
    % reference is undefined
    \ltx@ReturnAfterElseFi{%
      \ltx@zero
      #3%
    }%
  \else
    \ltx@ReturnAfterFi{%
      \ifx#2\relax
        % extract method is missing
        \ltx@ReturnAfterElseFi{%
          \ltx@zero
          #3%
        }%
      \else
        \ltx@ReturnAfterFi{%
          \expandafter
          \rc@generic#1{#3}{#3}{#3}{#3}{#3}\@nil#2{#3}%
        }%
      \fi
    }%
  \fi
}
%    \end{macrocode}
%    \end{macro}
%    \begin{macro}{\rc@generic}
%\begin{quote}
%\begin{tabular}[t]{@{}l@{: }l@{}}
% |#1|& first item in \cs{r@<...>}\\
% |#2|& remaining items in \cs{r@<...>}\\
% |#3|& \cs{rc@extract@<...>}\\
% |#4|& default\\
%\end{tabular}
%\end{quote}
%    \begin{macrocode}
\long\def\rc@generic#1#2\@nil#3#4{%
  #3{#1\TR@TitleReference\@empty{#4}\@nil}{#1}#2\@nil
}
%    \end{macrocode}
%    \end{macro}
%    \begin{macro}{\rc@extract@}
%    \begin{macrocode}
\long\def\rc@extract@#1#2#3\@nil{%
  \ltx@zero
  #2%
}
%    \end{macrocode}
%    \end{macro}
%    \begin{macro}{\rc@extract@page}
%    \begin{macrocode}
\long\def\rc@extract@page#1#2#3#4\@nil{%
  \ltx@zero
  #3%
}
%    \end{macrocode}
%    \end{macro}
%    \begin{macro}{\rc@extract@name}
%    \begin{macrocode}
\long\def\rc@extract@name#1#2#3#4#5\@nil{%
  \ltx@zero
  #4%
}
%    \end{macrocode}
%    \end{macro}
%    \begin{macro}{\rc@extract@anchor}
%    \begin{macrocode}
\long\def\rc@extract@anchor#1#2#3#4#5#6\@nil{%
  \ltx@zero
  #5%
}
%    \end{macrocode}
%    \end{macro}
%    \begin{macro}{\rc@extract@url}
%    \begin{macrocode}
\long\def\rc@extract@url#1#2#3#4#5#6#7\@nil{%
  \ltx@zero
  #6%
}
%    \end{macrocode}
%    \end{macro}
%    \begin{macro}{\rc@extract@title}
%    \begin{macrocode}
\long\def\rc@extract@title#1#2\@nil{%
  \rc@@extract@title#1%
}
%    \end{macrocode}
%    \end{macro}
%    \begin{macro}{\rc@@extract@title}
%    \begin{macrocode}
\long\def\rc@@extract@title#1\TR@TitleReference#2#3#4\@nil{%
  \ltx@zero
  #3%
}
%    \end{macrocode}
%    \end{macro}
%
% \subsection{Macros for checking undefined references}
%
%    \begin{macro}{\IfRefUndefinedExpandable}
%    \begin{macrocode}
\rc@newcommand*{\IfRefUndefinedExpandable}[1]{%
  \ltx@ifundefined{r@#1}\ltx@firstoftwo\ltx@secondoftwo
}
%    \end{macrocode}
%    \end{macro}
%    \begin{macro}{\IfRefUndefinedBabel}
%    \begin{macrocode}
\rc@RobustDefOne{}*\IfRefUndefinedBabel{%
  \begingroup
    \csname safe@actives@true\endcsname
  \expandafter\expandafter\expandafter\endgroup
  \expandafter\ifx\csname r@#1\endcsname\relax
    \expandafter\ltx@firstoftwo
  \else
    \expandafter\ltx@secondoftwo
  \fi
}
%    \end{macrocode}
%    \end{macro}
%    \begin{macrocode}
\rc@AtEnd%
%</package>
%    \end{macrocode}
%
% \section{Test}
%
% \subsection{Catcode checks for loading}
%
%    \begin{macrocode}
%<*test1>
%    \end{macrocode}
%    \begin{macrocode}
\catcode`\{=1 %
\catcode`\}=2 %
\catcode`\#=6 %
\catcode`\@=11 %
\expandafter\ifx\csname count@\endcsname\relax
  \countdef\count@=255 %
\fi
\expandafter\ifx\csname @gobble\endcsname\relax
  \long\def\@gobble#1{}%
\fi
\expandafter\ifx\csname @firstofone\endcsname\relax
  \long\def\@firstofone#1{#1}%
\fi
\expandafter\ifx\csname loop\endcsname\relax
  \expandafter\@firstofone
\else
  \expandafter\@gobble
\fi
{%
  \def\loop#1\repeat{%
    \def\body{#1}%
    \iterate
  }%
  \def\iterate{%
    \body
      \let\next\iterate
    \else
      \let\next\relax
    \fi
    \next
  }%
  \let\repeat=\fi
}%
\def\RestoreCatcodes{}
\count@=0 %
\loop
  \edef\RestoreCatcodes{%
    \RestoreCatcodes
    \catcode\the\count@=\the\catcode\count@\relax
  }%
\ifnum\count@<255 %
  \advance\count@ 1 %
\repeat

\def\RangeCatcodeInvalid#1#2{%
  \count@=#1\relax
  \loop
    \catcode\count@=15 %
  \ifnum\count@<#2\relax
    \advance\count@ 1 %
  \repeat
}
\def\RangeCatcodeCheck#1#2#3{%
  \count@=#1\relax
  \loop
    \ifnum#3=\catcode\count@
    \else
      \errmessage{%
        Character \the\count@\space
        with wrong catcode \the\catcode\count@\space
        instead of \number#3%
      }%
    \fi
  \ifnum\count@<#2\relax
    \advance\count@ 1 %
  \repeat
}
\def\space{ }
\expandafter\ifx\csname LoadCommand\endcsname\relax
  \def\LoadCommand{\input refcount.sty\relax}%
\fi
\def\Test{%
  \RangeCatcodeInvalid{0}{47}%
  \RangeCatcodeInvalid{58}{64}%
  \RangeCatcodeInvalid{91}{96}%
  \RangeCatcodeInvalid{123}{255}%
  \catcode`\@=12 %
  \catcode`\\=0 %
  \catcode`\%=14 %
  \LoadCommand
  \RangeCatcodeCheck{0}{36}{15}%
  \RangeCatcodeCheck{37}{37}{14}%
  \RangeCatcodeCheck{38}{47}{15}%
  \RangeCatcodeCheck{48}{57}{12}%
  \RangeCatcodeCheck{58}{63}{15}%
  \RangeCatcodeCheck{64}{64}{12}%
  \RangeCatcodeCheck{65}{90}{11}%
  \RangeCatcodeCheck{91}{91}{15}%
  \RangeCatcodeCheck{92}{92}{0}%
  \RangeCatcodeCheck{93}{96}{15}%
  \RangeCatcodeCheck{97}{122}{11}%
  \RangeCatcodeCheck{123}{255}{15}%
  \RestoreCatcodes
}
\Test
\csname @@end\endcsname
\end
%    \end{macrocode}
%    \begin{macrocode}
%</test1>
%    \end{macrocode}
%
% \subsection{Macro tests}
%
%    \begin{macrocode}
%<*test2>
\errorcontextlines=10000 %
\showboxbreadth=10000 %
\showboxdepth=10000 %
\begingroup\expandafter\expandafter\expandafter\endgroup
\expandafter\ifx\csname RequirePackage\endcsname\relax
  \input refcount.sty\relax
\else
  \RequirePackage{refcount}[2011/10/16]%
\fi
\catcode`\@=11 %
\begingroup\expandafter\expandafter\expandafter\endgroup
\expandafter\ifx\csname @onelevel@sanitize\endcsname\relax
  \begingroup\expandafter\expandafter\expandafter\endgroup
  \expandafter\ifx\csname detokenize\endcsname\relax
    \def\strip@prefix#1->{}%
    \def\@onelevel@sanitize#1{%
      \edef#1{%
        \expandafter\strip@prefix\meaning#1%
      }%
    }%
  \else
    \def\@onelevel@sanitize#1{%
      \edef#1{%
        \detokenize\expandafter{#1}%
      }%
    }%
  \fi
\fi
\def\msg#{\immediate\write16}
\def\empty{}
\def\space{ }
%    \end{macrocode}
%    \begin{macrocode}
\def\r@foo{{\empty 1}{\empty 2}}
\long\def\test#1#2{%
  \begingroup
    \setbox0=\hbox{%
      \def\TestTask{#1}%
      \@onelevel@sanitize\TestTask
      \msg{* \TestTask}%
      \expandafter\expandafter\expandafter\def
      \expandafter\expandafter\expandafter\TestResult
      \expandafter\expandafter\expandafter{%
        #1%
      }%
      \def\TestExpected{#2}%
      \ifx\TestResult\TestExpected
        \msg{ \space ok.}%
      \else
        \@onelevel@sanitize\TestResult
        \@onelevel@sanitize\TestExpected
        \msg{ \space Result: \space\space[\TestResult]}%
        \msg{ \space Expected: [\TestExpected]}%
        \errmessage{Test failed!}%
      \fi
    }%
    \ifdim\wd0=0pt %
    \else
      \showbox0 %
    \fi
  \endgroup
}
\test{\getrefnumber{foo}}{\empty 1}
\test{\getpagerefnumber{foo}}{\empty 2}
\test{\getrefbykeydefault{foo}{}{\empty default}}{\empty 1}
\test{\getrefbykeydefault{foo}{page}{\empty default}}{\empty 2}
\test{\getrefbykeydefault{foo}{name}{\empty default}}{\empty default}
\test{\getrefbykeydefault{foo}{anchor}{\empty default}}{\empty default}
\test{\getrefbykeydefault{foo}{url}{\empty default}}{\empty default}
\test{\getrefbykeydefault{foo}{title}{\empty default}}{\empty default}
\msg{}
\def\r@foo{{}{}{}{}{}{}{}{}{}{}}
\def\Test#1#2\\{%
  \test{#1{foo}#2}{}%
}
\def\TestGroup{%
  \Test\getrefnumber\\%
  \Test\getpagerefnumber\\%
  \Test\getrefbykeydefault{}{}\\%
  \Test\getrefbykeydefault{page}{}\\%
  \Test\getrefbykeydefault{anchor}{}\\%
  \Test\getrefbykeydefault{name}{}\\%
  \Test\getrefbykeydefault{url}{}\\%
}
\TestGroup
\Test\getrefbykeydefault{title}{}\\%
\msg{}
\def\r@foo{\par\par\par\par\par\par\par\par}
\long\def\Test#1#2\\{%
  \test{#1{foo}#2}{\par}%
}
\TestGroup
\test{\getrefbykeydefault{title}{}{}}{}
\msg{}
\def\r@foo{{ }{ }{ }{ }{ }}
\def\Test#1#2\\{%
  \test{#1{foo}#2}{ }%
}
\TestGroup
\msg{}
\long\def\TestDefault#1{%
  \begingroup
    \setrefcountdefault{#1}%
    \test{\getrefnumber{foo}}{#1}%
    \test{\getpagerefnumber{foo}}{#1}%
  \endgroup
}
\def\TestDefaultX{%
  \TestDefault{}%
  \TestDefault{\par}%
  \TestDefault{ }%
  \TestDefault{\space}%
}
\let\r@foo\@undefined
\TestDefaultX
\let\r@foo\relax
\TestDefaultX
\def\r@foo{}
\TestDefaultX
%    \end{macrocode}
%    \begin{macrocode}
\msg{}
\long\def\Test#1#2#3#4{%
  \begingroup
    \def\TestTask{#1}%
    \@onelevel@sanitize\TestTask
    \msg{* [\TestTask]}%
    \edef\TestResultA{\IfRefUndefinedExpandable{#1}{#2}{#3}}%
    \IfRefUndefinedBabel{#1}{%
      \def\TestResultB{#2}%
    }{%
      \def\TestResultB{#3}%
    }%
    \def\TestExpected{#4}%
    \ifx\TestResultA\TestExpected
      \msg{ \space ok.}%
    \else
      \begingroup
        \@onelevel@sanitize\TestResultA
        \@onelevel@sanitize\TestExpected
        \msg{ \space Result: \space\space[\TestResultA]}%
        \msg{ \space Expected: [\TestExpected]}%
        \errmessage{Test failed!}%
      \endgroup
    \fi
    \ifx\TestResultB\TestExpected
      \msg{ \space ok.}%
    \else
      \begingroup
        \@onelevel@sanitize\TestResultB
        \@onelevel@sanitize\TestExpected
        \msg{ \space Result: \space\space[\TestResultB]}%
        \msg{ \space Expected: [\TestExpected]}%
        \errmessage{Test failed!}%
      \endgroup
    \fi
  \endgroup
}
\begingroup
  \def\r@foo{{}{}}%
  \let\r@bar\@undefined
  \let\r@xyz\relax
  \Test{foo}{true}{false}{false}%
  \Test{bar}{true}{false}{true}%
  \Test{xyz}{true}{false}{true}%
\endgroup
%    \end{macrocode}
%    \begin{macrocode}
\csname @@end\endcsname\end
%</test2>
%    \end{macrocode}
% \subsection{Test with package \xpackage{titleref}}
%
%    \begin{macrocode}
%<*test3>
\NeedsTeXFormat{LaTeX2e}
\documentclass{article}
\usepackage{refcount}[2011/10/16]
%<test4>\usepackage{nameref}
\usepackage{titleref}
\begin{document}
\section{Hello World}
\label{sec:hello}
\section{\hbox{xy}}
\label{sec:foo}
%
\makeatletter
\@ifundefined{r@sec:hello}{%
  \typeout{==> Compile twice!}%
}{%
  \def\test#1#2{%
    \begingroup
      \def\TestTask{#1}%
      \@onelevel@sanitize\TestTask
      \typeout{* \TestTask}%
      \expandafter\expandafter\expandafter\def
      \expandafter\expandafter\expandafter\TestResult
      \expandafter\expandafter\expandafter{%
        #1%
      }%
      \def\TestExpected{#2}%
      \ifx\TestResult\TestExpected
        \typeout{ \space ok.}%
      \else
        \@onelevel@sanitize\TestResult
        \@onelevel@sanitize\TestExpected
        \typeout{ \space Result: \space\space[\TestResult]}%
        \typeout{ \space Expected: [\TestExpected]}%
        \errmessage{Test failed!}%
      \fi
    \endgroup
  }%
  \test{\getrefbykeydefault{sec:hello}{title}{}}{Hello World}%
  \test{\getrefbykeydefault{sec:foo}{title}{}}{\hbox{xy}}%
  \begingroup
    \def\hbox#1{[#1]}% hash-ok
    \test{\getrefbykeydefault{sec:foo}{title}{}}{\hbox{xy}}%
  \endgroup
}
\makeatother
%    \end{macrocode}
%    \begin{macrocode}
\end{document}
%</test3>
%    \end{macrocode}
%    \begin{macrocode}
%<*test5>
\NeedsTeXFormat{LaTeX2e}
\documentclass{book}
\usepackage{refcount}[2011/10/16]
\usepackage{zref-runs}
\newcounter{test}
\begin{document}
\ifnum\zruns>1 %
  \makeatletter
  \def\Test#1#2#3{%
    \begingroup
      \setcounter{test}{10}%
      \sbox0{%
        #1{test}{#2}%
        \ifnum#3=\value{test}%
        \else
          \PackageError{test}{\string#1{#2} <> #3 (\the\value{test})}%
        \fi
      }%
      \ifdim\wd0=0pt %
      \else
        \PackageError{test}{Non-empty box}\@ehc
      \fi
    \endgroup
  }%
  \makeatother
  \Test\setcounterpageref{ch:two}{1}%
  \Test\setcounterpageref{ch:three}{3}%
  \Test\setcounterpageref{ch:four}{5}%
  \Test\setcounterpageref{ch:five}{7}%
  \Test\setcounterpageref{ch:six}{9}%
  \Test\setcounterpageref{ch:seven}{13}%
  \Test\addtocounterpageref{ch:two}{11}%
  \Test\addtocounterpageref{ch:three}{13}%
  \Test\addtocounterpageref{ch:four}{15}%
  \Test\addtocounterpageref{ch:five}{17}%
  \Test\addtocounterpageref{ch:six}{19}%
  \Test\addtocounterpageref{ch:seven}{23}%
  \Test\setcounterref{ch:two}{1}%
  \Test\setcounterref{ch:three}{2}%
  \Test\setcounterref{ch:four}{11}%
  \Test\addtocounterref{ch:two}{11}%
  \Test\addtocounterref{ch:three}{12}%
  \Test\addtocounterref{ch:four}{21}%
\fi
\frontmatter
\chapter{Chapter one}\label{ch:one}
\cleardoublepage
\mainmatter
\chapter{Chapter two}\label{ch:two}
\cleardoublepage
\chapter{Chapter three}\label{ch:three}
\cleardoublepage
\setcounter{chapter}{10}
\chapter{Chapter four}\label{ch:four}
\cleardoublepage
\appendix
\chapter{Chapter five}\label{ch:five}
\cleardoublepage
\chapter{Chapter six}\label{ch:six}
\cleardoublepage
\null
\cleardoublepage
\chapter{Chapter seven}\label{ch:seven}
\end{document}
%</test5>
%    \end{macrocode}
%
% \section{Installation}
%
% \subsection{Download}
%
% \paragraph{Package.} This package is available on
% CTAN\footnote{\url{ftp://ftp.ctan.org/tex-archive/}}:
% \begin{description}
% \item[\CTAN{macros/latex/contrib/oberdiek/refcount.dtx}] The source file.
% \item[\CTAN{macros/latex/contrib/oberdiek/refcount.pdf}] Documentation.
% \end{description}
%
%
% \paragraph{Bundle.} All the packages of the bundle `oberdiek'
% are also available in a TDS compliant ZIP archive. There
% the packages are already unpacked and the documentation files
% are generated. The files and directories obey the TDS standard.
% \begin{description}
% \item[\CTAN{install/macros/latex/contrib/oberdiek.tds.zip}]
% \end{description}
% \emph{TDS} refers to the standard ``A Directory Structure
% for \TeX\ Files'' (\CTAN{tds/tds.pdf}). Directories
% with \xfile{texmf} in their name are usually organized this way.
%
% \subsection{Bundle installation}
%
% \paragraph{Unpacking.} Unpack the \xfile{oberdiek.tds.zip} in the
% TDS tree (also known as \xfile{texmf} tree) of your choice.
% Example (linux):
% \begin{quote}
%   |unzip oberdiek.tds.zip -d ~/texmf|
% \end{quote}
%
% \paragraph{Script installation.}
% Check the directory \xfile{TDS:scripts/oberdiek/} for
% scripts that need further installation steps.
% Package \xpackage{attachfile2} comes with the Perl script
% \xfile{pdfatfi.pl} that should be installed in such a way
% that it can be called as \texttt{pdfatfi}.
% Example (linux):
% \begin{quote}
%   |chmod +x scripts/oberdiek/pdfatfi.pl|\\
%   |cp scripts/oberdiek/pdfatfi.pl /usr/local/bin/|
% \end{quote}
%
% \subsection{Package installation}
%
% \paragraph{Unpacking.} The \xfile{.dtx} file is a self-extracting
% \docstrip\ archive. The files are extracted by running the
% \xfile{.dtx} through \plainTeX:
% \begin{quote}
%   \verb|tex refcount.dtx|
% \end{quote}
%
% \paragraph{TDS.} Now the different files must be moved into
% the different directories in your installation TDS tree
% (also known as \xfile{texmf} tree):
% \begin{quote}
% \def\t{^^A
% \begin{tabular}{@{}>{\ttfamily}l@{ $\rightarrow$ }>{\ttfamily}l@{}}
%   refcount.sty & tex/latex/oberdiek/refcount.sty\\
%   refcount.pdf & doc/latex/oberdiek/refcount.pdf\\
%   test/refcount-test1.tex & doc/latex/oberdiek/test/refcount-test1.tex\\
%   test/refcount-test2.tex & doc/latex/oberdiek/test/refcount-test2.tex\\
%   test/refcount-test3.tex & doc/latex/oberdiek/test/refcount-test3.tex\\
%   test/refcount-test4.tex & doc/latex/oberdiek/test/refcount-test4.tex\\
%   test/refcount-test5.tex & doc/latex/oberdiek/test/refcount-test5.tex\\
%   refcount.dtx & source/latex/oberdiek/refcount.dtx\\
% \end{tabular}^^A
% }^^A
% \sbox0{\t}^^A
% \ifdim\wd0>\linewidth
%   \begingroup
%     \advance\linewidth by\leftmargin
%     \advance\linewidth by\rightmargin
%   \edef\x{\endgroup
%     \def\noexpand\lw{\the\linewidth}^^A
%   }\x
%   \def\lwbox{^^A
%     \leavevmode
%     \hbox to \linewidth{^^A
%       \kern-\leftmargin\relax
%       \hss
%       \usebox0
%       \hss
%       \kern-\rightmargin\relax
%     }^^A
%   }^^A
%   \ifdim\wd0>\lw
%     \sbox0{\small\t}^^A
%     \ifdim\wd0>\linewidth
%       \ifdim\wd0>\lw
%         \sbox0{\footnotesize\t}^^A
%         \ifdim\wd0>\linewidth
%           \ifdim\wd0>\lw
%             \sbox0{\scriptsize\t}^^A
%             \ifdim\wd0>\linewidth
%               \ifdim\wd0>\lw
%                 \sbox0{\tiny\t}^^A
%                 \ifdim\wd0>\linewidth
%                   \lwbox
%                 \else
%                   \usebox0
%                 \fi
%               \else
%                 \lwbox
%               \fi
%             \else
%               \usebox0
%             \fi
%           \else
%             \lwbox
%           \fi
%         \else
%           \usebox0
%         \fi
%       \else
%         \lwbox
%       \fi
%     \else
%       \usebox0
%     \fi
%   \else
%     \lwbox
%   \fi
% \else
%   \usebox0
% \fi
% \end{quote}
% If you have a \xfile{docstrip.cfg} that configures and enables \docstrip's
% TDS installing feature, then some files can already be in the right
% place, see the documentation of \docstrip.
%
% \subsection{Refresh file name databases}
%
% If your \TeX~distribution
% (\teTeX, \mikTeX, \dots) relies on file name databases, you must refresh
% these. For example, \teTeX\ users run \verb|texhash| or
% \verb|mktexlsr|.
%
% \subsection{Some details for the interested}
%
% \paragraph{Attached source.}
%
% The PDF documentation on CTAN also includes the
% \xfile{.dtx} source file. It can be extracted by
% AcrobatReader 6 or higher. Another option is \textsf{pdftk},
% e.g. unpack the file into the current directory:
% \begin{quote}
%   \verb|pdftk refcount.pdf unpack_files output .|
% \end{quote}
%
% \paragraph{Unpacking with \LaTeX.}
% The \xfile{.dtx} chooses its action depending on the format:
% \begin{description}
% \item[\plainTeX:] Run \docstrip\ and extract the files.
% \item[\LaTeX:] Generate the documentation.
% \end{description}
% If you insist on using \LaTeX\ for \docstrip\ (really,
% \docstrip\ does not need \LaTeX), then inform the autodetect routine
% about your intention:
% \begin{quote}
%   \verb|latex \let\install=y% \iffalse meta-comment
%
% File: refcount.dtx
% Version: 2011/10/16 v3.4
% Info: Data extraction from label references
%
% Copyright (C) 1998, 2000, 2006, 2008, 2010, 2011 by
%    Heiko Oberdiek <heiko.oberdiek at googlemail.com>
%
% This work may be distributed and/or modified under the
% conditions of the LaTeX Project Public License, either
% version 1.3c of this license or (at your option) any later
% version. This version of this license is in
%    http://www.latex-project.org/lppl/lppl-1-3c.txt
% and the latest version of this license is in
%    http://www.latex-project.org/lppl.txt
% and version 1.3 or later is part of all distributions of
% LaTeX version 2005/12/01 or later.
%
% This work has the LPPL maintenance status "maintained".
%
% This Current Maintainer of this work is Heiko Oberdiek.
%
% This work consists of the main source file refcount.dtx
% and the derived files
%    refcount.sty, refcount.pdf, refcount.ins, refcount.drv,
%    refcount-test1.tex, refcount-test2.tex, refcount-test3.tex,
%    refcount-test4.tex, refcount-test5.tex.
%
% Distribution:
%    CTAN:macros/latex/contrib/oberdiek/refcount.dtx
%    CTAN:macros/latex/contrib/oberdiek/refcount.pdf
%
% Unpacking:
%    (a) If refcount.ins is present:
%           tex refcount.ins
%    (b) Without refcount.ins:
%           tex refcount.dtx
%    (c) If you insist on using LaTeX
%           latex \let\install=y\input{refcount.dtx}
%        (quote the arguments according to the demands of your shell)
%
% Documentation:
%    (a) If refcount.drv is present:
%           latex refcount.drv
%    (b) Without refcount.drv:
%           latex refcount.dtx; ...
%    The class ltxdoc loads the configuration file ltxdoc.cfg
%    if available. Here you can specify further options, e.g.
%    use A4 as paper format:
%       \PassOptionsToClass{a4paper}{article}
%
%    Programm calls to get the documentation (example):
%       pdflatex refcount.dtx
%       makeindex -s gind.ist refcount.idx
%       pdflatex refcount.dtx
%       makeindex -s gind.ist refcount.idx
%       pdflatex refcount.dtx
%
% Installation:
%    TDS:tex/latex/oberdiek/refcount.sty
%    TDS:doc/latex/oberdiek/refcount.pdf
%    TDS:doc/latex/oberdiek/test/refcount-test1.tex
%    TDS:doc/latex/oberdiek/test/refcount-test2.tex
%    TDS:doc/latex/oberdiek/test/refcount-test3.tex
%    TDS:doc/latex/oberdiek/test/refcount-test4.tex
%    TDS:doc/latex/oberdiek/test/refcount-test5.tex
%    TDS:source/latex/oberdiek/refcount.dtx
%
%<*ignore>
\begingroup
  \catcode123=1 %
  \catcode125=2 %
  \def\x{LaTeX2e}%
\expandafter\endgroup
\ifcase 0\ifx\install y1\fi\expandafter
         \ifx\csname processbatchFile\endcsname\relax\else1\fi
         \ifx\fmtname\x\else 1\fi\relax
\else\csname fi\endcsname
%</ignore>
%<*install>
\input docstrip.tex
\Msg{************************************************************************}
\Msg{* Installation}
\Msg{* Package: refcount 2011/10/16 v3.4 Data extraction from label references (HO)}
\Msg{************************************************************************}

\keepsilent
\askforoverwritefalse

\let\MetaPrefix\relax
\preamble

This is a generated file.

Project: refcount
Version: 2011/10/16 v3.4

Copyright (C) 1998, 2000, 2006, 2008, 2010, 2011 by
   Heiko Oberdiek <heiko.oberdiek at googlemail.com>

This work may be distributed and/or modified under the
conditions of the LaTeX Project Public License, either
version 1.3c of this license or (at your option) any later
version. This version of this license is in
   http://www.latex-project.org/lppl/lppl-1-3c.txt
and the latest version of this license is in
   http://www.latex-project.org/lppl.txt
and version 1.3 or later is part of all distributions of
LaTeX version 2005/12/01 or later.

This work has the LPPL maintenance status "maintained".

This Current Maintainer of this work is Heiko Oberdiek.

This work consists of the main source file refcount.dtx
and the derived files
   refcount.sty, refcount.pdf, refcount.ins, refcount.drv,
   refcount-test1.tex, refcount-test2.tex, refcount-test3.tex,
   refcount-test4.tex, refcount-test5.tex.

\endpreamble
\let\MetaPrefix\DoubleperCent

\generate{%
  \file{refcount.ins}{\from{refcount.dtx}{install}}%
  \file{refcount.drv}{\from{refcount.dtx}{driver}}%
  \usedir{tex/latex/oberdiek}%
  \file{refcount.sty}{\from{refcount.dtx}{package}}%
  \usedir{doc/latex/oberdiek/test}%
  \file{refcount-test1.tex}{\from{refcount.dtx}{test1}}%
  \file{refcount-test2.tex}{\from{refcount.dtx}{test2}}%
  \file{refcount-test3.tex}{\from{refcount.dtx}{test3}}%
  \file{refcount-test4.tex}{\from{refcount.dtx}{test3,test4}}%
  \file{refcount-test5.tex}{\from{refcount.dtx}{test5}}%
  \nopreamble
  \nopostamble
  \usedir{source/latex/oberdiek/catalogue}%
  \file{refcount.xml}{\from{refcount.dtx}{catalogue}}%
}

\catcode32=13\relax% active space
\let =\space%
\Msg{************************************************************************}
\Msg{*}
\Msg{* To finish the installation you have to move the following}
\Msg{* file into a directory searched by TeX:}
\Msg{*}
\Msg{*     refcount.sty}
\Msg{*}
\Msg{* To produce the documentation run the file `refcount.drv'}
\Msg{* through LaTeX.}
\Msg{*}
\Msg{* Happy TeXing!}
\Msg{*}
\Msg{************************************************************************}

\endbatchfile
%</install>
%<*ignore>
\fi
%</ignore>
%<*driver>
\NeedsTeXFormat{LaTeX2e}
\ProvidesFile{refcount.drv}%
  [2011/10/16 v3.4 Data extraction from label references (HO)]%
\documentclass{ltxdoc}
\usepackage{holtxdoc}[2011/11/22]
\begin{document}
  \DocInput{refcount.dtx}%
\end{document}
%</driver>
% \fi
%
% \CheckSum{1185}
%
% \CharacterTable
%  {Upper-case    \A\B\C\D\E\F\G\H\I\J\K\L\M\N\O\P\Q\R\S\T\U\V\W\X\Y\Z
%   Lower-case    \a\b\c\d\e\f\g\h\i\j\k\l\m\n\o\p\q\r\s\t\u\v\w\x\y\z
%   Digits        \0\1\2\3\4\5\6\7\8\9
%   Exclamation   \!     Double quote  \"     Hash (number) \#
%   Dollar        \$     Percent       \%     Ampersand     \&
%   Acute accent  \'     Left paren    \(     Right paren   \)
%   Asterisk      \*     Plus          \+     Comma         \,
%   Minus         \-     Point         \.     Solidus       \/
%   Colon         \:     Semicolon     \;     Less than     \<
%   Equals        \=     Greater than  \>     Question mark \?
%   Commercial at \@     Left bracket  \[     Backslash     \\
%   Right bracket \]     Circumflex    \^     Underscore    \_
%   Grave accent  \`     Left brace    \{     Vertical bar  \|
%   Right brace   \}     Tilde         \~}
%
% \GetFileInfo{refcount.drv}
%
% \title{The \xpackage{refcount} package}
% \date{2011/10/16 v3.4}
% \author{Heiko Oberdiek\\\xemail{heiko.oberdiek at googlemail.com}}
%
% \maketitle
%
% \begin{abstract}
% References are not numbers, however they often store numerical
% data such as section or page numbers. \cs{ref} or \cs{pageref}
% cannot be used for counter assignments or calculations because
% they are not expandable, generate warnings, or can even be links.
% The package provides expandable macros to extract the data
% from references. Packages \xpackage{hyperref}, \xpackage{nameref},
% \xpackage{titleref}, and \xpackage{babel} are supported.
% \end{abstract}
%
% \tableofcontents
%
% \section{Usage}
%
% \subsection{Setting counters}
%
% The following commands are similar to \LaTeX's
% \cs{setcounter} and \cs{addtocounter},
% but they extract the number value from a reference:
% \begin{quote}
%   \cs{setcounterref}, \cs{addtocounterref}\\
%   \cs{setcounterpageref}, \cs{addtocounterpageref}
% \end{quote}
% They take two arguments:
% \begin{quote}
%    \cs{...counter...ref} |{|\meta{\LaTeX\ counter}|}|
%    |{|\meta{reference}|}|
% \end{quote}
% An undefined references produces the usual LaTeX warning
% and its value is assumed to be zero.
% Example:
% \begin{quote}
%\begin{verbatim}
%\newcounter{ctrA}
%\newcounter{ctrB}
%\refstepcounter{ctrA}\label{ref:A}
%\setcounterref{ctrB}{ref:A}
%\addtocounterpageref{ctrB}{ref:A}
%\end{verbatim}
% \end{quote}
%
% \subsection{Expandable commands}
%
% These commands that can be used in expandible contexts
% (inside calculations, \cs{edef}, \cs{csname}, \cs{write}, \dots):
% \begin{quote}
%   \cs{getrefnumber}, \cs{getpagerefnumber}
% \end{quote}
% They take one argument, the reference:
% \begin{quote}
%   \cs{get...refnumber} |{|\meta{reference}|}|
% \end{quote}
% The default for undefined references can be changed
% with macro \cs{setrefcountdefault}, for example this
% package calls:
% \begin{quote}
%   \cs{setrefcountdefault}|{0}|
% \end{quote}
%
% Since version 2.0 of this package there is a new
% command:
% \begin{quote}
%   \cs{getrefbykeydefault} |{|\meta{reference}|}|
%   |{|\meta{key}|}| |{|\meta{default}|}|
% \end{quote}
% This generalized version allows the extraction
% of further properties of a reference than the
% two standard ones. Thus the following properties
% are supported, if they are available:
% \begin{quote}
% \begin{tabular}{@{}l|l|l@{}}
%    Key & Description & Package\\
% \hline
%   \meta{empty} & same as \cs{ref} & \LaTeX\\
%   |page| & same as \cs{pageref} & \LaTeX\\
%   |title| & section and caption titles & \xpackage{titleref}\\
%   |name| & section and caption titles & \xpackage{nameref}\\
%   |anchor| & anchor name & \xpackage{hyperref}\\
%   |url| & url/file & \xpackage{hyperref}/\xpackage{xr}
% \end{tabular}
% \end{quote}
%
% Since version 3.2 the expandable macros described before
% in this section
% are expandable in exact two expansion steps.
%
% \subsection{Undefined references}
%
% Because warnings and assignments cannot be used in
% expandible contexts, undefined references do not
% produce a warning, their values are assumed to be zero.
% Example:
% \begin{quote}
%\begin{verbatim}
%\label{ref:here}% somewhere
%\refused{ref:here}% see below
%\ifodd\getpagerefnumber{ref:here}%
%  reference is on an odd page
%\else
%  reference is on an even page
%\fi
%\end{verbatim}
% \end{quote}
%
% In case of undefined references the user usually want's
% to be informed. Also \LaTeX\ prints a warning at
% the end of the \LaTeX\ run. To notify \LaTeX\ and
% get a normal warning, just use
% \begin{quote}
%   \cs{refused} |{|\meta{reference}|}|
% \end{quote}
% outside the expanding context. Example, see above.
%
% \subsubsection{Check for undefined references}
%
% In version 3.2 macros were added, that test, whether references
% are defined.
% \begin{declcs}{IfRefUndefinedExpandable} \M{refname} \M{then} \M{else}\\
%   \cs{IfRefUndefinedBabel} \M{refname} \M{then} \M{else}
% \end{declcs}
% If the reference is not available and therefore undefined, then
% argument \meta{then} is executed, otherwise argument \meta{else}
% is called. Macro \cs{IfRefUndefinedExpandable} is expandable,
% but \meta{refname} must not contain babel shorthand characters.
% Macro \cs{IfRefUndefinedBabel} supports shorthand characters of
% babel, but it is not expandable.
%
% \subsection{Notes}
%
% \begin{itemize}
% \item
%   The method of extracting the number in this
%   package also works in cases, where the
%   reference cannot be used directly, because
%   a package such as \xpackage{hyperref} has added
%   extra stuff (hyper link), so that the reference cannot
%   be used as number any more.
% \item
%   If the reference does not contain a number,
%   assignments to a counter will fail of course.
% \end{itemize}
%
%
% \StopEventually{
% }
%
% \section{Implementation}
%
%    \begin{macrocode}
%<*package>
%    \end{macrocode}
%    Reload check, especially if the package is not used with \LaTeX.
%    \begin{macrocode}
\begingroup\catcode61\catcode48\catcode32=10\relax%
  \catcode13=5 % ^^M
  \endlinechar=13 %
  \catcode35=6 % #
  \catcode39=12 % '
  \catcode44=12 % ,
  \catcode45=12 % -
  \catcode46=12 % .
  \catcode58=12 % :
  \catcode64=11 % @
  \catcode123=1 % {
  \catcode125=2 % }
  \expandafter\let\expandafter\x\csname ver@refcount.sty\endcsname
  \ifx\x\relax % plain-TeX, first loading
  \else
    \def\empty{}%
    \ifx\x\empty % LaTeX, first loading,
      % variable is initialized, but \ProvidesPackage not yet seen
    \else
      \expandafter\ifx\csname PackageInfo\endcsname\relax
        \def\x#1#2{%
          \immediate\write-1{Package #1 Info: #2.}%
        }%
      \else
        \def\x#1#2{\PackageInfo{#1}{#2, stopped}}%
      \fi
      \x{refcount}{The package is already loaded}%
      \aftergroup\endinput
    \fi
  \fi
\endgroup%
%    \end{macrocode}
%    Package identification:
%    \begin{macrocode}
\begingroup\catcode61\catcode48\catcode32=10\relax%
  \catcode13=5 % ^^M
  \endlinechar=13 %
  \catcode35=6 % #
  \catcode39=12 % '
  \catcode40=12 % (
  \catcode41=12 % )
  \catcode44=12 % ,
  \catcode45=12 % -
  \catcode46=12 % .
  \catcode47=12 % /
  \catcode58=12 % :
  \catcode64=11 % @
  \catcode91=12 % [
  \catcode93=12 % ]
  \catcode123=1 % {
  \catcode125=2 % }
  \expandafter\ifx\csname ProvidesPackage\endcsname\relax
    \def\x#1#2#3[#4]{\endgroup
      \immediate\write-1{Package: #3 #4}%
      \xdef#1{#4}%
    }%
  \else
    \def\x#1#2[#3]{\endgroup
      #2[{#3}]%
      \ifx#1\@undefined
        \xdef#1{#3}%
      \fi
      \ifx#1\relax
        \xdef#1{#3}%
      \fi
    }%
  \fi
\expandafter\x\csname ver@refcount.sty\endcsname
\ProvidesPackage{refcount}%
  [2011/10/16 v3.4 Data extraction from label references (HO)]%
%    \end{macrocode}
%
%    \begin{macrocode}
\begingroup\catcode61\catcode48\catcode32=10\relax%
  \catcode13=5 % ^^M
  \endlinechar=13 %
  \catcode123=1 % {
  \catcode125=2 % }
  \catcode64=11 % @
  \def\x{\endgroup
    \expandafter\edef\csname rc@AtEnd\endcsname{%
      \endlinechar=\the\endlinechar\relax
      \catcode13=\the\catcode13\relax
      \catcode32=\the\catcode32\relax
      \catcode35=\the\catcode35\relax
      \catcode61=\the\catcode61\relax
      \catcode64=\the\catcode64\relax
      \catcode123=\the\catcode123\relax
      \catcode125=\the\catcode125\relax
    }%
  }%
\x\catcode61\catcode48\catcode32=10\relax%
\catcode13=5 % ^^M
\endlinechar=13 %
\catcode35=6 % #
\catcode64=11 % @
\catcode123=1 % {
\catcode125=2 % }
\def\TMP@EnsureCode#1#2{%
  \edef\rc@AtEnd{%
    \rc@AtEnd
    \catcode#1=\the\catcode#1\relax
  }%
  \catcode#1=#2\relax
}
\TMP@EnsureCode{33}{12}% !
\TMP@EnsureCode{39}{12}% '
\TMP@EnsureCode{42}{12}% *
\TMP@EnsureCode{45}{12}% -
\TMP@EnsureCode{46}{12}% .
\TMP@EnsureCode{47}{12}% /
\TMP@EnsureCode{91}{12}% [
\TMP@EnsureCode{93}{12}% ]
\TMP@EnsureCode{96}{12}% `
\edef\rc@AtEnd{\rc@AtEnd\noexpand\endinput}
%    \end{macrocode}
%
% \subsection{Loading packages}
%
%    \begin{macrocode}
\begingroup\expandafter\expandafter\expandafter\endgroup
\expandafter\ifx\csname RequirePackage\endcsname\relax
  \input ltxcmds.sty\relax
  \input infwarerr.sty\relax
\else
  \RequirePackage{ltxcmds}[2011/11/09]%
  \RequirePackage{infwarerr}[2010/04/08]%
\fi
%    \end{macrocode}
%
% \subsection{Defining commands}
%
%    \begin{macro}{\rc@IfDefinable}
%    \begin{macrocode}
\ltx@IfUndefined{@ifdefinable}{%
  \def\rc@IfDefinable#1{%
    \ifx#1\ltx@undefined
      \expandafter\ltx@firstofone
    \else
      \ifx#1\relax
        \expandafter\expandafter\expandafter\ltx@firstofone
      \else
        \@PackageError{refcount}{%
          Command \string#1 is already defined.\MessageBreak
          It will not redefined by this package%
        }\@ehc
        \expandafter\expandafter\expandafter\ltx@gobble
      \fi
    \fi
  }%
}{%
  \let\rc@IfDefinable\@ifdefinable
}
%    \end{macrocode}
%    \end{macro}
%
%    \begin{macro}{\rc@RobustDefOne}
%    \begin{macro}{\rc@RobustDefZero}
%    \begin{macrocode}
\ltx@IfUndefined{protected}{%
  \ltx@IfUndefined{DeclareRobustCommand}{%
    \def\rc@RobustDefOne#1#2#3#4{%
      \rc@IfDefinable#3{%
        #1\def#3##1{#4}%
      }%
    }%
    \def\rc@RobustDefZero#1#2{%
      \rc@IfDefinable#1{%
        \def#1{#2}%
      }%
    }%
  }{%
    \def\rc@RobustDefOne#1#2#3#4{%
      \rc@IfDefinable#3{%
        \DeclareRobustCommand#2#3[1]{#4}%
      }%
    }%
    \def\rc@RobustDefZero#1#2{%
      \rc@IfDefinable#1{%
        \DeclareRobustCommand#1{#2}%
      }%
    }%
  }%
}{%
  \def\rc@RobustDefOne#1#2#3#4{%
    \rc@IfDefinable#3{%
      \protected#1\def#3##1{#4}%
    }%
  }%
  \def\rc@RobustDefZero#1#2{%
    \rc@IfDefinable#1{%
      \protected\def#1{#2}%
    }%
  }%
}
%    \end{macrocode}
%    \end{macro}
%    \end{macro}
%
%    \begin{macro}{\rc@newcommand}
%    \begin{macrocode}
\ltx@IfUndefined{newcommand}{%
  \def\rc@newcommand*#1[#2]#3{% hash-ok
    \rc@IfDefinable#1{%
      \ifcase#2 %
        \def#1{#3}%
      \or
        \def#1##1{#3}%
      \or
        \def#1##1##2{#3}%
      \else
        \rc@InternalError
      \fi
    }%
  }%
}{%
  \let\rc@newcommand\newcommand
}
%    \end{macrocode}
%    \end{macro}
%
% \subsection{\cs{setrefcountdefault}}
%
%    \begin{macro}{\setrefcountdefault}
%    \begin{macrocode}
\rc@RobustDefOne\long{}\setrefcountdefault{%
  \def\rc@default{#1}%
}
%    \end{macrocode}
%    \end{macro}
%    \begin{macrocode}
\setrefcountdefault{0}
%    \end{macrocode}
%
% \subsection{\cs{refused}}
%
%    \begin{macro}{\refused}
%    \begin{macrocode}
\ltx@IfUndefined{G@refundefinedtrue}{%
  \rc@RobustDefOne{}{*}\refused{%
    \begingroup
      \csname @safe@activestrue\endcsname
      \ltx@IfUndefined{r@#1}{%
        \protect\G@refundefinedtrue
        \rc@WarningUndefined{#1}%
      }{}%
    \endgroup
  }%
}{%
  \rc@RobustDefOne{}{*}\refused{%
    \begingroup
      \csname @safe@activestrue\endcsname
      \ltx@IfUndefined{r@#1}{%
        \csname protect\expandafter\endcsname
        \csname G@refundefinedtrue\endcsname
        \rc@WarningUndefined{#1}%
      }{}%
    \endgroup
  }%
}
%    \end{macrocode}
%    \end{macro}
%    \begin{macro}{\rc@WarningUndefined}
%    \begin{macrocode}
\ltx@IfUndefined{@latex@warning}{%
  \def\rc@WarningUndefined#1{%
    \ltx@ifundefined{thepage}{%
      \def\thepage{\number\count0 }%
    }{}%
    \@PackageWarning{refcount}{%
      Reference `#1' on page \thepage\space undefined%
    }%
  }%
}{%
  \def\rc@WarningUndefined#1{%
    \@latex@warning{%
      Reference `#1' on page \thepage\space undefined%
    }%
  }%
}
%    \end{macrocode}
%    \end{macro}
%
% \subsection{Setting counters by reference data}
%
% \subsubsection{Generic setting}
%
%    \begin{macro}{\rc@set}
% Generic command for
% |\|$\{$|set|$,$|addto|$\}$|counter|$\{$|page|$,\}$|ref|:
%\begin{quote}
%\begin{tabular}[t]{@{}l@{: }l@{}}
% |#1|& \cs{setcounter}, \cs{addtocounter}\\
% |#2|& \cs{ltx@car} (for \cs{ref}), \cs{ltx@cartwo} (for \cs{pageref})\\
% |#3|& \hologo{LaTeX} counter\\
% |#4|& reference\\
%\end{tabular}
%\end{quote}
%    \begin{macrocode}
\def\rc@set#1#2#3#4{%
  \begingroup
    \csname @safe@activestrue\endcsname
    \refused{#4}%
    \expandafter\rc@@set\csname r@#4\endcsname{#1}{#2}{#3}%
  \endgroup
}
%    \end{macrocode}
%    \end{macro}
%    \begin{macro}{\rc@@set}
%\begin{quote}
%\begin{tabular}[t]{@{}l@{: }l@{}}
% |#1|& \cs{r@<...>}\\
% |#2|& \cs{setcounter}, \cs{addtocounter}\\
% |#3|& \cs{ltx@car} (for \cs{ref}), \cs{ltx@carsecond} (for \cs{pageref})\\
% |#4|& \hologo{LaTeX} counter\\
%\end{tabular}
%\end{quote}
%    \begin{macrocode}
\def\rc@@set#1#2#3#4{%
  \ifx#1\relax
    #2{#4}{\rc@default}%
  \else
    #2{#4}{%
      \expandafter#3#1\rc@default\rc@default\@nil
    }%
  \fi
}
%    \end{macrocode}
%    \end{macro}
%
% \subsubsection{User commands}
%
%    \begin{macro}{\setcounterref}
%    \begin{macrocode}
\rc@RobustDefZero\setcounterref{%
  \rc@set\setcounter\ltx@car
}
%    \end{macrocode}
%    \end{macro}
%    \begin{macro}{\addtocounterref}
%    \begin{macrocode}
\rc@RobustDefZero\addtocounterref{%
  \rc@set\addtocounter\ltx@car
}
%    \end{macrocode}
%    \end{macro}
%    \begin{macro}{\setcounterpageref}
%    \begin{macrocode}
\rc@RobustDefZero\setcounterpageref{%
  \rc@set\setcounter\ltx@carsecond
}
%    \end{macrocode}
%    \end{macro}
%    \begin{macro}{\addtocounterpageref}
%    \begin{macrocode}
\rc@RobustDefZero\addtocounterpageref{%
  \rc@set\addtocounter\ltx@carsecond
}
%    \end{macrocode}
%    \end{macro}
%
% \subsection{Extracting references}
%
%    \begin{macro}{\getrefnumber}
%    \begin{macrocode}
\rc@newcommand*{\getrefnumber}[1]{%
  \romannumeral
  \ltx@ifundefined{r@#1}{%
    \expandafter\ltx@zero
    \rc@default
  }{%
    \expandafter\expandafter\expandafter\rc@extract@
    \expandafter\expandafter\expandafter!%
    \csname r@#1\expandafter\endcsname
    \expandafter{\rc@default}\@nil
  }%
}
%    \end{macrocode}
%    \end{macro}
%    \begin{macro}{\getpagerefnumber}
%    \begin{macrocode}
\rc@newcommand*{\getpagerefnumber}[1]{%
  \romannumeral
  \ltx@ifundefined{r@#1}{%
    \expandafter\ltx@zero
    \rc@default
  }{%
    \expandafter\expandafter\expandafter\rc@extract@page
    \expandafter\expandafter\expandafter!%
    \csname r@#1\expandafter\expandafter\expandafter\endcsname
    \expandafter\expandafter\expandafter{%
      \expandafter\rc@default
    \expandafter}\expandafter{\rc@default}\@nil
  }%
}
%    \end{macrocode}
%    \end{macro}
%    \begin{macro}{\getrefbykeydefault}
%    \begin{macrocode}
\rc@newcommand*{\getrefbykeydefault}[2]{%
  \romannumeral
  \expandafter\rc@getrefbykeydefault
    \csname r@#1\expandafter\endcsname
    \csname rc@extract@#2\endcsname
}
%    \end{macrocode}
%    \end{macro}
%    \begin{macro}{\rc@getrefbykeydefault}
%\begin{quote}
%\begin{tabular}[t]{@{}l@{: }l@{}}
% |#1|& \cs{r@<...>}\\
% |#2|& \cs{rc@extract@<...>}\\
% |#3|& default\\
%\end{tabular}
%\end{quote}
%    \begin{macrocode}
\long\def\rc@getrefbykeydefault#1#2#3{%
  \ifx#1\relax
    % reference is undefined
    \ltx@ReturnAfterElseFi{%
      \ltx@zero
      #3%
    }%
  \else
    \ltx@ReturnAfterFi{%
      \ifx#2\relax
        % extract method is missing
        \ltx@ReturnAfterElseFi{%
          \ltx@zero
          #3%
        }%
      \else
        \ltx@ReturnAfterFi{%
          \expandafter
          \rc@generic#1{#3}{#3}{#3}{#3}{#3}\@nil#2{#3}%
        }%
      \fi
    }%
  \fi
}
%    \end{macrocode}
%    \end{macro}
%    \begin{macro}{\rc@generic}
%\begin{quote}
%\begin{tabular}[t]{@{}l@{: }l@{}}
% |#1|& first item in \cs{r@<...>}\\
% |#2|& remaining items in \cs{r@<...>}\\
% |#3|& \cs{rc@extract@<...>}\\
% |#4|& default\\
%\end{tabular}
%\end{quote}
%    \begin{macrocode}
\long\def\rc@generic#1#2\@nil#3#4{%
  #3{#1\TR@TitleReference\@empty{#4}\@nil}{#1}#2\@nil
}
%    \end{macrocode}
%    \end{macro}
%    \begin{macro}{\rc@extract@}
%    \begin{macrocode}
\long\def\rc@extract@#1#2#3\@nil{%
  \ltx@zero
  #2%
}
%    \end{macrocode}
%    \end{macro}
%    \begin{macro}{\rc@extract@page}
%    \begin{macrocode}
\long\def\rc@extract@page#1#2#3#4\@nil{%
  \ltx@zero
  #3%
}
%    \end{macrocode}
%    \end{macro}
%    \begin{macro}{\rc@extract@name}
%    \begin{macrocode}
\long\def\rc@extract@name#1#2#3#4#5\@nil{%
  \ltx@zero
  #4%
}
%    \end{macrocode}
%    \end{macro}
%    \begin{macro}{\rc@extract@anchor}
%    \begin{macrocode}
\long\def\rc@extract@anchor#1#2#3#4#5#6\@nil{%
  \ltx@zero
  #5%
}
%    \end{macrocode}
%    \end{macro}
%    \begin{macro}{\rc@extract@url}
%    \begin{macrocode}
\long\def\rc@extract@url#1#2#3#4#5#6#7\@nil{%
  \ltx@zero
  #6%
}
%    \end{macrocode}
%    \end{macro}
%    \begin{macro}{\rc@extract@title}
%    \begin{macrocode}
\long\def\rc@extract@title#1#2\@nil{%
  \rc@@extract@title#1%
}
%    \end{macrocode}
%    \end{macro}
%    \begin{macro}{\rc@@extract@title}
%    \begin{macrocode}
\long\def\rc@@extract@title#1\TR@TitleReference#2#3#4\@nil{%
  \ltx@zero
  #3%
}
%    \end{macrocode}
%    \end{macro}
%
% \subsection{Macros for checking undefined references}
%
%    \begin{macro}{\IfRefUndefinedExpandable}
%    \begin{macrocode}
\rc@newcommand*{\IfRefUndefinedExpandable}[1]{%
  \ltx@ifundefined{r@#1}\ltx@firstoftwo\ltx@secondoftwo
}
%    \end{macrocode}
%    \end{macro}
%    \begin{macro}{\IfRefUndefinedBabel}
%    \begin{macrocode}
\rc@RobustDefOne{}*\IfRefUndefinedBabel{%
  \begingroup
    \csname safe@actives@true\endcsname
  \expandafter\expandafter\expandafter\endgroup
  \expandafter\ifx\csname r@#1\endcsname\relax
    \expandafter\ltx@firstoftwo
  \else
    \expandafter\ltx@secondoftwo
  \fi
}
%    \end{macrocode}
%    \end{macro}
%    \begin{macrocode}
\rc@AtEnd%
%</package>
%    \end{macrocode}
%
% \section{Test}
%
% \subsection{Catcode checks for loading}
%
%    \begin{macrocode}
%<*test1>
%    \end{macrocode}
%    \begin{macrocode}
\catcode`\{=1 %
\catcode`\}=2 %
\catcode`\#=6 %
\catcode`\@=11 %
\expandafter\ifx\csname count@\endcsname\relax
  \countdef\count@=255 %
\fi
\expandafter\ifx\csname @gobble\endcsname\relax
  \long\def\@gobble#1{}%
\fi
\expandafter\ifx\csname @firstofone\endcsname\relax
  \long\def\@firstofone#1{#1}%
\fi
\expandafter\ifx\csname loop\endcsname\relax
  \expandafter\@firstofone
\else
  \expandafter\@gobble
\fi
{%
  \def\loop#1\repeat{%
    \def\body{#1}%
    \iterate
  }%
  \def\iterate{%
    \body
      \let\next\iterate
    \else
      \let\next\relax
    \fi
    \next
  }%
  \let\repeat=\fi
}%
\def\RestoreCatcodes{}
\count@=0 %
\loop
  \edef\RestoreCatcodes{%
    \RestoreCatcodes
    \catcode\the\count@=\the\catcode\count@\relax
  }%
\ifnum\count@<255 %
  \advance\count@ 1 %
\repeat

\def\RangeCatcodeInvalid#1#2{%
  \count@=#1\relax
  \loop
    \catcode\count@=15 %
  \ifnum\count@<#2\relax
    \advance\count@ 1 %
  \repeat
}
\def\RangeCatcodeCheck#1#2#3{%
  \count@=#1\relax
  \loop
    \ifnum#3=\catcode\count@
    \else
      \errmessage{%
        Character \the\count@\space
        with wrong catcode \the\catcode\count@\space
        instead of \number#3%
      }%
    \fi
  \ifnum\count@<#2\relax
    \advance\count@ 1 %
  \repeat
}
\def\space{ }
\expandafter\ifx\csname LoadCommand\endcsname\relax
  \def\LoadCommand{\input refcount.sty\relax}%
\fi
\def\Test{%
  \RangeCatcodeInvalid{0}{47}%
  \RangeCatcodeInvalid{58}{64}%
  \RangeCatcodeInvalid{91}{96}%
  \RangeCatcodeInvalid{123}{255}%
  \catcode`\@=12 %
  \catcode`\\=0 %
  \catcode`\%=14 %
  \LoadCommand
  \RangeCatcodeCheck{0}{36}{15}%
  \RangeCatcodeCheck{37}{37}{14}%
  \RangeCatcodeCheck{38}{47}{15}%
  \RangeCatcodeCheck{48}{57}{12}%
  \RangeCatcodeCheck{58}{63}{15}%
  \RangeCatcodeCheck{64}{64}{12}%
  \RangeCatcodeCheck{65}{90}{11}%
  \RangeCatcodeCheck{91}{91}{15}%
  \RangeCatcodeCheck{92}{92}{0}%
  \RangeCatcodeCheck{93}{96}{15}%
  \RangeCatcodeCheck{97}{122}{11}%
  \RangeCatcodeCheck{123}{255}{15}%
  \RestoreCatcodes
}
\Test
\csname @@end\endcsname
\end
%    \end{macrocode}
%    \begin{macrocode}
%</test1>
%    \end{macrocode}
%
% \subsection{Macro tests}
%
%    \begin{macrocode}
%<*test2>
\errorcontextlines=10000 %
\showboxbreadth=10000 %
\showboxdepth=10000 %
\begingroup\expandafter\expandafter\expandafter\endgroup
\expandafter\ifx\csname RequirePackage\endcsname\relax
  \input refcount.sty\relax
\else
  \RequirePackage{refcount}[2011/10/16]%
\fi
\catcode`\@=11 %
\begingroup\expandafter\expandafter\expandafter\endgroup
\expandafter\ifx\csname @onelevel@sanitize\endcsname\relax
  \begingroup\expandafter\expandafter\expandafter\endgroup
  \expandafter\ifx\csname detokenize\endcsname\relax
    \def\strip@prefix#1->{}%
    \def\@onelevel@sanitize#1{%
      \edef#1{%
        \expandafter\strip@prefix\meaning#1%
      }%
    }%
  \else
    \def\@onelevel@sanitize#1{%
      \edef#1{%
        \detokenize\expandafter{#1}%
      }%
    }%
  \fi
\fi
\def\msg#{\immediate\write16}
\def\empty{}
\def\space{ }
%    \end{macrocode}
%    \begin{macrocode}
\def\r@foo{{\empty 1}{\empty 2}}
\long\def\test#1#2{%
  \begingroup
    \setbox0=\hbox{%
      \def\TestTask{#1}%
      \@onelevel@sanitize\TestTask
      \msg{* \TestTask}%
      \expandafter\expandafter\expandafter\def
      \expandafter\expandafter\expandafter\TestResult
      \expandafter\expandafter\expandafter{%
        #1%
      }%
      \def\TestExpected{#2}%
      \ifx\TestResult\TestExpected
        \msg{ \space ok.}%
      \else
        \@onelevel@sanitize\TestResult
        \@onelevel@sanitize\TestExpected
        \msg{ \space Result: \space\space[\TestResult]}%
        \msg{ \space Expected: [\TestExpected]}%
        \errmessage{Test failed!}%
      \fi
    }%
    \ifdim\wd0=0pt %
    \else
      \showbox0 %
    \fi
  \endgroup
}
\test{\getrefnumber{foo}}{\empty 1}
\test{\getpagerefnumber{foo}}{\empty 2}
\test{\getrefbykeydefault{foo}{}{\empty default}}{\empty 1}
\test{\getrefbykeydefault{foo}{page}{\empty default}}{\empty 2}
\test{\getrefbykeydefault{foo}{name}{\empty default}}{\empty default}
\test{\getrefbykeydefault{foo}{anchor}{\empty default}}{\empty default}
\test{\getrefbykeydefault{foo}{url}{\empty default}}{\empty default}
\test{\getrefbykeydefault{foo}{title}{\empty default}}{\empty default}
\msg{}
\def\r@foo{{}{}{}{}{}{}{}{}{}{}}
\def\Test#1#2\\{%
  \test{#1{foo}#2}{}%
}
\def\TestGroup{%
  \Test\getrefnumber\\%
  \Test\getpagerefnumber\\%
  \Test\getrefbykeydefault{}{}\\%
  \Test\getrefbykeydefault{page}{}\\%
  \Test\getrefbykeydefault{anchor}{}\\%
  \Test\getrefbykeydefault{name}{}\\%
  \Test\getrefbykeydefault{url}{}\\%
}
\TestGroup
\Test\getrefbykeydefault{title}{}\\%
\msg{}
\def\r@foo{\par\par\par\par\par\par\par\par}
\long\def\Test#1#2\\{%
  \test{#1{foo}#2}{\par}%
}
\TestGroup
\test{\getrefbykeydefault{title}{}{}}{}
\msg{}
\def\r@foo{{ }{ }{ }{ }{ }}
\def\Test#1#2\\{%
  \test{#1{foo}#2}{ }%
}
\TestGroup
\msg{}
\long\def\TestDefault#1{%
  \begingroup
    \setrefcountdefault{#1}%
    \test{\getrefnumber{foo}}{#1}%
    \test{\getpagerefnumber{foo}}{#1}%
  \endgroup
}
\def\TestDefaultX{%
  \TestDefault{}%
  \TestDefault{\par}%
  \TestDefault{ }%
  \TestDefault{\space}%
}
\let\r@foo\@undefined
\TestDefaultX
\let\r@foo\relax
\TestDefaultX
\def\r@foo{}
\TestDefaultX
%    \end{macrocode}
%    \begin{macrocode}
\msg{}
\long\def\Test#1#2#3#4{%
  \begingroup
    \def\TestTask{#1}%
    \@onelevel@sanitize\TestTask
    \msg{* [\TestTask]}%
    \edef\TestResultA{\IfRefUndefinedExpandable{#1}{#2}{#3}}%
    \IfRefUndefinedBabel{#1}{%
      \def\TestResultB{#2}%
    }{%
      \def\TestResultB{#3}%
    }%
    \def\TestExpected{#4}%
    \ifx\TestResultA\TestExpected
      \msg{ \space ok.}%
    \else
      \begingroup
        \@onelevel@sanitize\TestResultA
        \@onelevel@sanitize\TestExpected
        \msg{ \space Result: \space\space[\TestResultA]}%
        \msg{ \space Expected: [\TestExpected]}%
        \errmessage{Test failed!}%
      \endgroup
    \fi
    \ifx\TestResultB\TestExpected
      \msg{ \space ok.}%
    \else
      \begingroup
        \@onelevel@sanitize\TestResultB
        \@onelevel@sanitize\TestExpected
        \msg{ \space Result: \space\space[\TestResultB]}%
        \msg{ \space Expected: [\TestExpected]}%
        \errmessage{Test failed!}%
      \endgroup
    \fi
  \endgroup
}
\begingroup
  \def\r@foo{{}{}}%
  \let\r@bar\@undefined
  \let\r@xyz\relax
  \Test{foo}{true}{false}{false}%
  \Test{bar}{true}{false}{true}%
  \Test{xyz}{true}{false}{true}%
\endgroup
%    \end{macrocode}
%    \begin{macrocode}
\csname @@end\endcsname\end
%</test2>
%    \end{macrocode}
% \subsection{Test with package \xpackage{titleref}}
%
%    \begin{macrocode}
%<*test3>
\NeedsTeXFormat{LaTeX2e}
\documentclass{article}
\usepackage{refcount}[2011/10/16]
%<test4>\usepackage{nameref}
\usepackage{titleref}
\begin{document}
\section{Hello World}
\label{sec:hello}
\section{\hbox{xy}}
\label{sec:foo}
%
\makeatletter
\@ifundefined{r@sec:hello}{%
  \typeout{==> Compile twice!}%
}{%
  \def\test#1#2{%
    \begingroup
      \def\TestTask{#1}%
      \@onelevel@sanitize\TestTask
      \typeout{* \TestTask}%
      \expandafter\expandafter\expandafter\def
      \expandafter\expandafter\expandafter\TestResult
      \expandafter\expandafter\expandafter{%
        #1%
      }%
      \def\TestExpected{#2}%
      \ifx\TestResult\TestExpected
        \typeout{ \space ok.}%
      \else
        \@onelevel@sanitize\TestResult
        \@onelevel@sanitize\TestExpected
        \typeout{ \space Result: \space\space[\TestResult]}%
        \typeout{ \space Expected: [\TestExpected]}%
        \errmessage{Test failed!}%
      \fi
    \endgroup
  }%
  \test{\getrefbykeydefault{sec:hello}{title}{}}{Hello World}%
  \test{\getrefbykeydefault{sec:foo}{title}{}}{\hbox{xy}}%
  \begingroup
    \def\hbox#1{[#1]}% hash-ok
    \test{\getrefbykeydefault{sec:foo}{title}{}}{\hbox{xy}}%
  \endgroup
}
\makeatother
%    \end{macrocode}
%    \begin{macrocode}
\end{document}
%</test3>
%    \end{macrocode}
%    \begin{macrocode}
%<*test5>
\NeedsTeXFormat{LaTeX2e}
\documentclass{book}
\usepackage{refcount}[2011/10/16]
\usepackage{zref-runs}
\newcounter{test}
\begin{document}
\ifnum\zruns>1 %
  \makeatletter
  \def\Test#1#2#3{%
    \begingroup
      \setcounter{test}{10}%
      \sbox0{%
        #1{test}{#2}%
        \ifnum#3=\value{test}%
        \else
          \PackageError{test}{\string#1{#2} <> #3 (\the\value{test})}%
        \fi
      }%
      \ifdim\wd0=0pt %
      \else
        \PackageError{test}{Non-empty box}\@ehc
      \fi
    \endgroup
  }%
  \makeatother
  \Test\setcounterpageref{ch:two}{1}%
  \Test\setcounterpageref{ch:three}{3}%
  \Test\setcounterpageref{ch:four}{5}%
  \Test\setcounterpageref{ch:five}{7}%
  \Test\setcounterpageref{ch:six}{9}%
  \Test\setcounterpageref{ch:seven}{13}%
  \Test\addtocounterpageref{ch:two}{11}%
  \Test\addtocounterpageref{ch:three}{13}%
  \Test\addtocounterpageref{ch:four}{15}%
  \Test\addtocounterpageref{ch:five}{17}%
  \Test\addtocounterpageref{ch:six}{19}%
  \Test\addtocounterpageref{ch:seven}{23}%
  \Test\setcounterref{ch:two}{1}%
  \Test\setcounterref{ch:three}{2}%
  \Test\setcounterref{ch:four}{11}%
  \Test\addtocounterref{ch:two}{11}%
  \Test\addtocounterref{ch:three}{12}%
  \Test\addtocounterref{ch:four}{21}%
\fi
\frontmatter
\chapter{Chapter one}\label{ch:one}
\cleardoublepage
\mainmatter
\chapter{Chapter two}\label{ch:two}
\cleardoublepage
\chapter{Chapter three}\label{ch:three}
\cleardoublepage
\setcounter{chapter}{10}
\chapter{Chapter four}\label{ch:four}
\cleardoublepage
\appendix
\chapter{Chapter five}\label{ch:five}
\cleardoublepage
\chapter{Chapter six}\label{ch:six}
\cleardoublepage
\null
\cleardoublepage
\chapter{Chapter seven}\label{ch:seven}
\end{document}
%</test5>
%    \end{macrocode}
%
% \section{Installation}
%
% \subsection{Download}
%
% \paragraph{Package.} This package is available on
% CTAN\footnote{\url{ftp://ftp.ctan.org/tex-archive/}}:
% \begin{description}
% \item[\CTAN{macros/latex/contrib/oberdiek/refcount.dtx}] The source file.
% \item[\CTAN{macros/latex/contrib/oberdiek/refcount.pdf}] Documentation.
% \end{description}
%
%
% \paragraph{Bundle.} All the packages of the bundle `oberdiek'
% are also available in a TDS compliant ZIP archive. There
% the packages are already unpacked and the documentation files
% are generated. The files and directories obey the TDS standard.
% \begin{description}
% \item[\CTAN{install/macros/latex/contrib/oberdiek.tds.zip}]
% \end{description}
% \emph{TDS} refers to the standard ``A Directory Structure
% for \TeX\ Files'' (\CTAN{tds/tds.pdf}). Directories
% with \xfile{texmf} in their name are usually organized this way.
%
% \subsection{Bundle installation}
%
% \paragraph{Unpacking.} Unpack the \xfile{oberdiek.tds.zip} in the
% TDS tree (also known as \xfile{texmf} tree) of your choice.
% Example (linux):
% \begin{quote}
%   |unzip oberdiek.tds.zip -d ~/texmf|
% \end{quote}
%
% \paragraph{Script installation.}
% Check the directory \xfile{TDS:scripts/oberdiek/} for
% scripts that need further installation steps.
% Package \xpackage{attachfile2} comes with the Perl script
% \xfile{pdfatfi.pl} that should be installed in such a way
% that it can be called as \texttt{pdfatfi}.
% Example (linux):
% \begin{quote}
%   |chmod +x scripts/oberdiek/pdfatfi.pl|\\
%   |cp scripts/oberdiek/pdfatfi.pl /usr/local/bin/|
% \end{quote}
%
% \subsection{Package installation}
%
% \paragraph{Unpacking.} The \xfile{.dtx} file is a self-extracting
% \docstrip\ archive. The files are extracted by running the
% \xfile{.dtx} through \plainTeX:
% \begin{quote}
%   \verb|tex refcount.dtx|
% \end{quote}
%
% \paragraph{TDS.} Now the different files must be moved into
% the different directories in your installation TDS tree
% (also known as \xfile{texmf} tree):
% \begin{quote}
% \def\t{^^A
% \begin{tabular}{@{}>{\ttfamily}l@{ $\rightarrow$ }>{\ttfamily}l@{}}
%   refcount.sty & tex/latex/oberdiek/refcount.sty\\
%   refcount.pdf & doc/latex/oberdiek/refcount.pdf\\
%   test/refcount-test1.tex & doc/latex/oberdiek/test/refcount-test1.tex\\
%   test/refcount-test2.tex & doc/latex/oberdiek/test/refcount-test2.tex\\
%   test/refcount-test3.tex & doc/latex/oberdiek/test/refcount-test3.tex\\
%   test/refcount-test4.tex & doc/latex/oberdiek/test/refcount-test4.tex\\
%   test/refcount-test5.tex & doc/latex/oberdiek/test/refcount-test5.tex\\
%   refcount.dtx & source/latex/oberdiek/refcount.dtx\\
% \end{tabular}^^A
% }^^A
% \sbox0{\t}^^A
% \ifdim\wd0>\linewidth
%   \begingroup
%     \advance\linewidth by\leftmargin
%     \advance\linewidth by\rightmargin
%   \edef\x{\endgroup
%     \def\noexpand\lw{\the\linewidth}^^A
%   }\x
%   \def\lwbox{^^A
%     \leavevmode
%     \hbox to \linewidth{^^A
%       \kern-\leftmargin\relax
%       \hss
%       \usebox0
%       \hss
%       \kern-\rightmargin\relax
%     }^^A
%   }^^A
%   \ifdim\wd0>\lw
%     \sbox0{\small\t}^^A
%     \ifdim\wd0>\linewidth
%       \ifdim\wd0>\lw
%         \sbox0{\footnotesize\t}^^A
%         \ifdim\wd0>\linewidth
%           \ifdim\wd0>\lw
%             \sbox0{\scriptsize\t}^^A
%             \ifdim\wd0>\linewidth
%               \ifdim\wd0>\lw
%                 \sbox0{\tiny\t}^^A
%                 \ifdim\wd0>\linewidth
%                   \lwbox
%                 \else
%                   \usebox0
%                 \fi
%               \else
%                 \lwbox
%               \fi
%             \else
%               \usebox0
%             \fi
%           \else
%             \lwbox
%           \fi
%         \else
%           \usebox0
%         \fi
%       \else
%         \lwbox
%       \fi
%     \else
%       \usebox0
%     \fi
%   \else
%     \lwbox
%   \fi
% \else
%   \usebox0
% \fi
% \end{quote}
% If you have a \xfile{docstrip.cfg} that configures and enables \docstrip's
% TDS installing feature, then some files can already be in the right
% place, see the documentation of \docstrip.
%
% \subsection{Refresh file name databases}
%
% If your \TeX~distribution
% (\teTeX, \mikTeX, \dots) relies on file name databases, you must refresh
% these. For example, \teTeX\ users run \verb|texhash| or
% \verb|mktexlsr|.
%
% \subsection{Some details for the interested}
%
% \paragraph{Attached source.}
%
% The PDF documentation on CTAN also includes the
% \xfile{.dtx} source file. It can be extracted by
% AcrobatReader 6 or higher. Another option is \textsf{pdftk},
% e.g. unpack the file into the current directory:
% \begin{quote}
%   \verb|pdftk refcount.pdf unpack_files output .|
% \end{quote}
%
% \paragraph{Unpacking with \LaTeX.}
% The \xfile{.dtx} chooses its action depending on the format:
% \begin{description}
% \item[\plainTeX:] Run \docstrip\ and extract the files.
% \item[\LaTeX:] Generate the documentation.
% \end{description}
% If you insist on using \LaTeX\ for \docstrip\ (really,
% \docstrip\ does not need \LaTeX), then inform the autodetect routine
% about your intention:
% \begin{quote}
%   \verb|latex \let\install=y\input{refcount.dtx}|
% \end{quote}
% Do not forget to quote the argument according to the demands
% of your shell.
%
% \paragraph{Generating the documentation.}
% You can use both the \xfile{.dtx} or the \xfile{.drv} to generate
% the documentation. The process can be configured by the
% configuration file \xfile{ltxdoc.cfg}. For instance, put this
% line into this file, if you want to have A4 as paper format:
% \begin{quote}
%   \verb|\PassOptionsToClass{a4paper}{article}|
% \end{quote}
% An example follows how to generate the
% documentation with pdf\LaTeX:
% \begin{quote}
%\begin{verbatim}
%pdflatex refcount.dtx
%makeindex -s gind.ist refcount.idx
%pdflatex refcount.dtx
%makeindex -s gind.ist refcount.idx
%pdflatex refcount.dtx
%\end{verbatim}
% \end{quote}
%
% \section{Catalogue}
%
% The following XML file can be used as source for the
% \href{http://mirror.ctan.org/help/Catalogue/catalogue.html}{\TeX\ Catalogue}.
% The elements \texttt{caption} and \texttt{description} are imported
% from the original XML file from the Catalogue.
% The name of the XML file in the Catalogue is \xfile{refcount.xml}.
%    \begin{macrocode}
%<*catalogue>
<?xml version='1.0' encoding='us-ascii'?>
<!DOCTYPE entry SYSTEM 'catalogue.dtd'>
<entry datestamp='$Date$' modifier='$Author$' id='refcount'>
  <name>refcount</name>
  <caption>Counter operations with label references.</caption>
  <authorref id='auth:oberdiek'/>
  <copyright owner='Heiko Oberdiek' year='1998,2000,2006,2008,2010,2011'/>
  <license type='lppl1.3'/>
  <version number='3.4'/>
  <description>
    Provides commands <tt>\setcounterref</tt> and
    <tt>\addtocounterref</tt> which use the section (or whatever)
    number from the reference as the value to put into the counter, as
    in:

    <pre>
    ...\label{sec:foo}
    ...
    \setcounterref{foonum}{sec:foo}
    </pre>
    Commands <tt>\setcounterpageref</tt> and
    <tt>\addtocounterpageref</tt> do the corresponding thing with the
    page reference of the label.
    <p/>
    No <tt>.ins</tt> file is distributed; process the
    <tt>.dtx</tt> with plain TeX to create one.
    <p/>
    The package is part of the <xref refid='oberdiek'>oberdiek</xref>
    bundle.
  </description>
  <documentation details='Package documentation'
      href='ctan:/macros/latex/contrib/oberdiek/refcount.pdf'/>
  <ctan file='true' path='/macros/latex/contrib/oberdiek/refcount.dtx'/>
  <miktex location='oberdiek'/>
  <texlive location='oberdiek'/>
  <install path='/macros/latex/contrib/oberdiek/oberdiek.tds.zip'/>
</entry>
%</catalogue>
%    \end{macrocode}
%
% \begin{History}
%   \begin{Version}{1998/04/08 v1.0}
%   \item
%     First public release, written as answer in the
%     newsgroup \xnewsgroup{comp.text.tex}:
%     \URL{``\link{Re: Adding a \cs{ref} to a counter?}''}^^A
%     {http://groups.google.com/group/comp.text.tex/msg/c3f2a135ef5ee528}
%   \end{Version}
%   \begin{Version}{2000/09/07 v2.0}
%   \item
%     Documentation added.
%   \item
%     LPPL 1.2
%   \item
%     Package rewritten, new commands added.
%   \end{Version}
%   \begin{Version}{2006/02/20 v3.0}
%   \item
%     Support for \xpackage{hyperref} and \xpackage{nameref} improved.
%   \item
%     Support for \xpackage{titleref} and \xpackage{babel}'s shorthands added.
%   \item
%     New: \cs{refused}, \cs{getrefbykeydefault}
%   \end{Version}
%   \begin{Version}{2008/08/11 v3.1}
%   \item
%     Code is not changed.
%   \item
%     URLs updated.
%   \end{Version}
%   \begin{Version}{2010/12/01 v3.2}
%   \item
%     \cs{IfRefUndefinedExpandable} and \cs{IfRefUndefinedBabel} added.
%   \item
%     \cs{getrefnumber}, \cs{getpagerefnumber}, \cs{getrefbykeydefault}
%     are expandable in exact two expansion steps.
%   \item
%     Non-expandable macros are made robust.
%   \item
%     Test files added.
%   \end{Version}
%   \begin{Version}{2011/06/22 v3.3}
%   \item
%     Bug fix: \cs{rc@refused} is undefined for \cs{setcounterpageref}
%     and similar macros. (Bug found by Marc van Dongen.)
%   \end{Version}
%   \begin{Version}{2011/10/16 v3.4}
%   \item
%     Bug fix: \cs{setcounterpageref} and \cs{addtocounterpageref} fixed.
%     (Bug found by Staz.)
%   \item
%     Macros \cs{(set|addto)counter(page|)ref} are made robust.
%   \end{Version}
% \end{History}
%
% \PrintIndex
%
% \Finale
\endinput
|
% \end{quote}
% Do not forget to quote the argument according to the demands
% of your shell.
%
% \paragraph{Generating the documentation.}
% You can use both the \xfile{.dtx} or the \xfile{.drv} to generate
% the documentation. The process can be configured by the
% configuration file \xfile{ltxdoc.cfg}. For instance, put this
% line into this file, if you want to have A4 as paper format:
% \begin{quote}
%   \verb|\PassOptionsToClass{a4paper}{article}|
% \end{quote}
% An example follows how to generate the
% documentation with pdf\LaTeX:
% \begin{quote}
%\begin{verbatim}
%pdflatex refcount.dtx
%makeindex -s gind.ist refcount.idx
%pdflatex refcount.dtx
%makeindex -s gind.ist refcount.idx
%pdflatex refcount.dtx
%\end{verbatim}
% \end{quote}
%
% \section{Catalogue}
%
% The following XML file can be used as source for the
% \href{http://mirror.ctan.org/help/Catalogue/catalogue.html}{\TeX\ Catalogue}.
% The elements \texttt{caption} and \texttt{description} are imported
% from the original XML file from the Catalogue.
% The name of the XML file in the Catalogue is \xfile{refcount.xml}.
%    \begin{macrocode}
%<*catalogue>
<?xml version='1.0' encoding='us-ascii'?>
<!DOCTYPE entry SYSTEM 'catalogue.dtd'>
<entry datestamp='$Date$' modifier='$Author$' id='refcount'>
  <name>refcount</name>
  <caption>Counter operations with label references.</caption>
  <authorref id='auth:oberdiek'/>
  <copyright owner='Heiko Oberdiek' year='1998,2000,2006,2008,2010,2011'/>
  <license type='lppl1.3'/>
  <version number='3.4'/>
  <description>
    Provides commands <tt>\setcounterref</tt> and
    <tt>\addtocounterref</tt> which use the section (or whatever)
    number from the reference as the value to put into the counter, as
    in:

    <pre>
    ...\label{sec:foo}
    ...
    \setcounterref{foonum}{sec:foo}
    </pre>
    Commands <tt>\setcounterpageref</tt> and
    <tt>\addtocounterpageref</tt> do the corresponding thing with the
    page reference of the label.
    <p/>
    No <tt>.ins</tt> file is distributed; process the
    <tt>.dtx</tt> with plain TeX to create one.
    <p/>
    The package is part of the <xref refid='oberdiek'>oberdiek</xref>
    bundle.
  </description>
  <documentation details='Package documentation'
      href='ctan:/macros/latex/contrib/oberdiek/refcount.pdf'/>
  <ctan file='true' path='/macros/latex/contrib/oberdiek/refcount.dtx'/>
  <miktex location='oberdiek'/>
  <texlive location='oberdiek'/>
  <install path='/macros/latex/contrib/oberdiek/oberdiek.tds.zip'/>
</entry>
%</catalogue>
%    \end{macrocode}
%
% \begin{History}
%   \begin{Version}{1998/04/08 v1.0}
%   \item
%     First public release, written as answer in the
%     newsgroup \xnewsgroup{comp.text.tex}:
%     \URL{``\link{Re: Adding a \cs{ref} to a counter?}''}^^A
%     {http://groups.google.com/group/comp.text.tex/msg/c3f2a135ef5ee528}
%   \end{Version}
%   \begin{Version}{2000/09/07 v2.0}
%   \item
%     Documentation added.
%   \item
%     LPPL 1.2
%   \item
%     Package rewritten, new commands added.
%   \end{Version}
%   \begin{Version}{2006/02/20 v3.0}
%   \item
%     Support for \xpackage{hyperref} and \xpackage{nameref} improved.
%   \item
%     Support for \xpackage{titleref} and \xpackage{babel}'s shorthands added.
%   \item
%     New: \cs{refused}, \cs{getrefbykeydefault}
%   \end{Version}
%   \begin{Version}{2008/08/11 v3.1}
%   \item
%     Code is not changed.
%   \item
%     URLs updated.
%   \end{Version}
%   \begin{Version}{2010/12/01 v3.2}
%   \item
%     \cs{IfRefUndefinedExpandable} and \cs{IfRefUndefinedBabel} added.
%   \item
%     \cs{getrefnumber}, \cs{getpagerefnumber}, \cs{getrefbykeydefault}
%     are expandable in exact two expansion steps.
%   \item
%     Non-expandable macros are made robust.
%   \item
%     Test files added.
%   \end{Version}
%   \begin{Version}{2011/06/22 v3.3}
%   \item
%     Bug fix: \cs{rc@refused} is undefined for \cs{setcounterpageref}
%     and similar macros. (Bug found by Marc van Dongen.)
%   \end{Version}
%   \begin{Version}{2011/10/16 v3.4}
%   \item
%     Bug fix: \cs{setcounterpageref} and \cs{addtocounterpageref} fixed.
%     (Bug found by Staz.)
%   \item
%     Macros \cs{(set|addto)counter(page|)ref} are made robust.
%   \end{Version}
% \end{History}
%
% \PrintIndex
%
% \Finale
\endinput

%        (quote the arguments according to the demands of your shell)
%
% Documentation:
%    (a) If refcount.drv is present:
%           latex refcount.drv
%    (b) Without refcount.drv:
%           latex refcount.dtx; ...
%    The class ltxdoc loads the configuration file ltxdoc.cfg
%    if available. Here you can specify further options, e.g.
%    use A4 as paper format:
%       \PassOptionsToClass{a4paper}{article}
%
%    Programm calls to get the documentation (example):
%       pdflatex refcount.dtx
%       makeindex -s gind.ist refcount.idx
%       pdflatex refcount.dtx
%       makeindex -s gind.ist refcount.idx
%       pdflatex refcount.dtx
%
% Installation:
%    TDS:tex/latex/oberdiek/refcount.sty
%    TDS:doc/latex/oberdiek/refcount.pdf
%    TDS:doc/latex/oberdiek/test/refcount-test1.tex
%    TDS:doc/latex/oberdiek/test/refcount-test2.tex
%    TDS:doc/latex/oberdiek/test/refcount-test3.tex
%    TDS:doc/latex/oberdiek/test/refcount-test4.tex
%    TDS:doc/latex/oberdiek/test/refcount-test5.tex
%    TDS:source/latex/oberdiek/refcount.dtx
%
%<*ignore>
\begingroup
  \catcode123=1 %
  \catcode125=2 %
  \def\x{LaTeX2e}%
\expandafter\endgroup
\ifcase 0\ifx\install y1\fi\expandafter
         \ifx\csname processbatchFile\endcsname\relax\else1\fi
         \ifx\fmtname\x\else 1\fi\relax
\else\csname fi\endcsname
%</ignore>
%<*install>
\input docstrip.tex
\Msg{************************************************************************}
\Msg{* Installation}
\Msg{* Package: refcount 2011/10/16 v3.4 Data extraction from label references (HO)}
\Msg{************************************************************************}

\keepsilent
\askforoverwritefalse

\let\MetaPrefix\relax
\preamble

This is a generated file.

Project: refcount
Version: 2011/10/16 v3.4

Copyright (C) 1998, 2000, 2006, 2008, 2010, 2011 by
   Heiko Oberdiek <heiko.oberdiek at googlemail.com>

This work may be distributed and/or modified under the
conditions of the LaTeX Project Public License, either
version 1.3c of this license or (at your option) any later
version. This version of this license is in
   http://www.latex-project.org/lppl/lppl-1-3c.txt
and the latest version of this license is in
   http://www.latex-project.org/lppl.txt
and version 1.3 or later is part of all distributions of
LaTeX version 2005/12/01 or later.

This work has the LPPL maintenance status "maintained".

This Current Maintainer of this work is Heiko Oberdiek.

This work consists of the main source file refcount.dtx
and the derived files
   refcount.sty, refcount.pdf, refcount.ins, refcount.drv,
   refcount-test1.tex, refcount-test2.tex, refcount-test3.tex,
   refcount-test4.tex, refcount-test5.tex.

\endpreamble
\let\MetaPrefix\DoubleperCent

\generate{%
  \file{refcount.ins}{\from{refcount.dtx}{install}}%
  \file{refcount.drv}{\from{refcount.dtx}{driver}}%
  \usedir{tex/latex/oberdiek}%
  \file{refcount.sty}{\from{refcount.dtx}{package}}%
  \usedir{doc/latex/oberdiek/test}%
  \file{refcount-test1.tex}{\from{refcount.dtx}{test1}}%
  \file{refcount-test2.tex}{\from{refcount.dtx}{test2}}%
  \file{refcount-test3.tex}{\from{refcount.dtx}{test3}}%
  \file{refcount-test4.tex}{\from{refcount.dtx}{test3,test4}}%
  \file{refcount-test5.tex}{\from{refcount.dtx}{test5}}%
  \nopreamble
  \nopostamble
  \usedir{source/latex/oberdiek/catalogue}%
  \file{refcount.xml}{\from{refcount.dtx}{catalogue}}%
}

\catcode32=13\relax% active space
\let =\space%
\Msg{************************************************************************}
\Msg{*}
\Msg{* To finish the installation you have to move the following}
\Msg{* file into a directory searched by TeX:}
\Msg{*}
\Msg{*     refcount.sty}
\Msg{*}
\Msg{* To produce the documentation run the file `refcount.drv'}
\Msg{* through LaTeX.}
\Msg{*}
\Msg{* Happy TeXing!}
\Msg{*}
\Msg{************************************************************************}

\endbatchfile
%</install>
%<*ignore>
\fi
%</ignore>
%<*driver>
\NeedsTeXFormat{LaTeX2e}
\ProvidesFile{refcount.drv}%
  [2011/10/16 v3.4 Data extraction from label references (HO)]%
\documentclass{ltxdoc}
\usepackage{holtxdoc}[2011/11/22]
\begin{document}
  \DocInput{refcount.dtx}%
\end{document}
%</driver>
% \fi
%
% \CheckSum{1185}
%
% \CharacterTable
%  {Upper-case    \A\B\C\D\E\F\G\H\I\J\K\L\M\N\O\P\Q\R\S\T\U\V\W\X\Y\Z
%   Lower-case    \a\b\c\d\e\f\g\h\i\j\k\l\m\n\o\p\q\r\s\t\u\v\w\x\y\z
%   Digits        \0\1\2\3\4\5\6\7\8\9
%   Exclamation   \!     Double quote  \"     Hash (number) \#
%   Dollar        \$     Percent       \%     Ampersand     \&
%   Acute accent  \'     Left paren    \(     Right paren   \)
%   Asterisk      \*     Plus          \+     Comma         \,
%   Minus         \-     Point         \.     Solidus       \/
%   Colon         \:     Semicolon     \;     Less than     \<
%   Equals        \=     Greater than  \>     Question mark \?
%   Commercial at \@     Left bracket  \[     Backslash     \\
%   Right bracket \]     Circumflex    \^     Underscore    \_
%   Grave accent  \`     Left brace    \{     Vertical bar  \|
%   Right brace   \}     Tilde         \~}
%
% \GetFileInfo{refcount.drv}
%
% \title{The \xpackage{refcount} package}
% \date{2011/10/16 v3.4}
% \author{Heiko Oberdiek\\\xemail{heiko.oberdiek at googlemail.com}}
%
% \maketitle
%
% \begin{abstract}
% References are not numbers, however they often store numerical
% data such as section or page numbers. \cs{ref} or \cs{pageref}
% cannot be used for counter assignments or calculations because
% they are not expandable, generate warnings, or can even be links.
% The package provides expandable macros to extract the data
% from references. Packages \xpackage{hyperref}, \xpackage{nameref},
% \xpackage{titleref}, and \xpackage{babel} are supported.
% \end{abstract}
%
% \tableofcontents
%
% \section{Usage}
%
% \subsection{Setting counters}
%
% The following commands are similar to \LaTeX's
% \cs{setcounter} and \cs{addtocounter},
% but they extract the number value from a reference:
% \begin{quote}
%   \cs{setcounterref}, \cs{addtocounterref}\\
%   \cs{setcounterpageref}, \cs{addtocounterpageref}
% \end{quote}
% They take two arguments:
% \begin{quote}
%    \cs{...counter...ref} |{|\meta{\LaTeX\ counter}|}|
%    |{|\meta{reference}|}|
% \end{quote}
% An undefined references produces the usual LaTeX warning
% and its value is assumed to be zero.
% Example:
% \begin{quote}
%\begin{verbatim}
%\newcounter{ctrA}
%\newcounter{ctrB}
%\refstepcounter{ctrA}\label{ref:A}
%\setcounterref{ctrB}{ref:A}
%\addtocounterpageref{ctrB}{ref:A}
%\end{verbatim}
% \end{quote}
%
% \subsection{Expandable commands}
%
% These commands that can be used in expandible contexts
% (inside calculations, \cs{edef}, \cs{csname}, \cs{write}, \dots):
% \begin{quote}
%   \cs{getrefnumber}, \cs{getpagerefnumber}
% \end{quote}
% They take one argument, the reference:
% \begin{quote}
%   \cs{get...refnumber} |{|\meta{reference}|}|
% \end{quote}
% The default for undefined references can be changed
% with macro \cs{setrefcountdefault}, for example this
% package calls:
% \begin{quote}
%   \cs{setrefcountdefault}|{0}|
% \end{quote}
%
% Since version 2.0 of this package there is a new
% command:
% \begin{quote}
%   \cs{getrefbykeydefault} |{|\meta{reference}|}|
%   |{|\meta{key}|}| |{|\meta{default}|}|
% \end{quote}
% This generalized version allows the extraction
% of further properties of a reference than the
% two standard ones. Thus the following properties
% are supported, if they are available:
% \begin{quote}
% \begin{tabular}{@{}l|l|l@{}}
%    Key & Description & Package\\
% \hline
%   \meta{empty} & same as \cs{ref} & \LaTeX\\
%   |page| & same as \cs{pageref} & \LaTeX\\
%   |title| & section and caption titles & \xpackage{titleref}\\
%   |name| & section and caption titles & \xpackage{nameref}\\
%   |anchor| & anchor name & \xpackage{hyperref}\\
%   |url| & url/file & \xpackage{hyperref}/\xpackage{xr}
% \end{tabular}
% \end{quote}
%
% Since version 3.2 the expandable macros described before
% in this section
% are expandable in exact two expansion steps.
%
% \subsection{Undefined references}
%
% Because warnings and assignments cannot be used in
% expandible contexts, undefined references do not
% produce a warning, their values are assumed to be zero.
% Example:
% \begin{quote}
%\begin{verbatim}
%\label{ref:here}% somewhere
%\refused{ref:here}% see below
%\ifodd\getpagerefnumber{ref:here}%
%  reference is on an odd page
%\else
%  reference is on an even page
%\fi
%\end{verbatim}
% \end{quote}
%
% In case of undefined references the user usually want's
% to be informed. Also \LaTeX\ prints a warning at
% the end of the \LaTeX\ run. To notify \LaTeX\ and
% get a normal warning, just use
% \begin{quote}
%   \cs{refused} |{|\meta{reference}|}|
% \end{quote}
% outside the expanding context. Example, see above.
%
% \subsubsection{Check for undefined references}
%
% In version 3.2 macros were added, that test, whether references
% are defined.
% \begin{declcs}{IfRefUndefinedExpandable} \M{refname} \M{then} \M{else}\\
%   \cs{IfRefUndefinedBabel} \M{refname} \M{then} \M{else}
% \end{declcs}
% If the reference is not available and therefore undefined, then
% argument \meta{then} is executed, otherwise argument \meta{else}
% is called. Macro \cs{IfRefUndefinedExpandable} is expandable,
% but \meta{refname} must not contain babel shorthand characters.
% Macro \cs{IfRefUndefinedBabel} supports shorthand characters of
% babel, but it is not expandable.
%
% \subsection{Notes}
%
% \begin{itemize}
% \item
%   The method of extracting the number in this
%   package also works in cases, where the
%   reference cannot be used directly, because
%   a package such as \xpackage{hyperref} has added
%   extra stuff (hyper link), so that the reference cannot
%   be used as number any more.
% \item
%   If the reference does not contain a number,
%   assignments to a counter will fail of course.
% \end{itemize}
%
%
% \StopEventually{
% }
%
% \section{Implementation}
%
%    \begin{macrocode}
%<*package>
%    \end{macrocode}
%    Reload check, especially if the package is not used with \LaTeX.
%    \begin{macrocode}
\begingroup\catcode61\catcode48\catcode32=10\relax%
  \catcode13=5 % ^^M
  \endlinechar=13 %
  \catcode35=6 % #
  \catcode39=12 % '
  \catcode44=12 % ,
  \catcode45=12 % -
  \catcode46=12 % .
  \catcode58=12 % :
  \catcode64=11 % @
  \catcode123=1 % {
  \catcode125=2 % }
  \expandafter\let\expandafter\x\csname ver@refcount.sty\endcsname
  \ifx\x\relax % plain-TeX, first loading
  \else
    \def\empty{}%
    \ifx\x\empty % LaTeX, first loading,
      % variable is initialized, but \ProvidesPackage not yet seen
    \else
      \expandafter\ifx\csname PackageInfo\endcsname\relax
        \def\x#1#2{%
          \immediate\write-1{Package #1 Info: #2.}%
        }%
      \else
        \def\x#1#2{\PackageInfo{#1}{#2, stopped}}%
      \fi
      \x{refcount}{The package is already loaded}%
      \aftergroup\endinput
    \fi
  \fi
\endgroup%
%    \end{macrocode}
%    Package identification:
%    \begin{macrocode}
\begingroup\catcode61\catcode48\catcode32=10\relax%
  \catcode13=5 % ^^M
  \endlinechar=13 %
  \catcode35=6 % #
  \catcode39=12 % '
  \catcode40=12 % (
  \catcode41=12 % )
  \catcode44=12 % ,
  \catcode45=12 % -
  \catcode46=12 % .
  \catcode47=12 % /
  \catcode58=12 % :
  \catcode64=11 % @
  \catcode91=12 % [
  \catcode93=12 % ]
  \catcode123=1 % {
  \catcode125=2 % }
  \expandafter\ifx\csname ProvidesPackage\endcsname\relax
    \def\x#1#2#3[#4]{\endgroup
      \immediate\write-1{Package: #3 #4}%
      \xdef#1{#4}%
    }%
  \else
    \def\x#1#2[#3]{\endgroup
      #2[{#3}]%
      \ifx#1\@undefined
        \xdef#1{#3}%
      \fi
      \ifx#1\relax
        \xdef#1{#3}%
      \fi
    }%
  \fi
\expandafter\x\csname ver@refcount.sty\endcsname
\ProvidesPackage{refcount}%
  [2011/10/16 v3.4 Data extraction from label references (HO)]%
%    \end{macrocode}
%
%    \begin{macrocode}
\begingroup\catcode61\catcode48\catcode32=10\relax%
  \catcode13=5 % ^^M
  \endlinechar=13 %
  \catcode123=1 % {
  \catcode125=2 % }
  \catcode64=11 % @
  \def\x{\endgroup
    \expandafter\edef\csname rc@AtEnd\endcsname{%
      \endlinechar=\the\endlinechar\relax
      \catcode13=\the\catcode13\relax
      \catcode32=\the\catcode32\relax
      \catcode35=\the\catcode35\relax
      \catcode61=\the\catcode61\relax
      \catcode64=\the\catcode64\relax
      \catcode123=\the\catcode123\relax
      \catcode125=\the\catcode125\relax
    }%
  }%
\x\catcode61\catcode48\catcode32=10\relax%
\catcode13=5 % ^^M
\endlinechar=13 %
\catcode35=6 % #
\catcode64=11 % @
\catcode123=1 % {
\catcode125=2 % }
\def\TMP@EnsureCode#1#2{%
  \edef\rc@AtEnd{%
    \rc@AtEnd
    \catcode#1=\the\catcode#1\relax
  }%
  \catcode#1=#2\relax
}
\TMP@EnsureCode{33}{12}% !
\TMP@EnsureCode{39}{12}% '
\TMP@EnsureCode{42}{12}% *
\TMP@EnsureCode{45}{12}% -
\TMP@EnsureCode{46}{12}% .
\TMP@EnsureCode{47}{12}% /
\TMP@EnsureCode{91}{12}% [
\TMP@EnsureCode{93}{12}% ]
\TMP@EnsureCode{96}{12}% `
\edef\rc@AtEnd{\rc@AtEnd\noexpand\endinput}
%    \end{macrocode}
%
% \subsection{Loading packages}
%
%    \begin{macrocode}
\begingroup\expandafter\expandafter\expandafter\endgroup
\expandafter\ifx\csname RequirePackage\endcsname\relax
  \input ltxcmds.sty\relax
  \input infwarerr.sty\relax
\else
  \RequirePackage{ltxcmds}[2011/11/09]%
  \RequirePackage{infwarerr}[2010/04/08]%
\fi
%    \end{macrocode}
%
% \subsection{Defining commands}
%
%    \begin{macro}{\rc@IfDefinable}
%    \begin{macrocode}
\ltx@IfUndefined{@ifdefinable}{%
  \def\rc@IfDefinable#1{%
    \ifx#1\ltx@undefined
      \expandafter\ltx@firstofone
    \else
      \ifx#1\relax
        \expandafter\expandafter\expandafter\ltx@firstofone
      \else
        \@PackageError{refcount}{%
          Command \string#1 is already defined.\MessageBreak
          It will not redefined by this package%
        }\@ehc
        \expandafter\expandafter\expandafter\ltx@gobble
      \fi
    \fi
  }%
}{%
  \let\rc@IfDefinable\@ifdefinable
}
%    \end{macrocode}
%    \end{macro}
%
%    \begin{macro}{\rc@RobustDefOne}
%    \begin{macro}{\rc@RobustDefZero}
%    \begin{macrocode}
\ltx@IfUndefined{protected}{%
  \ltx@IfUndefined{DeclareRobustCommand}{%
    \def\rc@RobustDefOne#1#2#3#4{%
      \rc@IfDefinable#3{%
        #1\def#3##1{#4}%
      }%
    }%
    \def\rc@RobustDefZero#1#2{%
      \rc@IfDefinable#1{%
        \def#1{#2}%
      }%
    }%
  }{%
    \def\rc@RobustDefOne#1#2#3#4{%
      \rc@IfDefinable#3{%
        \DeclareRobustCommand#2#3[1]{#4}%
      }%
    }%
    \def\rc@RobustDefZero#1#2{%
      \rc@IfDefinable#1{%
        \DeclareRobustCommand#1{#2}%
      }%
    }%
  }%
}{%
  \def\rc@RobustDefOne#1#2#3#4{%
    \rc@IfDefinable#3{%
      \protected#1\def#3##1{#4}%
    }%
  }%
  \def\rc@RobustDefZero#1#2{%
    \rc@IfDefinable#1{%
      \protected\def#1{#2}%
    }%
  }%
}
%    \end{macrocode}
%    \end{macro}
%    \end{macro}
%
%    \begin{macro}{\rc@newcommand}
%    \begin{macrocode}
\ltx@IfUndefined{newcommand}{%
  \def\rc@newcommand*#1[#2]#3{% hash-ok
    \rc@IfDefinable#1{%
      \ifcase#2 %
        \def#1{#3}%
      \or
        \def#1##1{#3}%
      \or
        \def#1##1##2{#3}%
      \else
        \rc@InternalError
      \fi
    }%
  }%
}{%
  \let\rc@newcommand\newcommand
}
%    \end{macrocode}
%    \end{macro}
%
% \subsection{\cs{setrefcountdefault}}
%
%    \begin{macro}{\setrefcountdefault}
%    \begin{macrocode}
\rc@RobustDefOne\long{}\setrefcountdefault{%
  \def\rc@default{#1}%
}
%    \end{macrocode}
%    \end{macro}
%    \begin{macrocode}
\setrefcountdefault{0}
%    \end{macrocode}
%
% \subsection{\cs{refused}}
%
%    \begin{macro}{\refused}
%    \begin{macrocode}
\ltx@IfUndefined{G@refundefinedtrue}{%
  \rc@RobustDefOne{}{*}\refused{%
    \begingroup
      \csname @safe@activestrue\endcsname
      \ltx@IfUndefined{r@#1}{%
        \protect\G@refundefinedtrue
        \rc@WarningUndefined{#1}%
      }{}%
    \endgroup
  }%
}{%
  \rc@RobustDefOne{}{*}\refused{%
    \begingroup
      \csname @safe@activestrue\endcsname
      \ltx@IfUndefined{r@#1}{%
        \csname protect\expandafter\endcsname
        \csname G@refundefinedtrue\endcsname
        \rc@WarningUndefined{#1}%
      }{}%
    \endgroup
  }%
}
%    \end{macrocode}
%    \end{macro}
%    \begin{macro}{\rc@WarningUndefined}
%    \begin{macrocode}
\ltx@IfUndefined{@latex@warning}{%
  \def\rc@WarningUndefined#1{%
    \ltx@ifundefined{thepage}{%
      \def\thepage{\number\count0 }%
    }{}%
    \@PackageWarning{refcount}{%
      Reference `#1' on page \thepage\space undefined%
    }%
  }%
}{%
  \def\rc@WarningUndefined#1{%
    \@latex@warning{%
      Reference `#1' on page \thepage\space undefined%
    }%
  }%
}
%    \end{macrocode}
%    \end{macro}
%
% \subsection{Setting counters by reference data}
%
% \subsubsection{Generic setting}
%
%    \begin{macro}{\rc@set}
% Generic command for
% |\|$\{$|set|$,$|addto|$\}$|counter|$\{$|page|$,\}$|ref|:
%\begin{quote}
%\begin{tabular}[t]{@{}l@{: }l@{}}
% |#1|& \cs{setcounter}, \cs{addtocounter}\\
% |#2|& \cs{ltx@car} (for \cs{ref}), \cs{ltx@cartwo} (for \cs{pageref})\\
% |#3|& \hologo{LaTeX} counter\\
% |#4|& reference\\
%\end{tabular}
%\end{quote}
%    \begin{macrocode}
\def\rc@set#1#2#3#4{%
  \begingroup
    \csname @safe@activestrue\endcsname
    \refused{#4}%
    \expandafter\rc@@set\csname r@#4\endcsname{#1}{#2}{#3}%
  \endgroup
}
%    \end{macrocode}
%    \end{macro}
%    \begin{macro}{\rc@@set}
%\begin{quote}
%\begin{tabular}[t]{@{}l@{: }l@{}}
% |#1|& \cs{r@<...>}\\
% |#2|& \cs{setcounter}, \cs{addtocounter}\\
% |#3|& \cs{ltx@car} (for \cs{ref}), \cs{ltx@carsecond} (for \cs{pageref})\\
% |#4|& \hologo{LaTeX} counter\\
%\end{tabular}
%\end{quote}
%    \begin{macrocode}
\def\rc@@set#1#2#3#4{%
  \ifx#1\relax
    #2{#4}{\rc@default}%
  \else
    #2{#4}{%
      \expandafter#3#1\rc@default\rc@default\@nil
    }%
  \fi
}
%    \end{macrocode}
%    \end{macro}
%
% \subsubsection{User commands}
%
%    \begin{macro}{\setcounterref}
%    \begin{macrocode}
\rc@RobustDefZero\setcounterref{%
  \rc@set\setcounter\ltx@car
}
%    \end{macrocode}
%    \end{macro}
%    \begin{macro}{\addtocounterref}
%    \begin{macrocode}
\rc@RobustDefZero\addtocounterref{%
  \rc@set\addtocounter\ltx@car
}
%    \end{macrocode}
%    \end{macro}
%    \begin{macro}{\setcounterpageref}
%    \begin{macrocode}
\rc@RobustDefZero\setcounterpageref{%
  \rc@set\setcounter\ltx@carsecond
}
%    \end{macrocode}
%    \end{macro}
%    \begin{macro}{\addtocounterpageref}
%    \begin{macrocode}
\rc@RobustDefZero\addtocounterpageref{%
  \rc@set\addtocounter\ltx@carsecond
}
%    \end{macrocode}
%    \end{macro}
%
% \subsection{Extracting references}
%
%    \begin{macro}{\getrefnumber}
%    \begin{macrocode}
\rc@newcommand*{\getrefnumber}[1]{%
  \romannumeral
  \ltx@ifundefined{r@#1}{%
    \expandafter\ltx@zero
    \rc@default
  }{%
    \expandafter\expandafter\expandafter\rc@extract@
    \expandafter\expandafter\expandafter!%
    \csname r@#1\expandafter\endcsname
    \expandafter{\rc@default}\@nil
  }%
}
%    \end{macrocode}
%    \end{macro}
%    \begin{macro}{\getpagerefnumber}
%    \begin{macrocode}
\rc@newcommand*{\getpagerefnumber}[1]{%
  \romannumeral
  \ltx@ifundefined{r@#1}{%
    \expandafter\ltx@zero
    \rc@default
  }{%
    \expandafter\expandafter\expandafter\rc@extract@page
    \expandafter\expandafter\expandafter!%
    \csname r@#1\expandafter\expandafter\expandafter\endcsname
    \expandafter\expandafter\expandafter{%
      \expandafter\rc@default
    \expandafter}\expandafter{\rc@default}\@nil
  }%
}
%    \end{macrocode}
%    \end{macro}
%    \begin{macro}{\getrefbykeydefault}
%    \begin{macrocode}
\rc@newcommand*{\getrefbykeydefault}[2]{%
  \romannumeral
  \expandafter\rc@getrefbykeydefault
    \csname r@#1\expandafter\endcsname
    \csname rc@extract@#2\endcsname
}
%    \end{macrocode}
%    \end{macro}
%    \begin{macro}{\rc@getrefbykeydefault}
%\begin{quote}
%\begin{tabular}[t]{@{}l@{: }l@{}}
% |#1|& \cs{r@<...>}\\
% |#2|& \cs{rc@extract@<...>}\\
% |#3|& default\\
%\end{tabular}
%\end{quote}
%    \begin{macrocode}
\long\def\rc@getrefbykeydefault#1#2#3{%
  \ifx#1\relax
    % reference is undefined
    \ltx@ReturnAfterElseFi{%
      \ltx@zero
      #3%
    }%
  \else
    \ltx@ReturnAfterFi{%
      \ifx#2\relax
        % extract method is missing
        \ltx@ReturnAfterElseFi{%
          \ltx@zero
          #3%
        }%
      \else
        \ltx@ReturnAfterFi{%
          \expandafter
          \rc@generic#1{#3}{#3}{#3}{#3}{#3}\@nil#2{#3}%
        }%
      \fi
    }%
  \fi
}
%    \end{macrocode}
%    \end{macro}
%    \begin{macro}{\rc@generic}
%\begin{quote}
%\begin{tabular}[t]{@{}l@{: }l@{}}
% |#1|& first item in \cs{r@<...>}\\
% |#2|& remaining items in \cs{r@<...>}\\
% |#3|& \cs{rc@extract@<...>}\\
% |#4|& default\\
%\end{tabular}
%\end{quote}
%    \begin{macrocode}
\long\def\rc@generic#1#2\@nil#3#4{%
  #3{#1\TR@TitleReference\@empty{#4}\@nil}{#1}#2\@nil
}
%    \end{macrocode}
%    \end{macro}
%    \begin{macro}{\rc@extract@}
%    \begin{macrocode}
\long\def\rc@extract@#1#2#3\@nil{%
  \ltx@zero
  #2%
}
%    \end{macrocode}
%    \end{macro}
%    \begin{macro}{\rc@extract@page}
%    \begin{macrocode}
\long\def\rc@extract@page#1#2#3#4\@nil{%
  \ltx@zero
  #3%
}
%    \end{macrocode}
%    \end{macro}
%    \begin{macro}{\rc@extract@name}
%    \begin{macrocode}
\long\def\rc@extract@name#1#2#3#4#5\@nil{%
  \ltx@zero
  #4%
}
%    \end{macrocode}
%    \end{macro}
%    \begin{macro}{\rc@extract@anchor}
%    \begin{macrocode}
\long\def\rc@extract@anchor#1#2#3#4#5#6\@nil{%
  \ltx@zero
  #5%
}
%    \end{macrocode}
%    \end{macro}
%    \begin{macro}{\rc@extract@url}
%    \begin{macrocode}
\long\def\rc@extract@url#1#2#3#4#5#6#7\@nil{%
  \ltx@zero
  #6%
}
%    \end{macrocode}
%    \end{macro}
%    \begin{macro}{\rc@extract@title}
%    \begin{macrocode}
\long\def\rc@extract@title#1#2\@nil{%
  \rc@@extract@title#1%
}
%    \end{macrocode}
%    \end{macro}
%    \begin{macro}{\rc@@extract@title}
%    \begin{macrocode}
\long\def\rc@@extract@title#1\TR@TitleReference#2#3#4\@nil{%
  \ltx@zero
  #3%
}
%    \end{macrocode}
%    \end{macro}
%
% \subsection{Macros for checking undefined references}
%
%    \begin{macro}{\IfRefUndefinedExpandable}
%    \begin{macrocode}
\rc@newcommand*{\IfRefUndefinedExpandable}[1]{%
  \ltx@ifundefined{r@#1}\ltx@firstoftwo\ltx@secondoftwo
}
%    \end{macrocode}
%    \end{macro}
%    \begin{macro}{\IfRefUndefinedBabel}
%    \begin{macrocode}
\rc@RobustDefOne{}*\IfRefUndefinedBabel{%
  \begingroup
    \csname safe@actives@true\endcsname
  \expandafter\expandafter\expandafter\endgroup
  \expandafter\ifx\csname r@#1\endcsname\relax
    \expandafter\ltx@firstoftwo
  \else
    \expandafter\ltx@secondoftwo
  \fi
}
%    \end{macrocode}
%    \end{macro}
%    \begin{macrocode}
\rc@AtEnd%
%</package>
%    \end{macrocode}
%
% \section{Test}
%
% \subsection{Catcode checks for loading}
%
%    \begin{macrocode}
%<*test1>
%    \end{macrocode}
%    \begin{macrocode}
\catcode`\{=1 %
\catcode`\}=2 %
\catcode`\#=6 %
\catcode`\@=11 %
\expandafter\ifx\csname count@\endcsname\relax
  \countdef\count@=255 %
\fi
\expandafter\ifx\csname @gobble\endcsname\relax
  \long\def\@gobble#1{}%
\fi
\expandafter\ifx\csname @firstofone\endcsname\relax
  \long\def\@firstofone#1{#1}%
\fi
\expandafter\ifx\csname loop\endcsname\relax
  \expandafter\@firstofone
\else
  \expandafter\@gobble
\fi
{%
  \def\loop#1\repeat{%
    \def\body{#1}%
    \iterate
  }%
  \def\iterate{%
    \body
      \let\next\iterate
    \else
      \let\next\relax
    \fi
    \next
  }%
  \let\repeat=\fi
}%
\def\RestoreCatcodes{}
\count@=0 %
\loop
  \edef\RestoreCatcodes{%
    \RestoreCatcodes
    \catcode\the\count@=\the\catcode\count@\relax
  }%
\ifnum\count@<255 %
  \advance\count@ 1 %
\repeat

\def\RangeCatcodeInvalid#1#2{%
  \count@=#1\relax
  \loop
    \catcode\count@=15 %
  \ifnum\count@<#2\relax
    \advance\count@ 1 %
  \repeat
}
\def\RangeCatcodeCheck#1#2#3{%
  \count@=#1\relax
  \loop
    \ifnum#3=\catcode\count@
    \else
      \errmessage{%
        Character \the\count@\space
        with wrong catcode \the\catcode\count@\space
        instead of \number#3%
      }%
    \fi
  \ifnum\count@<#2\relax
    \advance\count@ 1 %
  \repeat
}
\def\space{ }
\expandafter\ifx\csname LoadCommand\endcsname\relax
  \def\LoadCommand{\input refcount.sty\relax}%
\fi
\def\Test{%
  \RangeCatcodeInvalid{0}{47}%
  \RangeCatcodeInvalid{58}{64}%
  \RangeCatcodeInvalid{91}{96}%
  \RangeCatcodeInvalid{123}{255}%
  \catcode`\@=12 %
  \catcode`\\=0 %
  \catcode`\%=14 %
  \LoadCommand
  \RangeCatcodeCheck{0}{36}{15}%
  \RangeCatcodeCheck{37}{37}{14}%
  \RangeCatcodeCheck{38}{47}{15}%
  \RangeCatcodeCheck{48}{57}{12}%
  \RangeCatcodeCheck{58}{63}{15}%
  \RangeCatcodeCheck{64}{64}{12}%
  \RangeCatcodeCheck{65}{90}{11}%
  \RangeCatcodeCheck{91}{91}{15}%
  \RangeCatcodeCheck{92}{92}{0}%
  \RangeCatcodeCheck{93}{96}{15}%
  \RangeCatcodeCheck{97}{122}{11}%
  \RangeCatcodeCheck{123}{255}{15}%
  \RestoreCatcodes
}
\Test
\csname @@end\endcsname
\end
%    \end{macrocode}
%    \begin{macrocode}
%</test1>
%    \end{macrocode}
%
% \subsection{Macro tests}
%
%    \begin{macrocode}
%<*test2>
\errorcontextlines=10000 %
\showboxbreadth=10000 %
\showboxdepth=10000 %
\begingroup\expandafter\expandafter\expandafter\endgroup
\expandafter\ifx\csname RequirePackage\endcsname\relax
  \input refcount.sty\relax
\else
  \RequirePackage{refcount}[2011/10/16]%
\fi
\catcode`\@=11 %
\begingroup\expandafter\expandafter\expandafter\endgroup
\expandafter\ifx\csname @onelevel@sanitize\endcsname\relax
  \begingroup\expandafter\expandafter\expandafter\endgroup
  \expandafter\ifx\csname detokenize\endcsname\relax
    \def\strip@prefix#1->{}%
    \def\@onelevel@sanitize#1{%
      \edef#1{%
        \expandafter\strip@prefix\meaning#1%
      }%
    }%
  \else
    \def\@onelevel@sanitize#1{%
      \edef#1{%
        \detokenize\expandafter{#1}%
      }%
    }%
  \fi
\fi
\def\msg#{\immediate\write16}
\def\empty{}
\def\space{ }
%    \end{macrocode}
%    \begin{macrocode}
\def\r@foo{{\empty 1}{\empty 2}}
\long\def\test#1#2{%
  \begingroup
    \setbox0=\hbox{%
      \def\TestTask{#1}%
      \@onelevel@sanitize\TestTask
      \msg{* \TestTask}%
      \expandafter\expandafter\expandafter\def
      \expandafter\expandafter\expandafter\TestResult
      \expandafter\expandafter\expandafter{%
        #1%
      }%
      \def\TestExpected{#2}%
      \ifx\TestResult\TestExpected
        \msg{ \space ok.}%
      \else
        \@onelevel@sanitize\TestResult
        \@onelevel@sanitize\TestExpected
        \msg{ \space Result: \space\space[\TestResult]}%
        \msg{ \space Expected: [\TestExpected]}%
        \errmessage{Test failed!}%
      \fi
    }%
    \ifdim\wd0=0pt %
    \else
      \showbox0 %
    \fi
  \endgroup
}
\test{\getrefnumber{foo}}{\empty 1}
\test{\getpagerefnumber{foo}}{\empty 2}
\test{\getrefbykeydefault{foo}{}{\empty default}}{\empty 1}
\test{\getrefbykeydefault{foo}{page}{\empty default}}{\empty 2}
\test{\getrefbykeydefault{foo}{name}{\empty default}}{\empty default}
\test{\getrefbykeydefault{foo}{anchor}{\empty default}}{\empty default}
\test{\getrefbykeydefault{foo}{url}{\empty default}}{\empty default}
\test{\getrefbykeydefault{foo}{title}{\empty default}}{\empty default}
\msg{}
\def\r@foo{{}{}{}{}{}{}{}{}{}{}}
\def\Test#1#2\\{%
  \test{#1{foo}#2}{}%
}
\def\TestGroup{%
  \Test\getrefnumber\\%
  \Test\getpagerefnumber\\%
  \Test\getrefbykeydefault{}{}\\%
  \Test\getrefbykeydefault{page}{}\\%
  \Test\getrefbykeydefault{anchor}{}\\%
  \Test\getrefbykeydefault{name}{}\\%
  \Test\getrefbykeydefault{url}{}\\%
}
\TestGroup
\Test\getrefbykeydefault{title}{}\\%
\msg{}
\def\r@foo{\par\par\par\par\par\par\par\par}
\long\def\Test#1#2\\{%
  \test{#1{foo}#2}{\par}%
}
\TestGroup
\test{\getrefbykeydefault{title}{}{}}{}
\msg{}
\def\r@foo{{ }{ }{ }{ }{ }}
\def\Test#1#2\\{%
  \test{#1{foo}#2}{ }%
}
\TestGroup
\msg{}
\long\def\TestDefault#1{%
  \begingroup
    \setrefcountdefault{#1}%
    \test{\getrefnumber{foo}}{#1}%
    \test{\getpagerefnumber{foo}}{#1}%
  \endgroup
}
\def\TestDefaultX{%
  \TestDefault{}%
  \TestDefault{\par}%
  \TestDefault{ }%
  \TestDefault{\space}%
}
\let\r@foo\@undefined
\TestDefaultX
\let\r@foo\relax
\TestDefaultX
\def\r@foo{}
\TestDefaultX
%    \end{macrocode}
%    \begin{macrocode}
\msg{}
\long\def\Test#1#2#3#4{%
  \begingroup
    \def\TestTask{#1}%
    \@onelevel@sanitize\TestTask
    \msg{* [\TestTask]}%
    \edef\TestResultA{\IfRefUndefinedExpandable{#1}{#2}{#3}}%
    \IfRefUndefinedBabel{#1}{%
      \def\TestResultB{#2}%
    }{%
      \def\TestResultB{#3}%
    }%
    \def\TestExpected{#4}%
    \ifx\TestResultA\TestExpected
      \msg{ \space ok.}%
    \else
      \begingroup
        \@onelevel@sanitize\TestResultA
        \@onelevel@sanitize\TestExpected
        \msg{ \space Result: \space\space[\TestResultA]}%
        \msg{ \space Expected: [\TestExpected]}%
        \errmessage{Test failed!}%
      \endgroup
    \fi
    \ifx\TestResultB\TestExpected
      \msg{ \space ok.}%
    \else
      \begingroup
        \@onelevel@sanitize\TestResultB
        \@onelevel@sanitize\TestExpected
        \msg{ \space Result: \space\space[\TestResultB]}%
        \msg{ \space Expected: [\TestExpected]}%
        \errmessage{Test failed!}%
      \endgroup
    \fi
  \endgroup
}
\begingroup
  \def\r@foo{{}{}}%
  \let\r@bar\@undefined
  \let\r@xyz\relax
  \Test{foo}{true}{false}{false}%
  \Test{bar}{true}{false}{true}%
  \Test{xyz}{true}{false}{true}%
\endgroup
%    \end{macrocode}
%    \begin{macrocode}
\csname @@end\endcsname\end
%</test2>
%    \end{macrocode}
% \subsection{Test with package \xpackage{titleref}}
%
%    \begin{macrocode}
%<*test3>
\NeedsTeXFormat{LaTeX2e}
\documentclass{article}
\usepackage{refcount}[2011/10/16]
%<test4>\usepackage{nameref}
\usepackage{titleref}
\begin{document}
\section{Hello World}
\label{sec:hello}
\section{\hbox{xy}}
\label{sec:foo}
%
\makeatletter
\@ifundefined{r@sec:hello}{%
  \typeout{==> Compile twice!}%
}{%
  \def\test#1#2{%
    \begingroup
      \def\TestTask{#1}%
      \@onelevel@sanitize\TestTask
      \typeout{* \TestTask}%
      \expandafter\expandafter\expandafter\def
      \expandafter\expandafter\expandafter\TestResult
      \expandafter\expandafter\expandafter{%
        #1%
      }%
      \def\TestExpected{#2}%
      \ifx\TestResult\TestExpected
        \typeout{ \space ok.}%
      \else
        \@onelevel@sanitize\TestResult
        \@onelevel@sanitize\TestExpected
        \typeout{ \space Result: \space\space[\TestResult]}%
        \typeout{ \space Expected: [\TestExpected]}%
        \errmessage{Test failed!}%
      \fi
    \endgroup
  }%
  \test{\getrefbykeydefault{sec:hello}{title}{}}{Hello World}%
  \test{\getrefbykeydefault{sec:foo}{title}{}}{\hbox{xy}}%
  \begingroup
    \def\hbox#1{[#1]}% hash-ok
    \test{\getrefbykeydefault{sec:foo}{title}{}}{\hbox{xy}}%
  \endgroup
}
\makeatother
%    \end{macrocode}
%    \begin{macrocode}
\end{document}
%</test3>
%    \end{macrocode}
%    \begin{macrocode}
%<*test5>
\NeedsTeXFormat{LaTeX2e}
\documentclass{book}
\usepackage{refcount}[2011/10/16]
\usepackage{zref-runs}
\newcounter{test}
\begin{document}
\ifnum\zruns>1 %
  \makeatletter
  \def\Test#1#2#3{%
    \begingroup
      \setcounter{test}{10}%
      \sbox0{%
        #1{test}{#2}%
        \ifnum#3=\value{test}%
        \else
          \PackageError{test}{\string#1{#2} <> #3 (\the\value{test})}%
        \fi
      }%
      \ifdim\wd0=0pt %
      \else
        \PackageError{test}{Non-empty box}\@ehc
      \fi
    \endgroup
  }%
  \makeatother
  \Test\setcounterpageref{ch:two}{1}%
  \Test\setcounterpageref{ch:three}{3}%
  \Test\setcounterpageref{ch:four}{5}%
  \Test\setcounterpageref{ch:five}{7}%
  \Test\setcounterpageref{ch:six}{9}%
  \Test\setcounterpageref{ch:seven}{13}%
  \Test\addtocounterpageref{ch:two}{11}%
  \Test\addtocounterpageref{ch:three}{13}%
  \Test\addtocounterpageref{ch:four}{15}%
  \Test\addtocounterpageref{ch:five}{17}%
  \Test\addtocounterpageref{ch:six}{19}%
  \Test\addtocounterpageref{ch:seven}{23}%
  \Test\setcounterref{ch:two}{1}%
  \Test\setcounterref{ch:three}{2}%
  \Test\setcounterref{ch:four}{11}%
  \Test\addtocounterref{ch:two}{11}%
  \Test\addtocounterref{ch:three}{12}%
  \Test\addtocounterref{ch:four}{21}%
\fi
\frontmatter
\chapter{Chapter one}\label{ch:one}
\cleardoublepage
\mainmatter
\chapter{Chapter two}\label{ch:two}
\cleardoublepage
\chapter{Chapter three}\label{ch:three}
\cleardoublepage
\setcounter{chapter}{10}
\chapter{Chapter four}\label{ch:four}
\cleardoublepage
\appendix
\chapter{Chapter five}\label{ch:five}
\cleardoublepage
\chapter{Chapter six}\label{ch:six}
\cleardoublepage
\null
\cleardoublepage
\chapter{Chapter seven}\label{ch:seven}
\end{document}
%</test5>
%    \end{macrocode}
%
% \section{Installation}
%
% \subsection{Download}
%
% \paragraph{Package.} This package is available on
% CTAN\footnote{\url{ftp://ftp.ctan.org/tex-archive/}}:
% \begin{description}
% \item[\CTAN{macros/latex/contrib/oberdiek/refcount.dtx}] The source file.
% \item[\CTAN{macros/latex/contrib/oberdiek/refcount.pdf}] Documentation.
% \end{description}
%
%
% \paragraph{Bundle.} All the packages of the bundle `oberdiek'
% are also available in a TDS compliant ZIP archive. There
% the packages are already unpacked and the documentation files
% are generated. The files and directories obey the TDS standard.
% \begin{description}
% \item[\CTAN{install/macros/latex/contrib/oberdiek.tds.zip}]
% \end{description}
% \emph{TDS} refers to the standard ``A Directory Structure
% for \TeX\ Files'' (\CTAN{tds/tds.pdf}). Directories
% with \xfile{texmf} in their name are usually organized this way.
%
% \subsection{Bundle installation}
%
% \paragraph{Unpacking.} Unpack the \xfile{oberdiek.tds.zip} in the
% TDS tree (also known as \xfile{texmf} tree) of your choice.
% Example (linux):
% \begin{quote}
%   |unzip oberdiek.tds.zip -d ~/texmf|
% \end{quote}
%
% \paragraph{Script installation.}
% Check the directory \xfile{TDS:scripts/oberdiek/} for
% scripts that need further installation steps.
% Package \xpackage{attachfile2} comes with the Perl script
% \xfile{pdfatfi.pl} that should be installed in such a way
% that it can be called as \texttt{pdfatfi}.
% Example (linux):
% \begin{quote}
%   |chmod +x scripts/oberdiek/pdfatfi.pl|\\
%   |cp scripts/oberdiek/pdfatfi.pl /usr/local/bin/|
% \end{quote}
%
% \subsection{Package installation}
%
% \paragraph{Unpacking.} The \xfile{.dtx} file is a self-extracting
% \docstrip\ archive. The files are extracted by running the
% \xfile{.dtx} through \plainTeX:
% \begin{quote}
%   \verb|tex refcount.dtx|
% \end{quote}
%
% \paragraph{TDS.} Now the different files must be moved into
% the different directories in your installation TDS tree
% (also known as \xfile{texmf} tree):
% \begin{quote}
% \def\t{^^A
% \begin{tabular}{@{}>{\ttfamily}l@{ $\rightarrow$ }>{\ttfamily}l@{}}
%   refcount.sty & tex/latex/oberdiek/refcount.sty\\
%   refcount.pdf & doc/latex/oberdiek/refcount.pdf\\
%   test/refcount-test1.tex & doc/latex/oberdiek/test/refcount-test1.tex\\
%   test/refcount-test2.tex & doc/latex/oberdiek/test/refcount-test2.tex\\
%   test/refcount-test3.tex & doc/latex/oberdiek/test/refcount-test3.tex\\
%   test/refcount-test4.tex & doc/latex/oberdiek/test/refcount-test4.tex\\
%   test/refcount-test5.tex & doc/latex/oberdiek/test/refcount-test5.tex\\
%   refcount.dtx & source/latex/oberdiek/refcount.dtx\\
% \end{tabular}^^A
% }^^A
% \sbox0{\t}^^A
% \ifdim\wd0>\linewidth
%   \begingroup
%     \advance\linewidth by\leftmargin
%     \advance\linewidth by\rightmargin
%   \edef\x{\endgroup
%     \def\noexpand\lw{\the\linewidth}^^A
%   }\x
%   \def\lwbox{^^A
%     \leavevmode
%     \hbox to \linewidth{^^A
%       \kern-\leftmargin\relax
%       \hss
%       \usebox0
%       \hss
%       \kern-\rightmargin\relax
%     }^^A
%   }^^A
%   \ifdim\wd0>\lw
%     \sbox0{\small\t}^^A
%     \ifdim\wd0>\linewidth
%       \ifdim\wd0>\lw
%         \sbox0{\footnotesize\t}^^A
%         \ifdim\wd0>\linewidth
%           \ifdim\wd0>\lw
%             \sbox0{\scriptsize\t}^^A
%             \ifdim\wd0>\linewidth
%               \ifdim\wd0>\lw
%                 \sbox0{\tiny\t}^^A
%                 \ifdim\wd0>\linewidth
%                   \lwbox
%                 \else
%                   \usebox0
%                 \fi
%               \else
%                 \lwbox
%               \fi
%             \else
%               \usebox0
%             \fi
%           \else
%             \lwbox
%           \fi
%         \else
%           \usebox0
%         \fi
%       \else
%         \lwbox
%       \fi
%     \else
%       \usebox0
%     \fi
%   \else
%     \lwbox
%   \fi
% \else
%   \usebox0
% \fi
% \end{quote}
% If you have a \xfile{docstrip.cfg} that configures and enables \docstrip's
% TDS installing feature, then some files can already be in the right
% place, see the documentation of \docstrip.
%
% \subsection{Refresh file name databases}
%
% If your \TeX~distribution
% (\teTeX, \mikTeX, \dots) relies on file name databases, you must refresh
% these. For example, \teTeX\ users run \verb|texhash| or
% \verb|mktexlsr|.
%
% \subsection{Some details for the interested}
%
% \paragraph{Attached source.}
%
% The PDF documentation on CTAN also includes the
% \xfile{.dtx} source file. It can be extracted by
% AcrobatReader 6 or higher. Another option is \textsf{pdftk},
% e.g. unpack the file into the current directory:
% \begin{quote}
%   \verb|pdftk refcount.pdf unpack_files output .|
% \end{quote}
%
% \paragraph{Unpacking with \LaTeX.}
% The \xfile{.dtx} chooses its action depending on the format:
% \begin{description}
% \item[\plainTeX:] Run \docstrip\ and extract the files.
% \item[\LaTeX:] Generate the documentation.
% \end{description}
% If you insist on using \LaTeX\ for \docstrip\ (really,
% \docstrip\ does not need \LaTeX), then inform the autodetect routine
% about your intention:
% \begin{quote}
%   \verb|latex \let\install=y% \iffalse meta-comment
%
% File: refcount.dtx
% Version: 2011/10/16 v3.4
% Info: Data extraction from label references
%
% Copyright (C) 1998, 2000, 2006, 2008, 2010, 2011 by
%    Heiko Oberdiek <heiko.oberdiek at googlemail.com>
%
% This work may be distributed and/or modified under the
% conditions of the LaTeX Project Public License, either
% version 1.3c of this license or (at your option) any later
% version. This version of this license is in
%    http://www.latex-project.org/lppl/lppl-1-3c.txt
% and the latest version of this license is in
%    http://www.latex-project.org/lppl.txt
% and version 1.3 or later is part of all distributions of
% LaTeX version 2005/12/01 or later.
%
% This work has the LPPL maintenance status "maintained".
%
% This Current Maintainer of this work is Heiko Oberdiek.
%
% This work consists of the main source file refcount.dtx
% and the derived files
%    refcount.sty, refcount.pdf, refcount.ins, refcount.drv,
%    refcount-test1.tex, refcount-test2.tex, refcount-test3.tex,
%    refcount-test4.tex, refcount-test5.tex.
%
% Distribution:
%    CTAN:macros/latex/contrib/oberdiek/refcount.dtx
%    CTAN:macros/latex/contrib/oberdiek/refcount.pdf
%
% Unpacking:
%    (a) If refcount.ins is present:
%           tex refcount.ins
%    (b) Without refcount.ins:
%           tex refcount.dtx
%    (c) If you insist on using LaTeX
%           latex \let\install=y% \iffalse meta-comment
%
% File: refcount.dtx
% Version: 2011/10/16 v3.4
% Info: Data extraction from label references
%
% Copyright (C) 1998, 2000, 2006, 2008, 2010, 2011 by
%    Heiko Oberdiek <heiko.oberdiek at googlemail.com>
%
% This work may be distributed and/or modified under the
% conditions of the LaTeX Project Public License, either
% version 1.3c of this license or (at your option) any later
% version. This version of this license is in
%    http://www.latex-project.org/lppl/lppl-1-3c.txt
% and the latest version of this license is in
%    http://www.latex-project.org/lppl.txt
% and version 1.3 or later is part of all distributions of
% LaTeX version 2005/12/01 or later.
%
% This work has the LPPL maintenance status "maintained".
%
% This Current Maintainer of this work is Heiko Oberdiek.
%
% This work consists of the main source file refcount.dtx
% and the derived files
%    refcount.sty, refcount.pdf, refcount.ins, refcount.drv,
%    refcount-test1.tex, refcount-test2.tex, refcount-test3.tex,
%    refcount-test4.tex, refcount-test5.tex.
%
% Distribution:
%    CTAN:macros/latex/contrib/oberdiek/refcount.dtx
%    CTAN:macros/latex/contrib/oberdiek/refcount.pdf
%
% Unpacking:
%    (a) If refcount.ins is present:
%           tex refcount.ins
%    (b) Without refcount.ins:
%           tex refcount.dtx
%    (c) If you insist on using LaTeX
%           latex \let\install=y\input{refcount.dtx}
%        (quote the arguments according to the demands of your shell)
%
% Documentation:
%    (a) If refcount.drv is present:
%           latex refcount.drv
%    (b) Without refcount.drv:
%           latex refcount.dtx; ...
%    The class ltxdoc loads the configuration file ltxdoc.cfg
%    if available. Here you can specify further options, e.g.
%    use A4 as paper format:
%       \PassOptionsToClass{a4paper}{article}
%
%    Programm calls to get the documentation (example):
%       pdflatex refcount.dtx
%       makeindex -s gind.ist refcount.idx
%       pdflatex refcount.dtx
%       makeindex -s gind.ist refcount.idx
%       pdflatex refcount.dtx
%
% Installation:
%    TDS:tex/latex/oberdiek/refcount.sty
%    TDS:doc/latex/oberdiek/refcount.pdf
%    TDS:doc/latex/oberdiek/test/refcount-test1.tex
%    TDS:doc/latex/oberdiek/test/refcount-test2.tex
%    TDS:doc/latex/oberdiek/test/refcount-test3.tex
%    TDS:doc/latex/oberdiek/test/refcount-test4.tex
%    TDS:doc/latex/oberdiek/test/refcount-test5.tex
%    TDS:source/latex/oberdiek/refcount.dtx
%
%<*ignore>
\begingroup
  \catcode123=1 %
  \catcode125=2 %
  \def\x{LaTeX2e}%
\expandafter\endgroup
\ifcase 0\ifx\install y1\fi\expandafter
         \ifx\csname processbatchFile\endcsname\relax\else1\fi
         \ifx\fmtname\x\else 1\fi\relax
\else\csname fi\endcsname
%</ignore>
%<*install>
\input docstrip.tex
\Msg{************************************************************************}
\Msg{* Installation}
\Msg{* Package: refcount 2011/10/16 v3.4 Data extraction from label references (HO)}
\Msg{************************************************************************}

\keepsilent
\askforoverwritefalse

\let\MetaPrefix\relax
\preamble

This is a generated file.

Project: refcount
Version: 2011/10/16 v3.4

Copyright (C) 1998, 2000, 2006, 2008, 2010, 2011 by
   Heiko Oberdiek <heiko.oberdiek at googlemail.com>

This work may be distributed and/or modified under the
conditions of the LaTeX Project Public License, either
version 1.3c of this license or (at your option) any later
version. This version of this license is in
   http://www.latex-project.org/lppl/lppl-1-3c.txt
and the latest version of this license is in
   http://www.latex-project.org/lppl.txt
and version 1.3 or later is part of all distributions of
LaTeX version 2005/12/01 or later.

This work has the LPPL maintenance status "maintained".

This Current Maintainer of this work is Heiko Oberdiek.

This work consists of the main source file refcount.dtx
and the derived files
   refcount.sty, refcount.pdf, refcount.ins, refcount.drv,
   refcount-test1.tex, refcount-test2.tex, refcount-test3.tex,
   refcount-test4.tex, refcount-test5.tex.

\endpreamble
\let\MetaPrefix\DoubleperCent

\generate{%
  \file{refcount.ins}{\from{refcount.dtx}{install}}%
  \file{refcount.drv}{\from{refcount.dtx}{driver}}%
  \usedir{tex/latex/oberdiek}%
  \file{refcount.sty}{\from{refcount.dtx}{package}}%
  \usedir{doc/latex/oberdiek/test}%
  \file{refcount-test1.tex}{\from{refcount.dtx}{test1}}%
  \file{refcount-test2.tex}{\from{refcount.dtx}{test2}}%
  \file{refcount-test3.tex}{\from{refcount.dtx}{test3}}%
  \file{refcount-test4.tex}{\from{refcount.dtx}{test3,test4}}%
  \file{refcount-test5.tex}{\from{refcount.dtx}{test5}}%
  \nopreamble
  \nopostamble
  \usedir{source/latex/oberdiek/catalogue}%
  \file{refcount.xml}{\from{refcount.dtx}{catalogue}}%
}

\catcode32=13\relax% active space
\let =\space%
\Msg{************************************************************************}
\Msg{*}
\Msg{* To finish the installation you have to move the following}
\Msg{* file into a directory searched by TeX:}
\Msg{*}
\Msg{*     refcount.sty}
\Msg{*}
\Msg{* To produce the documentation run the file `refcount.drv'}
\Msg{* through LaTeX.}
\Msg{*}
\Msg{* Happy TeXing!}
\Msg{*}
\Msg{************************************************************************}

\endbatchfile
%</install>
%<*ignore>
\fi
%</ignore>
%<*driver>
\NeedsTeXFormat{LaTeX2e}
\ProvidesFile{refcount.drv}%
  [2011/10/16 v3.4 Data extraction from label references (HO)]%
\documentclass{ltxdoc}
\usepackage{holtxdoc}[2011/11/22]
\begin{document}
  \DocInput{refcount.dtx}%
\end{document}
%</driver>
% \fi
%
% \CheckSum{1185}
%
% \CharacterTable
%  {Upper-case    \A\B\C\D\E\F\G\H\I\J\K\L\M\N\O\P\Q\R\S\T\U\V\W\X\Y\Z
%   Lower-case    \a\b\c\d\e\f\g\h\i\j\k\l\m\n\o\p\q\r\s\t\u\v\w\x\y\z
%   Digits        \0\1\2\3\4\5\6\7\8\9
%   Exclamation   \!     Double quote  \"     Hash (number) \#
%   Dollar        \$     Percent       \%     Ampersand     \&
%   Acute accent  \'     Left paren    \(     Right paren   \)
%   Asterisk      \*     Plus          \+     Comma         \,
%   Minus         \-     Point         \.     Solidus       \/
%   Colon         \:     Semicolon     \;     Less than     \<
%   Equals        \=     Greater than  \>     Question mark \?
%   Commercial at \@     Left bracket  \[     Backslash     \\
%   Right bracket \]     Circumflex    \^     Underscore    \_
%   Grave accent  \`     Left brace    \{     Vertical bar  \|
%   Right brace   \}     Tilde         \~}
%
% \GetFileInfo{refcount.drv}
%
% \title{The \xpackage{refcount} package}
% \date{2011/10/16 v3.4}
% \author{Heiko Oberdiek\\\xemail{heiko.oberdiek at googlemail.com}}
%
% \maketitle
%
% \begin{abstract}
% References are not numbers, however they often store numerical
% data such as section or page numbers. \cs{ref} or \cs{pageref}
% cannot be used for counter assignments or calculations because
% they are not expandable, generate warnings, or can even be links.
% The package provides expandable macros to extract the data
% from references. Packages \xpackage{hyperref}, \xpackage{nameref},
% \xpackage{titleref}, and \xpackage{babel} are supported.
% \end{abstract}
%
% \tableofcontents
%
% \section{Usage}
%
% \subsection{Setting counters}
%
% The following commands are similar to \LaTeX's
% \cs{setcounter} and \cs{addtocounter},
% but they extract the number value from a reference:
% \begin{quote}
%   \cs{setcounterref}, \cs{addtocounterref}\\
%   \cs{setcounterpageref}, \cs{addtocounterpageref}
% \end{quote}
% They take two arguments:
% \begin{quote}
%    \cs{...counter...ref} |{|\meta{\LaTeX\ counter}|}|
%    |{|\meta{reference}|}|
% \end{quote}
% An undefined references produces the usual LaTeX warning
% and its value is assumed to be zero.
% Example:
% \begin{quote}
%\begin{verbatim}
%\newcounter{ctrA}
%\newcounter{ctrB}
%\refstepcounter{ctrA}\label{ref:A}
%\setcounterref{ctrB}{ref:A}
%\addtocounterpageref{ctrB}{ref:A}
%\end{verbatim}
% \end{quote}
%
% \subsection{Expandable commands}
%
% These commands that can be used in expandible contexts
% (inside calculations, \cs{edef}, \cs{csname}, \cs{write}, \dots):
% \begin{quote}
%   \cs{getrefnumber}, \cs{getpagerefnumber}
% \end{quote}
% They take one argument, the reference:
% \begin{quote}
%   \cs{get...refnumber} |{|\meta{reference}|}|
% \end{quote}
% The default for undefined references can be changed
% with macro \cs{setrefcountdefault}, for example this
% package calls:
% \begin{quote}
%   \cs{setrefcountdefault}|{0}|
% \end{quote}
%
% Since version 2.0 of this package there is a new
% command:
% \begin{quote}
%   \cs{getrefbykeydefault} |{|\meta{reference}|}|
%   |{|\meta{key}|}| |{|\meta{default}|}|
% \end{quote}
% This generalized version allows the extraction
% of further properties of a reference than the
% two standard ones. Thus the following properties
% are supported, if they are available:
% \begin{quote}
% \begin{tabular}{@{}l|l|l@{}}
%    Key & Description & Package\\
% \hline
%   \meta{empty} & same as \cs{ref} & \LaTeX\\
%   |page| & same as \cs{pageref} & \LaTeX\\
%   |title| & section and caption titles & \xpackage{titleref}\\
%   |name| & section and caption titles & \xpackage{nameref}\\
%   |anchor| & anchor name & \xpackage{hyperref}\\
%   |url| & url/file & \xpackage{hyperref}/\xpackage{xr}
% \end{tabular}
% \end{quote}
%
% Since version 3.2 the expandable macros described before
% in this section
% are expandable in exact two expansion steps.
%
% \subsection{Undefined references}
%
% Because warnings and assignments cannot be used in
% expandible contexts, undefined references do not
% produce a warning, their values are assumed to be zero.
% Example:
% \begin{quote}
%\begin{verbatim}
%\label{ref:here}% somewhere
%\refused{ref:here}% see below
%\ifodd\getpagerefnumber{ref:here}%
%  reference is on an odd page
%\else
%  reference is on an even page
%\fi
%\end{verbatim}
% \end{quote}
%
% In case of undefined references the user usually want's
% to be informed. Also \LaTeX\ prints a warning at
% the end of the \LaTeX\ run. To notify \LaTeX\ and
% get a normal warning, just use
% \begin{quote}
%   \cs{refused} |{|\meta{reference}|}|
% \end{quote}
% outside the expanding context. Example, see above.
%
% \subsubsection{Check for undefined references}
%
% In version 3.2 macros were added, that test, whether references
% are defined.
% \begin{declcs}{IfRefUndefinedExpandable} \M{refname} \M{then} \M{else}\\
%   \cs{IfRefUndefinedBabel} \M{refname} \M{then} \M{else}
% \end{declcs}
% If the reference is not available and therefore undefined, then
% argument \meta{then} is executed, otherwise argument \meta{else}
% is called. Macro \cs{IfRefUndefinedExpandable} is expandable,
% but \meta{refname} must not contain babel shorthand characters.
% Macro \cs{IfRefUndefinedBabel} supports shorthand characters of
% babel, but it is not expandable.
%
% \subsection{Notes}
%
% \begin{itemize}
% \item
%   The method of extracting the number in this
%   package also works in cases, where the
%   reference cannot be used directly, because
%   a package such as \xpackage{hyperref} has added
%   extra stuff (hyper link), so that the reference cannot
%   be used as number any more.
% \item
%   If the reference does not contain a number,
%   assignments to a counter will fail of course.
% \end{itemize}
%
%
% \StopEventually{
% }
%
% \section{Implementation}
%
%    \begin{macrocode}
%<*package>
%    \end{macrocode}
%    Reload check, especially if the package is not used with \LaTeX.
%    \begin{macrocode}
\begingroup\catcode61\catcode48\catcode32=10\relax%
  \catcode13=5 % ^^M
  \endlinechar=13 %
  \catcode35=6 % #
  \catcode39=12 % '
  \catcode44=12 % ,
  \catcode45=12 % -
  \catcode46=12 % .
  \catcode58=12 % :
  \catcode64=11 % @
  \catcode123=1 % {
  \catcode125=2 % }
  \expandafter\let\expandafter\x\csname ver@refcount.sty\endcsname
  \ifx\x\relax % plain-TeX, first loading
  \else
    \def\empty{}%
    \ifx\x\empty % LaTeX, first loading,
      % variable is initialized, but \ProvidesPackage not yet seen
    \else
      \expandafter\ifx\csname PackageInfo\endcsname\relax
        \def\x#1#2{%
          \immediate\write-1{Package #1 Info: #2.}%
        }%
      \else
        \def\x#1#2{\PackageInfo{#1}{#2, stopped}}%
      \fi
      \x{refcount}{The package is already loaded}%
      \aftergroup\endinput
    \fi
  \fi
\endgroup%
%    \end{macrocode}
%    Package identification:
%    \begin{macrocode}
\begingroup\catcode61\catcode48\catcode32=10\relax%
  \catcode13=5 % ^^M
  \endlinechar=13 %
  \catcode35=6 % #
  \catcode39=12 % '
  \catcode40=12 % (
  \catcode41=12 % )
  \catcode44=12 % ,
  \catcode45=12 % -
  \catcode46=12 % .
  \catcode47=12 % /
  \catcode58=12 % :
  \catcode64=11 % @
  \catcode91=12 % [
  \catcode93=12 % ]
  \catcode123=1 % {
  \catcode125=2 % }
  \expandafter\ifx\csname ProvidesPackage\endcsname\relax
    \def\x#1#2#3[#4]{\endgroup
      \immediate\write-1{Package: #3 #4}%
      \xdef#1{#4}%
    }%
  \else
    \def\x#1#2[#3]{\endgroup
      #2[{#3}]%
      \ifx#1\@undefined
        \xdef#1{#3}%
      \fi
      \ifx#1\relax
        \xdef#1{#3}%
      \fi
    }%
  \fi
\expandafter\x\csname ver@refcount.sty\endcsname
\ProvidesPackage{refcount}%
  [2011/10/16 v3.4 Data extraction from label references (HO)]%
%    \end{macrocode}
%
%    \begin{macrocode}
\begingroup\catcode61\catcode48\catcode32=10\relax%
  \catcode13=5 % ^^M
  \endlinechar=13 %
  \catcode123=1 % {
  \catcode125=2 % }
  \catcode64=11 % @
  \def\x{\endgroup
    \expandafter\edef\csname rc@AtEnd\endcsname{%
      \endlinechar=\the\endlinechar\relax
      \catcode13=\the\catcode13\relax
      \catcode32=\the\catcode32\relax
      \catcode35=\the\catcode35\relax
      \catcode61=\the\catcode61\relax
      \catcode64=\the\catcode64\relax
      \catcode123=\the\catcode123\relax
      \catcode125=\the\catcode125\relax
    }%
  }%
\x\catcode61\catcode48\catcode32=10\relax%
\catcode13=5 % ^^M
\endlinechar=13 %
\catcode35=6 % #
\catcode64=11 % @
\catcode123=1 % {
\catcode125=2 % }
\def\TMP@EnsureCode#1#2{%
  \edef\rc@AtEnd{%
    \rc@AtEnd
    \catcode#1=\the\catcode#1\relax
  }%
  \catcode#1=#2\relax
}
\TMP@EnsureCode{33}{12}% !
\TMP@EnsureCode{39}{12}% '
\TMP@EnsureCode{42}{12}% *
\TMP@EnsureCode{45}{12}% -
\TMP@EnsureCode{46}{12}% .
\TMP@EnsureCode{47}{12}% /
\TMP@EnsureCode{91}{12}% [
\TMP@EnsureCode{93}{12}% ]
\TMP@EnsureCode{96}{12}% `
\edef\rc@AtEnd{\rc@AtEnd\noexpand\endinput}
%    \end{macrocode}
%
% \subsection{Loading packages}
%
%    \begin{macrocode}
\begingroup\expandafter\expandafter\expandafter\endgroup
\expandafter\ifx\csname RequirePackage\endcsname\relax
  \input ltxcmds.sty\relax
  \input infwarerr.sty\relax
\else
  \RequirePackage{ltxcmds}[2011/11/09]%
  \RequirePackage{infwarerr}[2010/04/08]%
\fi
%    \end{macrocode}
%
% \subsection{Defining commands}
%
%    \begin{macro}{\rc@IfDefinable}
%    \begin{macrocode}
\ltx@IfUndefined{@ifdefinable}{%
  \def\rc@IfDefinable#1{%
    \ifx#1\ltx@undefined
      \expandafter\ltx@firstofone
    \else
      \ifx#1\relax
        \expandafter\expandafter\expandafter\ltx@firstofone
      \else
        \@PackageError{refcount}{%
          Command \string#1 is already defined.\MessageBreak
          It will not redefined by this package%
        }\@ehc
        \expandafter\expandafter\expandafter\ltx@gobble
      \fi
    \fi
  }%
}{%
  \let\rc@IfDefinable\@ifdefinable
}
%    \end{macrocode}
%    \end{macro}
%
%    \begin{macro}{\rc@RobustDefOne}
%    \begin{macro}{\rc@RobustDefZero}
%    \begin{macrocode}
\ltx@IfUndefined{protected}{%
  \ltx@IfUndefined{DeclareRobustCommand}{%
    \def\rc@RobustDefOne#1#2#3#4{%
      \rc@IfDefinable#3{%
        #1\def#3##1{#4}%
      }%
    }%
    \def\rc@RobustDefZero#1#2{%
      \rc@IfDefinable#1{%
        \def#1{#2}%
      }%
    }%
  }{%
    \def\rc@RobustDefOne#1#2#3#4{%
      \rc@IfDefinable#3{%
        \DeclareRobustCommand#2#3[1]{#4}%
      }%
    }%
    \def\rc@RobustDefZero#1#2{%
      \rc@IfDefinable#1{%
        \DeclareRobustCommand#1{#2}%
      }%
    }%
  }%
}{%
  \def\rc@RobustDefOne#1#2#3#4{%
    \rc@IfDefinable#3{%
      \protected#1\def#3##1{#4}%
    }%
  }%
  \def\rc@RobustDefZero#1#2{%
    \rc@IfDefinable#1{%
      \protected\def#1{#2}%
    }%
  }%
}
%    \end{macrocode}
%    \end{macro}
%    \end{macro}
%
%    \begin{macro}{\rc@newcommand}
%    \begin{macrocode}
\ltx@IfUndefined{newcommand}{%
  \def\rc@newcommand*#1[#2]#3{% hash-ok
    \rc@IfDefinable#1{%
      \ifcase#2 %
        \def#1{#3}%
      \or
        \def#1##1{#3}%
      \or
        \def#1##1##2{#3}%
      \else
        \rc@InternalError
      \fi
    }%
  }%
}{%
  \let\rc@newcommand\newcommand
}
%    \end{macrocode}
%    \end{macro}
%
% \subsection{\cs{setrefcountdefault}}
%
%    \begin{macro}{\setrefcountdefault}
%    \begin{macrocode}
\rc@RobustDefOne\long{}\setrefcountdefault{%
  \def\rc@default{#1}%
}
%    \end{macrocode}
%    \end{macro}
%    \begin{macrocode}
\setrefcountdefault{0}
%    \end{macrocode}
%
% \subsection{\cs{refused}}
%
%    \begin{macro}{\refused}
%    \begin{macrocode}
\ltx@IfUndefined{G@refundefinedtrue}{%
  \rc@RobustDefOne{}{*}\refused{%
    \begingroup
      \csname @safe@activestrue\endcsname
      \ltx@IfUndefined{r@#1}{%
        \protect\G@refundefinedtrue
        \rc@WarningUndefined{#1}%
      }{}%
    \endgroup
  }%
}{%
  \rc@RobustDefOne{}{*}\refused{%
    \begingroup
      \csname @safe@activestrue\endcsname
      \ltx@IfUndefined{r@#1}{%
        \csname protect\expandafter\endcsname
        \csname G@refundefinedtrue\endcsname
        \rc@WarningUndefined{#1}%
      }{}%
    \endgroup
  }%
}
%    \end{macrocode}
%    \end{macro}
%    \begin{macro}{\rc@WarningUndefined}
%    \begin{macrocode}
\ltx@IfUndefined{@latex@warning}{%
  \def\rc@WarningUndefined#1{%
    \ltx@ifundefined{thepage}{%
      \def\thepage{\number\count0 }%
    }{}%
    \@PackageWarning{refcount}{%
      Reference `#1' on page \thepage\space undefined%
    }%
  }%
}{%
  \def\rc@WarningUndefined#1{%
    \@latex@warning{%
      Reference `#1' on page \thepage\space undefined%
    }%
  }%
}
%    \end{macrocode}
%    \end{macro}
%
% \subsection{Setting counters by reference data}
%
% \subsubsection{Generic setting}
%
%    \begin{macro}{\rc@set}
% Generic command for
% |\|$\{$|set|$,$|addto|$\}$|counter|$\{$|page|$,\}$|ref|:
%\begin{quote}
%\begin{tabular}[t]{@{}l@{: }l@{}}
% |#1|& \cs{setcounter}, \cs{addtocounter}\\
% |#2|& \cs{ltx@car} (for \cs{ref}), \cs{ltx@cartwo} (for \cs{pageref})\\
% |#3|& \hologo{LaTeX} counter\\
% |#4|& reference\\
%\end{tabular}
%\end{quote}
%    \begin{macrocode}
\def\rc@set#1#2#3#4{%
  \begingroup
    \csname @safe@activestrue\endcsname
    \refused{#4}%
    \expandafter\rc@@set\csname r@#4\endcsname{#1}{#2}{#3}%
  \endgroup
}
%    \end{macrocode}
%    \end{macro}
%    \begin{macro}{\rc@@set}
%\begin{quote}
%\begin{tabular}[t]{@{}l@{: }l@{}}
% |#1|& \cs{r@<...>}\\
% |#2|& \cs{setcounter}, \cs{addtocounter}\\
% |#3|& \cs{ltx@car} (for \cs{ref}), \cs{ltx@carsecond} (for \cs{pageref})\\
% |#4|& \hologo{LaTeX} counter\\
%\end{tabular}
%\end{quote}
%    \begin{macrocode}
\def\rc@@set#1#2#3#4{%
  \ifx#1\relax
    #2{#4}{\rc@default}%
  \else
    #2{#4}{%
      \expandafter#3#1\rc@default\rc@default\@nil
    }%
  \fi
}
%    \end{macrocode}
%    \end{macro}
%
% \subsubsection{User commands}
%
%    \begin{macro}{\setcounterref}
%    \begin{macrocode}
\rc@RobustDefZero\setcounterref{%
  \rc@set\setcounter\ltx@car
}
%    \end{macrocode}
%    \end{macro}
%    \begin{macro}{\addtocounterref}
%    \begin{macrocode}
\rc@RobustDefZero\addtocounterref{%
  \rc@set\addtocounter\ltx@car
}
%    \end{macrocode}
%    \end{macro}
%    \begin{macro}{\setcounterpageref}
%    \begin{macrocode}
\rc@RobustDefZero\setcounterpageref{%
  \rc@set\setcounter\ltx@carsecond
}
%    \end{macrocode}
%    \end{macro}
%    \begin{macro}{\addtocounterpageref}
%    \begin{macrocode}
\rc@RobustDefZero\addtocounterpageref{%
  \rc@set\addtocounter\ltx@carsecond
}
%    \end{macrocode}
%    \end{macro}
%
% \subsection{Extracting references}
%
%    \begin{macro}{\getrefnumber}
%    \begin{macrocode}
\rc@newcommand*{\getrefnumber}[1]{%
  \romannumeral
  \ltx@ifundefined{r@#1}{%
    \expandafter\ltx@zero
    \rc@default
  }{%
    \expandafter\expandafter\expandafter\rc@extract@
    \expandafter\expandafter\expandafter!%
    \csname r@#1\expandafter\endcsname
    \expandafter{\rc@default}\@nil
  }%
}
%    \end{macrocode}
%    \end{macro}
%    \begin{macro}{\getpagerefnumber}
%    \begin{macrocode}
\rc@newcommand*{\getpagerefnumber}[1]{%
  \romannumeral
  \ltx@ifundefined{r@#1}{%
    \expandafter\ltx@zero
    \rc@default
  }{%
    \expandafter\expandafter\expandafter\rc@extract@page
    \expandafter\expandafter\expandafter!%
    \csname r@#1\expandafter\expandafter\expandafter\endcsname
    \expandafter\expandafter\expandafter{%
      \expandafter\rc@default
    \expandafter}\expandafter{\rc@default}\@nil
  }%
}
%    \end{macrocode}
%    \end{macro}
%    \begin{macro}{\getrefbykeydefault}
%    \begin{macrocode}
\rc@newcommand*{\getrefbykeydefault}[2]{%
  \romannumeral
  \expandafter\rc@getrefbykeydefault
    \csname r@#1\expandafter\endcsname
    \csname rc@extract@#2\endcsname
}
%    \end{macrocode}
%    \end{macro}
%    \begin{macro}{\rc@getrefbykeydefault}
%\begin{quote}
%\begin{tabular}[t]{@{}l@{: }l@{}}
% |#1|& \cs{r@<...>}\\
% |#2|& \cs{rc@extract@<...>}\\
% |#3|& default\\
%\end{tabular}
%\end{quote}
%    \begin{macrocode}
\long\def\rc@getrefbykeydefault#1#2#3{%
  \ifx#1\relax
    % reference is undefined
    \ltx@ReturnAfterElseFi{%
      \ltx@zero
      #3%
    }%
  \else
    \ltx@ReturnAfterFi{%
      \ifx#2\relax
        % extract method is missing
        \ltx@ReturnAfterElseFi{%
          \ltx@zero
          #3%
        }%
      \else
        \ltx@ReturnAfterFi{%
          \expandafter
          \rc@generic#1{#3}{#3}{#3}{#3}{#3}\@nil#2{#3}%
        }%
      \fi
    }%
  \fi
}
%    \end{macrocode}
%    \end{macro}
%    \begin{macro}{\rc@generic}
%\begin{quote}
%\begin{tabular}[t]{@{}l@{: }l@{}}
% |#1|& first item in \cs{r@<...>}\\
% |#2|& remaining items in \cs{r@<...>}\\
% |#3|& \cs{rc@extract@<...>}\\
% |#4|& default\\
%\end{tabular}
%\end{quote}
%    \begin{macrocode}
\long\def\rc@generic#1#2\@nil#3#4{%
  #3{#1\TR@TitleReference\@empty{#4}\@nil}{#1}#2\@nil
}
%    \end{macrocode}
%    \end{macro}
%    \begin{macro}{\rc@extract@}
%    \begin{macrocode}
\long\def\rc@extract@#1#2#3\@nil{%
  \ltx@zero
  #2%
}
%    \end{macrocode}
%    \end{macro}
%    \begin{macro}{\rc@extract@page}
%    \begin{macrocode}
\long\def\rc@extract@page#1#2#3#4\@nil{%
  \ltx@zero
  #3%
}
%    \end{macrocode}
%    \end{macro}
%    \begin{macro}{\rc@extract@name}
%    \begin{macrocode}
\long\def\rc@extract@name#1#2#3#4#5\@nil{%
  \ltx@zero
  #4%
}
%    \end{macrocode}
%    \end{macro}
%    \begin{macro}{\rc@extract@anchor}
%    \begin{macrocode}
\long\def\rc@extract@anchor#1#2#3#4#5#6\@nil{%
  \ltx@zero
  #5%
}
%    \end{macrocode}
%    \end{macro}
%    \begin{macro}{\rc@extract@url}
%    \begin{macrocode}
\long\def\rc@extract@url#1#2#3#4#5#6#7\@nil{%
  \ltx@zero
  #6%
}
%    \end{macrocode}
%    \end{macro}
%    \begin{macro}{\rc@extract@title}
%    \begin{macrocode}
\long\def\rc@extract@title#1#2\@nil{%
  \rc@@extract@title#1%
}
%    \end{macrocode}
%    \end{macro}
%    \begin{macro}{\rc@@extract@title}
%    \begin{macrocode}
\long\def\rc@@extract@title#1\TR@TitleReference#2#3#4\@nil{%
  \ltx@zero
  #3%
}
%    \end{macrocode}
%    \end{macro}
%
% \subsection{Macros for checking undefined references}
%
%    \begin{macro}{\IfRefUndefinedExpandable}
%    \begin{macrocode}
\rc@newcommand*{\IfRefUndefinedExpandable}[1]{%
  \ltx@ifundefined{r@#1}\ltx@firstoftwo\ltx@secondoftwo
}
%    \end{macrocode}
%    \end{macro}
%    \begin{macro}{\IfRefUndefinedBabel}
%    \begin{macrocode}
\rc@RobustDefOne{}*\IfRefUndefinedBabel{%
  \begingroup
    \csname safe@actives@true\endcsname
  \expandafter\expandafter\expandafter\endgroup
  \expandafter\ifx\csname r@#1\endcsname\relax
    \expandafter\ltx@firstoftwo
  \else
    \expandafter\ltx@secondoftwo
  \fi
}
%    \end{macrocode}
%    \end{macro}
%    \begin{macrocode}
\rc@AtEnd%
%</package>
%    \end{macrocode}
%
% \section{Test}
%
% \subsection{Catcode checks for loading}
%
%    \begin{macrocode}
%<*test1>
%    \end{macrocode}
%    \begin{macrocode}
\catcode`\{=1 %
\catcode`\}=2 %
\catcode`\#=6 %
\catcode`\@=11 %
\expandafter\ifx\csname count@\endcsname\relax
  \countdef\count@=255 %
\fi
\expandafter\ifx\csname @gobble\endcsname\relax
  \long\def\@gobble#1{}%
\fi
\expandafter\ifx\csname @firstofone\endcsname\relax
  \long\def\@firstofone#1{#1}%
\fi
\expandafter\ifx\csname loop\endcsname\relax
  \expandafter\@firstofone
\else
  \expandafter\@gobble
\fi
{%
  \def\loop#1\repeat{%
    \def\body{#1}%
    \iterate
  }%
  \def\iterate{%
    \body
      \let\next\iterate
    \else
      \let\next\relax
    \fi
    \next
  }%
  \let\repeat=\fi
}%
\def\RestoreCatcodes{}
\count@=0 %
\loop
  \edef\RestoreCatcodes{%
    \RestoreCatcodes
    \catcode\the\count@=\the\catcode\count@\relax
  }%
\ifnum\count@<255 %
  \advance\count@ 1 %
\repeat

\def\RangeCatcodeInvalid#1#2{%
  \count@=#1\relax
  \loop
    \catcode\count@=15 %
  \ifnum\count@<#2\relax
    \advance\count@ 1 %
  \repeat
}
\def\RangeCatcodeCheck#1#2#3{%
  \count@=#1\relax
  \loop
    \ifnum#3=\catcode\count@
    \else
      \errmessage{%
        Character \the\count@\space
        with wrong catcode \the\catcode\count@\space
        instead of \number#3%
      }%
    \fi
  \ifnum\count@<#2\relax
    \advance\count@ 1 %
  \repeat
}
\def\space{ }
\expandafter\ifx\csname LoadCommand\endcsname\relax
  \def\LoadCommand{\input refcount.sty\relax}%
\fi
\def\Test{%
  \RangeCatcodeInvalid{0}{47}%
  \RangeCatcodeInvalid{58}{64}%
  \RangeCatcodeInvalid{91}{96}%
  \RangeCatcodeInvalid{123}{255}%
  \catcode`\@=12 %
  \catcode`\\=0 %
  \catcode`\%=14 %
  \LoadCommand
  \RangeCatcodeCheck{0}{36}{15}%
  \RangeCatcodeCheck{37}{37}{14}%
  \RangeCatcodeCheck{38}{47}{15}%
  \RangeCatcodeCheck{48}{57}{12}%
  \RangeCatcodeCheck{58}{63}{15}%
  \RangeCatcodeCheck{64}{64}{12}%
  \RangeCatcodeCheck{65}{90}{11}%
  \RangeCatcodeCheck{91}{91}{15}%
  \RangeCatcodeCheck{92}{92}{0}%
  \RangeCatcodeCheck{93}{96}{15}%
  \RangeCatcodeCheck{97}{122}{11}%
  \RangeCatcodeCheck{123}{255}{15}%
  \RestoreCatcodes
}
\Test
\csname @@end\endcsname
\end
%    \end{macrocode}
%    \begin{macrocode}
%</test1>
%    \end{macrocode}
%
% \subsection{Macro tests}
%
%    \begin{macrocode}
%<*test2>
\errorcontextlines=10000 %
\showboxbreadth=10000 %
\showboxdepth=10000 %
\begingroup\expandafter\expandafter\expandafter\endgroup
\expandafter\ifx\csname RequirePackage\endcsname\relax
  \input refcount.sty\relax
\else
  \RequirePackage{refcount}[2011/10/16]%
\fi
\catcode`\@=11 %
\begingroup\expandafter\expandafter\expandafter\endgroup
\expandafter\ifx\csname @onelevel@sanitize\endcsname\relax
  \begingroup\expandafter\expandafter\expandafter\endgroup
  \expandafter\ifx\csname detokenize\endcsname\relax
    \def\strip@prefix#1->{}%
    \def\@onelevel@sanitize#1{%
      \edef#1{%
        \expandafter\strip@prefix\meaning#1%
      }%
    }%
  \else
    \def\@onelevel@sanitize#1{%
      \edef#1{%
        \detokenize\expandafter{#1}%
      }%
    }%
  \fi
\fi
\def\msg#{\immediate\write16}
\def\empty{}
\def\space{ }
%    \end{macrocode}
%    \begin{macrocode}
\def\r@foo{{\empty 1}{\empty 2}}
\long\def\test#1#2{%
  \begingroup
    \setbox0=\hbox{%
      \def\TestTask{#1}%
      \@onelevel@sanitize\TestTask
      \msg{* \TestTask}%
      \expandafter\expandafter\expandafter\def
      \expandafter\expandafter\expandafter\TestResult
      \expandafter\expandafter\expandafter{%
        #1%
      }%
      \def\TestExpected{#2}%
      \ifx\TestResult\TestExpected
        \msg{ \space ok.}%
      \else
        \@onelevel@sanitize\TestResult
        \@onelevel@sanitize\TestExpected
        \msg{ \space Result: \space\space[\TestResult]}%
        \msg{ \space Expected: [\TestExpected]}%
        \errmessage{Test failed!}%
      \fi
    }%
    \ifdim\wd0=0pt %
    \else
      \showbox0 %
    \fi
  \endgroup
}
\test{\getrefnumber{foo}}{\empty 1}
\test{\getpagerefnumber{foo}}{\empty 2}
\test{\getrefbykeydefault{foo}{}{\empty default}}{\empty 1}
\test{\getrefbykeydefault{foo}{page}{\empty default}}{\empty 2}
\test{\getrefbykeydefault{foo}{name}{\empty default}}{\empty default}
\test{\getrefbykeydefault{foo}{anchor}{\empty default}}{\empty default}
\test{\getrefbykeydefault{foo}{url}{\empty default}}{\empty default}
\test{\getrefbykeydefault{foo}{title}{\empty default}}{\empty default}
\msg{}
\def\r@foo{{}{}{}{}{}{}{}{}{}{}}
\def\Test#1#2\\{%
  \test{#1{foo}#2}{}%
}
\def\TestGroup{%
  \Test\getrefnumber\\%
  \Test\getpagerefnumber\\%
  \Test\getrefbykeydefault{}{}\\%
  \Test\getrefbykeydefault{page}{}\\%
  \Test\getrefbykeydefault{anchor}{}\\%
  \Test\getrefbykeydefault{name}{}\\%
  \Test\getrefbykeydefault{url}{}\\%
}
\TestGroup
\Test\getrefbykeydefault{title}{}\\%
\msg{}
\def\r@foo{\par\par\par\par\par\par\par\par}
\long\def\Test#1#2\\{%
  \test{#1{foo}#2}{\par}%
}
\TestGroup
\test{\getrefbykeydefault{title}{}{}}{}
\msg{}
\def\r@foo{{ }{ }{ }{ }{ }}
\def\Test#1#2\\{%
  \test{#1{foo}#2}{ }%
}
\TestGroup
\msg{}
\long\def\TestDefault#1{%
  \begingroup
    \setrefcountdefault{#1}%
    \test{\getrefnumber{foo}}{#1}%
    \test{\getpagerefnumber{foo}}{#1}%
  \endgroup
}
\def\TestDefaultX{%
  \TestDefault{}%
  \TestDefault{\par}%
  \TestDefault{ }%
  \TestDefault{\space}%
}
\let\r@foo\@undefined
\TestDefaultX
\let\r@foo\relax
\TestDefaultX
\def\r@foo{}
\TestDefaultX
%    \end{macrocode}
%    \begin{macrocode}
\msg{}
\long\def\Test#1#2#3#4{%
  \begingroup
    \def\TestTask{#1}%
    \@onelevel@sanitize\TestTask
    \msg{* [\TestTask]}%
    \edef\TestResultA{\IfRefUndefinedExpandable{#1}{#2}{#3}}%
    \IfRefUndefinedBabel{#1}{%
      \def\TestResultB{#2}%
    }{%
      \def\TestResultB{#3}%
    }%
    \def\TestExpected{#4}%
    \ifx\TestResultA\TestExpected
      \msg{ \space ok.}%
    \else
      \begingroup
        \@onelevel@sanitize\TestResultA
        \@onelevel@sanitize\TestExpected
        \msg{ \space Result: \space\space[\TestResultA]}%
        \msg{ \space Expected: [\TestExpected]}%
        \errmessage{Test failed!}%
      \endgroup
    \fi
    \ifx\TestResultB\TestExpected
      \msg{ \space ok.}%
    \else
      \begingroup
        \@onelevel@sanitize\TestResultB
        \@onelevel@sanitize\TestExpected
        \msg{ \space Result: \space\space[\TestResultB]}%
        \msg{ \space Expected: [\TestExpected]}%
        \errmessage{Test failed!}%
      \endgroup
    \fi
  \endgroup
}
\begingroup
  \def\r@foo{{}{}}%
  \let\r@bar\@undefined
  \let\r@xyz\relax
  \Test{foo}{true}{false}{false}%
  \Test{bar}{true}{false}{true}%
  \Test{xyz}{true}{false}{true}%
\endgroup
%    \end{macrocode}
%    \begin{macrocode}
\csname @@end\endcsname\end
%</test2>
%    \end{macrocode}
% \subsection{Test with package \xpackage{titleref}}
%
%    \begin{macrocode}
%<*test3>
\NeedsTeXFormat{LaTeX2e}
\documentclass{article}
\usepackage{refcount}[2011/10/16]
%<test4>\usepackage{nameref}
\usepackage{titleref}
\begin{document}
\section{Hello World}
\label{sec:hello}
\section{\hbox{xy}}
\label{sec:foo}
%
\makeatletter
\@ifundefined{r@sec:hello}{%
  \typeout{==> Compile twice!}%
}{%
  \def\test#1#2{%
    \begingroup
      \def\TestTask{#1}%
      \@onelevel@sanitize\TestTask
      \typeout{* \TestTask}%
      \expandafter\expandafter\expandafter\def
      \expandafter\expandafter\expandafter\TestResult
      \expandafter\expandafter\expandafter{%
        #1%
      }%
      \def\TestExpected{#2}%
      \ifx\TestResult\TestExpected
        \typeout{ \space ok.}%
      \else
        \@onelevel@sanitize\TestResult
        \@onelevel@sanitize\TestExpected
        \typeout{ \space Result: \space\space[\TestResult]}%
        \typeout{ \space Expected: [\TestExpected]}%
        \errmessage{Test failed!}%
      \fi
    \endgroup
  }%
  \test{\getrefbykeydefault{sec:hello}{title}{}}{Hello World}%
  \test{\getrefbykeydefault{sec:foo}{title}{}}{\hbox{xy}}%
  \begingroup
    \def\hbox#1{[#1]}% hash-ok
    \test{\getrefbykeydefault{sec:foo}{title}{}}{\hbox{xy}}%
  \endgroup
}
\makeatother
%    \end{macrocode}
%    \begin{macrocode}
\end{document}
%</test3>
%    \end{macrocode}
%    \begin{macrocode}
%<*test5>
\NeedsTeXFormat{LaTeX2e}
\documentclass{book}
\usepackage{refcount}[2011/10/16]
\usepackage{zref-runs}
\newcounter{test}
\begin{document}
\ifnum\zruns>1 %
  \makeatletter
  \def\Test#1#2#3{%
    \begingroup
      \setcounter{test}{10}%
      \sbox0{%
        #1{test}{#2}%
        \ifnum#3=\value{test}%
        \else
          \PackageError{test}{\string#1{#2} <> #3 (\the\value{test})}%
        \fi
      }%
      \ifdim\wd0=0pt %
      \else
        \PackageError{test}{Non-empty box}\@ehc
      \fi
    \endgroup
  }%
  \makeatother
  \Test\setcounterpageref{ch:two}{1}%
  \Test\setcounterpageref{ch:three}{3}%
  \Test\setcounterpageref{ch:four}{5}%
  \Test\setcounterpageref{ch:five}{7}%
  \Test\setcounterpageref{ch:six}{9}%
  \Test\setcounterpageref{ch:seven}{13}%
  \Test\addtocounterpageref{ch:two}{11}%
  \Test\addtocounterpageref{ch:three}{13}%
  \Test\addtocounterpageref{ch:four}{15}%
  \Test\addtocounterpageref{ch:five}{17}%
  \Test\addtocounterpageref{ch:six}{19}%
  \Test\addtocounterpageref{ch:seven}{23}%
  \Test\setcounterref{ch:two}{1}%
  \Test\setcounterref{ch:three}{2}%
  \Test\setcounterref{ch:four}{11}%
  \Test\addtocounterref{ch:two}{11}%
  \Test\addtocounterref{ch:three}{12}%
  \Test\addtocounterref{ch:four}{21}%
\fi
\frontmatter
\chapter{Chapter one}\label{ch:one}
\cleardoublepage
\mainmatter
\chapter{Chapter two}\label{ch:two}
\cleardoublepage
\chapter{Chapter three}\label{ch:three}
\cleardoublepage
\setcounter{chapter}{10}
\chapter{Chapter four}\label{ch:four}
\cleardoublepage
\appendix
\chapter{Chapter five}\label{ch:five}
\cleardoublepage
\chapter{Chapter six}\label{ch:six}
\cleardoublepage
\null
\cleardoublepage
\chapter{Chapter seven}\label{ch:seven}
\end{document}
%</test5>
%    \end{macrocode}
%
% \section{Installation}
%
% \subsection{Download}
%
% \paragraph{Package.} This package is available on
% CTAN\footnote{\url{ftp://ftp.ctan.org/tex-archive/}}:
% \begin{description}
% \item[\CTAN{macros/latex/contrib/oberdiek/refcount.dtx}] The source file.
% \item[\CTAN{macros/latex/contrib/oberdiek/refcount.pdf}] Documentation.
% \end{description}
%
%
% \paragraph{Bundle.} All the packages of the bundle `oberdiek'
% are also available in a TDS compliant ZIP archive. There
% the packages are already unpacked and the documentation files
% are generated. The files and directories obey the TDS standard.
% \begin{description}
% \item[\CTAN{install/macros/latex/contrib/oberdiek.tds.zip}]
% \end{description}
% \emph{TDS} refers to the standard ``A Directory Structure
% for \TeX\ Files'' (\CTAN{tds/tds.pdf}). Directories
% with \xfile{texmf} in their name are usually organized this way.
%
% \subsection{Bundle installation}
%
% \paragraph{Unpacking.} Unpack the \xfile{oberdiek.tds.zip} in the
% TDS tree (also known as \xfile{texmf} tree) of your choice.
% Example (linux):
% \begin{quote}
%   |unzip oberdiek.tds.zip -d ~/texmf|
% \end{quote}
%
% \paragraph{Script installation.}
% Check the directory \xfile{TDS:scripts/oberdiek/} for
% scripts that need further installation steps.
% Package \xpackage{attachfile2} comes with the Perl script
% \xfile{pdfatfi.pl} that should be installed in such a way
% that it can be called as \texttt{pdfatfi}.
% Example (linux):
% \begin{quote}
%   |chmod +x scripts/oberdiek/pdfatfi.pl|\\
%   |cp scripts/oberdiek/pdfatfi.pl /usr/local/bin/|
% \end{quote}
%
% \subsection{Package installation}
%
% \paragraph{Unpacking.} The \xfile{.dtx} file is a self-extracting
% \docstrip\ archive. The files are extracted by running the
% \xfile{.dtx} through \plainTeX:
% \begin{quote}
%   \verb|tex refcount.dtx|
% \end{quote}
%
% \paragraph{TDS.} Now the different files must be moved into
% the different directories in your installation TDS tree
% (also known as \xfile{texmf} tree):
% \begin{quote}
% \def\t{^^A
% \begin{tabular}{@{}>{\ttfamily}l@{ $\rightarrow$ }>{\ttfamily}l@{}}
%   refcount.sty & tex/latex/oberdiek/refcount.sty\\
%   refcount.pdf & doc/latex/oberdiek/refcount.pdf\\
%   test/refcount-test1.tex & doc/latex/oberdiek/test/refcount-test1.tex\\
%   test/refcount-test2.tex & doc/latex/oberdiek/test/refcount-test2.tex\\
%   test/refcount-test3.tex & doc/latex/oberdiek/test/refcount-test3.tex\\
%   test/refcount-test4.tex & doc/latex/oberdiek/test/refcount-test4.tex\\
%   test/refcount-test5.tex & doc/latex/oberdiek/test/refcount-test5.tex\\
%   refcount.dtx & source/latex/oberdiek/refcount.dtx\\
% \end{tabular}^^A
% }^^A
% \sbox0{\t}^^A
% \ifdim\wd0>\linewidth
%   \begingroup
%     \advance\linewidth by\leftmargin
%     \advance\linewidth by\rightmargin
%   \edef\x{\endgroup
%     \def\noexpand\lw{\the\linewidth}^^A
%   }\x
%   \def\lwbox{^^A
%     \leavevmode
%     \hbox to \linewidth{^^A
%       \kern-\leftmargin\relax
%       \hss
%       \usebox0
%       \hss
%       \kern-\rightmargin\relax
%     }^^A
%   }^^A
%   \ifdim\wd0>\lw
%     \sbox0{\small\t}^^A
%     \ifdim\wd0>\linewidth
%       \ifdim\wd0>\lw
%         \sbox0{\footnotesize\t}^^A
%         \ifdim\wd0>\linewidth
%           \ifdim\wd0>\lw
%             \sbox0{\scriptsize\t}^^A
%             \ifdim\wd0>\linewidth
%               \ifdim\wd0>\lw
%                 \sbox0{\tiny\t}^^A
%                 \ifdim\wd0>\linewidth
%                   \lwbox
%                 \else
%                   \usebox0
%                 \fi
%               \else
%                 \lwbox
%               \fi
%             \else
%               \usebox0
%             \fi
%           \else
%             \lwbox
%           \fi
%         \else
%           \usebox0
%         \fi
%       \else
%         \lwbox
%       \fi
%     \else
%       \usebox0
%     \fi
%   \else
%     \lwbox
%   \fi
% \else
%   \usebox0
% \fi
% \end{quote}
% If you have a \xfile{docstrip.cfg} that configures and enables \docstrip's
% TDS installing feature, then some files can already be in the right
% place, see the documentation of \docstrip.
%
% \subsection{Refresh file name databases}
%
% If your \TeX~distribution
% (\teTeX, \mikTeX, \dots) relies on file name databases, you must refresh
% these. For example, \teTeX\ users run \verb|texhash| or
% \verb|mktexlsr|.
%
% \subsection{Some details for the interested}
%
% \paragraph{Attached source.}
%
% The PDF documentation on CTAN also includes the
% \xfile{.dtx} source file. It can be extracted by
% AcrobatReader 6 or higher. Another option is \textsf{pdftk},
% e.g. unpack the file into the current directory:
% \begin{quote}
%   \verb|pdftk refcount.pdf unpack_files output .|
% \end{quote}
%
% \paragraph{Unpacking with \LaTeX.}
% The \xfile{.dtx} chooses its action depending on the format:
% \begin{description}
% \item[\plainTeX:] Run \docstrip\ and extract the files.
% \item[\LaTeX:] Generate the documentation.
% \end{description}
% If you insist on using \LaTeX\ for \docstrip\ (really,
% \docstrip\ does not need \LaTeX), then inform the autodetect routine
% about your intention:
% \begin{quote}
%   \verb|latex \let\install=y\input{refcount.dtx}|
% \end{quote}
% Do not forget to quote the argument according to the demands
% of your shell.
%
% \paragraph{Generating the documentation.}
% You can use both the \xfile{.dtx} or the \xfile{.drv} to generate
% the documentation. The process can be configured by the
% configuration file \xfile{ltxdoc.cfg}. For instance, put this
% line into this file, if you want to have A4 as paper format:
% \begin{quote}
%   \verb|\PassOptionsToClass{a4paper}{article}|
% \end{quote}
% An example follows how to generate the
% documentation with pdf\LaTeX:
% \begin{quote}
%\begin{verbatim}
%pdflatex refcount.dtx
%makeindex -s gind.ist refcount.idx
%pdflatex refcount.dtx
%makeindex -s gind.ist refcount.idx
%pdflatex refcount.dtx
%\end{verbatim}
% \end{quote}
%
% \section{Catalogue}
%
% The following XML file can be used as source for the
% \href{http://mirror.ctan.org/help/Catalogue/catalogue.html}{\TeX\ Catalogue}.
% The elements \texttt{caption} and \texttt{description} are imported
% from the original XML file from the Catalogue.
% The name of the XML file in the Catalogue is \xfile{refcount.xml}.
%    \begin{macrocode}
%<*catalogue>
<?xml version='1.0' encoding='us-ascii'?>
<!DOCTYPE entry SYSTEM 'catalogue.dtd'>
<entry datestamp='$Date$' modifier='$Author$' id='refcount'>
  <name>refcount</name>
  <caption>Counter operations with label references.</caption>
  <authorref id='auth:oberdiek'/>
  <copyright owner='Heiko Oberdiek' year='1998,2000,2006,2008,2010,2011'/>
  <license type='lppl1.3'/>
  <version number='3.4'/>
  <description>
    Provides commands <tt>\setcounterref</tt> and
    <tt>\addtocounterref</tt> which use the section (or whatever)
    number from the reference as the value to put into the counter, as
    in:

    <pre>
    ...\label{sec:foo}
    ...
    \setcounterref{foonum}{sec:foo}
    </pre>
    Commands <tt>\setcounterpageref</tt> and
    <tt>\addtocounterpageref</tt> do the corresponding thing with the
    page reference of the label.
    <p/>
    No <tt>.ins</tt> file is distributed; process the
    <tt>.dtx</tt> with plain TeX to create one.
    <p/>
    The package is part of the <xref refid='oberdiek'>oberdiek</xref>
    bundle.
  </description>
  <documentation details='Package documentation'
      href='ctan:/macros/latex/contrib/oberdiek/refcount.pdf'/>
  <ctan file='true' path='/macros/latex/contrib/oberdiek/refcount.dtx'/>
  <miktex location='oberdiek'/>
  <texlive location='oberdiek'/>
  <install path='/macros/latex/contrib/oberdiek/oberdiek.tds.zip'/>
</entry>
%</catalogue>
%    \end{macrocode}
%
% \begin{History}
%   \begin{Version}{1998/04/08 v1.0}
%   \item
%     First public release, written as answer in the
%     newsgroup \xnewsgroup{comp.text.tex}:
%     \URL{``\link{Re: Adding a \cs{ref} to a counter?}''}^^A
%     {http://groups.google.com/group/comp.text.tex/msg/c3f2a135ef5ee528}
%   \end{Version}
%   \begin{Version}{2000/09/07 v2.0}
%   \item
%     Documentation added.
%   \item
%     LPPL 1.2
%   \item
%     Package rewritten, new commands added.
%   \end{Version}
%   \begin{Version}{2006/02/20 v3.0}
%   \item
%     Support for \xpackage{hyperref} and \xpackage{nameref} improved.
%   \item
%     Support for \xpackage{titleref} and \xpackage{babel}'s shorthands added.
%   \item
%     New: \cs{refused}, \cs{getrefbykeydefault}
%   \end{Version}
%   \begin{Version}{2008/08/11 v3.1}
%   \item
%     Code is not changed.
%   \item
%     URLs updated.
%   \end{Version}
%   \begin{Version}{2010/12/01 v3.2}
%   \item
%     \cs{IfRefUndefinedExpandable} and \cs{IfRefUndefinedBabel} added.
%   \item
%     \cs{getrefnumber}, \cs{getpagerefnumber}, \cs{getrefbykeydefault}
%     are expandable in exact two expansion steps.
%   \item
%     Non-expandable macros are made robust.
%   \item
%     Test files added.
%   \end{Version}
%   \begin{Version}{2011/06/22 v3.3}
%   \item
%     Bug fix: \cs{rc@refused} is undefined for \cs{setcounterpageref}
%     and similar macros. (Bug found by Marc van Dongen.)
%   \end{Version}
%   \begin{Version}{2011/10/16 v3.4}
%   \item
%     Bug fix: \cs{setcounterpageref} and \cs{addtocounterpageref} fixed.
%     (Bug found by Staz.)
%   \item
%     Macros \cs{(set|addto)counter(page|)ref} are made robust.
%   \end{Version}
% \end{History}
%
% \PrintIndex
%
% \Finale
\endinput

%        (quote the arguments according to the demands of your shell)
%
% Documentation:
%    (a) If refcount.drv is present:
%           latex refcount.drv
%    (b) Without refcount.drv:
%           latex refcount.dtx; ...
%    The class ltxdoc loads the configuration file ltxdoc.cfg
%    if available. Here you can specify further options, e.g.
%    use A4 as paper format:
%       \PassOptionsToClass{a4paper}{article}
%
%    Programm calls to get the documentation (example):
%       pdflatex refcount.dtx
%       makeindex -s gind.ist refcount.idx
%       pdflatex refcount.dtx
%       makeindex -s gind.ist refcount.idx
%       pdflatex refcount.dtx
%
% Installation:
%    TDS:tex/latex/oberdiek/refcount.sty
%    TDS:doc/latex/oberdiek/refcount.pdf
%    TDS:doc/latex/oberdiek/test/refcount-test1.tex
%    TDS:doc/latex/oberdiek/test/refcount-test2.tex
%    TDS:doc/latex/oberdiek/test/refcount-test3.tex
%    TDS:doc/latex/oberdiek/test/refcount-test4.tex
%    TDS:doc/latex/oberdiek/test/refcount-test5.tex
%    TDS:source/latex/oberdiek/refcount.dtx
%
%<*ignore>
\begingroup
  \catcode123=1 %
  \catcode125=2 %
  \def\x{LaTeX2e}%
\expandafter\endgroup
\ifcase 0\ifx\install y1\fi\expandafter
         \ifx\csname processbatchFile\endcsname\relax\else1\fi
         \ifx\fmtname\x\else 1\fi\relax
\else\csname fi\endcsname
%</ignore>
%<*install>
\input docstrip.tex
\Msg{************************************************************************}
\Msg{* Installation}
\Msg{* Package: refcount 2011/10/16 v3.4 Data extraction from label references (HO)}
\Msg{************************************************************************}

\keepsilent
\askforoverwritefalse

\let\MetaPrefix\relax
\preamble

This is a generated file.

Project: refcount
Version: 2011/10/16 v3.4

Copyright (C) 1998, 2000, 2006, 2008, 2010, 2011 by
   Heiko Oberdiek <heiko.oberdiek at googlemail.com>

This work may be distributed and/or modified under the
conditions of the LaTeX Project Public License, either
version 1.3c of this license or (at your option) any later
version. This version of this license is in
   http://www.latex-project.org/lppl/lppl-1-3c.txt
and the latest version of this license is in
   http://www.latex-project.org/lppl.txt
and version 1.3 or later is part of all distributions of
LaTeX version 2005/12/01 or later.

This work has the LPPL maintenance status "maintained".

This Current Maintainer of this work is Heiko Oberdiek.

This work consists of the main source file refcount.dtx
and the derived files
   refcount.sty, refcount.pdf, refcount.ins, refcount.drv,
   refcount-test1.tex, refcount-test2.tex, refcount-test3.tex,
   refcount-test4.tex, refcount-test5.tex.

\endpreamble
\let\MetaPrefix\DoubleperCent

\generate{%
  \file{refcount.ins}{\from{refcount.dtx}{install}}%
  \file{refcount.drv}{\from{refcount.dtx}{driver}}%
  \usedir{tex/latex/oberdiek}%
  \file{refcount.sty}{\from{refcount.dtx}{package}}%
  \usedir{doc/latex/oberdiek/test}%
  \file{refcount-test1.tex}{\from{refcount.dtx}{test1}}%
  \file{refcount-test2.tex}{\from{refcount.dtx}{test2}}%
  \file{refcount-test3.tex}{\from{refcount.dtx}{test3}}%
  \file{refcount-test4.tex}{\from{refcount.dtx}{test3,test4}}%
  \file{refcount-test5.tex}{\from{refcount.dtx}{test5}}%
  \nopreamble
  \nopostamble
  \usedir{source/latex/oberdiek/catalogue}%
  \file{refcount.xml}{\from{refcount.dtx}{catalogue}}%
}

\catcode32=13\relax% active space
\let =\space%
\Msg{************************************************************************}
\Msg{*}
\Msg{* To finish the installation you have to move the following}
\Msg{* file into a directory searched by TeX:}
\Msg{*}
\Msg{*     refcount.sty}
\Msg{*}
\Msg{* To produce the documentation run the file `refcount.drv'}
\Msg{* through LaTeX.}
\Msg{*}
\Msg{* Happy TeXing!}
\Msg{*}
\Msg{************************************************************************}

\endbatchfile
%</install>
%<*ignore>
\fi
%</ignore>
%<*driver>
\NeedsTeXFormat{LaTeX2e}
\ProvidesFile{refcount.drv}%
  [2011/10/16 v3.4 Data extraction from label references (HO)]%
\documentclass{ltxdoc}
\usepackage{holtxdoc}[2011/11/22]
\begin{document}
  \DocInput{refcount.dtx}%
\end{document}
%</driver>
% \fi
%
% \CheckSum{1185}
%
% \CharacterTable
%  {Upper-case    \A\B\C\D\E\F\G\H\I\J\K\L\M\N\O\P\Q\R\S\T\U\V\W\X\Y\Z
%   Lower-case    \a\b\c\d\e\f\g\h\i\j\k\l\m\n\o\p\q\r\s\t\u\v\w\x\y\z
%   Digits        \0\1\2\3\4\5\6\7\8\9
%   Exclamation   \!     Double quote  \"     Hash (number) \#
%   Dollar        \$     Percent       \%     Ampersand     \&
%   Acute accent  \'     Left paren    \(     Right paren   \)
%   Asterisk      \*     Plus          \+     Comma         \,
%   Minus         \-     Point         \.     Solidus       \/
%   Colon         \:     Semicolon     \;     Less than     \<
%   Equals        \=     Greater than  \>     Question mark \?
%   Commercial at \@     Left bracket  \[     Backslash     \\
%   Right bracket \]     Circumflex    \^     Underscore    \_
%   Grave accent  \`     Left brace    \{     Vertical bar  \|
%   Right brace   \}     Tilde         \~}
%
% \GetFileInfo{refcount.drv}
%
% \title{The \xpackage{refcount} package}
% \date{2011/10/16 v3.4}
% \author{Heiko Oberdiek\\\xemail{heiko.oberdiek at googlemail.com}}
%
% \maketitle
%
% \begin{abstract}
% References are not numbers, however they often store numerical
% data such as section or page numbers. \cs{ref} or \cs{pageref}
% cannot be used for counter assignments or calculations because
% they are not expandable, generate warnings, or can even be links.
% The package provides expandable macros to extract the data
% from references. Packages \xpackage{hyperref}, \xpackage{nameref},
% \xpackage{titleref}, and \xpackage{babel} are supported.
% \end{abstract}
%
% \tableofcontents
%
% \section{Usage}
%
% \subsection{Setting counters}
%
% The following commands are similar to \LaTeX's
% \cs{setcounter} and \cs{addtocounter},
% but they extract the number value from a reference:
% \begin{quote}
%   \cs{setcounterref}, \cs{addtocounterref}\\
%   \cs{setcounterpageref}, \cs{addtocounterpageref}
% \end{quote}
% They take two arguments:
% \begin{quote}
%    \cs{...counter...ref} |{|\meta{\LaTeX\ counter}|}|
%    |{|\meta{reference}|}|
% \end{quote}
% An undefined references produces the usual LaTeX warning
% and its value is assumed to be zero.
% Example:
% \begin{quote}
%\begin{verbatim}
%\newcounter{ctrA}
%\newcounter{ctrB}
%\refstepcounter{ctrA}\label{ref:A}
%\setcounterref{ctrB}{ref:A}
%\addtocounterpageref{ctrB}{ref:A}
%\end{verbatim}
% \end{quote}
%
% \subsection{Expandable commands}
%
% These commands that can be used in expandible contexts
% (inside calculations, \cs{edef}, \cs{csname}, \cs{write}, \dots):
% \begin{quote}
%   \cs{getrefnumber}, \cs{getpagerefnumber}
% \end{quote}
% They take one argument, the reference:
% \begin{quote}
%   \cs{get...refnumber} |{|\meta{reference}|}|
% \end{quote}
% The default for undefined references can be changed
% with macro \cs{setrefcountdefault}, for example this
% package calls:
% \begin{quote}
%   \cs{setrefcountdefault}|{0}|
% \end{quote}
%
% Since version 2.0 of this package there is a new
% command:
% \begin{quote}
%   \cs{getrefbykeydefault} |{|\meta{reference}|}|
%   |{|\meta{key}|}| |{|\meta{default}|}|
% \end{quote}
% This generalized version allows the extraction
% of further properties of a reference than the
% two standard ones. Thus the following properties
% are supported, if they are available:
% \begin{quote}
% \begin{tabular}{@{}l|l|l@{}}
%    Key & Description & Package\\
% \hline
%   \meta{empty} & same as \cs{ref} & \LaTeX\\
%   |page| & same as \cs{pageref} & \LaTeX\\
%   |title| & section and caption titles & \xpackage{titleref}\\
%   |name| & section and caption titles & \xpackage{nameref}\\
%   |anchor| & anchor name & \xpackage{hyperref}\\
%   |url| & url/file & \xpackage{hyperref}/\xpackage{xr}
% \end{tabular}
% \end{quote}
%
% Since version 3.2 the expandable macros described before
% in this section
% are expandable in exact two expansion steps.
%
% \subsection{Undefined references}
%
% Because warnings and assignments cannot be used in
% expandible contexts, undefined references do not
% produce a warning, their values are assumed to be zero.
% Example:
% \begin{quote}
%\begin{verbatim}
%\label{ref:here}% somewhere
%\refused{ref:here}% see below
%\ifodd\getpagerefnumber{ref:here}%
%  reference is on an odd page
%\else
%  reference is on an even page
%\fi
%\end{verbatim}
% \end{quote}
%
% In case of undefined references the user usually want's
% to be informed. Also \LaTeX\ prints a warning at
% the end of the \LaTeX\ run. To notify \LaTeX\ and
% get a normal warning, just use
% \begin{quote}
%   \cs{refused} |{|\meta{reference}|}|
% \end{quote}
% outside the expanding context. Example, see above.
%
% \subsubsection{Check for undefined references}
%
% In version 3.2 macros were added, that test, whether references
% are defined.
% \begin{declcs}{IfRefUndefinedExpandable} \M{refname} \M{then} \M{else}\\
%   \cs{IfRefUndefinedBabel} \M{refname} \M{then} \M{else}
% \end{declcs}
% If the reference is not available and therefore undefined, then
% argument \meta{then} is executed, otherwise argument \meta{else}
% is called. Macro \cs{IfRefUndefinedExpandable} is expandable,
% but \meta{refname} must not contain babel shorthand characters.
% Macro \cs{IfRefUndefinedBabel} supports shorthand characters of
% babel, but it is not expandable.
%
% \subsection{Notes}
%
% \begin{itemize}
% \item
%   The method of extracting the number in this
%   package also works in cases, where the
%   reference cannot be used directly, because
%   a package such as \xpackage{hyperref} has added
%   extra stuff (hyper link), so that the reference cannot
%   be used as number any more.
% \item
%   If the reference does not contain a number,
%   assignments to a counter will fail of course.
% \end{itemize}
%
%
% \StopEventually{
% }
%
% \section{Implementation}
%
%    \begin{macrocode}
%<*package>
%    \end{macrocode}
%    Reload check, especially if the package is not used with \LaTeX.
%    \begin{macrocode}
\begingroup\catcode61\catcode48\catcode32=10\relax%
  \catcode13=5 % ^^M
  \endlinechar=13 %
  \catcode35=6 % #
  \catcode39=12 % '
  \catcode44=12 % ,
  \catcode45=12 % -
  \catcode46=12 % .
  \catcode58=12 % :
  \catcode64=11 % @
  \catcode123=1 % {
  \catcode125=2 % }
  \expandafter\let\expandafter\x\csname ver@refcount.sty\endcsname
  \ifx\x\relax % plain-TeX, first loading
  \else
    \def\empty{}%
    \ifx\x\empty % LaTeX, first loading,
      % variable is initialized, but \ProvidesPackage not yet seen
    \else
      \expandafter\ifx\csname PackageInfo\endcsname\relax
        \def\x#1#2{%
          \immediate\write-1{Package #1 Info: #2.}%
        }%
      \else
        \def\x#1#2{\PackageInfo{#1}{#2, stopped}}%
      \fi
      \x{refcount}{The package is already loaded}%
      \aftergroup\endinput
    \fi
  \fi
\endgroup%
%    \end{macrocode}
%    Package identification:
%    \begin{macrocode}
\begingroup\catcode61\catcode48\catcode32=10\relax%
  \catcode13=5 % ^^M
  \endlinechar=13 %
  \catcode35=6 % #
  \catcode39=12 % '
  \catcode40=12 % (
  \catcode41=12 % )
  \catcode44=12 % ,
  \catcode45=12 % -
  \catcode46=12 % .
  \catcode47=12 % /
  \catcode58=12 % :
  \catcode64=11 % @
  \catcode91=12 % [
  \catcode93=12 % ]
  \catcode123=1 % {
  \catcode125=2 % }
  \expandafter\ifx\csname ProvidesPackage\endcsname\relax
    \def\x#1#2#3[#4]{\endgroup
      \immediate\write-1{Package: #3 #4}%
      \xdef#1{#4}%
    }%
  \else
    \def\x#1#2[#3]{\endgroup
      #2[{#3}]%
      \ifx#1\@undefined
        \xdef#1{#3}%
      \fi
      \ifx#1\relax
        \xdef#1{#3}%
      \fi
    }%
  \fi
\expandafter\x\csname ver@refcount.sty\endcsname
\ProvidesPackage{refcount}%
  [2011/10/16 v3.4 Data extraction from label references (HO)]%
%    \end{macrocode}
%
%    \begin{macrocode}
\begingroup\catcode61\catcode48\catcode32=10\relax%
  \catcode13=5 % ^^M
  \endlinechar=13 %
  \catcode123=1 % {
  \catcode125=2 % }
  \catcode64=11 % @
  \def\x{\endgroup
    \expandafter\edef\csname rc@AtEnd\endcsname{%
      \endlinechar=\the\endlinechar\relax
      \catcode13=\the\catcode13\relax
      \catcode32=\the\catcode32\relax
      \catcode35=\the\catcode35\relax
      \catcode61=\the\catcode61\relax
      \catcode64=\the\catcode64\relax
      \catcode123=\the\catcode123\relax
      \catcode125=\the\catcode125\relax
    }%
  }%
\x\catcode61\catcode48\catcode32=10\relax%
\catcode13=5 % ^^M
\endlinechar=13 %
\catcode35=6 % #
\catcode64=11 % @
\catcode123=1 % {
\catcode125=2 % }
\def\TMP@EnsureCode#1#2{%
  \edef\rc@AtEnd{%
    \rc@AtEnd
    \catcode#1=\the\catcode#1\relax
  }%
  \catcode#1=#2\relax
}
\TMP@EnsureCode{33}{12}% !
\TMP@EnsureCode{39}{12}% '
\TMP@EnsureCode{42}{12}% *
\TMP@EnsureCode{45}{12}% -
\TMP@EnsureCode{46}{12}% .
\TMP@EnsureCode{47}{12}% /
\TMP@EnsureCode{91}{12}% [
\TMP@EnsureCode{93}{12}% ]
\TMP@EnsureCode{96}{12}% `
\edef\rc@AtEnd{\rc@AtEnd\noexpand\endinput}
%    \end{macrocode}
%
% \subsection{Loading packages}
%
%    \begin{macrocode}
\begingroup\expandafter\expandafter\expandafter\endgroup
\expandafter\ifx\csname RequirePackage\endcsname\relax
  \input ltxcmds.sty\relax
  \input infwarerr.sty\relax
\else
  \RequirePackage{ltxcmds}[2011/11/09]%
  \RequirePackage{infwarerr}[2010/04/08]%
\fi
%    \end{macrocode}
%
% \subsection{Defining commands}
%
%    \begin{macro}{\rc@IfDefinable}
%    \begin{macrocode}
\ltx@IfUndefined{@ifdefinable}{%
  \def\rc@IfDefinable#1{%
    \ifx#1\ltx@undefined
      \expandafter\ltx@firstofone
    \else
      \ifx#1\relax
        \expandafter\expandafter\expandafter\ltx@firstofone
      \else
        \@PackageError{refcount}{%
          Command \string#1 is already defined.\MessageBreak
          It will not redefined by this package%
        }\@ehc
        \expandafter\expandafter\expandafter\ltx@gobble
      \fi
    \fi
  }%
}{%
  \let\rc@IfDefinable\@ifdefinable
}
%    \end{macrocode}
%    \end{macro}
%
%    \begin{macro}{\rc@RobustDefOne}
%    \begin{macro}{\rc@RobustDefZero}
%    \begin{macrocode}
\ltx@IfUndefined{protected}{%
  \ltx@IfUndefined{DeclareRobustCommand}{%
    \def\rc@RobustDefOne#1#2#3#4{%
      \rc@IfDefinable#3{%
        #1\def#3##1{#4}%
      }%
    }%
    \def\rc@RobustDefZero#1#2{%
      \rc@IfDefinable#1{%
        \def#1{#2}%
      }%
    }%
  }{%
    \def\rc@RobustDefOne#1#2#3#4{%
      \rc@IfDefinable#3{%
        \DeclareRobustCommand#2#3[1]{#4}%
      }%
    }%
    \def\rc@RobustDefZero#1#2{%
      \rc@IfDefinable#1{%
        \DeclareRobustCommand#1{#2}%
      }%
    }%
  }%
}{%
  \def\rc@RobustDefOne#1#2#3#4{%
    \rc@IfDefinable#3{%
      \protected#1\def#3##1{#4}%
    }%
  }%
  \def\rc@RobustDefZero#1#2{%
    \rc@IfDefinable#1{%
      \protected\def#1{#2}%
    }%
  }%
}
%    \end{macrocode}
%    \end{macro}
%    \end{macro}
%
%    \begin{macro}{\rc@newcommand}
%    \begin{macrocode}
\ltx@IfUndefined{newcommand}{%
  \def\rc@newcommand*#1[#2]#3{% hash-ok
    \rc@IfDefinable#1{%
      \ifcase#2 %
        \def#1{#3}%
      \or
        \def#1##1{#3}%
      \or
        \def#1##1##2{#3}%
      \else
        \rc@InternalError
      \fi
    }%
  }%
}{%
  \let\rc@newcommand\newcommand
}
%    \end{macrocode}
%    \end{macro}
%
% \subsection{\cs{setrefcountdefault}}
%
%    \begin{macro}{\setrefcountdefault}
%    \begin{macrocode}
\rc@RobustDefOne\long{}\setrefcountdefault{%
  \def\rc@default{#1}%
}
%    \end{macrocode}
%    \end{macro}
%    \begin{macrocode}
\setrefcountdefault{0}
%    \end{macrocode}
%
% \subsection{\cs{refused}}
%
%    \begin{macro}{\refused}
%    \begin{macrocode}
\ltx@IfUndefined{G@refundefinedtrue}{%
  \rc@RobustDefOne{}{*}\refused{%
    \begingroup
      \csname @safe@activestrue\endcsname
      \ltx@IfUndefined{r@#1}{%
        \protect\G@refundefinedtrue
        \rc@WarningUndefined{#1}%
      }{}%
    \endgroup
  }%
}{%
  \rc@RobustDefOne{}{*}\refused{%
    \begingroup
      \csname @safe@activestrue\endcsname
      \ltx@IfUndefined{r@#1}{%
        \csname protect\expandafter\endcsname
        \csname G@refundefinedtrue\endcsname
        \rc@WarningUndefined{#1}%
      }{}%
    \endgroup
  }%
}
%    \end{macrocode}
%    \end{macro}
%    \begin{macro}{\rc@WarningUndefined}
%    \begin{macrocode}
\ltx@IfUndefined{@latex@warning}{%
  \def\rc@WarningUndefined#1{%
    \ltx@ifundefined{thepage}{%
      \def\thepage{\number\count0 }%
    }{}%
    \@PackageWarning{refcount}{%
      Reference `#1' on page \thepage\space undefined%
    }%
  }%
}{%
  \def\rc@WarningUndefined#1{%
    \@latex@warning{%
      Reference `#1' on page \thepage\space undefined%
    }%
  }%
}
%    \end{macrocode}
%    \end{macro}
%
% \subsection{Setting counters by reference data}
%
% \subsubsection{Generic setting}
%
%    \begin{macro}{\rc@set}
% Generic command for
% |\|$\{$|set|$,$|addto|$\}$|counter|$\{$|page|$,\}$|ref|:
%\begin{quote}
%\begin{tabular}[t]{@{}l@{: }l@{}}
% |#1|& \cs{setcounter}, \cs{addtocounter}\\
% |#2|& \cs{ltx@car} (for \cs{ref}), \cs{ltx@cartwo} (for \cs{pageref})\\
% |#3|& \hologo{LaTeX} counter\\
% |#4|& reference\\
%\end{tabular}
%\end{quote}
%    \begin{macrocode}
\def\rc@set#1#2#3#4{%
  \begingroup
    \csname @safe@activestrue\endcsname
    \refused{#4}%
    \expandafter\rc@@set\csname r@#4\endcsname{#1}{#2}{#3}%
  \endgroup
}
%    \end{macrocode}
%    \end{macro}
%    \begin{macro}{\rc@@set}
%\begin{quote}
%\begin{tabular}[t]{@{}l@{: }l@{}}
% |#1|& \cs{r@<...>}\\
% |#2|& \cs{setcounter}, \cs{addtocounter}\\
% |#3|& \cs{ltx@car} (for \cs{ref}), \cs{ltx@carsecond} (for \cs{pageref})\\
% |#4|& \hologo{LaTeX} counter\\
%\end{tabular}
%\end{quote}
%    \begin{macrocode}
\def\rc@@set#1#2#3#4{%
  \ifx#1\relax
    #2{#4}{\rc@default}%
  \else
    #2{#4}{%
      \expandafter#3#1\rc@default\rc@default\@nil
    }%
  \fi
}
%    \end{macrocode}
%    \end{macro}
%
% \subsubsection{User commands}
%
%    \begin{macro}{\setcounterref}
%    \begin{macrocode}
\rc@RobustDefZero\setcounterref{%
  \rc@set\setcounter\ltx@car
}
%    \end{macrocode}
%    \end{macro}
%    \begin{macro}{\addtocounterref}
%    \begin{macrocode}
\rc@RobustDefZero\addtocounterref{%
  \rc@set\addtocounter\ltx@car
}
%    \end{macrocode}
%    \end{macro}
%    \begin{macro}{\setcounterpageref}
%    \begin{macrocode}
\rc@RobustDefZero\setcounterpageref{%
  \rc@set\setcounter\ltx@carsecond
}
%    \end{macrocode}
%    \end{macro}
%    \begin{macro}{\addtocounterpageref}
%    \begin{macrocode}
\rc@RobustDefZero\addtocounterpageref{%
  \rc@set\addtocounter\ltx@carsecond
}
%    \end{macrocode}
%    \end{macro}
%
% \subsection{Extracting references}
%
%    \begin{macro}{\getrefnumber}
%    \begin{macrocode}
\rc@newcommand*{\getrefnumber}[1]{%
  \romannumeral
  \ltx@ifundefined{r@#1}{%
    \expandafter\ltx@zero
    \rc@default
  }{%
    \expandafter\expandafter\expandafter\rc@extract@
    \expandafter\expandafter\expandafter!%
    \csname r@#1\expandafter\endcsname
    \expandafter{\rc@default}\@nil
  }%
}
%    \end{macrocode}
%    \end{macro}
%    \begin{macro}{\getpagerefnumber}
%    \begin{macrocode}
\rc@newcommand*{\getpagerefnumber}[1]{%
  \romannumeral
  \ltx@ifundefined{r@#1}{%
    \expandafter\ltx@zero
    \rc@default
  }{%
    \expandafter\expandafter\expandafter\rc@extract@page
    \expandafter\expandafter\expandafter!%
    \csname r@#1\expandafter\expandafter\expandafter\endcsname
    \expandafter\expandafter\expandafter{%
      \expandafter\rc@default
    \expandafter}\expandafter{\rc@default}\@nil
  }%
}
%    \end{macrocode}
%    \end{macro}
%    \begin{macro}{\getrefbykeydefault}
%    \begin{macrocode}
\rc@newcommand*{\getrefbykeydefault}[2]{%
  \romannumeral
  \expandafter\rc@getrefbykeydefault
    \csname r@#1\expandafter\endcsname
    \csname rc@extract@#2\endcsname
}
%    \end{macrocode}
%    \end{macro}
%    \begin{macro}{\rc@getrefbykeydefault}
%\begin{quote}
%\begin{tabular}[t]{@{}l@{: }l@{}}
% |#1|& \cs{r@<...>}\\
% |#2|& \cs{rc@extract@<...>}\\
% |#3|& default\\
%\end{tabular}
%\end{quote}
%    \begin{macrocode}
\long\def\rc@getrefbykeydefault#1#2#3{%
  \ifx#1\relax
    % reference is undefined
    \ltx@ReturnAfterElseFi{%
      \ltx@zero
      #3%
    }%
  \else
    \ltx@ReturnAfterFi{%
      \ifx#2\relax
        % extract method is missing
        \ltx@ReturnAfterElseFi{%
          \ltx@zero
          #3%
        }%
      \else
        \ltx@ReturnAfterFi{%
          \expandafter
          \rc@generic#1{#3}{#3}{#3}{#3}{#3}\@nil#2{#3}%
        }%
      \fi
    }%
  \fi
}
%    \end{macrocode}
%    \end{macro}
%    \begin{macro}{\rc@generic}
%\begin{quote}
%\begin{tabular}[t]{@{}l@{: }l@{}}
% |#1|& first item in \cs{r@<...>}\\
% |#2|& remaining items in \cs{r@<...>}\\
% |#3|& \cs{rc@extract@<...>}\\
% |#4|& default\\
%\end{tabular}
%\end{quote}
%    \begin{macrocode}
\long\def\rc@generic#1#2\@nil#3#4{%
  #3{#1\TR@TitleReference\@empty{#4}\@nil}{#1}#2\@nil
}
%    \end{macrocode}
%    \end{macro}
%    \begin{macro}{\rc@extract@}
%    \begin{macrocode}
\long\def\rc@extract@#1#2#3\@nil{%
  \ltx@zero
  #2%
}
%    \end{macrocode}
%    \end{macro}
%    \begin{macro}{\rc@extract@page}
%    \begin{macrocode}
\long\def\rc@extract@page#1#2#3#4\@nil{%
  \ltx@zero
  #3%
}
%    \end{macrocode}
%    \end{macro}
%    \begin{macro}{\rc@extract@name}
%    \begin{macrocode}
\long\def\rc@extract@name#1#2#3#4#5\@nil{%
  \ltx@zero
  #4%
}
%    \end{macrocode}
%    \end{macro}
%    \begin{macro}{\rc@extract@anchor}
%    \begin{macrocode}
\long\def\rc@extract@anchor#1#2#3#4#5#6\@nil{%
  \ltx@zero
  #5%
}
%    \end{macrocode}
%    \end{macro}
%    \begin{macro}{\rc@extract@url}
%    \begin{macrocode}
\long\def\rc@extract@url#1#2#3#4#5#6#7\@nil{%
  \ltx@zero
  #6%
}
%    \end{macrocode}
%    \end{macro}
%    \begin{macro}{\rc@extract@title}
%    \begin{macrocode}
\long\def\rc@extract@title#1#2\@nil{%
  \rc@@extract@title#1%
}
%    \end{macrocode}
%    \end{macro}
%    \begin{macro}{\rc@@extract@title}
%    \begin{macrocode}
\long\def\rc@@extract@title#1\TR@TitleReference#2#3#4\@nil{%
  \ltx@zero
  #3%
}
%    \end{macrocode}
%    \end{macro}
%
% \subsection{Macros for checking undefined references}
%
%    \begin{macro}{\IfRefUndefinedExpandable}
%    \begin{macrocode}
\rc@newcommand*{\IfRefUndefinedExpandable}[1]{%
  \ltx@ifundefined{r@#1}\ltx@firstoftwo\ltx@secondoftwo
}
%    \end{macrocode}
%    \end{macro}
%    \begin{macro}{\IfRefUndefinedBabel}
%    \begin{macrocode}
\rc@RobustDefOne{}*\IfRefUndefinedBabel{%
  \begingroup
    \csname safe@actives@true\endcsname
  \expandafter\expandafter\expandafter\endgroup
  \expandafter\ifx\csname r@#1\endcsname\relax
    \expandafter\ltx@firstoftwo
  \else
    \expandafter\ltx@secondoftwo
  \fi
}
%    \end{macrocode}
%    \end{macro}
%    \begin{macrocode}
\rc@AtEnd%
%</package>
%    \end{macrocode}
%
% \section{Test}
%
% \subsection{Catcode checks for loading}
%
%    \begin{macrocode}
%<*test1>
%    \end{macrocode}
%    \begin{macrocode}
\catcode`\{=1 %
\catcode`\}=2 %
\catcode`\#=6 %
\catcode`\@=11 %
\expandafter\ifx\csname count@\endcsname\relax
  \countdef\count@=255 %
\fi
\expandafter\ifx\csname @gobble\endcsname\relax
  \long\def\@gobble#1{}%
\fi
\expandafter\ifx\csname @firstofone\endcsname\relax
  \long\def\@firstofone#1{#1}%
\fi
\expandafter\ifx\csname loop\endcsname\relax
  \expandafter\@firstofone
\else
  \expandafter\@gobble
\fi
{%
  \def\loop#1\repeat{%
    \def\body{#1}%
    \iterate
  }%
  \def\iterate{%
    \body
      \let\next\iterate
    \else
      \let\next\relax
    \fi
    \next
  }%
  \let\repeat=\fi
}%
\def\RestoreCatcodes{}
\count@=0 %
\loop
  \edef\RestoreCatcodes{%
    \RestoreCatcodes
    \catcode\the\count@=\the\catcode\count@\relax
  }%
\ifnum\count@<255 %
  \advance\count@ 1 %
\repeat

\def\RangeCatcodeInvalid#1#2{%
  \count@=#1\relax
  \loop
    \catcode\count@=15 %
  \ifnum\count@<#2\relax
    \advance\count@ 1 %
  \repeat
}
\def\RangeCatcodeCheck#1#2#3{%
  \count@=#1\relax
  \loop
    \ifnum#3=\catcode\count@
    \else
      \errmessage{%
        Character \the\count@\space
        with wrong catcode \the\catcode\count@\space
        instead of \number#3%
      }%
    \fi
  \ifnum\count@<#2\relax
    \advance\count@ 1 %
  \repeat
}
\def\space{ }
\expandafter\ifx\csname LoadCommand\endcsname\relax
  \def\LoadCommand{\input refcount.sty\relax}%
\fi
\def\Test{%
  \RangeCatcodeInvalid{0}{47}%
  \RangeCatcodeInvalid{58}{64}%
  \RangeCatcodeInvalid{91}{96}%
  \RangeCatcodeInvalid{123}{255}%
  \catcode`\@=12 %
  \catcode`\\=0 %
  \catcode`\%=14 %
  \LoadCommand
  \RangeCatcodeCheck{0}{36}{15}%
  \RangeCatcodeCheck{37}{37}{14}%
  \RangeCatcodeCheck{38}{47}{15}%
  \RangeCatcodeCheck{48}{57}{12}%
  \RangeCatcodeCheck{58}{63}{15}%
  \RangeCatcodeCheck{64}{64}{12}%
  \RangeCatcodeCheck{65}{90}{11}%
  \RangeCatcodeCheck{91}{91}{15}%
  \RangeCatcodeCheck{92}{92}{0}%
  \RangeCatcodeCheck{93}{96}{15}%
  \RangeCatcodeCheck{97}{122}{11}%
  \RangeCatcodeCheck{123}{255}{15}%
  \RestoreCatcodes
}
\Test
\csname @@end\endcsname
\end
%    \end{macrocode}
%    \begin{macrocode}
%</test1>
%    \end{macrocode}
%
% \subsection{Macro tests}
%
%    \begin{macrocode}
%<*test2>
\errorcontextlines=10000 %
\showboxbreadth=10000 %
\showboxdepth=10000 %
\begingroup\expandafter\expandafter\expandafter\endgroup
\expandafter\ifx\csname RequirePackage\endcsname\relax
  \input refcount.sty\relax
\else
  \RequirePackage{refcount}[2011/10/16]%
\fi
\catcode`\@=11 %
\begingroup\expandafter\expandafter\expandafter\endgroup
\expandafter\ifx\csname @onelevel@sanitize\endcsname\relax
  \begingroup\expandafter\expandafter\expandafter\endgroup
  \expandafter\ifx\csname detokenize\endcsname\relax
    \def\strip@prefix#1->{}%
    \def\@onelevel@sanitize#1{%
      \edef#1{%
        \expandafter\strip@prefix\meaning#1%
      }%
    }%
  \else
    \def\@onelevel@sanitize#1{%
      \edef#1{%
        \detokenize\expandafter{#1}%
      }%
    }%
  \fi
\fi
\def\msg#{\immediate\write16}
\def\empty{}
\def\space{ }
%    \end{macrocode}
%    \begin{macrocode}
\def\r@foo{{\empty 1}{\empty 2}}
\long\def\test#1#2{%
  \begingroup
    \setbox0=\hbox{%
      \def\TestTask{#1}%
      \@onelevel@sanitize\TestTask
      \msg{* \TestTask}%
      \expandafter\expandafter\expandafter\def
      \expandafter\expandafter\expandafter\TestResult
      \expandafter\expandafter\expandafter{%
        #1%
      }%
      \def\TestExpected{#2}%
      \ifx\TestResult\TestExpected
        \msg{ \space ok.}%
      \else
        \@onelevel@sanitize\TestResult
        \@onelevel@sanitize\TestExpected
        \msg{ \space Result: \space\space[\TestResult]}%
        \msg{ \space Expected: [\TestExpected]}%
        \errmessage{Test failed!}%
      \fi
    }%
    \ifdim\wd0=0pt %
    \else
      \showbox0 %
    \fi
  \endgroup
}
\test{\getrefnumber{foo}}{\empty 1}
\test{\getpagerefnumber{foo}}{\empty 2}
\test{\getrefbykeydefault{foo}{}{\empty default}}{\empty 1}
\test{\getrefbykeydefault{foo}{page}{\empty default}}{\empty 2}
\test{\getrefbykeydefault{foo}{name}{\empty default}}{\empty default}
\test{\getrefbykeydefault{foo}{anchor}{\empty default}}{\empty default}
\test{\getrefbykeydefault{foo}{url}{\empty default}}{\empty default}
\test{\getrefbykeydefault{foo}{title}{\empty default}}{\empty default}
\msg{}
\def\r@foo{{}{}{}{}{}{}{}{}{}{}}
\def\Test#1#2\\{%
  \test{#1{foo}#2}{}%
}
\def\TestGroup{%
  \Test\getrefnumber\\%
  \Test\getpagerefnumber\\%
  \Test\getrefbykeydefault{}{}\\%
  \Test\getrefbykeydefault{page}{}\\%
  \Test\getrefbykeydefault{anchor}{}\\%
  \Test\getrefbykeydefault{name}{}\\%
  \Test\getrefbykeydefault{url}{}\\%
}
\TestGroup
\Test\getrefbykeydefault{title}{}\\%
\msg{}
\def\r@foo{\par\par\par\par\par\par\par\par}
\long\def\Test#1#2\\{%
  \test{#1{foo}#2}{\par}%
}
\TestGroup
\test{\getrefbykeydefault{title}{}{}}{}
\msg{}
\def\r@foo{{ }{ }{ }{ }{ }}
\def\Test#1#2\\{%
  \test{#1{foo}#2}{ }%
}
\TestGroup
\msg{}
\long\def\TestDefault#1{%
  \begingroup
    \setrefcountdefault{#1}%
    \test{\getrefnumber{foo}}{#1}%
    \test{\getpagerefnumber{foo}}{#1}%
  \endgroup
}
\def\TestDefaultX{%
  \TestDefault{}%
  \TestDefault{\par}%
  \TestDefault{ }%
  \TestDefault{\space}%
}
\let\r@foo\@undefined
\TestDefaultX
\let\r@foo\relax
\TestDefaultX
\def\r@foo{}
\TestDefaultX
%    \end{macrocode}
%    \begin{macrocode}
\msg{}
\long\def\Test#1#2#3#4{%
  \begingroup
    \def\TestTask{#1}%
    \@onelevel@sanitize\TestTask
    \msg{* [\TestTask]}%
    \edef\TestResultA{\IfRefUndefinedExpandable{#1}{#2}{#3}}%
    \IfRefUndefinedBabel{#1}{%
      \def\TestResultB{#2}%
    }{%
      \def\TestResultB{#3}%
    }%
    \def\TestExpected{#4}%
    \ifx\TestResultA\TestExpected
      \msg{ \space ok.}%
    \else
      \begingroup
        \@onelevel@sanitize\TestResultA
        \@onelevel@sanitize\TestExpected
        \msg{ \space Result: \space\space[\TestResultA]}%
        \msg{ \space Expected: [\TestExpected]}%
        \errmessage{Test failed!}%
      \endgroup
    \fi
    \ifx\TestResultB\TestExpected
      \msg{ \space ok.}%
    \else
      \begingroup
        \@onelevel@sanitize\TestResultB
        \@onelevel@sanitize\TestExpected
        \msg{ \space Result: \space\space[\TestResultB]}%
        \msg{ \space Expected: [\TestExpected]}%
        \errmessage{Test failed!}%
      \endgroup
    \fi
  \endgroup
}
\begingroup
  \def\r@foo{{}{}}%
  \let\r@bar\@undefined
  \let\r@xyz\relax
  \Test{foo}{true}{false}{false}%
  \Test{bar}{true}{false}{true}%
  \Test{xyz}{true}{false}{true}%
\endgroup
%    \end{macrocode}
%    \begin{macrocode}
\csname @@end\endcsname\end
%</test2>
%    \end{macrocode}
% \subsection{Test with package \xpackage{titleref}}
%
%    \begin{macrocode}
%<*test3>
\NeedsTeXFormat{LaTeX2e}
\documentclass{article}
\usepackage{refcount}[2011/10/16]
%<test4>\usepackage{nameref}
\usepackage{titleref}
\begin{document}
\section{Hello World}
\label{sec:hello}
\section{\hbox{xy}}
\label{sec:foo}
%
\makeatletter
\@ifundefined{r@sec:hello}{%
  \typeout{==> Compile twice!}%
}{%
  \def\test#1#2{%
    \begingroup
      \def\TestTask{#1}%
      \@onelevel@sanitize\TestTask
      \typeout{* \TestTask}%
      \expandafter\expandafter\expandafter\def
      \expandafter\expandafter\expandafter\TestResult
      \expandafter\expandafter\expandafter{%
        #1%
      }%
      \def\TestExpected{#2}%
      \ifx\TestResult\TestExpected
        \typeout{ \space ok.}%
      \else
        \@onelevel@sanitize\TestResult
        \@onelevel@sanitize\TestExpected
        \typeout{ \space Result: \space\space[\TestResult]}%
        \typeout{ \space Expected: [\TestExpected]}%
        \errmessage{Test failed!}%
      \fi
    \endgroup
  }%
  \test{\getrefbykeydefault{sec:hello}{title}{}}{Hello World}%
  \test{\getrefbykeydefault{sec:foo}{title}{}}{\hbox{xy}}%
  \begingroup
    \def\hbox#1{[#1]}% hash-ok
    \test{\getrefbykeydefault{sec:foo}{title}{}}{\hbox{xy}}%
  \endgroup
}
\makeatother
%    \end{macrocode}
%    \begin{macrocode}
\end{document}
%</test3>
%    \end{macrocode}
%    \begin{macrocode}
%<*test5>
\NeedsTeXFormat{LaTeX2e}
\documentclass{book}
\usepackage{refcount}[2011/10/16]
\usepackage{zref-runs}
\newcounter{test}
\begin{document}
\ifnum\zruns>1 %
  \makeatletter
  \def\Test#1#2#3{%
    \begingroup
      \setcounter{test}{10}%
      \sbox0{%
        #1{test}{#2}%
        \ifnum#3=\value{test}%
        \else
          \PackageError{test}{\string#1{#2} <> #3 (\the\value{test})}%
        \fi
      }%
      \ifdim\wd0=0pt %
      \else
        \PackageError{test}{Non-empty box}\@ehc
      \fi
    \endgroup
  }%
  \makeatother
  \Test\setcounterpageref{ch:two}{1}%
  \Test\setcounterpageref{ch:three}{3}%
  \Test\setcounterpageref{ch:four}{5}%
  \Test\setcounterpageref{ch:five}{7}%
  \Test\setcounterpageref{ch:six}{9}%
  \Test\setcounterpageref{ch:seven}{13}%
  \Test\addtocounterpageref{ch:two}{11}%
  \Test\addtocounterpageref{ch:three}{13}%
  \Test\addtocounterpageref{ch:four}{15}%
  \Test\addtocounterpageref{ch:five}{17}%
  \Test\addtocounterpageref{ch:six}{19}%
  \Test\addtocounterpageref{ch:seven}{23}%
  \Test\setcounterref{ch:two}{1}%
  \Test\setcounterref{ch:three}{2}%
  \Test\setcounterref{ch:four}{11}%
  \Test\addtocounterref{ch:two}{11}%
  \Test\addtocounterref{ch:three}{12}%
  \Test\addtocounterref{ch:four}{21}%
\fi
\frontmatter
\chapter{Chapter one}\label{ch:one}
\cleardoublepage
\mainmatter
\chapter{Chapter two}\label{ch:two}
\cleardoublepage
\chapter{Chapter three}\label{ch:three}
\cleardoublepage
\setcounter{chapter}{10}
\chapter{Chapter four}\label{ch:four}
\cleardoublepage
\appendix
\chapter{Chapter five}\label{ch:five}
\cleardoublepage
\chapter{Chapter six}\label{ch:six}
\cleardoublepage
\null
\cleardoublepage
\chapter{Chapter seven}\label{ch:seven}
\end{document}
%</test5>
%    \end{macrocode}
%
% \section{Installation}
%
% \subsection{Download}
%
% \paragraph{Package.} This package is available on
% CTAN\footnote{\url{ftp://ftp.ctan.org/tex-archive/}}:
% \begin{description}
% \item[\CTAN{macros/latex/contrib/oberdiek/refcount.dtx}] The source file.
% \item[\CTAN{macros/latex/contrib/oberdiek/refcount.pdf}] Documentation.
% \end{description}
%
%
% \paragraph{Bundle.} All the packages of the bundle `oberdiek'
% are also available in a TDS compliant ZIP archive. There
% the packages are already unpacked and the documentation files
% are generated. The files and directories obey the TDS standard.
% \begin{description}
% \item[\CTAN{install/macros/latex/contrib/oberdiek.tds.zip}]
% \end{description}
% \emph{TDS} refers to the standard ``A Directory Structure
% for \TeX\ Files'' (\CTAN{tds/tds.pdf}). Directories
% with \xfile{texmf} in their name are usually organized this way.
%
% \subsection{Bundle installation}
%
% \paragraph{Unpacking.} Unpack the \xfile{oberdiek.tds.zip} in the
% TDS tree (also known as \xfile{texmf} tree) of your choice.
% Example (linux):
% \begin{quote}
%   |unzip oberdiek.tds.zip -d ~/texmf|
% \end{quote}
%
% \paragraph{Script installation.}
% Check the directory \xfile{TDS:scripts/oberdiek/} for
% scripts that need further installation steps.
% Package \xpackage{attachfile2} comes with the Perl script
% \xfile{pdfatfi.pl} that should be installed in such a way
% that it can be called as \texttt{pdfatfi}.
% Example (linux):
% \begin{quote}
%   |chmod +x scripts/oberdiek/pdfatfi.pl|\\
%   |cp scripts/oberdiek/pdfatfi.pl /usr/local/bin/|
% \end{quote}
%
% \subsection{Package installation}
%
% \paragraph{Unpacking.} The \xfile{.dtx} file is a self-extracting
% \docstrip\ archive. The files are extracted by running the
% \xfile{.dtx} through \plainTeX:
% \begin{quote}
%   \verb|tex refcount.dtx|
% \end{quote}
%
% \paragraph{TDS.} Now the different files must be moved into
% the different directories in your installation TDS tree
% (also known as \xfile{texmf} tree):
% \begin{quote}
% \def\t{^^A
% \begin{tabular}{@{}>{\ttfamily}l@{ $\rightarrow$ }>{\ttfamily}l@{}}
%   refcount.sty & tex/latex/oberdiek/refcount.sty\\
%   refcount.pdf & doc/latex/oberdiek/refcount.pdf\\
%   test/refcount-test1.tex & doc/latex/oberdiek/test/refcount-test1.tex\\
%   test/refcount-test2.tex & doc/latex/oberdiek/test/refcount-test2.tex\\
%   test/refcount-test3.tex & doc/latex/oberdiek/test/refcount-test3.tex\\
%   test/refcount-test4.tex & doc/latex/oberdiek/test/refcount-test4.tex\\
%   test/refcount-test5.tex & doc/latex/oberdiek/test/refcount-test5.tex\\
%   refcount.dtx & source/latex/oberdiek/refcount.dtx\\
% \end{tabular}^^A
% }^^A
% \sbox0{\t}^^A
% \ifdim\wd0>\linewidth
%   \begingroup
%     \advance\linewidth by\leftmargin
%     \advance\linewidth by\rightmargin
%   \edef\x{\endgroup
%     \def\noexpand\lw{\the\linewidth}^^A
%   }\x
%   \def\lwbox{^^A
%     \leavevmode
%     \hbox to \linewidth{^^A
%       \kern-\leftmargin\relax
%       \hss
%       \usebox0
%       \hss
%       \kern-\rightmargin\relax
%     }^^A
%   }^^A
%   \ifdim\wd0>\lw
%     \sbox0{\small\t}^^A
%     \ifdim\wd0>\linewidth
%       \ifdim\wd0>\lw
%         \sbox0{\footnotesize\t}^^A
%         \ifdim\wd0>\linewidth
%           \ifdim\wd0>\lw
%             \sbox0{\scriptsize\t}^^A
%             \ifdim\wd0>\linewidth
%               \ifdim\wd0>\lw
%                 \sbox0{\tiny\t}^^A
%                 \ifdim\wd0>\linewidth
%                   \lwbox
%                 \else
%                   \usebox0
%                 \fi
%               \else
%                 \lwbox
%               \fi
%             \else
%               \usebox0
%             \fi
%           \else
%             \lwbox
%           \fi
%         \else
%           \usebox0
%         \fi
%       \else
%         \lwbox
%       \fi
%     \else
%       \usebox0
%     \fi
%   \else
%     \lwbox
%   \fi
% \else
%   \usebox0
% \fi
% \end{quote}
% If you have a \xfile{docstrip.cfg} that configures and enables \docstrip's
% TDS installing feature, then some files can already be in the right
% place, see the documentation of \docstrip.
%
% \subsection{Refresh file name databases}
%
% If your \TeX~distribution
% (\teTeX, \mikTeX, \dots) relies on file name databases, you must refresh
% these. For example, \teTeX\ users run \verb|texhash| or
% \verb|mktexlsr|.
%
% \subsection{Some details for the interested}
%
% \paragraph{Attached source.}
%
% The PDF documentation on CTAN also includes the
% \xfile{.dtx} source file. It can be extracted by
% AcrobatReader 6 or higher. Another option is \textsf{pdftk},
% e.g. unpack the file into the current directory:
% \begin{quote}
%   \verb|pdftk refcount.pdf unpack_files output .|
% \end{quote}
%
% \paragraph{Unpacking with \LaTeX.}
% The \xfile{.dtx} chooses its action depending on the format:
% \begin{description}
% \item[\plainTeX:] Run \docstrip\ and extract the files.
% \item[\LaTeX:] Generate the documentation.
% \end{description}
% If you insist on using \LaTeX\ for \docstrip\ (really,
% \docstrip\ does not need \LaTeX), then inform the autodetect routine
% about your intention:
% \begin{quote}
%   \verb|latex \let\install=y% \iffalse meta-comment
%
% File: refcount.dtx
% Version: 2011/10/16 v3.4
% Info: Data extraction from label references
%
% Copyright (C) 1998, 2000, 2006, 2008, 2010, 2011 by
%    Heiko Oberdiek <heiko.oberdiek at googlemail.com>
%
% This work may be distributed and/or modified under the
% conditions of the LaTeX Project Public License, either
% version 1.3c of this license or (at your option) any later
% version. This version of this license is in
%    http://www.latex-project.org/lppl/lppl-1-3c.txt
% and the latest version of this license is in
%    http://www.latex-project.org/lppl.txt
% and version 1.3 or later is part of all distributions of
% LaTeX version 2005/12/01 or later.
%
% This work has the LPPL maintenance status "maintained".
%
% This Current Maintainer of this work is Heiko Oberdiek.
%
% This work consists of the main source file refcount.dtx
% and the derived files
%    refcount.sty, refcount.pdf, refcount.ins, refcount.drv,
%    refcount-test1.tex, refcount-test2.tex, refcount-test3.tex,
%    refcount-test4.tex, refcount-test5.tex.
%
% Distribution:
%    CTAN:macros/latex/contrib/oberdiek/refcount.dtx
%    CTAN:macros/latex/contrib/oberdiek/refcount.pdf
%
% Unpacking:
%    (a) If refcount.ins is present:
%           tex refcount.ins
%    (b) Without refcount.ins:
%           tex refcount.dtx
%    (c) If you insist on using LaTeX
%           latex \let\install=y\input{refcount.dtx}
%        (quote the arguments according to the demands of your shell)
%
% Documentation:
%    (a) If refcount.drv is present:
%           latex refcount.drv
%    (b) Without refcount.drv:
%           latex refcount.dtx; ...
%    The class ltxdoc loads the configuration file ltxdoc.cfg
%    if available. Here you can specify further options, e.g.
%    use A4 as paper format:
%       \PassOptionsToClass{a4paper}{article}
%
%    Programm calls to get the documentation (example):
%       pdflatex refcount.dtx
%       makeindex -s gind.ist refcount.idx
%       pdflatex refcount.dtx
%       makeindex -s gind.ist refcount.idx
%       pdflatex refcount.dtx
%
% Installation:
%    TDS:tex/latex/oberdiek/refcount.sty
%    TDS:doc/latex/oberdiek/refcount.pdf
%    TDS:doc/latex/oberdiek/test/refcount-test1.tex
%    TDS:doc/latex/oberdiek/test/refcount-test2.tex
%    TDS:doc/latex/oberdiek/test/refcount-test3.tex
%    TDS:doc/latex/oberdiek/test/refcount-test4.tex
%    TDS:doc/latex/oberdiek/test/refcount-test5.tex
%    TDS:source/latex/oberdiek/refcount.dtx
%
%<*ignore>
\begingroup
  \catcode123=1 %
  \catcode125=2 %
  \def\x{LaTeX2e}%
\expandafter\endgroup
\ifcase 0\ifx\install y1\fi\expandafter
         \ifx\csname processbatchFile\endcsname\relax\else1\fi
         \ifx\fmtname\x\else 1\fi\relax
\else\csname fi\endcsname
%</ignore>
%<*install>
\input docstrip.tex
\Msg{************************************************************************}
\Msg{* Installation}
\Msg{* Package: refcount 2011/10/16 v3.4 Data extraction from label references (HO)}
\Msg{************************************************************************}

\keepsilent
\askforoverwritefalse

\let\MetaPrefix\relax
\preamble

This is a generated file.

Project: refcount
Version: 2011/10/16 v3.4

Copyright (C) 1998, 2000, 2006, 2008, 2010, 2011 by
   Heiko Oberdiek <heiko.oberdiek at googlemail.com>

This work may be distributed and/or modified under the
conditions of the LaTeX Project Public License, either
version 1.3c of this license or (at your option) any later
version. This version of this license is in
   http://www.latex-project.org/lppl/lppl-1-3c.txt
and the latest version of this license is in
   http://www.latex-project.org/lppl.txt
and version 1.3 or later is part of all distributions of
LaTeX version 2005/12/01 or later.

This work has the LPPL maintenance status "maintained".

This Current Maintainer of this work is Heiko Oberdiek.

This work consists of the main source file refcount.dtx
and the derived files
   refcount.sty, refcount.pdf, refcount.ins, refcount.drv,
   refcount-test1.tex, refcount-test2.tex, refcount-test3.tex,
   refcount-test4.tex, refcount-test5.tex.

\endpreamble
\let\MetaPrefix\DoubleperCent

\generate{%
  \file{refcount.ins}{\from{refcount.dtx}{install}}%
  \file{refcount.drv}{\from{refcount.dtx}{driver}}%
  \usedir{tex/latex/oberdiek}%
  \file{refcount.sty}{\from{refcount.dtx}{package}}%
  \usedir{doc/latex/oberdiek/test}%
  \file{refcount-test1.tex}{\from{refcount.dtx}{test1}}%
  \file{refcount-test2.tex}{\from{refcount.dtx}{test2}}%
  \file{refcount-test3.tex}{\from{refcount.dtx}{test3}}%
  \file{refcount-test4.tex}{\from{refcount.dtx}{test3,test4}}%
  \file{refcount-test5.tex}{\from{refcount.dtx}{test5}}%
  \nopreamble
  \nopostamble
  \usedir{source/latex/oberdiek/catalogue}%
  \file{refcount.xml}{\from{refcount.dtx}{catalogue}}%
}

\catcode32=13\relax% active space
\let =\space%
\Msg{************************************************************************}
\Msg{*}
\Msg{* To finish the installation you have to move the following}
\Msg{* file into a directory searched by TeX:}
\Msg{*}
\Msg{*     refcount.sty}
\Msg{*}
\Msg{* To produce the documentation run the file `refcount.drv'}
\Msg{* through LaTeX.}
\Msg{*}
\Msg{* Happy TeXing!}
\Msg{*}
\Msg{************************************************************************}

\endbatchfile
%</install>
%<*ignore>
\fi
%</ignore>
%<*driver>
\NeedsTeXFormat{LaTeX2e}
\ProvidesFile{refcount.drv}%
  [2011/10/16 v3.4 Data extraction from label references (HO)]%
\documentclass{ltxdoc}
\usepackage{holtxdoc}[2011/11/22]
\begin{document}
  \DocInput{refcount.dtx}%
\end{document}
%</driver>
% \fi
%
% \CheckSum{1185}
%
% \CharacterTable
%  {Upper-case    \A\B\C\D\E\F\G\H\I\J\K\L\M\N\O\P\Q\R\S\T\U\V\W\X\Y\Z
%   Lower-case    \a\b\c\d\e\f\g\h\i\j\k\l\m\n\o\p\q\r\s\t\u\v\w\x\y\z
%   Digits        \0\1\2\3\4\5\6\7\8\9
%   Exclamation   \!     Double quote  \"     Hash (number) \#
%   Dollar        \$     Percent       \%     Ampersand     \&
%   Acute accent  \'     Left paren    \(     Right paren   \)
%   Asterisk      \*     Plus          \+     Comma         \,
%   Minus         \-     Point         \.     Solidus       \/
%   Colon         \:     Semicolon     \;     Less than     \<
%   Equals        \=     Greater than  \>     Question mark \?
%   Commercial at \@     Left bracket  \[     Backslash     \\
%   Right bracket \]     Circumflex    \^     Underscore    \_
%   Grave accent  \`     Left brace    \{     Vertical bar  \|
%   Right brace   \}     Tilde         \~}
%
% \GetFileInfo{refcount.drv}
%
% \title{The \xpackage{refcount} package}
% \date{2011/10/16 v3.4}
% \author{Heiko Oberdiek\\\xemail{heiko.oberdiek at googlemail.com}}
%
% \maketitle
%
% \begin{abstract}
% References are not numbers, however they often store numerical
% data such as section or page numbers. \cs{ref} or \cs{pageref}
% cannot be used for counter assignments or calculations because
% they are not expandable, generate warnings, or can even be links.
% The package provides expandable macros to extract the data
% from references. Packages \xpackage{hyperref}, \xpackage{nameref},
% \xpackage{titleref}, and \xpackage{babel} are supported.
% \end{abstract}
%
% \tableofcontents
%
% \section{Usage}
%
% \subsection{Setting counters}
%
% The following commands are similar to \LaTeX's
% \cs{setcounter} and \cs{addtocounter},
% but they extract the number value from a reference:
% \begin{quote}
%   \cs{setcounterref}, \cs{addtocounterref}\\
%   \cs{setcounterpageref}, \cs{addtocounterpageref}
% \end{quote}
% They take two arguments:
% \begin{quote}
%    \cs{...counter...ref} |{|\meta{\LaTeX\ counter}|}|
%    |{|\meta{reference}|}|
% \end{quote}
% An undefined references produces the usual LaTeX warning
% and its value is assumed to be zero.
% Example:
% \begin{quote}
%\begin{verbatim}
%\newcounter{ctrA}
%\newcounter{ctrB}
%\refstepcounter{ctrA}\label{ref:A}
%\setcounterref{ctrB}{ref:A}
%\addtocounterpageref{ctrB}{ref:A}
%\end{verbatim}
% \end{quote}
%
% \subsection{Expandable commands}
%
% These commands that can be used in expandible contexts
% (inside calculations, \cs{edef}, \cs{csname}, \cs{write}, \dots):
% \begin{quote}
%   \cs{getrefnumber}, \cs{getpagerefnumber}
% \end{quote}
% They take one argument, the reference:
% \begin{quote}
%   \cs{get...refnumber} |{|\meta{reference}|}|
% \end{quote}
% The default for undefined references can be changed
% with macro \cs{setrefcountdefault}, for example this
% package calls:
% \begin{quote}
%   \cs{setrefcountdefault}|{0}|
% \end{quote}
%
% Since version 2.0 of this package there is a new
% command:
% \begin{quote}
%   \cs{getrefbykeydefault} |{|\meta{reference}|}|
%   |{|\meta{key}|}| |{|\meta{default}|}|
% \end{quote}
% This generalized version allows the extraction
% of further properties of a reference than the
% two standard ones. Thus the following properties
% are supported, if they are available:
% \begin{quote}
% \begin{tabular}{@{}l|l|l@{}}
%    Key & Description & Package\\
% \hline
%   \meta{empty} & same as \cs{ref} & \LaTeX\\
%   |page| & same as \cs{pageref} & \LaTeX\\
%   |title| & section and caption titles & \xpackage{titleref}\\
%   |name| & section and caption titles & \xpackage{nameref}\\
%   |anchor| & anchor name & \xpackage{hyperref}\\
%   |url| & url/file & \xpackage{hyperref}/\xpackage{xr}
% \end{tabular}
% \end{quote}
%
% Since version 3.2 the expandable macros described before
% in this section
% are expandable in exact two expansion steps.
%
% \subsection{Undefined references}
%
% Because warnings and assignments cannot be used in
% expandible contexts, undefined references do not
% produce a warning, their values are assumed to be zero.
% Example:
% \begin{quote}
%\begin{verbatim}
%\label{ref:here}% somewhere
%\refused{ref:here}% see below
%\ifodd\getpagerefnumber{ref:here}%
%  reference is on an odd page
%\else
%  reference is on an even page
%\fi
%\end{verbatim}
% \end{quote}
%
% In case of undefined references the user usually want's
% to be informed. Also \LaTeX\ prints a warning at
% the end of the \LaTeX\ run. To notify \LaTeX\ and
% get a normal warning, just use
% \begin{quote}
%   \cs{refused} |{|\meta{reference}|}|
% \end{quote}
% outside the expanding context. Example, see above.
%
% \subsubsection{Check for undefined references}
%
% In version 3.2 macros were added, that test, whether references
% are defined.
% \begin{declcs}{IfRefUndefinedExpandable} \M{refname} \M{then} \M{else}\\
%   \cs{IfRefUndefinedBabel} \M{refname} \M{then} \M{else}
% \end{declcs}
% If the reference is not available and therefore undefined, then
% argument \meta{then} is executed, otherwise argument \meta{else}
% is called. Macro \cs{IfRefUndefinedExpandable} is expandable,
% but \meta{refname} must not contain babel shorthand characters.
% Macro \cs{IfRefUndefinedBabel} supports shorthand characters of
% babel, but it is not expandable.
%
% \subsection{Notes}
%
% \begin{itemize}
% \item
%   The method of extracting the number in this
%   package also works in cases, where the
%   reference cannot be used directly, because
%   a package such as \xpackage{hyperref} has added
%   extra stuff (hyper link), so that the reference cannot
%   be used as number any more.
% \item
%   If the reference does not contain a number,
%   assignments to a counter will fail of course.
% \end{itemize}
%
%
% \StopEventually{
% }
%
% \section{Implementation}
%
%    \begin{macrocode}
%<*package>
%    \end{macrocode}
%    Reload check, especially if the package is not used with \LaTeX.
%    \begin{macrocode}
\begingroup\catcode61\catcode48\catcode32=10\relax%
  \catcode13=5 % ^^M
  \endlinechar=13 %
  \catcode35=6 % #
  \catcode39=12 % '
  \catcode44=12 % ,
  \catcode45=12 % -
  \catcode46=12 % .
  \catcode58=12 % :
  \catcode64=11 % @
  \catcode123=1 % {
  \catcode125=2 % }
  \expandafter\let\expandafter\x\csname ver@refcount.sty\endcsname
  \ifx\x\relax % plain-TeX, first loading
  \else
    \def\empty{}%
    \ifx\x\empty % LaTeX, first loading,
      % variable is initialized, but \ProvidesPackage not yet seen
    \else
      \expandafter\ifx\csname PackageInfo\endcsname\relax
        \def\x#1#2{%
          \immediate\write-1{Package #1 Info: #2.}%
        }%
      \else
        \def\x#1#2{\PackageInfo{#1}{#2, stopped}}%
      \fi
      \x{refcount}{The package is already loaded}%
      \aftergroup\endinput
    \fi
  \fi
\endgroup%
%    \end{macrocode}
%    Package identification:
%    \begin{macrocode}
\begingroup\catcode61\catcode48\catcode32=10\relax%
  \catcode13=5 % ^^M
  \endlinechar=13 %
  \catcode35=6 % #
  \catcode39=12 % '
  \catcode40=12 % (
  \catcode41=12 % )
  \catcode44=12 % ,
  \catcode45=12 % -
  \catcode46=12 % .
  \catcode47=12 % /
  \catcode58=12 % :
  \catcode64=11 % @
  \catcode91=12 % [
  \catcode93=12 % ]
  \catcode123=1 % {
  \catcode125=2 % }
  \expandafter\ifx\csname ProvidesPackage\endcsname\relax
    \def\x#1#2#3[#4]{\endgroup
      \immediate\write-1{Package: #3 #4}%
      \xdef#1{#4}%
    }%
  \else
    \def\x#1#2[#3]{\endgroup
      #2[{#3}]%
      \ifx#1\@undefined
        \xdef#1{#3}%
      \fi
      \ifx#1\relax
        \xdef#1{#3}%
      \fi
    }%
  \fi
\expandafter\x\csname ver@refcount.sty\endcsname
\ProvidesPackage{refcount}%
  [2011/10/16 v3.4 Data extraction from label references (HO)]%
%    \end{macrocode}
%
%    \begin{macrocode}
\begingroup\catcode61\catcode48\catcode32=10\relax%
  \catcode13=5 % ^^M
  \endlinechar=13 %
  \catcode123=1 % {
  \catcode125=2 % }
  \catcode64=11 % @
  \def\x{\endgroup
    \expandafter\edef\csname rc@AtEnd\endcsname{%
      \endlinechar=\the\endlinechar\relax
      \catcode13=\the\catcode13\relax
      \catcode32=\the\catcode32\relax
      \catcode35=\the\catcode35\relax
      \catcode61=\the\catcode61\relax
      \catcode64=\the\catcode64\relax
      \catcode123=\the\catcode123\relax
      \catcode125=\the\catcode125\relax
    }%
  }%
\x\catcode61\catcode48\catcode32=10\relax%
\catcode13=5 % ^^M
\endlinechar=13 %
\catcode35=6 % #
\catcode64=11 % @
\catcode123=1 % {
\catcode125=2 % }
\def\TMP@EnsureCode#1#2{%
  \edef\rc@AtEnd{%
    \rc@AtEnd
    \catcode#1=\the\catcode#1\relax
  }%
  \catcode#1=#2\relax
}
\TMP@EnsureCode{33}{12}% !
\TMP@EnsureCode{39}{12}% '
\TMP@EnsureCode{42}{12}% *
\TMP@EnsureCode{45}{12}% -
\TMP@EnsureCode{46}{12}% .
\TMP@EnsureCode{47}{12}% /
\TMP@EnsureCode{91}{12}% [
\TMP@EnsureCode{93}{12}% ]
\TMP@EnsureCode{96}{12}% `
\edef\rc@AtEnd{\rc@AtEnd\noexpand\endinput}
%    \end{macrocode}
%
% \subsection{Loading packages}
%
%    \begin{macrocode}
\begingroup\expandafter\expandafter\expandafter\endgroup
\expandafter\ifx\csname RequirePackage\endcsname\relax
  \input ltxcmds.sty\relax
  \input infwarerr.sty\relax
\else
  \RequirePackage{ltxcmds}[2011/11/09]%
  \RequirePackage{infwarerr}[2010/04/08]%
\fi
%    \end{macrocode}
%
% \subsection{Defining commands}
%
%    \begin{macro}{\rc@IfDefinable}
%    \begin{macrocode}
\ltx@IfUndefined{@ifdefinable}{%
  \def\rc@IfDefinable#1{%
    \ifx#1\ltx@undefined
      \expandafter\ltx@firstofone
    \else
      \ifx#1\relax
        \expandafter\expandafter\expandafter\ltx@firstofone
      \else
        \@PackageError{refcount}{%
          Command \string#1 is already defined.\MessageBreak
          It will not redefined by this package%
        }\@ehc
        \expandafter\expandafter\expandafter\ltx@gobble
      \fi
    \fi
  }%
}{%
  \let\rc@IfDefinable\@ifdefinable
}
%    \end{macrocode}
%    \end{macro}
%
%    \begin{macro}{\rc@RobustDefOne}
%    \begin{macro}{\rc@RobustDefZero}
%    \begin{macrocode}
\ltx@IfUndefined{protected}{%
  \ltx@IfUndefined{DeclareRobustCommand}{%
    \def\rc@RobustDefOne#1#2#3#4{%
      \rc@IfDefinable#3{%
        #1\def#3##1{#4}%
      }%
    }%
    \def\rc@RobustDefZero#1#2{%
      \rc@IfDefinable#1{%
        \def#1{#2}%
      }%
    }%
  }{%
    \def\rc@RobustDefOne#1#2#3#4{%
      \rc@IfDefinable#3{%
        \DeclareRobustCommand#2#3[1]{#4}%
      }%
    }%
    \def\rc@RobustDefZero#1#2{%
      \rc@IfDefinable#1{%
        \DeclareRobustCommand#1{#2}%
      }%
    }%
  }%
}{%
  \def\rc@RobustDefOne#1#2#3#4{%
    \rc@IfDefinable#3{%
      \protected#1\def#3##1{#4}%
    }%
  }%
  \def\rc@RobustDefZero#1#2{%
    \rc@IfDefinable#1{%
      \protected\def#1{#2}%
    }%
  }%
}
%    \end{macrocode}
%    \end{macro}
%    \end{macro}
%
%    \begin{macro}{\rc@newcommand}
%    \begin{macrocode}
\ltx@IfUndefined{newcommand}{%
  \def\rc@newcommand*#1[#2]#3{% hash-ok
    \rc@IfDefinable#1{%
      \ifcase#2 %
        \def#1{#3}%
      \or
        \def#1##1{#3}%
      \or
        \def#1##1##2{#3}%
      \else
        \rc@InternalError
      \fi
    }%
  }%
}{%
  \let\rc@newcommand\newcommand
}
%    \end{macrocode}
%    \end{macro}
%
% \subsection{\cs{setrefcountdefault}}
%
%    \begin{macro}{\setrefcountdefault}
%    \begin{macrocode}
\rc@RobustDefOne\long{}\setrefcountdefault{%
  \def\rc@default{#1}%
}
%    \end{macrocode}
%    \end{macro}
%    \begin{macrocode}
\setrefcountdefault{0}
%    \end{macrocode}
%
% \subsection{\cs{refused}}
%
%    \begin{macro}{\refused}
%    \begin{macrocode}
\ltx@IfUndefined{G@refundefinedtrue}{%
  \rc@RobustDefOne{}{*}\refused{%
    \begingroup
      \csname @safe@activestrue\endcsname
      \ltx@IfUndefined{r@#1}{%
        \protect\G@refundefinedtrue
        \rc@WarningUndefined{#1}%
      }{}%
    \endgroup
  }%
}{%
  \rc@RobustDefOne{}{*}\refused{%
    \begingroup
      \csname @safe@activestrue\endcsname
      \ltx@IfUndefined{r@#1}{%
        \csname protect\expandafter\endcsname
        \csname G@refundefinedtrue\endcsname
        \rc@WarningUndefined{#1}%
      }{}%
    \endgroup
  }%
}
%    \end{macrocode}
%    \end{macro}
%    \begin{macro}{\rc@WarningUndefined}
%    \begin{macrocode}
\ltx@IfUndefined{@latex@warning}{%
  \def\rc@WarningUndefined#1{%
    \ltx@ifundefined{thepage}{%
      \def\thepage{\number\count0 }%
    }{}%
    \@PackageWarning{refcount}{%
      Reference `#1' on page \thepage\space undefined%
    }%
  }%
}{%
  \def\rc@WarningUndefined#1{%
    \@latex@warning{%
      Reference `#1' on page \thepage\space undefined%
    }%
  }%
}
%    \end{macrocode}
%    \end{macro}
%
% \subsection{Setting counters by reference data}
%
% \subsubsection{Generic setting}
%
%    \begin{macro}{\rc@set}
% Generic command for
% |\|$\{$|set|$,$|addto|$\}$|counter|$\{$|page|$,\}$|ref|:
%\begin{quote}
%\begin{tabular}[t]{@{}l@{: }l@{}}
% |#1|& \cs{setcounter}, \cs{addtocounter}\\
% |#2|& \cs{ltx@car} (for \cs{ref}), \cs{ltx@cartwo} (for \cs{pageref})\\
% |#3|& \hologo{LaTeX} counter\\
% |#4|& reference\\
%\end{tabular}
%\end{quote}
%    \begin{macrocode}
\def\rc@set#1#2#3#4{%
  \begingroup
    \csname @safe@activestrue\endcsname
    \refused{#4}%
    \expandafter\rc@@set\csname r@#4\endcsname{#1}{#2}{#3}%
  \endgroup
}
%    \end{macrocode}
%    \end{macro}
%    \begin{macro}{\rc@@set}
%\begin{quote}
%\begin{tabular}[t]{@{}l@{: }l@{}}
% |#1|& \cs{r@<...>}\\
% |#2|& \cs{setcounter}, \cs{addtocounter}\\
% |#3|& \cs{ltx@car} (for \cs{ref}), \cs{ltx@carsecond} (for \cs{pageref})\\
% |#4|& \hologo{LaTeX} counter\\
%\end{tabular}
%\end{quote}
%    \begin{macrocode}
\def\rc@@set#1#2#3#4{%
  \ifx#1\relax
    #2{#4}{\rc@default}%
  \else
    #2{#4}{%
      \expandafter#3#1\rc@default\rc@default\@nil
    }%
  \fi
}
%    \end{macrocode}
%    \end{macro}
%
% \subsubsection{User commands}
%
%    \begin{macro}{\setcounterref}
%    \begin{macrocode}
\rc@RobustDefZero\setcounterref{%
  \rc@set\setcounter\ltx@car
}
%    \end{macrocode}
%    \end{macro}
%    \begin{macro}{\addtocounterref}
%    \begin{macrocode}
\rc@RobustDefZero\addtocounterref{%
  \rc@set\addtocounter\ltx@car
}
%    \end{macrocode}
%    \end{macro}
%    \begin{macro}{\setcounterpageref}
%    \begin{macrocode}
\rc@RobustDefZero\setcounterpageref{%
  \rc@set\setcounter\ltx@carsecond
}
%    \end{macrocode}
%    \end{macro}
%    \begin{macro}{\addtocounterpageref}
%    \begin{macrocode}
\rc@RobustDefZero\addtocounterpageref{%
  \rc@set\addtocounter\ltx@carsecond
}
%    \end{macrocode}
%    \end{macro}
%
% \subsection{Extracting references}
%
%    \begin{macro}{\getrefnumber}
%    \begin{macrocode}
\rc@newcommand*{\getrefnumber}[1]{%
  \romannumeral
  \ltx@ifundefined{r@#1}{%
    \expandafter\ltx@zero
    \rc@default
  }{%
    \expandafter\expandafter\expandafter\rc@extract@
    \expandafter\expandafter\expandafter!%
    \csname r@#1\expandafter\endcsname
    \expandafter{\rc@default}\@nil
  }%
}
%    \end{macrocode}
%    \end{macro}
%    \begin{macro}{\getpagerefnumber}
%    \begin{macrocode}
\rc@newcommand*{\getpagerefnumber}[1]{%
  \romannumeral
  \ltx@ifundefined{r@#1}{%
    \expandafter\ltx@zero
    \rc@default
  }{%
    \expandafter\expandafter\expandafter\rc@extract@page
    \expandafter\expandafter\expandafter!%
    \csname r@#1\expandafter\expandafter\expandafter\endcsname
    \expandafter\expandafter\expandafter{%
      \expandafter\rc@default
    \expandafter}\expandafter{\rc@default}\@nil
  }%
}
%    \end{macrocode}
%    \end{macro}
%    \begin{macro}{\getrefbykeydefault}
%    \begin{macrocode}
\rc@newcommand*{\getrefbykeydefault}[2]{%
  \romannumeral
  \expandafter\rc@getrefbykeydefault
    \csname r@#1\expandafter\endcsname
    \csname rc@extract@#2\endcsname
}
%    \end{macrocode}
%    \end{macro}
%    \begin{macro}{\rc@getrefbykeydefault}
%\begin{quote}
%\begin{tabular}[t]{@{}l@{: }l@{}}
% |#1|& \cs{r@<...>}\\
% |#2|& \cs{rc@extract@<...>}\\
% |#3|& default\\
%\end{tabular}
%\end{quote}
%    \begin{macrocode}
\long\def\rc@getrefbykeydefault#1#2#3{%
  \ifx#1\relax
    % reference is undefined
    \ltx@ReturnAfterElseFi{%
      \ltx@zero
      #3%
    }%
  \else
    \ltx@ReturnAfterFi{%
      \ifx#2\relax
        % extract method is missing
        \ltx@ReturnAfterElseFi{%
          \ltx@zero
          #3%
        }%
      \else
        \ltx@ReturnAfterFi{%
          \expandafter
          \rc@generic#1{#3}{#3}{#3}{#3}{#3}\@nil#2{#3}%
        }%
      \fi
    }%
  \fi
}
%    \end{macrocode}
%    \end{macro}
%    \begin{macro}{\rc@generic}
%\begin{quote}
%\begin{tabular}[t]{@{}l@{: }l@{}}
% |#1|& first item in \cs{r@<...>}\\
% |#2|& remaining items in \cs{r@<...>}\\
% |#3|& \cs{rc@extract@<...>}\\
% |#4|& default\\
%\end{tabular}
%\end{quote}
%    \begin{macrocode}
\long\def\rc@generic#1#2\@nil#3#4{%
  #3{#1\TR@TitleReference\@empty{#4}\@nil}{#1}#2\@nil
}
%    \end{macrocode}
%    \end{macro}
%    \begin{macro}{\rc@extract@}
%    \begin{macrocode}
\long\def\rc@extract@#1#2#3\@nil{%
  \ltx@zero
  #2%
}
%    \end{macrocode}
%    \end{macro}
%    \begin{macro}{\rc@extract@page}
%    \begin{macrocode}
\long\def\rc@extract@page#1#2#3#4\@nil{%
  \ltx@zero
  #3%
}
%    \end{macrocode}
%    \end{macro}
%    \begin{macro}{\rc@extract@name}
%    \begin{macrocode}
\long\def\rc@extract@name#1#2#3#4#5\@nil{%
  \ltx@zero
  #4%
}
%    \end{macrocode}
%    \end{macro}
%    \begin{macro}{\rc@extract@anchor}
%    \begin{macrocode}
\long\def\rc@extract@anchor#1#2#3#4#5#6\@nil{%
  \ltx@zero
  #5%
}
%    \end{macrocode}
%    \end{macro}
%    \begin{macro}{\rc@extract@url}
%    \begin{macrocode}
\long\def\rc@extract@url#1#2#3#4#5#6#7\@nil{%
  \ltx@zero
  #6%
}
%    \end{macrocode}
%    \end{macro}
%    \begin{macro}{\rc@extract@title}
%    \begin{macrocode}
\long\def\rc@extract@title#1#2\@nil{%
  \rc@@extract@title#1%
}
%    \end{macrocode}
%    \end{macro}
%    \begin{macro}{\rc@@extract@title}
%    \begin{macrocode}
\long\def\rc@@extract@title#1\TR@TitleReference#2#3#4\@nil{%
  \ltx@zero
  #3%
}
%    \end{macrocode}
%    \end{macro}
%
% \subsection{Macros for checking undefined references}
%
%    \begin{macro}{\IfRefUndefinedExpandable}
%    \begin{macrocode}
\rc@newcommand*{\IfRefUndefinedExpandable}[1]{%
  \ltx@ifundefined{r@#1}\ltx@firstoftwo\ltx@secondoftwo
}
%    \end{macrocode}
%    \end{macro}
%    \begin{macro}{\IfRefUndefinedBabel}
%    \begin{macrocode}
\rc@RobustDefOne{}*\IfRefUndefinedBabel{%
  \begingroup
    \csname safe@actives@true\endcsname
  \expandafter\expandafter\expandafter\endgroup
  \expandafter\ifx\csname r@#1\endcsname\relax
    \expandafter\ltx@firstoftwo
  \else
    \expandafter\ltx@secondoftwo
  \fi
}
%    \end{macrocode}
%    \end{macro}
%    \begin{macrocode}
\rc@AtEnd%
%</package>
%    \end{macrocode}
%
% \section{Test}
%
% \subsection{Catcode checks for loading}
%
%    \begin{macrocode}
%<*test1>
%    \end{macrocode}
%    \begin{macrocode}
\catcode`\{=1 %
\catcode`\}=2 %
\catcode`\#=6 %
\catcode`\@=11 %
\expandafter\ifx\csname count@\endcsname\relax
  \countdef\count@=255 %
\fi
\expandafter\ifx\csname @gobble\endcsname\relax
  \long\def\@gobble#1{}%
\fi
\expandafter\ifx\csname @firstofone\endcsname\relax
  \long\def\@firstofone#1{#1}%
\fi
\expandafter\ifx\csname loop\endcsname\relax
  \expandafter\@firstofone
\else
  \expandafter\@gobble
\fi
{%
  \def\loop#1\repeat{%
    \def\body{#1}%
    \iterate
  }%
  \def\iterate{%
    \body
      \let\next\iterate
    \else
      \let\next\relax
    \fi
    \next
  }%
  \let\repeat=\fi
}%
\def\RestoreCatcodes{}
\count@=0 %
\loop
  \edef\RestoreCatcodes{%
    \RestoreCatcodes
    \catcode\the\count@=\the\catcode\count@\relax
  }%
\ifnum\count@<255 %
  \advance\count@ 1 %
\repeat

\def\RangeCatcodeInvalid#1#2{%
  \count@=#1\relax
  \loop
    \catcode\count@=15 %
  \ifnum\count@<#2\relax
    \advance\count@ 1 %
  \repeat
}
\def\RangeCatcodeCheck#1#2#3{%
  \count@=#1\relax
  \loop
    \ifnum#3=\catcode\count@
    \else
      \errmessage{%
        Character \the\count@\space
        with wrong catcode \the\catcode\count@\space
        instead of \number#3%
      }%
    \fi
  \ifnum\count@<#2\relax
    \advance\count@ 1 %
  \repeat
}
\def\space{ }
\expandafter\ifx\csname LoadCommand\endcsname\relax
  \def\LoadCommand{\input refcount.sty\relax}%
\fi
\def\Test{%
  \RangeCatcodeInvalid{0}{47}%
  \RangeCatcodeInvalid{58}{64}%
  \RangeCatcodeInvalid{91}{96}%
  \RangeCatcodeInvalid{123}{255}%
  \catcode`\@=12 %
  \catcode`\\=0 %
  \catcode`\%=14 %
  \LoadCommand
  \RangeCatcodeCheck{0}{36}{15}%
  \RangeCatcodeCheck{37}{37}{14}%
  \RangeCatcodeCheck{38}{47}{15}%
  \RangeCatcodeCheck{48}{57}{12}%
  \RangeCatcodeCheck{58}{63}{15}%
  \RangeCatcodeCheck{64}{64}{12}%
  \RangeCatcodeCheck{65}{90}{11}%
  \RangeCatcodeCheck{91}{91}{15}%
  \RangeCatcodeCheck{92}{92}{0}%
  \RangeCatcodeCheck{93}{96}{15}%
  \RangeCatcodeCheck{97}{122}{11}%
  \RangeCatcodeCheck{123}{255}{15}%
  \RestoreCatcodes
}
\Test
\csname @@end\endcsname
\end
%    \end{macrocode}
%    \begin{macrocode}
%</test1>
%    \end{macrocode}
%
% \subsection{Macro tests}
%
%    \begin{macrocode}
%<*test2>
\errorcontextlines=10000 %
\showboxbreadth=10000 %
\showboxdepth=10000 %
\begingroup\expandafter\expandafter\expandafter\endgroup
\expandafter\ifx\csname RequirePackage\endcsname\relax
  \input refcount.sty\relax
\else
  \RequirePackage{refcount}[2011/10/16]%
\fi
\catcode`\@=11 %
\begingroup\expandafter\expandafter\expandafter\endgroup
\expandafter\ifx\csname @onelevel@sanitize\endcsname\relax
  \begingroup\expandafter\expandafter\expandafter\endgroup
  \expandafter\ifx\csname detokenize\endcsname\relax
    \def\strip@prefix#1->{}%
    \def\@onelevel@sanitize#1{%
      \edef#1{%
        \expandafter\strip@prefix\meaning#1%
      }%
    }%
  \else
    \def\@onelevel@sanitize#1{%
      \edef#1{%
        \detokenize\expandafter{#1}%
      }%
    }%
  \fi
\fi
\def\msg#{\immediate\write16}
\def\empty{}
\def\space{ }
%    \end{macrocode}
%    \begin{macrocode}
\def\r@foo{{\empty 1}{\empty 2}}
\long\def\test#1#2{%
  \begingroup
    \setbox0=\hbox{%
      \def\TestTask{#1}%
      \@onelevel@sanitize\TestTask
      \msg{* \TestTask}%
      \expandafter\expandafter\expandafter\def
      \expandafter\expandafter\expandafter\TestResult
      \expandafter\expandafter\expandafter{%
        #1%
      }%
      \def\TestExpected{#2}%
      \ifx\TestResult\TestExpected
        \msg{ \space ok.}%
      \else
        \@onelevel@sanitize\TestResult
        \@onelevel@sanitize\TestExpected
        \msg{ \space Result: \space\space[\TestResult]}%
        \msg{ \space Expected: [\TestExpected]}%
        \errmessage{Test failed!}%
      \fi
    }%
    \ifdim\wd0=0pt %
    \else
      \showbox0 %
    \fi
  \endgroup
}
\test{\getrefnumber{foo}}{\empty 1}
\test{\getpagerefnumber{foo}}{\empty 2}
\test{\getrefbykeydefault{foo}{}{\empty default}}{\empty 1}
\test{\getrefbykeydefault{foo}{page}{\empty default}}{\empty 2}
\test{\getrefbykeydefault{foo}{name}{\empty default}}{\empty default}
\test{\getrefbykeydefault{foo}{anchor}{\empty default}}{\empty default}
\test{\getrefbykeydefault{foo}{url}{\empty default}}{\empty default}
\test{\getrefbykeydefault{foo}{title}{\empty default}}{\empty default}
\msg{}
\def\r@foo{{}{}{}{}{}{}{}{}{}{}}
\def\Test#1#2\\{%
  \test{#1{foo}#2}{}%
}
\def\TestGroup{%
  \Test\getrefnumber\\%
  \Test\getpagerefnumber\\%
  \Test\getrefbykeydefault{}{}\\%
  \Test\getrefbykeydefault{page}{}\\%
  \Test\getrefbykeydefault{anchor}{}\\%
  \Test\getrefbykeydefault{name}{}\\%
  \Test\getrefbykeydefault{url}{}\\%
}
\TestGroup
\Test\getrefbykeydefault{title}{}\\%
\msg{}
\def\r@foo{\par\par\par\par\par\par\par\par}
\long\def\Test#1#2\\{%
  \test{#1{foo}#2}{\par}%
}
\TestGroup
\test{\getrefbykeydefault{title}{}{}}{}
\msg{}
\def\r@foo{{ }{ }{ }{ }{ }}
\def\Test#1#2\\{%
  \test{#1{foo}#2}{ }%
}
\TestGroup
\msg{}
\long\def\TestDefault#1{%
  \begingroup
    \setrefcountdefault{#1}%
    \test{\getrefnumber{foo}}{#1}%
    \test{\getpagerefnumber{foo}}{#1}%
  \endgroup
}
\def\TestDefaultX{%
  \TestDefault{}%
  \TestDefault{\par}%
  \TestDefault{ }%
  \TestDefault{\space}%
}
\let\r@foo\@undefined
\TestDefaultX
\let\r@foo\relax
\TestDefaultX
\def\r@foo{}
\TestDefaultX
%    \end{macrocode}
%    \begin{macrocode}
\msg{}
\long\def\Test#1#2#3#4{%
  \begingroup
    \def\TestTask{#1}%
    \@onelevel@sanitize\TestTask
    \msg{* [\TestTask]}%
    \edef\TestResultA{\IfRefUndefinedExpandable{#1}{#2}{#3}}%
    \IfRefUndefinedBabel{#1}{%
      \def\TestResultB{#2}%
    }{%
      \def\TestResultB{#3}%
    }%
    \def\TestExpected{#4}%
    \ifx\TestResultA\TestExpected
      \msg{ \space ok.}%
    \else
      \begingroup
        \@onelevel@sanitize\TestResultA
        \@onelevel@sanitize\TestExpected
        \msg{ \space Result: \space\space[\TestResultA]}%
        \msg{ \space Expected: [\TestExpected]}%
        \errmessage{Test failed!}%
      \endgroup
    \fi
    \ifx\TestResultB\TestExpected
      \msg{ \space ok.}%
    \else
      \begingroup
        \@onelevel@sanitize\TestResultB
        \@onelevel@sanitize\TestExpected
        \msg{ \space Result: \space\space[\TestResultB]}%
        \msg{ \space Expected: [\TestExpected]}%
        \errmessage{Test failed!}%
      \endgroup
    \fi
  \endgroup
}
\begingroup
  \def\r@foo{{}{}}%
  \let\r@bar\@undefined
  \let\r@xyz\relax
  \Test{foo}{true}{false}{false}%
  \Test{bar}{true}{false}{true}%
  \Test{xyz}{true}{false}{true}%
\endgroup
%    \end{macrocode}
%    \begin{macrocode}
\csname @@end\endcsname\end
%</test2>
%    \end{macrocode}
% \subsection{Test with package \xpackage{titleref}}
%
%    \begin{macrocode}
%<*test3>
\NeedsTeXFormat{LaTeX2e}
\documentclass{article}
\usepackage{refcount}[2011/10/16]
%<test4>\usepackage{nameref}
\usepackage{titleref}
\begin{document}
\section{Hello World}
\label{sec:hello}
\section{\hbox{xy}}
\label{sec:foo}
%
\makeatletter
\@ifundefined{r@sec:hello}{%
  \typeout{==> Compile twice!}%
}{%
  \def\test#1#2{%
    \begingroup
      \def\TestTask{#1}%
      \@onelevel@sanitize\TestTask
      \typeout{* \TestTask}%
      \expandafter\expandafter\expandafter\def
      \expandafter\expandafter\expandafter\TestResult
      \expandafter\expandafter\expandafter{%
        #1%
      }%
      \def\TestExpected{#2}%
      \ifx\TestResult\TestExpected
        \typeout{ \space ok.}%
      \else
        \@onelevel@sanitize\TestResult
        \@onelevel@sanitize\TestExpected
        \typeout{ \space Result: \space\space[\TestResult]}%
        \typeout{ \space Expected: [\TestExpected]}%
        \errmessage{Test failed!}%
      \fi
    \endgroup
  }%
  \test{\getrefbykeydefault{sec:hello}{title}{}}{Hello World}%
  \test{\getrefbykeydefault{sec:foo}{title}{}}{\hbox{xy}}%
  \begingroup
    \def\hbox#1{[#1]}% hash-ok
    \test{\getrefbykeydefault{sec:foo}{title}{}}{\hbox{xy}}%
  \endgroup
}
\makeatother
%    \end{macrocode}
%    \begin{macrocode}
\end{document}
%</test3>
%    \end{macrocode}
%    \begin{macrocode}
%<*test5>
\NeedsTeXFormat{LaTeX2e}
\documentclass{book}
\usepackage{refcount}[2011/10/16]
\usepackage{zref-runs}
\newcounter{test}
\begin{document}
\ifnum\zruns>1 %
  \makeatletter
  \def\Test#1#2#3{%
    \begingroup
      \setcounter{test}{10}%
      \sbox0{%
        #1{test}{#2}%
        \ifnum#3=\value{test}%
        \else
          \PackageError{test}{\string#1{#2} <> #3 (\the\value{test})}%
        \fi
      }%
      \ifdim\wd0=0pt %
      \else
        \PackageError{test}{Non-empty box}\@ehc
      \fi
    \endgroup
  }%
  \makeatother
  \Test\setcounterpageref{ch:two}{1}%
  \Test\setcounterpageref{ch:three}{3}%
  \Test\setcounterpageref{ch:four}{5}%
  \Test\setcounterpageref{ch:five}{7}%
  \Test\setcounterpageref{ch:six}{9}%
  \Test\setcounterpageref{ch:seven}{13}%
  \Test\addtocounterpageref{ch:two}{11}%
  \Test\addtocounterpageref{ch:three}{13}%
  \Test\addtocounterpageref{ch:four}{15}%
  \Test\addtocounterpageref{ch:five}{17}%
  \Test\addtocounterpageref{ch:six}{19}%
  \Test\addtocounterpageref{ch:seven}{23}%
  \Test\setcounterref{ch:two}{1}%
  \Test\setcounterref{ch:three}{2}%
  \Test\setcounterref{ch:four}{11}%
  \Test\addtocounterref{ch:two}{11}%
  \Test\addtocounterref{ch:three}{12}%
  \Test\addtocounterref{ch:four}{21}%
\fi
\frontmatter
\chapter{Chapter one}\label{ch:one}
\cleardoublepage
\mainmatter
\chapter{Chapter two}\label{ch:two}
\cleardoublepage
\chapter{Chapter three}\label{ch:three}
\cleardoublepage
\setcounter{chapter}{10}
\chapter{Chapter four}\label{ch:four}
\cleardoublepage
\appendix
\chapter{Chapter five}\label{ch:five}
\cleardoublepage
\chapter{Chapter six}\label{ch:six}
\cleardoublepage
\null
\cleardoublepage
\chapter{Chapter seven}\label{ch:seven}
\end{document}
%</test5>
%    \end{macrocode}
%
% \section{Installation}
%
% \subsection{Download}
%
% \paragraph{Package.} This package is available on
% CTAN\footnote{\url{ftp://ftp.ctan.org/tex-archive/}}:
% \begin{description}
% \item[\CTAN{macros/latex/contrib/oberdiek/refcount.dtx}] The source file.
% \item[\CTAN{macros/latex/contrib/oberdiek/refcount.pdf}] Documentation.
% \end{description}
%
%
% \paragraph{Bundle.} All the packages of the bundle `oberdiek'
% are also available in a TDS compliant ZIP archive. There
% the packages are already unpacked and the documentation files
% are generated. The files and directories obey the TDS standard.
% \begin{description}
% \item[\CTAN{install/macros/latex/contrib/oberdiek.tds.zip}]
% \end{description}
% \emph{TDS} refers to the standard ``A Directory Structure
% for \TeX\ Files'' (\CTAN{tds/tds.pdf}). Directories
% with \xfile{texmf} in their name are usually organized this way.
%
% \subsection{Bundle installation}
%
% \paragraph{Unpacking.} Unpack the \xfile{oberdiek.tds.zip} in the
% TDS tree (also known as \xfile{texmf} tree) of your choice.
% Example (linux):
% \begin{quote}
%   |unzip oberdiek.tds.zip -d ~/texmf|
% \end{quote}
%
% \paragraph{Script installation.}
% Check the directory \xfile{TDS:scripts/oberdiek/} for
% scripts that need further installation steps.
% Package \xpackage{attachfile2} comes with the Perl script
% \xfile{pdfatfi.pl} that should be installed in such a way
% that it can be called as \texttt{pdfatfi}.
% Example (linux):
% \begin{quote}
%   |chmod +x scripts/oberdiek/pdfatfi.pl|\\
%   |cp scripts/oberdiek/pdfatfi.pl /usr/local/bin/|
% \end{quote}
%
% \subsection{Package installation}
%
% \paragraph{Unpacking.} The \xfile{.dtx} file is a self-extracting
% \docstrip\ archive. The files are extracted by running the
% \xfile{.dtx} through \plainTeX:
% \begin{quote}
%   \verb|tex refcount.dtx|
% \end{quote}
%
% \paragraph{TDS.} Now the different files must be moved into
% the different directories in your installation TDS tree
% (also known as \xfile{texmf} tree):
% \begin{quote}
% \def\t{^^A
% \begin{tabular}{@{}>{\ttfamily}l@{ $\rightarrow$ }>{\ttfamily}l@{}}
%   refcount.sty & tex/latex/oberdiek/refcount.sty\\
%   refcount.pdf & doc/latex/oberdiek/refcount.pdf\\
%   test/refcount-test1.tex & doc/latex/oberdiek/test/refcount-test1.tex\\
%   test/refcount-test2.tex & doc/latex/oberdiek/test/refcount-test2.tex\\
%   test/refcount-test3.tex & doc/latex/oberdiek/test/refcount-test3.tex\\
%   test/refcount-test4.tex & doc/latex/oberdiek/test/refcount-test4.tex\\
%   test/refcount-test5.tex & doc/latex/oberdiek/test/refcount-test5.tex\\
%   refcount.dtx & source/latex/oberdiek/refcount.dtx\\
% \end{tabular}^^A
% }^^A
% \sbox0{\t}^^A
% \ifdim\wd0>\linewidth
%   \begingroup
%     \advance\linewidth by\leftmargin
%     \advance\linewidth by\rightmargin
%   \edef\x{\endgroup
%     \def\noexpand\lw{\the\linewidth}^^A
%   }\x
%   \def\lwbox{^^A
%     \leavevmode
%     \hbox to \linewidth{^^A
%       \kern-\leftmargin\relax
%       \hss
%       \usebox0
%       \hss
%       \kern-\rightmargin\relax
%     }^^A
%   }^^A
%   \ifdim\wd0>\lw
%     \sbox0{\small\t}^^A
%     \ifdim\wd0>\linewidth
%       \ifdim\wd0>\lw
%         \sbox0{\footnotesize\t}^^A
%         \ifdim\wd0>\linewidth
%           \ifdim\wd0>\lw
%             \sbox0{\scriptsize\t}^^A
%             \ifdim\wd0>\linewidth
%               \ifdim\wd0>\lw
%                 \sbox0{\tiny\t}^^A
%                 \ifdim\wd0>\linewidth
%                   \lwbox
%                 \else
%                   \usebox0
%                 \fi
%               \else
%                 \lwbox
%               \fi
%             \else
%               \usebox0
%             \fi
%           \else
%             \lwbox
%           \fi
%         \else
%           \usebox0
%         \fi
%       \else
%         \lwbox
%       \fi
%     \else
%       \usebox0
%     \fi
%   \else
%     \lwbox
%   \fi
% \else
%   \usebox0
% \fi
% \end{quote}
% If you have a \xfile{docstrip.cfg} that configures and enables \docstrip's
% TDS installing feature, then some files can already be in the right
% place, see the documentation of \docstrip.
%
% \subsection{Refresh file name databases}
%
% If your \TeX~distribution
% (\teTeX, \mikTeX, \dots) relies on file name databases, you must refresh
% these. For example, \teTeX\ users run \verb|texhash| or
% \verb|mktexlsr|.
%
% \subsection{Some details for the interested}
%
% \paragraph{Attached source.}
%
% The PDF documentation on CTAN also includes the
% \xfile{.dtx} source file. It can be extracted by
% AcrobatReader 6 or higher. Another option is \textsf{pdftk},
% e.g. unpack the file into the current directory:
% \begin{quote}
%   \verb|pdftk refcount.pdf unpack_files output .|
% \end{quote}
%
% \paragraph{Unpacking with \LaTeX.}
% The \xfile{.dtx} chooses its action depending on the format:
% \begin{description}
% \item[\plainTeX:] Run \docstrip\ and extract the files.
% \item[\LaTeX:] Generate the documentation.
% \end{description}
% If you insist on using \LaTeX\ for \docstrip\ (really,
% \docstrip\ does not need \LaTeX), then inform the autodetect routine
% about your intention:
% \begin{quote}
%   \verb|latex \let\install=y\input{refcount.dtx}|
% \end{quote}
% Do not forget to quote the argument according to the demands
% of your shell.
%
% \paragraph{Generating the documentation.}
% You can use both the \xfile{.dtx} or the \xfile{.drv} to generate
% the documentation. The process can be configured by the
% configuration file \xfile{ltxdoc.cfg}. For instance, put this
% line into this file, if you want to have A4 as paper format:
% \begin{quote}
%   \verb|\PassOptionsToClass{a4paper}{article}|
% \end{quote}
% An example follows how to generate the
% documentation with pdf\LaTeX:
% \begin{quote}
%\begin{verbatim}
%pdflatex refcount.dtx
%makeindex -s gind.ist refcount.idx
%pdflatex refcount.dtx
%makeindex -s gind.ist refcount.idx
%pdflatex refcount.dtx
%\end{verbatim}
% \end{quote}
%
% \section{Catalogue}
%
% The following XML file can be used as source for the
% \href{http://mirror.ctan.org/help/Catalogue/catalogue.html}{\TeX\ Catalogue}.
% The elements \texttt{caption} and \texttt{description} are imported
% from the original XML file from the Catalogue.
% The name of the XML file in the Catalogue is \xfile{refcount.xml}.
%    \begin{macrocode}
%<*catalogue>
<?xml version='1.0' encoding='us-ascii'?>
<!DOCTYPE entry SYSTEM 'catalogue.dtd'>
<entry datestamp='$Date$' modifier='$Author$' id='refcount'>
  <name>refcount</name>
  <caption>Counter operations with label references.</caption>
  <authorref id='auth:oberdiek'/>
  <copyright owner='Heiko Oberdiek' year='1998,2000,2006,2008,2010,2011'/>
  <license type='lppl1.3'/>
  <version number='3.4'/>
  <description>
    Provides commands <tt>\setcounterref</tt> and
    <tt>\addtocounterref</tt> which use the section (or whatever)
    number from the reference as the value to put into the counter, as
    in:

    <pre>
    ...\label{sec:foo}
    ...
    \setcounterref{foonum}{sec:foo}
    </pre>
    Commands <tt>\setcounterpageref</tt> and
    <tt>\addtocounterpageref</tt> do the corresponding thing with the
    page reference of the label.
    <p/>
    No <tt>.ins</tt> file is distributed; process the
    <tt>.dtx</tt> with plain TeX to create one.
    <p/>
    The package is part of the <xref refid='oberdiek'>oberdiek</xref>
    bundle.
  </description>
  <documentation details='Package documentation'
      href='ctan:/macros/latex/contrib/oberdiek/refcount.pdf'/>
  <ctan file='true' path='/macros/latex/contrib/oberdiek/refcount.dtx'/>
  <miktex location='oberdiek'/>
  <texlive location='oberdiek'/>
  <install path='/macros/latex/contrib/oberdiek/oberdiek.tds.zip'/>
</entry>
%</catalogue>
%    \end{macrocode}
%
% \begin{History}
%   \begin{Version}{1998/04/08 v1.0}
%   \item
%     First public release, written as answer in the
%     newsgroup \xnewsgroup{comp.text.tex}:
%     \URL{``\link{Re: Adding a \cs{ref} to a counter?}''}^^A
%     {http://groups.google.com/group/comp.text.tex/msg/c3f2a135ef5ee528}
%   \end{Version}
%   \begin{Version}{2000/09/07 v2.0}
%   \item
%     Documentation added.
%   \item
%     LPPL 1.2
%   \item
%     Package rewritten, new commands added.
%   \end{Version}
%   \begin{Version}{2006/02/20 v3.0}
%   \item
%     Support for \xpackage{hyperref} and \xpackage{nameref} improved.
%   \item
%     Support for \xpackage{titleref} and \xpackage{babel}'s shorthands added.
%   \item
%     New: \cs{refused}, \cs{getrefbykeydefault}
%   \end{Version}
%   \begin{Version}{2008/08/11 v3.1}
%   \item
%     Code is not changed.
%   \item
%     URLs updated.
%   \end{Version}
%   \begin{Version}{2010/12/01 v3.2}
%   \item
%     \cs{IfRefUndefinedExpandable} and \cs{IfRefUndefinedBabel} added.
%   \item
%     \cs{getrefnumber}, \cs{getpagerefnumber}, \cs{getrefbykeydefault}
%     are expandable in exact two expansion steps.
%   \item
%     Non-expandable macros are made robust.
%   \item
%     Test files added.
%   \end{Version}
%   \begin{Version}{2011/06/22 v3.3}
%   \item
%     Bug fix: \cs{rc@refused} is undefined for \cs{setcounterpageref}
%     and similar macros. (Bug found by Marc van Dongen.)
%   \end{Version}
%   \begin{Version}{2011/10/16 v3.4}
%   \item
%     Bug fix: \cs{setcounterpageref} and \cs{addtocounterpageref} fixed.
%     (Bug found by Staz.)
%   \item
%     Macros \cs{(set|addto)counter(page|)ref} are made robust.
%   \end{Version}
% \end{History}
%
% \PrintIndex
%
% \Finale
\endinput
|
% \end{quote}
% Do not forget to quote the argument according to the demands
% of your shell.
%
% \paragraph{Generating the documentation.}
% You can use both the \xfile{.dtx} or the \xfile{.drv} to generate
% the documentation. The process can be configured by the
% configuration file \xfile{ltxdoc.cfg}. For instance, put this
% line into this file, if you want to have A4 as paper format:
% \begin{quote}
%   \verb|\PassOptionsToClass{a4paper}{article}|
% \end{quote}
% An example follows how to generate the
% documentation with pdf\LaTeX:
% \begin{quote}
%\begin{verbatim}
%pdflatex refcount.dtx
%makeindex -s gind.ist refcount.idx
%pdflatex refcount.dtx
%makeindex -s gind.ist refcount.idx
%pdflatex refcount.dtx
%\end{verbatim}
% \end{quote}
%
% \section{Catalogue}
%
% The following XML file can be used as source for the
% \href{http://mirror.ctan.org/help/Catalogue/catalogue.html}{\TeX\ Catalogue}.
% The elements \texttt{caption} and \texttt{description} are imported
% from the original XML file from the Catalogue.
% The name of the XML file in the Catalogue is \xfile{refcount.xml}.
%    \begin{macrocode}
%<*catalogue>
<?xml version='1.0' encoding='us-ascii'?>
<!DOCTYPE entry SYSTEM 'catalogue.dtd'>
<entry datestamp='$Date$' modifier='$Author$' id='refcount'>
  <name>refcount</name>
  <caption>Counter operations with label references.</caption>
  <authorref id='auth:oberdiek'/>
  <copyright owner='Heiko Oberdiek' year='1998,2000,2006,2008,2010,2011'/>
  <license type='lppl1.3'/>
  <version number='3.4'/>
  <description>
    Provides commands <tt>\setcounterref</tt> and
    <tt>\addtocounterref</tt> which use the section (or whatever)
    number from the reference as the value to put into the counter, as
    in:

    <pre>
    ...\label{sec:foo}
    ...
    \setcounterref{foonum}{sec:foo}
    </pre>
    Commands <tt>\setcounterpageref</tt> and
    <tt>\addtocounterpageref</tt> do the corresponding thing with the
    page reference of the label.
    <p/>
    No <tt>.ins</tt> file is distributed; process the
    <tt>.dtx</tt> with plain TeX to create one.
    <p/>
    The package is part of the <xref refid='oberdiek'>oberdiek</xref>
    bundle.
  </description>
  <documentation details='Package documentation'
      href='ctan:/macros/latex/contrib/oberdiek/refcount.pdf'/>
  <ctan file='true' path='/macros/latex/contrib/oberdiek/refcount.dtx'/>
  <miktex location='oberdiek'/>
  <texlive location='oberdiek'/>
  <install path='/macros/latex/contrib/oberdiek/oberdiek.tds.zip'/>
</entry>
%</catalogue>
%    \end{macrocode}
%
% \begin{History}
%   \begin{Version}{1998/04/08 v1.0}
%   \item
%     First public release, written as answer in the
%     newsgroup \xnewsgroup{comp.text.tex}:
%     \URL{``\link{Re: Adding a \cs{ref} to a counter?}''}^^A
%     {http://groups.google.com/group/comp.text.tex/msg/c3f2a135ef5ee528}
%   \end{Version}
%   \begin{Version}{2000/09/07 v2.0}
%   \item
%     Documentation added.
%   \item
%     LPPL 1.2
%   \item
%     Package rewritten, new commands added.
%   \end{Version}
%   \begin{Version}{2006/02/20 v3.0}
%   \item
%     Support for \xpackage{hyperref} and \xpackage{nameref} improved.
%   \item
%     Support for \xpackage{titleref} and \xpackage{babel}'s shorthands added.
%   \item
%     New: \cs{refused}, \cs{getrefbykeydefault}
%   \end{Version}
%   \begin{Version}{2008/08/11 v3.1}
%   \item
%     Code is not changed.
%   \item
%     URLs updated.
%   \end{Version}
%   \begin{Version}{2010/12/01 v3.2}
%   \item
%     \cs{IfRefUndefinedExpandable} and \cs{IfRefUndefinedBabel} added.
%   \item
%     \cs{getrefnumber}, \cs{getpagerefnumber}, \cs{getrefbykeydefault}
%     are expandable in exact two expansion steps.
%   \item
%     Non-expandable macros are made robust.
%   \item
%     Test files added.
%   \end{Version}
%   \begin{Version}{2011/06/22 v3.3}
%   \item
%     Bug fix: \cs{rc@refused} is undefined for \cs{setcounterpageref}
%     and similar macros. (Bug found by Marc van Dongen.)
%   \end{Version}
%   \begin{Version}{2011/10/16 v3.4}
%   \item
%     Bug fix: \cs{setcounterpageref} and \cs{addtocounterpageref} fixed.
%     (Bug found by Staz.)
%   \item
%     Macros \cs{(set|addto)counter(page|)ref} are made robust.
%   \end{Version}
% \end{History}
%
% \PrintIndex
%
% \Finale
\endinput
|
% \end{quote}
% Do not forget to quote the argument according to the demands
% of your shell.
%
% \paragraph{Generating the documentation.}
% You can use both the \xfile{.dtx} or the \xfile{.drv} to generate
% the documentation. The process can be configured by the
% configuration file \xfile{ltxdoc.cfg}. For instance, put this
% line into this file, if you want to have A4 as paper format:
% \begin{quote}
%   \verb|\PassOptionsToClass{a4paper}{article}|
% \end{quote}
% An example follows how to generate the
% documentation with pdf\LaTeX:
% \begin{quote}
%\begin{verbatim}
%pdflatex refcount.dtx
%makeindex -s gind.ist refcount.idx
%pdflatex refcount.dtx
%makeindex -s gind.ist refcount.idx
%pdflatex refcount.dtx
%\end{verbatim}
% \end{quote}
%
% \section{Catalogue}
%
% The following XML file can be used as source for the
% \href{http://mirror.ctan.org/help/Catalogue/catalogue.html}{\TeX\ Catalogue}.
% The elements \texttt{caption} and \texttt{description} are imported
% from the original XML file from the Catalogue.
% The name of the XML file in the Catalogue is \xfile{refcount.xml}.
%    \begin{macrocode}
%<*catalogue>
<?xml version='1.0' encoding='us-ascii'?>
<!DOCTYPE entry SYSTEM 'catalogue.dtd'>
<entry datestamp='$Date$' modifier='$Author$' id='refcount'>
  <name>refcount</name>
  <caption>Counter operations with label references.</caption>
  <authorref id='auth:oberdiek'/>
  <copyright owner='Heiko Oberdiek' year='1998,2000,2006,2008,2010,2011'/>
  <license type='lppl1.3'/>
  <version number='3.4'/>
  <description>
    Provides commands <tt>\setcounterref</tt> and
    <tt>\addtocounterref</tt> which use the section (or whatever)
    number from the reference as the value to put into the counter, as
    in:

    <pre>
    ...\label{sec:foo}
    ...
    \setcounterref{foonum}{sec:foo}
    </pre>
    Commands <tt>\setcounterpageref</tt> and
    <tt>\addtocounterpageref</tt> do the corresponding thing with the
    page reference of the label.
    <p/>
    No <tt>.ins</tt> file is distributed; process the
    <tt>.dtx</tt> with plain TeX to create one.
    <p/>
    The package is part of the <xref refid='oberdiek'>oberdiek</xref>
    bundle.
  </description>
  <documentation details='Package documentation'
      href='ctan:/macros/latex/contrib/oberdiek/refcount.pdf'/>
  <ctan file='true' path='/macros/latex/contrib/oberdiek/refcount.dtx'/>
  <miktex location='oberdiek'/>
  <texlive location='oberdiek'/>
  <install path='/macros/latex/contrib/oberdiek/oberdiek.tds.zip'/>
</entry>
%</catalogue>
%    \end{macrocode}
%
% \begin{History}
%   \begin{Version}{1998/04/08 v1.0}
%   \item
%     First public release, written as answer in the
%     newsgroup \xnewsgroup{comp.text.tex}:
%     \URL{``\link{Re: Adding a \cs{ref} to a counter?}''}^^A
%     {http://groups.google.com/group/comp.text.tex/msg/c3f2a135ef5ee528}
%   \end{Version}
%   \begin{Version}{2000/09/07 v2.0}
%   \item
%     Documentation added.
%   \item
%     LPPL 1.2
%   \item
%     Package rewritten, new commands added.
%   \end{Version}
%   \begin{Version}{2006/02/20 v3.0}
%   \item
%     Support for \xpackage{hyperref} and \xpackage{nameref} improved.
%   \item
%     Support for \xpackage{titleref} and \xpackage{babel}'s shorthands added.
%   \item
%     New: \cs{refused}, \cs{getrefbykeydefault}
%   \end{Version}
%   \begin{Version}{2008/08/11 v3.1}
%   \item
%     Code is not changed.
%   \item
%     URLs updated.
%   \end{Version}
%   \begin{Version}{2010/12/01 v3.2}
%   \item
%     \cs{IfRefUndefinedExpandable} and \cs{IfRefUndefinedBabel} added.
%   \item
%     \cs{getrefnumber}, \cs{getpagerefnumber}, \cs{getrefbykeydefault}
%     are expandable in exact two expansion steps.
%   \item
%     Non-expandable macros are made robust.
%   \item
%     Test files added.
%   \end{Version}
%   \begin{Version}{2011/06/22 v3.3}
%   \item
%     Bug fix: \cs{rc@refused} is undefined for \cs{setcounterpageref}
%     and similar macros. (Bug found by Marc van Dongen.)
%   \end{Version}
%   \begin{Version}{2011/10/16 v3.4}
%   \item
%     Bug fix: \cs{setcounterpageref} and \cs{addtocounterpageref} fixed.
%     (Bug found by Staz.)
%   \item
%     Macros \cs{(set|addto)counter(page|)ref} are made robust.
%   \end{Version}
% \end{History}
%
% \PrintIndex
%
% \Finale
\endinput
|
% \end{quote}
% Do not forget to quote the argument according to the demands
% of your shell.
%
% \paragraph{Generating the documentation.}
% You can use both the \xfile{.dtx} or the \xfile{.drv} to generate
% the documentation. The process can be configured by the
% configuration file \xfile{ltxdoc.cfg}. For instance, put this
% line into this file, if you want to have A4 as paper format:
% \begin{quote}
%   \verb|\PassOptionsToClass{a4paper}{article}|
% \end{quote}
% An example follows how to generate the
% documentation with pdf\LaTeX:
% \begin{quote}
%\begin{verbatim}
%pdflatex refcount.dtx
%makeindex -s gind.ist refcount.idx
%pdflatex refcount.dtx
%makeindex -s gind.ist refcount.idx
%pdflatex refcount.dtx
%\end{verbatim}
% \end{quote}
%
% \section{Catalogue}
%
% The following XML file can be used as source for the
% \href{http://mirror.ctan.org/help/Catalogue/catalogue.html}{\TeX\ Catalogue}.
% The elements \texttt{caption} and \texttt{description} are imported
% from the original XML file from the Catalogue.
% The name of the XML file in the Catalogue is \xfile{refcount.xml}.
%    \begin{macrocode}
%<*catalogue>
<?xml version='1.0' encoding='us-ascii'?>
<!DOCTYPE entry SYSTEM 'catalogue.dtd'>
<entry datestamp='$Date$' modifier='$Author$' id='refcount'>
  <name>refcount</name>
  <caption>Counter operations with label references.</caption>
  <authorref id='auth:oberdiek'/>
  <copyright owner='Heiko Oberdiek' year='1998,2000,2006,2008,2010,2011'/>
  <license type='lppl1.3'/>
  <version number='3.4'/>
  <description>
    Provides commands <tt>\setcounterref</tt> and
    <tt>\addtocounterref</tt> which use the section (or whatever)
    number from the reference as the value to put into the counter, as
    in:

    <pre>
    ...\label{sec:foo}
    ...
    \setcounterref{foonum}{sec:foo}
    </pre>
    Commands <tt>\setcounterpageref</tt> and
    <tt>\addtocounterpageref</tt> do the corresponding thing with the
    page reference of the label.
    <p/>
    No <tt>.ins</tt> file is distributed; process the
    <tt>.dtx</tt> with plain TeX to create one.
    <p/>
    The package is part of the <xref refid='oberdiek'>oberdiek</xref>
    bundle.
  </description>
  <documentation details='Package documentation'
      href='ctan:/macros/latex/contrib/oberdiek/refcount.pdf'/>
  <ctan file='true' path='/macros/latex/contrib/oberdiek/refcount.dtx'/>
  <miktex location='oberdiek'/>
  <texlive location='oberdiek'/>
  <install path='/macros/latex/contrib/oberdiek/oberdiek.tds.zip'/>
</entry>
%</catalogue>
%    \end{macrocode}
%
% \begin{History}
%   \begin{Version}{1998/04/08 v1.0}
%   \item
%     First public release, written as answer in the
%     newsgroup \xnewsgroup{comp.text.tex}:
%     \URL{``\link{Re: Adding a \cs{ref} to a counter?}''}^^A
%     {http://groups.google.com/group/comp.text.tex/msg/c3f2a135ef5ee528}
%   \end{Version}
%   \begin{Version}{2000/09/07 v2.0}
%   \item
%     Documentation added.
%   \item
%     LPPL 1.2
%   \item
%     Package rewritten, new commands added.
%   \end{Version}
%   \begin{Version}{2006/02/20 v3.0}
%   \item
%     Support for \xpackage{hyperref} and \xpackage{nameref} improved.
%   \item
%     Support for \xpackage{titleref} and \xpackage{babel}'s shorthands added.
%   \item
%     New: \cs{refused}, \cs{getrefbykeydefault}
%   \end{Version}
%   \begin{Version}{2008/08/11 v3.1}
%   \item
%     Code is not changed.
%   \item
%     URLs updated.
%   \end{Version}
%   \begin{Version}{2010/12/01 v3.2}
%   \item
%     \cs{IfRefUndefinedExpandable} and \cs{IfRefUndefinedBabel} added.
%   \item
%     \cs{getrefnumber}, \cs{getpagerefnumber}, \cs{getrefbykeydefault}
%     are expandable in exact two expansion steps.
%   \item
%     Non-expandable macros are made robust.
%   \item
%     Test files added.
%   \end{Version}
%   \begin{Version}{2011/06/22 v3.3}
%   \item
%     Bug fix: \cs{rc@refused} is undefined for \cs{setcounterpageref}
%     and similar macros. (Bug found by Marc van Dongen.)
%   \end{Version}
%   \begin{Version}{2011/10/16 v3.4}
%   \item
%     Bug fix: \cs{setcounterpageref} and \cs{addtocounterpageref} fixed.
%     (Bug found by Staz.)
%   \item
%     Macros \cs{(set|addto)counter(page|)ref} are made robust.
%   \end{Version}
% \end{History}
%
% \PrintIndex
%
% \Finale
\endinput

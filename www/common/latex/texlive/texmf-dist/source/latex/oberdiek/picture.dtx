% \iffalse meta-comment
%
% File: picture.dtx
% Version: 2009/10/11 v1.3
% Info: Dimens for picture macros
%
% Copyright (C) 2006-2009 by
%    Heiko Oberdiek <heiko.oberdiek at googlemail.com>
%
% This work may be distributed and/or modified under the
% conditions of the LaTeX Project Public License, either
% version 1.3c of this license or (at your option) any later
% version. This version of this license is in
%    http://www.latex-project.org/lppl/lppl-1-3c.txt
% and the latest version of this license is in
%    http://www.latex-project.org/lppl.txt
% and version 1.3 or later is part of all distributions of
% LaTeX version 2005/12/01 or later.
%
% This work has the LPPL maintenance status "maintained".
%
% This Current Maintainer of this work is Heiko Oberdiek.
%
% This work consists of the main source file picture.dtx
% and the derived files
%    picture.sty, picture.pdf, picture.ins, picture.drv,
%    picture-example.tex.
%
% Distribution:
%    CTAN:macros/latex/contrib/oberdiek/picture.dtx
%    CTAN:macros/latex/contrib/oberdiek/picture.pdf
%
% Unpacking:
%    (a) If picture.ins is present:
%           tex picture.ins
%    (b) Without picture.ins:
%           tex picture.dtx
%    (c) If you insist on using LaTeX
%           latex \let\install=y% \iffalse meta-comment
%
% File: picture.dtx
% Version: 2009/10/11 v1.3
% Info: Dimens for picture macros
%
% Copyright (C) 2006-2009 by
%    Heiko Oberdiek <heiko.oberdiek at googlemail.com>
%
% This work may be distributed and/or modified under the
% conditions of the LaTeX Project Public License, either
% version 1.3c of this license or (at your option) any later
% version. This version of this license is in
%    http://www.latex-project.org/lppl/lppl-1-3c.txt
% and the latest version of this license is in
%    http://www.latex-project.org/lppl.txt
% and version 1.3 or later is part of all distributions of
% LaTeX version 2005/12/01 or later.
%
% This work has the LPPL maintenance status "maintained".
%
% This Current Maintainer of this work is Heiko Oberdiek.
%
% This work consists of the main source file picture.dtx
% and the derived files
%    picture.sty, picture.pdf, picture.ins, picture.drv,
%    picture-example.tex.
%
% Distribution:
%    CTAN:macros/latex/contrib/oberdiek/picture.dtx
%    CTAN:macros/latex/contrib/oberdiek/picture.pdf
%
% Unpacking:
%    (a) If picture.ins is present:
%           tex picture.ins
%    (b) Without picture.ins:
%           tex picture.dtx
%    (c) If you insist on using LaTeX
%           latex \let\install=y% \iffalse meta-comment
%
% File: picture.dtx
% Version: 2009/10/11 v1.3
% Info: Dimens for picture macros
%
% Copyright (C) 2006-2009 by
%    Heiko Oberdiek <heiko.oberdiek at googlemail.com>
%
% This work may be distributed and/or modified under the
% conditions of the LaTeX Project Public License, either
% version 1.3c of this license or (at your option) any later
% version. This version of this license is in
%    http://www.latex-project.org/lppl/lppl-1-3c.txt
% and the latest version of this license is in
%    http://www.latex-project.org/lppl.txt
% and version 1.3 or later is part of all distributions of
% LaTeX version 2005/12/01 or later.
%
% This work has the LPPL maintenance status "maintained".
%
% This Current Maintainer of this work is Heiko Oberdiek.
%
% This work consists of the main source file picture.dtx
% and the derived files
%    picture.sty, picture.pdf, picture.ins, picture.drv,
%    picture-example.tex.
%
% Distribution:
%    CTAN:macros/latex/contrib/oberdiek/picture.dtx
%    CTAN:macros/latex/contrib/oberdiek/picture.pdf
%
% Unpacking:
%    (a) If picture.ins is present:
%           tex picture.ins
%    (b) Without picture.ins:
%           tex picture.dtx
%    (c) If you insist on using LaTeX
%           latex \let\install=y% \iffalse meta-comment
%
% File: picture.dtx
% Version: 2009/10/11 v1.3
% Info: Dimens for picture macros
%
% Copyright (C) 2006-2009 by
%    Heiko Oberdiek <heiko.oberdiek at googlemail.com>
%
% This work may be distributed and/or modified under the
% conditions of the LaTeX Project Public License, either
% version 1.3c of this license or (at your option) any later
% version. This version of this license is in
%    http://www.latex-project.org/lppl/lppl-1-3c.txt
% and the latest version of this license is in
%    http://www.latex-project.org/lppl.txt
% and version 1.3 or later is part of all distributions of
% LaTeX version 2005/12/01 or later.
%
% This work has the LPPL maintenance status "maintained".
%
% This Current Maintainer of this work is Heiko Oberdiek.
%
% This work consists of the main source file picture.dtx
% and the derived files
%    picture.sty, picture.pdf, picture.ins, picture.drv,
%    picture-example.tex.
%
% Distribution:
%    CTAN:macros/latex/contrib/oberdiek/picture.dtx
%    CTAN:macros/latex/contrib/oberdiek/picture.pdf
%
% Unpacking:
%    (a) If picture.ins is present:
%           tex picture.ins
%    (b) Without picture.ins:
%           tex picture.dtx
%    (c) If you insist on using LaTeX
%           latex \let\install=y\input{picture.dtx}
%        (quote the arguments according to the demands of your shell)
%
% Documentation:
%    (a) If picture.drv is present:
%           latex picture.drv
%    (b) Without picture.drv:
%           latex picture.dtx; ...
%    The class ltxdoc loads the configuration file ltxdoc.cfg
%    if available. Here you can specify further options, e.g.
%    use A4 as paper format:
%       \PassOptionsToClass{a4paper}{article}
%
%    Programm calls to get the documentation (example):
%       pdflatex picture.dtx
%       makeindex -s gind.ist picture.idx
%       pdflatex picture.dtx
%       makeindex -s gind.ist picture.idx
%       pdflatex picture.dtx
%
% Installation:
%    TDS:tex/latex/oberdiek/picture.sty
%    TDS:doc/latex/oberdiek/picture.pdf
%    TDS:doc/latex/oberdiek/picture-example.tex
%    TDS:source/latex/oberdiek/picture.dtx
%
%<*ignore>
\begingroup
  \catcode123=1 %
  \catcode125=2 %
  \def\x{LaTeX2e}%
\expandafter\endgroup
\ifcase 0\ifx\install y1\fi\expandafter
         \ifx\csname processbatchFile\endcsname\relax\else1\fi
         \ifx\fmtname\x\else 1\fi\relax
\else\csname fi\endcsname
%</ignore>
%<*install>
\input docstrip.tex
\Msg{************************************************************************}
\Msg{* Installation}
\Msg{* Package: picture 2009/10/11 v1.3 Dimens for picture macros (HO)}
\Msg{************************************************************************}

\keepsilent
\askforoverwritefalse

\let\MetaPrefix\relax
\preamble

This is a generated file.

Project: picture
Version: 2009/10/11 v1.3

Copyright (C) 2006-2009 by
   Heiko Oberdiek <heiko.oberdiek at googlemail.com>

This work may be distributed and/or modified under the
conditions of the LaTeX Project Public License, either
version 1.3c of this license or (at your option) any later
version. This version of this license is in
   http://www.latex-project.org/lppl/lppl-1-3c.txt
and the latest version of this license is in
   http://www.latex-project.org/lppl.txt
and version 1.3 or later is part of all distributions of
LaTeX version 2005/12/01 or later.

This work has the LPPL maintenance status "maintained".

This Current Maintainer of this work is Heiko Oberdiek.

This work consists of the main source file picture.dtx
and the derived files
   picture.sty, picture.pdf, picture.ins, picture.drv,
   picture-example.tex.

\endpreamble
\let\MetaPrefix\DoubleperCent

\generate{%
  \file{picture.ins}{\from{picture.dtx}{install}}%
  \file{picture.drv}{\from{picture.dtx}{driver}}%
  \usedir{tex/latex/oberdiek}%
  \file{picture.sty}{\from{picture.dtx}{package}}%
  \usedir{doc/latex/oberdiek}%
  \file{picture-example.tex}{\from{picture.dtx}{example}}%
  \nopreamble
  \nopostamble
  \usedir{source/latex/oberdiek/catalogue}%
  \file{picture.xml}{\from{picture.dtx}{catalogue}}%
}

\catcode32=13\relax% active space
\let =\space%
\Msg{************************************************************************}
\Msg{*}
\Msg{* To finish the installation you have to move the following}
\Msg{* file into a directory searched by TeX:}
\Msg{*}
\Msg{*     picture.sty}
\Msg{*}
\Msg{* To produce the documentation run the file `picture.drv'}
\Msg{* through LaTeX.}
\Msg{*}
\Msg{* Happy TeXing!}
\Msg{*}
\Msg{************************************************************************}

\endbatchfile
%</install>
%<*ignore>
\fi
%</ignore>
%<*driver>
\NeedsTeXFormat{LaTeX2e}
\ProvidesFile{picture.drv}%
  [2009/10/11 v1.3 Dimens for picture macros (HO)]%
\documentclass{ltxdoc}
\usepackage{holtxdoc}[2011/11/22]
\begin{document}
  \DocInput{picture.dtx}%
\end{document}
%</driver>
% \fi
%
% \CheckSum{322}
%
% \CharacterTable
%  {Upper-case    \A\B\C\D\E\F\G\H\I\J\K\L\M\N\O\P\Q\R\S\T\U\V\W\X\Y\Z
%   Lower-case    \a\b\c\d\e\f\g\h\i\j\k\l\m\n\o\p\q\r\s\t\u\v\w\x\y\z
%   Digits        \0\1\2\3\4\5\6\7\8\9
%   Exclamation   \!     Double quote  \"     Hash (number) \#
%   Dollar        \$     Percent       \%     Ampersand     \&
%   Acute accent  \'     Left paren    \(     Right paren   \)
%   Asterisk      \*     Plus          \+     Comma         \,
%   Minus         \-     Point         \.     Solidus       \/
%   Colon         \:     Semicolon     \;     Less than     \<
%   Equals        \=     Greater than  \>     Question mark \?
%   Commercial at \@     Left bracket  \[     Backslash     \\
%   Right bracket \]     Circumflex    \^     Underscore    \_
%   Grave accent  \`     Left brace    \{     Vertical bar  \|
%   Right brace   \}     Tilde         \~}
%
% \GetFileInfo{picture.drv}
%
% \title{The \xpackage{picture} package}
% \date{2009/10/11 v1.3}
% \author{Heiko Oberdiek\\\xemail{heiko.oberdiek at googlemail.com}}
%
% \maketitle
%
% \begin{abstract}
% There are macro and environment arguments that expect numbers
% that will internally be multiplicated with \cs{unitlength}.
% This package extends the syntax of these arguments that
% dimens with calculation support can be added for these arguments.
% \end{abstract}
%
% \tableofcontents
%
% \section{User interface}
%
% \subsection{Introduction}
%
% The environment |picture| and macros such as \cs{put}, \cs{line},
% \cs{vector} and other macros have arguments that expect numbers
% that are used as factor for \cs{unitlength}.
% This package redefines such macros and adds code that detects
% whether such an argument is given as number or as length.
% In the latter case, the length is used directly without multiplying
% with \cs{unitlength}.
%
% \subsection{Options}
%
% Depending on the available features, also length expressions
% can be given. Option \xoption{calc} loads package \xpackage{calc}.
% Then expressions of these package may be used.
% Otherwise \xoption{etex} wraps the length argument inside
% \cs{dimexpr}\dots\cs{relax}, if \eTeX\ is available. Otherwise
% option \xoption{plain} uses plain assignments without calculation
% support.
%
% The default is \xoption{calc} if package \xpackage{calc}
% is loaded before package \xoption{picture}. If you specify
% option \xoption{calc} the loading of \xpackage{calc} is ensured.
% Otherwise package \xpackage{picture} looks whether \cs{dimexpr}
% is available and uses then option \xoption{etex} as default.
% If \eTeX\ also could not be found, then \xoption{plain} is used.
%
% \subsection{Example}
%
%    \begin{macrocode}
%<*example>
\documentclass{article}

\usepackage[calc]{picture}

\begin{document}

\setlength{\unitlength}{1pt}

\begin{picture}(\widthof{Hello World}, 10mm)
  \put(0, 0){\makebox(0,0)[lb]{Hello World}}%
  \put(0, \heightof{Hello World} + \fboxsep){%
    \line(1, 0){\widthof{Hello World}}%
  }%
  \put(\widthof{Hello World}, 10mm){%
    \line(0, -1){10mm}%
  }%
\end{picture}

\end{document}
%</example>
%    \end{macrocode}
%
% \subsection{Supported packages}
%
% Packages \xpackage{pspicture} and \xpackage{pict2e} are supported,
% but they must be loaded before package \xpackage{picture}.
%
% New macros can be supported by \cs{picture@redefine}.
% The first argument is the macro which contains the arguments
% in its parameter text that you want to support by package \xpackage{picture}.
% The second argument contains the parameter text. Change \# to \& for
% the arguments in question.
% Examples (already used by package \xpackage{picture}):
% \begin{quote}
% |\picture@redefine\put{(&1,&2)}|\\
% |\picture@redefine\line{(#1,#2)&3}|
% \end{quote}
%
% \StopEventually{
% }
%
% \section{Implementation}
%
% \subsection{Identification}
%
%    \begin{macrocode}
%<*package>
\NeedsTeXFormat{LaTeX2e}
\ProvidesPackage{picture}%
  [2009/10/11 v1.3 Dimens for picture macros (HO)]%
%    \end{macrocode}
%
% \subsection{Options}
%
%    \begin{macrocode}
\def\Pc@calcname{calc}
\def\Pc@etexname{etex}
\def\Pc@plainname{plain}
%    \end{macrocode}
%    \begin{macro}{\Pc@method}
%    Macro \cs{Pc@method} stores the method to use for calculations.
%    Check which features are available and set the default for \cs{Pc@method}.
%    \begin{macrocode}
\@ifpackageloaded{calc}{%
  \let\Pc@method\Pc@calcname
}{%
  \begingroup\expandafter\expandafter\expandafter\endgroup
  \expandafter\ifx\csname dimexpr\endcsname\relax
    \let\Pc@method\Pc@plainname
  \else
    \let\Pc@method\Pc@etexname
  \fi
}
%    \end{macrocode}
%    \end{macro}
%    \begin{macrocode}
\DeclareOption{plain}{%
  \let\Pc@method\Pc@plainname
}
\DeclareOption{etex}{%
  \begingroup\expandafter\expandafter\expandafter\endgroup
  \expandafter\ifx\csname dimexpr\endcsname\relax
    \PackageError{picture}{%
      e-TeX is not available%
    }\@ehc
  \else
    \let\Pc@method\Pc@etexname
  \fi
}
\DeclareOption{calc}{%
  \let\Pc@method\Pc@calcname
}
\ProcessOptions*
\begingroup
  \let\on@line\@empty
  \PackageInfo{picture}{Calculation method: \Pc@method}%
\endgroup
%    \end{macrocode}
%
% \subsection{Calculation method}
%
%    \begin{macrocode}
\ifx\Pc@method\Pc@calcname
  \RequirePackage{calc}%
\fi
%    \end{macrocode}
%
% \subsubsection{Method \xoption{calc}}
%
%    \begin{macrocode}
\ifx\Pc@method\Pc@calcname
  \def\Pc@tokslength#1{%
    \begingroup
      \let\calc@error\Pc@calc@error
      \setlength\dimen@{#1\unitlength}\Pc@next\Pc@nil{#1}%
  }%
  \let\PcOrg@calc@error\calc@error
  \@ifpackagelater{calc}{2007/08/22}{% v4.3
    \def\Pc@calc@error#1{%
      \expandafter\ifx\expandafter\unitlength\noexpand#1\relax
        \def\calc@next##1!{%
          \endgroup
          \aftergroup\afterassignment
          \aftergroup\Pc@next
        }%
        \expandafter\@firstoftwo
      \else
        \expandafter\@secondoftwo
      \fi
      {%
        \calc@next{#1}%
      }{%
        \PcOrg@calc@error{#1}%
      }%
    }%
  }{%
    \def\Pc@calc@error#1{%
      \expandafter\ifx\expandafter\unitlength\noexpand#1\relax
        \def\calc@next##1!{%
          \endgroup
          \aftergroup\afterassignment
          \aftergroup\Pc@next
        }%
        \expandafter\@gobble
      \else
        \expandafter\@firstofone
      \fi
      {%
        \PcOrg@calc@error{#1}%
      }%
    }%
  }%
\fi
%    \end{macrocode}
%
% \subsubsection{Method \xoption{etex}}
%
%    \begin{macrocode}
\ifx\Pc@method\Pc@etexname
  \def\Pc@tokslength#1{%
    \begingroup
      \afterassignment\Pc@next
      \dimen@=\dimexpr#1\unitlength\Pc@nil{#1}%
  }%
\fi
%    \end{macrocode}
%
% \subsubsection{Method \xoption{plain}}
%
%    \begin{macrocode}
\ifx\Pc@method\Pc@plainname
  \def\Pc@tokslength#1{%
    \begingroup
      \afterassignment\Pc@next
      \dimen@=#1\unitlength\Pc@nil{#1}%
  }%
\fi
%    \end{macrocode}
%
% \subsubsection{Help macros}
%
%    \begin{macrocode}
\def\Pc@next#1\Pc@nil#2{%
  \ifx\\#1\\%
    \endgroup
    \Pc@addtoks{{#2}}%
  \else
    \expandafter\endgroup
    \expandafter\Pc@addtoks\expandafter{%
      \expandafter{\the\dimen@\@gobble}%
    }%
  \fi
}
%    \end{macrocode}
%    \begin{macro}{\Pc@nil}
%    \cs{Pc@nil} must not have the meaning of \cs{relax} because of
%    \cs{dimexpr}.
%    \begin{macrocode}
\let\Pc@nil\message
%    \end{macrocode}
%    \end{macro}
%    \begin{macro}{\Pc@addtoks}
%    \begin{macrocode}
\def\Pc@addtoks#1{%
  \toks@=\expandafter{\the\toks@#1}%
}
%    \end{macrocode}
%    \end{macro}
%    \begin{macro}{\Pc@init}
%    \begin{macrocode}
\def\Pc@init#1{%
  \begingroup
    \toks@={#1}%
}
%    \end{macrocode}
%    \end{macro}
%    \begin{macro}{\Pc@finish}
%    \begin{macrocode}
\def\Pc@finish#1{%
  \expandafter\endgroup
  \expandafter#1\the\toks@
}
%    \end{macrocode}
%    \end{macro}
%
% \subsection{Redefinitions}
%
%    \begin{macro}{\picture@redefine}
%    \begin{tabular}[t]{@{}l@{ }l@{}}
%    |#1|:& command name\\
%    |#2|:& parameter text, length parameter with \& instead of \#
%    \end{tabular}
%    \begin{macrocode}
\def\picture@redefine#1#2{%
  \begingroup
    \edef\reserved@a{%
      \noexpand\noexpand
      \expandafter\noexpand
          \csname PcOrg@\expandafter\@gobble\string#1\endcsname
    }%
    \toks0{#1}%
    \Pc@first#2&0%
}
%    \end{macrocode}
%    \end{macro}
%    \begin{macro}{\Pc@first}
%    \begin{macrocode}
\def\Pc@first#1&{%
  \toks1={#1}%
  \toks2={\Pc@init{#1}}%
  \Pc@scanlength
}
%    \end{macrocode}
%    \end{macro}
%    \begin{macro}{\Pc@scanlength}
%    |#1|: number of length parameter or zero
%    \begin{macrocode}
\def\Pc@scanlength#1{%
  \ifcase#1 %
    \expandafter\Pc@last
  \else
    \toks1=\expandafter{\the\toks1 ###1}%
    \toks2=\expandafter{\the\toks2 \Pc@tokslength{###1}}%
    \expandafter\Pc@scannext
  \fi
}
%    \end{macrocode}
%    \end{macro}
%    \begin{macro}{\Pc@scannext}
%    \begin{macrocode}
\def\Pc@scannext#1&{%
  \ifx\\#1\\%
  \else
    \toks1=\expandafter{\the\toks1 #1}%
    \toks2=\expandafter{\the\toks2 \Pc@addtoks{#1}}%
  \fi
  \Pc@scanlength
}
%    \end{macrocode}
%    \end{macro}
%    \begin{macro}{\Pc@last}
%    \begin{macrocode}
\def\Pc@last{%
  \edef\x{%
    \endgroup
    \let\reserved@a\the\toks0 %
    \def\the\toks0 \the\toks1 {%
      \the\toks2 %
      \noexpand\Pc@finish\reserved@a
    }%
  }%
  \x
}
%    \end{macrocode}
%    \end{macro}
%
% \subsubsection{\LaTeX\ base macros}
%
%    \begin{macrocode}
\picture@redefine\@picture{(&1,&2)(&3,&4)}
\picture@redefine\put{(&1,&2)}
\picture@redefine\multiput{(&1,&2)}
\picture@redefine\@multiput{(&1,&2)}
\picture@redefine\line{(#1,#2)&3}
\picture@redefine\vector{(#1,#2)&3}
\picture@redefine\dashbox{&1(&2,&3)}
\picture@redefine\@circle{&1}
\picture@redefine\@dot{&1}
\picture@redefine\@bezier{#1(&2,&3)(&4,&5)(&6,&7)}
\picture@redefine\@imakepicbox{(&1,&2)}
%    \end{macrocode}
%
% \subsubsection{Package \xpackage{pspicture}}
%
%     Package \xpackage{pspicture} changes the signature of \cs{@oval}
%     by adding an optional argument.
%    \begin{macrocode}
\@ifpackageloaded{pspicture}{%
  \picture@redefine\@oval{[&1](&2,&3)}%
  \picture@redefine\Line{(&1,&2)}%
  \picture@redefine\Curve{(&1,&2)}%
  \picture@redefine\Vector{(&1,&2)}%
}{%
  \picture@redefine\@oval{(&1,&2)}%
}
%    \end{macrocode}
%
% \subsection{Check package loading order}
%
%    \begin{macro}{\PC@checkpackage}
%    \begin{macrocode}
\def\Pc@checkpackage#1{%
  \@ifpackageloaded{#1}{%
  }{%
    \AtBeginDocument{%
      \@ifpackageloaded{#1}{%
        \PackageWarningNoLine{picture}{%
          Package `#1' is loaded after `picture'.\MessageBreak
          Load package `picture' afterwards to get full support%
          \MessageBreak
          of its additional syntax with length specifications%
        }%
      }{}%
    }%
  }%
}
%    \end{macrocode}
%    \end{macro}
%    \begin{macrocode}
\Pc@checkpackage{pict2e}
\Pc@checkpackage{pspicture}
%    \end{macrocode}
%
%    \begin{macrocode}
%</package>
%    \end{macrocode}
%
% \section{Installation}
%
% \subsection{Download}
%
% \paragraph{Package.} This package is available on
% CTAN\footnote{\url{ftp://ftp.ctan.org/tex-archive/}}:
% \begin{description}
% \item[\CTAN{macros/latex/contrib/oberdiek/picture.dtx}] The source file.
% \item[\CTAN{macros/latex/contrib/oberdiek/picture.pdf}] Documentation.
% \end{description}
%
%
% \paragraph{Bundle.} All the packages of the bundle `oberdiek'
% are also available in a TDS compliant ZIP archive. There
% the packages are already unpacked and the documentation files
% are generated. The files and directories obey the TDS standard.
% \begin{description}
% \item[\CTAN{install/macros/latex/contrib/oberdiek.tds.zip}]
% \end{description}
% \emph{TDS} refers to the standard ``A Directory Structure
% for \TeX\ Files'' (\CTAN{tds/tds.pdf}). Directories
% with \xfile{texmf} in their name are usually organized this way.
%
% \subsection{Bundle installation}
%
% \paragraph{Unpacking.} Unpack the \xfile{oberdiek.tds.zip} in the
% TDS tree (also known as \xfile{texmf} tree) of your choice.
% Example (linux):
% \begin{quote}
%   |unzip oberdiek.tds.zip -d ~/texmf|
% \end{quote}
%
% \paragraph{Script installation.}
% Check the directory \xfile{TDS:scripts/oberdiek/} for
% scripts that need further installation steps.
% Package \xpackage{attachfile2} comes with the Perl script
% \xfile{pdfatfi.pl} that should be installed in such a way
% that it can be called as \texttt{pdfatfi}.
% Example (linux):
% \begin{quote}
%   |chmod +x scripts/oberdiek/pdfatfi.pl|\\
%   |cp scripts/oberdiek/pdfatfi.pl /usr/local/bin/|
% \end{quote}
%
% \subsection{Package installation}
%
% \paragraph{Unpacking.} The \xfile{.dtx} file is a self-extracting
% \docstrip\ archive. The files are extracted by running the
% \xfile{.dtx} through \plainTeX:
% \begin{quote}
%   \verb|tex picture.dtx|
% \end{quote}
%
% \paragraph{TDS.} Now the different files must be moved into
% the different directories in your installation TDS tree
% (also known as \xfile{texmf} tree):
% \begin{quote}
% \def\t{^^A
% \begin{tabular}{@{}>{\ttfamily}l@{ $\rightarrow$ }>{\ttfamily}l@{}}
%   picture.sty & tex/latex/oberdiek/picture.sty\\
%   picture.pdf & doc/latex/oberdiek/picture.pdf\\
%   picture-example.tex & doc/latex/oberdiek/picture-example.tex\\
%   picture.dtx & source/latex/oberdiek/picture.dtx\\
% \end{tabular}^^A
% }^^A
% \sbox0{\t}^^A
% \ifdim\wd0>\linewidth
%   \begingroup
%     \advance\linewidth by\leftmargin
%     \advance\linewidth by\rightmargin
%   \edef\x{\endgroup
%     \def\noexpand\lw{\the\linewidth}^^A
%   }\x
%   \def\lwbox{^^A
%     \leavevmode
%     \hbox to \linewidth{^^A
%       \kern-\leftmargin\relax
%       \hss
%       \usebox0
%       \hss
%       \kern-\rightmargin\relax
%     }^^A
%   }^^A
%   \ifdim\wd0>\lw
%     \sbox0{\small\t}^^A
%     \ifdim\wd0>\linewidth
%       \ifdim\wd0>\lw
%         \sbox0{\footnotesize\t}^^A
%         \ifdim\wd0>\linewidth
%           \ifdim\wd0>\lw
%             \sbox0{\scriptsize\t}^^A
%             \ifdim\wd0>\linewidth
%               \ifdim\wd0>\lw
%                 \sbox0{\tiny\t}^^A
%                 \ifdim\wd0>\linewidth
%                   \lwbox
%                 \else
%                   \usebox0
%                 \fi
%               \else
%                 \lwbox
%               \fi
%             \else
%               \usebox0
%             \fi
%           \else
%             \lwbox
%           \fi
%         \else
%           \usebox0
%         \fi
%       \else
%         \lwbox
%       \fi
%     \else
%       \usebox0
%     \fi
%   \else
%     \lwbox
%   \fi
% \else
%   \usebox0
% \fi
% \end{quote}
% If you have a \xfile{docstrip.cfg} that configures and enables \docstrip's
% TDS installing feature, then some files can already be in the right
% place, see the documentation of \docstrip.
%
% \subsection{Refresh file name databases}
%
% If your \TeX~distribution
% (\teTeX, \mikTeX, \dots) relies on file name databases, you must refresh
% these. For example, \teTeX\ users run \verb|texhash| or
% \verb|mktexlsr|.
%
% \subsection{Some details for the interested}
%
% \paragraph{Attached source.}
%
% The PDF documentation on CTAN also includes the
% \xfile{.dtx} source file. It can be extracted by
% AcrobatReader 6 or higher. Another option is \textsf{pdftk},
% e.g. unpack the file into the current directory:
% \begin{quote}
%   \verb|pdftk picture.pdf unpack_files output .|
% \end{quote}
%
% \paragraph{Unpacking with \LaTeX.}
% The \xfile{.dtx} chooses its action depending on the format:
% \begin{description}
% \item[\plainTeX:] Run \docstrip\ and extract the files.
% \item[\LaTeX:] Generate the documentation.
% \end{description}
% If you insist on using \LaTeX\ for \docstrip\ (really,
% \docstrip\ does not need \LaTeX), then inform the autodetect routine
% about your intention:
% \begin{quote}
%   \verb|latex \let\install=y\input{picture.dtx}|
% \end{quote}
% Do not forget to quote the argument according to the demands
% of your shell.
%
% \paragraph{Generating the documentation.}
% You can use both the \xfile{.dtx} or the \xfile{.drv} to generate
% the documentation. The process can be configured by the
% configuration file \xfile{ltxdoc.cfg}. For instance, put this
% line into this file, if you want to have A4 as paper format:
% \begin{quote}
%   \verb|\PassOptionsToClass{a4paper}{article}|
% \end{quote}
% An example follows how to generate the
% documentation with pdf\LaTeX:
% \begin{quote}
%\begin{verbatim}
%pdflatex picture.dtx
%makeindex -s gind.ist picture.idx
%pdflatex picture.dtx
%makeindex -s gind.ist picture.idx
%pdflatex picture.dtx
%\end{verbatim}
% \end{quote}
%
% \section{Catalogue}
%
% The following XML file can be used as source for the
% \href{http://mirror.ctan.org/help/Catalogue/catalogue.html}{\TeX\ Catalogue}.
% The elements \texttt{caption} and \texttt{description} are imported
% from the original XML file from the Catalogue.
% The name of the XML file in the Catalogue is \xfile{picture.xml}.
%    \begin{macrocode}
%<*catalogue>
<?xml version='1.0' encoding='us-ascii'?>
<!DOCTYPE entry SYSTEM 'catalogue.dtd'>
<entry datestamp='$Date$' modifier='$Author$' id='picture'>
  <name>picture</name>
  <caption>Dimens for picture macros.</caption>
  <authorref id='auth:oberdiek'/>
  <copyright owner='Heiko Oberdiek' year='2006-2009'/>
  <license type='lppl1.3'/>
  <version number='1.3'/>
  <description>
    There are macro and environment arguments that expect numbers
    that will internally be multiplied by <tt>\unitlength</tt>.
    This package extends the syntax of these arguments, so that
    dimensions with calculation support may be used for these arguments.
    <p/>
    The package is part of the <xref refid='oberdiek'>oberdiek</xref> bundle.
  </description>
  <documentation details='Package documentation'
      href='ctan:/macros/latex/contrib/oberdiek/picture.pdf'/>
  <ctan file='true' path='/macros/latex/contrib/oberdiek/picture.dtx'/>
  <miktex location='oberdiek'/>
  <texlive location='oberdiek'/>
  <install path='/macros/latex/contrib/oberdiek/oberdiek.tds.zip'/>
</entry>
%</catalogue>
%    \end{macrocode}
%
% \begin{History}
%   \begin{Version}{2006/08/26 v1.0}
%   \item
%     First released version. (First start of the project was June/July 2002.)
%   \end{Version}
%   \begin{Version}{2007/04/11 v1.1}
%   \item
%     Line ends sanitized.
%   \end{Version}
%   \begin{Version}{2008/11/26 v1.2}
%   \item
%     Package \xpackage{pict2e} added to documentation section ``Supported
%     packages''.
%   \item
%     Package order of supported packages is checked.
%   \end{Version}
%   \begin{Version}{2009/10/11 v1.3}
%   \item
%     Fix because of new version v4.3 of package \xpackage{calc}.
%   \end{Version}
% \end{History}
%
% \PrintIndex
%
% \Finale
\endinput

%        (quote the arguments according to the demands of your shell)
%
% Documentation:
%    (a) If picture.drv is present:
%           latex picture.drv
%    (b) Without picture.drv:
%           latex picture.dtx; ...
%    The class ltxdoc loads the configuration file ltxdoc.cfg
%    if available. Here you can specify further options, e.g.
%    use A4 as paper format:
%       \PassOptionsToClass{a4paper}{article}
%
%    Programm calls to get the documentation (example):
%       pdflatex picture.dtx
%       makeindex -s gind.ist picture.idx
%       pdflatex picture.dtx
%       makeindex -s gind.ist picture.idx
%       pdflatex picture.dtx
%
% Installation:
%    TDS:tex/latex/oberdiek/picture.sty
%    TDS:doc/latex/oberdiek/picture.pdf
%    TDS:doc/latex/oberdiek/picture-example.tex
%    TDS:source/latex/oberdiek/picture.dtx
%
%<*ignore>
\begingroup
  \catcode123=1 %
  \catcode125=2 %
  \def\x{LaTeX2e}%
\expandafter\endgroup
\ifcase 0\ifx\install y1\fi\expandafter
         \ifx\csname processbatchFile\endcsname\relax\else1\fi
         \ifx\fmtname\x\else 1\fi\relax
\else\csname fi\endcsname
%</ignore>
%<*install>
\input docstrip.tex
\Msg{************************************************************************}
\Msg{* Installation}
\Msg{* Package: picture 2009/10/11 v1.3 Dimens for picture macros (HO)}
\Msg{************************************************************************}

\keepsilent
\askforoverwritefalse

\let\MetaPrefix\relax
\preamble

This is a generated file.

Project: picture
Version: 2009/10/11 v1.3

Copyright (C) 2006-2009 by
   Heiko Oberdiek <heiko.oberdiek at googlemail.com>

This work may be distributed and/or modified under the
conditions of the LaTeX Project Public License, either
version 1.3c of this license or (at your option) any later
version. This version of this license is in
   http://www.latex-project.org/lppl/lppl-1-3c.txt
and the latest version of this license is in
   http://www.latex-project.org/lppl.txt
and version 1.3 or later is part of all distributions of
LaTeX version 2005/12/01 or later.

This work has the LPPL maintenance status "maintained".

This Current Maintainer of this work is Heiko Oberdiek.

This work consists of the main source file picture.dtx
and the derived files
   picture.sty, picture.pdf, picture.ins, picture.drv,
   picture-example.tex.

\endpreamble
\let\MetaPrefix\DoubleperCent

\generate{%
  \file{picture.ins}{\from{picture.dtx}{install}}%
  \file{picture.drv}{\from{picture.dtx}{driver}}%
  \usedir{tex/latex/oberdiek}%
  \file{picture.sty}{\from{picture.dtx}{package}}%
  \usedir{doc/latex/oberdiek}%
  \file{picture-example.tex}{\from{picture.dtx}{example}}%
  \nopreamble
  \nopostamble
  \usedir{source/latex/oberdiek/catalogue}%
  \file{picture.xml}{\from{picture.dtx}{catalogue}}%
}

\catcode32=13\relax% active space
\let =\space%
\Msg{************************************************************************}
\Msg{*}
\Msg{* To finish the installation you have to move the following}
\Msg{* file into a directory searched by TeX:}
\Msg{*}
\Msg{*     picture.sty}
\Msg{*}
\Msg{* To produce the documentation run the file `picture.drv'}
\Msg{* through LaTeX.}
\Msg{*}
\Msg{* Happy TeXing!}
\Msg{*}
\Msg{************************************************************************}

\endbatchfile
%</install>
%<*ignore>
\fi
%</ignore>
%<*driver>
\NeedsTeXFormat{LaTeX2e}
\ProvidesFile{picture.drv}%
  [2009/10/11 v1.3 Dimens for picture macros (HO)]%
\documentclass{ltxdoc}
\usepackage{holtxdoc}[2011/11/22]
\begin{document}
  \DocInput{picture.dtx}%
\end{document}
%</driver>
% \fi
%
% \CheckSum{322}
%
% \CharacterTable
%  {Upper-case    \A\B\C\D\E\F\G\H\I\J\K\L\M\N\O\P\Q\R\S\T\U\V\W\X\Y\Z
%   Lower-case    \a\b\c\d\e\f\g\h\i\j\k\l\m\n\o\p\q\r\s\t\u\v\w\x\y\z
%   Digits        \0\1\2\3\4\5\6\7\8\9
%   Exclamation   \!     Double quote  \"     Hash (number) \#
%   Dollar        \$     Percent       \%     Ampersand     \&
%   Acute accent  \'     Left paren    \(     Right paren   \)
%   Asterisk      \*     Plus          \+     Comma         \,
%   Minus         \-     Point         \.     Solidus       \/
%   Colon         \:     Semicolon     \;     Less than     \<
%   Equals        \=     Greater than  \>     Question mark \?
%   Commercial at \@     Left bracket  \[     Backslash     \\
%   Right bracket \]     Circumflex    \^     Underscore    \_
%   Grave accent  \`     Left brace    \{     Vertical bar  \|
%   Right brace   \}     Tilde         \~}
%
% \GetFileInfo{picture.drv}
%
% \title{The \xpackage{picture} package}
% \date{2009/10/11 v1.3}
% \author{Heiko Oberdiek\\\xemail{heiko.oberdiek at googlemail.com}}
%
% \maketitle
%
% \begin{abstract}
% There are macro and environment arguments that expect numbers
% that will internally be multiplicated with \cs{unitlength}.
% This package extends the syntax of these arguments that
% dimens with calculation support can be added for these arguments.
% \end{abstract}
%
% \tableofcontents
%
% \section{User interface}
%
% \subsection{Introduction}
%
% The environment |picture| and macros such as \cs{put}, \cs{line},
% \cs{vector} and other macros have arguments that expect numbers
% that are used as factor for \cs{unitlength}.
% This package redefines such macros and adds code that detects
% whether such an argument is given as number or as length.
% In the latter case, the length is used directly without multiplying
% with \cs{unitlength}.
%
% \subsection{Options}
%
% Depending on the available features, also length expressions
% can be given. Option \xoption{calc} loads package \xpackage{calc}.
% Then expressions of these package may be used.
% Otherwise \xoption{etex} wraps the length argument inside
% \cs{dimexpr}\dots\cs{relax}, if \eTeX\ is available. Otherwise
% option \xoption{plain} uses plain assignments without calculation
% support.
%
% The default is \xoption{calc} if package \xpackage{calc}
% is loaded before package \xoption{picture}. If you specify
% option \xoption{calc} the loading of \xpackage{calc} is ensured.
% Otherwise package \xpackage{picture} looks whether \cs{dimexpr}
% is available and uses then option \xoption{etex} as default.
% If \eTeX\ also could not be found, then \xoption{plain} is used.
%
% \subsection{Example}
%
%    \begin{macrocode}
%<*example>
\documentclass{article}

\usepackage[calc]{picture}

\begin{document}

\setlength{\unitlength}{1pt}

\begin{picture}(\widthof{Hello World}, 10mm)
  \put(0, 0){\makebox(0,0)[lb]{Hello World}}%
  \put(0, \heightof{Hello World} + \fboxsep){%
    \line(1, 0){\widthof{Hello World}}%
  }%
  \put(\widthof{Hello World}, 10mm){%
    \line(0, -1){10mm}%
  }%
\end{picture}

\end{document}
%</example>
%    \end{macrocode}
%
% \subsection{Supported packages}
%
% Packages \xpackage{pspicture} and \xpackage{pict2e} are supported,
% but they must be loaded before package \xpackage{picture}.
%
% New macros can be supported by \cs{picture@redefine}.
% The first argument is the macro which contains the arguments
% in its parameter text that you want to support by package \xpackage{picture}.
% The second argument contains the parameter text. Change \# to \& for
% the arguments in question.
% Examples (already used by package \xpackage{picture}):
% \begin{quote}
% |\picture@redefine\put{(&1,&2)}|\\
% |\picture@redefine\line{(#1,#2)&3}|
% \end{quote}
%
% \StopEventually{
% }
%
% \section{Implementation}
%
% \subsection{Identification}
%
%    \begin{macrocode}
%<*package>
\NeedsTeXFormat{LaTeX2e}
\ProvidesPackage{picture}%
  [2009/10/11 v1.3 Dimens for picture macros (HO)]%
%    \end{macrocode}
%
% \subsection{Options}
%
%    \begin{macrocode}
\def\Pc@calcname{calc}
\def\Pc@etexname{etex}
\def\Pc@plainname{plain}
%    \end{macrocode}
%    \begin{macro}{\Pc@method}
%    Macro \cs{Pc@method} stores the method to use for calculations.
%    Check which features are available and set the default for \cs{Pc@method}.
%    \begin{macrocode}
\@ifpackageloaded{calc}{%
  \let\Pc@method\Pc@calcname
}{%
  \begingroup\expandafter\expandafter\expandafter\endgroup
  \expandafter\ifx\csname dimexpr\endcsname\relax
    \let\Pc@method\Pc@plainname
  \else
    \let\Pc@method\Pc@etexname
  \fi
}
%    \end{macrocode}
%    \end{macro}
%    \begin{macrocode}
\DeclareOption{plain}{%
  \let\Pc@method\Pc@plainname
}
\DeclareOption{etex}{%
  \begingroup\expandafter\expandafter\expandafter\endgroup
  \expandafter\ifx\csname dimexpr\endcsname\relax
    \PackageError{picture}{%
      e-TeX is not available%
    }\@ehc
  \else
    \let\Pc@method\Pc@etexname
  \fi
}
\DeclareOption{calc}{%
  \let\Pc@method\Pc@calcname
}
\ProcessOptions*
\begingroup
  \let\on@line\@empty
  \PackageInfo{picture}{Calculation method: \Pc@method}%
\endgroup
%    \end{macrocode}
%
% \subsection{Calculation method}
%
%    \begin{macrocode}
\ifx\Pc@method\Pc@calcname
  \RequirePackage{calc}%
\fi
%    \end{macrocode}
%
% \subsubsection{Method \xoption{calc}}
%
%    \begin{macrocode}
\ifx\Pc@method\Pc@calcname
  \def\Pc@tokslength#1{%
    \begingroup
      \let\calc@error\Pc@calc@error
      \setlength\dimen@{#1\unitlength}\Pc@next\Pc@nil{#1}%
  }%
  \let\PcOrg@calc@error\calc@error
  \@ifpackagelater{calc}{2007/08/22}{% v4.3
    \def\Pc@calc@error#1{%
      \expandafter\ifx\expandafter\unitlength\noexpand#1\relax
        \def\calc@next##1!{%
          \endgroup
          \aftergroup\afterassignment
          \aftergroup\Pc@next
        }%
        \expandafter\@firstoftwo
      \else
        \expandafter\@secondoftwo
      \fi
      {%
        \calc@next{#1}%
      }{%
        \PcOrg@calc@error{#1}%
      }%
    }%
  }{%
    \def\Pc@calc@error#1{%
      \expandafter\ifx\expandafter\unitlength\noexpand#1\relax
        \def\calc@next##1!{%
          \endgroup
          \aftergroup\afterassignment
          \aftergroup\Pc@next
        }%
        \expandafter\@gobble
      \else
        \expandafter\@firstofone
      \fi
      {%
        \PcOrg@calc@error{#1}%
      }%
    }%
  }%
\fi
%    \end{macrocode}
%
% \subsubsection{Method \xoption{etex}}
%
%    \begin{macrocode}
\ifx\Pc@method\Pc@etexname
  \def\Pc@tokslength#1{%
    \begingroup
      \afterassignment\Pc@next
      \dimen@=\dimexpr#1\unitlength\Pc@nil{#1}%
  }%
\fi
%    \end{macrocode}
%
% \subsubsection{Method \xoption{plain}}
%
%    \begin{macrocode}
\ifx\Pc@method\Pc@plainname
  \def\Pc@tokslength#1{%
    \begingroup
      \afterassignment\Pc@next
      \dimen@=#1\unitlength\Pc@nil{#1}%
  }%
\fi
%    \end{macrocode}
%
% \subsubsection{Help macros}
%
%    \begin{macrocode}
\def\Pc@next#1\Pc@nil#2{%
  \ifx\\#1\\%
    \endgroup
    \Pc@addtoks{{#2}}%
  \else
    \expandafter\endgroup
    \expandafter\Pc@addtoks\expandafter{%
      \expandafter{\the\dimen@\@gobble}%
    }%
  \fi
}
%    \end{macrocode}
%    \begin{macro}{\Pc@nil}
%    \cs{Pc@nil} must not have the meaning of \cs{relax} because of
%    \cs{dimexpr}.
%    \begin{macrocode}
\let\Pc@nil\message
%    \end{macrocode}
%    \end{macro}
%    \begin{macro}{\Pc@addtoks}
%    \begin{macrocode}
\def\Pc@addtoks#1{%
  \toks@=\expandafter{\the\toks@#1}%
}
%    \end{macrocode}
%    \end{macro}
%    \begin{macro}{\Pc@init}
%    \begin{macrocode}
\def\Pc@init#1{%
  \begingroup
    \toks@={#1}%
}
%    \end{macrocode}
%    \end{macro}
%    \begin{macro}{\Pc@finish}
%    \begin{macrocode}
\def\Pc@finish#1{%
  \expandafter\endgroup
  \expandafter#1\the\toks@
}
%    \end{macrocode}
%    \end{macro}
%
% \subsection{Redefinitions}
%
%    \begin{macro}{\picture@redefine}
%    \begin{tabular}[t]{@{}l@{ }l@{}}
%    |#1|:& command name\\
%    |#2|:& parameter text, length parameter with \& instead of \#
%    \end{tabular}
%    \begin{macrocode}
\def\picture@redefine#1#2{%
  \begingroup
    \edef\reserved@a{%
      \noexpand\noexpand
      \expandafter\noexpand
          \csname PcOrg@\expandafter\@gobble\string#1\endcsname
    }%
    \toks0{#1}%
    \Pc@first#2&0%
}
%    \end{macrocode}
%    \end{macro}
%    \begin{macro}{\Pc@first}
%    \begin{macrocode}
\def\Pc@first#1&{%
  \toks1={#1}%
  \toks2={\Pc@init{#1}}%
  \Pc@scanlength
}
%    \end{macrocode}
%    \end{macro}
%    \begin{macro}{\Pc@scanlength}
%    |#1|: number of length parameter or zero
%    \begin{macrocode}
\def\Pc@scanlength#1{%
  \ifcase#1 %
    \expandafter\Pc@last
  \else
    \toks1=\expandafter{\the\toks1 ###1}%
    \toks2=\expandafter{\the\toks2 \Pc@tokslength{###1}}%
    \expandafter\Pc@scannext
  \fi
}
%    \end{macrocode}
%    \end{macro}
%    \begin{macro}{\Pc@scannext}
%    \begin{macrocode}
\def\Pc@scannext#1&{%
  \ifx\\#1\\%
  \else
    \toks1=\expandafter{\the\toks1 #1}%
    \toks2=\expandafter{\the\toks2 \Pc@addtoks{#1}}%
  \fi
  \Pc@scanlength
}
%    \end{macrocode}
%    \end{macro}
%    \begin{macro}{\Pc@last}
%    \begin{macrocode}
\def\Pc@last{%
  \edef\x{%
    \endgroup
    \let\reserved@a\the\toks0 %
    \def\the\toks0 \the\toks1 {%
      \the\toks2 %
      \noexpand\Pc@finish\reserved@a
    }%
  }%
  \x
}
%    \end{macrocode}
%    \end{macro}
%
% \subsubsection{\LaTeX\ base macros}
%
%    \begin{macrocode}
\picture@redefine\@picture{(&1,&2)(&3,&4)}
\picture@redefine\put{(&1,&2)}
\picture@redefine\multiput{(&1,&2)}
\picture@redefine\@multiput{(&1,&2)}
\picture@redefine\line{(#1,#2)&3}
\picture@redefine\vector{(#1,#2)&3}
\picture@redefine\dashbox{&1(&2,&3)}
\picture@redefine\@circle{&1}
\picture@redefine\@dot{&1}
\picture@redefine\@bezier{#1(&2,&3)(&4,&5)(&6,&7)}
\picture@redefine\@imakepicbox{(&1,&2)}
%    \end{macrocode}
%
% \subsubsection{Package \xpackage{pspicture}}
%
%     Package \xpackage{pspicture} changes the signature of \cs{@oval}
%     by adding an optional argument.
%    \begin{macrocode}
\@ifpackageloaded{pspicture}{%
  \picture@redefine\@oval{[&1](&2,&3)}%
  \picture@redefine\Line{(&1,&2)}%
  \picture@redefine\Curve{(&1,&2)}%
  \picture@redefine\Vector{(&1,&2)}%
}{%
  \picture@redefine\@oval{(&1,&2)}%
}
%    \end{macrocode}
%
% \subsection{Check package loading order}
%
%    \begin{macro}{\PC@checkpackage}
%    \begin{macrocode}
\def\Pc@checkpackage#1{%
  \@ifpackageloaded{#1}{%
  }{%
    \AtBeginDocument{%
      \@ifpackageloaded{#1}{%
        \PackageWarningNoLine{picture}{%
          Package `#1' is loaded after `picture'.\MessageBreak
          Load package `picture' afterwards to get full support%
          \MessageBreak
          of its additional syntax with length specifications%
        }%
      }{}%
    }%
  }%
}
%    \end{macrocode}
%    \end{macro}
%    \begin{macrocode}
\Pc@checkpackage{pict2e}
\Pc@checkpackage{pspicture}
%    \end{macrocode}
%
%    \begin{macrocode}
%</package>
%    \end{macrocode}
%
% \section{Installation}
%
% \subsection{Download}
%
% \paragraph{Package.} This package is available on
% CTAN\footnote{\url{ftp://ftp.ctan.org/tex-archive/}}:
% \begin{description}
% \item[\CTAN{macros/latex/contrib/oberdiek/picture.dtx}] The source file.
% \item[\CTAN{macros/latex/contrib/oberdiek/picture.pdf}] Documentation.
% \end{description}
%
%
% \paragraph{Bundle.} All the packages of the bundle `oberdiek'
% are also available in a TDS compliant ZIP archive. There
% the packages are already unpacked and the documentation files
% are generated. The files and directories obey the TDS standard.
% \begin{description}
% \item[\CTAN{install/macros/latex/contrib/oberdiek.tds.zip}]
% \end{description}
% \emph{TDS} refers to the standard ``A Directory Structure
% for \TeX\ Files'' (\CTAN{tds/tds.pdf}). Directories
% with \xfile{texmf} in their name are usually organized this way.
%
% \subsection{Bundle installation}
%
% \paragraph{Unpacking.} Unpack the \xfile{oberdiek.tds.zip} in the
% TDS tree (also known as \xfile{texmf} tree) of your choice.
% Example (linux):
% \begin{quote}
%   |unzip oberdiek.tds.zip -d ~/texmf|
% \end{quote}
%
% \paragraph{Script installation.}
% Check the directory \xfile{TDS:scripts/oberdiek/} for
% scripts that need further installation steps.
% Package \xpackage{attachfile2} comes with the Perl script
% \xfile{pdfatfi.pl} that should be installed in such a way
% that it can be called as \texttt{pdfatfi}.
% Example (linux):
% \begin{quote}
%   |chmod +x scripts/oberdiek/pdfatfi.pl|\\
%   |cp scripts/oberdiek/pdfatfi.pl /usr/local/bin/|
% \end{quote}
%
% \subsection{Package installation}
%
% \paragraph{Unpacking.} The \xfile{.dtx} file is a self-extracting
% \docstrip\ archive. The files are extracted by running the
% \xfile{.dtx} through \plainTeX:
% \begin{quote}
%   \verb|tex picture.dtx|
% \end{quote}
%
% \paragraph{TDS.} Now the different files must be moved into
% the different directories in your installation TDS tree
% (also known as \xfile{texmf} tree):
% \begin{quote}
% \def\t{^^A
% \begin{tabular}{@{}>{\ttfamily}l@{ $\rightarrow$ }>{\ttfamily}l@{}}
%   picture.sty & tex/latex/oberdiek/picture.sty\\
%   picture.pdf & doc/latex/oberdiek/picture.pdf\\
%   picture-example.tex & doc/latex/oberdiek/picture-example.tex\\
%   picture.dtx & source/latex/oberdiek/picture.dtx\\
% \end{tabular}^^A
% }^^A
% \sbox0{\t}^^A
% \ifdim\wd0>\linewidth
%   \begingroup
%     \advance\linewidth by\leftmargin
%     \advance\linewidth by\rightmargin
%   \edef\x{\endgroup
%     \def\noexpand\lw{\the\linewidth}^^A
%   }\x
%   \def\lwbox{^^A
%     \leavevmode
%     \hbox to \linewidth{^^A
%       \kern-\leftmargin\relax
%       \hss
%       \usebox0
%       \hss
%       \kern-\rightmargin\relax
%     }^^A
%   }^^A
%   \ifdim\wd0>\lw
%     \sbox0{\small\t}^^A
%     \ifdim\wd0>\linewidth
%       \ifdim\wd0>\lw
%         \sbox0{\footnotesize\t}^^A
%         \ifdim\wd0>\linewidth
%           \ifdim\wd0>\lw
%             \sbox0{\scriptsize\t}^^A
%             \ifdim\wd0>\linewidth
%               \ifdim\wd0>\lw
%                 \sbox0{\tiny\t}^^A
%                 \ifdim\wd0>\linewidth
%                   \lwbox
%                 \else
%                   \usebox0
%                 \fi
%               \else
%                 \lwbox
%               \fi
%             \else
%               \usebox0
%             \fi
%           \else
%             \lwbox
%           \fi
%         \else
%           \usebox0
%         \fi
%       \else
%         \lwbox
%       \fi
%     \else
%       \usebox0
%     \fi
%   \else
%     \lwbox
%   \fi
% \else
%   \usebox0
% \fi
% \end{quote}
% If you have a \xfile{docstrip.cfg} that configures and enables \docstrip's
% TDS installing feature, then some files can already be in the right
% place, see the documentation of \docstrip.
%
% \subsection{Refresh file name databases}
%
% If your \TeX~distribution
% (\teTeX, \mikTeX, \dots) relies on file name databases, you must refresh
% these. For example, \teTeX\ users run \verb|texhash| or
% \verb|mktexlsr|.
%
% \subsection{Some details for the interested}
%
% \paragraph{Attached source.}
%
% The PDF documentation on CTAN also includes the
% \xfile{.dtx} source file. It can be extracted by
% AcrobatReader 6 or higher. Another option is \textsf{pdftk},
% e.g. unpack the file into the current directory:
% \begin{quote}
%   \verb|pdftk picture.pdf unpack_files output .|
% \end{quote}
%
% \paragraph{Unpacking with \LaTeX.}
% The \xfile{.dtx} chooses its action depending on the format:
% \begin{description}
% \item[\plainTeX:] Run \docstrip\ and extract the files.
% \item[\LaTeX:] Generate the documentation.
% \end{description}
% If you insist on using \LaTeX\ for \docstrip\ (really,
% \docstrip\ does not need \LaTeX), then inform the autodetect routine
% about your intention:
% \begin{quote}
%   \verb|latex \let\install=y% \iffalse meta-comment
%
% File: picture.dtx
% Version: 2009/10/11 v1.3
% Info: Dimens for picture macros
%
% Copyright (C) 2006-2009 by
%    Heiko Oberdiek <heiko.oberdiek at googlemail.com>
%
% This work may be distributed and/or modified under the
% conditions of the LaTeX Project Public License, either
% version 1.3c of this license or (at your option) any later
% version. This version of this license is in
%    http://www.latex-project.org/lppl/lppl-1-3c.txt
% and the latest version of this license is in
%    http://www.latex-project.org/lppl.txt
% and version 1.3 or later is part of all distributions of
% LaTeX version 2005/12/01 or later.
%
% This work has the LPPL maintenance status "maintained".
%
% This Current Maintainer of this work is Heiko Oberdiek.
%
% This work consists of the main source file picture.dtx
% and the derived files
%    picture.sty, picture.pdf, picture.ins, picture.drv,
%    picture-example.tex.
%
% Distribution:
%    CTAN:macros/latex/contrib/oberdiek/picture.dtx
%    CTAN:macros/latex/contrib/oberdiek/picture.pdf
%
% Unpacking:
%    (a) If picture.ins is present:
%           tex picture.ins
%    (b) Without picture.ins:
%           tex picture.dtx
%    (c) If you insist on using LaTeX
%           latex \let\install=y\input{picture.dtx}
%        (quote the arguments according to the demands of your shell)
%
% Documentation:
%    (a) If picture.drv is present:
%           latex picture.drv
%    (b) Without picture.drv:
%           latex picture.dtx; ...
%    The class ltxdoc loads the configuration file ltxdoc.cfg
%    if available. Here you can specify further options, e.g.
%    use A4 as paper format:
%       \PassOptionsToClass{a4paper}{article}
%
%    Programm calls to get the documentation (example):
%       pdflatex picture.dtx
%       makeindex -s gind.ist picture.idx
%       pdflatex picture.dtx
%       makeindex -s gind.ist picture.idx
%       pdflatex picture.dtx
%
% Installation:
%    TDS:tex/latex/oberdiek/picture.sty
%    TDS:doc/latex/oberdiek/picture.pdf
%    TDS:doc/latex/oberdiek/picture-example.tex
%    TDS:source/latex/oberdiek/picture.dtx
%
%<*ignore>
\begingroup
  \catcode123=1 %
  \catcode125=2 %
  \def\x{LaTeX2e}%
\expandafter\endgroup
\ifcase 0\ifx\install y1\fi\expandafter
         \ifx\csname processbatchFile\endcsname\relax\else1\fi
         \ifx\fmtname\x\else 1\fi\relax
\else\csname fi\endcsname
%</ignore>
%<*install>
\input docstrip.tex
\Msg{************************************************************************}
\Msg{* Installation}
\Msg{* Package: picture 2009/10/11 v1.3 Dimens for picture macros (HO)}
\Msg{************************************************************************}

\keepsilent
\askforoverwritefalse

\let\MetaPrefix\relax
\preamble

This is a generated file.

Project: picture
Version: 2009/10/11 v1.3

Copyright (C) 2006-2009 by
   Heiko Oberdiek <heiko.oberdiek at googlemail.com>

This work may be distributed and/or modified under the
conditions of the LaTeX Project Public License, either
version 1.3c of this license or (at your option) any later
version. This version of this license is in
   http://www.latex-project.org/lppl/lppl-1-3c.txt
and the latest version of this license is in
   http://www.latex-project.org/lppl.txt
and version 1.3 or later is part of all distributions of
LaTeX version 2005/12/01 or later.

This work has the LPPL maintenance status "maintained".

This Current Maintainer of this work is Heiko Oberdiek.

This work consists of the main source file picture.dtx
and the derived files
   picture.sty, picture.pdf, picture.ins, picture.drv,
   picture-example.tex.

\endpreamble
\let\MetaPrefix\DoubleperCent

\generate{%
  \file{picture.ins}{\from{picture.dtx}{install}}%
  \file{picture.drv}{\from{picture.dtx}{driver}}%
  \usedir{tex/latex/oberdiek}%
  \file{picture.sty}{\from{picture.dtx}{package}}%
  \usedir{doc/latex/oberdiek}%
  \file{picture-example.tex}{\from{picture.dtx}{example}}%
  \nopreamble
  \nopostamble
  \usedir{source/latex/oberdiek/catalogue}%
  \file{picture.xml}{\from{picture.dtx}{catalogue}}%
}

\catcode32=13\relax% active space
\let =\space%
\Msg{************************************************************************}
\Msg{*}
\Msg{* To finish the installation you have to move the following}
\Msg{* file into a directory searched by TeX:}
\Msg{*}
\Msg{*     picture.sty}
\Msg{*}
\Msg{* To produce the documentation run the file `picture.drv'}
\Msg{* through LaTeX.}
\Msg{*}
\Msg{* Happy TeXing!}
\Msg{*}
\Msg{************************************************************************}

\endbatchfile
%</install>
%<*ignore>
\fi
%</ignore>
%<*driver>
\NeedsTeXFormat{LaTeX2e}
\ProvidesFile{picture.drv}%
  [2009/10/11 v1.3 Dimens for picture macros (HO)]%
\documentclass{ltxdoc}
\usepackage{holtxdoc}[2011/11/22]
\begin{document}
  \DocInput{picture.dtx}%
\end{document}
%</driver>
% \fi
%
% \CheckSum{322}
%
% \CharacterTable
%  {Upper-case    \A\B\C\D\E\F\G\H\I\J\K\L\M\N\O\P\Q\R\S\T\U\V\W\X\Y\Z
%   Lower-case    \a\b\c\d\e\f\g\h\i\j\k\l\m\n\o\p\q\r\s\t\u\v\w\x\y\z
%   Digits        \0\1\2\3\4\5\6\7\8\9
%   Exclamation   \!     Double quote  \"     Hash (number) \#
%   Dollar        \$     Percent       \%     Ampersand     \&
%   Acute accent  \'     Left paren    \(     Right paren   \)
%   Asterisk      \*     Plus          \+     Comma         \,
%   Minus         \-     Point         \.     Solidus       \/
%   Colon         \:     Semicolon     \;     Less than     \<
%   Equals        \=     Greater than  \>     Question mark \?
%   Commercial at \@     Left bracket  \[     Backslash     \\
%   Right bracket \]     Circumflex    \^     Underscore    \_
%   Grave accent  \`     Left brace    \{     Vertical bar  \|
%   Right brace   \}     Tilde         \~}
%
% \GetFileInfo{picture.drv}
%
% \title{The \xpackage{picture} package}
% \date{2009/10/11 v1.3}
% \author{Heiko Oberdiek\\\xemail{heiko.oberdiek at googlemail.com}}
%
% \maketitle
%
% \begin{abstract}
% There are macro and environment arguments that expect numbers
% that will internally be multiplicated with \cs{unitlength}.
% This package extends the syntax of these arguments that
% dimens with calculation support can be added for these arguments.
% \end{abstract}
%
% \tableofcontents
%
% \section{User interface}
%
% \subsection{Introduction}
%
% The environment |picture| and macros such as \cs{put}, \cs{line},
% \cs{vector} and other macros have arguments that expect numbers
% that are used as factor for \cs{unitlength}.
% This package redefines such macros and adds code that detects
% whether such an argument is given as number or as length.
% In the latter case, the length is used directly without multiplying
% with \cs{unitlength}.
%
% \subsection{Options}
%
% Depending on the available features, also length expressions
% can be given. Option \xoption{calc} loads package \xpackage{calc}.
% Then expressions of these package may be used.
% Otherwise \xoption{etex} wraps the length argument inside
% \cs{dimexpr}\dots\cs{relax}, if \eTeX\ is available. Otherwise
% option \xoption{plain} uses plain assignments without calculation
% support.
%
% The default is \xoption{calc} if package \xpackage{calc}
% is loaded before package \xoption{picture}. If you specify
% option \xoption{calc} the loading of \xpackage{calc} is ensured.
% Otherwise package \xpackage{picture} looks whether \cs{dimexpr}
% is available and uses then option \xoption{etex} as default.
% If \eTeX\ also could not be found, then \xoption{plain} is used.
%
% \subsection{Example}
%
%    \begin{macrocode}
%<*example>
\documentclass{article}

\usepackage[calc]{picture}

\begin{document}

\setlength{\unitlength}{1pt}

\begin{picture}(\widthof{Hello World}, 10mm)
  \put(0, 0){\makebox(0,0)[lb]{Hello World}}%
  \put(0, \heightof{Hello World} + \fboxsep){%
    \line(1, 0){\widthof{Hello World}}%
  }%
  \put(\widthof{Hello World}, 10mm){%
    \line(0, -1){10mm}%
  }%
\end{picture}

\end{document}
%</example>
%    \end{macrocode}
%
% \subsection{Supported packages}
%
% Packages \xpackage{pspicture} and \xpackage{pict2e} are supported,
% but they must be loaded before package \xpackage{picture}.
%
% New macros can be supported by \cs{picture@redefine}.
% The first argument is the macro which contains the arguments
% in its parameter text that you want to support by package \xpackage{picture}.
% The second argument contains the parameter text. Change \# to \& for
% the arguments in question.
% Examples (already used by package \xpackage{picture}):
% \begin{quote}
% |\picture@redefine\put{(&1,&2)}|\\
% |\picture@redefine\line{(#1,#2)&3}|
% \end{quote}
%
% \StopEventually{
% }
%
% \section{Implementation}
%
% \subsection{Identification}
%
%    \begin{macrocode}
%<*package>
\NeedsTeXFormat{LaTeX2e}
\ProvidesPackage{picture}%
  [2009/10/11 v1.3 Dimens for picture macros (HO)]%
%    \end{macrocode}
%
% \subsection{Options}
%
%    \begin{macrocode}
\def\Pc@calcname{calc}
\def\Pc@etexname{etex}
\def\Pc@plainname{plain}
%    \end{macrocode}
%    \begin{macro}{\Pc@method}
%    Macro \cs{Pc@method} stores the method to use for calculations.
%    Check which features are available and set the default for \cs{Pc@method}.
%    \begin{macrocode}
\@ifpackageloaded{calc}{%
  \let\Pc@method\Pc@calcname
}{%
  \begingroup\expandafter\expandafter\expandafter\endgroup
  \expandafter\ifx\csname dimexpr\endcsname\relax
    \let\Pc@method\Pc@plainname
  \else
    \let\Pc@method\Pc@etexname
  \fi
}
%    \end{macrocode}
%    \end{macro}
%    \begin{macrocode}
\DeclareOption{plain}{%
  \let\Pc@method\Pc@plainname
}
\DeclareOption{etex}{%
  \begingroup\expandafter\expandafter\expandafter\endgroup
  \expandafter\ifx\csname dimexpr\endcsname\relax
    \PackageError{picture}{%
      e-TeX is not available%
    }\@ehc
  \else
    \let\Pc@method\Pc@etexname
  \fi
}
\DeclareOption{calc}{%
  \let\Pc@method\Pc@calcname
}
\ProcessOptions*
\begingroup
  \let\on@line\@empty
  \PackageInfo{picture}{Calculation method: \Pc@method}%
\endgroup
%    \end{macrocode}
%
% \subsection{Calculation method}
%
%    \begin{macrocode}
\ifx\Pc@method\Pc@calcname
  \RequirePackage{calc}%
\fi
%    \end{macrocode}
%
% \subsubsection{Method \xoption{calc}}
%
%    \begin{macrocode}
\ifx\Pc@method\Pc@calcname
  \def\Pc@tokslength#1{%
    \begingroup
      \let\calc@error\Pc@calc@error
      \setlength\dimen@{#1\unitlength}\Pc@next\Pc@nil{#1}%
  }%
  \let\PcOrg@calc@error\calc@error
  \@ifpackagelater{calc}{2007/08/22}{% v4.3
    \def\Pc@calc@error#1{%
      \expandafter\ifx\expandafter\unitlength\noexpand#1\relax
        \def\calc@next##1!{%
          \endgroup
          \aftergroup\afterassignment
          \aftergroup\Pc@next
        }%
        \expandafter\@firstoftwo
      \else
        \expandafter\@secondoftwo
      \fi
      {%
        \calc@next{#1}%
      }{%
        \PcOrg@calc@error{#1}%
      }%
    }%
  }{%
    \def\Pc@calc@error#1{%
      \expandafter\ifx\expandafter\unitlength\noexpand#1\relax
        \def\calc@next##1!{%
          \endgroup
          \aftergroup\afterassignment
          \aftergroup\Pc@next
        }%
        \expandafter\@gobble
      \else
        \expandafter\@firstofone
      \fi
      {%
        \PcOrg@calc@error{#1}%
      }%
    }%
  }%
\fi
%    \end{macrocode}
%
% \subsubsection{Method \xoption{etex}}
%
%    \begin{macrocode}
\ifx\Pc@method\Pc@etexname
  \def\Pc@tokslength#1{%
    \begingroup
      \afterassignment\Pc@next
      \dimen@=\dimexpr#1\unitlength\Pc@nil{#1}%
  }%
\fi
%    \end{macrocode}
%
% \subsubsection{Method \xoption{plain}}
%
%    \begin{macrocode}
\ifx\Pc@method\Pc@plainname
  \def\Pc@tokslength#1{%
    \begingroup
      \afterassignment\Pc@next
      \dimen@=#1\unitlength\Pc@nil{#1}%
  }%
\fi
%    \end{macrocode}
%
% \subsubsection{Help macros}
%
%    \begin{macrocode}
\def\Pc@next#1\Pc@nil#2{%
  \ifx\\#1\\%
    \endgroup
    \Pc@addtoks{{#2}}%
  \else
    \expandafter\endgroup
    \expandafter\Pc@addtoks\expandafter{%
      \expandafter{\the\dimen@\@gobble}%
    }%
  \fi
}
%    \end{macrocode}
%    \begin{macro}{\Pc@nil}
%    \cs{Pc@nil} must not have the meaning of \cs{relax} because of
%    \cs{dimexpr}.
%    \begin{macrocode}
\let\Pc@nil\message
%    \end{macrocode}
%    \end{macro}
%    \begin{macro}{\Pc@addtoks}
%    \begin{macrocode}
\def\Pc@addtoks#1{%
  \toks@=\expandafter{\the\toks@#1}%
}
%    \end{macrocode}
%    \end{macro}
%    \begin{macro}{\Pc@init}
%    \begin{macrocode}
\def\Pc@init#1{%
  \begingroup
    \toks@={#1}%
}
%    \end{macrocode}
%    \end{macro}
%    \begin{macro}{\Pc@finish}
%    \begin{macrocode}
\def\Pc@finish#1{%
  \expandafter\endgroup
  \expandafter#1\the\toks@
}
%    \end{macrocode}
%    \end{macro}
%
% \subsection{Redefinitions}
%
%    \begin{macro}{\picture@redefine}
%    \begin{tabular}[t]{@{}l@{ }l@{}}
%    |#1|:& command name\\
%    |#2|:& parameter text, length parameter with \& instead of \#
%    \end{tabular}
%    \begin{macrocode}
\def\picture@redefine#1#2{%
  \begingroup
    \edef\reserved@a{%
      \noexpand\noexpand
      \expandafter\noexpand
          \csname PcOrg@\expandafter\@gobble\string#1\endcsname
    }%
    \toks0{#1}%
    \Pc@first#2&0%
}
%    \end{macrocode}
%    \end{macro}
%    \begin{macro}{\Pc@first}
%    \begin{macrocode}
\def\Pc@first#1&{%
  \toks1={#1}%
  \toks2={\Pc@init{#1}}%
  \Pc@scanlength
}
%    \end{macrocode}
%    \end{macro}
%    \begin{macro}{\Pc@scanlength}
%    |#1|: number of length parameter or zero
%    \begin{macrocode}
\def\Pc@scanlength#1{%
  \ifcase#1 %
    \expandafter\Pc@last
  \else
    \toks1=\expandafter{\the\toks1 ###1}%
    \toks2=\expandafter{\the\toks2 \Pc@tokslength{###1}}%
    \expandafter\Pc@scannext
  \fi
}
%    \end{macrocode}
%    \end{macro}
%    \begin{macro}{\Pc@scannext}
%    \begin{macrocode}
\def\Pc@scannext#1&{%
  \ifx\\#1\\%
  \else
    \toks1=\expandafter{\the\toks1 #1}%
    \toks2=\expandafter{\the\toks2 \Pc@addtoks{#1}}%
  \fi
  \Pc@scanlength
}
%    \end{macrocode}
%    \end{macro}
%    \begin{macro}{\Pc@last}
%    \begin{macrocode}
\def\Pc@last{%
  \edef\x{%
    \endgroup
    \let\reserved@a\the\toks0 %
    \def\the\toks0 \the\toks1 {%
      \the\toks2 %
      \noexpand\Pc@finish\reserved@a
    }%
  }%
  \x
}
%    \end{macrocode}
%    \end{macro}
%
% \subsubsection{\LaTeX\ base macros}
%
%    \begin{macrocode}
\picture@redefine\@picture{(&1,&2)(&3,&4)}
\picture@redefine\put{(&1,&2)}
\picture@redefine\multiput{(&1,&2)}
\picture@redefine\@multiput{(&1,&2)}
\picture@redefine\line{(#1,#2)&3}
\picture@redefine\vector{(#1,#2)&3}
\picture@redefine\dashbox{&1(&2,&3)}
\picture@redefine\@circle{&1}
\picture@redefine\@dot{&1}
\picture@redefine\@bezier{#1(&2,&3)(&4,&5)(&6,&7)}
\picture@redefine\@imakepicbox{(&1,&2)}
%    \end{macrocode}
%
% \subsubsection{Package \xpackage{pspicture}}
%
%     Package \xpackage{pspicture} changes the signature of \cs{@oval}
%     by adding an optional argument.
%    \begin{macrocode}
\@ifpackageloaded{pspicture}{%
  \picture@redefine\@oval{[&1](&2,&3)}%
  \picture@redefine\Line{(&1,&2)}%
  \picture@redefine\Curve{(&1,&2)}%
  \picture@redefine\Vector{(&1,&2)}%
}{%
  \picture@redefine\@oval{(&1,&2)}%
}
%    \end{macrocode}
%
% \subsection{Check package loading order}
%
%    \begin{macro}{\PC@checkpackage}
%    \begin{macrocode}
\def\Pc@checkpackage#1{%
  \@ifpackageloaded{#1}{%
  }{%
    \AtBeginDocument{%
      \@ifpackageloaded{#1}{%
        \PackageWarningNoLine{picture}{%
          Package `#1' is loaded after `picture'.\MessageBreak
          Load package `picture' afterwards to get full support%
          \MessageBreak
          of its additional syntax with length specifications%
        }%
      }{}%
    }%
  }%
}
%    \end{macrocode}
%    \end{macro}
%    \begin{macrocode}
\Pc@checkpackage{pict2e}
\Pc@checkpackage{pspicture}
%    \end{macrocode}
%
%    \begin{macrocode}
%</package>
%    \end{macrocode}
%
% \section{Installation}
%
% \subsection{Download}
%
% \paragraph{Package.} This package is available on
% CTAN\footnote{\url{ftp://ftp.ctan.org/tex-archive/}}:
% \begin{description}
% \item[\CTAN{macros/latex/contrib/oberdiek/picture.dtx}] The source file.
% \item[\CTAN{macros/latex/contrib/oberdiek/picture.pdf}] Documentation.
% \end{description}
%
%
% \paragraph{Bundle.} All the packages of the bundle `oberdiek'
% are also available in a TDS compliant ZIP archive. There
% the packages are already unpacked and the documentation files
% are generated. The files and directories obey the TDS standard.
% \begin{description}
% \item[\CTAN{install/macros/latex/contrib/oberdiek.tds.zip}]
% \end{description}
% \emph{TDS} refers to the standard ``A Directory Structure
% for \TeX\ Files'' (\CTAN{tds/tds.pdf}). Directories
% with \xfile{texmf} in their name are usually organized this way.
%
% \subsection{Bundle installation}
%
% \paragraph{Unpacking.} Unpack the \xfile{oberdiek.tds.zip} in the
% TDS tree (also known as \xfile{texmf} tree) of your choice.
% Example (linux):
% \begin{quote}
%   |unzip oberdiek.tds.zip -d ~/texmf|
% \end{quote}
%
% \paragraph{Script installation.}
% Check the directory \xfile{TDS:scripts/oberdiek/} for
% scripts that need further installation steps.
% Package \xpackage{attachfile2} comes with the Perl script
% \xfile{pdfatfi.pl} that should be installed in such a way
% that it can be called as \texttt{pdfatfi}.
% Example (linux):
% \begin{quote}
%   |chmod +x scripts/oberdiek/pdfatfi.pl|\\
%   |cp scripts/oberdiek/pdfatfi.pl /usr/local/bin/|
% \end{quote}
%
% \subsection{Package installation}
%
% \paragraph{Unpacking.} The \xfile{.dtx} file is a self-extracting
% \docstrip\ archive. The files are extracted by running the
% \xfile{.dtx} through \plainTeX:
% \begin{quote}
%   \verb|tex picture.dtx|
% \end{quote}
%
% \paragraph{TDS.} Now the different files must be moved into
% the different directories in your installation TDS tree
% (also known as \xfile{texmf} tree):
% \begin{quote}
% \def\t{^^A
% \begin{tabular}{@{}>{\ttfamily}l@{ $\rightarrow$ }>{\ttfamily}l@{}}
%   picture.sty & tex/latex/oberdiek/picture.sty\\
%   picture.pdf & doc/latex/oberdiek/picture.pdf\\
%   picture-example.tex & doc/latex/oberdiek/picture-example.tex\\
%   picture.dtx & source/latex/oberdiek/picture.dtx\\
% \end{tabular}^^A
% }^^A
% \sbox0{\t}^^A
% \ifdim\wd0>\linewidth
%   \begingroup
%     \advance\linewidth by\leftmargin
%     \advance\linewidth by\rightmargin
%   \edef\x{\endgroup
%     \def\noexpand\lw{\the\linewidth}^^A
%   }\x
%   \def\lwbox{^^A
%     \leavevmode
%     \hbox to \linewidth{^^A
%       \kern-\leftmargin\relax
%       \hss
%       \usebox0
%       \hss
%       \kern-\rightmargin\relax
%     }^^A
%   }^^A
%   \ifdim\wd0>\lw
%     \sbox0{\small\t}^^A
%     \ifdim\wd0>\linewidth
%       \ifdim\wd0>\lw
%         \sbox0{\footnotesize\t}^^A
%         \ifdim\wd0>\linewidth
%           \ifdim\wd0>\lw
%             \sbox0{\scriptsize\t}^^A
%             \ifdim\wd0>\linewidth
%               \ifdim\wd0>\lw
%                 \sbox0{\tiny\t}^^A
%                 \ifdim\wd0>\linewidth
%                   \lwbox
%                 \else
%                   \usebox0
%                 \fi
%               \else
%                 \lwbox
%               \fi
%             \else
%               \usebox0
%             \fi
%           \else
%             \lwbox
%           \fi
%         \else
%           \usebox0
%         \fi
%       \else
%         \lwbox
%       \fi
%     \else
%       \usebox0
%     \fi
%   \else
%     \lwbox
%   \fi
% \else
%   \usebox0
% \fi
% \end{quote}
% If you have a \xfile{docstrip.cfg} that configures and enables \docstrip's
% TDS installing feature, then some files can already be in the right
% place, see the documentation of \docstrip.
%
% \subsection{Refresh file name databases}
%
% If your \TeX~distribution
% (\teTeX, \mikTeX, \dots) relies on file name databases, you must refresh
% these. For example, \teTeX\ users run \verb|texhash| or
% \verb|mktexlsr|.
%
% \subsection{Some details for the interested}
%
% \paragraph{Attached source.}
%
% The PDF documentation on CTAN also includes the
% \xfile{.dtx} source file. It can be extracted by
% AcrobatReader 6 or higher. Another option is \textsf{pdftk},
% e.g. unpack the file into the current directory:
% \begin{quote}
%   \verb|pdftk picture.pdf unpack_files output .|
% \end{quote}
%
% \paragraph{Unpacking with \LaTeX.}
% The \xfile{.dtx} chooses its action depending on the format:
% \begin{description}
% \item[\plainTeX:] Run \docstrip\ and extract the files.
% \item[\LaTeX:] Generate the documentation.
% \end{description}
% If you insist on using \LaTeX\ for \docstrip\ (really,
% \docstrip\ does not need \LaTeX), then inform the autodetect routine
% about your intention:
% \begin{quote}
%   \verb|latex \let\install=y\input{picture.dtx}|
% \end{quote}
% Do not forget to quote the argument according to the demands
% of your shell.
%
% \paragraph{Generating the documentation.}
% You can use both the \xfile{.dtx} or the \xfile{.drv} to generate
% the documentation. The process can be configured by the
% configuration file \xfile{ltxdoc.cfg}. For instance, put this
% line into this file, if you want to have A4 as paper format:
% \begin{quote}
%   \verb|\PassOptionsToClass{a4paper}{article}|
% \end{quote}
% An example follows how to generate the
% documentation with pdf\LaTeX:
% \begin{quote}
%\begin{verbatim}
%pdflatex picture.dtx
%makeindex -s gind.ist picture.idx
%pdflatex picture.dtx
%makeindex -s gind.ist picture.idx
%pdflatex picture.dtx
%\end{verbatim}
% \end{quote}
%
% \section{Catalogue}
%
% The following XML file can be used as source for the
% \href{http://mirror.ctan.org/help/Catalogue/catalogue.html}{\TeX\ Catalogue}.
% The elements \texttt{caption} and \texttt{description} are imported
% from the original XML file from the Catalogue.
% The name of the XML file in the Catalogue is \xfile{picture.xml}.
%    \begin{macrocode}
%<*catalogue>
<?xml version='1.0' encoding='us-ascii'?>
<!DOCTYPE entry SYSTEM 'catalogue.dtd'>
<entry datestamp='$Date$' modifier='$Author$' id='picture'>
  <name>picture</name>
  <caption>Dimens for picture macros.</caption>
  <authorref id='auth:oberdiek'/>
  <copyright owner='Heiko Oberdiek' year='2006-2009'/>
  <license type='lppl1.3'/>
  <version number='1.3'/>
  <description>
    There are macro and environment arguments that expect numbers
    that will internally be multiplied by <tt>\unitlength</tt>.
    This package extends the syntax of these arguments, so that
    dimensions with calculation support may be used for these arguments.
    <p/>
    The package is part of the <xref refid='oberdiek'>oberdiek</xref> bundle.
  </description>
  <documentation details='Package documentation'
      href='ctan:/macros/latex/contrib/oberdiek/picture.pdf'/>
  <ctan file='true' path='/macros/latex/contrib/oberdiek/picture.dtx'/>
  <miktex location='oberdiek'/>
  <texlive location='oberdiek'/>
  <install path='/macros/latex/contrib/oberdiek/oberdiek.tds.zip'/>
</entry>
%</catalogue>
%    \end{macrocode}
%
% \begin{History}
%   \begin{Version}{2006/08/26 v1.0}
%   \item
%     First released version. (First start of the project was June/July 2002.)
%   \end{Version}
%   \begin{Version}{2007/04/11 v1.1}
%   \item
%     Line ends sanitized.
%   \end{Version}
%   \begin{Version}{2008/11/26 v1.2}
%   \item
%     Package \xpackage{pict2e} added to documentation section ``Supported
%     packages''.
%   \item
%     Package order of supported packages is checked.
%   \end{Version}
%   \begin{Version}{2009/10/11 v1.3}
%   \item
%     Fix because of new version v4.3 of package \xpackage{calc}.
%   \end{Version}
% \end{History}
%
% \PrintIndex
%
% \Finale
\endinput
|
% \end{quote}
% Do not forget to quote the argument according to the demands
% of your shell.
%
% \paragraph{Generating the documentation.}
% You can use both the \xfile{.dtx} or the \xfile{.drv} to generate
% the documentation. The process can be configured by the
% configuration file \xfile{ltxdoc.cfg}. For instance, put this
% line into this file, if you want to have A4 as paper format:
% \begin{quote}
%   \verb|\PassOptionsToClass{a4paper}{article}|
% \end{quote}
% An example follows how to generate the
% documentation with pdf\LaTeX:
% \begin{quote}
%\begin{verbatim}
%pdflatex picture.dtx
%makeindex -s gind.ist picture.idx
%pdflatex picture.dtx
%makeindex -s gind.ist picture.idx
%pdflatex picture.dtx
%\end{verbatim}
% \end{quote}
%
% \section{Catalogue}
%
% The following XML file can be used as source for the
% \href{http://mirror.ctan.org/help/Catalogue/catalogue.html}{\TeX\ Catalogue}.
% The elements \texttt{caption} and \texttt{description} are imported
% from the original XML file from the Catalogue.
% The name of the XML file in the Catalogue is \xfile{picture.xml}.
%    \begin{macrocode}
%<*catalogue>
<?xml version='1.0' encoding='us-ascii'?>
<!DOCTYPE entry SYSTEM 'catalogue.dtd'>
<entry datestamp='$Date$' modifier='$Author$' id='picture'>
  <name>picture</name>
  <caption>Dimens for picture macros.</caption>
  <authorref id='auth:oberdiek'/>
  <copyright owner='Heiko Oberdiek' year='2006-2009'/>
  <license type='lppl1.3'/>
  <version number='1.3'/>
  <description>
    There are macro and environment arguments that expect numbers
    that will internally be multiplied by <tt>\unitlength</tt>.
    This package extends the syntax of these arguments, so that
    dimensions with calculation support may be used for these arguments.
    <p/>
    The package is part of the <xref refid='oberdiek'>oberdiek</xref> bundle.
  </description>
  <documentation details='Package documentation'
      href='ctan:/macros/latex/contrib/oberdiek/picture.pdf'/>
  <ctan file='true' path='/macros/latex/contrib/oberdiek/picture.dtx'/>
  <miktex location='oberdiek'/>
  <texlive location='oberdiek'/>
  <install path='/macros/latex/contrib/oberdiek/oberdiek.tds.zip'/>
</entry>
%</catalogue>
%    \end{macrocode}
%
% \begin{History}
%   \begin{Version}{2006/08/26 v1.0}
%   \item
%     First released version. (First start of the project was June/July 2002.)
%   \end{Version}
%   \begin{Version}{2007/04/11 v1.1}
%   \item
%     Line ends sanitized.
%   \end{Version}
%   \begin{Version}{2008/11/26 v1.2}
%   \item
%     Package \xpackage{pict2e} added to documentation section ``Supported
%     packages''.
%   \item
%     Package order of supported packages is checked.
%   \end{Version}
%   \begin{Version}{2009/10/11 v1.3}
%   \item
%     Fix because of new version v4.3 of package \xpackage{calc}.
%   \end{Version}
% \end{History}
%
% \PrintIndex
%
% \Finale
\endinput

%        (quote the arguments according to the demands of your shell)
%
% Documentation:
%    (a) If picture.drv is present:
%           latex picture.drv
%    (b) Without picture.drv:
%           latex picture.dtx; ...
%    The class ltxdoc loads the configuration file ltxdoc.cfg
%    if available. Here you can specify further options, e.g.
%    use A4 as paper format:
%       \PassOptionsToClass{a4paper}{article}
%
%    Programm calls to get the documentation (example):
%       pdflatex picture.dtx
%       makeindex -s gind.ist picture.idx
%       pdflatex picture.dtx
%       makeindex -s gind.ist picture.idx
%       pdflatex picture.dtx
%
% Installation:
%    TDS:tex/latex/oberdiek/picture.sty
%    TDS:doc/latex/oberdiek/picture.pdf
%    TDS:doc/latex/oberdiek/picture-example.tex
%    TDS:source/latex/oberdiek/picture.dtx
%
%<*ignore>
\begingroup
  \catcode123=1 %
  \catcode125=2 %
  \def\x{LaTeX2e}%
\expandafter\endgroup
\ifcase 0\ifx\install y1\fi\expandafter
         \ifx\csname processbatchFile\endcsname\relax\else1\fi
         \ifx\fmtname\x\else 1\fi\relax
\else\csname fi\endcsname
%</ignore>
%<*install>
\input docstrip.tex
\Msg{************************************************************************}
\Msg{* Installation}
\Msg{* Package: picture 2009/10/11 v1.3 Dimens for picture macros (HO)}
\Msg{************************************************************************}

\keepsilent
\askforoverwritefalse

\let\MetaPrefix\relax
\preamble

This is a generated file.

Project: picture
Version: 2009/10/11 v1.3

Copyright (C) 2006-2009 by
   Heiko Oberdiek <heiko.oberdiek at googlemail.com>

This work may be distributed and/or modified under the
conditions of the LaTeX Project Public License, either
version 1.3c of this license or (at your option) any later
version. This version of this license is in
   http://www.latex-project.org/lppl/lppl-1-3c.txt
and the latest version of this license is in
   http://www.latex-project.org/lppl.txt
and version 1.3 or later is part of all distributions of
LaTeX version 2005/12/01 or later.

This work has the LPPL maintenance status "maintained".

This Current Maintainer of this work is Heiko Oberdiek.

This work consists of the main source file picture.dtx
and the derived files
   picture.sty, picture.pdf, picture.ins, picture.drv,
   picture-example.tex.

\endpreamble
\let\MetaPrefix\DoubleperCent

\generate{%
  \file{picture.ins}{\from{picture.dtx}{install}}%
  \file{picture.drv}{\from{picture.dtx}{driver}}%
  \usedir{tex/latex/oberdiek}%
  \file{picture.sty}{\from{picture.dtx}{package}}%
  \usedir{doc/latex/oberdiek}%
  \file{picture-example.tex}{\from{picture.dtx}{example}}%
  \nopreamble
  \nopostamble
  \usedir{source/latex/oberdiek/catalogue}%
  \file{picture.xml}{\from{picture.dtx}{catalogue}}%
}

\catcode32=13\relax% active space
\let =\space%
\Msg{************************************************************************}
\Msg{*}
\Msg{* To finish the installation you have to move the following}
\Msg{* file into a directory searched by TeX:}
\Msg{*}
\Msg{*     picture.sty}
\Msg{*}
\Msg{* To produce the documentation run the file `picture.drv'}
\Msg{* through LaTeX.}
\Msg{*}
\Msg{* Happy TeXing!}
\Msg{*}
\Msg{************************************************************************}

\endbatchfile
%</install>
%<*ignore>
\fi
%</ignore>
%<*driver>
\NeedsTeXFormat{LaTeX2e}
\ProvidesFile{picture.drv}%
  [2009/10/11 v1.3 Dimens for picture macros (HO)]%
\documentclass{ltxdoc}
\usepackage{holtxdoc}[2011/11/22]
\begin{document}
  \DocInput{picture.dtx}%
\end{document}
%</driver>
% \fi
%
% \CheckSum{322}
%
% \CharacterTable
%  {Upper-case    \A\B\C\D\E\F\G\H\I\J\K\L\M\N\O\P\Q\R\S\T\U\V\W\X\Y\Z
%   Lower-case    \a\b\c\d\e\f\g\h\i\j\k\l\m\n\o\p\q\r\s\t\u\v\w\x\y\z
%   Digits        \0\1\2\3\4\5\6\7\8\9
%   Exclamation   \!     Double quote  \"     Hash (number) \#
%   Dollar        \$     Percent       \%     Ampersand     \&
%   Acute accent  \'     Left paren    \(     Right paren   \)
%   Asterisk      \*     Plus          \+     Comma         \,
%   Minus         \-     Point         \.     Solidus       \/
%   Colon         \:     Semicolon     \;     Less than     \<
%   Equals        \=     Greater than  \>     Question mark \?
%   Commercial at \@     Left bracket  \[     Backslash     \\
%   Right bracket \]     Circumflex    \^     Underscore    \_
%   Grave accent  \`     Left brace    \{     Vertical bar  \|
%   Right brace   \}     Tilde         \~}
%
% \GetFileInfo{picture.drv}
%
% \title{The \xpackage{picture} package}
% \date{2009/10/11 v1.3}
% \author{Heiko Oberdiek\\\xemail{heiko.oberdiek at googlemail.com}}
%
% \maketitle
%
% \begin{abstract}
% There are macro and environment arguments that expect numbers
% that will internally be multiplicated with \cs{unitlength}.
% This package extends the syntax of these arguments that
% dimens with calculation support can be added for these arguments.
% \end{abstract}
%
% \tableofcontents
%
% \section{User interface}
%
% \subsection{Introduction}
%
% The environment |picture| and macros such as \cs{put}, \cs{line},
% \cs{vector} and other macros have arguments that expect numbers
% that are used as factor for \cs{unitlength}.
% This package redefines such macros and adds code that detects
% whether such an argument is given as number or as length.
% In the latter case, the length is used directly without multiplying
% with \cs{unitlength}.
%
% \subsection{Options}
%
% Depending on the available features, also length expressions
% can be given. Option \xoption{calc} loads package \xpackage{calc}.
% Then expressions of these package may be used.
% Otherwise \xoption{etex} wraps the length argument inside
% \cs{dimexpr}\dots\cs{relax}, if \eTeX\ is available. Otherwise
% option \xoption{plain} uses plain assignments without calculation
% support.
%
% The default is \xoption{calc} if package \xpackage{calc}
% is loaded before package \xoption{picture}. If you specify
% option \xoption{calc} the loading of \xpackage{calc} is ensured.
% Otherwise package \xpackage{picture} looks whether \cs{dimexpr}
% is available and uses then option \xoption{etex} as default.
% If \eTeX\ also could not be found, then \xoption{plain} is used.
%
% \subsection{Example}
%
%    \begin{macrocode}
%<*example>
\documentclass{article}

\usepackage[calc]{picture}

\begin{document}

\setlength{\unitlength}{1pt}

\begin{picture}(\widthof{Hello World}, 10mm)
  \put(0, 0){\makebox(0,0)[lb]{Hello World}}%
  \put(0, \heightof{Hello World} + \fboxsep){%
    \line(1, 0){\widthof{Hello World}}%
  }%
  \put(\widthof{Hello World}, 10mm){%
    \line(0, -1){10mm}%
  }%
\end{picture}

\end{document}
%</example>
%    \end{macrocode}
%
% \subsection{Supported packages}
%
% Packages \xpackage{pspicture} and \xpackage{pict2e} are supported,
% but they must be loaded before package \xpackage{picture}.
%
% New macros can be supported by \cs{picture@redefine}.
% The first argument is the macro which contains the arguments
% in its parameter text that you want to support by package \xpackage{picture}.
% The second argument contains the parameter text. Change \# to \& for
% the arguments in question.
% Examples (already used by package \xpackage{picture}):
% \begin{quote}
% |\picture@redefine\put{(&1,&2)}|\\
% |\picture@redefine\line{(#1,#2)&3}|
% \end{quote}
%
% \StopEventually{
% }
%
% \section{Implementation}
%
% \subsection{Identification}
%
%    \begin{macrocode}
%<*package>
\NeedsTeXFormat{LaTeX2e}
\ProvidesPackage{picture}%
  [2009/10/11 v1.3 Dimens for picture macros (HO)]%
%    \end{macrocode}
%
% \subsection{Options}
%
%    \begin{macrocode}
\def\Pc@calcname{calc}
\def\Pc@etexname{etex}
\def\Pc@plainname{plain}
%    \end{macrocode}
%    \begin{macro}{\Pc@method}
%    Macro \cs{Pc@method} stores the method to use for calculations.
%    Check which features are available and set the default for \cs{Pc@method}.
%    \begin{macrocode}
\@ifpackageloaded{calc}{%
  \let\Pc@method\Pc@calcname
}{%
  \begingroup\expandafter\expandafter\expandafter\endgroup
  \expandafter\ifx\csname dimexpr\endcsname\relax
    \let\Pc@method\Pc@plainname
  \else
    \let\Pc@method\Pc@etexname
  \fi
}
%    \end{macrocode}
%    \end{macro}
%    \begin{macrocode}
\DeclareOption{plain}{%
  \let\Pc@method\Pc@plainname
}
\DeclareOption{etex}{%
  \begingroup\expandafter\expandafter\expandafter\endgroup
  \expandafter\ifx\csname dimexpr\endcsname\relax
    \PackageError{picture}{%
      e-TeX is not available%
    }\@ehc
  \else
    \let\Pc@method\Pc@etexname
  \fi
}
\DeclareOption{calc}{%
  \let\Pc@method\Pc@calcname
}
\ProcessOptions*
\begingroup
  \let\on@line\@empty
  \PackageInfo{picture}{Calculation method: \Pc@method}%
\endgroup
%    \end{macrocode}
%
% \subsection{Calculation method}
%
%    \begin{macrocode}
\ifx\Pc@method\Pc@calcname
  \RequirePackage{calc}%
\fi
%    \end{macrocode}
%
% \subsubsection{Method \xoption{calc}}
%
%    \begin{macrocode}
\ifx\Pc@method\Pc@calcname
  \def\Pc@tokslength#1{%
    \begingroup
      \let\calc@error\Pc@calc@error
      \setlength\dimen@{#1\unitlength}\Pc@next\Pc@nil{#1}%
  }%
  \let\PcOrg@calc@error\calc@error
  \@ifpackagelater{calc}{2007/08/22}{% v4.3
    \def\Pc@calc@error#1{%
      \expandafter\ifx\expandafter\unitlength\noexpand#1\relax
        \def\calc@next##1!{%
          \endgroup
          \aftergroup\afterassignment
          \aftergroup\Pc@next
        }%
        \expandafter\@firstoftwo
      \else
        \expandafter\@secondoftwo
      \fi
      {%
        \calc@next{#1}%
      }{%
        \PcOrg@calc@error{#1}%
      }%
    }%
  }{%
    \def\Pc@calc@error#1{%
      \expandafter\ifx\expandafter\unitlength\noexpand#1\relax
        \def\calc@next##1!{%
          \endgroup
          \aftergroup\afterassignment
          \aftergroup\Pc@next
        }%
        \expandafter\@gobble
      \else
        \expandafter\@firstofone
      \fi
      {%
        \PcOrg@calc@error{#1}%
      }%
    }%
  }%
\fi
%    \end{macrocode}
%
% \subsubsection{Method \xoption{etex}}
%
%    \begin{macrocode}
\ifx\Pc@method\Pc@etexname
  \def\Pc@tokslength#1{%
    \begingroup
      \afterassignment\Pc@next
      \dimen@=\dimexpr#1\unitlength\Pc@nil{#1}%
  }%
\fi
%    \end{macrocode}
%
% \subsubsection{Method \xoption{plain}}
%
%    \begin{macrocode}
\ifx\Pc@method\Pc@plainname
  \def\Pc@tokslength#1{%
    \begingroup
      \afterassignment\Pc@next
      \dimen@=#1\unitlength\Pc@nil{#1}%
  }%
\fi
%    \end{macrocode}
%
% \subsubsection{Help macros}
%
%    \begin{macrocode}
\def\Pc@next#1\Pc@nil#2{%
  \ifx\\#1\\%
    \endgroup
    \Pc@addtoks{{#2}}%
  \else
    \expandafter\endgroup
    \expandafter\Pc@addtoks\expandafter{%
      \expandafter{\the\dimen@\@gobble}%
    }%
  \fi
}
%    \end{macrocode}
%    \begin{macro}{\Pc@nil}
%    \cs{Pc@nil} must not have the meaning of \cs{relax} because of
%    \cs{dimexpr}.
%    \begin{macrocode}
\let\Pc@nil\message
%    \end{macrocode}
%    \end{macro}
%    \begin{macro}{\Pc@addtoks}
%    \begin{macrocode}
\def\Pc@addtoks#1{%
  \toks@=\expandafter{\the\toks@#1}%
}
%    \end{macrocode}
%    \end{macro}
%    \begin{macro}{\Pc@init}
%    \begin{macrocode}
\def\Pc@init#1{%
  \begingroup
    \toks@={#1}%
}
%    \end{macrocode}
%    \end{macro}
%    \begin{macro}{\Pc@finish}
%    \begin{macrocode}
\def\Pc@finish#1{%
  \expandafter\endgroup
  \expandafter#1\the\toks@
}
%    \end{macrocode}
%    \end{macro}
%
% \subsection{Redefinitions}
%
%    \begin{macro}{\picture@redefine}
%    \begin{tabular}[t]{@{}l@{ }l@{}}
%    |#1|:& command name\\
%    |#2|:& parameter text, length parameter with \& instead of \#
%    \end{tabular}
%    \begin{macrocode}
\def\picture@redefine#1#2{%
  \begingroup
    \edef\reserved@a{%
      \noexpand\noexpand
      \expandafter\noexpand
          \csname PcOrg@\expandafter\@gobble\string#1\endcsname
    }%
    \toks0{#1}%
    \Pc@first#2&0%
}
%    \end{macrocode}
%    \end{macro}
%    \begin{macro}{\Pc@first}
%    \begin{macrocode}
\def\Pc@first#1&{%
  \toks1={#1}%
  \toks2={\Pc@init{#1}}%
  \Pc@scanlength
}
%    \end{macrocode}
%    \end{macro}
%    \begin{macro}{\Pc@scanlength}
%    |#1|: number of length parameter or zero
%    \begin{macrocode}
\def\Pc@scanlength#1{%
  \ifcase#1 %
    \expandafter\Pc@last
  \else
    \toks1=\expandafter{\the\toks1 ###1}%
    \toks2=\expandafter{\the\toks2 \Pc@tokslength{###1}}%
    \expandafter\Pc@scannext
  \fi
}
%    \end{macrocode}
%    \end{macro}
%    \begin{macro}{\Pc@scannext}
%    \begin{macrocode}
\def\Pc@scannext#1&{%
  \ifx\\#1\\%
  \else
    \toks1=\expandafter{\the\toks1 #1}%
    \toks2=\expandafter{\the\toks2 \Pc@addtoks{#1}}%
  \fi
  \Pc@scanlength
}
%    \end{macrocode}
%    \end{macro}
%    \begin{macro}{\Pc@last}
%    \begin{macrocode}
\def\Pc@last{%
  \edef\x{%
    \endgroup
    \let\reserved@a\the\toks0 %
    \def\the\toks0 \the\toks1 {%
      \the\toks2 %
      \noexpand\Pc@finish\reserved@a
    }%
  }%
  \x
}
%    \end{macrocode}
%    \end{macro}
%
% \subsubsection{\LaTeX\ base macros}
%
%    \begin{macrocode}
\picture@redefine\@picture{(&1,&2)(&3,&4)}
\picture@redefine\put{(&1,&2)}
\picture@redefine\multiput{(&1,&2)}
\picture@redefine\@multiput{(&1,&2)}
\picture@redefine\line{(#1,#2)&3}
\picture@redefine\vector{(#1,#2)&3}
\picture@redefine\dashbox{&1(&2,&3)}
\picture@redefine\@circle{&1}
\picture@redefine\@dot{&1}
\picture@redefine\@bezier{#1(&2,&3)(&4,&5)(&6,&7)}
\picture@redefine\@imakepicbox{(&1,&2)}
%    \end{macrocode}
%
% \subsubsection{Package \xpackage{pspicture}}
%
%     Package \xpackage{pspicture} changes the signature of \cs{@oval}
%     by adding an optional argument.
%    \begin{macrocode}
\@ifpackageloaded{pspicture}{%
  \picture@redefine\@oval{[&1](&2,&3)}%
  \picture@redefine\Line{(&1,&2)}%
  \picture@redefine\Curve{(&1,&2)}%
  \picture@redefine\Vector{(&1,&2)}%
}{%
  \picture@redefine\@oval{(&1,&2)}%
}
%    \end{macrocode}
%
% \subsection{Check package loading order}
%
%    \begin{macro}{\PC@checkpackage}
%    \begin{macrocode}
\def\Pc@checkpackage#1{%
  \@ifpackageloaded{#1}{%
  }{%
    \AtBeginDocument{%
      \@ifpackageloaded{#1}{%
        \PackageWarningNoLine{picture}{%
          Package `#1' is loaded after `picture'.\MessageBreak
          Load package `picture' afterwards to get full support%
          \MessageBreak
          of its additional syntax with length specifications%
        }%
      }{}%
    }%
  }%
}
%    \end{macrocode}
%    \end{macro}
%    \begin{macrocode}
\Pc@checkpackage{pict2e}
\Pc@checkpackage{pspicture}
%    \end{macrocode}
%
%    \begin{macrocode}
%</package>
%    \end{macrocode}
%
% \section{Installation}
%
% \subsection{Download}
%
% \paragraph{Package.} This package is available on
% CTAN\footnote{\url{ftp://ftp.ctan.org/tex-archive/}}:
% \begin{description}
% \item[\CTAN{macros/latex/contrib/oberdiek/picture.dtx}] The source file.
% \item[\CTAN{macros/latex/contrib/oberdiek/picture.pdf}] Documentation.
% \end{description}
%
%
% \paragraph{Bundle.} All the packages of the bundle `oberdiek'
% are also available in a TDS compliant ZIP archive. There
% the packages are already unpacked and the documentation files
% are generated. The files and directories obey the TDS standard.
% \begin{description}
% \item[\CTAN{install/macros/latex/contrib/oberdiek.tds.zip}]
% \end{description}
% \emph{TDS} refers to the standard ``A Directory Structure
% for \TeX\ Files'' (\CTAN{tds/tds.pdf}). Directories
% with \xfile{texmf} in their name are usually organized this way.
%
% \subsection{Bundle installation}
%
% \paragraph{Unpacking.} Unpack the \xfile{oberdiek.tds.zip} in the
% TDS tree (also known as \xfile{texmf} tree) of your choice.
% Example (linux):
% \begin{quote}
%   |unzip oberdiek.tds.zip -d ~/texmf|
% \end{quote}
%
% \paragraph{Script installation.}
% Check the directory \xfile{TDS:scripts/oberdiek/} for
% scripts that need further installation steps.
% Package \xpackage{attachfile2} comes with the Perl script
% \xfile{pdfatfi.pl} that should be installed in such a way
% that it can be called as \texttt{pdfatfi}.
% Example (linux):
% \begin{quote}
%   |chmod +x scripts/oberdiek/pdfatfi.pl|\\
%   |cp scripts/oberdiek/pdfatfi.pl /usr/local/bin/|
% \end{quote}
%
% \subsection{Package installation}
%
% \paragraph{Unpacking.} The \xfile{.dtx} file is a self-extracting
% \docstrip\ archive. The files are extracted by running the
% \xfile{.dtx} through \plainTeX:
% \begin{quote}
%   \verb|tex picture.dtx|
% \end{quote}
%
% \paragraph{TDS.} Now the different files must be moved into
% the different directories in your installation TDS tree
% (also known as \xfile{texmf} tree):
% \begin{quote}
% \def\t{^^A
% \begin{tabular}{@{}>{\ttfamily}l@{ $\rightarrow$ }>{\ttfamily}l@{}}
%   picture.sty & tex/latex/oberdiek/picture.sty\\
%   picture.pdf & doc/latex/oberdiek/picture.pdf\\
%   picture-example.tex & doc/latex/oberdiek/picture-example.tex\\
%   picture.dtx & source/latex/oberdiek/picture.dtx\\
% \end{tabular}^^A
% }^^A
% \sbox0{\t}^^A
% \ifdim\wd0>\linewidth
%   \begingroup
%     \advance\linewidth by\leftmargin
%     \advance\linewidth by\rightmargin
%   \edef\x{\endgroup
%     \def\noexpand\lw{\the\linewidth}^^A
%   }\x
%   \def\lwbox{^^A
%     \leavevmode
%     \hbox to \linewidth{^^A
%       \kern-\leftmargin\relax
%       \hss
%       \usebox0
%       \hss
%       \kern-\rightmargin\relax
%     }^^A
%   }^^A
%   \ifdim\wd0>\lw
%     \sbox0{\small\t}^^A
%     \ifdim\wd0>\linewidth
%       \ifdim\wd0>\lw
%         \sbox0{\footnotesize\t}^^A
%         \ifdim\wd0>\linewidth
%           \ifdim\wd0>\lw
%             \sbox0{\scriptsize\t}^^A
%             \ifdim\wd0>\linewidth
%               \ifdim\wd0>\lw
%                 \sbox0{\tiny\t}^^A
%                 \ifdim\wd0>\linewidth
%                   \lwbox
%                 \else
%                   \usebox0
%                 \fi
%               \else
%                 \lwbox
%               \fi
%             \else
%               \usebox0
%             \fi
%           \else
%             \lwbox
%           \fi
%         \else
%           \usebox0
%         \fi
%       \else
%         \lwbox
%       \fi
%     \else
%       \usebox0
%     \fi
%   \else
%     \lwbox
%   \fi
% \else
%   \usebox0
% \fi
% \end{quote}
% If you have a \xfile{docstrip.cfg} that configures and enables \docstrip's
% TDS installing feature, then some files can already be in the right
% place, see the documentation of \docstrip.
%
% \subsection{Refresh file name databases}
%
% If your \TeX~distribution
% (\teTeX, \mikTeX, \dots) relies on file name databases, you must refresh
% these. For example, \teTeX\ users run \verb|texhash| or
% \verb|mktexlsr|.
%
% \subsection{Some details for the interested}
%
% \paragraph{Attached source.}
%
% The PDF documentation on CTAN also includes the
% \xfile{.dtx} source file. It can be extracted by
% AcrobatReader 6 or higher. Another option is \textsf{pdftk},
% e.g. unpack the file into the current directory:
% \begin{quote}
%   \verb|pdftk picture.pdf unpack_files output .|
% \end{quote}
%
% \paragraph{Unpacking with \LaTeX.}
% The \xfile{.dtx} chooses its action depending on the format:
% \begin{description}
% \item[\plainTeX:] Run \docstrip\ and extract the files.
% \item[\LaTeX:] Generate the documentation.
% \end{description}
% If you insist on using \LaTeX\ for \docstrip\ (really,
% \docstrip\ does not need \LaTeX), then inform the autodetect routine
% about your intention:
% \begin{quote}
%   \verb|latex \let\install=y% \iffalse meta-comment
%
% File: picture.dtx
% Version: 2009/10/11 v1.3
% Info: Dimens for picture macros
%
% Copyright (C) 2006-2009 by
%    Heiko Oberdiek <heiko.oberdiek at googlemail.com>
%
% This work may be distributed and/or modified under the
% conditions of the LaTeX Project Public License, either
% version 1.3c of this license or (at your option) any later
% version. This version of this license is in
%    http://www.latex-project.org/lppl/lppl-1-3c.txt
% and the latest version of this license is in
%    http://www.latex-project.org/lppl.txt
% and version 1.3 or later is part of all distributions of
% LaTeX version 2005/12/01 or later.
%
% This work has the LPPL maintenance status "maintained".
%
% This Current Maintainer of this work is Heiko Oberdiek.
%
% This work consists of the main source file picture.dtx
% and the derived files
%    picture.sty, picture.pdf, picture.ins, picture.drv,
%    picture-example.tex.
%
% Distribution:
%    CTAN:macros/latex/contrib/oberdiek/picture.dtx
%    CTAN:macros/latex/contrib/oberdiek/picture.pdf
%
% Unpacking:
%    (a) If picture.ins is present:
%           tex picture.ins
%    (b) Without picture.ins:
%           tex picture.dtx
%    (c) If you insist on using LaTeX
%           latex \let\install=y% \iffalse meta-comment
%
% File: picture.dtx
% Version: 2009/10/11 v1.3
% Info: Dimens for picture macros
%
% Copyright (C) 2006-2009 by
%    Heiko Oberdiek <heiko.oberdiek at googlemail.com>
%
% This work may be distributed and/or modified under the
% conditions of the LaTeX Project Public License, either
% version 1.3c of this license or (at your option) any later
% version. This version of this license is in
%    http://www.latex-project.org/lppl/lppl-1-3c.txt
% and the latest version of this license is in
%    http://www.latex-project.org/lppl.txt
% and version 1.3 or later is part of all distributions of
% LaTeX version 2005/12/01 or later.
%
% This work has the LPPL maintenance status "maintained".
%
% This Current Maintainer of this work is Heiko Oberdiek.
%
% This work consists of the main source file picture.dtx
% and the derived files
%    picture.sty, picture.pdf, picture.ins, picture.drv,
%    picture-example.tex.
%
% Distribution:
%    CTAN:macros/latex/contrib/oberdiek/picture.dtx
%    CTAN:macros/latex/contrib/oberdiek/picture.pdf
%
% Unpacking:
%    (a) If picture.ins is present:
%           tex picture.ins
%    (b) Without picture.ins:
%           tex picture.dtx
%    (c) If you insist on using LaTeX
%           latex \let\install=y\input{picture.dtx}
%        (quote the arguments according to the demands of your shell)
%
% Documentation:
%    (a) If picture.drv is present:
%           latex picture.drv
%    (b) Without picture.drv:
%           latex picture.dtx; ...
%    The class ltxdoc loads the configuration file ltxdoc.cfg
%    if available. Here you can specify further options, e.g.
%    use A4 as paper format:
%       \PassOptionsToClass{a4paper}{article}
%
%    Programm calls to get the documentation (example):
%       pdflatex picture.dtx
%       makeindex -s gind.ist picture.idx
%       pdflatex picture.dtx
%       makeindex -s gind.ist picture.idx
%       pdflatex picture.dtx
%
% Installation:
%    TDS:tex/latex/oberdiek/picture.sty
%    TDS:doc/latex/oberdiek/picture.pdf
%    TDS:doc/latex/oberdiek/picture-example.tex
%    TDS:source/latex/oberdiek/picture.dtx
%
%<*ignore>
\begingroup
  \catcode123=1 %
  \catcode125=2 %
  \def\x{LaTeX2e}%
\expandafter\endgroup
\ifcase 0\ifx\install y1\fi\expandafter
         \ifx\csname processbatchFile\endcsname\relax\else1\fi
         \ifx\fmtname\x\else 1\fi\relax
\else\csname fi\endcsname
%</ignore>
%<*install>
\input docstrip.tex
\Msg{************************************************************************}
\Msg{* Installation}
\Msg{* Package: picture 2009/10/11 v1.3 Dimens for picture macros (HO)}
\Msg{************************************************************************}

\keepsilent
\askforoverwritefalse

\let\MetaPrefix\relax
\preamble

This is a generated file.

Project: picture
Version: 2009/10/11 v1.3

Copyright (C) 2006-2009 by
   Heiko Oberdiek <heiko.oberdiek at googlemail.com>

This work may be distributed and/or modified under the
conditions of the LaTeX Project Public License, either
version 1.3c of this license or (at your option) any later
version. This version of this license is in
   http://www.latex-project.org/lppl/lppl-1-3c.txt
and the latest version of this license is in
   http://www.latex-project.org/lppl.txt
and version 1.3 or later is part of all distributions of
LaTeX version 2005/12/01 or later.

This work has the LPPL maintenance status "maintained".

This Current Maintainer of this work is Heiko Oberdiek.

This work consists of the main source file picture.dtx
and the derived files
   picture.sty, picture.pdf, picture.ins, picture.drv,
   picture-example.tex.

\endpreamble
\let\MetaPrefix\DoubleperCent

\generate{%
  \file{picture.ins}{\from{picture.dtx}{install}}%
  \file{picture.drv}{\from{picture.dtx}{driver}}%
  \usedir{tex/latex/oberdiek}%
  \file{picture.sty}{\from{picture.dtx}{package}}%
  \usedir{doc/latex/oberdiek}%
  \file{picture-example.tex}{\from{picture.dtx}{example}}%
  \nopreamble
  \nopostamble
  \usedir{source/latex/oberdiek/catalogue}%
  \file{picture.xml}{\from{picture.dtx}{catalogue}}%
}

\catcode32=13\relax% active space
\let =\space%
\Msg{************************************************************************}
\Msg{*}
\Msg{* To finish the installation you have to move the following}
\Msg{* file into a directory searched by TeX:}
\Msg{*}
\Msg{*     picture.sty}
\Msg{*}
\Msg{* To produce the documentation run the file `picture.drv'}
\Msg{* through LaTeX.}
\Msg{*}
\Msg{* Happy TeXing!}
\Msg{*}
\Msg{************************************************************************}

\endbatchfile
%</install>
%<*ignore>
\fi
%</ignore>
%<*driver>
\NeedsTeXFormat{LaTeX2e}
\ProvidesFile{picture.drv}%
  [2009/10/11 v1.3 Dimens for picture macros (HO)]%
\documentclass{ltxdoc}
\usepackage{holtxdoc}[2011/11/22]
\begin{document}
  \DocInput{picture.dtx}%
\end{document}
%</driver>
% \fi
%
% \CheckSum{322}
%
% \CharacterTable
%  {Upper-case    \A\B\C\D\E\F\G\H\I\J\K\L\M\N\O\P\Q\R\S\T\U\V\W\X\Y\Z
%   Lower-case    \a\b\c\d\e\f\g\h\i\j\k\l\m\n\o\p\q\r\s\t\u\v\w\x\y\z
%   Digits        \0\1\2\3\4\5\6\7\8\9
%   Exclamation   \!     Double quote  \"     Hash (number) \#
%   Dollar        \$     Percent       \%     Ampersand     \&
%   Acute accent  \'     Left paren    \(     Right paren   \)
%   Asterisk      \*     Plus          \+     Comma         \,
%   Minus         \-     Point         \.     Solidus       \/
%   Colon         \:     Semicolon     \;     Less than     \<
%   Equals        \=     Greater than  \>     Question mark \?
%   Commercial at \@     Left bracket  \[     Backslash     \\
%   Right bracket \]     Circumflex    \^     Underscore    \_
%   Grave accent  \`     Left brace    \{     Vertical bar  \|
%   Right brace   \}     Tilde         \~}
%
% \GetFileInfo{picture.drv}
%
% \title{The \xpackage{picture} package}
% \date{2009/10/11 v1.3}
% \author{Heiko Oberdiek\\\xemail{heiko.oberdiek at googlemail.com}}
%
% \maketitle
%
% \begin{abstract}
% There are macro and environment arguments that expect numbers
% that will internally be multiplicated with \cs{unitlength}.
% This package extends the syntax of these arguments that
% dimens with calculation support can be added for these arguments.
% \end{abstract}
%
% \tableofcontents
%
% \section{User interface}
%
% \subsection{Introduction}
%
% The environment |picture| and macros such as \cs{put}, \cs{line},
% \cs{vector} and other macros have arguments that expect numbers
% that are used as factor for \cs{unitlength}.
% This package redefines such macros and adds code that detects
% whether such an argument is given as number or as length.
% In the latter case, the length is used directly without multiplying
% with \cs{unitlength}.
%
% \subsection{Options}
%
% Depending on the available features, also length expressions
% can be given. Option \xoption{calc} loads package \xpackage{calc}.
% Then expressions of these package may be used.
% Otherwise \xoption{etex} wraps the length argument inside
% \cs{dimexpr}\dots\cs{relax}, if \eTeX\ is available. Otherwise
% option \xoption{plain} uses plain assignments without calculation
% support.
%
% The default is \xoption{calc} if package \xpackage{calc}
% is loaded before package \xoption{picture}. If you specify
% option \xoption{calc} the loading of \xpackage{calc} is ensured.
% Otherwise package \xpackage{picture} looks whether \cs{dimexpr}
% is available and uses then option \xoption{etex} as default.
% If \eTeX\ also could not be found, then \xoption{plain} is used.
%
% \subsection{Example}
%
%    \begin{macrocode}
%<*example>
\documentclass{article}

\usepackage[calc]{picture}

\begin{document}

\setlength{\unitlength}{1pt}

\begin{picture}(\widthof{Hello World}, 10mm)
  \put(0, 0){\makebox(0,0)[lb]{Hello World}}%
  \put(0, \heightof{Hello World} + \fboxsep){%
    \line(1, 0){\widthof{Hello World}}%
  }%
  \put(\widthof{Hello World}, 10mm){%
    \line(0, -1){10mm}%
  }%
\end{picture}

\end{document}
%</example>
%    \end{macrocode}
%
% \subsection{Supported packages}
%
% Packages \xpackage{pspicture} and \xpackage{pict2e} are supported,
% but they must be loaded before package \xpackage{picture}.
%
% New macros can be supported by \cs{picture@redefine}.
% The first argument is the macro which contains the arguments
% in its parameter text that you want to support by package \xpackage{picture}.
% The second argument contains the parameter text. Change \# to \& for
% the arguments in question.
% Examples (already used by package \xpackage{picture}):
% \begin{quote}
% |\picture@redefine\put{(&1,&2)}|\\
% |\picture@redefine\line{(#1,#2)&3}|
% \end{quote}
%
% \StopEventually{
% }
%
% \section{Implementation}
%
% \subsection{Identification}
%
%    \begin{macrocode}
%<*package>
\NeedsTeXFormat{LaTeX2e}
\ProvidesPackage{picture}%
  [2009/10/11 v1.3 Dimens for picture macros (HO)]%
%    \end{macrocode}
%
% \subsection{Options}
%
%    \begin{macrocode}
\def\Pc@calcname{calc}
\def\Pc@etexname{etex}
\def\Pc@plainname{plain}
%    \end{macrocode}
%    \begin{macro}{\Pc@method}
%    Macro \cs{Pc@method} stores the method to use for calculations.
%    Check which features are available and set the default for \cs{Pc@method}.
%    \begin{macrocode}
\@ifpackageloaded{calc}{%
  \let\Pc@method\Pc@calcname
}{%
  \begingroup\expandafter\expandafter\expandafter\endgroup
  \expandafter\ifx\csname dimexpr\endcsname\relax
    \let\Pc@method\Pc@plainname
  \else
    \let\Pc@method\Pc@etexname
  \fi
}
%    \end{macrocode}
%    \end{macro}
%    \begin{macrocode}
\DeclareOption{plain}{%
  \let\Pc@method\Pc@plainname
}
\DeclareOption{etex}{%
  \begingroup\expandafter\expandafter\expandafter\endgroup
  \expandafter\ifx\csname dimexpr\endcsname\relax
    \PackageError{picture}{%
      e-TeX is not available%
    }\@ehc
  \else
    \let\Pc@method\Pc@etexname
  \fi
}
\DeclareOption{calc}{%
  \let\Pc@method\Pc@calcname
}
\ProcessOptions*
\begingroup
  \let\on@line\@empty
  \PackageInfo{picture}{Calculation method: \Pc@method}%
\endgroup
%    \end{macrocode}
%
% \subsection{Calculation method}
%
%    \begin{macrocode}
\ifx\Pc@method\Pc@calcname
  \RequirePackage{calc}%
\fi
%    \end{macrocode}
%
% \subsubsection{Method \xoption{calc}}
%
%    \begin{macrocode}
\ifx\Pc@method\Pc@calcname
  \def\Pc@tokslength#1{%
    \begingroup
      \let\calc@error\Pc@calc@error
      \setlength\dimen@{#1\unitlength}\Pc@next\Pc@nil{#1}%
  }%
  \let\PcOrg@calc@error\calc@error
  \@ifpackagelater{calc}{2007/08/22}{% v4.3
    \def\Pc@calc@error#1{%
      \expandafter\ifx\expandafter\unitlength\noexpand#1\relax
        \def\calc@next##1!{%
          \endgroup
          \aftergroup\afterassignment
          \aftergroup\Pc@next
        }%
        \expandafter\@firstoftwo
      \else
        \expandafter\@secondoftwo
      \fi
      {%
        \calc@next{#1}%
      }{%
        \PcOrg@calc@error{#1}%
      }%
    }%
  }{%
    \def\Pc@calc@error#1{%
      \expandafter\ifx\expandafter\unitlength\noexpand#1\relax
        \def\calc@next##1!{%
          \endgroup
          \aftergroup\afterassignment
          \aftergroup\Pc@next
        }%
        \expandafter\@gobble
      \else
        \expandafter\@firstofone
      \fi
      {%
        \PcOrg@calc@error{#1}%
      }%
    }%
  }%
\fi
%    \end{macrocode}
%
% \subsubsection{Method \xoption{etex}}
%
%    \begin{macrocode}
\ifx\Pc@method\Pc@etexname
  \def\Pc@tokslength#1{%
    \begingroup
      \afterassignment\Pc@next
      \dimen@=\dimexpr#1\unitlength\Pc@nil{#1}%
  }%
\fi
%    \end{macrocode}
%
% \subsubsection{Method \xoption{plain}}
%
%    \begin{macrocode}
\ifx\Pc@method\Pc@plainname
  \def\Pc@tokslength#1{%
    \begingroup
      \afterassignment\Pc@next
      \dimen@=#1\unitlength\Pc@nil{#1}%
  }%
\fi
%    \end{macrocode}
%
% \subsubsection{Help macros}
%
%    \begin{macrocode}
\def\Pc@next#1\Pc@nil#2{%
  \ifx\\#1\\%
    \endgroup
    \Pc@addtoks{{#2}}%
  \else
    \expandafter\endgroup
    \expandafter\Pc@addtoks\expandafter{%
      \expandafter{\the\dimen@\@gobble}%
    }%
  \fi
}
%    \end{macrocode}
%    \begin{macro}{\Pc@nil}
%    \cs{Pc@nil} must not have the meaning of \cs{relax} because of
%    \cs{dimexpr}.
%    \begin{macrocode}
\let\Pc@nil\message
%    \end{macrocode}
%    \end{macro}
%    \begin{macro}{\Pc@addtoks}
%    \begin{macrocode}
\def\Pc@addtoks#1{%
  \toks@=\expandafter{\the\toks@#1}%
}
%    \end{macrocode}
%    \end{macro}
%    \begin{macro}{\Pc@init}
%    \begin{macrocode}
\def\Pc@init#1{%
  \begingroup
    \toks@={#1}%
}
%    \end{macrocode}
%    \end{macro}
%    \begin{macro}{\Pc@finish}
%    \begin{macrocode}
\def\Pc@finish#1{%
  \expandafter\endgroup
  \expandafter#1\the\toks@
}
%    \end{macrocode}
%    \end{macro}
%
% \subsection{Redefinitions}
%
%    \begin{macro}{\picture@redefine}
%    \begin{tabular}[t]{@{}l@{ }l@{}}
%    |#1|:& command name\\
%    |#2|:& parameter text, length parameter with \& instead of \#
%    \end{tabular}
%    \begin{macrocode}
\def\picture@redefine#1#2{%
  \begingroup
    \edef\reserved@a{%
      \noexpand\noexpand
      \expandafter\noexpand
          \csname PcOrg@\expandafter\@gobble\string#1\endcsname
    }%
    \toks0{#1}%
    \Pc@first#2&0%
}
%    \end{macrocode}
%    \end{macro}
%    \begin{macro}{\Pc@first}
%    \begin{macrocode}
\def\Pc@first#1&{%
  \toks1={#1}%
  \toks2={\Pc@init{#1}}%
  \Pc@scanlength
}
%    \end{macrocode}
%    \end{macro}
%    \begin{macro}{\Pc@scanlength}
%    |#1|: number of length parameter or zero
%    \begin{macrocode}
\def\Pc@scanlength#1{%
  \ifcase#1 %
    \expandafter\Pc@last
  \else
    \toks1=\expandafter{\the\toks1 ###1}%
    \toks2=\expandafter{\the\toks2 \Pc@tokslength{###1}}%
    \expandafter\Pc@scannext
  \fi
}
%    \end{macrocode}
%    \end{macro}
%    \begin{macro}{\Pc@scannext}
%    \begin{macrocode}
\def\Pc@scannext#1&{%
  \ifx\\#1\\%
  \else
    \toks1=\expandafter{\the\toks1 #1}%
    \toks2=\expandafter{\the\toks2 \Pc@addtoks{#1}}%
  \fi
  \Pc@scanlength
}
%    \end{macrocode}
%    \end{macro}
%    \begin{macro}{\Pc@last}
%    \begin{macrocode}
\def\Pc@last{%
  \edef\x{%
    \endgroup
    \let\reserved@a\the\toks0 %
    \def\the\toks0 \the\toks1 {%
      \the\toks2 %
      \noexpand\Pc@finish\reserved@a
    }%
  }%
  \x
}
%    \end{macrocode}
%    \end{macro}
%
% \subsubsection{\LaTeX\ base macros}
%
%    \begin{macrocode}
\picture@redefine\@picture{(&1,&2)(&3,&4)}
\picture@redefine\put{(&1,&2)}
\picture@redefine\multiput{(&1,&2)}
\picture@redefine\@multiput{(&1,&2)}
\picture@redefine\line{(#1,#2)&3}
\picture@redefine\vector{(#1,#2)&3}
\picture@redefine\dashbox{&1(&2,&3)}
\picture@redefine\@circle{&1}
\picture@redefine\@dot{&1}
\picture@redefine\@bezier{#1(&2,&3)(&4,&5)(&6,&7)}
\picture@redefine\@imakepicbox{(&1,&2)}
%    \end{macrocode}
%
% \subsubsection{Package \xpackage{pspicture}}
%
%     Package \xpackage{pspicture} changes the signature of \cs{@oval}
%     by adding an optional argument.
%    \begin{macrocode}
\@ifpackageloaded{pspicture}{%
  \picture@redefine\@oval{[&1](&2,&3)}%
  \picture@redefine\Line{(&1,&2)}%
  \picture@redefine\Curve{(&1,&2)}%
  \picture@redefine\Vector{(&1,&2)}%
}{%
  \picture@redefine\@oval{(&1,&2)}%
}
%    \end{macrocode}
%
% \subsection{Check package loading order}
%
%    \begin{macro}{\PC@checkpackage}
%    \begin{macrocode}
\def\Pc@checkpackage#1{%
  \@ifpackageloaded{#1}{%
  }{%
    \AtBeginDocument{%
      \@ifpackageloaded{#1}{%
        \PackageWarningNoLine{picture}{%
          Package `#1' is loaded after `picture'.\MessageBreak
          Load package `picture' afterwards to get full support%
          \MessageBreak
          of its additional syntax with length specifications%
        }%
      }{}%
    }%
  }%
}
%    \end{macrocode}
%    \end{macro}
%    \begin{macrocode}
\Pc@checkpackage{pict2e}
\Pc@checkpackage{pspicture}
%    \end{macrocode}
%
%    \begin{macrocode}
%</package>
%    \end{macrocode}
%
% \section{Installation}
%
% \subsection{Download}
%
% \paragraph{Package.} This package is available on
% CTAN\footnote{\url{ftp://ftp.ctan.org/tex-archive/}}:
% \begin{description}
% \item[\CTAN{macros/latex/contrib/oberdiek/picture.dtx}] The source file.
% \item[\CTAN{macros/latex/contrib/oberdiek/picture.pdf}] Documentation.
% \end{description}
%
%
% \paragraph{Bundle.} All the packages of the bundle `oberdiek'
% are also available in a TDS compliant ZIP archive. There
% the packages are already unpacked and the documentation files
% are generated. The files and directories obey the TDS standard.
% \begin{description}
% \item[\CTAN{install/macros/latex/contrib/oberdiek.tds.zip}]
% \end{description}
% \emph{TDS} refers to the standard ``A Directory Structure
% for \TeX\ Files'' (\CTAN{tds/tds.pdf}). Directories
% with \xfile{texmf} in their name are usually organized this way.
%
% \subsection{Bundle installation}
%
% \paragraph{Unpacking.} Unpack the \xfile{oberdiek.tds.zip} in the
% TDS tree (also known as \xfile{texmf} tree) of your choice.
% Example (linux):
% \begin{quote}
%   |unzip oberdiek.tds.zip -d ~/texmf|
% \end{quote}
%
% \paragraph{Script installation.}
% Check the directory \xfile{TDS:scripts/oberdiek/} for
% scripts that need further installation steps.
% Package \xpackage{attachfile2} comes with the Perl script
% \xfile{pdfatfi.pl} that should be installed in such a way
% that it can be called as \texttt{pdfatfi}.
% Example (linux):
% \begin{quote}
%   |chmod +x scripts/oberdiek/pdfatfi.pl|\\
%   |cp scripts/oberdiek/pdfatfi.pl /usr/local/bin/|
% \end{quote}
%
% \subsection{Package installation}
%
% \paragraph{Unpacking.} The \xfile{.dtx} file is a self-extracting
% \docstrip\ archive. The files are extracted by running the
% \xfile{.dtx} through \plainTeX:
% \begin{quote}
%   \verb|tex picture.dtx|
% \end{quote}
%
% \paragraph{TDS.} Now the different files must be moved into
% the different directories in your installation TDS tree
% (also known as \xfile{texmf} tree):
% \begin{quote}
% \def\t{^^A
% \begin{tabular}{@{}>{\ttfamily}l@{ $\rightarrow$ }>{\ttfamily}l@{}}
%   picture.sty & tex/latex/oberdiek/picture.sty\\
%   picture.pdf & doc/latex/oberdiek/picture.pdf\\
%   picture-example.tex & doc/latex/oberdiek/picture-example.tex\\
%   picture.dtx & source/latex/oberdiek/picture.dtx\\
% \end{tabular}^^A
% }^^A
% \sbox0{\t}^^A
% \ifdim\wd0>\linewidth
%   \begingroup
%     \advance\linewidth by\leftmargin
%     \advance\linewidth by\rightmargin
%   \edef\x{\endgroup
%     \def\noexpand\lw{\the\linewidth}^^A
%   }\x
%   \def\lwbox{^^A
%     \leavevmode
%     \hbox to \linewidth{^^A
%       \kern-\leftmargin\relax
%       \hss
%       \usebox0
%       \hss
%       \kern-\rightmargin\relax
%     }^^A
%   }^^A
%   \ifdim\wd0>\lw
%     \sbox0{\small\t}^^A
%     \ifdim\wd0>\linewidth
%       \ifdim\wd0>\lw
%         \sbox0{\footnotesize\t}^^A
%         \ifdim\wd0>\linewidth
%           \ifdim\wd0>\lw
%             \sbox0{\scriptsize\t}^^A
%             \ifdim\wd0>\linewidth
%               \ifdim\wd0>\lw
%                 \sbox0{\tiny\t}^^A
%                 \ifdim\wd0>\linewidth
%                   \lwbox
%                 \else
%                   \usebox0
%                 \fi
%               \else
%                 \lwbox
%               \fi
%             \else
%               \usebox0
%             \fi
%           \else
%             \lwbox
%           \fi
%         \else
%           \usebox0
%         \fi
%       \else
%         \lwbox
%       \fi
%     \else
%       \usebox0
%     \fi
%   \else
%     \lwbox
%   \fi
% \else
%   \usebox0
% \fi
% \end{quote}
% If you have a \xfile{docstrip.cfg} that configures and enables \docstrip's
% TDS installing feature, then some files can already be in the right
% place, see the documentation of \docstrip.
%
% \subsection{Refresh file name databases}
%
% If your \TeX~distribution
% (\teTeX, \mikTeX, \dots) relies on file name databases, you must refresh
% these. For example, \teTeX\ users run \verb|texhash| or
% \verb|mktexlsr|.
%
% \subsection{Some details for the interested}
%
% \paragraph{Attached source.}
%
% The PDF documentation on CTAN also includes the
% \xfile{.dtx} source file. It can be extracted by
% AcrobatReader 6 or higher. Another option is \textsf{pdftk},
% e.g. unpack the file into the current directory:
% \begin{quote}
%   \verb|pdftk picture.pdf unpack_files output .|
% \end{quote}
%
% \paragraph{Unpacking with \LaTeX.}
% The \xfile{.dtx} chooses its action depending on the format:
% \begin{description}
% \item[\plainTeX:] Run \docstrip\ and extract the files.
% \item[\LaTeX:] Generate the documentation.
% \end{description}
% If you insist on using \LaTeX\ for \docstrip\ (really,
% \docstrip\ does not need \LaTeX), then inform the autodetect routine
% about your intention:
% \begin{quote}
%   \verb|latex \let\install=y\input{picture.dtx}|
% \end{quote}
% Do not forget to quote the argument according to the demands
% of your shell.
%
% \paragraph{Generating the documentation.}
% You can use both the \xfile{.dtx} or the \xfile{.drv} to generate
% the documentation. The process can be configured by the
% configuration file \xfile{ltxdoc.cfg}. For instance, put this
% line into this file, if you want to have A4 as paper format:
% \begin{quote}
%   \verb|\PassOptionsToClass{a4paper}{article}|
% \end{quote}
% An example follows how to generate the
% documentation with pdf\LaTeX:
% \begin{quote}
%\begin{verbatim}
%pdflatex picture.dtx
%makeindex -s gind.ist picture.idx
%pdflatex picture.dtx
%makeindex -s gind.ist picture.idx
%pdflatex picture.dtx
%\end{verbatim}
% \end{quote}
%
% \section{Catalogue}
%
% The following XML file can be used as source for the
% \href{http://mirror.ctan.org/help/Catalogue/catalogue.html}{\TeX\ Catalogue}.
% The elements \texttt{caption} and \texttt{description} are imported
% from the original XML file from the Catalogue.
% The name of the XML file in the Catalogue is \xfile{picture.xml}.
%    \begin{macrocode}
%<*catalogue>
<?xml version='1.0' encoding='us-ascii'?>
<!DOCTYPE entry SYSTEM 'catalogue.dtd'>
<entry datestamp='$Date$' modifier='$Author$' id='picture'>
  <name>picture</name>
  <caption>Dimens for picture macros.</caption>
  <authorref id='auth:oberdiek'/>
  <copyright owner='Heiko Oberdiek' year='2006-2009'/>
  <license type='lppl1.3'/>
  <version number='1.3'/>
  <description>
    There are macro and environment arguments that expect numbers
    that will internally be multiplied by <tt>\unitlength</tt>.
    This package extends the syntax of these arguments, so that
    dimensions with calculation support may be used for these arguments.
    <p/>
    The package is part of the <xref refid='oberdiek'>oberdiek</xref> bundle.
  </description>
  <documentation details='Package documentation'
      href='ctan:/macros/latex/contrib/oberdiek/picture.pdf'/>
  <ctan file='true' path='/macros/latex/contrib/oberdiek/picture.dtx'/>
  <miktex location='oberdiek'/>
  <texlive location='oberdiek'/>
  <install path='/macros/latex/contrib/oberdiek/oberdiek.tds.zip'/>
</entry>
%</catalogue>
%    \end{macrocode}
%
% \begin{History}
%   \begin{Version}{2006/08/26 v1.0}
%   \item
%     First released version. (First start of the project was June/July 2002.)
%   \end{Version}
%   \begin{Version}{2007/04/11 v1.1}
%   \item
%     Line ends sanitized.
%   \end{Version}
%   \begin{Version}{2008/11/26 v1.2}
%   \item
%     Package \xpackage{pict2e} added to documentation section ``Supported
%     packages''.
%   \item
%     Package order of supported packages is checked.
%   \end{Version}
%   \begin{Version}{2009/10/11 v1.3}
%   \item
%     Fix because of new version v4.3 of package \xpackage{calc}.
%   \end{Version}
% \end{History}
%
% \PrintIndex
%
% \Finale
\endinput

%        (quote the arguments according to the demands of your shell)
%
% Documentation:
%    (a) If picture.drv is present:
%           latex picture.drv
%    (b) Without picture.drv:
%           latex picture.dtx; ...
%    The class ltxdoc loads the configuration file ltxdoc.cfg
%    if available. Here you can specify further options, e.g.
%    use A4 as paper format:
%       \PassOptionsToClass{a4paper}{article}
%
%    Programm calls to get the documentation (example):
%       pdflatex picture.dtx
%       makeindex -s gind.ist picture.idx
%       pdflatex picture.dtx
%       makeindex -s gind.ist picture.idx
%       pdflatex picture.dtx
%
% Installation:
%    TDS:tex/latex/oberdiek/picture.sty
%    TDS:doc/latex/oberdiek/picture.pdf
%    TDS:doc/latex/oberdiek/picture-example.tex
%    TDS:source/latex/oberdiek/picture.dtx
%
%<*ignore>
\begingroup
  \catcode123=1 %
  \catcode125=2 %
  \def\x{LaTeX2e}%
\expandafter\endgroup
\ifcase 0\ifx\install y1\fi\expandafter
         \ifx\csname processbatchFile\endcsname\relax\else1\fi
         \ifx\fmtname\x\else 1\fi\relax
\else\csname fi\endcsname
%</ignore>
%<*install>
\input docstrip.tex
\Msg{************************************************************************}
\Msg{* Installation}
\Msg{* Package: picture 2009/10/11 v1.3 Dimens for picture macros (HO)}
\Msg{************************************************************************}

\keepsilent
\askforoverwritefalse

\let\MetaPrefix\relax
\preamble

This is a generated file.

Project: picture
Version: 2009/10/11 v1.3

Copyright (C) 2006-2009 by
   Heiko Oberdiek <heiko.oberdiek at googlemail.com>

This work may be distributed and/or modified under the
conditions of the LaTeX Project Public License, either
version 1.3c of this license or (at your option) any later
version. This version of this license is in
   http://www.latex-project.org/lppl/lppl-1-3c.txt
and the latest version of this license is in
   http://www.latex-project.org/lppl.txt
and version 1.3 or later is part of all distributions of
LaTeX version 2005/12/01 or later.

This work has the LPPL maintenance status "maintained".

This Current Maintainer of this work is Heiko Oberdiek.

This work consists of the main source file picture.dtx
and the derived files
   picture.sty, picture.pdf, picture.ins, picture.drv,
   picture-example.tex.

\endpreamble
\let\MetaPrefix\DoubleperCent

\generate{%
  \file{picture.ins}{\from{picture.dtx}{install}}%
  \file{picture.drv}{\from{picture.dtx}{driver}}%
  \usedir{tex/latex/oberdiek}%
  \file{picture.sty}{\from{picture.dtx}{package}}%
  \usedir{doc/latex/oberdiek}%
  \file{picture-example.tex}{\from{picture.dtx}{example}}%
  \nopreamble
  \nopostamble
  \usedir{source/latex/oberdiek/catalogue}%
  \file{picture.xml}{\from{picture.dtx}{catalogue}}%
}

\catcode32=13\relax% active space
\let =\space%
\Msg{************************************************************************}
\Msg{*}
\Msg{* To finish the installation you have to move the following}
\Msg{* file into a directory searched by TeX:}
\Msg{*}
\Msg{*     picture.sty}
\Msg{*}
\Msg{* To produce the documentation run the file `picture.drv'}
\Msg{* through LaTeX.}
\Msg{*}
\Msg{* Happy TeXing!}
\Msg{*}
\Msg{************************************************************************}

\endbatchfile
%</install>
%<*ignore>
\fi
%</ignore>
%<*driver>
\NeedsTeXFormat{LaTeX2e}
\ProvidesFile{picture.drv}%
  [2009/10/11 v1.3 Dimens for picture macros (HO)]%
\documentclass{ltxdoc}
\usepackage{holtxdoc}[2011/11/22]
\begin{document}
  \DocInput{picture.dtx}%
\end{document}
%</driver>
% \fi
%
% \CheckSum{322}
%
% \CharacterTable
%  {Upper-case    \A\B\C\D\E\F\G\H\I\J\K\L\M\N\O\P\Q\R\S\T\U\V\W\X\Y\Z
%   Lower-case    \a\b\c\d\e\f\g\h\i\j\k\l\m\n\o\p\q\r\s\t\u\v\w\x\y\z
%   Digits        \0\1\2\3\4\5\6\7\8\9
%   Exclamation   \!     Double quote  \"     Hash (number) \#
%   Dollar        \$     Percent       \%     Ampersand     \&
%   Acute accent  \'     Left paren    \(     Right paren   \)
%   Asterisk      \*     Plus          \+     Comma         \,
%   Minus         \-     Point         \.     Solidus       \/
%   Colon         \:     Semicolon     \;     Less than     \<
%   Equals        \=     Greater than  \>     Question mark \?
%   Commercial at \@     Left bracket  \[     Backslash     \\
%   Right bracket \]     Circumflex    \^     Underscore    \_
%   Grave accent  \`     Left brace    \{     Vertical bar  \|
%   Right brace   \}     Tilde         \~}
%
% \GetFileInfo{picture.drv}
%
% \title{The \xpackage{picture} package}
% \date{2009/10/11 v1.3}
% \author{Heiko Oberdiek\\\xemail{heiko.oberdiek at googlemail.com}}
%
% \maketitle
%
% \begin{abstract}
% There are macro and environment arguments that expect numbers
% that will internally be multiplicated with \cs{unitlength}.
% This package extends the syntax of these arguments that
% dimens with calculation support can be added for these arguments.
% \end{abstract}
%
% \tableofcontents
%
% \section{User interface}
%
% \subsection{Introduction}
%
% The environment |picture| and macros such as \cs{put}, \cs{line},
% \cs{vector} and other macros have arguments that expect numbers
% that are used as factor for \cs{unitlength}.
% This package redefines such macros and adds code that detects
% whether such an argument is given as number or as length.
% In the latter case, the length is used directly without multiplying
% with \cs{unitlength}.
%
% \subsection{Options}
%
% Depending on the available features, also length expressions
% can be given. Option \xoption{calc} loads package \xpackage{calc}.
% Then expressions of these package may be used.
% Otherwise \xoption{etex} wraps the length argument inside
% \cs{dimexpr}\dots\cs{relax}, if \eTeX\ is available. Otherwise
% option \xoption{plain} uses plain assignments without calculation
% support.
%
% The default is \xoption{calc} if package \xpackage{calc}
% is loaded before package \xoption{picture}. If you specify
% option \xoption{calc} the loading of \xpackage{calc} is ensured.
% Otherwise package \xpackage{picture} looks whether \cs{dimexpr}
% is available and uses then option \xoption{etex} as default.
% If \eTeX\ also could not be found, then \xoption{plain} is used.
%
% \subsection{Example}
%
%    \begin{macrocode}
%<*example>
\documentclass{article}

\usepackage[calc]{picture}

\begin{document}

\setlength{\unitlength}{1pt}

\begin{picture}(\widthof{Hello World}, 10mm)
  \put(0, 0){\makebox(0,0)[lb]{Hello World}}%
  \put(0, \heightof{Hello World} + \fboxsep){%
    \line(1, 0){\widthof{Hello World}}%
  }%
  \put(\widthof{Hello World}, 10mm){%
    \line(0, -1){10mm}%
  }%
\end{picture}

\end{document}
%</example>
%    \end{macrocode}
%
% \subsection{Supported packages}
%
% Packages \xpackage{pspicture} and \xpackage{pict2e} are supported,
% but they must be loaded before package \xpackage{picture}.
%
% New macros can be supported by \cs{picture@redefine}.
% The first argument is the macro which contains the arguments
% in its parameter text that you want to support by package \xpackage{picture}.
% The second argument contains the parameter text. Change \# to \& for
% the arguments in question.
% Examples (already used by package \xpackage{picture}):
% \begin{quote}
% |\picture@redefine\put{(&1,&2)}|\\
% |\picture@redefine\line{(#1,#2)&3}|
% \end{quote}
%
% \StopEventually{
% }
%
% \section{Implementation}
%
% \subsection{Identification}
%
%    \begin{macrocode}
%<*package>
\NeedsTeXFormat{LaTeX2e}
\ProvidesPackage{picture}%
  [2009/10/11 v1.3 Dimens for picture macros (HO)]%
%    \end{macrocode}
%
% \subsection{Options}
%
%    \begin{macrocode}
\def\Pc@calcname{calc}
\def\Pc@etexname{etex}
\def\Pc@plainname{plain}
%    \end{macrocode}
%    \begin{macro}{\Pc@method}
%    Macro \cs{Pc@method} stores the method to use for calculations.
%    Check which features are available and set the default for \cs{Pc@method}.
%    \begin{macrocode}
\@ifpackageloaded{calc}{%
  \let\Pc@method\Pc@calcname
}{%
  \begingroup\expandafter\expandafter\expandafter\endgroup
  \expandafter\ifx\csname dimexpr\endcsname\relax
    \let\Pc@method\Pc@plainname
  \else
    \let\Pc@method\Pc@etexname
  \fi
}
%    \end{macrocode}
%    \end{macro}
%    \begin{macrocode}
\DeclareOption{plain}{%
  \let\Pc@method\Pc@plainname
}
\DeclareOption{etex}{%
  \begingroup\expandafter\expandafter\expandafter\endgroup
  \expandafter\ifx\csname dimexpr\endcsname\relax
    \PackageError{picture}{%
      e-TeX is not available%
    }\@ehc
  \else
    \let\Pc@method\Pc@etexname
  \fi
}
\DeclareOption{calc}{%
  \let\Pc@method\Pc@calcname
}
\ProcessOptions*
\begingroup
  \let\on@line\@empty
  \PackageInfo{picture}{Calculation method: \Pc@method}%
\endgroup
%    \end{macrocode}
%
% \subsection{Calculation method}
%
%    \begin{macrocode}
\ifx\Pc@method\Pc@calcname
  \RequirePackage{calc}%
\fi
%    \end{macrocode}
%
% \subsubsection{Method \xoption{calc}}
%
%    \begin{macrocode}
\ifx\Pc@method\Pc@calcname
  \def\Pc@tokslength#1{%
    \begingroup
      \let\calc@error\Pc@calc@error
      \setlength\dimen@{#1\unitlength}\Pc@next\Pc@nil{#1}%
  }%
  \let\PcOrg@calc@error\calc@error
  \@ifpackagelater{calc}{2007/08/22}{% v4.3
    \def\Pc@calc@error#1{%
      \expandafter\ifx\expandafter\unitlength\noexpand#1\relax
        \def\calc@next##1!{%
          \endgroup
          \aftergroup\afterassignment
          \aftergroup\Pc@next
        }%
        \expandafter\@firstoftwo
      \else
        \expandafter\@secondoftwo
      \fi
      {%
        \calc@next{#1}%
      }{%
        \PcOrg@calc@error{#1}%
      }%
    }%
  }{%
    \def\Pc@calc@error#1{%
      \expandafter\ifx\expandafter\unitlength\noexpand#1\relax
        \def\calc@next##1!{%
          \endgroup
          \aftergroup\afterassignment
          \aftergroup\Pc@next
        }%
        \expandafter\@gobble
      \else
        \expandafter\@firstofone
      \fi
      {%
        \PcOrg@calc@error{#1}%
      }%
    }%
  }%
\fi
%    \end{macrocode}
%
% \subsubsection{Method \xoption{etex}}
%
%    \begin{macrocode}
\ifx\Pc@method\Pc@etexname
  \def\Pc@tokslength#1{%
    \begingroup
      \afterassignment\Pc@next
      \dimen@=\dimexpr#1\unitlength\Pc@nil{#1}%
  }%
\fi
%    \end{macrocode}
%
% \subsubsection{Method \xoption{plain}}
%
%    \begin{macrocode}
\ifx\Pc@method\Pc@plainname
  \def\Pc@tokslength#1{%
    \begingroup
      \afterassignment\Pc@next
      \dimen@=#1\unitlength\Pc@nil{#1}%
  }%
\fi
%    \end{macrocode}
%
% \subsubsection{Help macros}
%
%    \begin{macrocode}
\def\Pc@next#1\Pc@nil#2{%
  \ifx\\#1\\%
    \endgroup
    \Pc@addtoks{{#2}}%
  \else
    \expandafter\endgroup
    \expandafter\Pc@addtoks\expandafter{%
      \expandafter{\the\dimen@\@gobble}%
    }%
  \fi
}
%    \end{macrocode}
%    \begin{macro}{\Pc@nil}
%    \cs{Pc@nil} must not have the meaning of \cs{relax} because of
%    \cs{dimexpr}.
%    \begin{macrocode}
\let\Pc@nil\message
%    \end{macrocode}
%    \end{macro}
%    \begin{macro}{\Pc@addtoks}
%    \begin{macrocode}
\def\Pc@addtoks#1{%
  \toks@=\expandafter{\the\toks@#1}%
}
%    \end{macrocode}
%    \end{macro}
%    \begin{macro}{\Pc@init}
%    \begin{macrocode}
\def\Pc@init#1{%
  \begingroup
    \toks@={#1}%
}
%    \end{macrocode}
%    \end{macro}
%    \begin{macro}{\Pc@finish}
%    \begin{macrocode}
\def\Pc@finish#1{%
  \expandafter\endgroup
  \expandafter#1\the\toks@
}
%    \end{macrocode}
%    \end{macro}
%
% \subsection{Redefinitions}
%
%    \begin{macro}{\picture@redefine}
%    \begin{tabular}[t]{@{}l@{ }l@{}}
%    |#1|:& command name\\
%    |#2|:& parameter text, length parameter with \& instead of \#
%    \end{tabular}
%    \begin{macrocode}
\def\picture@redefine#1#2{%
  \begingroup
    \edef\reserved@a{%
      \noexpand\noexpand
      \expandafter\noexpand
          \csname PcOrg@\expandafter\@gobble\string#1\endcsname
    }%
    \toks0{#1}%
    \Pc@first#2&0%
}
%    \end{macrocode}
%    \end{macro}
%    \begin{macro}{\Pc@first}
%    \begin{macrocode}
\def\Pc@first#1&{%
  \toks1={#1}%
  \toks2={\Pc@init{#1}}%
  \Pc@scanlength
}
%    \end{macrocode}
%    \end{macro}
%    \begin{macro}{\Pc@scanlength}
%    |#1|: number of length parameter or zero
%    \begin{macrocode}
\def\Pc@scanlength#1{%
  \ifcase#1 %
    \expandafter\Pc@last
  \else
    \toks1=\expandafter{\the\toks1 ###1}%
    \toks2=\expandafter{\the\toks2 \Pc@tokslength{###1}}%
    \expandafter\Pc@scannext
  \fi
}
%    \end{macrocode}
%    \end{macro}
%    \begin{macro}{\Pc@scannext}
%    \begin{macrocode}
\def\Pc@scannext#1&{%
  \ifx\\#1\\%
  \else
    \toks1=\expandafter{\the\toks1 #1}%
    \toks2=\expandafter{\the\toks2 \Pc@addtoks{#1}}%
  \fi
  \Pc@scanlength
}
%    \end{macrocode}
%    \end{macro}
%    \begin{macro}{\Pc@last}
%    \begin{macrocode}
\def\Pc@last{%
  \edef\x{%
    \endgroup
    \let\reserved@a\the\toks0 %
    \def\the\toks0 \the\toks1 {%
      \the\toks2 %
      \noexpand\Pc@finish\reserved@a
    }%
  }%
  \x
}
%    \end{macrocode}
%    \end{macro}
%
% \subsubsection{\LaTeX\ base macros}
%
%    \begin{macrocode}
\picture@redefine\@picture{(&1,&2)(&3,&4)}
\picture@redefine\put{(&1,&2)}
\picture@redefine\multiput{(&1,&2)}
\picture@redefine\@multiput{(&1,&2)}
\picture@redefine\line{(#1,#2)&3}
\picture@redefine\vector{(#1,#2)&3}
\picture@redefine\dashbox{&1(&2,&3)}
\picture@redefine\@circle{&1}
\picture@redefine\@dot{&1}
\picture@redefine\@bezier{#1(&2,&3)(&4,&5)(&6,&7)}
\picture@redefine\@imakepicbox{(&1,&2)}
%    \end{macrocode}
%
% \subsubsection{Package \xpackage{pspicture}}
%
%     Package \xpackage{pspicture} changes the signature of \cs{@oval}
%     by adding an optional argument.
%    \begin{macrocode}
\@ifpackageloaded{pspicture}{%
  \picture@redefine\@oval{[&1](&2,&3)}%
  \picture@redefine\Line{(&1,&2)}%
  \picture@redefine\Curve{(&1,&2)}%
  \picture@redefine\Vector{(&1,&2)}%
}{%
  \picture@redefine\@oval{(&1,&2)}%
}
%    \end{macrocode}
%
% \subsection{Check package loading order}
%
%    \begin{macro}{\PC@checkpackage}
%    \begin{macrocode}
\def\Pc@checkpackage#1{%
  \@ifpackageloaded{#1}{%
  }{%
    \AtBeginDocument{%
      \@ifpackageloaded{#1}{%
        \PackageWarningNoLine{picture}{%
          Package `#1' is loaded after `picture'.\MessageBreak
          Load package `picture' afterwards to get full support%
          \MessageBreak
          of its additional syntax with length specifications%
        }%
      }{}%
    }%
  }%
}
%    \end{macrocode}
%    \end{macro}
%    \begin{macrocode}
\Pc@checkpackage{pict2e}
\Pc@checkpackage{pspicture}
%    \end{macrocode}
%
%    \begin{macrocode}
%</package>
%    \end{macrocode}
%
% \section{Installation}
%
% \subsection{Download}
%
% \paragraph{Package.} This package is available on
% CTAN\footnote{\url{ftp://ftp.ctan.org/tex-archive/}}:
% \begin{description}
% \item[\CTAN{macros/latex/contrib/oberdiek/picture.dtx}] The source file.
% \item[\CTAN{macros/latex/contrib/oberdiek/picture.pdf}] Documentation.
% \end{description}
%
%
% \paragraph{Bundle.} All the packages of the bundle `oberdiek'
% are also available in a TDS compliant ZIP archive. There
% the packages are already unpacked and the documentation files
% are generated. The files and directories obey the TDS standard.
% \begin{description}
% \item[\CTAN{install/macros/latex/contrib/oberdiek.tds.zip}]
% \end{description}
% \emph{TDS} refers to the standard ``A Directory Structure
% for \TeX\ Files'' (\CTAN{tds/tds.pdf}). Directories
% with \xfile{texmf} in their name are usually organized this way.
%
% \subsection{Bundle installation}
%
% \paragraph{Unpacking.} Unpack the \xfile{oberdiek.tds.zip} in the
% TDS tree (also known as \xfile{texmf} tree) of your choice.
% Example (linux):
% \begin{quote}
%   |unzip oberdiek.tds.zip -d ~/texmf|
% \end{quote}
%
% \paragraph{Script installation.}
% Check the directory \xfile{TDS:scripts/oberdiek/} for
% scripts that need further installation steps.
% Package \xpackage{attachfile2} comes with the Perl script
% \xfile{pdfatfi.pl} that should be installed in such a way
% that it can be called as \texttt{pdfatfi}.
% Example (linux):
% \begin{quote}
%   |chmod +x scripts/oberdiek/pdfatfi.pl|\\
%   |cp scripts/oberdiek/pdfatfi.pl /usr/local/bin/|
% \end{quote}
%
% \subsection{Package installation}
%
% \paragraph{Unpacking.} The \xfile{.dtx} file is a self-extracting
% \docstrip\ archive. The files are extracted by running the
% \xfile{.dtx} through \plainTeX:
% \begin{quote}
%   \verb|tex picture.dtx|
% \end{quote}
%
% \paragraph{TDS.} Now the different files must be moved into
% the different directories in your installation TDS tree
% (also known as \xfile{texmf} tree):
% \begin{quote}
% \def\t{^^A
% \begin{tabular}{@{}>{\ttfamily}l@{ $\rightarrow$ }>{\ttfamily}l@{}}
%   picture.sty & tex/latex/oberdiek/picture.sty\\
%   picture.pdf & doc/latex/oberdiek/picture.pdf\\
%   picture-example.tex & doc/latex/oberdiek/picture-example.tex\\
%   picture.dtx & source/latex/oberdiek/picture.dtx\\
% \end{tabular}^^A
% }^^A
% \sbox0{\t}^^A
% \ifdim\wd0>\linewidth
%   \begingroup
%     \advance\linewidth by\leftmargin
%     \advance\linewidth by\rightmargin
%   \edef\x{\endgroup
%     \def\noexpand\lw{\the\linewidth}^^A
%   }\x
%   \def\lwbox{^^A
%     \leavevmode
%     \hbox to \linewidth{^^A
%       \kern-\leftmargin\relax
%       \hss
%       \usebox0
%       \hss
%       \kern-\rightmargin\relax
%     }^^A
%   }^^A
%   \ifdim\wd0>\lw
%     \sbox0{\small\t}^^A
%     \ifdim\wd0>\linewidth
%       \ifdim\wd0>\lw
%         \sbox0{\footnotesize\t}^^A
%         \ifdim\wd0>\linewidth
%           \ifdim\wd0>\lw
%             \sbox0{\scriptsize\t}^^A
%             \ifdim\wd0>\linewidth
%               \ifdim\wd0>\lw
%                 \sbox0{\tiny\t}^^A
%                 \ifdim\wd0>\linewidth
%                   \lwbox
%                 \else
%                   \usebox0
%                 \fi
%               \else
%                 \lwbox
%               \fi
%             \else
%               \usebox0
%             \fi
%           \else
%             \lwbox
%           \fi
%         \else
%           \usebox0
%         \fi
%       \else
%         \lwbox
%       \fi
%     \else
%       \usebox0
%     \fi
%   \else
%     \lwbox
%   \fi
% \else
%   \usebox0
% \fi
% \end{quote}
% If you have a \xfile{docstrip.cfg} that configures and enables \docstrip's
% TDS installing feature, then some files can already be in the right
% place, see the documentation of \docstrip.
%
% \subsection{Refresh file name databases}
%
% If your \TeX~distribution
% (\teTeX, \mikTeX, \dots) relies on file name databases, you must refresh
% these. For example, \teTeX\ users run \verb|texhash| or
% \verb|mktexlsr|.
%
% \subsection{Some details for the interested}
%
% \paragraph{Attached source.}
%
% The PDF documentation on CTAN also includes the
% \xfile{.dtx} source file. It can be extracted by
% AcrobatReader 6 or higher. Another option is \textsf{pdftk},
% e.g. unpack the file into the current directory:
% \begin{quote}
%   \verb|pdftk picture.pdf unpack_files output .|
% \end{quote}
%
% \paragraph{Unpacking with \LaTeX.}
% The \xfile{.dtx} chooses its action depending on the format:
% \begin{description}
% \item[\plainTeX:] Run \docstrip\ and extract the files.
% \item[\LaTeX:] Generate the documentation.
% \end{description}
% If you insist on using \LaTeX\ for \docstrip\ (really,
% \docstrip\ does not need \LaTeX), then inform the autodetect routine
% about your intention:
% \begin{quote}
%   \verb|latex \let\install=y% \iffalse meta-comment
%
% File: picture.dtx
% Version: 2009/10/11 v1.3
% Info: Dimens for picture macros
%
% Copyright (C) 2006-2009 by
%    Heiko Oberdiek <heiko.oberdiek at googlemail.com>
%
% This work may be distributed and/or modified under the
% conditions of the LaTeX Project Public License, either
% version 1.3c of this license or (at your option) any later
% version. This version of this license is in
%    http://www.latex-project.org/lppl/lppl-1-3c.txt
% and the latest version of this license is in
%    http://www.latex-project.org/lppl.txt
% and version 1.3 or later is part of all distributions of
% LaTeX version 2005/12/01 or later.
%
% This work has the LPPL maintenance status "maintained".
%
% This Current Maintainer of this work is Heiko Oberdiek.
%
% This work consists of the main source file picture.dtx
% and the derived files
%    picture.sty, picture.pdf, picture.ins, picture.drv,
%    picture-example.tex.
%
% Distribution:
%    CTAN:macros/latex/contrib/oberdiek/picture.dtx
%    CTAN:macros/latex/contrib/oberdiek/picture.pdf
%
% Unpacking:
%    (a) If picture.ins is present:
%           tex picture.ins
%    (b) Without picture.ins:
%           tex picture.dtx
%    (c) If you insist on using LaTeX
%           latex \let\install=y\input{picture.dtx}
%        (quote the arguments according to the demands of your shell)
%
% Documentation:
%    (a) If picture.drv is present:
%           latex picture.drv
%    (b) Without picture.drv:
%           latex picture.dtx; ...
%    The class ltxdoc loads the configuration file ltxdoc.cfg
%    if available. Here you can specify further options, e.g.
%    use A4 as paper format:
%       \PassOptionsToClass{a4paper}{article}
%
%    Programm calls to get the documentation (example):
%       pdflatex picture.dtx
%       makeindex -s gind.ist picture.idx
%       pdflatex picture.dtx
%       makeindex -s gind.ist picture.idx
%       pdflatex picture.dtx
%
% Installation:
%    TDS:tex/latex/oberdiek/picture.sty
%    TDS:doc/latex/oberdiek/picture.pdf
%    TDS:doc/latex/oberdiek/picture-example.tex
%    TDS:source/latex/oberdiek/picture.dtx
%
%<*ignore>
\begingroup
  \catcode123=1 %
  \catcode125=2 %
  \def\x{LaTeX2e}%
\expandafter\endgroup
\ifcase 0\ifx\install y1\fi\expandafter
         \ifx\csname processbatchFile\endcsname\relax\else1\fi
         \ifx\fmtname\x\else 1\fi\relax
\else\csname fi\endcsname
%</ignore>
%<*install>
\input docstrip.tex
\Msg{************************************************************************}
\Msg{* Installation}
\Msg{* Package: picture 2009/10/11 v1.3 Dimens for picture macros (HO)}
\Msg{************************************************************************}

\keepsilent
\askforoverwritefalse

\let\MetaPrefix\relax
\preamble

This is a generated file.

Project: picture
Version: 2009/10/11 v1.3

Copyright (C) 2006-2009 by
   Heiko Oberdiek <heiko.oberdiek at googlemail.com>

This work may be distributed and/or modified under the
conditions of the LaTeX Project Public License, either
version 1.3c of this license or (at your option) any later
version. This version of this license is in
   http://www.latex-project.org/lppl/lppl-1-3c.txt
and the latest version of this license is in
   http://www.latex-project.org/lppl.txt
and version 1.3 or later is part of all distributions of
LaTeX version 2005/12/01 or later.

This work has the LPPL maintenance status "maintained".

This Current Maintainer of this work is Heiko Oberdiek.

This work consists of the main source file picture.dtx
and the derived files
   picture.sty, picture.pdf, picture.ins, picture.drv,
   picture-example.tex.

\endpreamble
\let\MetaPrefix\DoubleperCent

\generate{%
  \file{picture.ins}{\from{picture.dtx}{install}}%
  \file{picture.drv}{\from{picture.dtx}{driver}}%
  \usedir{tex/latex/oberdiek}%
  \file{picture.sty}{\from{picture.dtx}{package}}%
  \usedir{doc/latex/oberdiek}%
  \file{picture-example.tex}{\from{picture.dtx}{example}}%
  \nopreamble
  \nopostamble
  \usedir{source/latex/oberdiek/catalogue}%
  \file{picture.xml}{\from{picture.dtx}{catalogue}}%
}

\catcode32=13\relax% active space
\let =\space%
\Msg{************************************************************************}
\Msg{*}
\Msg{* To finish the installation you have to move the following}
\Msg{* file into a directory searched by TeX:}
\Msg{*}
\Msg{*     picture.sty}
\Msg{*}
\Msg{* To produce the documentation run the file `picture.drv'}
\Msg{* through LaTeX.}
\Msg{*}
\Msg{* Happy TeXing!}
\Msg{*}
\Msg{************************************************************************}

\endbatchfile
%</install>
%<*ignore>
\fi
%</ignore>
%<*driver>
\NeedsTeXFormat{LaTeX2e}
\ProvidesFile{picture.drv}%
  [2009/10/11 v1.3 Dimens for picture macros (HO)]%
\documentclass{ltxdoc}
\usepackage{holtxdoc}[2011/11/22]
\begin{document}
  \DocInput{picture.dtx}%
\end{document}
%</driver>
% \fi
%
% \CheckSum{322}
%
% \CharacterTable
%  {Upper-case    \A\B\C\D\E\F\G\H\I\J\K\L\M\N\O\P\Q\R\S\T\U\V\W\X\Y\Z
%   Lower-case    \a\b\c\d\e\f\g\h\i\j\k\l\m\n\o\p\q\r\s\t\u\v\w\x\y\z
%   Digits        \0\1\2\3\4\5\6\7\8\9
%   Exclamation   \!     Double quote  \"     Hash (number) \#
%   Dollar        \$     Percent       \%     Ampersand     \&
%   Acute accent  \'     Left paren    \(     Right paren   \)
%   Asterisk      \*     Plus          \+     Comma         \,
%   Minus         \-     Point         \.     Solidus       \/
%   Colon         \:     Semicolon     \;     Less than     \<
%   Equals        \=     Greater than  \>     Question mark \?
%   Commercial at \@     Left bracket  \[     Backslash     \\
%   Right bracket \]     Circumflex    \^     Underscore    \_
%   Grave accent  \`     Left brace    \{     Vertical bar  \|
%   Right brace   \}     Tilde         \~}
%
% \GetFileInfo{picture.drv}
%
% \title{The \xpackage{picture} package}
% \date{2009/10/11 v1.3}
% \author{Heiko Oberdiek\\\xemail{heiko.oberdiek at googlemail.com}}
%
% \maketitle
%
% \begin{abstract}
% There are macro and environment arguments that expect numbers
% that will internally be multiplicated with \cs{unitlength}.
% This package extends the syntax of these arguments that
% dimens with calculation support can be added for these arguments.
% \end{abstract}
%
% \tableofcontents
%
% \section{User interface}
%
% \subsection{Introduction}
%
% The environment |picture| and macros such as \cs{put}, \cs{line},
% \cs{vector} and other macros have arguments that expect numbers
% that are used as factor for \cs{unitlength}.
% This package redefines such macros and adds code that detects
% whether such an argument is given as number or as length.
% In the latter case, the length is used directly without multiplying
% with \cs{unitlength}.
%
% \subsection{Options}
%
% Depending on the available features, also length expressions
% can be given. Option \xoption{calc} loads package \xpackage{calc}.
% Then expressions of these package may be used.
% Otherwise \xoption{etex} wraps the length argument inside
% \cs{dimexpr}\dots\cs{relax}, if \eTeX\ is available. Otherwise
% option \xoption{plain} uses plain assignments without calculation
% support.
%
% The default is \xoption{calc} if package \xpackage{calc}
% is loaded before package \xoption{picture}. If you specify
% option \xoption{calc} the loading of \xpackage{calc} is ensured.
% Otherwise package \xpackage{picture} looks whether \cs{dimexpr}
% is available and uses then option \xoption{etex} as default.
% If \eTeX\ also could not be found, then \xoption{plain} is used.
%
% \subsection{Example}
%
%    \begin{macrocode}
%<*example>
\documentclass{article}

\usepackage[calc]{picture}

\begin{document}

\setlength{\unitlength}{1pt}

\begin{picture}(\widthof{Hello World}, 10mm)
  \put(0, 0){\makebox(0,0)[lb]{Hello World}}%
  \put(0, \heightof{Hello World} + \fboxsep){%
    \line(1, 0){\widthof{Hello World}}%
  }%
  \put(\widthof{Hello World}, 10mm){%
    \line(0, -1){10mm}%
  }%
\end{picture}

\end{document}
%</example>
%    \end{macrocode}
%
% \subsection{Supported packages}
%
% Packages \xpackage{pspicture} and \xpackage{pict2e} are supported,
% but they must be loaded before package \xpackage{picture}.
%
% New macros can be supported by \cs{picture@redefine}.
% The first argument is the macro which contains the arguments
% in its parameter text that you want to support by package \xpackage{picture}.
% The second argument contains the parameter text. Change \# to \& for
% the arguments in question.
% Examples (already used by package \xpackage{picture}):
% \begin{quote}
% |\picture@redefine\put{(&1,&2)}|\\
% |\picture@redefine\line{(#1,#2)&3}|
% \end{quote}
%
% \StopEventually{
% }
%
% \section{Implementation}
%
% \subsection{Identification}
%
%    \begin{macrocode}
%<*package>
\NeedsTeXFormat{LaTeX2e}
\ProvidesPackage{picture}%
  [2009/10/11 v1.3 Dimens for picture macros (HO)]%
%    \end{macrocode}
%
% \subsection{Options}
%
%    \begin{macrocode}
\def\Pc@calcname{calc}
\def\Pc@etexname{etex}
\def\Pc@plainname{plain}
%    \end{macrocode}
%    \begin{macro}{\Pc@method}
%    Macro \cs{Pc@method} stores the method to use for calculations.
%    Check which features are available and set the default for \cs{Pc@method}.
%    \begin{macrocode}
\@ifpackageloaded{calc}{%
  \let\Pc@method\Pc@calcname
}{%
  \begingroup\expandafter\expandafter\expandafter\endgroup
  \expandafter\ifx\csname dimexpr\endcsname\relax
    \let\Pc@method\Pc@plainname
  \else
    \let\Pc@method\Pc@etexname
  \fi
}
%    \end{macrocode}
%    \end{macro}
%    \begin{macrocode}
\DeclareOption{plain}{%
  \let\Pc@method\Pc@plainname
}
\DeclareOption{etex}{%
  \begingroup\expandafter\expandafter\expandafter\endgroup
  \expandafter\ifx\csname dimexpr\endcsname\relax
    \PackageError{picture}{%
      e-TeX is not available%
    }\@ehc
  \else
    \let\Pc@method\Pc@etexname
  \fi
}
\DeclareOption{calc}{%
  \let\Pc@method\Pc@calcname
}
\ProcessOptions*
\begingroup
  \let\on@line\@empty
  \PackageInfo{picture}{Calculation method: \Pc@method}%
\endgroup
%    \end{macrocode}
%
% \subsection{Calculation method}
%
%    \begin{macrocode}
\ifx\Pc@method\Pc@calcname
  \RequirePackage{calc}%
\fi
%    \end{macrocode}
%
% \subsubsection{Method \xoption{calc}}
%
%    \begin{macrocode}
\ifx\Pc@method\Pc@calcname
  \def\Pc@tokslength#1{%
    \begingroup
      \let\calc@error\Pc@calc@error
      \setlength\dimen@{#1\unitlength}\Pc@next\Pc@nil{#1}%
  }%
  \let\PcOrg@calc@error\calc@error
  \@ifpackagelater{calc}{2007/08/22}{% v4.3
    \def\Pc@calc@error#1{%
      \expandafter\ifx\expandafter\unitlength\noexpand#1\relax
        \def\calc@next##1!{%
          \endgroup
          \aftergroup\afterassignment
          \aftergroup\Pc@next
        }%
        \expandafter\@firstoftwo
      \else
        \expandafter\@secondoftwo
      \fi
      {%
        \calc@next{#1}%
      }{%
        \PcOrg@calc@error{#1}%
      }%
    }%
  }{%
    \def\Pc@calc@error#1{%
      \expandafter\ifx\expandafter\unitlength\noexpand#1\relax
        \def\calc@next##1!{%
          \endgroup
          \aftergroup\afterassignment
          \aftergroup\Pc@next
        }%
        \expandafter\@gobble
      \else
        \expandafter\@firstofone
      \fi
      {%
        \PcOrg@calc@error{#1}%
      }%
    }%
  }%
\fi
%    \end{macrocode}
%
% \subsubsection{Method \xoption{etex}}
%
%    \begin{macrocode}
\ifx\Pc@method\Pc@etexname
  \def\Pc@tokslength#1{%
    \begingroup
      \afterassignment\Pc@next
      \dimen@=\dimexpr#1\unitlength\Pc@nil{#1}%
  }%
\fi
%    \end{macrocode}
%
% \subsubsection{Method \xoption{plain}}
%
%    \begin{macrocode}
\ifx\Pc@method\Pc@plainname
  \def\Pc@tokslength#1{%
    \begingroup
      \afterassignment\Pc@next
      \dimen@=#1\unitlength\Pc@nil{#1}%
  }%
\fi
%    \end{macrocode}
%
% \subsubsection{Help macros}
%
%    \begin{macrocode}
\def\Pc@next#1\Pc@nil#2{%
  \ifx\\#1\\%
    \endgroup
    \Pc@addtoks{{#2}}%
  \else
    \expandafter\endgroup
    \expandafter\Pc@addtoks\expandafter{%
      \expandafter{\the\dimen@\@gobble}%
    }%
  \fi
}
%    \end{macrocode}
%    \begin{macro}{\Pc@nil}
%    \cs{Pc@nil} must not have the meaning of \cs{relax} because of
%    \cs{dimexpr}.
%    \begin{macrocode}
\let\Pc@nil\message
%    \end{macrocode}
%    \end{macro}
%    \begin{macro}{\Pc@addtoks}
%    \begin{macrocode}
\def\Pc@addtoks#1{%
  \toks@=\expandafter{\the\toks@#1}%
}
%    \end{macrocode}
%    \end{macro}
%    \begin{macro}{\Pc@init}
%    \begin{macrocode}
\def\Pc@init#1{%
  \begingroup
    \toks@={#1}%
}
%    \end{macrocode}
%    \end{macro}
%    \begin{macro}{\Pc@finish}
%    \begin{macrocode}
\def\Pc@finish#1{%
  \expandafter\endgroup
  \expandafter#1\the\toks@
}
%    \end{macrocode}
%    \end{macro}
%
% \subsection{Redefinitions}
%
%    \begin{macro}{\picture@redefine}
%    \begin{tabular}[t]{@{}l@{ }l@{}}
%    |#1|:& command name\\
%    |#2|:& parameter text, length parameter with \& instead of \#
%    \end{tabular}
%    \begin{macrocode}
\def\picture@redefine#1#2{%
  \begingroup
    \edef\reserved@a{%
      \noexpand\noexpand
      \expandafter\noexpand
          \csname PcOrg@\expandafter\@gobble\string#1\endcsname
    }%
    \toks0{#1}%
    \Pc@first#2&0%
}
%    \end{macrocode}
%    \end{macro}
%    \begin{macro}{\Pc@first}
%    \begin{macrocode}
\def\Pc@first#1&{%
  \toks1={#1}%
  \toks2={\Pc@init{#1}}%
  \Pc@scanlength
}
%    \end{macrocode}
%    \end{macro}
%    \begin{macro}{\Pc@scanlength}
%    |#1|: number of length parameter or zero
%    \begin{macrocode}
\def\Pc@scanlength#1{%
  \ifcase#1 %
    \expandafter\Pc@last
  \else
    \toks1=\expandafter{\the\toks1 ###1}%
    \toks2=\expandafter{\the\toks2 \Pc@tokslength{###1}}%
    \expandafter\Pc@scannext
  \fi
}
%    \end{macrocode}
%    \end{macro}
%    \begin{macro}{\Pc@scannext}
%    \begin{macrocode}
\def\Pc@scannext#1&{%
  \ifx\\#1\\%
  \else
    \toks1=\expandafter{\the\toks1 #1}%
    \toks2=\expandafter{\the\toks2 \Pc@addtoks{#1}}%
  \fi
  \Pc@scanlength
}
%    \end{macrocode}
%    \end{macro}
%    \begin{macro}{\Pc@last}
%    \begin{macrocode}
\def\Pc@last{%
  \edef\x{%
    \endgroup
    \let\reserved@a\the\toks0 %
    \def\the\toks0 \the\toks1 {%
      \the\toks2 %
      \noexpand\Pc@finish\reserved@a
    }%
  }%
  \x
}
%    \end{macrocode}
%    \end{macro}
%
% \subsubsection{\LaTeX\ base macros}
%
%    \begin{macrocode}
\picture@redefine\@picture{(&1,&2)(&3,&4)}
\picture@redefine\put{(&1,&2)}
\picture@redefine\multiput{(&1,&2)}
\picture@redefine\@multiput{(&1,&2)}
\picture@redefine\line{(#1,#2)&3}
\picture@redefine\vector{(#1,#2)&3}
\picture@redefine\dashbox{&1(&2,&3)}
\picture@redefine\@circle{&1}
\picture@redefine\@dot{&1}
\picture@redefine\@bezier{#1(&2,&3)(&4,&5)(&6,&7)}
\picture@redefine\@imakepicbox{(&1,&2)}
%    \end{macrocode}
%
% \subsubsection{Package \xpackage{pspicture}}
%
%     Package \xpackage{pspicture} changes the signature of \cs{@oval}
%     by adding an optional argument.
%    \begin{macrocode}
\@ifpackageloaded{pspicture}{%
  \picture@redefine\@oval{[&1](&2,&3)}%
  \picture@redefine\Line{(&1,&2)}%
  \picture@redefine\Curve{(&1,&2)}%
  \picture@redefine\Vector{(&1,&2)}%
}{%
  \picture@redefine\@oval{(&1,&2)}%
}
%    \end{macrocode}
%
% \subsection{Check package loading order}
%
%    \begin{macro}{\PC@checkpackage}
%    \begin{macrocode}
\def\Pc@checkpackage#1{%
  \@ifpackageloaded{#1}{%
  }{%
    \AtBeginDocument{%
      \@ifpackageloaded{#1}{%
        \PackageWarningNoLine{picture}{%
          Package `#1' is loaded after `picture'.\MessageBreak
          Load package `picture' afterwards to get full support%
          \MessageBreak
          of its additional syntax with length specifications%
        }%
      }{}%
    }%
  }%
}
%    \end{macrocode}
%    \end{macro}
%    \begin{macrocode}
\Pc@checkpackage{pict2e}
\Pc@checkpackage{pspicture}
%    \end{macrocode}
%
%    \begin{macrocode}
%</package>
%    \end{macrocode}
%
% \section{Installation}
%
% \subsection{Download}
%
% \paragraph{Package.} This package is available on
% CTAN\footnote{\url{ftp://ftp.ctan.org/tex-archive/}}:
% \begin{description}
% \item[\CTAN{macros/latex/contrib/oberdiek/picture.dtx}] The source file.
% \item[\CTAN{macros/latex/contrib/oberdiek/picture.pdf}] Documentation.
% \end{description}
%
%
% \paragraph{Bundle.} All the packages of the bundle `oberdiek'
% are also available in a TDS compliant ZIP archive. There
% the packages are already unpacked and the documentation files
% are generated. The files and directories obey the TDS standard.
% \begin{description}
% \item[\CTAN{install/macros/latex/contrib/oberdiek.tds.zip}]
% \end{description}
% \emph{TDS} refers to the standard ``A Directory Structure
% for \TeX\ Files'' (\CTAN{tds/tds.pdf}). Directories
% with \xfile{texmf} in their name are usually organized this way.
%
% \subsection{Bundle installation}
%
% \paragraph{Unpacking.} Unpack the \xfile{oberdiek.tds.zip} in the
% TDS tree (also known as \xfile{texmf} tree) of your choice.
% Example (linux):
% \begin{quote}
%   |unzip oberdiek.tds.zip -d ~/texmf|
% \end{quote}
%
% \paragraph{Script installation.}
% Check the directory \xfile{TDS:scripts/oberdiek/} for
% scripts that need further installation steps.
% Package \xpackage{attachfile2} comes with the Perl script
% \xfile{pdfatfi.pl} that should be installed in such a way
% that it can be called as \texttt{pdfatfi}.
% Example (linux):
% \begin{quote}
%   |chmod +x scripts/oberdiek/pdfatfi.pl|\\
%   |cp scripts/oberdiek/pdfatfi.pl /usr/local/bin/|
% \end{quote}
%
% \subsection{Package installation}
%
% \paragraph{Unpacking.} The \xfile{.dtx} file is a self-extracting
% \docstrip\ archive. The files are extracted by running the
% \xfile{.dtx} through \plainTeX:
% \begin{quote}
%   \verb|tex picture.dtx|
% \end{quote}
%
% \paragraph{TDS.} Now the different files must be moved into
% the different directories in your installation TDS tree
% (also known as \xfile{texmf} tree):
% \begin{quote}
% \def\t{^^A
% \begin{tabular}{@{}>{\ttfamily}l@{ $\rightarrow$ }>{\ttfamily}l@{}}
%   picture.sty & tex/latex/oberdiek/picture.sty\\
%   picture.pdf & doc/latex/oberdiek/picture.pdf\\
%   picture-example.tex & doc/latex/oberdiek/picture-example.tex\\
%   picture.dtx & source/latex/oberdiek/picture.dtx\\
% \end{tabular}^^A
% }^^A
% \sbox0{\t}^^A
% \ifdim\wd0>\linewidth
%   \begingroup
%     \advance\linewidth by\leftmargin
%     \advance\linewidth by\rightmargin
%   \edef\x{\endgroup
%     \def\noexpand\lw{\the\linewidth}^^A
%   }\x
%   \def\lwbox{^^A
%     \leavevmode
%     \hbox to \linewidth{^^A
%       \kern-\leftmargin\relax
%       \hss
%       \usebox0
%       \hss
%       \kern-\rightmargin\relax
%     }^^A
%   }^^A
%   \ifdim\wd0>\lw
%     \sbox0{\small\t}^^A
%     \ifdim\wd0>\linewidth
%       \ifdim\wd0>\lw
%         \sbox0{\footnotesize\t}^^A
%         \ifdim\wd0>\linewidth
%           \ifdim\wd0>\lw
%             \sbox0{\scriptsize\t}^^A
%             \ifdim\wd0>\linewidth
%               \ifdim\wd0>\lw
%                 \sbox0{\tiny\t}^^A
%                 \ifdim\wd0>\linewidth
%                   \lwbox
%                 \else
%                   \usebox0
%                 \fi
%               \else
%                 \lwbox
%               \fi
%             \else
%               \usebox0
%             \fi
%           \else
%             \lwbox
%           \fi
%         \else
%           \usebox0
%         \fi
%       \else
%         \lwbox
%       \fi
%     \else
%       \usebox0
%     \fi
%   \else
%     \lwbox
%   \fi
% \else
%   \usebox0
% \fi
% \end{quote}
% If you have a \xfile{docstrip.cfg} that configures and enables \docstrip's
% TDS installing feature, then some files can already be in the right
% place, see the documentation of \docstrip.
%
% \subsection{Refresh file name databases}
%
% If your \TeX~distribution
% (\teTeX, \mikTeX, \dots) relies on file name databases, you must refresh
% these. For example, \teTeX\ users run \verb|texhash| or
% \verb|mktexlsr|.
%
% \subsection{Some details for the interested}
%
% \paragraph{Attached source.}
%
% The PDF documentation on CTAN also includes the
% \xfile{.dtx} source file. It can be extracted by
% AcrobatReader 6 or higher. Another option is \textsf{pdftk},
% e.g. unpack the file into the current directory:
% \begin{quote}
%   \verb|pdftk picture.pdf unpack_files output .|
% \end{quote}
%
% \paragraph{Unpacking with \LaTeX.}
% The \xfile{.dtx} chooses its action depending on the format:
% \begin{description}
% \item[\plainTeX:] Run \docstrip\ and extract the files.
% \item[\LaTeX:] Generate the documentation.
% \end{description}
% If you insist on using \LaTeX\ for \docstrip\ (really,
% \docstrip\ does not need \LaTeX), then inform the autodetect routine
% about your intention:
% \begin{quote}
%   \verb|latex \let\install=y\input{picture.dtx}|
% \end{quote}
% Do not forget to quote the argument according to the demands
% of your shell.
%
% \paragraph{Generating the documentation.}
% You can use both the \xfile{.dtx} or the \xfile{.drv} to generate
% the documentation. The process can be configured by the
% configuration file \xfile{ltxdoc.cfg}. For instance, put this
% line into this file, if you want to have A4 as paper format:
% \begin{quote}
%   \verb|\PassOptionsToClass{a4paper}{article}|
% \end{quote}
% An example follows how to generate the
% documentation with pdf\LaTeX:
% \begin{quote}
%\begin{verbatim}
%pdflatex picture.dtx
%makeindex -s gind.ist picture.idx
%pdflatex picture.dtx
%makeindex -s gind.ist picture.idx
%pdflatex picture.dtx
%\end{verbatim}
% \end{quote}
%
% \section{Catalogue}
%
% The following XML file can be used as source for the
% \href{http://mirror.ctan.org/help/Catalogue/catalogue.html}{\TeX\ Catalogue}.
% The elements \texttt{caption} and \texttt{description} are imported
% from the original XML file from the Catalogue.
% The name of the XML file in the Catalogue is \xfile{picture.xml}.
%    \begin{macrocode}
%<*catalogue>
<?xml version='1.0' encoding='us-ascii'?>
<!DOCTYPE entry SYSTEM 'catalogue.dtd'>
<entry datestamp='$Date$' modifier='$Author$' id='picture'>
  <name>picture</name>
  <caption>Dimens for picture macros.</caption>
  <authorref id='auth:oberdiek'/>
  <copyright owner='Heiko Oberdiek' year='2006-2009'/>
  <license type='lppl1.3'/>
  <version number='1.3'/>
  <description>
    There are macro and environment arguments that expect numbers
    that will internally be multiplied by <tt>\unitlength</tt>.
    This package extends the syntax of these arguments, so that
    dimensions with calculation support may be used for these arguments.
    <p/>
    The package is part of the <xref refid='oberdiek'>oberdiek</xref> bundle.
  </description>
  <documentation details='Package documentation'
      href='ctan:/macros/latex/contrib/oberdiek/picture.pdf'/>
  <ctan file='true' path='/macros/latex/contrib/oberdiek/picture.dtx'/>
  <miktex location='oberdiek'/>
  <texlive location='oberdiek'/>
  <install path='/macros/latex/contrib/oberdiek/oberdiek.tds.zip'/>
</entry>
%</catalogue>
%    \end{macrocode}
%
% \begin{History}
%   \begin{Version}{2006/08/26 v1.0}
%   \item
%     First released version. (First start of the project was June/July 2002.)
%   \end{Version}
%   \begin{Version}{2007/04/11 v1.1}
%   \item
%     Line ends sanitized.
%   \end{Version}
%   \begin{Version}{2008/11/26 v1.2}
%   \item
%     Package \xpackage{pict2e} added to documentation section ``Supported
%     packages''.
%   \item
%     Package order of supported packages is checked.
%   \end{Version}
%   \begin{Version}{2009/10/11 v1.3}
%   \item
%     Fix because of new version v4.3 of package \xpackage{calc}.
%   \end{Version}
% \end{History}
%
% \PrintIndex
%
% \Finale
\endinput
|
% \end{quote}
% Do not forget to quote the argument according to the demands
% of your shell.
%
% \paragraph{Generating the documentation.}
% You can use both the \xfile{.dtx} or the \xfile{.drv} to generate
% the documentation. The process can be configured by the
% configuration file \xfile{ltxdoc.cfg}. For instance, put this
% line into this file, if you want to have A4 as paper format:
% \begin{quote}
%   \verb|\PassOptionsToClass{a4paper}{article}|
% \end{quote}
% An example follows how to generate the
% documentation with pdf\LaTeX:
% \begin{quote}
%\begin{verbatim}
%pdflatex picture.dtx
%makeindex -s gind.ist picture.idx
%pdflatex picture.dtx
%makeindex -s gind.ist picture.idx
%pdflatex picture.dtx
%\end{verbatim}
% \end{quote}
%
% \section{Catalogue}
%
% The following XML file can be used as source for the
% \href{http://mirror.ctan.org/help/Catalogue/catalogue.html}{\TeX\ Catalogue}.
% The elements \texttt{caption} and \texttt{description} are imported
% from the original XML file from the Catalogue.
% The name of the XML file in the Catalogue is \xfile{picture.xml}.
%    \begin{macrocode}
%<*catalogue>
<?xml version='1.0' encoding='us-ascii'?>
<!DOCTYPE entry SYSTEM 'catalogue.dtd'>
<entry datestamp='$Date$' modifier='$Author$' id='picture'>
  <name>picture</name>
  <caption>Dimens for picture macros.</caption>
  <authorref id='auth:oberdiek'/>
  <copyright owner='Heiko Oberdiek' year='2006-2009'/>
  <license type='lppl1.3'/>
  <version number='1.3'/>
  <description>
    There are macro and environment arguments that expect numbers
    that will internally be multiplied by <tt>\unitlength</tt>.
    This package extends the syntax of these arguments, so that
    dimensions with calculation support may be used for these arguments.
    <p/>
    The package is part of the <xref refid='oberdiek'>oberdiek</xref> bundle.
  </description>
  <documentation details='Package documentation'
      href='ctan:/macros/latex/contrib/oberdiek/picture.pdf'/>
  <ctan file='true' path='/macros/latex/contrib/oberdiek/picture.dtx'/>
  <miktex location='oberdiek'/>
  <texlive location='oberdiek'/>
  <install path='/macros/latex/contrib/oberdiek/oberdiek.tds.zip'/>
</entry>
%</catalogue>
%    \end{macrocode}
%
% \begin{History}
%   \begin{Version}{2006/08/26 v1.0}
%   \item
%     First released version. (First start of the project was June/July 2002.)
%   \end{Version}
%   \begin{Version}{2007/04/11 v1.1}
%   \item
%     Line ends sanitized.
%   \end{Version}
%   \begin{Version}{2008/11/26 v1.2}
%   \item
%     Package \xpackage{pict2e} added to documentation section ``Supported
%     packages''.
%   \item
%     Package order of supported packages is checked.
%   \end{Version}
%   \begin{Version}{2009/10/11 v1.3}
%   \item
%     Fix because of new version v4.3 of package \xpackage{calc}.
%   \end{Version}
% \end{History}
%
% \PrintIndex
%
% \Finale
\endinput
|
% \end{quote}
% Do not forget to quote the argument according to the demands
% of your shell.
%
% \paragraph{Generating the documentation.}
% You can use both the \xfile{.dtx} or the \xfile{.drv} to generate
% the documentation. The process can be configured by the
% configuration file \xfile{ltxdoc.cfg}. For instance, put this
% line into this file, if you want to have A4 as paper format:
% \begin{quote}
%   \verb|\PassOptionsToClass{a4paper}{article}|
% \end{quote}
% An example follows how to generate the
% documentation with pdf\LaTeX:
% \begin{quote}
%\begin{verbatim}
%pdflatex picture.dtx
%makeindex -s gind.ist picture.idx
%pdflatex picture.dtx
%makeindex -s gind.ist picture.idx
%pdflatex picture.dtx
%\end{verbatim}
% \end{quote}
%
% \section{Catalogue}
%
% The following XML file can be used as source for the
% \href{http://mirror.ctan.org/help/Catalogue/catalogue.html}{\TeX\ Catalogue}.
% The elements \texttt{caption} and \texttt{description} are imported
% from the original XML file from the Catalogue.
% The name of the XML file in the Catalogue is \xfile{picture.xml}.
%    \begin{macrocode}
%<*catalogue>
<?xml version='1.0' encoding='us-ascii'?>
<!DOCTYPE entry SYSTEM 'catalogue.dtd'>
<entry datestamp='$Date$' modifier='$Author$' id='picture'>
  <name>picture</name>
  <caption>Dimens for picture macros.</caption>
  <authorref id='auth:oberdiek'/>
  <copyright owner='Heiko Oberdiek' year='2006-2009'/>
  <license type='lppl1.3'/>
  <version number='1.3'/>
  <description>
    There are macro and environment arguments that expect numbers
    that will internally be multiplied by <tt>\unitlength</tt>.
    This package extends the syntax of these arguments, so that
    dimensions with calculation support may be used for these arguments.
    <p/>
    The package is part of the <xref refid='oberdiek'>oberdiek</xref> bundle.
  </description>
  <documentation details='Package documentation'
      href='ctan:/macros/latex/contrib/oberdiek/picture.pdf'/>
  <ctan file='true' path='/macros/latex/contrib/oberdiek/picture.dtx'/>
  <miktex location='oberdiek'/>
  <texlive location='oberdiek'/>
  <install path='/macros/latex/contrib/oberdiek/oberdiek.tds.zip'/>
</entry>
%</catalogue>
%    \end{macrocode}
%
% \begin{History}
%   \begin{Version}{2006/08/26 v1.0}
%   \item
%     First released version. (First start of the project was June/July 2002.)
%   \end{Version}
%   \begin{Version}{2007/04/11 v1.1}
%   \item
%     Line ends sanitized.
%   \end{Version}
%   \begin{Version}{2008/11/26 v1.2}
%   \item
%     Package \xpackage{pict2e} added to documentation section ``Supported
%     packages''.
%   \item
%     Package order of supported packages is checked.
%   \end{Version}
%   \begin{Version}{2009/10/11 v1.3}
%   \item
%     Fix because of new version v4.3 of package \xpackage{calc}.
%   \end{Version}
% \end{History}
%
% \PrintIndex
%
% \Finale
\endinput

%        (quote the arguments according to the demands of your shell)
%
% Documentation:
%    (a) If picture.drv is present:
%           latex picture.drv
%    (b) Without picture.drv:
%           latex picture.dtx; ...
%    The class ltxdoc loads the configuration file ltxdoc.cfg
%    if available. Here you can specify further options, e.g.
%    use A4 as paper format:
%       \PassOptionsToClass{a4paper}{article}
%
%    Programm calls to get the documentation (example):
%       pdflatex picture.dtx
%       makeindex -s gind.ist picture.idx
%       pdflatex picture.dtx
%       makeindex -s gind.ist picture.idx
%       pdflatex picture.dtx
%
% Installation:
%    TDS:tex/latex/oberdiek/picture.sty
%    TDS:doc/latex/oberdiek/picture.pdf
%    TDS:doc/latex/oberdiek/picture-example.tex
%    TDS:source/latex/oberdiek/picture.dtx
%
%<*ignore>
\begingroup
  \catcode123=1 %
  \catcode125=2 %
  \def\x{LaTeX2e}%
\expandafter\endgroup
\ifcase 0\ifx\install y1\fi\expandafter
         \ifx\csname processbatchFile\endcsname\relax\else1\fi
         \ifx\fmtname\x\else 1\fi\relax
\else\csname fi\endcsname
%</ignore>
%<*install>
\input docstrip.tex
\Msg{************************************************************************}
\Msg{* Installation}
\Msg{* Package: picture 2009/10/11 v1.3 Dimens for picture macros (HO)}
\Msg{************************************************************************}

\keepsilent
\askforoverwritefalse

\let\MetaPrefix\relax
\preamble

This is a generated file.

Project: picture
Version: 2009/10/11 v1.3

Copyright (C) 2006-2009 by
   Heiko Oberdiek <heiko.oberdiek at googlemail.com>

This work may be distributed and/or modified under the
conditions of the LaTeX Project Public License, either
version 1.3c of this license or (at your option) any later
version. This version of this license is in
   http://www.latex-project.org/lppl/lppl-1-3c.txt
and the latest version of this license is in
   http://www.latex-project.org/lppl.txt
and version 1.3 or later is part of all distributions of
LaTeX version 2005/12/01 or later.

This work has the LPPL maintenance status "maintained".

This Current Maintainer of this work is Heiko Oberdiek.

This work consists of the main source file picture.dtx
and the derived files
   picture.sty, picture.pdf, picture.ins, picture.drv,
   picture-example.tex.

\endpreamble
\let\MetaPrefix\DoubleperCent

\generate{%
  \file{picture.ins}{\from{picture.dtx}{install}}%
  \file{picture.drv}{\from{picture.dtx}{driver}}%
  \usedir{tex/latex/oberdiek}%
  \file{picture.sty}{\from{picture.dtx}{package}}%
  \usedir{doc/latex/oberdiek}%
  \file{picture-example.tex}{\from{picture.dtx}{example}}%
  \nopreamble
  \nopostamble
  \usedir{source/latex/oberdiek/catalogue}%
  \file{picture.xml}{\from{picture.dtx}{catalogue}}%
}

\catcode32=13\relax% active space
\let =\space%
\Msg{************************************************************************}
\Msg{*}
\Msg{* To finish the installation you have to move the following}
\Msg{* file into a directory searched by TeX:}
\Msg{*}
\Msg{*     picture.sty}
\Msg{*}
\Msg{* To produce the documentation run the file `picture.drv'}
\Msg{* through LaTeX.}
\Msg{*}
\Msg{* Happy TeXing!}
\Msg{*}
\Msg{************************************************************************}

\endbatchfile
%</install>
%<*ignore>
\fi
%</ignore>
%<*driver>
\NeedsTeXFormat{LaTeX2e}
\ProvidesFile{picture.drv}%
  [2009/10/11 v1.3 Dimens for picture macros (HO)]%
\documentclass{ltxdoc}
\usepackage{holtxdoc}[2011/11/22]
\begin{document}
  \DocInput{picture.dtx}%
\end{document}
%</driver>
% \fi
%
% \CheckSum{322}
%
% \CharacterTable
%  {Upper-case    \A\B\C\D\E\F\G\H\I\J\K\L\M\N\O\P\Q\R\S\T\U\V\W\X\Y\Z
%   Lower-case    \a\b\c\d\e\f\g\h\i\j\k\l\m\n\o\p\q\r\s\t\u\v\w\x\y\z
%   Digits        \0\1\2\3\4\5\6\7\8\9
%   Exclamation   \!     Double quote  \"     Hash (number) \#
%   Dollar        \$     Percent       \%     Ampersand     \&
%   Acute accent  \'     Left paren    \(     Right paren   \)
%   Asterisk      \*     Plus          \+     Comma         \,
%   Minus         \-     Point         \.     Solidus       \/
%   Colon         \:     Semicolon     \;     Less than     \<
%   Equals        \=     Greater than  \>     Question mark \?
%   Commercial at \@     Left bracket  \[     Backslash     \\
%   Right bracket \]     Circumflex    \^     Underscore    \_
%   Grave accent  \`     Left brace    \{     Vertical bar  \|
%   Right brace   \}     Tilde         \~}
%
% \GetFileInfo{picture.drv}
%
% \title{The \xpackage{picture} package}
% \date{2009/10/11 v1.3}
% \author{Heiko Oberdiek\\\xemail{heiko.oberdiek at googlemail.com}}
%
% \maketitle
%
% \begin{abstract}
% There are macro and environment arguments that expect numbers
% that will internally be multiplicated with \cs{unitlength}.
% This package extends the syntax of these arguments that
% dimens with calculation support can be added for these arguments.
% \end{abstract}
%
% \tableofcontents
%
% \section{User interface}
%
% \subsection{Introduction}
%
% The environment |picture| and macros such as \cs{put}, \cs{line},
% \cs{vector} and other macros have arguments that expect numbers
% that are used as factor for \cs{unitlength}.
% This package redefines such macros and adds code that detects
% whether such an argument is given as number or as length.
% In the latter case, the length is used directly without multiplying
% with \cs{unitlength}.
%
% \subsection{Options}
%
% Depending on the available features, also length expressions
% can be given. Option \xoption{calc} loads package \xpackage{calc}.
% Then expressions of these package may be used.
% Otherwise \xoption{etex} wraps the length argument inside
% \cs{dimexpr}\dots\cs{relax}, if \eTeX\ is available. Otherwise
% option \xoption{plain} uses plain assignments without calculation
% support.
%
% The default is \xoption{calc} if package \xpackage{calc}
% is loaded before package \xoption{picture}. If you specify
% option \xoption{calc} the loading of \xpackage{calc} is ensured.
% Otherwise package \xpackage{picture} looks whether \cs{dimexpr}
% is available and uses then option \xoption{etex} as default.
% If \eTeX\ also could not be found, then \xoption{plain} is used.
%
% \subsection{Example}
%
%    \begin{macrocode}
%<*example>
\documentclass{article}

\usepackage[calc]{picture}

\begin{document}

\setlength{\unitlength}{1pt}

\begin{picture}(\widthof{Hello World}, 10mm)
  \put(0, 0){\makebox(0,0)[lb]{Hello World}}%
  \put(0, \heightof{Hello World} + \fboxsep){%
    \line(1, 0){\widthof{Hello World}}%
  }%
  \put(\widthof{Hello World}, 10mm){%
    \line(0, -1){10mm}%
  }%
\end{picture}

\end{document}
%</example>
%    \end{macrocode}
%
% \subsection{Supported packages}
%
% Packages \xpackage{pspicture} and \xpackage{pict2e} are supported,
% but they must be loaded before package \xpackage{picture}.
%
% New macros can be supported by \cs{picture@redefine}.
% The first argument is the macro which contains the arguments
% in its parameter text that you want to support by package \xpackage{picture}.
% The second argument contains the parameter text. Change \# to \& for
% the arguments in question.
% Examples (already used by package \xpackage{picture}):
% \begin{quote}
% |\picture@redefine\put{(&1,&2)}|\\
% |\picture@redefine\line{(#1,#2)&3}|
% \end{quote}
%
% \StopEventually{
% }
%
% \section{Implementation}
%
% \subsection{Identification}
%
%    \begin{macrocode}
%<*package>
\NeedsTeXFormat{LaTeX2e}
\ProvidesPackage{picture}%
  [2009/10/11 v1.3 Dimens for picture macros (HO)]%
%    \end{macrocode}
%
% \subsection{Options}
%
%    \begin{macrocode}
\def\Pc@calcname{calc}
\def\Pc@etexname{etex}
\def\Pc@plainname{plain}
%    \end{macrocode}
%    \begin{macro}{\Pc@method}
%    Macro \cs{Pc@method} stores the method to use for calculations.
%    Check which features are available and set the default for \cs{Pc@method}.
%    \begin{macrocode}
\@ifpackageloaded{calc}{%
  \let\Pc@method\Pc@calcname
}{%
  \begingroup\expandafter\expandafter\expandafter\endgroup
  \expandafter\ifx\csname dimexpr\endcsname\relax
    \let\Pc@method\Pc@plainname
  \else
    \let\Pc@method\Pc@etexname
  \fi
}
%    \end{macrocode}
%    \end{macro}
%    \begin{macrocode}
\DeclareOption{plain}{%
  \let\Pc@method\Pc@plainname
}
\DeclareOption{etex}{%
  \begingroup\expandafter\expandafter\expandafter\endgroup
  \expandafter\ifx\csname dimexpr\endcsname\relax
    \PackageError{picture}{%
      e-TeX is not available%
    }\@ehc
  \else
    \let\Pc@method\Pc@etexname
  \fi
}
\DeclareOption{calc}{%
  \let\Pc@method\Pc@calcname
}
\ProcessOptions*
\begingroup
  \let\on@line\@empty
  \PackageInfo{picture}{Calculation method: \Pc@method}%
\endgroup
%    \end{macrocode}
%
% \subsection{Calculation method}
%
%    \begin{macrocode}
\ifx\Pc@method\Pc@calcname
  \RequirePackage{calc}%
\fi
%    \end{macrocode}
%
% \subsubsection{Method \xoption{calc}}
%
%    \begin{macrocode}
\ifx\Pc@method\Pc@calcname
  \def\Pc@tokslength#1{%
    \begingroup
      \let\calc@error\Pc@calc@error
      \setlength\dimen@{#1\unitlength}\Pc@next\Pc@nil{#1}%
  }%
  \let\PcOrg@calc@error\calc@error
  \@ifpackagelater{calc}{2007/08/22}{% v4.3
    \def\Pc@calc@error#1{%
      \expandafter\ifx\expandafter\unitlength\noexpand#1\relax
        \def\calc@next##1!{%
          \endgroup
          \aftergroup\afterassignment
          \aftergroup\Pc@next
        }%
        \expandafter\@firstoftwo
      \else
        \expandafter\@secondoftwo
      \fi
      {%
        \calc@next{#1}%
      }{%
        \PcOrg@calc@error{#1}%
      }%
    }%
  }{%
    \def\Pc@calc@error#1{%
      \expandafter\ifx\expandafter\unitlength\noexpand#1\relax
        \def\calc@next##1!{%
          \endgroup
          \aftergroup\afterassignment
          \aftergroup\Pc@next
        }%
        \expandafter\@gobble
      \else
        \expandafter\@firstofone
      \fi
      {%
        \PcOrg@calc@error{#1}%
      }%
    }%
  }%
\fi
%    \end{macrocode}
%
% \subsubsection{Method \xoption{etex}}
%
%    \begin{macrocode}
\ifx\Pc@method\Pc@etexname
  \def\Pc@tokslength#1{%
    \begingroup
      \afterassignment\Pc@next
      \dimen@=\dimexpr#1\unitlength\Pc@nil{#1}%
  }%
\fi
%    \end{macrocode}
%
% \subsubsection{Method \xoption{plain}}
%
%    \begin{macrocode}
\ifx\Pc@method\Pc@plainname
  \def\Pc@tokslength#1{%
    \begingroup
      \afterassignment\Pc@next
      \dimen@=#1\unitlength\Pc@nil{#1}%
  }%
\fi
%    \end{macrocode}
%
% \subsubsection{Help macros}
%
%    \begin{macrocode}
\def\Pc@next#1\Pc@nil#2{%
  \ifx\\#1\\%
    \endgroup
    \Pc@addtoks{{#2}}%
  \else
    \expandafter\endgroup
    \expandafter\Pc@addtoks\expandafter{%
      \expandafter{\the\dimen@\@gobble}%
    }%
  \fi
}
%    \end{macrocode}
%    \begin{macro}{\Pc@nil}
%    \cs{Pc@nil} must not have the meaning of \cs{relax} because of
%    \cs{dimexpr}.
%    \begin{macrocode}
\let\Pc@nil\message
%    \end{macrocode}
%    \end{macro}
%    \begin{macro}{\Pc@addtoks}
%    \begin{macrocode}
\def\Pc@addtoks#1{%
  \toks@=\expandafter{\the\toks@#1}%
}
%    \end{macrocode}
%    \end{macro}
%    \begin{macro}{\Pc@init}
%    \begin{macrocode}
\def\Pc@init#1{%
  \begingroup
    \toks@={#1}%
}
%    \end{macrocode}
%    \end{macro}
%    \begin{macro}{\Pc@finish}
%    \begin{macrocode}
\def\Pc@finish#1{%
  \expandafter\endgroup
  \expandafter#1\the\toks@
}
%    \end{macrocode}
%    \end{macro}
%
% \subsection{Redefinitions}
%
%    \begin{macro}{\picture@redefine}
%    \begin{tabular}[t]{@{}l@{ }l@{}}
%    |#1|:& command name\\
%    |#2|:& parameter text, length parameter with \& instead of \#
%    \end{tabular}
%    \begin{macrocode}
\def\picture@redefine#1#2{%
  \begingroup
    \edef\reserved@a{%
      \noexpand\noexpand
      \expandafter\noexpand
          \csname PcOrg@\expandafter\@gobble\string#1\endcsname
    }%
    \toks0{#1}%
    \Pc@first#2&0%
}
%    \end{macrocode}
%    \end{macro}
%    \begin{macro}{\Pc@first}
%    \begin{macrocode}
\def\Pc@first#1&{%
  \toks1={#1}%
  \toks2={\Pc@init{#1}}%
  \Pc@scanlength
}
%    \end{macrocode}
%    \end{macro}
%    \begin{macro}{\Pc@scanlength}
%    |#1|: number of length parameter or zero
%    \begin{macrocode}
\def\Pc@scanlength#1{%
  \ifcase#1 %
    \expandafter\Pc@last
  \else
    \toks1=\expandafter{\the\toks1 ###1}%
    \toks2=\expandafter{\the\toks2 \Pc@tokslength{###1}}%
    \expandafter\Pc@scannext
  \fi
}
%    \end{macrocode}
%    \end{macro}
%    \begin{macro}{\Pc@scannext}
%    \begin{macrocode}
\def\Pc@scannext#1&{%
  \ifx\\#1\\%
  \else
    \toks1=\expandafter{\the\toks1 #1}%
    \toks2=\expandafter{\the\toks2 \Pc@addtoks{#1}}%
  \fi
  \Pc@scanlength
}
%    \end{macrocode}
%    \end{macro}
%    \begin{macro}{\Pc@last}
%    \begin{macrocode}
\def\Pc@last{%
  \edef\x{%
    \endgroup
    \let\reserved@a\the\toks0 %
    \def\the\toks0 \the\toks1 {%
      \the\toks2 %
      \noexpand\Pc@finish\reserved@a
    }%
  }%
  \x
}
%    \end{macrocode}
%    \end{macro}
%
% \subsubsection{\LaTeX\ base macros}
%
%    \begin{macrocode}
\picture@redefine\@picture{(&1,&2)(&3,&4)}
\picture@redefine\put{(&1,&2)}
\picture@redefine\multiput{(&1,&2)}
\picture@redefine\@multiput{(&1,&2)}
\picture@redefine\line{(#1,#2)&3}
\picture@redefine\vector{(#1,#2)&3}
\picture@redefine\dashbox{&1(&2,&3)}
\picture@redefine\@circle{&1}
\picture@redefine\@dot{&1}
\picture@redefine\@bezier{#1(&2,&3)(&4,&5)(&6,&7)}
\picture@redefine\@imakepicbox{(&1,&2)}
%    \end{macrocode}
%
% \subsubsection{Package \xpackage{pspicture}}
%
%     Package \xpackage{pspicture} changes the signature of \cs{@oval}
%     by adding an optional argument.
%    \begin{macrocode}
\@ifpackageloaded{pspicture}{%
  \picture@redefine\@oval{[&1](&2,&3)}%
  \picture@redefine\Line{(&1,&2)}%
  \picture@redefine\Curve{(&1,&2)}%
  \picture@redefine\Vector{(&1,&2)}%
}{%
  \picture@redefine\@oval{(&1,&2)}%
}
%    \end{macrocode}
%
% \subsection{Check package loading order}
%
%    \begin{macro}{\PC@checkpackage}
%    \begin{macrocode}
\def\Pc@checkpackage#1{%
  \@ifpackageloaded{#1}{%
  }{%
    \AtBeginDocument{%
      \@ifpackageloaded{#1}{%
        \PackageWarningNoLine{picture}{%
          Package `#1' is loaded after `picture'.\MessageBreak
          Load package `picture' afterwards to get full support%
          \MessageBreak
          of its additional syntax with length specifications%
        }%
      }{}%
    }%
  }%
}
%    \end{macrocode}
%    \end{macro}
%    \begin{macrocode}
\Pc@checkpackage{pict2e}
\Pc@checkpackage{pspicture}
%    \end{macrocode}
%
%    \begin{macrocode}
%</package>
%    \end{macrocode}
%
% \section{Installation}
%
% \subsection{Download}
%
% \paragraph{Package.} This package is available on
% CTAN\footnote{\url{ftp://ftp.ctan.org/tex-archive/}}:
% \begin{description}
% \item[\CTAN{macros/latex/contrib/oberdiek/picture.dtx}] The source file.
% \item[\CTAN{macros/latex/contrib/oberdiek/picture.pdf}] Documentation.
% \end{description}
%
%
% \paragraph{Bundle.} All the packages of the bundle `oberdiek'
% are also available in a TDS compliant ZIP archive. There
% the packages are already unpacked and the documentation files
% are generated. The files and directories obey the TDS standard.
% \begin{description}
% \item[\CTAN{install/macros/latex/contrib/oberdiek.tds.zip}]
% \end{description}
% \emph{TDS} refers to the standard ``A Directory Structure
% for \TeX\ Files'' (\CTAN{tds/tds.pdf}). Directories
% with \xfile{texmf} in their name are usually organized this way.
%
% \subsection{Bundle installation}
%
% \paragraph{Unpacking.} Unpack the \xfile{oberdiek.tds.zip} in the
% TDS tree (also known as \xfile{texmf} tree) of your choice.
% Example (linux):
% \begin{quote}
%   |unzip oberdiek.tds.zip -d ~/texmf|
% \end{quote}
%
% \paragraph{Script installation.}
% Check the directory \xfile{TDS:scripts/oberdiek/} for
% scripts that need further installation steps.
% Package \xpackage{attachfile2} comes with the Perl script
% \xfile{pdfatfi.pl} that should be installed in such a way
% that it can be called as \texttt{pdfatfi}.
% Example (linux):
% \begin{quote}
%   |chmod +x scripts/oberdiek/pdfatfi.pl|\\
%   |cp scripts/oberdiek/pdfatfi.pl /usr/local/bin/|
% \end{quote}
%
% \subsection{Package installation}
%
% \paragraph{Unpacking.} The \xfile{.dtx} file is a self-extracting
% \docstrip\ archive. The files are extracted by running the
% \xfile{.dtx} through \plainTeX:
% \begin{quote}
%   \verb|tex picture.dtx|
% \end{quote}
%
% \paragraph{TDS.} Now the different files must be moved into
% the different directories in your installation TDS tree
% (also known as \xfile{texmf} tree):
% \begin{quote}
% \def\t{^^A
% \begin{tabular}{@{}>{\ttfamily}l@{ $\rightarrow$ }>{\ttfamily}l@{}}
%   picture.sty & tex/latex/oberdiek/picture.sty\\
%   picture.pdf & doc/latex/oberdiek/picture.pdf\\
%   picture-example.tex & doc/latex/oberdiek/picture-example.tex\\
%   picture.dtx & source/latex/oberdiek/picture.dtx\\
% \end{tabular}^^A
% }^^A
% \sbox0{\t}^^A
% \ifdim\wd0>\linewidth
%   \begingroup
%     \advance\linewidth by\leftmargin
%     \advance\linewidth by\rightmargin
%   \edef\x{\endgroup
%     \def\noexpand\lw{\the\linewidth}^^A
%   }\x
%   \def\lwbox{^^A
%     \leavevmode
%     \hbox to \linewidth{^^A
%       \kern-\leftmargin\relax
%       \hss
%       \usebox0
%       \hss
%       \kern-\rightmargin\relax
%     }^^A
%   }^^A
%   \ifdim\wd0>\lw
%     \sbox0{\small\t}^^A
%     \ifdim\wd0>\linewidth
%       \ifdim\wd0>\lw
%         \sbox0{\footnotesize\t}^^A
%         \ifdim\wd0>\linewidth
%           \ifdim\wd0>\lw
%             \sbox0{\scriptsize\t}^^A
%             \ifdim\wd0>\linewidth
%               \ifdim\wd0>\lw
%                 \sbox0{\tiny\t}^^A
%                 \ifdim\wd0>\linewidth
%                   \lwbox
%                 \else
%                   \usebox0
%                 \fi
%               \else
%                 \lwbox
%               \fi
%             \else
%               \usebox0
%             \fi
%           \else
%             \lwbox
%           \fi
%         \else
%           \usebox0
%         \fi
%       \else
%         \lwbox
%       \fi
%     \else
%       \usebox0
%     \fi
%   \else
%     \lwbox
%   \fi
% \else
%   \usebox0
% \fi
% \end{quote}
% If you have a \xfile{docstrip.cfg} that configures and enables \docstrip's
% TDS installing feature, then some files can already be in the right
% place, see the documentation of \docstrip.
%
% \subsection{Refresh file name databases}
%
% If your \TeX~distribution
% (\teTeX, \mikTeX, \dots) relies on file name databases, you must refresh
% these. For example, \teTeX\ users run \verb|texhash| or
% \verb|mktexlsr|.
%
% \subsection{Some details for the interested}
%
% \paragraph{Attached source.}
%
% The PDF documentation on CTAN also includes the
% \xfile{.dtx} source file. It can be extracted by
% AcrobatReader 6 or higher. Another option is \textsf{pdftk},
% e.g. unpack the file into the current directory:
% \begin{quote}
%   \verb|pdftk picture.pdf unpack_files output .|
% \end{quote}
%
% \paragraph{Unpacking with \LaTeX.}
% The \xfile{.dtx} chooses its action depending on the format:
% \begin{description}
% \item[\plainTeX:] Run \docstrip\ and extract the files.
% \item[\LaTeX:] Generate the documentation.
% \end{description}
% If you insist on using \LaTeX\ for \docstrip\ (really,
% \docstrip\ does not need \LaTeX), then inform the autodetect routine
% about your intention:
% \begin{quote}
%   \verb|latex \let\install=y% \iffalse meta-comment
%
% File: picture.dtx
% Version: 2009/10/11 v1.3
% Info: Dimens for picture macros
%
% Copyright (C) 2006-2009 by
%    Heiko Oberdiek <heiko.oberdiek at googlemail.com>
%
% This work may be distributed and/or modified under the
% conditions of the LaTeX Project Public License, either
% version 1.3c of this license or (at your option) any later
% version. This version of this license is in
%    http://www.latex-project.org/lppl/lppl-1-3c.txt
% and the latest version of this license is in
%    http://www.latex-project.org/lppl.txt
% and version 1.3 or later is part of all distributions of
% LaTeX version 2005/12/01 or later.
%
% This work has the LPPL maintenance status "maintained".
%
% This Current Maintainer of this work is Heiko Oberdiek.
%
% This work consists of the main source file picture.dtx
% and the derived files
%    picture.sty, picture.pdf, picture.ins, picture.drv,
%    picture-example.tex.
%
% Distribution:
%    CTAN:macros/latex/contrib/oberdiek/picture.dtx
%    CTAN:macros/latex/contrib/oberdiek/picture.pdf
%
% Unpacking:
%    (a) If picture.ins is present:
%           tex picture.ins
%    (b) Without picture.ins:
%           tex picture.dtx
%    (c) If you insist on using LaTeX
%           latex \let\install=y% \iffalse meta-comment
%
% File: picture.dtx
% Version: 2009/10/11 v1.3
% Info: Dimens for picture macros
%
% Copyright (C) 2006-2009 by
%    Heiko Oberdiek <heiko.oberdiek at googlemail.com>
%
% This work may be distributed and/or modified under the
% conditions of the LaTeX Project Public License, either
% version 1.3c of this license or (at your option) any later
% version. This version of this license is in
%    http://www.latex-project.org/lppl/lppl-1-3c.txt
% and the latest version of this license is in
%    http://www.latex-project.org/lppl.txt
% and version 1.3 or later is part of all distributions of
% LaTeX version 2005/12/01 or later.
%
% This work has the LPPL maintenance status "maintained".
%
% This Current Maintainer of this work is Heiko Oberdiek.
%
% This work consists of the main source file picture.dtx
% and the derived files
%    picture.sty, picture.pdf, picture.ins, picture.drv,
%    picture-example.tex.
%
% Distribution:
%    CTAN:macros/latex/contrib/oberdiek/picture.dtx
%    CTAN:macros/latex/contrib/oberdiek/picture.pdf
%
% Unpacking:
%    (a) If picture.ins is present:
%           tex picture.ins
%    (b) Without picture.ins:
%           tex picture.dtx
%    (c) If you insist on using LaTeX
%           latex \let\install=y% \iffalse meta-comment
%
% File: picture.dtx
% Version: 2009/10/11 v1.3
% Info: Dimens for picture macros
%
% Copyright (C) 2006-2009 by
%    Heiko Oberdiek <heiko.oberdiek at googlemail.com>
%
% This work may be distributed and/or modified under the
% conditions of the LaTeX Project Public License, either
% version 1.3c of this license or (at your option) any later
% version. This version of this license is in
%    http://www.latex-project.org/lppl/lppl-1-3c.txt
% and the latest version of this license is in
%    http://www.latex-project.org/lppl.txt
% and version 1.3 or later is part of all distributions of
% LaTeX version 2005/12/01 or later.
%
% This work has the LPPL maintenance status "maintained".
%
% This Current Maintainer of this work is Heiko Oberdiek.
%
% This work consists of the main source file picture.dtx
% and the derived files
%    picture.sty, picture.pdf, picture.ins, picture.drv,
%    picture-example.tex.
%
% Distribution:
%    CTAN:macros/latex/contrib/oberdiek/picture.dtx
%    CTAN:macros/latex/contrib/oberdiek/picture.pdf
%
% Unpacking:
%    (a) If picture.ins is present:
%           tex picture.ins
%    (b) Without picture.ins:
%           tex picture.dtx
%    (c) If you insist on using LaTeX
%           latex \let\install=y\input{picture.dtx}
%        (quote the arguments according to the demands of your shell)
%
% Documentation:
%    (a) If picture.drv is present:
%           latex picture.drv
%    (b) Without picture.drv:
%           latex picture.dtx; ...
%    The class ltxdoc loads the configuration file ltxdoc.cfg
%    if available. Here you can specify further options, e.g.
%    use A4 as paper format:
%       \PassOptionsToClass{a4paper}{article}
%
%    Programm calls to get the documentation (example):
%       pdflatex picture.dtx
%       makeindex -s gind.ist picture.idx
%       pdflatex picture.dtx
%       makeindex -s gind.ist picture.idx
%       pdflatex picture.dtx
%
% Installation:
%    TDS:tex/latex/oberdiek/picture.sty
%    TDS:doc/latex/oberdiek/picture.pdf
%    TDS:doc/latex/oberdiek/picture-example.tex
%    TDS:source/latex/oberdiek/picture.dtx
%
%<*ignore>
\begingroup
  \catcode123=1 %
  \catcode125=2 %
  \def\x{LaTeX2e}%
\expandafter\endgroup
\ifcase 0\ifx\install y1\fi\expandafter
         \ifx\csname processbatchFile\endcsname\relax\else1\fi
         \ifx\fmtname\x\else 1\fi\relax
\else\csname fi\endcsname
%</ignore>
%<*install>
\input docstrip.tex
\Msg{************************************************************************}
\Msg{* Installation}
\Msg{* Package: picture 2009/10/11 v1.3 Dimens for picture macros (HO)}
\Msg{************************************************************************}

\keepsilent
\askforoverwritefalse

\let\MetaPrefix\relax
\preamble

This is a generated file.

Project: picture
Version: 2009/10/11 v1.3

Copyright (C) 2006-2009 by
   Heiko Oberdiek <heiko.oberdiek at googlemail.com>

This work may be distributed and/or modified under the
conditions of the LaTeX Project Public License, either
version 1.3c of this license or (at your option) any later
version. This version of this license is in
   http://www.latex-project.org/lppl/lppl-1-3c.txt
and the latest version of this license is in
   http://www.latex-project.org/lppl.txt
and version 1.3 or later is part of all distributions of
LaTeX version 2005/12/01 or later.

This work has the LPPL maintenance status "maintained".

This Current Maintainer of this work is Heiko Oberdiek.

This work consists of the main source file picture.dtx
and the derived files
   picture.sty, picture.pdf, picture.ins, picture.drv,
   picture-example.tex.

\endpreamble
\let\MetaPrefix\DoubleperCent

\generate{%
  \file{picture.ins}{\from{picture.dtx}{install}}%
  \file{picture.drv}{\from{picture.dtx}{driver}}%
  \usedir{tex/latex/oberdiek}%
  \file{picture.sty}{\from{picture.dtx}{package}}%
  \usedir{doc/latex/oberdiek}%
  \file{picture-example.tex}{\from{picture.dtx}{example}}%
  \nopreamble
  \nopostamble
  \usedir{source/latex/oberdiek/catalogue}%
  \file{picture.xml}{\from{picture.dtx}{catalogue}}%
}

\catcode32=13\relax% active space
\let =\space%
\Msg{************************************************************************}
\Msg{*}
\Msg{* To finish the installation you have to move the following}
\Msg{* file into a directory searched by TeX:}
\Msg{*}
\Msg{*     picture.sty}
\Msg{*}
\Msg{* To produce the documentation run the file `picture.drv'}
\Msg{* through LaTeX.}
\Msg{*}
\Msg{* Happy TeXing!}
\Msg{*}
\Msg{************************************************************************}

\endbatchfile
%</install>
%<*ignore>
\fi
%</ignore>
%<*driver>
\NeedsTeXFormat{LaTeX2e}
\ProvidesFile{picture.drv}%
  [2009/10/11 v1.3 Dimens for picture macros (HO)]%
\documentclass{ltxdoc}
\usepackage{holtxdoc}[2011/11/22]
\begin{document}
  \DocInput{picture.dtx}%
\end{document}
%</driver>
% \fi
%
% \CheckSum{322}
%
% \CharacterTable
%  {Upper-case    \A\B\C\D\E\F\G\H\I\J\K\L\M\N\O\P\Q\R\S\T\U\V\W\X\Y\Z
%   Lower-case    \a\b\c\d\e\f\g\h\i\j\k\l\m\n\o\p\q\r\s\t\u\v\w\x\y\z
%   Digits        \0\1\2\3\4\5\6\7\8\9
%   Exclamation   \!     Double quote  \"     Hash (number) \#
%   Dollar        \$     Percent       \%     Ampersand     \&
%   Acute accent  \'     Left paren    \(     Right paren   \)
%   Asterisk      \*     Plus          \+     Comma         \,
%   Minus         \-     Point         \.     Solidus       \/
%   Colon         \:     Semicolon     \;     Less than     \<
%   Equals        \=     Greater than  \>     Question mark \?
%   Commercial at \@     Left bracket  \[     Backslash     \\
%   Right bracket \]     Circumflex    \^     Underscore    \_
%   Grave accent  \`     Left brace    \{     Vertical bar  \|
%   Right brace   \}     Tilde         \~}
%
% \GetFileInfo{picture.drv}
%
% \title{The \xpackage{picture} package}
% \date{2009/10/11 v1.3}
% \author{Heiko Oberdiek\\\xemail{heiko.oberdiek at googlemail.com}}
%
% \maketitle
%
% \begin{abstract}
% There are macro and environment arguments that expect numbers
% that will internally be multiplicated with \cs{unitlength}.
% This package extends the syntax of these arguments that
% dimens with calculation support can be added for these arguments.
% \end{abstract}
%
% \tableofcontents
%
% \section{User interface}
%
% \subsection{Introduction}
%
% The environment |picture| and macros such as \cs{put}, \cs{line},
% \cs{vector} and other macros have arguments that expect numbers
% that are used as factor for \cs{unitlength}.
% This package redefines such macros and adds code that detects
% whether such an argument is given as number or as length.
% In the latter case, the length is used directly without multiplying
% with \cs{unitlength}.
%
% \subsection{Options}
%
% Depending on the available features, also length expressions
% can be given. Option \xoption{calc} loads package \xpackage{calc}.
% Then expressions of these package may be used.
% Otherwise \xoption{etex} wraps the length argument inside
% \cs{dimexpr}\dots\cs{relax}, if \eTeX\ is available. Otherwise
% option \xoption{plain} uses plain assignments without calculation
% support.
%
% The default is \xoption{calc} if package \xpackage{calc}
% is loaded before package \xoption{picture}. If you specify
% option \xoption{calc} the loading of \xpackage{calc} is ensured.
% Otherwise package \xpackage{picture} looks whether \cs{dimexpr}
% is available and uses then option \xoption{etex} as default.
% If \eTeX\ also could not be found, then \xoption{plain} is used.
%
% \subsection{Example}
%
%    \begin{macrocode}
%<*example>
\documentclass{article}

\usepackage[calc]{picture}

\begin{document}

\setlength{\unitlength}{1pt}

\begin{picture}(\widthof{Hello World}, 10mm)
  \put(0, 0){\makebox(0,0)[lb]{Hello World}}%
  \put(0, \heightof{Hello World} + \fboxsep){%
    \line(1, 0){\widthof{Hello World}}%
  }%
  \put(\widthof{Hello World}, 10mm){%
    \line(0, -1){10mm}%
  }%
\end{picture}

\end{document}
%</example>
%    \end{macrocode}
%
% \subsection{Supported packages}
%
% Packages \xpackage{pspicture} and \xpackage{pict2e} are supported,
% but they must be loaded before package \xpackage{picture}.
%
% New macros can be supported by \cs{picture@redefine}.
% The first argument is the macro which contains the arguments
% in its parameter text that you want to support by package \xpackage{picture}.
% The second argument contains the parameter text. Change \# to \& for
% the arguments in question.
% Examples (already used by package \xpackage{picture}):
% \begin{quote}
% |\picture@redefine\put{(&1,&2)}|\\
% |\picture@redefine\line{(#1,#2)&3}|
% \end{quote}
%
% \StopEventually{
% }
%
% \section{Implementation}
%
% \subsection{Identification}
%
%    \begin{macrocode}
%<*package>
\NeedsTeXFormat{LaTeX2e}
\ProvidesPackage{picture}%
  [2009/10/11 v1.3 Dimens for picture macros (HO)]%
%    \end{macrocode}
%
% \subsection{Options}
%
%    \begin{macrocode}
\def\Pc@calcname{calc}
\def\Pc@etexname{etex}
\def\Pc@plainname{plain}
%    \end{macrocode}
%    \begin{macro}{\Pc@method}
%    Macro \cs{Pc@method} stores the method to use for calculations.
%    Check which features are available and set the default for \cs{Pc@method}.
%    \begin{macrocode}
\@ifpackageloaded{calc}{%
  \let\Pc@method\Pc@calcname
}{%
  \begingroup\expandafter\expandafter\expandafter\endgroup
  \expandafter\ifx\csname dimexpr\endcsname\relax
    \let\Pc@method\Pc@plainname
  \else
    \let\Pc@method\Pc@etexname
  \fi
}
%    \end{macrocode}
%    \end{macro}
%    \begin{macrocode}
\DeclareOption{plain}{%
  \let\Pc@method\Pc@plainname
}
\DeclareOption{etex}{%
  \begingroup\expandafter\expandafter\expandafter\endgroup
  \expandafter\ifx\csname dimexpr\endcsname\relax
    \PackageError{picture}{%
      e-TeX is not available%
    }\@ehc
  \else
    \let\Pc@method\Pc@etexname
  \fi
}
\DeclareOption{calc}{%
  \let\Pc@method\Pc@calcname
}
\ProcessOptions*
\begingroup
  \let\on@line\@empty
  \PackageInfo{picture}{Calculation method: \Pc@method}%
\endgroup
%    \end{macrocode}
%
% \subsection{Calculation method}
%
%    \begin{macrocode}
\ifx\Pc@method\Pc@calcname
  \RequirePackage{calc}%
\fi
%    \end{macrocode}
%
% \subsubsection{Method \xoption{calc}}
%
%    \begin{macrocode}
\ifx\Pc@method\Pc@calcname
  \def\Pc@tokslength#1{%
    \begingroup
      \let\calc@error\Pc@calc@error
      \setlength\dimen@{#1\unitlength}\Pc@next\Pc@nil{#1}%
  }%
  \let\PcOrg@calc@error\calc@error
  \@ifpackagelater{calc}{2007/08/22}{% v4.3
    \def\Pc@calc@error#1{%
      \expandafter\ifx\expandafter\unitlength\noexpand#1\relax
        \def\calc@next##1!{%
          \endgroup
          \aftergroup\afterassignment
          \aftergroup\Pc@next
        }%
        \expandafter\@firstoftwo
      \else
        \expandafter\@secondoftwo
      \fi
      {%
        \calc@next{#1}%
      }{%
        \PcOrg@calc@error{#1}%
      }%
    }%
  }{%
    \def\Pc@calc@error#1{%
      \expandafter\ifx\expandafter\unitlength\noexpand#1\relax
        \def\calc@next##1!{%
          \endgroup
          \aftergroup\afterassignment
          \aftergroup\Pc@next
        }%
        \expandafter\@gobble
      \else
        \expandafter\@firstofone
      \fi
      {%
        \PcOrg@calc@error{#1}%
      }%
    }%
  }%
\fi
%    \end{macrocode}
%
% \subsubsection{Method \xoption{etex}}
%
%    \begin{macrocode}
\ifx\Pc@method\Pc@etexname
  \def\Pc@tokslength#1{%
    \begingroup
      \afterassignment\Pc@next
      \dimen@=\dimexpr#1\unitlength\Pc@nil{#1}%
  }%
\fi
%    \end{macrocode}
%
% \subsubsection{Method \xoption{plain}}
%
%    \begin{macrocode}
\ifx\Pc@method\Pc@plainname
  \def\Pc@tokslength#1{%
    \begingroup
      \afterassignment\Pc@next
      \dimen@=#1\unitlength\Pc@nil{#1}%
  }%
\fi
%    \end{macrocode}
%
% \subsubsection{Help macros}
%
%    \begin{macrocode}
\def\Pc@next#1\Pc@nil#2{%
  \ifx\\#1\\%
    \endgroup
    \Pc@addtoks{{#2}}%
  \else
    \expandafter\endgroup
    \expandafter\Pc@addtoks\expandafter{%
      \expandafter{\the\dimen@\@gobble}%
    }%
  \fi
}
%    \end{macrocode}
%    \begin{macro}{\Pc@nil}
%    \cs{Pc@nil} must not have the meaning of \cs{relax} because of
%    \cs{dimexpr}.
%    \begin{macrocode}
\let\Pc@nil\message
%    \end{macrocode}
%    \end{macro}
%    \begin{macro}{\Pc@addtoks}
%    \begin{macrocode}
\def\Pc@addtoks#1{%
  \toks@=\expandafter{\the\toks@#1}%
}
%    \end{macrocode}
%    \end{macro}
%    \begin{macro}{\Pc@init}
%    \begin{macrocode}
\def\Pc@init#1{%
  \begingroup
    \toks@={#1}%
}
%    \end{macrocode}
%    \end{macro}
%    \begin{macro}{\Pc@finish}
%    \begin{macrocode}
\def\Pc@finish#1{%
  \expandafter\endgroup
  \expandafter#1\the\toks@
}
%    \end{macrocode}
%    \end{macro}
%
% \subsection{Redefinitions}
%
%    \begin{macro}{\picture@redefine}
%    \begin{tabular}[t]{@{}l@{ }l@{}}
%    |#1|:& command name\\
%    |#2|:& parameter text, length parameter with \& instead of \#
%    \end{tabular}
%    \begin{macrocode}
\def\picture@redefine#1#2{%
  \begingroup
    \edef\reserved@a{%
      \noexpand\noexpand
      \expandafter\noexpand
          \csname PcOrg@\expandafter\@gobble\string#1\endcsname
    }%
    \toks0{#1}%
    \Pc@first#2&0%
}
%    \end{macrocode}
%    \end{macro}
%    \begin{macro}{\Pc@first}
%    \begin{macrocode}
\def\Pc@first#1&{%
  \toks1={#1}%
  \toks2={\Pc@init{#1}}%
  \Pc@scanlength
}
%    \end{macrocode}
%    \end{macro}
%    \begin{macro}{\Pc@scanlength}
%    |#1|: number of length parameter or zero
%    \begin{macrocode}
\def\Pc@scanlength#1{%
  \ifcase#1 %
    \expandafter\Pc@last
  \else
    \toks1=\expandafter{\the\toks1 ###1}%
    \toks2=\expandafter{\the\toks2 \Pc@tokslength{###1}}%
    \expandafter\Pc@scannext
  \fi
}
%    \end{macrocode}
%    \end{macro}
%    \begin{macro}{\Pc@scannext}
%    \begin{macrocode}
\def\Pc@scannext#1&{%
  \ifx\\#1\\%
  \else
    \toks1=\expandafter{\the\toks1 #1}%
    \toks2=\expandafter{\the\toks2 \Pc@addtoks{#1}}%
  \fi
  \Pc@scanlength
}
%    \end{macrocode}
%    \end{macro}
%    \begin{macro}{\Pc@last}
%    \begin{macrocode}
\def\Pc@last{%
  \edef\x{%
    \endgroup
    \let\reserved@a\the\toks0 %
    \def\the\toks0 \the\toks1 {%
      \the\toks2 %
      \noexpand\Pc@finish\reserved@a
    }%
  }%
  \x
}
%    \end{macrocode}
%    \end{macro}
%
% \subsubsection{\LaTeX\ base macros}
%
%    \begin{macrocode}
\picture@redefine\@picture{(&1,&2)(&3,&4)}
\picture@redefine\put{(&1,&2)}
\picture@redefine\multiput{(&1,&2)}
\picture@redefine\@multiput{(&1,&2)}
\picture@redefine\line{(#1,#2)&3}
\picture@redefine\vector{(#1,#2)&3}
\picture@redefine\dashbox{&1(&2,&3)}
\picture@redefine\@circle{&1}
\picture@redefine\@dot{&1}
\picture@redefine\@bezier{#1(&2,&3)(&4,&5)(&6,&7)}
\picture@redefine\@imakepicbox{(&1,&2)}
%    \end{macrocode}
%
% \subsubsection{Package \xpackage{pspicture}}
%
%     Package \xpackage{pspicture} changes the signature of \cs{@oval}
%     by adding an optional argument.
%    \begin{macrocode}
\@ifpackageloaded{pspicture}{%
  \picture@redefine\@oval{[&1](&2,&3)}%
  \picture@redefine\Line{(&1,&2)}%
  \picture@redefine\Curve{(&1,&2)}%
  \picture@redefine\Vector{(&1,&2)}%
}{%
  \picture@redefine\@oval{(&1,&2)}%
}
%    \end{macrocode}
%
% \subsection{Check package loading order}
%
%    \begin{macro}{\PC@checkpackage}
%    \begin{macrocode}
\def\Pc@checkpackage#1{%
  \@ifpackageloaded{#1}{%
  }{%
    \AtBeginDocument{%
      \@ifpackageloaded{#1}{%
        \PackageWarningNoLine{picture}{%
          Package `#1' is loaded after `picture'.\MessageBreak
          Load package `picture' afterwards to get full support%
          \MessageBreak
          of its additional syntax with length specifications%
        }%
      }{}%
    }%
  }%
}
%    \end{macrocode}
%    \end{macro}
%    \begin{macrocode}
\Pc@checkpackage{pict2e}
\Pc@checkpackage{pspicture}
%    \end{macrocode}
%
%    \begin{macrocode}
%</package>
%    \end{macrocode}
%
% \section{Installation}
%
% \subsection{Download}
%
% \paragraph{Package.} This package is available on
% CTAN\footnote{\url{ftp://ftp.ctan.org/tex-archive/}}:
% \begin{description}
% \item[\CTAN{macros/latex/contrib/oberdiek/picture.dtx}] The source file.
% \item[\CTAN{macros/latex/contrib/oberdiek/picture.pdf}] Documentation.
% \end{description}
%
%
% \paragraph{Bundle.} All the packages of the bundle `oberdiek'
% are also available in a TDS compliant ZIP archive. There
% the packages are already unpacked and the documentation files
% are generated. The files and directories obey the TDS standard.
% \begin{description}
% \item[\CTAN{install/macros/latex/contrib/oberdiek.tds.zip}]
% \end{description}
% \emph{TDS} refers to the standard ``A Directory Structure
% for \TeX\ Files'' (\CTAN{tds/tds.pdf}). Directories
% with \xfile{texmf} in their name are usually organized this way.
%
% \subsection{Bundle installation}
%
% \paragraph{Unpacking.} Unpack the \xfile{oberdiek.tds.zip} in the
% TDS tree (also known as \xfile{texmf} tree) of your choice.
% Example (linux):
% \begin{quote}
%   |unzip oberdiek.tds.zip -d ~/texmf|
% \end{quote}
%
% \paragraph{Script installation.}
% Check the directory \xfile{TDS:scripts/oberdiek/} for
% scripts that need further installation steps.
% Package \xpackage{attachfile2} comes with the Perl script
% \xfile{pdfatfi.pl} that should be installed in such a way
% that it can be called as \texttt{pdfatfi}.
% Example (linux):
% \begin{quote}
%   |chmod +x scripts/oberdiek/pdfatfi.pl|\\
%   |cp scripts/oberdiek/pdfatfi.pl /usr/local/bin/|
% \end{quote}
%
% \subsection{Package installation}
%
% \paragraph{Unpacking.} The \xfile{.dtx} file is a self-extracting
% \docstrip\ archive. The files are extracted by running the
% \xfile{.dtx} through \plainTeX:
% \begin{quote}
%   \verb|tex picture.dtx|
% \end{quote}
%
% \paragraph{TDS.} Now the different files must be moved into
% the different directories in your installation TDS tree
% (also known as \xfile{texmf} tree):
% \begin{quote}
% \def\t{^^A
% \begin{tabular}{@{}>{\ttfamily}l@{ $\rightarrow$ }>{\ttfamily}l@{}}
%   picture.sty & tex/latex/oberdiek/picture.sty\\
%   picture.pdf & doc/latex/oberdiek/picture.pdf\\
%   picture-example.tex & doc/latex/oberdiek/picture-example.tex\\
%   picture.dtx & source/latex/oberdiek/picture.dtx\\
% \end{tabular}^^A
% }^^A
% \sbox0{\t}^^A
% \ifdim\wd0>\linewidth
%   \begingroup
%     \advance\linewidth by\leftmargin
%     \advance\linewidth by\rightmargin
%   \edef\x{\endgroup
%     \def\noexpand\lw{\the\linewidth}^^A
%   }\x
%   \def\lwbox{^^A
%     \leavevmode
%     \hbox to \linewidth{^^A
%       \kern-\leftmargin\relax
%       \hss
%       \usebox0
%       \hss
%       \kern-\rightmargin\relax
%     }^^A
%   }^^A
%   \ifdim\wd0>\lw
%     \sbox0{\small\t}^^A
%     \ifdim\wd0>\linewidth
%       \ifdim\wd0>\lw
%         \sbox0{\footnotesize\t}^^A
%         \ifdim\wd0>\linewidth
%           \ifdim\wd0>\lw
%             \sbox0{\scriptsize\t}^^A
%             \ifdim\wd0>\linewidth
%               \ifdim\wd0>\lw
%                 \sbox0{\tiny\t}^^A
%                 \ifdim\wd0>\linewidth
%                   \lwbox
%                 \else
%                   \usebox0
%                 \fi
%               \else
%                 \lwbox
%               \fi
%             \else
%               \usebox0
%             \fi
%           \else
%             \lwbox
%           \fi
%         \else
%           \usebox0
%         \fi
%       \else
%         \lwbox
%       \fi
%     \else
%       \usebox0
%     \fi
%   \else
%     \lwbox
%   \fi
% \else
%   \usebox0
% \fi
% \end{quote}
% If you have a \xfile{docstrip.cfg} that configures and enables \docstrip's
% TDS installing feature, then some files can already be in the right
% place, see the documentation of \docstrip.
%
% \subsection{Refresh file name databases}
%
% If your \TeX~distribution
% (\teTeX, \mikTeX, \dots) relies on file name databases, you must refresh
% these. For example, \teTeX\ users run \verb|texhash| or
% \verb|mktexlsr|.
%
% \subsection{Some details for the interested}
%
% \paragraph{Attached source.}
%
% The PDF documentation on CTAN also includes the
% \xfile{.dtx} source file. It can be extracted by
% AcrobatReader 6 or higher. Another option is \textsf{pdftk},
% e.g. unpack the file into the current directory:
% \begin{quote}
%   \verb|pdftk picture.pdf unpack_files output .|
% \end{quote}
%
% \paragraph{Unpacking with \LaTeX.}
% The \xfile{.dtx} chooses its action depending on the format:
% \begin{description}
% \item[\plainTeX:] Run \docstrip\ and extract the files.
% \item[\LaTeX:] Generate the documentation.
% \end{description}
% If you insist on using \LaTeX\ for \docstrip\ (really,
% \docstrip\ does not need \LaTeX), then inform the autodetect routine
% about your intention:
% \begin{quote}
%   \verb|latex \let\install=y\input{picture.dtx}|
% \end{quote}
% Do not forget to quote the argument according to the demands
% of your shell.
%
% \paragraph{Generating the documentation.}
% You can use both the \xfile{.dtx} or the \xfile{.drv} to generate
% the documentation. The process can be configured by the
% configuration file \xfile{ltxdoc.cfg}. For instance, put this
% line into this file, if you want to have A4 as paper format:
% \begin{quote}
%   \verb|\PassOptionsToClass{a4paper}{article}|
% \end{quote}
% An example follows how to generate the
% documentation with pdf\LaTeX:
% \begin{quote}
%\begin{verbatim}
%pdflatex picture.dtx
%makeindex -s gind.ist picture.idx
%pdflatex picture.dtx
%makeindex -s gind.ist picture.idx
%pdflatex picture.dtx
%\end{verbatim}
% \end{quote}
%
% \section{Catalogue}
%
% The following XML file can be used as source for the
% \href{http://mirror.ctan.org/help/Catalogue/catalogue.html}{\TeX\ Catalogue}.
% The elements \texttt{caption} and \texttt{description} are imported
% from the original XML file from the Catalogue.
% The name of the XML file in the Catalogue is \xfile{picture.xml}.
%    \begin{macrocode}
%<*catalogue>
<?xml version='1.0' encoding='us-ascii'?>
<!DOCTYPE entry SYSTEM 'catalogue.dtd'>
<entry datestamp='$Date$' modifier='$Author$' id='picture'>
  <name>picture</name>
  <caption>Dimens for picture macros.</caption>
  <authorref id='auth:oberdiek'/>
  <copyright owner='Heiko Oberdiek' year='2006-2009'/>
  <license type='lppl1.3'/>
  <version number='1.3'/>
  <description>
    There are macro and environment arguments that expect numbers
    that will internally be multiplied by <tt>\unitlength</tt>.
    This package extends the syntax of these arguments, so that
    dimensions with calculation support may be used for these arguments.
    <p/>
    The package is part of the <xref refid='oberdiek'>oberdiek</xref> bundle.
  </description>
  <documentation details='Package documentation'
      href='ctan:/macros/latex/contrib/oberdiek/picture.pdf'/>
  <ctan file='true' path='/macros/latex/contrib/oberdiek/picture.dtx'/>
  <miktex location='oberdiek'/>
  <texlive location='oberdiek'/>
  <install path='/macros/latex/contrib/oberdiek/oberdiek.tds.zip'/>
</entry>
%</catalogue>
%    \end{macrocode}
%
% \begin{History}
%   \begin{Version}{2006/08/26 v1.0}
%   \item
%     First released version. (First start of the project was June/July 2002.)
%   \end{Version}
%   \begin{Version}{2007/04/11 v1.1}
%   \item
%     Line ends sanitized.
%   \end{Version}
%   \begin{Version}{2008/11/26 v1.2}
%   \item
%     Package \xpackage{pict2e} added to documentation section ``Supported
%     packages''.
%   \item
%     Package order of supported packages is checked.
%   \end{Version}
%   \begin{Version}{2009/10/11 v1.3}
%   \item
%     Fix because of new version v4.3 of package \xpackage{calc}.
%   \end{Version}
% \end{History}
%
% \PrintIndex
%
% \Finale
\endinput

%        (quote the arguments according to the demands of your shell)
%
% Documentation:
%    (a) If picture.drv is present:
%           latex picture.drv
%    (b) Without picture.drv:
%           latex picture.dtx; ...
%    The class ltxdoc loads the configuration file ltxdoc.cfg
%    if available. Here you can specify further options, e.g.
%    use A4 as paper format:
%       \PassOptionsToClass{a4paper}{article}
%
%    Programm calls to get the documentation (example):
%       pdflatex picture.dtx
%       makeindex -s gind.ist picture.idx
%       pdflatex picture.dtx
%       makeindex -s gind.ist picture.idx
%       pdflatex picture.dtx
%
% Installation:
%    TDS:tex/latex/oberdiek/picture.sty
%    TDS:doc/latex/oberdiek/picture.pdf
%    TDS:doc/latex/oberdiek/picture-example.tex
%    TDS:source/latex/oberdiek/picture.dtx
%
%<*ignore>
\begingroup
  \catcode123=1 %
  \catcode125=2 %
  \def\x{LaTeX2e}%
\expandafter\endgroup
\ifcase 0\ifx\install y1\fi\expandafter
         \ifx\csname processbatchFile\endcsname\relax\else1\fi
         \ifx\fmtname\x\else 1\fi\relax
\else\csname fi\endcsname
%</ignore>
%<*install>
\input docstrip.tex
\Msg{************************************************************************}
\Msg{* Installation}
\Msg{* Package: picture 2009/10/11 v1.3 Dimens for picture macros (HO)}
\Msg{************************************************************************}

\keepsilent
\askforoverwritefalse

\let\MetaPrefix\relax
\preamble

This is a generated file.

Project: picture
Version: 2009/10/11 v1.3

Copyright (C) 2006-2009 by
   Heiko Oberdiek <heiko.oberdiek at googlemail.com>

This work may be distributed and/or modified under the
conditions of the LaTeX Project Public License, either
version 1.3c of this license or (at your option) any later
version. This version of this license is in
   http://www.latex-project.org/lppl/lppl-1-3c.txt
and the latest version of this license is in
   http://www.latex-project.org/lppl.txt
and version 1.3 or later is part of all distributions of
LaTeX version 2005/12/01 or later.

This work has the LPPL maintenance status "maintained".

This Current Maintainer of this work is Heiko Oberdiek.

This work consists of the main source file picture.dtx
and the derived files
   picture.sty, picture.pdf, picture.ins, picture.drv,
   picture-example.tex.

\endpreamble
\let\MetaPrefix\DoubleperCent

\generate{%
  \file{picture.ins}{\from{picture.dtx}{install}}%
  \file{picture.drv}{\from{picture.dtx}{driver}}%
  \usedir{tex/latex/oberdiek}%
  \file{picture.sty}{\from{picture.dtx}{package}}%
  \usedir{doc/latex/oberdiek}%
  \file{picture-example.tex}{\from{picture.dtx}{example}}%
  \nopreamble
  \nopostamble
  \usedir{source/latex/oberdiek/catalogue}%
  \file{picture.xml}{\from{picture.dtx}{catalogue}}%
}

\catcode32=13\relax% active space
\let =\space%
\Msg{************************************************************************}
\Msg{*}
\Msg{* To finish the installation you have to move the following}
\Msg{* file into a directory searched by TeX:}
\Msg{*}
\Msg{*     picture.sty}
\Msg{*}
\Msg{* To produce the documentation run the file `picture.drv'}
\Msg{* through LaTeX.}
\Msg{*}
\Msg{* Happy TeXing!}
\Msg{*}
\Msg{************************************************************************}

\endbatchfile
%</install>
%<*ignore>
\fi
%</ignore>
%<*driver>
\NeedsTeXFormat{LaTeX2e}
\ProvidesFile{picture.drv}%
  [2009/10/11 v1.3 Dimens for picture macros (HO)]%
\documentclass{ltxdoc}
\usepackage{holtxdoc}[2011/11/22]
\begin{document}
  \DocInput{picture.dtx}%
\end{document}
%</driver>
% \fi
%
% \CheckSum{322}
%
% \CharacterTable
%  {Upper-case    \A\B\C\D\E\F\G\H\I\J\K\L\M\N\O\P\Q\R\S\T\U\V\W\X\Y\Z
%   Lower-case    \a\b\c\d\e\f\g\h\i\j\k\l\m\n\o\p\q\r\s\t\u\v\w\x\y\z
%   Digits        \0\1\2\3\4\5\6\7\8\9
%   Exclamation   \!     Double quote  \"     Hash (number) \#
%   Dollar        \$     Percent       \%     Ampersand     \&
%   Acute accent  \'     Left paren    \(     Right paren   \)
%   Asterisk      \*     Plus          \+     Comma         \,
%   Minus         \-     Point         \.     Solidus       \/
%   Colon         \:     Semicolon     \;     Less than     \<
%   Equals        \=     Greater than  \>     Question mark \?
%   Commercial at \@     Left bracket  \[     Backslash     \\
%   Right bracket \]     Circumflex    \^     Underscore    \_
%   Grave accent  \`     Left brace    \{     Vertical bar  \|
%   Right brace   \}     Tilde         \~}
%
% \GetFileInfo{picture.drv}
%
% \title{The \xpackage{picture} package}
% \date{2009/10/11 v1.3}
% \author{Heiko Oberdiek\\\xemail{heiko.oberdiek at googlemail.com}}
%
% \maketitle
%
% \begin{abstract}
% There are macro and environment arguments that expect numbers
% that will internally be multiplicated with \cs{unitlength}.
% This package extends the syntax of these arguments that
% dimens with calculation support can be added for these arguments.
% \end{abstract}
%
% \tableofcontents
%
% \section{User interface}
%
% \subsection{Introduction}
%
% The environment |picture| and macros such as \cs{put}, \cs{line},
% \cs{vector} and other macros have arguments that expect numbers
% that are used as factor for \cs{unitlength}.
% This package redefines such macros and adds code that detects
% whether such an argument is given as number or as length.
% In the latter case, the length is used directly without multiplying
% with \cs{unitlength}.
%
% \subsection{Options}
%
% Depending on the available features, also length expressions
% can be given. Option \xoption{calc} loads package \xpackage{calc}.
% Then expressions of these package may be used.
% Otherwise \xoption{etex} wraps the length argument inside
% \cs{dimexpr}\dots\cs{relax}, if \eTeX\ is available. Otherwise
% option \xoption{plain} uses plain assignments without calculation
% support.
%
% The default is \xoption{calc} if package \xpackage{calc}
% is loaded before package \xoption{picture}. If you specify
% option \xoption{calc} the loading of \xpackage{calc} is ensured.
% Otherwise package \xpackage{picture} looks whether \cs{dimexpr}
% is available and uses then option \xoption{etex} as default.
% If \eTeX\ also could not be found, then \xoption{plain} is used.
%
% \subsection{Example}
%
%    \begin{macrocode}
%<*example>
\documentclass{article}

\usepackage[calc]{picture}

\begin{document}

\setlength{\unitlength}{1pt}

\begin{picture}(\widthof{Hello World}, 10mm)
  \put(0, 0){\makebox(0,0)[lb]{Hello World}}%
  \put(0, \heightof{Hello World} + \fboxsep){%
    \line(1, 0){\widthof{Hello World}}%
  }%
  \put(\widthof{Hello World}, 10mm){%
    \line(0, -1){10mm}%
  }%
\end{picture}

\end{document}
%</example>
%    \end{macrocode}
%
% \subsection{Supported packages}
%
% Packages \xpackage{pspicture} and \xpackage{pict2e} are supported,
% but they must be loaded before package \xpackage{picture}.
%
% New macros can be supported by \cs{picture@redefine}.
% The first argument is the macro which contains the arguments
% in its parameter text that you want to support by package \xpackage{picture}.
% The second argument contains the parameter text. Change \# to \& for
% the arguments in question.
% Examples (already used by package \xpackage{picture}):
% \begin{quote}
% |\picture@redefine\put{(&1,&2)}|\\
% |\picture@redefine\line{(#1,#2)&3}|
% \end{quote}
%
% \StopEventually{
% }
%
% \section{Implementation}
%
% \subsection{Identification}
%
%    \begin{macrocode}
%<*package>
\NeedsTeXFormat{LaTeX2e}
\ProvidesPackage{picture}%
  [2009/10/11 v1.3 Dimens for picture macros (HO)]%
%    \end{macrocode}
%
% \subsection{Options}
%
%    \begin{macrocode}
\def\Pc@calcname{calc}
\def\Pc@etexname{etex}
\def\Pc@plainname{plain}
%    \end{macrocode}
%    \begin{macro}{\Pc@method}
%    Macro \cs{Pc@method} stores the method to use for calculations.
%    Check which features are available and set the default for \cs{Pc@method}.
%    \begin{macrocode}
\@ifpackageloaded{calc}{%
  \let\Pc@method\Pc@calcname
}{%
  \begingroup\expandafter\expandafter\expandafter\endgroup
  \expandafter\ifx\csname dimexpr\endcsname\relax
    \let\Pc@method\Pc@plainname
  \else
    \let\Pc@method\Pc@etexname
  \fi
}
%    \end{macrocode}
%    \end{macro}
%    \begin{macrocode}
\DeclareOption{plain}{%
  \let\Pc@method\Pc@plainname
}
\DeclareOption{etex}{%
  \begingroup\expandafter\expandafter\expandafter\endgroup
  \expandafter\ifx\csname dimexpr\endcsname\relax
    \PackageError{picture}{%
      e-TeX is not available%
    }\@ehc
  \else
    \let\Pc@method\Pc@etexname
  \fi
}
\DeclareOption{calc}{%
  \let\Pc@method\Pc@calcname
}
\ProcessOptions*
\begingroup
  \let\on@line\@empty
  \PackageInfo{picture}{Calculation method: \Pc@method}%
\endgroup
%    \end{macrocode}
%
% \subsection{Calculation method}
%
%    \begin{macrocode}
\ifx\Pc@method\Pc@calcname
  \RequirePackage{calc}%
\fi
%    \end{macrocode}
%
% \subsubsection{Method \xoption{calc}}
%
%    \begin{macrocode}
\ifx\Pc@method\Pc@calcname
  \def\Pc@tokslength#1{%
    \begingroup
      \let\calc@error\Pc@calc@error
      \setlength\dimen@{#1\unitlength}\Pc@next\Pc@nil{#1}%
  }%
  \let\PcOrg@calc@error\calc@error
  \@ifpackagelater{calc}{2007/08/22}{% v4.3
    \def\Pc@calc@error#1{%
      \expandafter\ifx\expandafter\unitlength\noexpand#1\relax
        \def\calc@next##1!{%
          \endgroup
          \aftergroup\afterassignment
          \aftergroup\Pc@next
        }%
        \expandafter\@firstoftwo
      \else
        \expandafter\@secondoftwo
      \fi
      {%
        \calc@next{#1}%
      }{%
        \PcOrg@calc@error{#1}%
      }%
    }%
  }{%
    \def\Pc@calc@error#1{%
      \expandafter\ifx\expandafter\unitlength\noexpand#1\relax
        \def\calc@next##1!{%
          \endgroup
          \aftergroup\afterassignment
          \aftergroup\Pc@next
        }%
        \expandafter\@gobble
      \else
        \expandafter\@firstofone
      \fi
      {%
        \PcOrg@calc@error{#1}%
      }%
    }%
  }%
\fi
%    \end{macrocode}
%
% \subsubsection{Method \xoption{etex}}
%
%    \begin{macrocode}
\ifx\Pc@method\Pc@etexname
  \def\Pc@tokslength#1{%
    \begingroup
      \afterassignment\Pc@next
      \dimen@=\dimexpr#1\unitlength\Pc@nil{#1}%
  }%
\fi
%    \end{macrocode}
%
% \subsubsection{Method \xoption{plain}}
%
%    \begin{macrocode}
\ifx\Pc@method\Pc@plainname
  \def\Pc@tokslength#1{%
    \begingroup
      \afterassignment\Pc@next
      \dimen@=#1\unitlength\Pc@nil{#1}%
  }%
\fi
%    \end{macrocode}
%
% \subsubsection{Help macros}
%
%    \begin{macrocode}
\def\Pc@next#1\Pc@nil#2{%
  \ifx\\#1\\%
    \endgroup
    \Pc@addtoks{{#2}}%
  \else
    \expandafter\endgroup
    \expandafter\Pc@addtoks\expandafter{%
      \expandafter{\the\dimen@\@gobble}%
    }%
  \fi
}
%    \end{macrocode}
%    \begin{macro}{\Pc@nil}
%    \cs{Pc@nil} must not have the meaning of \cs{relax} because of
%    \cs{dimexpr}.
%    \begin{macrocode}
\let\Pc@nil\message
%    \end{macrocode}
%    \end{macro}
%    \begin{macro}{\Pc@addtoks}
%    \begin{macrocode}
\def\Pc@addtoks#1{%
  \toks@=\expandafter{\the\toks@#1}%
}
%    \end{macrocode}
%    \end{macro}
%    \begin{macro}{\Pc@init}
%    \begin{macrocode}
\def\Pc@init#1{%
  \begingroup
    \toks@={#1}%
}
%    \end{macrocode}
%    \end{macro}
%    \begin{macro}{\Pc@finish}
%    \begin{macrocode}
\def\Pc@finish#1{%
  \expandafter\endgroup
  \expandafter#1\the\toks@
}
%    \end{macrocode}
%    \end{macro}
%
% \subsection{Redefinitions}
%
%    \begin{macro}{\picture@redefine}
%    \begin{tabular}[t]{@{}l@{ }l@{}}
%    |#1|:& command name\\
%    |#2|:& parameter text, length parameter with \& instead of \#
%    \end{tabular}
%    \begin{macrocode}
\def\picture@redefine#1#2{%
  \begingroup
    \edef\reserved@a{%
      \noexpand\noexpand
      \expandafter\noexpand
          \csname PcOrg@\expandafter\@gobble\string#1\endcsname
    }%
    \toks0{#1}%
    \Pc@first#2&0%
}
%    \end{macrocode}
%    \end{macro}
%    \begin{macro}{\Pc@first}
%    \begin{macrocode}
\def\Pc@first#1&{%
  \toks1={#1}%
  \toks2={\Pc@init{#1}}%
  \Pc@scanlength
}
%    \end{macrocode}
%    \end{macro}
%    \begin{macro}{\Pc@scanlength}
%    |#1|: number of length parameter or zero
%    \begin{macrocode}
\def\Pc@scanlength#1{%
  \ifcase#1 %
    \expandafter\Pc@last
  \else
    \toks1=\expandafter{\the\toks1 ###1}%
    \toks2=\expandafter{\the\toks2 \Pc@tokslength{###1}}%
    \expandafter\Pc@scannext
  \fi
}
%    \end{macrocode}
%    \end{macro}
%    \begin{macro}{\Pc@scannext}
%    \begin{macrocode}
\def\Pc@scannext#1&{%
  \ifx\\#1\\%
  \else
    \toks1=\expandafter{\the\toks1 #1}%
    \toks2=\expandafter{\the\toks2 \Pc@addtoks{#1}}%
  \fi
  \Pc@scanlength
}
%    \end{macrocode}
%    \end{macro}
%    \begin{macro}{\Pc@last}
%    \begin{macrocode}
\def\Pc@last{%
  \edef\x{%
    \endgroup
    \let\reserved@a\the\toks0 %
    \def\the\toks0 \the\toks1 {%
      \the\toks2 %
      \noexpand\Pc@finish\reserved@a
    }%
  }%
  \x
}
%    \end{macrocode}
%    \end{macro}
%
% \subsubsection{\LaTeX\ base macros}
%
%    \begin{macrocode}
\picture@redefine\@picture{(&1,&2)(&3,&4)}
\picture@redefine\put{(&1,&2)}
\picture@redefine\multiput{(&1,&2)}
\picture@redefine\@multiput{(&1,&2)}
\picture@redefine\line{(#1,#2)&3}
\picture@redefine\vector{(#1,#2)&3}
\picture@redefine\dashbox{&1(&2,&3)}
\picture@redefine\@circle{&1}
\picture@redefine\@dot{&1}
\picture@redefine\@bezier{#1(&2,&3)(&4,&5)(&6,&7)}
\picture@redefine\@imakepicbox{(&1,&2)}
%    \end{macrocode}
%
% \subsubsection{Package \xpackage{pspicture}}
%
%     Package \xpackage{pspicture} changes the signature of \cs{@oval}
%     by adding an optional argument.
%    \begin{macrocode}
\@ifpackageloaded{pspicture}{%
  \picture@redefine\@oval{[&1](&2,&3)}%
  \picture@redefine\Line{(&1,&2)}%
  \picture@redefine\Curve{(&1,&2)}%
  \picture@redefine\Vector{(&1,&2)}%
}{%
  \picture@redefine\@oval{(&1,&2)}%
}
%    \end{macrocode}
%
% \subsection{Check package loading order}
%
%    \begin{macro}{\PC@checkpackage}
%    \begin{macrocode}
\def\Pc@checkpackage#1{%
  \@ifpackageloaded{#1}{%
  }{%
    \AtBeginDocument{%
      \@ifpackageloaded{#1}{%
        \PackageWarningNoLine{picture}{%
          Package `#1' is loaded after `picture'.\MessageBreak
          Load package `picture' afterwards to get full support%
          \MessageBreak
          of its additional syntax with length specifications%
        }%
      }{}%
    }%
  }%
}
%    \end{macrocode}
%    \end{macro}
%    \begin{macrocode}
\Pc@checkpackage{pict2e}
\Pc@checkpackage{pspicture}
%    \end{macrocode}
%
%    \begin{macrocode}
%</package>
%    \end{macrocode}
%
% \section{Installation}
%
% \subsection{Download}
%
% \paragraph{Package.} This package is available on
% CTAN\footnote{\url{ftp://ftp.ctan.org/tex-archive/}}:
% \begin{description}
% \item[\CTAN{macros/latex/contrib/oberdiek/picture.dtx}] The source file.
% \item[\CTAN{macros/latex/contrib/oberdiek/picture.pdf}] Documentation.
% \end{description}
%
%
% \paragraph{Bundle.} All the packages of the bundle `oberdiek'
% are also available in a TDS compliant ZIP archive. There
% the packages are already unpacked and the documentation files
% are generated. The files and directories obey the TDS standard.
% \begin{description}
% \item[\CTAN{install/macros/latex/contrib/oberdiek.tds.zip}]
% \end{description}
% \emph{TDS} refers to the standard ``A Directory Structure
% for \TeX\ Files'' (\CTAN{tds/tds.pdf}). Directories
% with \xfile{texmf} in their name are usually organized this way.
%
% \subsection{Bundle installation}
%
% \paragraph{Unpacking.} Unpack the \xfile{oberdiek.tds.zip} in the
% TDS tree (also known as \xfile{texmf} tree) of your choice.
% Example (linux):
% \begin{quote}
%   |unzip oberdiek.tds.zip -d ~/texmf|
% \end{quote}
%
% \paragraph{Script installation.}
% Check the directory \xfile{TDS:scripts/oberdiek/} for
% scripts that need further installation steps.
% Package \xpackage{attachfile2} comes with the Perl script
% \xfile{pdfatfi.pl} that should be installed in such a way
% that it can be called as \texttt{pdfatfi}.
% Example (linux):
% \begin{quote}
%   |chmod +x scripts/oberdiek/pdfatfi.pl|\\
%   |cp scripts/oberdiek/pdfatfi.pl /usr/local/bin/|
% \end{quote}
%
% \subsection{Package installation}
%
% \paragraph{Unpacking.} The \xfile{.dtx} file is a self-extracting
% \docstrip\ archive. The files are extracted by running the
% \xfile{.dtx} through \plainTeX:
% \begin{quote}
%   \verb|tex picture.dtx|
% \end{quote}
%
% \paragraph{TDS.} Now the different files must be moved into
% the different directories in your installation TDS tree
% (also known as \xfile{texmf} tree):
% \begin{quote}
% \def\t{^^A
% \begin{tabular}{@{}>{\ttfamily}l@{ $\rightarrow$ }>{\ttfamily}l@{}}
%   picture.sty & tex/latex/oberdiek/picture.sty\\
%   picture.pdf & doc/latex/oberdiek/picture.pdf\\
%   picture-example.tex & doc/latex/oberdiek/picture-example.tex\\
%   picture.dtx & source/latex/oberdiek/picture.dtx\\
% \end{tabular}^^A
% }^^A
% \sbox0{\t}^^A
% \ifdim\wd0>\linewidth
%   \begingroup
%     \advance\linewidth by\leftmargin
%     \advance\linewidth by\rightmargin
%   \edef\x{\endgroup
%     \def\noexpand\lw{\the\linewidth}^^A
%   }\x
%   \def\lwbox{^^A
%     \leavevmode
%     \hbox to \linewidth{^^A
%       \kern-\leftmargin\relax
%       \hss
%       \usebox0
%       \hss
%       \kern-\rightmargin\relax
%     }^^A
%   }^^A
%   \ifdim\wd0>\lw
%     \sbox0{\small\t}^^A
%     \ifdim\wd0>\linewidth
%       \ifdim\wd0>\lw
%         \sbox0{\footnotesize\t}^^A
%         \ifdim\wd0>\linewidth
%           \ifdim\wd0>\lw
%             \sbox0{\scriptsize\t}^^A
%             \ifdim\wd0>\linewidth
%               \ifdim\wd0>\lw
%                 \sbox0{\tiny\t}^^A
%                 \ifdim\wd0>\linewidth
%                   \lwbox
%                 \else
%                   \usebox0
%                 \fi
%               \else
%                 \lwbox
%               \fi
%             \else
%               \usebox0
%             \fi
%           \else
%             \lwbox
%           \fi
%         \else
%           \usebox0
%         \fi
%       \else
%         \lwbox
%       \fi
%     \else
%       \usebox0
%     \fi
%   \else
%     \lwbox
%   \fi
% \else
%   \usebox0
% \fi
% \end{quote}
% If you have a \xfile{docstrip.cfg} that configures and enables \docstrip's
% TDS installing feature, then some files can already be in the right
% place, see the documentation of \docstrip.
%
% \subsection{Refresh file name databases}
%
% If your \TeX~distribution
% (\teTeX, \mikTeX, \dots) relies on file name databases, you must refresh
% these. For example, \teTeX\ users run \verb|texhash| or
% \verb|mktexlsr|.
%
% \subsection{Some details for the interested}
%
% \paragraph{Attached source.}
%
% The PDF documentation on CTAN also includes the
% \xfile{.dtx} source file. It can be extracted by
% AcrobatReader 6 or higher. Another option is \textsf{pdftk},
% e.g. unpack the file into the current directory:
% \begin{quote}
%   \verb|pdftk picture.pdf unpack_files output .|
% \end{quote}
%
% \paragraph{Unpacking with \LaTeX.}
% The \xfile{.dtx} chooses its action depending on the format:
% \begin{description}
% \item[\plainTeX:] Run \docstrip\ and extract the files.
% \item[\LaTeX:] Generate the documentation.
% \end{description}
% If you insist on using \LaTeX\ for \docstrip\ (really,
% \docstrip\ does not need \LaTeX), then inform the autodetect routine
% about your intention:
% \begin{quote}
%   \verb|latex \let\install=y% \iffalse meta-comment
%
% File: picture.dtx
% Version: 2009/10/11 v1.3
% Info: Dimens for picture macros
%
% Copyright (C) 2006-2009 by
%    Heiko Oberdiek <heiko.oberdiek at googlemail.com>
%
% This work may be distributed and/or modified under the
% conditions of the LaTeX Project Public License, either
% version 1.3c of this license or (at your option) any later
% version. This version of this license is in
%    http://www.latex-project.org/lppl/lppl-1-3c.txt
% and the latest version of this license is in
%    http://www.latex-project.org/lppl.txt
% and version 1.3 or later is part of all distributions of
% LaTeX version 2005/12/01 or later.
%
% This work has the LPPL maintenance status "maintained".
%
% This Current Maintainer of this work is Heiko Oberdiek.
%
% This work consists of the main source file picture.dtx
% and the derived files
%    picture.sty, picture.pdf, picture.ins, picture.drv,
%    picture-example.tex.
%
% Distribution:
%    CTAN:macros/latex/contrib/oberdiek/picture.dtx
%    CTAN:macros/latex/contrib/oberdiek/picture.pdf
%
% Unpacking:
%    (a) If picture.ins is present:
%           tex picture.ins
%    (b) Without picture.ins:
%           tex picture.dtx
%    (c) If you insist on using LaTeX
%           latex \let\install=y\input{picture.dtx}
%        (quote the arguments according to the demands of your shell)
%
% Documentation:
%    (a) If picture.drv is present:
%           latex picture.drv
%    (b) Without picture.drv:
%           latex picture.dtx; ...
%    The class ltxdoc loads the configuration file ltxdoc.cfg
%    if available. Here you can specify further options, e.g.
%    use A4 as paper format:
%       \PassOptionsToClass{a4paper}{article}
%
%    Programm calls to get the documentation (example):
%       pdflatex picture.dtx
%       makeindex -s gind.ist picture.idx
%       pdflatex picture.dtx
%       makeindex -s gind.ist picture.idx
%       pdflatex picture.dtx
%
% Installation:
%    TDS:tex/latex/oberdiek/picture.sty
%    TDS:doc/latex/oberdiek/picture.pdf
%    TDS:doc/latex/oberdiek/picture-example.tex
%    TDS:source/latex/oberdiek/picture.dtx
%
%<*ignore>
\begingroup
  \catcode123=1 %
  \catcode125=2 %
  \def\x{LaTeX2e}%
\expandafter\endgroup
\ifcase 0\ifx\install y1\fi\expandafter
         \ifx\csname processbatchFile\endcsname\relax\else1\fi
         \ifx\fmtname\x\else 1\fi\relax
\else\csname fi\endcsname
%</ignore>
%<*install>
\input docstrip.tex
\Msg{************************************************************************}
\Msg{* Installation}
\Msg{* Package: picture 2009/10/11 v1.3 Dimens for picture macros (HO)}
\Msg{************************************************************************}

\keepsilent
\askforoverwritefalse

\let\MetaPrefix\relax
\preamble

This is a generated file.

Project: picture
Version: 2009/10/11 v1.3

Copyright (C) 2006-2009 by
   Heiko Oberdiek <heiko.oberdiek at googlemail.com>

This work may be distributed and/or modified under the
conditions of the LaTeX Project Public License, either
version 1.3c of this license or (at your option) any later
version. This version of this license is in
   http://www.latex-project.org/lppl/lppl-1-3c.txt
and the latest version of this license is in
   http://www.latex-project.org/lppl.txt
and version 1.3 or later is part of all distributions of
LaTeX version 2005/12/01 or later.

This work has the LPPL maintenance status "maintained".

This Current Maintainer of this work is Heiko Oberdiek.

This work consists of the main source file picture.dtx
and the derived files
   picture.sty, picture.pdf, picture.ins, picture.drv,
   picture-example.tex.

\endpreamble
\let\MetaPrefix\DoubleperCent

\generate{%
  \file{picture.ins}{\from{picture.dtx}{install}}%
  \file{picture.drv}{\from{picture.dtx}{driver}}%
  \usedir{tex/latex/oberdiek}%
  \file{picture.sty}{\from{picture.dtx}{package}}%
  \usedir{doc/latex/oberdiek}%
  \file{picture-example.tex}{\from{picture.dtx}{example}}%
  \nopreamble
  \nopostamble
  \usedir{source/latex/oberdiek/catalogue}%
  \file{picture.xml}{\from{picture.dtx}{catalogue}}%
}

\catcode32=13\relax% active space
\let =\space%
\Msg{************************************************************************}
\Msg{*}
\Msg{* To finish the installation you have to move the following}
\Msg{* file into a directory searched by TeX:}
\Msg{*}
\Msg{*     picture.sty}
\Msg{*}
\Msg{* To produce the documentation run the file `picture.drv'}
\Msg{* through LaTeX.}
\Msg{*}
\Msg{* Happy TeXing!}
\Msg{*}
\Msg{************************************************************************}

\endbatchfile
%</install>
%<*ignore>
\fi
%</ignore>
%<*driver>
\NeedsTeXFormat{LaTeX2e}
\ProvidesFile{picture.drv}%
  [2009/10/11 v1.3 Dimens for picture macros (HO)]%
\documentclass{ltxdoc}
\usepackage{holtxdoc}[2011/11/22]
\begin{document}
  \DocInput{picture.dtx}%
\end{document}
%</driver>
% \fi
%
% \CheckSum{322}
%
% \CharacterTable
%  {Upper-case    \A\B\C\D\E\F\G\H\I\J\K\L\M\N\O\P\Q\R\S\T\U\V\W\X\Y\Z
%   Lower-case    \a\b\c\d\e\f\g\h\i\j\k\l\m\n\o\p\q\r\s\t\u\v\w\x\y\z
%   Digits        \0\1\2\3\4\5\6\7\8\9
%   Exclamation   \!     Double quote  \"     Hash (number) \#
%   Dollar        \$     Percent       \%     Ampersand     \&
%   Acute accent  \'     Left paren    \(     Right paren   \)
%   Asterisk      \*     Plus          \+     Comma         \,
%   Minus         \-     Point         \.     Solidus       \/
%   Colon         \:     Semicolon     \;     Less than     \<
%   Equals        \=     Greater than  \>     Question mark \?
%   Commercial at \@     Left bracket  \[     Backslash     \\
%   Right bracket \]     Circumflex    \^     Underscore    \_
%   Grave accent  \`     Left brace    \{     Vertical bar  \|
%   Right brace   \}     Tilde         \~}
%
% \GetFileInfo{picture.drv}
%
% \title{The \xpackage{picture} package}
% \date{2009/10/11 v1.3}
% \author{Heiko Oberdiek\\\xemail{heiko.oberdiek at googlemail.com}}
%
% \maketitle
%
% \begin{abstract}
% There are macro and environment arguments that expect numbers
% that will internally be multiplicated with \cs{unitlength}.
% This package extends the syntax of these arguments that
% dimens with calculation support can be added for these arguments.
% \end{abstract}
%
% \tableofcontents
%
% \section{User interface}
%
% \subsection{Introduction}
%
% The environment |picture| and macros such as \cs{put}, \cs{line},
% \cs{vector} and other macros have arguments that expect numbers
% that are used as factor for \cs{unitlength}.
% This package redefines such macros and adds code that detects
% whether such an argument is given as number or as length.
% In the latter case, the length is used directly without multiplying
% with \cs{unitlength}.
%
% \subsection{Options}
%
% Depending on the available features, also length expressions
% can be given. Option \xoption{calc} loads package \xpackage{calc}.
% Then expressions of these package may be used.
% Otherwise \xoption{etex} wraps the length argument inside
% \cs{dimexpr}\dots\cs{relax}, if \eTeX\ is available. Otherwise
% option \xoption{plain} uses plain assignments without calculation
% support.
%
% The default is \xoption{calc} if package \xpackage{calc}
% is loaded before package \xoption{picture}. If you specify
% option \xoption{calc} the loading of \xpackage{calc} is ensured.
% Otherwise package \xpackage{picture} looks whether \cs{dimexpr}
% is available and uses then option \xoption{etex} as default.
% If \eTeX\ also could not be found, then \xoption{plain} is used.
%
% \subsection{Example}
%
%    \begin{macrocode}
%<*example>
\documentclass{article}

\usepackage[calc]{picture}

\begin{document}

\setlength{\unitlength}{1pt}

\begin{picture}(\widthof{Hello World}, 10mm)
  \put(0, 0){\makebox(0,0)[lb]{Hello World}}%
  \put(0, \heightof{Hello World} + \fboxsep){%
    \line(1, 0){\widthof{Hello World}}%
  }%
  \put(\widthof{Hello World}, 10mm){%
    \line(0, -1){10mm}%
  }%
\end{picture}

\end{document}
%</example>
%    \end{macrocode}
%
% \subsection{Supported packages}
%
% Packages \xpackage{pspicture} and \xpackage{pict2e} are supported,
% but they must be loaded before package \xpackage{picture}.
%
% New macros can be supported by \cs{picture@redefine}.
% The first argument is the macro which contains the arguments
% in its parameter text that you want to support by package \xpackage{picture}.
% The second argument contains the parameter text. Change \# to \& for
% the arguments in question.
% Examples (already used by package \xpackage{picture}):
% \begin{quote}
% |\picture@redefine\put{(&1,&2)}|\\
% |\picture@redefine\line{(#1,#2)&3}|
% \end{quote}
%
% \StopEventually{
% }
%
% \section{Implementation}
%
% \subsection{Identification}
%
%    \begin{macrocode}
%<*package>
\NeedsTeXFormat{LaTeX2e}
\ProvidesPackage{picture}%
  [2009/10/11 v1.3 Dimens for picture macros (HO)]%
%    \end{macrocode}
%
% \subsection{Options}
%
%    \begin{macrocode}
\def\Pc@calcname{calc}
\def\Pc@etexname{etex}
\def\Pc@plainname{plain}
%    \end{macrocode}
%    \begin{macro}{\Pc@method}
%    Macro \cs{Pc@method} stores the method to use for calculations.
%    Check which features are available and set the default for \cs{Pc@method}.
%    \begin{macrocode}
\@ifpackageloaded{calc}{%
  \let\Pc@method\Pc@calcname
}{%
  \begingroup\expandafter\expandafter\expandafter\endgroup
  \expandafter\ifx\csname dimexpr\endcsname\relax
    \let\Pc@method\Pc@plainname
  \else
    \let\Pc@method\Pc@etexname
  \fi
}
%    \end{macrocode}
%    \end{macro}
%    \begin{macrocode}
\DeclareOption{plain}{%
  \let\Pc@method\Pc@plainname
}
\DeclareOption{etex}{%
  \begingroup\expandafter\expandafter\expandafter\endgroup
  \expandafter\ifx\csname dimexpr\endcsname\relax
    \PackageError{picture}{%
      e-TeX is not available%
    }\@ehc
  \else
    \let\Pc@method\Pc@etexname
  \fi
}
\DeclareOption{calc}{%
  \let\Pc@method\Pc@calcname
}
\ProcessOptions*
\begingroup
  \let\on@line\@empty
  \PackageInfo{picture}{Calculation method: \Pc@method}%
\endgroup
%    \end{macrocode}
%
% \subsection{Calculation method}
%
%    \begin{macrocode}
\ifx\Pc@method\Pc@calcname
  \RequirePackage{calc}%
\fi
%    \end{macrocode}
%
% \subsubsection{Method \xoption{calc}}
%
%    \begin{macrocode}
\ifx\Pc@method\Pc@calcname
  \def\Pc@tokslength#1{%
    \begingroup
      \let\calc@error\Pc@calc@error
      \setlength\dimen@{#1\unitlength}\Pc@next\Pc@nil{#1}%
  }%
  \let\PcOrg@calc@error\calc@error
  \@ifpackagelater{calc}{2007/08/22}{% v4.3
    \def\Pc@calc@error#1{%
      \expandafter\ifx\expandafter\unitlength\noexpand#1\relax
        \def\calc@next##1!{%
          \endgroup
          \aftergroup\afterassignment
          \aftergroup\Pc@next
        }%
        \expandafter\@firstoftwo
      \else
        \expandafter\@secondoftwo
      \fi
      {%
        \calc@next{#1}%
      }{%
        \PcOrg@calc@error{#1}%
      }%
    }%
  }{%
    \def\Pc@calc@error#1{%
      \expandafter\ifx\expandafter\unitlength\noexpand#1\relax
        \def\calc@next##1!{%
          \endgroup
          \aftergroup\afterassignment
          \aftergroup\Pc@next
        }%
        \expandafter\@gobble
      \else
        \expandafter\@firstofone
      \fi
      {%
        \PcOrg@calc@error{#1}%
      }%
    }%
  }%
\fi
%    \end{macrocode}
%
% \subsubsection{Method \xoption{etex}}
%
%    \begin{macrocode}
\ifx\Pc@method\Pc@etexname
  \def\Pc@tokslength#1{%
    \begingroup
      \afterassignment\Pc@next
      \dimen@=\dimexpr#1\unitlength\Pc@nil{#1}%
  }%
\fi
%    \end{macrocode}
%
% \subsubsection{Method \xoption{plain}}
%
%    \begin{macrocode}
\ifx\Pc@method\Pc@plainname
  \def\Pc@tokslength#1{%
    \begingroup
      \afterassignment\Pc@next
      \dimen@=#1\unitlength\Pc@nil{#1}%
  }%
\fi
%    \end{macrocode}
%
% \subsubsection{Help macros}
%
%    \begin{macrocode}
\def\Pc@next#1\Pc@nil#2{%
  \ifx\\#1\\%
    \endgroup
    \Pc@addtoks{{#2}}%
  \else
    \expandafter\endgroup
    \expandafter\Pc@addtoks\expandafter{%
      \expandafter{\the\dimen@\@gobble}%
    }%
  \fi
}
%    \end{macrocode}
%    \begin{macro}{\Pc@nil}
%    \cs{Pc@nil} must not have the meaning of \cs{relax} because of
%    \cs{dimexpr}.
%    \begin{macrocode}
\let\Pc@nil\message
%    \end{macrocode}
%    \end{macro}
%    \begin{macro}{\Pc@addtoks}
%    \begin{macrocode}
\def\Pc@addtoks#1{%
  \toks@=\expandafter{\the\toks@#1}%
}
%    \end{macrocode}
%    \end{macro}
%    \begin{macro}{\Pc@init}
%    \begin{macrocode}
\def\Pc@init#1{%
  \begingroup
    \toks@={#1}%
}
%    \end{macrocode}
%    \end{macro}
%    \begin{macro}{\Pc@finish}
%    \begin{macrocode}
\def\Pc@finish#1{%
  \expandafter\endgroup
  \expandafter#1\the\toks@
}
%    \end{macrocode}
%    \end{macro}
%
% \subsection{Redefinitions}
%
%    \begin{macro}{\picture@redefine}
%    \begin{tabular}[t]{@{}l@{ }l@{}}
%    |#1|:& command name\\
%    |#2|:& parameter text, length parameter with \& instead of \#
%    \end{tabular}
%    \begin{macrocode}
\def\picture@redefine#1#2{%
  \begingroup
    \edef\reserved@a{%
      \noexpand\noexpand
      \expandafter\noexpand
          \csname PcOrg@\expandafter\@gobble\string#1\endcsname
    }%
    \toks0{#1}%
    \Pc@first#2&0%
}
%    \end{macrocode}
%    \end{macro}
%    \begin{macro}{\Pc@first}
%    \begin{macrocode}
\def\Pc@first#1&{%
  \toks1={#1}%
  \toks2={\Pc@init{#1}}%
  \Pc@scanlength
}
%    \end{macrocode}
%    \end{macro}
%    \begin{macro}{\Pc@scanlength}
%    |#1|: number of length parameter or zero
%    \begin{macrocode}
\def\Pc@scanlength#1{%
  \ifcase#1 %
    \expandafter\Pc@last
  \else
    \toks1=\expandafter{\the\toks1 ###1}%
    \toks2=\expandafter{\the\toks2 \Pc@tokslength{###1}}%
    \expandafter\Pc@scannext
  \fi
}
%    \end{macrocode}
%    \end{macro}
%    \begin{macro}{\Pc@scannext}
%    \begin{macrocode}
\def\Pc@scannext#1&{%
  \ifx\\#1\\%
  \else
    \toks1=\expandafter{\the\toks1 #1}%
    \toks2=\expandafter{\the\toks2 \Pc@addtoks{#1}}%
  \fi
  \Pc@scanlength
}
%    \end{macrocode}
%    \end{macro}
%    \begin{macro}{\Pc@last}
%    \begin{macrocode}
\def\Pc@last{%
  \edef\x{%
    \endgroup
    \let\reserved@a\the\toks0 %
    \def\the\toks0 \the\toks1 {%
      \the\toks2 %
      \noexpand\Pc@finish\reserved@a
    }%
  }%
  \x
}
%    \end{macrocode}
%    \end{macro}
%
% \subsubsection{\LaTeX\ base macros}
%
%    \begin{macrocode}
\picture@redefine\@picture{(&1,&2)(&3,&4)}
\picture@redefine\put{(&1,&2)}
\picture@redefine\multiput{(&1,&2)}
\picture@redefine\@multiput{(&1,&2)}
\picture@redefine\line{(#1,#2)&3}
\picture@redefine\vector{(#1,#2)&3}
\picture@redefine\dashbox{&1(&2,&3)}
\picture@redefine\@circle{&1}
\picture@redefine\@dot{&1}
\picture@redefine\@bezier{#1(&2,&3)(&4,&5)(&6,&7)}
\picture@redefine\@imakepicbox{(&1,&2)}
%    \end{macrocode}
%
% \subsubsection{Package \xpackage{pspicture}}
%
%     Package \xpackage{pspicture} changes the signature of \cs{@oval}
%     by adding an optional argument.
%    \begin{macrocode}
\@ifpackageloaded{pspicture}{%
  \picture@redefine\@oval{[&1](&2,&3)}%
  \picture@redefine\Line{(&1,&2)}%
  \picture@redefine\Curve{(&1,&2)}%
  \picture@redefine\Vector{(&1,&2)}%
}{%
  \picture@redefine\@oval{(&1,&2)}%
}
%    \end{macrocode}
%
% \subsection{Check package loading order}
%
%    \begin{macro}{\PC@checkpackage}
%    \begin{macrocode}
\def\Pc@checkpackage#1{%
  \@ifpackageloaded{#1}{%
  }{%
    \AtBeginDocument{%
      \@ifpackageloaded{#1}{%
        \PackageWarningNoLine{picture}{%
          Package `#1' is loaded after `picture'.\MessageBreak
          Load package `picture' afterwards to get full support%
          \MessageBreak
          of its additional syntax with length specifications%
        }%
      }{}%
    }%
  }%
}
%    \end{macrocode}
%    \end{macro}
%    \begin{macrocode}
\Pc@checkpackage{pict2e}
\Pc@checkpackage{pspicture}
%    \end{macrocode}
%
%    \begin{macrocode}
%</package>
%    \end{macrocode}
%
% \section{Installation}
%
% \subsection{Download}
%
% \paragraph{Package.} This package is available on
% CTAN\footnote{\url{ftp://ftp.ctan.org/tex-archive/}}:
% \begin{description}
% \item[\CTAN{macros/latex/contrib/oberdiek/picture.dtx}] The source file.
% \item[\CTAN{macros/latex/contrib/oberdiek/picture.pdf}] Documentation.
% \end{description}
%
%
% \paragraph{Bundle.} All the packages of the bundle `oberdiek'
% are also available in a TDS compliant ZIP archive. There
% the packages are already unpacked and the documentation files
% are generated. The files and directories obey the TDS standard.
% \begin{description}
% \item[\CTAN{install/macros/latex/contrib/oberdiek.tds.zip}]
% \end{description}
% \emph{TDS} refers to the standard ``A Directory Structure
% for \TeX\ Files'' (\CTAN{tds/tds.pdf}). Directories
% with \xfile{texmf} in their name are usually organized this way.
%
% \subsection{Bundle installation}
%
% \paragraph{Unpacking.} Unpack the \xfile{oberdiek.tds.zip} in the
% TDS tree (also known as \xfile{texmf} tree) of your choice.
% Example (linux):
% \begin{quote}
%   |unzip oberdiek.tds.zip -d ~/texmf|
% \end{quote}
%
% \paragraph{Script installation.}
% Check the directory \xfile{TDS:scripts/oberdiek/} for
% scripts that need further installation steps.
% Package \xpackage{attachfile2} comes with the Perl script
% \xfile{pdfatfi.pl} that should be installed in such a way
% that it can be called as \texttt{pdfatfi}.
% Example (linux):
% \begin{quote}
%   |chmod +x scripts/oberdiek/pdfatfi.pl|\\
%   |cp scripts/oberdiek/pdfatfi.pl /usr/local/bin/|
% \end{quote}
%
% \subsection{Package installation}
%
% \paragraph{Unpacking.} The \xfile{.dtx} file is a self-extracting
% \docstrip\ archive. The files are extracted by running the
% \xfile{.dtx} through \plainTeX:
% \begin{quote}
%   \verb|tex picture.dtx|
% \end{quote}
%
% \paragraph{TDS.} Now the different files must be moved into
% the different directories in your installation TDS tree
% (also known as \xfile{texmf} tree):
% \begin{quote}
% \def\t{^^A
% \begin{tabular}{@{}>{\ttfamily}l@{ $\rightarrow$ }>{\ttfamily}l@{}}
%   picture.sty & tex/latex/oberdiek/picture.sty\\
%   picture.pdf & doc/latex/oberdiek/picture.pdf\\
%   picture-example.tex & doc/latex/oberdiek/picture-example.tex\\
%   picture.dtx & source/latex/oberdiek/picture.dtx\\
% \end{tabular}^^A
% }^^A
% \sbox0{\t}^^A
% \ifdim\wd0>\linewidth
%   \begingroup
%     \advance\linewidth by\leftmargin
%     \advance\linewidth by\rightmargin
%   \edef\x{\endgroup
%     \def\noexpand\lw{\the\linewidth}^^A
%   }\x
%   \def\lwbox{^^A
%     \leavevmode
%     \hbox to \linewidth{^^A
%       \kern-\leftmargin\relax
%       \hss
%       \usebox0
%       \hss
%       \kern-\rightmargin\relax
%     }^^A
%   }^^A
%   \ifdim\wd0>\lw
%     \sbox0{\small\t}^^A
%     \ifdim\wd0>\linewidth
%       \ifdim\wd0>\lw
%         \sbox0{\footnotesize\t}^^A
%         \ifdim\wd0>\linewidth
%           \ifdim\wd0>\lw
%             \sbox0{\scriptsize\t}^^A
%             \ifdim\wd0>\linewidth
%               \ifdim\wd0>\lw
%                 \sbox0{\tiny\t}^^A
%                 \ifdim\wd0>\linewidth
%                   \lwbox
%                 \else
%                   \usebox0
%                 \fi
%               \else
%                 \lwbox
%               \fi
%             \else
%               \usebox0
%             \fi
%           \else
%             \lwbox
%           \fi
%         \else
%           \usebox0
%         \fi
%       \else
%         \lwbox
%       \fi
%     \else
%       \usebox0
%     \fi
%   \else
%     \lwbox
%   \fi
% \else
%   \usebox0
% \fi
% \end{quote}
% If you have a \xfile{docstrip.cfg} that configures and enables \docstrip's
% TDS installing feature, then some files can already be in the right
% place, see the documentation of \docstrip.
%
% \subsection{Refresh file name databases}
%
% If your \TeX~distribution
% (\teTeX, \mikTeX, \dots) relies on file name databases, you must refresh
% these. For example, \teTeX\ users run \verb|texhash| or
% \verb|mktexlsr|.
%
% \subsection{Some details for the interested}
%
% \paragraph{Attached source.}
%
% The PDF documentation on CTAN also includes the
% \xfile{.dtx} source file. It can be extracted by
% AcrobatReader 6 or higher. Another option is \textsf{pdftk},
% e.g. unpack the file into the current directory:
% \begin{quote}
%   \verb|pdftk picture.pdf unpack_files output .|
% \end{quote}
%
% \paragraph{Unpacking with \LaTeX.}
% The \xfile{.dtx} chooses its action depending on the format:
% \begin{description}
% \item[\plainTeX:] Run \docstrip\ and extract the files.
% \item[\LaTeX:] Generate the documentation.
% \end{description}
% If you insist on using \LaTeX\ for \docstrip\ (really,
% \docstrip\ does not need \LaTeX), then inform the autodetect routine
% about your intention:
% \begin{quote}
%   \verb|latex \let\install=y\input{picture.dtx}|
% \end{quote}
% Do not forget to quote the argument according to the demands
% of your shell.
%
% \paragraph{Generating the documentation.}
% You can use both the \xfile{.dtx} or the \xfile{.drv} to generate
% the documentation. The process can be configured by the
% configuration file \xfile{ltxdoc.cfg}. For instance, put this
% line into this file, if you want to have A4 as paper format:
% \begin{quote}
%   \verb|\PassOptionsToClass{a4paper}{article}|
% \end{quote}
% An example follows how to generate the
% documentation with pdf\LaTeX:
% \begin{quote}
%\begin{verbatim}
%pdflatex picture.dtx
%makeindex -s gind.ist picture.idx
%pdflatex picture.dtx
%makeindex -s gind.ist picture.idx
%pdflatex picture.dtx
%\end{verbatim}
% \end{quote}
%
% \section{Catalogue}
%
% The following XML file can be used as source for the
% \href{http://mirror.ctan.org/help/Catalogue/catalogue.html}{\TeX\ Catalogue}.
% The elements \texttt{caption} and \texttt{description} are imported
% from the original XML file from the Catalogue.
% The name of the XML file in the Catalogue is \xfile{picture.xml}.
%    \begin{macrocode}
%<*catalogue>
<?xml version='1.0' encoding='us-ascii'?>
<!DOCTYPE entry SYSTEM 'catalogue.dtd'>
<entry datestamp='$Date$' modifier='$Author$' id='picture'>
  <name>picture</name>
  <caption>Dimens for picture macros.</caption>
  <authorref id='auth:oberdiek'/>
  <copyright owner='Heiko Oberdiek' year='2006-2009'/>
  <license type='lppl1.3'/>
  <version number='1.3'/>
  <description>
    There are macro and environment arguments that expect numbers
    that will internally be multiplied by <tt>\unitlength</tt>.
    This package extends the syntax of these arguments, so that
    dimensions with calculation support may be used for these arguments.
    <p/>
    The package is part of the <xref refid='oberdiek'>oberdiek</xref> bundle.
  </description>
  <documentation details='Package documentation'
      href='ctan:/macros/latex/contrib/oberdiek/picture.pdf'/>
  <ctan file='true' path='/macros/latex/contrib/oberdiek/picture.dtx'/>
  <miktex location='oberdiek'/>
  <texlive location='oberdiek'/>
  <install path='/macros/latex/contrib/oberdiek/oberdiek.tds.zip'/>
</entry>
%</catalogue>
%    \end{macrocode}
%
% \begin{History}
%   \begin{Version}{2006/08/26 v1.0}
%   \item
%     First released version. (First start of the project was June/July 2002.)
%   \end{Version}
%   \begin{Version}{2007/04/11 v1.1}
%   \item
%     Line ends sanitized.
%   \end{Version}
%   \begin{Version}{2008/11/26 v1.2}
%   \item
%     Package \xpackage{pict2e} added to documentation section ``Supported
%     packages''.
%   \item
%     Package order of supported packages is checked.
%   \end{Version}
%   \begin{Version}{2009/10/11 v1.3}
%   \item
%     Fix because of new version v4.3 of package \xpackage{calc}.
%   \end{Version}
% \end{History}
%
% \PrintIndex
%
% \Finale
\endinput
|
% \end{quote}
% Do not forget to quote the argument according to the demands
% of your shell.
%
% \paragraph{Generating the documentation.}
% You can use both the \xfile{.dtx} or the \xfile{.drv} to generate
% the documentation. The process can be configured by the
% configuration file \xfile{ltxdoc.cfg}. For instance, put this
% line into this file, if you want to have A4 as paper format:
% \begin{quote}
%   \verb|\PassOptionsToClass{a4paper}{article}|
% \end{quote}
% An example follows how to generate the
% documentation with pdf\LaTeX:
% \begin{quote}
%\begin{verbatim}
%pdflatex picture.dtx
%makeindex -s gind.ist picture.idx
%pdflatex picture.dtx
%makeindex -s gind.ist picture.idx
%pdflatex picture.dtx
%\end{verbatim}
% \end{quote}
%
% \section{Catalogue}
%
% The following XML file can be used as source for the
% \href{http://mirror.ctan.org/help/Catalogue/catalogue.html}{\TeX\ Catalogue}.
% The elements \texttt{caption} and \texttt{description} are imported
% from the original XML file from the Catalogue.
% The name of the XML file in the Catalogue is \xfile{picture.xml}.
%    \begin{macrocode}
%<*catalogue>
<?xml version='1.0' encoding='us-ascii'?>
<!DOCTYPE entry SYSTEM 'catalogue.dtd'>
<entry datestamp='$Date$' modifier='$Author$' id='picture'>
  <name>picture</name>
  <caption>Dimens for picture macros.</caption>
  <authorref id='auth:oberdiek'/>
  <copyright owner='Heiko Oberdiek' year='2006-2009'/>
  <license type='lppl1.3'/>
  <version number='1.3'/>
  <description>
    There are macro and environment arguments that expect numbers
    that will internally be multiplied by <tt>\unitlength</tt>.
    This package extends the syntax of these arguments, so that
    dimensions with calculation support may be used for these arguments.
    <p/>
    The package is part of the <xref refid='oberdiek'>oberdiek</xref> bundle.
  </description>
  <documentation details='Package documentation'
      href='ctan:/macros/latex/contrib/oberdiek/picture.pdf'/>
  <ctan file='true' path='/macros/latex/contrib/oberdiek/picture.dtx'/>
  <miktex location='oberdiek'/>
  <texlive location='oberdiek'/>
  <install path='/macros/latex/contrib/oberdiek/oberdiek.tds.zip'/>
</entry>
%</catalogue>
%    \end{macrocode}
%
% \begin{History}
%   \begin{Version}{2006/08/26 v1.0}
%   \item
%     First released version. (First start of the project was June/July 2002.)
%   \end{Version}
%   \begin{Version}{2007/04/11 v1.1}
%   \item
%     Line ends sanitized.
%   \end{Version}
%   \begin{Version}{2008/11/26 v1.2}
%   \item
%     Package \xpackage{pict2e} added to documentation section ``Supported
%     packages''.
%   \item
%     Package order of supported packages is checked.
%   \end{Version}
%   \begin{Version}{2009/10/11 v1.3}
%   \item
%     Fix because of new version v4.3 of package \xpackage{calc}.
%   \end{Version}
% \end{History}
%
% \PrintIndex
%
% \Finale
\endinput

%        (quote the arguments according to the demands of your shell)
%
% Documentation:
%    (a) If picture.drv is present:
%           latex picture.drv
%    (b) Without picture.drv:
%           latex picture.dtx; ...
%    The class ltxdoc loads the configuration file ltxdoc.cfg
%    if available. Here you can specify further options, e.g.
%    use A4 as paper format:
%       \PassOptionsToClass{a4paper}{article}
%
%    Programm calls to get the documentation (example):
%       pdflatex picture.dtx
%       makeindex -s gind.ist picture.idx
%       pdflatex picture.dtx
%       makeindex -s gind.ist picture.idx
%       pdflatex picture.dtx
%
% Installation:
%    TDS:tex/latex/oberdiek/picture.sty
%    TDS:doc/latex/oberdiek/picture.pdf
%    TDS:doc/latex/oberdiek/picture-example.tex
%    TDS:source/latex/oberdiek/picture.dtx
%
%<*ignore>
\begingroup
  \catcode123=1 %
  \catcode125=2 %
  \def\x{LaTeX2e}%
\expandafter\endgroup
\ifcase 0\ifx\install y1\fi\expandafter
         \ifx\csname processbatchFile\endcsname\relax\else1\fi
         \ifx\fmtname\x\else 1\fi\relax
\else\csname fi\endcsname
%</ignore>
%<*install>
\input docstrip.tex
\Msg{************************************************************************}
\Msg{* Installation}
\Msg{* Package: picture 2009/10/11 v1.3 Dimens for picture macros (HO)}
\Msg{************************************************************************}

\keepsilent
\askforoverwritefalse

\let\MetaPrefix\relax
\preamble

This is a generated file.

Project: picture
Version: 2009/10/11 v1.3

Copyright (C) 2006-2009 by
   Heiko Oberdiek <heiko.oberdiek at googlemail.com>

This work may be distributed and/or modified under the
conditions of the LaTeX Project Public License, either
version 1.3c of this license or (at your option) any later
version. This version of this license is in
   http://www.latex-project.org/lppl/lppl-1-3c.txt
and the latest version of this license is in
   http://www.latex-project.org/lppl.txt
and version 1.3 or later is part of all distributions of
LaTeX version 2005/12/01 or later.

This work has the LPPL maintenance status "maintained".

This Current Maintainer of this work is Heiko Oberdiek.

This work consists of the main source file picture.dtx
and the derived files
   picture.sty, picture.pdf, picture.ins, picture.drv,
   picture-example.tex.

\endpreamble
\let\MetaPrefix\DoubleperCent

\generate{%
  \file{picture.ins}{\from{picture.dtx}{install}}%
  \file{picture.drv}{\from{picture.dtx}{driver}}%
  \usedir{tex/latex/oberdiek}%
  \file{picture.sty}{\from{picture.dtx}{package}}%
  \usedir{doc/latex/oberdiek}%
  \file{picture-example.tex}{\from{picture.dtx}{example}}%
  \nopreamble
  \nopostamble
  \usedir{source/latex/oberdiek/catalogue}%
  \file{picture.xml}{\from{picture.dtx}{catalogue}}%
}

\catcode32=13\relax% active space
\let =\space%
\Msg{************************************************************************}
\Msg{*}
\Msg{* To finish the installation you have to move the following}
\Msg{* file into a directory searched by TeX:}
\Msg{*}
\Msg{*     picture.sty}
\Msg{*}
\Msg{* To produce the documentation run the file `picture.drv'}
\Msg{* through LaTeX.}
\Msg{*}
\Msg{* Happy TeXing!}
\Msg{*}
\Msg{************************************************************************}

\endbatchfile
%</install>
%<*ignore>
\fi
%</ignore>
%<*driver>
\NeedsTeXFormat{LaTeX2e}
\ProvidesFile{picture.drv}%
  [2009/10/11 v1.3 Dimens for picture macros (HO)]%
\documentclass{ltxdoc}
\usepackage{holtxdoc}[2011/11/22]
\begin{document}
  \DocInput{picture.dtx}%
\end{document}
%</driver>
% \fi
%
% \CheckSum{322}
%
% \CharacterTable
%  {Upper-case    \A\B\C\D\E\F\G\H\I\J\K\L\M\N\O\P\Q\R\S\T\U\V\W\X\Y\Z
%   Lower-case    \a\b\c\d\e\f\g\h\i\j\k\l\m\n\o\p\q\r\s\t\u\v\w\x\y\z
%   Digits        \0\1\2\3\4\5\6\7\8\9
%   Exclamation   \!     Double quote  \"     Hash (number) \#
%   Dollar        \$     Percent       \%     Ampersand     \&
%   Acute accent  \'     Left paren    \(     Right paren   \)
%   Asterisk      \*     Plus          \+     Comma         \,
%   Minus         \-     Point         \.     Solidus       \/
%   Colon         \:     Semicolon     \;     Less than     \<
%   Equals        \=     Greater than  \>     Question mark \?
%   Commercial at \@     Left bracket  \[     Backslash     \\
%   Right bracket \]     Circumflex    \^     Underscore    \_
%   Grave accent  \`     Left brace    \{     Vertical bar  \|
%   Right brace   \}     Tilde         \~}
%
% \GetFileInfo{picture.drv}
%
% \title{The \xpackage{picture} package}
% \date{2009/10/11 v1.3}
% \author{Heiko Oberdiek\\\xemail{heiko.oberdiek at googlemail.com}}
%
% \maketitle
%
% \begin{abstract}
% There are macro and environment arguments that expect numbers
% that will internally be multiplicated with \cs{unitlength}.
% This package extends the syntax of these arguments that
% dimens with calculation support can be added for these arguments.
% \end{abstract}
%
% \tableofcontents
%
% \section{User interface}
%
% \subsection{Introduction}
%
% The environment |picture| and macros such as \cs{put}, \cs{line},
% \cs{vector} and other macros have arguments that expect numbers
% that are used as factor for \cs{unitlength}.
% This package redefines such macros and adds code that detects
% whether such an argument is given as number or as length.
% In the latter case, the length is used directly without multiplying
% with \cs{unitlength}.
%
% \subsection{Options}
%
% Depending on the available features, also length expressions
% can be given. Option \xoption{calc} loads package \xpackage{calc}.
% Then expressions of these package may be used.
% Otherwise \xoption{etex} wraps the length argument inside
% \cs{dimexpr}\dots\cs{relax}, if \eTeX\ is available. Otherwise
% option \xoption{plain} uses plain assignments without calculation
% support.
%
% The default is \xoption{calc} if package \xpackage{calc}
% is loaded before package \xoption{picture}. If you specify
% option \xoption{calc} the loading of \xpackage{calc} is ensured.
% Otherwise package \xpackage{picture} looks whether \cs{dimexpr}
% is available and uses then option \xoption{etex} as default.
% If \eTeX\ also could not be found, then \xoption{plain} is used.
%
% \subsection{Example}
%
%    \begin{macrocode}
%<*example>
\documentclass{article}

\usepackage[calc]{picture}

\begin{document}

\setlength{\unitlength}{1pt}

\begin{picture}(\widthof{Hello World}, 10mm)
  \put(0, 0){\makebox(0,0)[lb]{Hello World}}%
  \put(0, \heightof{Hello World} + \fboxsep){%
    \line(1, 0){\widthof{Hello World}}%
  }%
  \put(\widthof{Hello World}, 10mm){%
    \line(0, -1){10mm}%
  }%
\end{picture}

\end{document}
%</example>
%    \end{macrocode}
%
% \subsection{Supported packages}
%
% Packages \xpackage{pspicture} and \xpackage{pict2e} are supported,
% but they must be loaded before package \xpackage{picture}.
%
% New macros can be supported by \cs{picture@redefine}.
% The first argument is the macro which contains the arguments
% in its parameter text that you want to support by package \xpackage{picture}.
% The second argument contains the parameter text. Change \# to \& for
% the arguments in question.
% Examples (already used by package \xpackage{picture}):
% \begin{quote}
% |\picture@redefine\put{(&1,&2)}|\\
% |\picture@redefine\line{(#1,#2)&3}|
% \end{quote}
%
% \StopEventually{
% }
%
% \section{Implementation}
%
% \subsection{Identification}
%
%    \begin{macrocode}
%<*package>
\NeedsTeXFormat{LaTeX2e}
\ProvidesPackage{picture}%
  [2009/10/11 v1.3 Dimens for picture macros (HO)]%
%    \end{macrocode}
%
% \subsection{Options}
%
%    \begin{macrocode}
\def\Pc@calcname{calc}
\def\Pc@etexname{etex}
\def\Pc@plainname{plain}
%    \end{macrocode}
%    \begin{macro}{\Pc@method}
%    Macro \cs{Pc@method} stores the method to use for calculations.
%    Check which features are available and set the default for \cs{Pc@method}.
%    \begin{macrocode}
\@ifpackageloaded{calc}{%
  \let\Pc@method\Pc@calcname
}{%
  \begingroup\expandafter\expandafter\expandafter\endgroup
  \expandafter\ifx\csname dimexpr\endcsname\relax
    \let\Pc@method\Pc@plainname
  \else
    \let\Pc@method\Pc@etexname
  \fi
}
%    \end{macrocode}
%    \end{macro}
%    \begin{macrocode}
\DeclareOption{plain}{%
  \let\Pc@method\Pc@plainname
}
\DeclareOption{etex}{%
  \begingroup\expandafter\expandafter\expandafter\endgroup
  \expandafter\ifx\csname dimexpr\endcsname\relax
    \PackageError{picture}{%
      e-TeX is not available%
    }\@ehc
  \else
    \let\Pc@method\Pc@etexname
  \fi
}
\DeclareOption{calc}{%
  \let\Pc@method\Pc@calcname
}
\ProcessOptions*
\begingroup
  \let\on@line\@empty
  \PackageInfo{picture}{Calculation method: \Pc@method}%
\endgroup
%    \end{macrocode}
%
% \subsection{Calculation method}
%
%    \begin{macrocode}
\ifx\Pc@method\Pc@calcname
  \RequirePackage{calc}%
\fi
%    \end{macrocode}
%
% \subsubsection{Method \xoption{calc}}
%
%    \begin{macrocode}
\ifx\Pc@method\Pc@calcname
  \def\Pc@tokslength#1{%
    \begingroup
      \let\calc@error\Pc@calc@error
      \setlength\dimen@{#1\unitlength}\Pc@next\Pc@nil{#1}%
  }%
  \let\PcOrg@calc@error\calc@error
  \@ifpackagelater{calc}{2007/08/22}{% v4.3
    \def\Pc@calc@error#1{%
      \expandafter\ifx\expandafter\unitlength\noexpand#1\relax
        \def\calc@next##1!{%
          \endgroup
          \aftergroup\afterassignment
          \aftergroup\Pc@next
        }%
        \expandafter\@firstoftwo
      \else
        \expandafter\@secondoftwo
      \fi
      {%
        \calc@next{#1}%
      }{%
        \PcOrg@calc@error{#1}%
      }%
    }%
  }{%
    \def\Pc@calc@error#1{%
      \expandafter\ifx\expandafter\unitlength\noexpand#1\relax
        \def\calc@next##1!{%
          \endgroup
          \aftergroup\afterassignment
          \aftergroup\Pc@next
        }%
        \expandafter\@gobble
      \else
        \expandafter\@firstofone
      \fi
      {%
        \PcOrg@calc@error{#1}%
      }%
    }%
  }%
\fi
%    \end{macrocode}
%
% \subsubsection{Method \xoption{etex}}
%
%    \begin{macrocode}
\ifx\Pc@method\Pc@etexname
  \def\Pc@tokslength#1{%
    \begingroup
      \afterassignment\Pc@next
      \dimen@=\dimexpr#1\unitlength\Pc@nil{#1}%
  }%
\fi
%    \end{macrocode}
%
% \subsubsection{Method \xoption{plain}}
%
%    \begin{macrocode}
\ifx\Pc@method\Pc@plainname
  \def\Pc@tokslength#1{%
    \begingroup
      \afterassignment\Pc@next
      \dimen@=#1\unitlength\Pc@nil{#1}%
  }%
\fi
%    \end{macrocode}
%
% \subsubsection{Help macros}
%
%    \begin{macrocode}
\def\Pc@next#1\Pc@nil#2{%
  \ifx\\#1\\%
    \endgroup
    \Pc@addtoks{{#2}}%
  \else
    \expandafter\endgroup
    \expandafter\Pc@addtoks\expandafter{%
      \expandafter{\the\dimen@\@gobble}%
    }%
  \fi
}
%    \end{macrocode}
%    \begin{macro}{\Pc@nil}
%    \cs{Pc@nil} must not have the meaning of \cs{relax} because of
%    \cs{dimexpr}.
%    \begin{macrocode}
\let\Pc@nil\message
%    \end{macrocode}
%    \end{macro}
%    \begin{macro}{\Pc@addtoks}
%    \begin{macrocode}
\def\Pc@addtoks#1{%
  \toks@=\expandafter{\the\toks@#1}%
}
%    \end{macrocode}
%    \end{macro}
%    \begin{macro}{\Pc@init}
%    \begin{macrocode}
\def\Pc@init#1{%
  \begingroup
    \toks@={#1}%
}
%    \end{macrocode}
%    \end{macro}
%    \begin{macro}{\Pc@finish}
%    \begin{macrocode}
\def\Pc@finish#1{%
  \expandafter\endgroup
  \expandafter#1\the\toks@
}
%    \end{macrocode}
%    \end{macro}
%
% \subsection{Redefinitions}
%
%    \begin{macro}{\picture@redefine}
%    \begin{tabular}[t]{@{}l@{ }l@{}}
%    |#1|:& command name\\
%    |#2|:& parameter text, length parameter with \& instead of \#
%    \end{tabular}
%    \begin{macrocode}
\def\picture@redefine#1#2{%
  \begingroup
    \edef\reserved@a{%
      \noexpand\noexpand
      \expandafter\noexpand
          \csname PcOrg@\expandafter\@gobble\string#1\endcsname
    }%
    \toks0{#1}%
    \Pc@first#2&0%
}
%    \end{macrocode}
%    \end{macro}
%    \begin{macro}{\Pc@first}
%    \begin{macrocode}
\def\Pc@first#1&{%
  \toks1={#1}%
  \toks2={\Pc@init{#1}}%
  \Pc@scanlength
}
%    \end{macrocode}
%    \end{macro}
%    \begin{macro}{\Pc@scanlength}
%    |#1|: number of length parameter or zero
%    \begin{macrocode}
\def\Pc@scanlength#1{%
  \ifcase#1 %
    \expandafter\Pc@last
  \else
    \toks1=\expandafter{\the\toks1 ###1}%
    \toks2=\expandafter{\the\toks2 \Pc@tokslength{###1}}%
    \expandafter\Pc@scannext
  \fi
}
%    \end{macrocode}
%    \end{macro}
%    \begin{macro}{\Pc@scannext}
%    \begin{macrocode}
\def\Pc@scannext#1&{%
  \ifx\\#1\\%
  \else
    \toks1=\expandafter{\the\toks1 #1}%
    \toks2=\expandafter{\the\toks2 \Pc@addtoks{#1}}%
  \fi
  \Pc@scanlength
}
%    \end{macrocode}
%    \end{macro}
%    \begin{macro}{\Pc@last}
%    \begin{macrocode}
\def\Pc@last{%
  \edef\x{%
    \endgroup
    \let\reserved@a\the\toks0 %
    \def\the\toks0 \the\toks1 {%
      \the\toks2 %
      \noexpand\Pc@finish\reserved@a
    }%
  }%
  \x
}
%    \end{macrocode}
%    \end{macro}
%
% \subsubsection{\LaTeX\ base macros}
%
%    \begin{macrocode}
\picture@redefine\@picture{(&1,&2)(&3,&4)}
\picture@redefine\put{(&1,&2)}
\picture@redefine\multiput{(&1,&2)}
\picture@redefine\@multiput{(&1,&2)}
\picture@redefine\line{(#1,#2)&3}
\picture@redefine\vector{(#1,#2)&3}
\picture@redefine\dashbox{&1(&2,&3)}
\picture@redefine\@circle{&1}
\picture@redefine\@dot{&1}
\picture@redefine\@bezier{#1(&2,&3)(&4,&5)(&6,&7)}
\picture@redefine\@imakepicbox{(&1,&2)}
%    \end{macrocode}
%
% \subsubsection{Package \xpackage{pspicture}}
%
%     Package \xpackage{pspicture} changes the signature of \cs{@oval}
%     by adding an optional argument.
%    \begin{macrocode}
\@ifpackageloaded{pspicture}{%
  \picture@redefine\@oval{[&1](&2,&3)}%
  \picture@redefine\Line{(&1,&2)}%
  \picture@redefine\Curve{(&1,&2)}%
  \picture@redefine\Vector{(&1,&2)}%
}{%
  \picture@redefine\@oval{(&1,&2)}%
}
%    \end{macrocode}
%
% \subsection{Check package loading order}
%
%    \begin{macro}{\PC@checkpackage}
%    \begin{macrocode}
\def\Pc@checkpackage#1{%
  \@ifpackageloaded{#1}{%
  }{%
    \AtBeginDocument{%
      \@ifpackageloaded{#1}{%
        \PackageWarningNoLine{picture}{%
          Package `#1' is loaded after `picture'.\MessageBreak
          Load package `picture' afterwards to get full support%
          \MessageBreak
          of its additional syntax with length specifications%
        }%
      }{}%
    }%
  }%
}
%    \end{macrocode}
%    \end{macro}
%    \begin{macrocode}
\Pc@checkpackage{pict2e}
\Pc@checkpackage{pspicture}
%    \end{macrocode}
%
%    \begin{macrocode}
%</package>
%    \end{macrocode}
%
% \section{Installation}
%
% \subsection{Download}
%
% \paragraph{Package.} This package is available on
% CTAN\footnote{\url{ftp://ftp.ctan.org/tex-archive/}}:
% \begin{description}
% \item[\CTAN{macros/latex/contrib/oberdiek/picture.dtx}] The source file.
% \item[\CTAN{macros/latex/contrib/oberdiek/picture.pdf}] Documentation.
% \end{description}
%
%
% \paragraph{Bundle.} All the packages of the bundle `oberdiek'
% are also available in a TDS compliant ZIP archive. There
% the packages are already unpacked and the documentation files
% are generated. The files and directories obey the TDS standard.
% \begin{description}
% \item[\CTAN{install/macros/latex/contrib/oberdiek.tds.zip}]
% \end{description}
% \emph{TDS} refers to the standard ``A Directory Structure
% for \TeX\ Files'' (\CTAN{tds/tds.pdf}). Directories
% with \xfile{texmf} in their name are usually organized this way.
%
% \subsection{Bundle installation}
%
% \paragraph{Unpacking.} Unpack the \xfile{oberdiek.tds.zip} in the
% TDS tree (also known as \xfile{texmf} tree) of your choice.
% Example (linux):
% \begin{quote}
%   |unzip oberdiek.tds.zip -d ~/texmf|
% \end{quote}
%
% \paragraph{Script installation.}
% Check the directory \xfile{TDS:scripts/oberdiek/} for
% scripts that need further installation steps.
% Package \xpackage{attachfile2} comes with the Perl script
% \xfile{pdfatfi.pl} that should be installed in such a way
% that it can be called as \texttt{pdfatfi}.
% Example (linux):
% \begin{quote}
%   |chmod +x scripts/oberdiek/pdfatfi.pl|\\
%   |cp scripts/oberdiek/pdfatfi.pl /usr/local/bin/|
% \end{quote}
%
% \subsection{Package installation}
%
% \paragraph{Unpacking.} The \xfile{.dtx} file is a self-extracting
% \docstrip\ archive. The files are extracted by running the
% \xfile{.dtx} through \plainTeX:
% \begin{quote}
%   \verb|tex picture.dtx|
% \end{quote}
%
% \paragraph{TDS.} Now the different files must be moved into
% the different directories in your installation TDS tree
% (also known as \xfile{texmf} tree):
% \begin{quote}
% \def\t{^^A
% \begin{tabular}{@{}>{\ttfamily}l@{ $\rightarrow$ }>{\ttfamily}l@{}}
%   picture.sty & tex/latex/oberdiek/picture.sty\\
%   picture.pdf & doc/latex/oberdiek/picture.pdf\\
%   picture-example.tex & doc/latex/oberdiek/picture-example.tex\\
%   picture.dtx & source/latex/oberdiek/picture.dtx\\
% \end{tabular}^^A
% }^^A
% \sbox0{\t}^^A
% \ifdim\wd0>\linewidth
%   \begingroup
%     \advance\linewidth by\leftmargin
%     \advance\linewidth by\rightmargin
%   \edef\x{\endgroup
%     \def\noexpand\lw{\the\linewidth}^^A
%   }\x
%   \def\lwbox{^^A
%     \leavevmode
%     \hbox to \linewidth{^^A
%       \kern-\leftmargin\relax
%       \hss
%       \usebox0
%       \hss
%       \kern-\rightmargin\relax
%     }^^A
%   }^^A
%   \ifdim\wd0>\lw
%     \sbox0{\small\t}^^A
%     \ifdim\wd0>\linewidth
%       \ifdim\wd0>\lw
%         \sbox0{\footnotesize\t}^^A
%         \ifdim\wd0>\linewidth
%           \ifdim\wd0>\lw
%             \sbox0{\scriptsize\t}^^A
%             \ifdim\wd0>\linewidth
%               \ifdim\wd0>\lw
%                 \sbox0{\tiny\t}^^A
%                 \ifdim\wd0>\linewidth
%                   \lwbox
%                 \else
%                   \usebox0
%                 \fi
%               \else
%                 \lwbox
%               \fi
%             \else
%               \usebox0
%             \fi
%           \else
%             \lwbox
%           \fi
%         \else
%           \usebox0
%         \fi
%       \else
%         \lwbox
%       \fi
%     \else
%       \usebox0
%     \fi
%   \else
%     \lwbox
%   \fi
% \else
%   \usebox0
% \fi
% \end{quote}
% If you have a \xfile{docstrip.cfg} that configures and enables \docstrip's
% TDS installing feature, then some files can already be in the right
% place, see the documentation of \docstrip.
%
% \subsection{Refresh file name databases}
%
% If your \TeX~distribution
% (\teTeX, \mikTeX, \dots) relies on file name databases, you must refresh
% these. For example, \teTeX\ users run \verb|texhash| or
% \verb|mktexlsr|.
%
% \subsection{Some details for the interested}
%
% \paragraph{Attached source.}
%
% The PDF documentation on CTAN also includes the
% \xfile{.dtx} source file. It can be extracted by
% AcrobatReader 6 or higher. Another option is \textsf{pdftk},
% e.g. unpack the file into the current directory:
% \begin{quote}
%   \verb|pdftk picture.pdf unpack_files output .|
% \end{quote}
%
% \paragraph{Unpacking with \LaTeX.}
% The \xfile{.dtx} chooses its action depending on the format:
% \begin{description}
% \item[\plainTeX:] Run \docstrip\ and extract the files.
% \item[\LaTeX:] Generate the documentation.
% \end{description}
% If you insist on using \LaTeX\ for \docstrip\ (really,
% \docstrip\ does not need \LaTeX), then inform the autodetect routine
% about your intention:
% \begin{quote}
%   \verb|latex \let\install=y% \iffalse meta-comment
%
% File: picture.dtx
% Version: 2009/10/11 v1.3
% Info: Dimens for picture macros
%
% Copyright (C) 2006-2009 by
%    Heiko Oberdiek <heiko.oberdiek at googlemail.com>
%
% This work may be distributed and/or modified under the
% conditions of the LaTeX Project Public License, either
% version 1.3c of this license or (at your option) any later
% version. This version of this license is in
%    http://www.latex-project.org/lppl/lppl-1-3c.txt
% and the latest version of this license is in
%    http://www.latex-project.org/lppl.txt
% and version 1.3 or later is part of all distributions of
% LaTeX version 2005/12/01 or later.
%
% This work has the LPPL maintenance status "maintained".
%
% This Current Maintainer of this work is Heiko Oberdiek.
%
% This work consists of the main source file picture.dtx
% and the derived files
%    picture.sty, picture.pdf, picture.ins, picture.drv,
%    picture-example.tex.
%
% Distribution:
%    CTAN:macros/latex/contrib/oberdiek/picture.dtx
%    CTAN:macros/latex/contrib/oberdiek/picture.pdf
%
% Unpacking:
%    (a) If picture.ins is present:
%           tex picture.ins
%    (b) Without picture.ins:
%           tex picture.dtx
%    (c) If you insist on using LaTeX
%           latex \let\install=y% \iffalse meta-comment
%
% File: picture.dtx
% Version: 2009/10/11 v1.3
% Info: Dimens for picture macros
%
% Copyright (C) 2006-2009 by
%    Heiko Oberdiek <heiko.oberdiek at googlemail.com>
%
% This work may be distributed and/or modified under the
% conditions of the LaTeX Project Public License, either
% version 1.3c of this license or (at your option) any later
% version. This version of this license is in
%    http://www.latex-project.org/lppl/lppl-1-3c.txt
% and the latest version of this license is in
%    http://www.latex-project.org/lppl.txt
% and version 1.3 or later is part of all distributions of
% LaTeX version 2005/12/01 or later.
%
% This work has the LPPL maintenance status "maintained".
%
% This Current Maintainer of this work is Heiko Oberdiek.
%
% This work consists of the main source file picture.dtx
% and the derived files
%    picture.sty, picture.pdf, picture.ins, picture.drv,
%    picture-example.tex.
%
% Distribution:
%    CTAN:macros/latex/contrib/oberdiek/picture.dtx
%    CTAN:macros/latex/contrib/oberdiek/picture.pdf
%
% Unpacking:
%    (a) If picture.ins is present:
%           tex picture.ins
%    (b) Without picture.ins:
%           tex picture.dtx
%    (c) If you insist on using LaTeX
%           latex \let\install=y\input{picture.dtx}
%        (quote the arguments according to the demands of your shell)
%
% Documentation:
%    (a) If picture.drv is present:
%           latex picture.drv
%    (b) Without picture.drv:
%           latex picture.dtx; ...
%    The class ltxdoc loads the configuration file ltxdoc.cfg
%    if available. Here you can specify further options, e.g.
%    use A4 as paper format:
%       \PassOptionsToClass{a4paper}{article}
%
%    Programm calls to get the documentation (example):
%       pdflatex picture.dtx
%       makeindex -s gind.ist picture.idx
%       pdflatex picture.dtx
%       makeindex -s gind.ist picture.idx
%       pdflatex picture.dtx
%
% Installation:
%    TDS:tex/latex/oberdiek/picture.sty
%    TDS:doc/latex/oberdiek/picture.pdf
%    TDS:doc/latex/oberdiek/picture-example.tex
%    TDS:source/latex/oberdiek/picture.dtx
%
%<*ignore>
\begingroup
  \catcode123=1 %
  \catcode125=2 %
  \def\x{LaTeX2e}%
\expandafter\endgroup
\ifcase 0\ifx\install y1\fi\expandafter
         \ifx\csname processbatchFile\endcsname\relax\else1\fi
         \ifx\fmtname\x\else 1\fi\relax
\else\csname fi\endcsname
%</ignore>
%<*install>
\input docstrip.tex
\Msg{************************************************************************}
\Msg{* Installation}
\Msg{* Package: picture 2009/10/11 v1.3 Dimens for picture macros (HO)}
\Msg{************************************************************************}

\keepsilent
\askforoverwritefalse

\let\MetaPrefix\relax
\preamble

This is a generated file.

Project: picture
Version: 2009/10/11 v1.3

Copyright (C) 2006-2009 by
   Heiko Oberdiek <heiko.oberdiek at googlemail.com>

This work may be distributed and/or modified under the
conditions of the LaTeX Project Public License, either
version 1.3c of this license or (at your option) any later
version. This version of this license is in
   http://www.latex-project.org/lppl/lppl-1-3c.txt
and the latest version of this license is in
   http://www.latex-project.org/lppl.txt
and version 1.3 or later is part of all distributions of
LaTeX version 2005/12/01 or later.

This work has the LPPL maintenance status "maintained".

This Current Maintainer of this work is Heiko Oberdiek.

This work consists of the main source file picture.dtx
and the derived files
   picture.sty, picture.pdf, picture.ins, picture.drv,
   picture-example.tex.

\endpreamble
\let\MetaPrefix\DoubleperCent

\generate{%
  \file{picture.ins}{\from{picture.dtx}{install}}%
  \file{picture.drv}{\from{picture.dtx}{driver}}%
  \usedir{tex/latex/oberdiek}%
  \file{picture.sty}{\from{picture.dtx}{package}}%
  \usedir{doc/latex/oberdiek}%
  \file{picture-example.tex}{\from{picture.dtx}{example}}%
  \nopreamble
  \nopostamble
  \usedir{source/latex/oberdiek/catalogue}%
  \file{picture.xml}{\from{picture.dtx}{catalogue}}%
}

\catcode32=13\relax% active space
\let =\space%
\Msg{************************************************************************}
\Msg{*}
\Msg{* To finish the installation you have to move the following}
\Msg{* file into a directory searched by TeX:}
\Msg{*}
\Msg{*     picture.sty}
\Msg{*}
\Msg{* To produce the documentation run the file `picture.drv'}
\Msg{* through LaTeX.}
\Msg{*}
\Msg{* Happy TeXing!}
\Msg{*}
\Msg{************************************************************************}

\endbatchfile
%</install>
%<*ignore>
\fi
%</ignore>
%<*driver>
\NeedsTeXFormat{LaTeX2e}
\ProvidesFile{picture.drv}%
  [2009/10/11 v1.3 Dimens for picture macros (HO)]%
\documentclass{ltxdoc}
\usepackage{holtxdoc}[2011/11/22]
\begin{document}
  \DocInput{picture.dtx}%
\end{document}
%</driver>
% \fi
%
% \CheckSum{322}
%
% \CharacterTable
%  {Upper-case    \A\B\C\D\E\F\G\H\I\J\K\L\M\N\O\P\Q\R\S\T\U\V\W\X\Y\Z
%   Lower-case    \a\b\c\d\e\f\g\h\i\j\k\l\m\n\o\p\q\r\s\t\u\v\w\x\y\z
%   Digits        \0\1\2\3\4\5\6\7\8\9
%   Exclamation   \!     Double quote  \"     Hash (number) \#
%   Dollar        \$     Percent       \%     Ampersand     \&
%   Acute accent  \'     Left paren    \(     Right paren   \)
%   Asterisk      \*     Plus          \+     Comma         \,
%   Minus         \-     Point         \.     Solidus       \/
%   Colon         \:     Semicolon     \;     Less than     \<
%   Equals        \=     Greater than  \>     Question mark \?
%   Commercial at \@     Left bracket  \[     Backslash     \\
%   Right bracket \]     Circumflex    \^     Underscore    \_
%   Grave accent  \`     Left brace    \{     Vertical bar  \|
%   Right brace   \}     Tilde         \~}
%
% \GetFileInfo{picture.drv}
%
% \title{The \xpackage{picture} package}
% \date{2009/10/11 v1.3}
% \author{Heiko Oberdiek\\\xemail{heiko.oberdiek at googlemail.com}}
%
% \maketitle
%
% \begin{abstract}
% There are macro and environment arguments that expect numbers
% that will internally be multiplicated with \cs{unitlength}.
% This package extends the syntax of these arguments that
% dimens with calculation support can be added for these arguments.
% \end{abstract}
%
% \tableofcontents
%
% \section{User interface}
%
% \subsection{Introduction}
%
% The environment |picture| and macros such as \cs{put}, \cs{line},
% \cs{vector} and other macros have arguments that expect numbers
% that are used as factor for \cs{unitlength}.
% This package redefines such macros and adds code that detects
% whether such an argument is given as number or as length.
% In the latter case, the length is used directly without multiplying
% with \cs{unitlength}.
%
% \subsection{Options}
%
% Depending on the available features, also length expressions
% can be given. Option \xoption{calc} loads package \xpackage{calc}.
% Then expressions of these package may be used.
% Otherwise \xoption{etex} wraps the length argument inside
% \cs{dimexpr}\dots\cs{relax}, if \eTeX\ is available. Otherwise
% option \xoption{plain} uses plain assignments without calculation
% support.
%
% The default is \xoption{calc} if package \xpackage{calc}
% is loaded before package \xoption{picture}. If you specify
% option \xoption{calc} the loading of \xpackage{calc} is ensured.
% Otherwise package \xpackage{picture} looks whether \cs{dimexpr}
% is available and uses then option \xoption{etex} as default.
% If \eTeX\ also could not be found, then \xoption{plain} is used.
%
% \subsection{Example}
%
%    \begin{macrocode}
%<*example>
\documentclass{article}

\usepackage[calc]{picture}

\begin{document}

\setlength{\unitlength}{1pt}

\begin{picture}(\widthof{Hello World}, 10mm)
  \put(0, 0){\makebox(0,0)[lb]{Hello World}}%
  \put(0, \heightof{Hello World} + \fboxsep){%
    \line(1, 0){\widthof{Hello World}}%
  }%
  \put(\widthof{Hello World}, 10mm){%
    \line(0, -1){10mm}%
  }%
\end{picture}

\end{document}
%</example>
%    \end{macrocode}
%
% \subsection{Supported packages}
%
% Packages \xpackage{pspicture} and \xpackage{pict2e} are supported,
% but they must be loaded before package \xpackage{picture}.
%
% New macros can be supported by \cs{picture@redefine}.
% The first argument is the macro which contains the arguments
% in its parameter text that you want to support by package \xpackage{picture}.
% The second argument contains the parameter text. Change \# to \& for
% the arguments in question.
% Examples (already used by package \xpackage{picture}):
% \begin{quote}
% |\picture@redefine\put{(&1,&2)}|\\
% |\picture@redefine\line{(#1,#2)&3}|
% \end{quote}
%
% \StopEventually{
% }
%
% \section{Implementation}
%
% \subsection{Identification}
%
%    \begin{macrocode}
%<*package>
\NeedsTeXFormat{LaTeX2e}
\ProvidesPackage{picture}%
  [2009/10/11 v1.3 Dimens for picture macros (HO)]%
%    \end{macrocode}
%
% \subsection{Options}
%
%    \begin{macrocode}
\def\Pc@calcname{calc}
\def\Pc@etexname{etex}
\def\Pc@plainname{plain}
%    \end{macrocode}
%    \begin{macro}{\Pc@method}
%    Macro \cs{Pc@method} stores the method to use for calculations.
%    Check which features are available and set the default for \cs{Pc@method}.
%    \begin{macrocode}
\@ifpackageloaded{calc}{%
  \let\Pc@method\Pc@calcname
}{%
  \begingroup\expandafter\expandafter\expandafter\endgroup
  \expandafter\ifx\csname dimexpr\endcsname\relax
    \let\Pc@method\Pc@plainname
  \else
    \let\Pc@method\Pc@etexname
  \fi
}
%    \end{macrocode}
%    \end{macro}
%    \begin{macrocode}
\DeclareOption{plain}{%
  \let\Pc@method\Pc@plainname
}
\DeclareOption{etex}{%
  \begingroup\expandafter\expandafter\expandafter\endgroup
  \expandafter\ifx\csname dimexpr\endcsname\relax
    \PackageError{picture}{%
      e-TeX is not available%
    }\@ehc
  \else
    \let\Pc@method\Pc@etexname
  \fi
}
\DeclareOption{calc}{%
  \let\Pc@method\Pc@calcname
}
\ProcessOptions*
\begingroup
  \let\on@line\@empty
  \PackageInfo{picture}{Calculation method: \Pc@method}%
\endgroup
%    \end{macrocode}
%
% \subsection{Calculation method}
%
%    \begin{macrocode}
\ifx\Pc@method\Pc@calcname
  \RequirePackage{calc}%
\fi
%    \end{macrocode}
%
% \subsubsection{Method \xoption{calc}}
%
%    \begin{macrocode}
\ifx\Pc@method\Pc@calcname
  \def\Pc@tokslength#1{%
    \begingroup
      \let\calc@error\Pc@calc@error
      \setlength\dimen@{#1\unitlength}\Pc@next\Pc@nil{#1}%
  }%
  \let\PcOrg@calc@error\calc@error
  \@ifpackagelater{calc}{2007/08/22}{% v4.3
    \def\Pc@calc@error#1{%
      \expandafter\ifx\expandafter\unitlength\noexpand#1\relax
        \def\calc@next##1!{%
          \endgroup
          \aftergroup\afterassignment
          \aftergroup\Pc@next
        }%
        \expandafter\@firstoftwo
      \else
        \expandafter\@secondoftwo
      \fi
      {%
        \calc@next{#1}%
      }{%
        \PcOrg@calc@error{#1}%
      }%
    }%
  }{%
    \def\Pc@calc@error#1{%
      \expandafter\ifx\expandafter\unitlength\noexpand#1\relax
        \def\calc@next##1!{%
          \endgroup
          \aftergroup\afterassignment
          \aftergroup\Pc@next
        }%
        \expandafter\@gobble
      \else
        \expandafter\@firstofone
      \fi
      {%
        \PcOrg@calc@error{#1}%
      }%
    }%
  }%
\fi
%    \end{macrocode}
%
% \subsubsection{Method \xoption{etex}}
%
%    \begin{macrocode}
\ifx\Pc@method\Pc@etexname
  \def\Pc@tokslength#1{%
    \begingroup
      \afterassignment\Pc@next
      \dimen@=\dimexpr#1\unitlength\Pc@nil{#1}%
  }%
\fi
%    \end{macrocode}
%
% \subsubsection{Method \xoption{plain}}
%
%    \begin{macrocode}
\ifx\Pc@method\Pc@plainname
  \def\Pc@tokslength#1{%
    \begingroup
      \afterassignment\Pc@next
      \dimen@=#1\unitlength\Pc@nil{#1}%
  }%
\fi
%    \end{macrocode}
%
% \subsubsection{Help macros}
%
%    \begin{macrocode}
\def\Pc@next#1\Pc@nil#2{%
  \ifx\\#1\\%
    \endgroup
    \Pc@addtoks{{#2}}%
  \else
    \expandafter\endgroup
    \expandafter\Pc@addtoks\expandafter{%
      \expandafter{\the\dimen@\@gobble}%
    }%
  \fi
}
%    \end{macrocode}
%    \begin{macro}{\Pc@nil}
%    \cs{Pc@nil} must not have the meaning of \cs{relax} because of
%    \cs{dimexpr}.
%    \begin{macrocode}
\let\Pc@nil\message
%    \end{macrocode}
%    \end{macro}
%    \begin{macro}{\Pc@addtoks}
%    \begin{macrocode}
\def\Pc@addtoks#1{%
  \toks@=\expandafter{\the\toks@#1}%
}
%    \end{macrocode}
%    \end{macro}
%    \begin{macro}{\Pc@init}
%    \begin{macrocode}
\def\Pc@init#1{%
  \begingroup
    \toks@={#1}%
}
%    \end{macrocode}
%    \end{macro}
%    \begin{macro}{\Pc@finish}
%    \begin{macrocode}
\def\Pc@finish#1{%
  \expandafter\endgroup
  \expandafter#1\the\toks@
}
%    \end{macrocode}
%    \end{macro}
%
% \subsection{Redefinitions}
%
%    \begin{macro}{\picture@redefine}
%    \begin{tabular}[t]{@{}l@{ }l@{}}
%    |#1|:& command name\\
%    |#2|:& parameter text, length parameter with \& instead of \#
%    \end{tabular}
%    \begin{macrocode}
\def\picture@redefine#1#2{%
  \begingroup
    \edef\reserved@a{%
      \noexpand\noexpand
      \expandafter\noexpand
          \csname PcOrg@\expandafter\@gobble\string#1\endcsname
    }%
    \toks0{#1}%
    \Pc@first#2&0%
}
%    \end{macrocode}
%    \end{macro}
%    \begin{macro}{\Pc@first}
%    \begin{macrocode}
\def\Pc@first#1&{%
  \toks1={#1}%
  \toks2={\Pc@init{#1}}%
  \Pc@scanlength
}
%    \end{macrocode}
%    \end{macro}
%    \begin{macro}{\Pc@scanlength}
%    |#1|: number of length parameter or zero
%    \begin{macrocode}
\def\Pc@scanlength#1{%
  \ifcase#1 %
    \expandafter\Pc@last
  \else
    \toks1=\expandafter{\the\toks1 ###1}%
    \toks2=\expandafter{\the\toks2 \Pc@tokslength{###1}}%
    \expandafter\Pc@scannext
  \fi
}
%    \end{macrocode}
%    \end{macro}
%    \begin{macro}{\Pc@scannext}
%    \begin{macrocode}
\def\Pc@scannext#1&{%
  \ifx\\#1\\%
  \else
    \toks1=\expandafter{\the\toks1 #1}%
    \toks2=\expandafter{\the\toks2 \Pc@addtoks{#1}}%
  \fi
  \Pc@scanlength
}
%    \end{macrocode}
%    \end{macro}
%    \begin{macro}{\Pc@last}
%    \begin{macrocode}
\def\Pc@last{%
  \edef\x{%
    \endgroup
    \let\reserved@a\the\toks0 %
    \def\the\toks0 \the\toks1 {%
      \the\toks2 %
      \noexpand\Pc@finish\reserved@a
    }%
  }%
  \x
}
%    \end{macrocode}
%    \end{macro}
%
% \subsubsection{\LaTeX\ base macros}
%
%    \begin{macrocode}
\picture@redefine\@picture{(&1,&2)(&3,&4)}
\picture@redefine\put{(&1,&2)}
\picture@redefine\multiput{(&1,&2)}
\picture@redefine\@multiput{(&1,&2)}
\picture@redefine\line{(#1,#2)&3}
\picture@redefine\vector{(#1,#2)&3}
\picture@redefine\dashbox{&1(&2,&3)}
\picture@redefine\@circle{&1}
\picture@redefine\@dot{&1}
\picture@redefine\@bezier{#1(&2,&3)(&4,&5)(&6,&7)}
\picture@redefine\@imakepicbox{(&1,&2)}
%    \end{macrocode}
%
% \subsubsection{Package \xpackage{pspicture}}
%
%     Package \xpackage{pspicture} changes the signature of \cs{@oval}
%     by adding an optional argument.
%    \begin{macrocode}
\@ifpackageloaded{pspicture}{%
  \picture@redefine\@oval{[&1](&2,&3)}%
  \picture@redefine\Line{(&1,&2)}%
  \picture@redefine\Curve{(&1,&2)}%
  \picture@redefine\Vector{(&1,&2)}%
}{%
  \picture@redefine\@oval{(&1,&2)}%
}
%    \end{macrocode}
%
% \subsection{Check package loading order}
%
%    \begin{macro}{\PC@checkpackage}
%    \begin{macrocode}
\def\Pc@checkpackage#1{%
  \@ifpackageloaded{#1}{%
  }{%
    \AtBeginDocument{%
      \@ifpackageloaded{#1}{%
        \PackageWarningNoLine{picture}{%
          Package `#1' is loaded after `picture'.\MessageBreak
          Load package `picture' afterwards to get full support%
          \MessageBreak
          of its additional syntax with length specifications%
        }%
      }{}%
    }%
  }%
}
%    \end{macrocode}
%    \end{macro}
%    \begin{macrocode}
\Pc@checkpackage{pict2e}
\Pc@checkpackage{pspicture}
%    \end{macrocode}
%
%    \begin{macrocode}
%</package>
%    \end{macrocode}
%
% \section{Installation}
%
% \subsection{Download}
%
% \paragraph{Package.} This package is available on
% CTAN\footnote{\url{ftp://ftp.ctan.org/tex-archive/}}:
% \begin{description}
% \item[\CTAN{macros/latex/contrib/oberdiek/picture.dtx}] The source file.
% \item[\CTAN{macros/latex/contrib/oberdiek/picture.pdf}] Documentation.
% \end{description}
%
%
% \paragraph{Bundle.} All the packages of the bundle `oberdiek'
% are also available in a TDS compliant ZIP archive. There
% the packages are already unpacked and the documentation files
% are generated. The files and directories obey the TDS standard.
% \begin{description}
% \item[\CTAN{install/macros/latex/contrib/oberdiek.tds.zip}]
% \end{description}
% \emph{TDS} refers to the standard ``A Directory Structure
% for \TeX\ Files'' (\CTAN{tds/tds.pdf}). Directories
% with \xfile{texmf} in their name are usually organized this way.
%
% \subsection{Bundle installation}
%
% \paragraph{Unpacking.} Unpack the \xfile{oberdiek.tds.zip} in the
% TDS tree (also known as \xfile{texmf} tree) of your choice.
% Example (linux):
% \begin{quote}
%   |unzip oberdiek.tds.zip -d ~/texmf|
% \end{quote}
%
% \paragraph{Script installation.}
% Check the directory \xfile{TDS:scripts/oberdiek/} for
% scripts that need further installation steps.
% Package \xpackage{attachfile2} comes with the Perl script
% \xfile{pdfatfi.pl} that should be installed in such a way
% that it can be called as \texttt{pdfatfi}.
% Example (linux):
% \begin{quote}
%   |chmod +x scripts/oberdiek/pdfatfi.pl|\\
%   |cp scripts/oberdiek/pdfatfi.pl /usr/local/bin/|
% \end{quote}
%
% \subsection{Package installation}
%
% \paragraph{Unpacking.} The \xfile{.dtx} file is a self-extracting
% \docstrip\ archive. The files are extracted by running the
% \xfile{.dtx} through \plainTeX:
% \begin{quote}
%   \verb|tex picture.dtx|
% \end{quote}
%
% \paragraph{TDS.} Now the different files must be moved into
% the different directories in your installation TDS tree
% (also known as \xfile{texmf} tree):
% \begin{quote}
% \def\t{^^A
% \begin{tabular}{@{}>{\ttfamily}l@{ $\rightarrow$ }>{\ttfamily}l@{}}
%   picture.sty & tex/latex/oberdiek/picture.sty\\
%   picture.pdf & doc/latex/oberdiek/picture.pdf\\
%   picture-example.tex & doc/latex/oberdiek/picture-example.tex\\
%   picture.dtx & source/latex/oberdiek/picture.dtx\\
% \end{tabular}^^A
% }^^A
% \sbox0{\t}^^A
% \ifdim\wd0>\linewidth
%   \begingroup
%     \advance\linewidth by\leftmargin
%     \advance\linewidth by\rightmargin
%   \edef\x{\endgroup
%     \def\noexpand\lw{\the\linewidth}^^A
%   }\x
%   \def\lwbox{^^A
%     \leavevmode
%     \hbox to \linewidth{^^A
%       \kern-\leftmargin\relax
%       \hss
%       \usebox0
%       \hss
%       \kern-\rightmargin\relax
%     }^^A
%   }^^A
%   \ifdim\wd0>\lw
%     \sbox0{\small\t}^^A
%     \ifdim\wd0>\linewidth
%       \ifdim\wd0>\lw
%         \sbox0{\footnotesize\t}^^A
%         \ifdim\wd0>\linewidth
%           \ifdim\wd0>\lw
%             \sbox0{\scriptsize\t}^^A
%             \ifdim\wd0>\linewidth
%               \ifdim\wd0>\lw
%                 \sbox0{\tiny\t}^^A
%                 \ifdim\wd0>\linewidth
%                   \lwbox
%                 \else
%                   \usebox0
%                 \fi
%               \else
%                 \lwbox
%               \fi
%             \else
%               \usebox0
%             \fi
%           \else
%             \lwbox
%           \fi
%         \else
%           \usebox0
%         \fi
%       \else
%         \lwbox
%       \fi
%     \else
%       \usebox0
%     \fi
%   \else
%     \lwbox
%   \fi
% \else
%   \usebox0
% \fi
% \end{quote}
% If you have a \xfile{docstrip.cfg} that configures and enables \docstrip's
% TDS installing feature, then some files can already be in the right
% place, see the documentation of \docstrip.
%
% \subsection{Refresh file name databases}
%
% If your \TeX~distribution
% (\teTeX, \mikTeX, \dots) relies on file name databases, you must refresh
% these. For example, \teTeX\ users run \verb|texhash| or
% \verb|mktexlsr|.
%
% \subsection{Some details for the interested}
%
% \paragraph{Attached source.}
%
% The PDF documentation on CTAN also includes the
% \xfile{.dtx} source file. It can be extracted by
% AcrobatReader 6 or higher. Another option is \textsf{pdftk},
% e.g. unpack the file into the current directory:
% \begin{quote}
%   \verb|pdftk picture.pdf unpack_files output .|
% \end{quote}
%
% \paragraph{Unpacking with \LaTeX.}
% The \xfile{.dtx} chooses its action depending on the format:
% \begin{description}
% \item[\plainTeX:] Run \docstrip\ and extract the files.
% \item[\LaTeX:] Generate the documentation.
% \end{description}
% If you insist on using \LaTeX\ for \docstrip\ (really,
% \docstrip\ does not need \LaTeX), then inform the autodetect routine
% about your intention:
% \begin{quote}
%   \verb|latex \let\install=y\input{picture.dtx}|
% \end{quote}
% Do not forget to quote the argument according to the demands
% of your shell.
%
% \paragraph{Generating the documentation.}
% You can use both the \xfile{.dtx} or the \xfile{.drv} to generate
% the documentation. The process can be configured by the
% configuration file \xfile{ltxdoc.cfg}. For instance, put this
% line into this file, if you want to have A4 as paper format:
% \begin{quote}
%   \verb|\PassOptionsToClass{a4paper}{article}|
% \end{quote}
% An example follows how to generate the
% documentation with pdf\LaTeX:
% \begin{quote}
%\begin{verbatim}
%pdflatex picture.dtx
%makeindex -s gind.ist picture.idx
%pdflatex picture.dtx
%makeindex -s gind.ist picture.idx
%pdflatex picture.dtx
%\end{verbatim}
% \end{quote}
%
% \section{Catalogue}
%
% The following XML file can be used as source for the
% \href{http://mirror.ctan.org/help/Catalogue/catalogue.html}{\TeX\ Catalogue}.
% The elements \texttt{caption} and \texttt{description} are imported
% from the original XML file from the Catalogue.
% The name of the XML file in the Catalogue is \xfile{picture.xml}.
%    \begin{macrocode}
%<*catalogue>
<?xml version='1.0' encoding='us-ascii'?>
<!DOCTYPE entry SYSTEM 'catalogue.dtd'>
<entry datestamp='$Date$' modifier='$Author$' id='picture'>
  <name>picture</name>
  <caption>Dimens for picture macros.</caption>
  <authorref id='auth:oberdiek'/>
  <copyright owner='Heiko Oberdiek' year='2006-2009'/>
  <license type='lppl1.3'/>
  <version number='1.3'/>
  <description>
    There are macro and environment arguments that expect numbers
    that will internally be multiplied by <tt>\unitlength</tt>.
    This package extends the syntax of these arguments, so that
    dimensions with calculation support may be used for these arguments.
    <p/>
    The package is part of the <xref refid='oberdiek'>oberdiek</xref> bundle.
  </description>
  <documentation details='Package documentation'
      href='ctan:/macros/latex/contrib/oberdiek/picture.pdf'/>
  <ctan file='true' path='/macros/latex/contrib/oberdiek/picture.dtx'/>
  <miktex location='oberdiek'/>
  <texlive location='oberdiek'/>
  <install path='/macros/latex/contrib/oberdiek/oberdiek.tds.zip'/>
</entry>
%</catalogue>
%    \end{macrocode}
%
% \begin{History}
%   \begin{Version}{2006/08/26 v1.0}
%   \item
%     First released version. (First start of the project was June/July 2002.)
%   \end{Version}
%   \begin{Version}{2007/04/11 v1.1}
%   \item
%     Line ends sanitized.
%   \end{Version}
%   \begin{Version}{2008/11/26 v1.2}
%   \item
%     Package \xpackage{pict2e} added to documentation section ``Supported
%     packages''.
%   \item
%     Package order of supported packages is checked.
%   \end{Version}
%   \begin{Version}{2009/10/11 v1.3}
%   \item
%     Fix because of new version v4.3 of package \xpackage{calc}.
%   \end{Version}
% \end{History}
%
% \PrintIndex
%
% \Finale
\endinput

%        (quote the arguments according to the demands of your shell)
%
% Documentation:
%    (a) If picture.drv is present:
%           latex picture.drv
%    (b) Without picture.drv:
%           latex picture.dtx; ...
%    The class ltxdoc loads the configuration file ltxdoc.cfg
%    if available. Here you can specify further options, e.g.
%    use A4 as paper format:
%       \PassOptionsToClass{a4paper}{article}
%
%    Programm calls to get the documentation (example):
%       pdflatex picture.dtx
%       makeindex -s gind.ist picture.idx
%       pdflatex picture.dtx
%       makeindex -s gind.ist picture.idx
%       pdflatex picture.dtx
%
% Installation:
%    TDS:tex/latex/oberdiek/picture.sty
%    TDS:doc/latex/oberdiek/picture.pdf
%    TDS:doc/latex/oberdiek/picture-example.tex
%    TDS:source/latex/oberdiek/picture.dtx
%
%<*ignore>
\begingroup
  \catcode123=1 %
  \catcode125=2 %
  \def\x{LaTeX2e}%
\expandafter\endgroup
\ifcase 0\ifx\install y1\fi\expandafter
         \ifx\csname processbatchFile\endcsname\relax\else1\fi
         \ifx\fmtname\x\else 1\fi\relax
\else\csname fi\endcsname
%</ignore>
%<*install>
\input docstrip.tex
\Msg{************************************************************************}
\Msg{* Installation}
\Msg{* Package: picture 2009/10/11 v1.3 Dimens for picture macros (HO)}
\Msg{************************************************************************}

\keepsilent
\askforoverwritefalse

\let\MetaPrefix\relax
\preamble

This is a generated file.

Project: picture
Version: 2009/10/11 v1.3

Copyright (C) 2006-2009 by
   Heiko Oberdiek <heiko.oberdiek at googlemail.com>

This work may be distributed and/or modified under the
conditions of the LaTeX Project Public License, either
version 1.3c of this license or (at your option) any later
version. This version of this license is in
   http://www.latex-project.org/lppl/lppl-1-3c.txt
and the latest version of this license is in
   http://www.latex-project.org/lppl.txt
and version 1.3 or later is part of all distributions of
LaTeX version 2005/12/01 or later.

This work has the LPPL maintenance status "maintained".

This Current Maintainer of this work is Heiko Oberdiek.

This work consists of the main source file picture.dtx
and the derived files
   picture.sty, picture.pdf, picture.ins, picture.drv,
   picture-example.tex.

\endpreamble
\let\MetaPrefix\DoubleperCent

\generate{%
  \file{picture.ins}{\from{picture.dtx}{install}}%
  \file{picture.drv}{\from{picture.dtx}{driver}}%
  \usedir{tex/latex/oberdiek}%
  \file{picture.sty}{\from{picture.dtx}{package}}%
  \usedir{doc/latex/oberdiek}%
  \file{picture-example.tex}{\from{picture.dtx}{example}}%
  \nopreamble
  \nopostamble
  \usedir{source/latex/oberdiek/catalogue}%
  \file{picture.xml}{\from{picture.dtx}{catalogue}}%
}

\catcode32=13\relax% active space
\let =\space%
\Msg{************************************************************************}
\Msg{*}
\Msg{* To finish the installation you have to move the following}
\Msg{* file into a directory searched by TeX:}
\Msg{*}
\Msg{*     picture.sty}
\Msg{*}
\Msg{* To produce the documentation run the file `picture.drv'}
\Msg{* through LaTeX.}
\Msg{*}
\Msg{* Happy TeXing!}
\Msg{*}
\Msg{************************************************************************}

\endbatchfile
%</install>
%<*ignore>
\fi
%</ignore>
%<*driver>
\NeedsTeXFormat{LaTeX2e}
\ProvidesFile{picture.drv}%
  [2009/10/11 v1.3 Dimens for picture macros (HO)]%
\documentclass{ltxdoc}
\usepackage{holtxdoc}[2011/11/22]
\begin{document}
  \DocInput{picture.dtx}%
\end{document}
%</driver>
% \fi
%
% \CheckSum{322}
%
% \CharacterTable
%  {Upper-case    \A\B\C\D\E\F\G\H\I\J\K\L\M\N\O\P\Q\R\S\T\U\V\W\X\Y\Z
%   Lower-case    \a\b\c\d\e\f\g\h\i\j\k\l\m\n\o\p\q\r\s\t\u\v\w\x\y\z
%   Digits        \0\1\2\3\4\5\6\7\8\9
%   Exclamation   \!     Double quote  \"     Hash (number) \#
%   Dollar        \$     Percent       \%     Ampersand     \&
%   Acute accent  \'     Left paren    \(     Right paren   \)
%   Asterisk      \*     Plus          \+     Comma         \,
%   Minus         \-     Point         \.     Solidus       \/
%   Colon         \:     Semicolon     \;     Less than     \<
%   Equals        \=     Greater than  \>     Question mark \?
%   Commercial at \@     Left bracket  \[     Backslash     \\
%   Right bracket \]     Circumflex    \^     Underscore    \_
%   Grave accent  \`     Left brace    \{     Vertical bar  \|
%   Right brace   \}     Tilde         \~}
%
% \GetFileInfo{picture.drv}
%
% \title{The \xpackage{picture} package}
% \date{2009/10/11 v1.3}
% \author{Heiko Oberdiek\\\xemail{heiko.oberdiek at googlemail.com}}
%
% \maketitle
%
% \begin{abstract}
% There are macro and environment arguments that expect numbers
% that will internally be multiplicated with \cs{unitlength}.
% This package extends the syntax of these arguments that
% dimens with calculation support can be added for these arguments.
% \end{abstract}
%
% \tableofcontents
%
% \section{User interface}
%
% \subsection{Introduction}
%
% The environment |picture| and macros such as \cs{put}, \cs{line},
% \cs{vector} and other macros have arguments that expect numbers
% that are used as factor for \cs{unitlength}.
% This package redefines such macros and adds code that detects
% whether such an argument is given as number or as length.
% In the latter case, the length is used directly without multiplying
% with \cs{unitlength}.
%
% \subsection{Options}
%
% Depending on the available features, also length expressions
% can be given. Option \xoption{calc} loads package \xpackage{calc}.
% Then expressions of these package may be used.
% Otherwise \xoption{etex} wraps the length argument inside
% \cs{dimexpr}\dots\cs{relax}, if \eTeX\ is available. Otherwise
% option \xoption{plain} uses plain assignments without calculation
% support.
%
% The default is \xoption{calc} if package \xpackage{calc}
% is loaded before package \xoption{picture}. If you specify
% option \xoption{calc} the loading of \xpackage{calc} is ensured.
% Otherwise package \xpackage{picture} looks whether \cs{dimexpr}
% is available and uses then option \xoption{etex} as default.
% If \eTeX\ also could not be found, then \xoption{plain} is used.
%
% \subsection{Example}
%
%    \begin{macrocode}
%<*example>
\documentclass{article}

\usepackage[calc]{picture}

\begin{document}

\setlength{\unitlength}{1pt}

\begin{picture}(\widthof{Hello World}, 10mm)
  \put(0, 0){\makebox(0,0)[lb]{Hello World}}%
  \put(0, \heightof{Hello World} + \fboxsep){%
    \line(1, 0){\widthof{Hello World}}%
  }%
  \put(\widthof{Hello World}, 10mm){%
    \line(0, -1){10mm}%
  }%
\end{picture}

\end{document}
%</example>
%    \end{macrocode}
%
% \subsection{Supported packages}
%
% Packages \xpackage{pspicture} and \xpackage{pict2e} are supported,
% but they must be loaded before package \xpackage{picture}.
%
% New macros can be supported by \cs{picture@redefine}.
% The first argument is the macro which contains the arguments
% in its parameter text that you want to support by package \xpackage{picture}.
% The second argument contains the parameter text. Change \# to \& for
% the arguments in question.
% Examples (already used by package \xpackage{picture}):
% \begin{quote}
% |\picture@redefine\put{(&1,&2)}|\\
% |\picture@redefine\line{(#1,#2)&3}|
% \end{quote}
%
% \StopEventually{
% }
%
% \section{Implementation}
%
% \subsection{Identification}
%
%    \begin{macrocode}
%<*package>
\NeedsTeXFormat{LaTeX2e}
\ProvidesPackage{picture}%
  [2009/10/11 v1.3 Dimens for picture macros (HO)]%
%    \end{macrocode}
%
% \subsection{Options}
%
%    \begin{macrocode}
\def\Pc@calcname{calc}
\def\Pc@etexname{etex}
\def\Pc@plainname{plain}
%    \end{macrocode}
%    \begin{macro}{\Pc@method}
%    Macro \cs{Pc@method} stores the method to use for calculations.
%    Check which features are available and set the default for \cs{Pc@method}.
%    \begin{macrocode}
\@ifpackageloaded{calc}{%
  \let\Pc@method\Pc@calcname
}{%
  \begingroup\expandafter\expandafter\expandafter\endgroup
  \expandafter\ifx\csname dimexpr\endcsname\relax
    \let\Pc@method\Pc@plainname
  \else
    \let\Pc@method\Pc@etexname
  \fi
}
%    \end{macrocode}
%    \end{macro}
%    \begin{macrocode}
\DeclareOption{plain}{%
  \let\Pc@method\Pc@plainname
}
\DeclareOption{etex}{%
  \begingroup\expandafter\expandafter\expandafter\endgroup
  \expandafter\ifx\csname dimexpr\endcsname\relax
    \PackageError{picture}{%
      e-TeX is not available%
    }\@ehc
  \else
    \let\Pc@method\Pc@etexname
  \fi
}
\DeclareOption{calc}{%
  \let\Pc@method\Pc@calcname
}
\ProcessOptions*
\begingroup
  \let\on@line\@empty
  \PackageInfo{picture}{Calculation method: \Pc@method}%
\endgroup
%    \end{macrocode}
%
% \subsection{Calculation method}
%
%    \begin{macrocode}
\ifx\Pc@method\Pc@calcname
  \RequirePackage{calc}%
\fi
%    \end{macrocode}
%
% \subsubsection{Method \xoption{calc}}
%
%    \begin{macrocode}
\ifx\Pc@method\Pc@calcname
  \def\Pc@tokslength#1{%
    \begingroup
      \let\calc@error\Pc@calc@error
      \setlength\dimen@{#1\unitlength}\Pc@next\Pc@nil{#1}%
  }%
  \let\PcOrg@calc@error\calc@error
  \@ifpackagelater{calc}{2007/08/22}{% v4.3
    \def\Pc@calc@error#1{%
      \expandafter\ifx\expandafter\unitlength\noexpand#1\relax
        \def\calc@next##1!{%
          \endgroup
          \aftergroup\afterassignment
          \aftergroup\Pc@next
        }%
        \expandafter\@firstoftwo
      \else
        \expandafter\@secondoftwo
      \fi
      {%
        \calc@next{#1}%
      }{%
        \PcOrg@calc@error{#1}%
      }%
    }%
  }{%
    \def\Pc@calc@error#1{%
      \expandafter\ifx\expandafter\unitlength\noexpand#1\relax
        \def\calc@next##1!{%
          \endgroup
          \aftergroup\afterassignment
          \aftergroup\Pc@next
        }%
        \expandafter\@gobble
      \else
        \expandafter\@firstofone
      \fi
      {%
        \PcOrg@calc@error{#1}%
      }%
    }%
  }%
\fi
%    \end{macrocode}
%
% \subsubsection{Method \xoption{etex}}
%
%    \begin{macrocode}
\ifx\Pc@method\Pc@etexname
  \def\Pc@tokslength#1{%
    \begingroup
      \afterassignment\Pc@next
      \dimen@=\dimexpr#1\unitlength\Pc@nil{#1}%
  }%
\fi
%    \end{macrocode}
%
% \subsubsection{Method \xoption{plain}}
%
%    \begin{macrocode}
\ifx\Pc@method\Pc@plainname
  \def\Pc@tokslength#1{%
    \begingroup
      \afterassignment\Pc@next
      \dimen@=#1\unitlength\Pc@nil{#1}%
  }%
\fi
%    \end{macrocode}
%
% \subsubsection{Help macros}
%
%    \begin{macrocode}
\def\Pc@next#1\Pc@nil#2{%
  \ifx\\#1\\%
    \endgroup
    \Pc@addtoks{{#2}}%
  \else
    \expandafter\endgroup
    \expandafter\Pc@addtoks\expandafter{%
      \expandafter{\the\dimen@\@gobble}%
    }%
  \fi
}
%    \end{macrocode}
%    \begin{macro}{\Pc@nil}
%    \cs{Pc@nil} must not have the meaning of \cs{relax} because of
%    \cs{dimexpr}.
%    \begin{macrocode}
\let\Pc@nil\message
%    \end{macrocode}
%    \end{macro}
%    \begin{macro}{\Pc@addtoks}
%    \begin{macrocode}
\def\Pc@addtoks#1{%
  \toks@=\expandafter{\the\toks@#1}%
}
%    \end{macrocode}
%    \end{macro}
%    \begin{macro}{\Pc@init}
%    \begin{macrocode}
\def\Pc@init#1{%
  \begingroup
    \toks@={#1}%
}
%    \end{macrocode}
%    \end{macro}
%    \begin{macro}{\Pc@finish}
%    \begin{macrocode}
\def\Pc@finish#1{%
  \expandafter\endgroup
  \expandafter#1\the\toks@
}
%    \end{macrocode}
%    \end{macro}
%
% \subsection{Redefinitions}
%
%    \begin{macro}{\picture@redefine}
%    \begin{tabular}[t]{@{}l@{ }l@{}}
%    |#1|:& command name\\
%    |#2|:& parameter text, length parameter with \& instead of \#
%    \end{tabular}
%    \begin{macrocode}
\def\picture@redefine#1#2{%
  \begingroup
    \edef\reserved@a{%
      \noexpand\noexpand
      \expandafter\noexpand
          \csname PcOrg@\expandafter\@gobble\string#1\endcsname
    }%
    \toks0{#1}%
    \Pc@first#2&0%
}
%    \end{macrocode}
%    \end{macro}
%    \begin{macro}{\Pc@first}
%    \begin{macrocode}
\def\Pc@first#1&{%
  \toks1={#1}%
  \toks2={\Pc@init{#1}}%
  \Pc@scanlength
}
%    \end{macrocode}
%    \end{macro}
%    \begin{macro}{\Pc@scanlength}
%    |#1|: number of length parameter or zero
%    \begin{macrocode}
\def\Pc@scanlength#1{%
  \ifcase#1 %
    \expandafter\Pc@last
  \else
    \toks1=\expandafter{\the\toks1 ###1}%
    \toks2=\expandafter{\the\toks2 \Pc@tokslength{###1}}%
    \expandafter\Pc@scannext
  \fi
}
%    \end{macrocode}
%    \end{macro}
%    \begin{macro}{\Pc@scannext}
%    \begin{macrocode}
\def\Pc@scannext#1&{%
  \ifx\\#1\\%
  \else
    \toks1=\expandafter{\the\toks1 #1}%
    \toks2=\expandafter{\the\toks2 \Pc@addtoks{#1}}%
  \fi
  \Pc@scanlength
}
%    \end{macrocode}
%    \end{macro}
%    \begin{macro}{\Pc@last}
%    \begin{macrocode}
\def\Pc@last{%
  \edef\x{%
    \endgroup
    \let\reserved@a\the\toks0 %
    \def\the\toks0 \the\toks1 {%
      \the\toks2 %
      \noexpand\Pc@finish\reserved@a
    }%
  }%
  \x
}
%    \end{macrocode}
%    \end{macro}
%
% \subsubsection{\LaTeX\ base macros}
%
%    \begin{macrocode}
\picture@redefine\@picture{(&1,&2)(&3,&4)}
\picture@redefine\put{(&1,&2)}
\picture@redefine\multiput{(&1,&2)}
\picture@redefine\@multiput{(&1,&2)}
\picture@redefine\line{(#1,#2)&3}
\picture@redefine\vector{(#1,#2)&3}
\picture@redefine\dashbox{&1(&2,&3)}
\picture@redefine\@circle{&1}
\picture@redefine\@dot{&1}
\picture@redefine\@bezier{#1(&2,&3)(&4,&5)(&6,&7)}
\picture@redefine\@imakepicbox{(&1,&2)}
%    \end{macrocode}
%
% \subsubsection{Package \xpackage{pspicture}}
%
%     Package \xpackage{pspicture} changes the signature of \cs{@oval}
%     by adding an optional argument.
%    \begin{macrocode}
\@ifpackageloaded{pspicture}{%
  \picture@redefine\@oval{[&1](&2,&3)}%
  \picture@redefine\Line{(&1,&2)}%
  \picture@redefine\Curve{(&1,&2)}%
  \picture@redefine\Vector{(&1,&2)}%
}{%
  \picture@redefine\@oval{(&1,&2)}%
}
%    \end{macrocode}
%
% \subsection{Check package loading order}
%
%    \begin{macro}{\PC@checkpackage}
%    \begin{macrocode}
\def\Pc@checkpackage#1{%
  \@ifpackageloaded{#1}{%
  }{%
    \AtBeginDocument{%
      \@ifpackageloaded{#1}{%
        \PackageWarningNoLine{picture}{%
          Package `#1' is loaded after `picture'.\MessageBreak
          Load package `picture' afterwards to get full support%
          \MessageBreak
          of its additional syntax with length specifications%
        }%
      }{}%
    }%
  }%
}
%    \end{macrocode}
%    \end{macro}
%    \begin{macrocode}
\Pc@checkpackage{pict2e}
\Pc@checkpackage{pspicture}
%    \end{macrocode}
%
%    \begin{macrocode}
%</package>
%    \end{macrocode}
%
% \section{Installation}
%
% \subsection{Download}
%
% \paragraph{Package.} This package is available on
% CTAN\footnote{\url{ftp://ftp.ctan.org/tex-archive/}}:
% \begin{description}
% \item[\CTAN{macros/latex/contrib/oberdiek/picture.dtx}] The source file.
% \item[\CTAN{macros/latex/contrib/oberdiek/picture.pdf}] Documentation.
% \end{description}
%
%
% \paragraph{Bundle.} All the packages of the bundle `oberdiek'
% are also available in a TDS compliant ZIP archive. There
% the packages are already unpacked and the documentation files
% are generated. The files and directories obey the TDS standard.
% \begin{description}
% \item[\CTAN{install/macros/latex/contrib/oberdiek.tds.zip}]
% \end{description}
% \emph{TDS} refers to the standard ``A Directory Structure
% for \TeX\ Files'' (\CTAN{tds/tds.pdf}). Directories
% with \xfile{texmf} in their name are usually organized this way.
%
% \subsection{Bundle installation}
%
% \paragraph{Unpacking.} Unpack the \xfile{oberdiek.tds.zip} in the
% TDS tree (also known as \xfile{texmf} tree) of your choice.
% Example (linux):
% \begin{quote}
%   |unzip oberdiek.tds.zip -d ~/texmf|
% \end{quote}
%
% \paragraph{Script installation.}
% Check the directory \xfile{TDS:scripts/oberdiek/} for
% scripts that need further installation steps.
% Package \xpackage{attachfile2} comes with the Perl script
% \xfile{pdfatfi.pl} that should be installed in such a way
% that it can be called as \texttt{pdfatfi}.
% Example (linux):
% \begin{quote}
%   |chmod +x scripts/oberdiek/pdfatfi.pl|\\
%   |cp scripts/oberdiek/pdfatfi.pl /usr/local/bin/|
% \end{quote}
%
% \subsection{Package installation}
%
% \paragraph{Unpacking.} The \xfile{.dtx} file is a self-extracting
% \docstrip\ archive. The files are extracted by running the
% \xfile{.dtx} through \plainTeX:
% \begin{quote}
%   \verb|tex picture.dtx|
% \end{quote}
%
% \paragraph{TDS.} Now the different files must be moved into
% the different directories in your installation TDS tree
% (also known as \xfile{texmf} tree):
% \begin{quote}
% \def\t{^^A
% \begin{tabular}{@{}>{\ttfamily}l@{ $\rightarrow$ }>{\ttfamily}l@{}}
%   picture.sty & tex/latex/oberdiek/picture.sty\\
%   picture.pdf & doc/latex/oberdiek/picture.pdf\\
%   picture-example.tex & doc/latex/oberdiek/picture-example.tex\\
%   picture.dtx & source/latex/oberdiek/picture.dtx\\
% \end{tabular}^^A
% }^^A
% \sbox0{\t}^^A
% \ifdim\wd0>\linewidth
%   \begingroup
%     \advance\linewidth by\leftmargin
%     \advance\linewidth by\rightmargin
%   \edef\x{\endgroup
%     \def\noexpand\lw{\the\linewidth}^^A
%   }\x
%   \def\lwbox{^^A
%     \leavevmode
%     \hbox to \linewidth{^^A
%       \kern-\leftmargin\relax
%       \hss
%       \usebox0
%       \hss
%       \kern-\rightmargin\relax
%     }^^A
%   }^^A
%   \ifdim\wd0>\lw
%     \sbox0{\small\t}^^A
%     \ifdim\wd0>\linewidth
%       \ifdim\wd0>\lw
%         \sbox0{\footnotesize\t}^^A
%         \ifdim\wd0>\linewidth
%           \ifdim\wd0>\lw
%             \sbox0{\scriptsize\t}^^A
%             \ifdim\wd0>\linewidth
%               \ifdim\wd0>\lw
%                 \sbox0{\tiny\t}^^A
%                 \ifdim\wd0>\linewidth
%                   \lwbox
%                 \else
%                   \usebox0
%                 \fi
%               \else
%                 \lwbox
%               \fi
%             \else
%               \usebox0
%             \fi
%           \else
%             \lwbox
%           \fi
%         \else
%           \usebox0
%         \fi
%       \else
%         \lwbox
%       \fi
%     \else
%       \usebox0
%     \fi
%   \else
%     \lwbox
%   \fi
% \else
%   \usebox0
% \fi
% \end{quote}
% If you have a \xfile{docstrip.cfg} that configures and enables \docstrip's
% TDS installing feature, then some files can already be in the right
% place, see the documentation of \docstrip.
%
% \subsection{Refresh file name databases}
%
% If your \TeX~distribution
% (\teTeX, \mikTeX, \dots) relies on file name databases, you must refresh
% these. For example, \teTeX\ users run \verb|texhash| or
% \verb|mktexlsr|.
%
% \subsection{Some details for the interested}
%
% \paragraph{Attached source.}
%
% The PDF documentation on CTAN also includes the
% \xfile{.dtx} source file. It can be extracted by
% AcrobatReader 6 or higher. Another option is \textsf{pdftk},
% e.g. unpack the file into the current directory:
% \begin{quote}
%   \verb|pdftk picture.pdf unpack_files output .|
% \end{quote}
%
% \paragraph{Unpacking with \LaTeX.}
% The \xfile{.dtx} chooses its action depending on the format:
% \begin{description}
% \item[\plainTeX:] Run \docstrip\ and extract the files.
% \item[\LaTeX:] Generate the documentation.
% \end{description}
% If you insist on using \LaTeX\ for \docstrip\ (really,
% \docstrip\ does not need \LaTeX), then inform the autodetect routine
% about your intention:
% \begin{quote}
%   \verb|latex \let\install=y% \iffalse meta-comment
%
% File: picture.dtx
% Version: 2009/10/11 v1.3
% Info: Dimens for picture macros
%
% Copyright (C) 2006-2009 by
%    Heiko Oberdiek <heiko.oberdiek at googlemail.com>
%
% This work may be distributed and/or modified under the
% conditions of the LaTeX Project Public License, either
% version 1.3c of this license or (at your option) any later
% version. This version of this license is in
%    http://www.latex-project.org/lppl/lppl-1-3c.txt
% and the latest version of this license is in
%    http://www.latex-project.org/lppl.txt
% and version 1.3 or later is part of all distributions of
% LaTeX version 2005/12/01 or later.
%
% This work has the LPPL maintenance status "maintained".
%
% This Current Maintainer of this work is Heiko Oberdiek.
%
% This work consists of the main source file picture.dtx
% and the derived files
%    picture.sty, picture.pdf, picture.ins, picture.drv,
%    picture-example.tex.
%
% Distribution:
%    CTAN:macros/latex/contrib/oberdiek/picture.dtx
%    CTAN:macros/latex/contrib/oberdiek/picture.pdf
%
% Unpacking:
%    (a) If picture.ins is present:
%           tex picture.ins
%    (b) Without picture.ins:
%           tex picture.dtx
%    (c) If you insist on using LaTeX
%           latex \let\install=y\input{picture.dtx}
%        (quote the arguments according to the demands of your shell)
%
% Documentation:
%    (a) If picture.drv is present:
%           latex picture.drv
%    (b) Without picture.drv:
%           latex picture.dtx; ...
%    The class ltxdoc loads the configuration file ltxdoc.cfg
%    if available. Here you can specify further options, e.g.
%    use A4 as paper format:
%       \PassOptionsToClass{a4paper}{article}
%
%    Programm calls to get the documentation (example):
%       pdflatex picture.dtx
%       makeindex -s gind.ist picture.idx
%       pdflatex picture.dtx
%       makeindex -s gind.ist picture.idx
%       pdflatex picture.dtx
%
% Installation:
%    TDS:tex/latex/oberdiek/picture.sty
%    TDS:doc/latex/oberdiek/picture.pdf
%    TDS:doc/latex/oberdiek/picture-example.tex
%    TDS:source/latex/oberdiek/picture.dtx
%
%<*ignore>
\begingroup
  \catcode123=1 %
  \catcode125=2 %
  \def\x{LaTeX2e}%
\expandafter\endgroup
\ifcase 0\ifx\install y1\fi\expandafter
         \ifx\csname processbatchFile\endcsname\relax\else1\fi
         \ifx\fmtname\x\else 1\fi\relax
\else\csname fi\endcsname
%</ignore>
%<*install>
\input docstrip.tex
\Msg{************************************************************************}
\Msg{* Installation}
\Msg{* Package: picture 2009/10/11 v1.3 Dimens for picture macros (HO)}
\Msg{************************************************************************}

\keepsilent
\askforoverwritefalse

\let\MetaPrefix\relax
\preamble

This is a generated file.

Project: picture
Version: 2009/10/11 v1.3

Copyright (C) 2006-2009 by
   Heiko Oberdiek <heiko.oberdiek at googlemail.com>

This work may be distributed and/or modified under the
conditions of the LaTeX Project Public License, either
version 1.3c of this license or (at your option) any later
version. This version of this license is in
   http://www.latex-project.org/lppl/lppl-1-3c.txt
and the latest version of this license is in
   http://www.latex-project.org/lppl.txt
and version 1.3 or later is part of all distributions of
LaTeX version 2005/12/01 or later.

This work has the LPPL maintenance status "maintained".

This Current Maintainer of this work is Heiko Oberdiek.

This work consists of the main source file picture.dtx
and the derived files
   picture.sty, picture.pdf, picture.ins, picture.drv,
   picture-example.tex.

\endpreamble
\let\MetaPrefix\DoubleperCent

\generate{%
  \file{picture.ins}{\from{picture.dtx}{install}}%
  \file{picture.drv}{\from{picture.dtx}{driver}}%
  \usedir{tex/latex/oberdiek}%
  \file{picture.sty}{\from{picture.dtx}{package}}%
  \usedir{doc/latex/oberdiek}%
  \file{picture-example.tex}{\from{picture.dtx}{example}}%
  \nopreamble
  \nopostamble
  \usedir{source/latex/oberdiek/catalogue}%
  \file{picture.xml}{\from{picture.dtx}{catalogue}}%
}

\catcode32=13\relax% active space
\let =\space%
\Msg{************************************************************************}
\Msg{*}
\Msg{* To finish the installation you have to move the following}
\Msg{* file into a directory searched by TeX:}
\Msg{*}
\Msg{*     picture.sty}
\Msg{*}
\Msg{* To produce the documentation run the file `picture.drv'}
\Msg{* through LaTeX.}
\Msg{*}
\Msg{* Happy TeXing!}
\Msg{*}
\Msg{************************************************************************}

\endbatchfile
%</install>
%<*ignore>
\fi
%</ignore>
%<*driver>
\NeedsTeXFormat{LaTeX2e}
\ProvidesFile{picture.drv}%
  [2009/10/11 v1.3 Dimens for picture macros (HO)]%
\documentclass{ltxdoc}
\usepackage{holtxdoc}[2011/11/22]
\begin{document}
  \DocInput{picture.dtx}%
\end{document}
%</driver>
% \fi
%
% \CheckSum{322}
%
% \CharacterTable
%  {Upper-case    \A\B\C\D\E\F\G\H\I\J\K\L\M\N\O\P\Q\R\S\T\U\V\W\X\Y\Z
%   Lower-case    \a\b\c\d\e\f\g\h\i\j\k\l\m\n\o\p\q\r\s\t\u\v\w\x\y\z
%   Digits        \0\1\2\3\4\5\6\7\8\9
%   Exclamation   \!     Double quote  \"     Hash (number) \#
%   Dollar        \$     Percent       \%     Ampersand     \&
%   Acute accent  \'     Left paren    \(     Right paren   \)
%   Asterisk      \*     Plus          \+     Comma         \,
%   Minus         \-     Point         \.     Solidus       \/
%   Colon         \:     Semicolon     \;     Less than     \<
%   Equals        \=     Greater than  \>     Question mark \?
%   Commercial at \@     Left bracket  \[     Backslash     \\
%   Right bracket \]     Circumflex    \^     Underscore    \_
%   Grave accent  \`     Left brace    \{     Vertical bar  \|
%   Right brace   \}     Tilde         \~}
%
% \GetFileInfo{picture.drv}
%
% \title{The \xpackage{picture} package}
% \date{2009/10/11 v1.3}
% \author{Heiko Oberdiek\\\xemail{heiko.oberdiek at googlemail.com}}
%
% \maketitle
%
% \begin{abstract}
% There are macro and environment arguments that expect numbers
% that will internally be multiplicated with \cs{unitlength}.
% This package extends the syntax of these arguments that
% dimens with calculation support can be added for these arguments.
% \end{abstract}
%
% \tableofcontents
%
% \section{User interface}
%
% \subsection{Introduction}
%
% The environment |picture| and macros such as \cs{put}, \cs{line},
% \cs{vector} and other macros have arguments that expect numbers
% that are used as factor for \cs{unitlength}.
% This package redefines such macros and adds code that detects
% whether such an argument is given as number or as length.
% In the latter case, the length is used directly without multiplying
% with \cs{unitlength}.
%
% \subsection{Options}
%
% Depending on the available features, also length expressions
% can be given. Option \xoption{calc} loads package \xpackage{calc}.
% Then expressions of these package may be used.
% Otherwise \xoption{etex} wraps the length argument inside
% \cs{dimexpr}\dots\cs{relax}, if \eTeX\ is available. Otherwise
% option \xoption{plain} uses plain assignments without calculation
% support.
%
% The default is \xoption{calc} if package \xpackage{calc}
% is loaded before package \xoption{picture}. If you specify
% option \xoption{calc} the loading of \xpackage{calc} is ensured.
% Otherwise package \xpackage{picture} looks whether \cs{dimexpr}
% is available and uses then option \xoption{etex} as default.
% If \eTeX\ also could not be found, then \xoption{plain} is used.
%
% \subsection{Example}
%
%    \begin{macrocode}
%<*example>
\documentclass{article}

\usepackage[calc]{picture}

\begin{document}

\setlength{\unitlength}{1pt}

\begin{picture}(\widthof{Hello World}, 10mm)
  \put(0, 0){\makebox(0,0)[lb]{Hello World}}%
  \put(0, \heightof{Hello World} + \fboxsep){%
    \line(1, 0){\widthof{Hello World}}%
  }%
  \put(\widthof{Hello World}, 10mm){%
    \line(0, -1){10mm}%
  }%
\end{picture}

\end{document}
%</example>
%    \end{macrocode}
%
% \subsection{Supported packages}
%
% Packages \xpackage{pspicture} and \xpackage{pict2e} are supported,
% but they must be loaded before package \xpackage{picture}.
%
% New macros can be supported by \cs{picture@redefine}.
% The first argument is the macro which contains the arguments
% in its parameter text that you want to support by package \xpackage{picture}.
% The second argument contains the parameter text. Change \# to \& for
% the arguments in question.
% Examples (already used by package \xpackage{picture}):
% \begin{quote}
% |\picture@redefine\put{(&1,&2)}|\\
% |\picture@redefine\line{(#1,#2)&3}|
% \end{quote}
%
% \StopEventually{
% }
%
% \section{Implementation}
%
% \subsection{Identification}
%
%    \begin{macrocode}
%<*package>
\NeedsTeXFormat{LaTeX2e}
\ProvidesPackage{picture}%
  [2009/10/11 v1.3 Dimens for picture macros (HO)]%
%    \end{macrocode}
%
% \subsection{Options}
%
%    \begin{macrocode}
\def\Pc@calcname{calc}
\def\Pc@etexname{etex}
\def\Pc@plainname{plain}
%    \end{macrocode}
%    \begin{macro}{\Pc@method}
%    Macro \cs{Pc@method} stores the method to use for calculations.
%    Check which features are available and set the default for \cs{Pc@method}.
%    \begin{macrocode}
\@ifpackageloaded{calc}{%
  \let\Pc@method\Pc@calcname
}{%
  \begingroup\expandafter\expandafter\expandafter\endgroup
  \expandafter\ifx\csname dimexpr\endcsname\relax
    \let\Pc@method\Pc@plainname
  \else
    \let\Pc@method\Pc@etexname
  \fi
}
%    \end{macrocode}
%    \end{macro}
%    \begin{macrocode}
\DeclareOption{plain}{%
  \let\Pc@method\Pc@plainname
}
\DeclareOption{etex}{%
  \begingroup\expandafter\expandafter\expandafter\endgroup
  \expandafter\ifx\csname dimexpr\endcsname\relax
    \PackageError{picture}{%
      e-TeX is not available%
    }\@ehc
  \else
    \let\Pc@method\Pc@etexname
  \fi
}
\DeclareOption{calc}{%
  \let\Pc@method\Pc@calcname
}
\ProcessOptions*
\begingroup
  \let\on@line\@empty
  \PackageInfo{picture}{Calculation method: \Pc@method}%
\endgroup
%    \end{macrocode}
%
% \subsection{Calculation method}
%
%    \begin{macrocode}
\ifx\Pc@method\Pc@calcname
  \RequirePackage{calc}%
\fi
%    \end{macrocode}
%
% \subsubsection{Method \xoption{calc}}
%
%    \begin{macrocode}
\ifx\Pc@method\Pc@calcname
  \def\Pc@tokslength#1{%
    \begingroup
      \let\calc@error\Pc@calc@error
      \setlength\dimen@{#1\unitlength}\Pc@next\Pc@nil{#1}%
  }%
  \let\PcOrg@calc@error\calc@error
  \@ifpackagelater{calc}{2007/08/22}{% v4.3
    \def\Pc@calc@error#1{%
      \expandafter\ifx\expandafter\unitlength\noexpand#1\relax
        \def\calc@next##1!{%
          \endgroup
          \aftergroup\afterassignment
          \aftergroup\Pc@next
        }%
        \expandafter\@firstoftwo
      \else
        \expandafter\@secondoftwo
      \fi
      {%
        \calc@next{#1}%
      }{%
        \PcOrg@calc@error{#1}%
      }%
    }%
  }{%
    \def\Pc@calc@error#1{%
      \expandafter\ifx\expandafter\unitlength\noexpand#1\relax
        \def\calc@next##1!{%
          \endgroup
          \aftergroup\afterassignment
          \aftergroup\Pc@next
        }%
        \expandafter\@gobble
      \else
        \expandafter\@firstofone
      \fi
      {%
        \PcOrg@calc@error{#1}%
      }%
    }%
  }%
\fi
%    \end{macrocode}
%
% \subsubsection{Method \xoption{etex}}
%
%    \begin{macrocode}
\ifx\Pc@method\Pc@etexname
  \def\Pc@tokslength#1{%
    \begingroup
      \afterassignment\Pc@next
      \dimen@=\dimexpr#1\unitlength\Pc@nil{#1}%
  }%
\fi
%    \end{macrocode}
%
% \subsubsection{Method \xoption{plain}}
%
%    \begin{macrocode}
\ifx\Pc@method\Pc@plainname
  \def\Pc@tokslength#1{%
    \begingroup
      \afterassignment\Pc@next
      \dimen@=#1\unitlength\Pc@nil{#1}%
  }%
\fi
%    \end{macrocode}
%
% \subsubsection{Help macros}
%
%    \begin{macrocode}
\def\Pc@next#1\Pc@nil#2{%
  \ifx\\#1\\%
    \endgroup
    \Pc@addtoks{{#2}}%
  \else
    \expandafter\endgroup
    \expandafter\Pc@addtoks\expandafter{%
      \expandafter{\the\dimen@\@gobble}%
    }%
  \fi
}
%    \end{macrocode}
%    \begin{macro}{\Pc@nil}
%    \cs{Pc@nil} must not have the meaning of \cs{relax} because of
%    \cs{dimexpr}.
%    \begin{macrocode}
\let\Pc@nil\message
%    \end{macrocode}
%    \end{macro}
%    \begin{macro}{\Pc@addtoks}
%    \begin{macrocode}
\def\Pc@addtoks#1{%
  \toks@=\expandafter{\the\toks@#1}%
}
%    \end{macrocode}
%    \end{macro}
%    \begin{macro}{\Pc@init}
%    \begin{macrocode}
\def\Pc@init#1{%
  \begingroup
    \toks@={#1}%
}
%    \end{macrocode}
%    \end{macro}
%    \begin{macro}{\Pc@finish}
%    \begin{macrocode}
\def\Pc@finish#1{%
  \expandafter\endgroup
  \expandafter#1\the\toks@
}
%    \end{macrocode}
%    \end{macro}
%
% \subsection{Redefinitions}
%
%    \begin{macro}{\picture@redefine}
%    \begin{tabular}[t]{@{}l@{ }l@{}}
%    |#1|:& command name\\
%    |#2|:& parameter text, length parameter with \& instead of \#
%    \end{tabular}
%    \begin{macrocode}
\def\picture@redefine#1#2{%
  \begingroup
    \edef\reserved@a{%
      \noexpand\noexpand
      \expandafter\noexpand
          \csname PcOrg@\expandafter\@gobble\string#1\endcsname
    }%
    \toks0{#1}%
    \Pc@first#2&0%
}
%    \end{macrocode}
%    \end{macro}
%    \begin{macro}{\Pc@first}
%    \begin{macrocode}
\def\Pc@first#1&{%
  \toks1={#1}%
  \toks2={\Pc@init{#1}}%
  \Pc@scanlength
}
%    \end{macrocode}
%    \end{macro}
%    \begin{macro}{\Pc@scanlength}
%    |#1|: number of length parameter or zero
%    \begin{macrocode}
\def\Pc@scanlength#1{%
  \ifcase#1 %
    \expandafter\Pc@last
  \else
    \toks1=\expandafter{\the\toks1 ###1}%
    \toks2=\expandafter{\the\toks2 \Pc@tokslength{###1}}%
    \expandafter\Pc@scannext
  \fi
}
%    \end{macrocode}
%    \end{macro}
%    \begin{macro}{\Pc@scannext}
%    \begin{macrocode}
\def\Pc@scannext#1&{%
  \ifx\\#1\\%
  \else
    \toks1=\expandafter{\the\toks1 #1}%
    \toks2=\expandafter{\the\toks2 \Pc@addtoks{#1}}%
  \fi
  \Pc@scanlength
}
%    \end{macrocode}
%    \end{macro}
%    \begin{macro}{\Pc@last}
%    \begin{macrocode}
\def\Pc@last{%
  \edef\x{%
    \endgroup
    \let\reserved@a\the\toks0 %
    \def\the\toks0 \the\toks1 {%
      \the\toks2 %
      \noexpand\Pc@finish\reserved@a
    }%
  }%
  \x
}
%    \end{macrocode}
%    \end{macro}
%
% \subsubsection{\LaTeX\ base macros}
%
%    \begin{macrocode}
\picture@redefine\@picture{(&1,&2)(&3,&4)}
\picture@redefine\put{(&1,&2)}
\picture@redefine\multiput{(&1,&2)}
\picture@redefine\@multiput{(&1,&2)}
\picture@redefine\line{(#1,#2)&3}
\picture@redefine\vector{(#1,#2)&3}
\picture@redefine\dashbox{&1(&2,&3)}
\picture@redefine\@circle{&1}
\picture@redefine\@dot{&1}
\picture@redefine\@bezier{#1(&2,&3)(&4,&5)(&6,&7)}
\picture@redefine\@imakepicbox{(&1,&2)}
%    \end{macrocode}
%
% \subsubsection{Package \xpackage{pspicture}}
%
%     Package \xpackage{pspicture} changes the signature of \cs{@oval}
%     by adding an optional argument.
%    \begin{macrocode}
\@ifpackageloaded{pspicture}{%
  \picture@redefine\@oval{[&1](&2,&3)}%
  \picture@redefine\Line{(&1,&2)}%
  \picture@redefine\Curve{(&1,&2)}%
  \picture@redefine\Vector{(&1,&2)}%
}{%
  \picture@redefine\@oval{(&1,&2)}%
}
%    \end{macrocode}
%
% \subsection{Check package loading order}
%
%    \begin{macro}{\PC@checkpackage}
%    \begin{macrocode}
\def\Pc@checkpackage#1{%
  \@ifpackageloaded{#1}{%
  }{%
    \AtBeginDocument{%
      \@ifpackageloaded{#1}{%
        \PackageWarningNoLine{picture}{%
          Package `#1' is loaded after `picture'.\MessageBreak
          Load package `picture' afterwards to get full support%
          \MessageBreak
          of its additional syntax with length specifications%
        }%
      }{}%
    }%
  }%
}
%    \end{macrocode}
%    \end{macro}
%    \begin{macrocode}
\Pc@checkpackage{pict2e}
\Pc@checkpackage{pspicture}
%    \end{macrocode}
%
%    \begin{macrocode}
%</package>
%    \end{macrocode}
%
% \section{Installation}
%
% \subsection{Download}
%
% \paragraph{Package.} This package is available on
% CTAN\footnote{\url{ftp://ftp.ctan.org/tex-archive/}}:
% \begin{description}
% \item[\CTAN{macros/latex/contrib/oberdiek/picture.dtx}] The source file.
% \item[\CTAN{macros/latex/contrib/oberdiek/picture.pdf}] Documentation.
% \end{description}
%
%
% \paragraph{Bundle.} All the packages of the bundle `oberdiek'
% are also available in a TDS compliant ZIP archive. There
% the packages are already unpacked and the documentation files
% are generated. The files and directories obey the TDS standard.
% \begin{description}
% \item[\CTAN{install/macros/latex/contrib/oberdiek.tds.zip}]
% \end{description}
% \emph{TDS} refers to the standard ``A Directory Structure
% for \TeX\ Files'' (\CTAN{tds/tds.pdf}). Directories
% with \xfile{texmf} in their name are usually organized this way.
%
% \subsection{Bundle installation}
%
% \paragraph{Unpacking.} Unpack the \xfile{oberdiek.tds.zip} in the
% TDS tree (also known as \xfile{texmf} tree) of your choice.
% Example (linux):
% \begin{quote}
%   |unzip oberdiek.tds.zip -d ~/texmf|
% \end{quote}
%
% \paragraph{Script installation.}
% Check the directory \xfile{TDS:scripts/oberdiek/} for
% scripts that need further installation steps.
% Package \xpackage{attachfile2} comes with the Perl script
% \xfile{pdfatfi.pl} that should be installed in such a way
% that it can be called as \texttt{pdfatfi}.
% Example (linux):
% \begin{quote}
%   |chmod +x scripts/oberdiek/pdfatfi.pl|\\
%   |cp scripts/oberdiek/pdfatfi.pl /usr/local/bin/|
% \end{quote}
%
% \subsection{Package installation}
%
% \paragraph{Unpacking.} The \xfile{.dtx} file is a self-extracting
% \docstrip\ archive. The files are extracted by running the
% \xfile{.dtx} through \plainTeX:
% \begin{quote}
%   \verb|tex picture.dtx|
% \end{quote}
%
% \paragraph{TDS.} Now the different files must be moved into
% the different directories in your installation TDS tree
% (also known as \xfile{texmf} tree):
% \begin{quote}
% \def\t{^^A
% \begin{tabular}{@{}>{\ttfamily}l@{ $\rightarrow$ }>{\ttfamily}l@{}}
%   picture.sty & tex/latex/oberdiek/picture.sty\\
%   picture.pdf & doc/latex/oberdiek/picture.pdf\\
%   picture-example.tex & doc/latex/oberdiek/picture-example.tex\\
%   picture.dtx & source/latex/oberdiek/picture.dtx\\
% \end{tabular}^^A
% }^^A
% \sbox0{\t}^^A
% \ifdim\wd0>\linewidth
%   \begingroup
%     \advance\linewidth by\leftmargin
%     \advance\linewidth by\rightmargin
%   \edef\x{\endgroup
%     \def\noexpand\lw{\the\linewidth}^^A
%   }\x
%   \def\lwbox{^^A
%     \leavevmode
%     \hbox to \linewidth{^^A
%       \kern-\leftmargin\relax
%       \hss
%       \usebox0
%       \hss
%       \kern-\rightmargin\relax
%     }^^A
%   }^^A
%   \ifdim\wd0>\lw
%     \sbox0{\small\t}^^A
%     \ifdim\wd0>\linewidth
%       \ifdim\wd0>\lw
%         \sbox0{\footnotesize\t}^^A
%         \ifdim\wd0>\linewidth
%           \ifdim\wd0>\lw
%             \sbox0{\scriptsize\t}^^A
%             \ifdim\wd0>\linewidth
%               \ifdim\wd0>\lw
%                 \sbox0{\tiny\t}^^A
%                 \ifdim\wd0>\linewidth
%                   \lwbox
%                 \else
%                   \usebox0
%                 \fi
%               \else
%                 \lwbox
%               \fi
%             \else
%               \usebox0
%             \fi
%           \else
%             \lwbox
%           \fi
%         \else
%           \usebox0
%         \fi
%       \else
%         \lwbox
%       \fi
%     \else
%       \usebox0
%     \fi
%   \else
%     \lwbox
%   \fi
% \else
%   \usebox0
% \fi
% \end{quote}
% If you have a \xfile{docstrip.cfg} that configures and enables \docstrip's
% TDS installing feature, then some files can already be in the right
% place, see the documentation of \docstrip.
%
% \subsection{Refresh file name databases}
%
% If your \TeX~distribution
% (\teTeX, \mikTeX, \dots) relies on file name databases, you must refresh
% these. For example, \teTeX\ users run \verb|texhash| or
% \verb|mktexlsr|.
%
% \subsection{Some details for the interested}
%
% \paragraph{Attached source.}
%
% The PDF documentation on CTAN also includes the
% \xfile{.dtx} source file. It can be extracted by
% AcrobatReader 6 or higher. Another option is \textsf{pdftk},
% e.g. unpack the file into the current directory:
% \begin{quote}
%   \verb|pdftk picture.pdf unpack_files output .|
% \end{quote}
%
% \paragraph{Unpacking with \LaTeX.}
% The \xfile{.dtx} chooses its action depending on the format:
% \begin{description}
% \item[\plainTeX:] Run \docstrip\ and extract the files.
% \item[\LaTeX:] Generate the documentation.
% \end{description}
% If you insist on using \LaTeX\ for \docstrip\ (really,
% \docstrip\ does not need \LaTeX), then inform the autodetect routine
% about your intention:
% \begin{quote}
%   \verb|latex \let\install=y\input{picture.dtx}|
% \end{quote}
% Do not forget to quote the argument according to the demands
% of your shell.
%
% \paragraph{Generating the documentation.}
% You can use both the \xfile{.dtx} or the \xfile{.drv} to generate
% the documentation. The process can be configured by the
% configuration file \xfile{ltxdoc.cfg}. For instance, put this
% line into this file, if you want to have A4 as paper format:
% \begin{quote}
%   \verb|\PassOptionsToClass{a4paper}{article}|
% \end{quote}
% An example follows how to generate the
% documentation with pdf\LaTeX:
% \begin{quote}
%\begin{verbatim}
%pdflatex picture.dtx
%makeindex -s gind.ist picture.idx
%pdflatex picture.dtx
%makeindex -s gind.ist picture.idx
%pdflatex picture.dtx
%\end{verbatim}
% \end{quote}
%
% \section{Catalogue}
%
% The following XML file can be used as source for the
% \href{http://mirror.ctan.org/help/Catalogue/catalogue.html}{\TeX\ Catalogue}.
% The elements \texttt{caption} and \texttt{description} are imported
% from the original XML file from the Catalogue.
% The name of the XML file in the Catalogue is \xfile{picture.xml}.
%    \begin{macrocode}
%<*catalogue>
<?xml version='1.0' encoding='us-ascii'?>
<!DOCTYPE entry SYSTEM 'catalogue.dtd'>
<entry datestamp='$Date$' modifier='$Author$' id='picture'>
  <name>picture</name>
  <caption>Dimens for picture macros.</caption>
  <authorref id='auth:oberdiek'/>
  <copyright owner='Heiko Oberdiek' year='2006-2009'/>
  <license type='lppl1.3'/>
  <version number='1.3'/>
  <description>
    There are macro and environment arguments that expect numbers
    that will internally be multiplied by <tt>\unitlength</tt>.
    This package extends the syntax of these arguments, so that
    dimensions with calculation support may be used for these arguments.
    <p/>
    The package is part of the <xref refid='oberdiek'>oberdiek</xref> bundle.
  </description>
  <documentation details='Package documentation'
      href='ctan:/macros/latex/contrib/oberdiek/picture.pdf'/>
  <ctan file='true' path='/macros/latex/contrib/oberdiek/picture.dtx'/>
  <miktex location='oberdiek'/>
  <texlive location='oberdiek'/>
  <install path='/macros/latex/contrib/oberdiek/oberdiek.tds.zip'/>
</entry>
%</catalogue>
%    \end{macrocode}
%
% \begin{History}
%   \begin{Version}{2006/08/26 v1.0}
%   \item
%     First released version. (First start of the project was June/July 2002.)
%   \end{Version}
%   \begin{Version}{2007/04/11 v1.1}
%   \item
%     Line ends sanitized.
%   \end{Version}
%   \begin{Version}{2008/11/26 v1.2}
%   \item
%     Package \xpackage{pict2e} added to documentation section ``Supported
%     packages''.
%   \item
%     Package order of supported packages is checked.
%   \end{Version}
%   \begin{Version}{2009/10/11 v1.3}
%   \item
%     Fix because of new version v4.3 of package \xpackage{calc}.
%   \end{Version}
% \end{History}
%
% \PrintIndex
%
% \Finale
\endinput
|
% \end{quote}
% Do not forget to quote the argument according to the demands
% of your shell.
%
% \paragraph{Generating the documentation.}
% You can use both the \xfile{.dtx} or the \xfile{.drv} to generate
% the documentation. The process can be configured by the
% configuration file \xfile{ltxdoc.cfg}. For instance, put this
% line into this file, if you want to have A4 as paper format:
% \begin{quote}
%   \verb|\PassOptionsToClass{a4paper}{article}|
% \end{quote}
% An example follows how to generate the
% documentation with pdf\LaTeX:
% \begin{quote}
%\begin{verbatim}
%pdflatex picture.dtx
%makeindex -s gind.ist picture.idx
%pdflatex picture.dtx
%makeindex -s gind.ist picture.idx
%pdflatex picture.dtx
%\end{verbatim}
% \end{quote}
%
% \section{Catalogue}
%
% The following XML file can be used as source for the
% \href{http://mirror.ctan.org/help/Catalogue/catalogue.html}{\TeX\ Catalogue}.
% The elements \texttt{caption} and \texttt{description} are imported
% from the original XML file from the Catalogue.
% The name of the XML file in the Catalogue is \xfile{picture.xml}.
%    \begin{macrocode}
%<*catalogue>
<?xml version='1.0' encoding='us-ascii'?>
<!DOCTYPE entry SYSTEM 'catalogue.dtd'>
<entry datestamp='$Date$' modifier='$Author$' id='picture'>
  <name>picture</name>
  <caption>Dimens for picture macros.</caption>
  <authorref id='auth:oberdiek'/>
  <copyright owner='Heiko Oberdiek' year='2006-2009'/>
  <license type='lppl1.3'/>
  <version number='1.3'/>
  <description>
    There are macro and environment arguments that expect numbers
    that will internally be multiplied by <tt>\unitlength</tt>.
    This package extends the syntax of these arguments, so that
    dimensions with calculation support may be used for these arguments.
    <p/>
    The package is part of the <xref refid='oberdiek'>oberdiek</xref> bundle.
  </description>
  <documentation details='Package documentation'
      href='ctan:/macros/latex/contrib/oberdiek/picture.pdf'/>
  <ctan file='true' path='/macros/latex/contrib/oberdiek/picture.dtx'/>
  <miktex location='oberdiek'/>
  <texlive location='oberdiek'/>
  <install path='/macros/latex/contrib/oberdiek/oberdiek.tds.zip'/>
</entry>
%</catalogue>
%    \end{macrocode}
%
% \begin{History}
%   \begin{Version}{2006/08/26 v1.0}
%   \item
%     First released version. (First start of the project was June/July 2002.)
%   \end{Version}
%   \begin{Version}{2007/04/11 v1.1}
%   \item
%     Line ends sanitized.
%   \end{Version}
%   \begin{Version}{2008/11/26 v1.2}
%   \item
%     Package \xpackage{pict2e} added to documentation section ``Supported
%     packages''.
%   \item
%     Package order of supported packages is checked.
%   \end{Version}
%   \begin{Version}{2009/10/11 v1.3}
%   \item
%     Fix because of new version v4.3 of package \xpackage{calc}.
%   \end{Version}
% \end{History}
%
% \PrintIndex
%
% \Finale
\endinput
|
% \end{quote}
% Do not forget to quote the argument according to the demands
% of your shell.
%
% \paragraph{Generating the documentation.}
% You can use both the \xfile{.dtx} or the \xfile{.drv} to generate
% the documentation. The process can be configured by the
% configuration file \xfile{ltxdoc.cfg}. For instance, put this
% line into this file, if you want to have A4 as paper format:
% \begin{quote}
%   \verb|\PassOptionsToClass{a4paper}{article}|
% \end{quote}
% An example follows how to generate the
% documentation with pdf\LaTeX:
% \begin{quote}
%\begin{verbatim}
%pdflatex picture.dtx
%makeindex -s gind.ist picture.idx
%pdflatex picture.dtx
%makeindex -s gind.ist picture.idx
%pdflatex picture.dtx
%\end{verbatim}
% \end{quote}
%
% \section{Catalogue}
%
% The following XML file can be used as source for the
% \href{http://mirror.ctan.org/help/Catalogue/catalogue.html}{\TeX\ Catalogue}.
% The elements \texttt{caption} and \texttt{description} are imported
% from the original XML file from the Catalogue.
% The name of the XML file in the Catalogue is \xfile{picture.xml}.
%    \begin{macrocode}
%<*catalogue>
<?xml version='1.0' encoding='us-ascii'?>
<!DOCTYPE entry SYSTEM 'catalogue.dtd'>
<entry datestamp='$Date$' modifier='$Author$' id='picture'>
  <name>picture</name>
  <caption>Dimens for picture macros.</caption>
  <authorref id='auth:oberdiek'/>
  <copyright owner='Heiko Oberdiek' year='2006-2009'/>
  <license type='lppl1.3'/>
  <version number='1.3'/>
  <description>
    There are macro and environment arguments that expect numbers
    that will internally be multiplied by <tt>\unitlength</tt>.
    This package extends the syntax of these arguments, so that
    dimensions with calculation support may be used for these arguments.
    <p/>
    The package is part of the <xref refid='oberdiek'>oberdiek</xref> bundle.
  </description>
  <documentation details='Package documentation'
      href='ctan:/macros/latex/contrib/oberdiek/picture.pdf'/>
  <ctan file='true' path='/macros/latex/contrib/oberdiek/picture.dtx'/>
  <miktex location='oberdiek'/>
  <texlive location='oberdiek'/>
  <install path='/macros/latex/contrib/oberdiek/oberdiek.tds.zip'/>
</entry>
%</catalogue>
%    \end{macrocode}
%
% \begin{History}
%   \begin{Version}{2006/08/26 v1.0}
%   \item
%     First released version. (First start of the project was June/July 2002.)
%   \end{Version}
%   \begin{Version}{2007/04/11 v1.1}
%   \item
%     Line ends sanitized.
%   \end{Version}
%   \begin{Version}{2008/11/26 v1.2}
%   \item
%     Package \xpackage{pict2e} added to documentation section ``Supported
%     packages''.
%   \item
%     Package order of supported packages is checked.
%   \end{Version}
%   \begin{Version}{2009/10/11 v1.3}
%   \item
%     Fix because of new version v4.3 of package \xpackage{calc}.
%   \end{Version}
% \end{History}
%
% \PrintIndex
%
% \Finale
\endinput
|
% \end{quote}
% Do not forget to quote the argument according to the demands
% of your shell.
%
% \paragraph{Generating the documentation.}
% You can use both the \xfile{.dtx} or the \xfile{.drv} to generate
% the documentation. The process can be configured by the
% configuration file \xfile{ltxdoc.cfg}. For instance, put this
% line into this file, if you want to have A4 as paper format:
% \begin{quote}
%   \verb|\PassOptionsToClass{a4paper}{article}|
% \end{quote}
% An example follows how to generate the
% documentation with pdf\LaTeX:
% \begin{quote}
%\begin{verbatim}
%pdflatex picture.dtx
%makeindex -s gind.ist picture.idx
%pdflatex picture.dtx
%makeindex -s gind.ist picture.idx
%pdflatex picture.dtx
%\end{verbatim}
% \end{quote}
%
% \section{Catalogue}
%
% The following XML file can be used as source for the
% \href{http://mirror.ctan.org/help/Catalogue/catalogue.html}{\TeX\ Catalogue}.
% The elements \texttt{caption} and \texttt{description} are imported
% from the original XML file from the Catalogue.
% The name of the XML file in the Catalogue is \xfile{picture.xml}.
%    \begin{macrocode}
%<*catalogue>
<?xml version='1.0' encoding='us-ascii'?>
<!DOCTYPE entry SYSTEM 'catalogue.dtd'>
<entry datestamp='$Date$' modifier='$Author$' id='picture'>
  <name>picture</name>
  <caption>Dimens for picture macros.</caption>
  <authorref id='auth:oberdiek'/>
  <copyright owner='Heiko Oberdiek' year='2006-2009'/>
  <license type='lppl1.3'/>
  <version number='1.3'/>
  <description>
    There are macro and environment arguments that expect numbers
    that will internally be multiplied by <tt>\unitlength</tt>.
    This package extends the syntax of these arguments, so that
    dimensions with calculation support may be used for these arguments.
    <p/>
    The package is part of the <xref refid='oberdiek'>oberdiek</xref> bundle.
  </description>
  <documentation details='Package documentation'
      href='ctan:/macros/latex/contrib/oberdiek/picture.pdf'/>
  <ctan file='true' path='/macros/latex/contrib/oberdiek/picture.dtx'/>
  <miktex location='oberdiek'/>
  <texlive location='oberdiek'/>
  <install path='/macros/latex/contrib/oberdiek/oberdiek.tds.zip'/>
</entry>
%</catalogue>
%    \end{macrocode}
%
% \begin{History}
%   \begin{Version}{2006/08/26 v1.0}
%   \item
%     First released version. (First start of the project was June/July 2002.)
%   \end{Version}
%   \begin{Version}{2007/04/11 v1.1}
%   \item
%     Line ends sanitized.
%   \end{Version}
%   \begin{Version}{2008/11/26 v1.2}
%   \item
%     Package \xpackage{pict2e} added to documentation section ``Supported
%     packages''.
%   \item
%     Package order of supported packages is checked.
%   \end{Version}
%   \begin{Version}{2009/10/11 v1.3}
%   \item
%     Fix because of new version v4.3 of package \xpackage{calc}.
%   \end{Version}
% \end{History}
%
% \PrintIndex
%
% \Finale
\endinput

% \iffalse meta-comment
%
% File: embedfile.dtx
% Version: 2011/04/13 v2.6
% Info: Embed files into PDF
%
% Copyright (C) 2006-2011 by
%    Heiko Oberdiek <heiko.oberdiek at googlemail.com>
%
% This work may be distributed and/or modified under the
% conditions of the LaTeX Project Public License, either
% version 1.3c of this license or (at your option) any later
% version. This version of this license is in
%    http://www.latex-project.org/lppl/lppl-1-3c.txt
% and the latest version of this license is in
%    http://www.latex-project.org/lppl.txt
% and version 1.3 or later is part of all distributions of
% LaTeX version 2005/12/01 or later.
%
% This work has the LPPL maintenance status "maintained".
%
% This Current Maintainer of this work is Heiko Oberdiek.
%
% The Base Interpreter refers to any `TeX-Format',
% because some files are installed in TDS:tex/generic//.
%
% This work consists of the main source file embedfile.dtx
% and the derived files
%    embedfile.sty, embedfile.pdf, embedfile.ins, embedfile.drv,
%    dtx-attach.sty, embedfile-example-plain.tex,
%    embedfile-example-collection.tex, embedfile-test1.tex,
%    embedfile-test2.tex, embedfile-test3.tex,
%    embedfile-test4.tex.
%
% Distribution:
%    CTAN:macros/latex/contrib/oberdiek/embedfile.dtx
%    CTAN:macros/latex/contrib/oberdiek/embedfile.pdf
%
% Unpacking:
%    (a) If embedfile.ins is present:
%           tex embedfile.ins
%    (b) Without embedfile.ins:
%           tex embedfile.dtx
%    (c) If you insist on using LaTeX
%           latex \let\install=y% \iffalse meta-comment
%
% File: embedfile.dtx
% Version: 2011/04/13 v2.6
% Info: Embed files into PDF
%
% Copyright (C) 2006-2011 by
%    Heiko Oberdiek <heiko.oberdiek at googlemail.com>
%
% This work may be distributed and/or modified under the
% conditions of the LaTeX Project Public License, either
% version 1.3c of this license or (at your option) any later
% version. This version of this license is in
%    http://www.latex-project.org/lppl/lppl-1-3c.txt
% and the latest version of this license is in
%    http://www.latex-project.org/lppl.txt
% and version 1.3 or later is part of all distributions of
% LaTeX version 2005/12/01 or later.
%
% This work has the LPPL maintenance status "maintained".
%
% This Current Maintainer of this work is Heiko Oberdiek.
%
% The Base Interpreter refers to any `TeX-Format',
% because some files are installed in TDS:tex/generic//.
%
% This work consists of the main source file embedfile.dtx
% and the derived files
%    embedfile.sty, embedfile.pdf, embedfile.ins, embedfile.drv,
%    dtx-attach.sty, embedfile-example-plain.tex,
%    embedfile-example-collection.tex, embedfile-test1.tex,
%    embedfile-test2.tex, embedfile-test3.tex,
%    embedfile-test4.tex.
%
% Distribution:
%    CTAN:macros/latex/contrib/oberdiek/embedfile.dtx
%    CTAN:macros/latex/contrib/oberdiek/embedfile.pdf
%
% Unpacking:
%    (a) If embedfile.ins is present:
%           tex embedfile.ins
%    (b) Without embedfile.ins:
%           tex embedfile.dtx
%    (c) If you insist on using LaTeX
%           latex \let\install=y% \iffalse meta-comment
%
% File: embedfile.dtx
% Version: 2011/04/13 v2.6
% Info: Embed files into PDF
%
% Copyright (C) 2006-2011 by
%    Heiko Oberdiek <heiko.oberdiek at googlemail.com>
%
% This work may be distributed and/or modified under the
% conditions of the LaTeX Project Public License, either
% version 1.3c of this license or (at your option) any later
% version. This version of this license is in
%    http://www.latex-project.org/lppl/lppl-1-3c.txt
% and the latest version of this license is in
%    http://www.latex-project.org/lppl.txt
% and version 1.3 or later is part of all distributions of
% LaTeX version 2005/12/01 or later.
%
% This work has the LPPL maintenance status "maintained".
%
% This Current Maintainer of this work is Heiko Oberdiek.
%
% The Base Interpreter refers to any `TeX-Format',
% because some files are installed in TDS:tex/generic//.
%
% This work consists of the main source file embedfile.dtx
% and the derived files
%    embedfile.sty, embedfile.pdf, embedfile.ins, embedfile.drv,
%    dtx-attach.sty, embedfile-example-plain.tex,
%    embedfile-example-collection.tex, embedfile-test1.tex,
%    embedfile-test2.tex, embedfile-test3.tex,
%    embedfile-test4.tex.
%
% Distribution:
%    CTAN:macros/latex/contrib/oberdiek/embedfile.dtx
%    CTAN:macros/latex/contrib/oberdiek/embedfile.pdf
%
% Unpacking:
%    (a) If embedfile.ins is present:
%           tex embedfile.ins
%    (b) Without embedfile.ins:
%           tex embedfile.dtx
%    (c) If you insist on using LaTeX
%           latex \let\install=y% \iffalse meta-comment
%
% File: embedfile.dtx
% Version: 2011/04/13 v2.6
% Info: Embed files into PDF
%
% Copyright (C) 2006-2011 by
%    Heiko Oberdiek <heiko.oberdiek at googlemail.com>
%
% This work may be distributed and/or modified under the
% conditions of the LaTeX Project Public License, either
% version 1.3c of this license or (at your option) any later
% version. This version of this license is in
%    http://www.latex-project.org/lppl/lppl-1-3c.txt
% and the latest version of this license is in
%    http://www.latex-project.org/lppl.txt
% and version 1.3 or later is part of all distributions of
% LaTeX version 2005/12/01 or later.
%
% This work has the LPPL maintenance status "maintained".
%
% This Current Maintainer of this work is Heiko Oberdiek.
%
% The Base Interpreter refers to any `TeX-Format',
% because some files are installed in TDS:tex/generic//.
%
% This work consists of the main source file embedfile.dtx
% and the derived files
%    embedfile.sty, embedfile.pdf, embedfile.ins, embedfile.drv,
%    dtx-attach.sty, embedfile-example-plain.tex,
%    embedfile-example-collection.tex, embedfile-test1.tex,
%    embedfile-test2.tex, embedfile-test3.tex,
%    embedfile-test4.tex.
%
% Distribution:
%    CTAN:macros/latex/contrib/oberdiek/embedfile.dtx
%    CTAN:macros/latex/contrib/oberdiek/embedfile.pdf
%
% Unpacking:
%    (a) If embedfile.ins is present:
%           tex embedfile.ins
%    (b) Without embedfile.ins:
%           tex embedfile.dtx
%    (c) If you insist on using LaTeX
%           latex \let\install=y\input{embedfile.dtx}
%        (quote the arguments according to the demands of your shell)
%
% Documentation:
%    (a) If embedfile.drv is present:
%           latex embedfile.drv
%    (b) Without embedfile.drv:
%           latex embedfile.dtx; ...
%    The class ltxdoc loads the configuration file ltxdoc.cfg
%    if available. Here you can specify further options, e.g.
%    use A4 as paper format:
%       \PassOptionsToClass{a4paper}{article}
%
%    Programm calls to get the documentation (example):
%       pdflatex embedfile.dtx
%       makeindex -s gind.ist embedfile.idx
%       pdflatex embedfile.dtx
%       makeindex -s gind.ist embedfile.idx
%       pdflatex embedfile.dtx
%
% Installation:
%    TDS:tex/generic/oberdiek/embedfile.sty
%    TDS:tex/generic/oberdiek/dtx-attach.sty
%    TDS:doc/latex/oberdiek/embedfile.pdf
%    TDS:doc/latex/oberdiek/embedfile-example-plain.tex
%    TDS:doc/latex/oberdiek/embedfile-example-collection.tex
%    TDS:doc/latex/oberdiek/test/embedfile-test1.tex
%    TDS:doc/latex/oberdiek/test/embedfile-test2.tex
%    TDS:doc/latex/oberdiek/test/embedfile-test3.tex
%    TDS:doc/latex/oberdiek/test/embedfile-test4.tex
%    TDS:source/latex/oberdiek/embedfile.dtx
%
%<*ignore>
\begingroup
  \catcode123=1 %
  \catcode125=2 %
  \def\x{LaTeX2e}%
\expandafter\endgroup
\ifcase 0\ifx\install y1\fi\expandafter
         \ifx\csname processbatchFile\endcsname\relax\else1\fi
         \ifx\fmtname\x\else 1\fi\relax
\else\csname fi\endcsname
%</ignore>
%<*install>
\input docstrip.tex
\Msg{************************************************************************}
\Msg{* Installation}
\Msg{* Package: embedfile 2011/04/13 v2.6 Embed files into PDF (HO)}
\Msg{************************************************************************}

\keepsilent
\askforoverwritefalse

\let\MetaPrefix\relax
\preamble

This is a generated file.

Project: embedfile
Version: 2011/04/13 v2.6

Copyright (C) 2006-2011 by
   Heiko Oberdiek <heiko.oberdiek at googlemail.com>

This work may be distributed and/or modified under the
conditions of the LaTeX Project Public License, either
version 1.3c of this license or (at your option) any later
version. This version of this license is in
   http://www.latex-project.org/lppl/lppl-1-3c.txt
and the latest version of this license is in
   http://www.latex-project.org/lppl.txt
and version 1.3 or later is part of all distributions of
LaTeX version 2005/12/01 or later.

This work has the LPPL maintenance status "maintained".

This Current Maintainer of this work is Heiko Oberdiek.

The Base Interpreter refers to any `TeX-Format',
because some files are installed in TDS:tex/generic//.

This work consists of the main source file embedfile.dtx
and the derived files
   embedfile.sty, embedfile.pdf, embedfile.ins, embedfile.drv,
   dtx-attach.sty, embedfile-example-plain.tex,
   embedfile-example-collection.tex, embedfile-test1.tex,
   embedfile-test2.tex, embedfile-test3.tex,
   embedfile-test4.tex.

\endpreamble
\let\MetaPrefix\DoubleperCent

\generate{%
  \file{embedfile.ins}{\from{embedfile.dtx}{install}}%
  \file{embedfile.drv}{\from{embedfile.dtx}{driver}}%
  \usedir{tex/generic/oberdiek}%
  \file{embedfile.sty}{\from{embedfile.dtx}{package}}%
  \usedir{tex/latex/oberdiek}%
  \file{dtx-attach.sty}{\from{embedfile.dtx}{dtxattach}}%
  \usedir{doc/latex/oberdiek}%
  \file{embedfile-example-plain.tex}{\from{embedfile.dtx}{exampleplain}}%
  \file{embedfile-example-collection.tex}{\from{embedfile.dtx}{examplecollection}}%
  \usedir{doc/latex/oberdiek/test}%
  \file{embedfile-test1.tex}{\from{embedfile.dtx}{test1}}%
  \file{embedfile-test2.tex}{\from{embedfile.dtx}{test2}}%
  \file{embedfile-test3.tex}{\from{embedfile.dtx}{test3}}%
  \file{embedfile-test4.tex}{\from{embedfile.dtx}{test4}}%
  \nopreamble
  \nopostamble
  \usedir{source/latex/oberdiek/catalogue}%
  \file{embedfile.xml}{\from{embedfile.dtx}{catalogue}}%
}

\catcode32=13\relax% active space
\let =\space%
\Msg{************************************************************************}
\Msg{*}
\Msg{* To finish the installation you have to move the following}
\Msg{* files into a directory searched by TeX:}
\Msg{*}
\Msg{*     embedfile.sty, dtx-attach.sty}
\Msg{*}
\Msg{* To produce the documentation run the file `embedfile.drv'}
\Msg{* through LaTeX.}
\Msg{*}
\Msg{* Happy TeXing!}
\Msg{*}
\Msg{************************************************************************}

\endbatchfile
%</install>
%<*ignore>
\fi
%</ignore>
%<*driver>
\NeedsTeXFormat{LaTeX2e}
\ProvidesFile{embedfile.drv}%
  [2011/04/13 v2.6 Embed files into PDF (HO)]%
\documentclass{ltxdoc}
\usepackage{holtxdoc}[2011/11/22]
\begin{document}
  \DocInput{embedfile.dtx}%
\end{document}
%</driver>
% \fi
%
% \CheckSum{1363}
%
% \CharacterTable
%  {Upper-case    \A\B\C\D\E\F\G\H\I\J\K\L\M\N\O\P\Q\R\S\T\U\V\W\X\Y\Z
%   Lower-case    \a\b\c\d\e\f\g\h\i\j\k\l\m\n\o\p\q\r\s\t\u\v\w\x\y\z
%   Digits        \0\1\2\3\4\5\6\7\8\9
%   Exclamation   \!     Double quote  \"     Hash (number) \#
%   Dollar        \$     Percent       \%     Ampersand     \&
%   Acute accent  \'     Left paren    \(     Right paren   \)
%   Asterisk      \*     Plus          \+     Comma         \,
%   Minus         \-     Point         \.     Solidus       \/
%   Colon         \:     Semicolon     \;     Less than     \<
%   Equals        \=     Greater than  \>     Question mark \?
%   Commercial at \@     Left bracket  \[     Backslash     \\
%   Right bracket \]     Circumflex    \^     Underscore    \_
%   Grave accent  \`     Left brace    \{     Vertical bar  \|
%   Right brace   \}     Tilde         \~}
%
% \GetFileInfo{embedfile.drv}
%
% \title{The \xpackage{embedfile} package}
% \date{2011/04/13 v2.6}
% \author{Heiko Oberdiek\\\xemail{heiko.oberdiek at googlemail.com}}
%
% \maketitle
%
% \begin{abstract}
% This package embeds files to a PDF document.
% Currently the only supported driver is \pdfTeX\ $>=$ 1.30 in PDF mode.
% \end{abstract}
%
% \tableofcontents
%
% \section{Documentation}
%
% \subsection{Introduction}
%
%    The PDF format (\cite{pdfspec}) allows the inclusion of files
%    inside the PDF document. The included files can be bound to an
%    annotation on a page. Or they can be recorded in a sorted
%    list of embedded files. The packages \xpackage{attachfile}
%    or \xpackage{attachfile2} follow the first approach, this
%    package uses the latter method.
%
% \subsubsection{Future development}
%
%    My dream is a large package that merges the features of
%    all these packages meantioned before:
%    \begin{itemize}
%    \item Files can be attached to a page.
%    \item Files can be attached to the document.
%    \item An easy user interface for simple, common tasks and
%          beginners.
%    \item An interface for the advanced users that want to setup
%          every detail.
%    \item Support of many drivers (pdftex, dvips, dvipdfm, \dots).
%    \item \dots
%    \end{itemize}
%    However, I have not managed to take the time for this project.
%    Instead:
%    \begin{itemize}
%    \item First I experimented with package \xpackage{attachfile},
%          adding driver support, fixing bugs, \dots. The result is
%          currently named as \xpackage{attachfile2}. It uses an external
%          script to get file properties (size, date, checksum, \dots).
%    \item In order to avoid an external program for getting basic
%          file properties I provided a patch ``EscapeAndOther'' for
%          pdfTeX that was accepted for version 1.30.
%    \item This package closes a gap left by the packages for attaching
%          files and allows the embedding of files to the document.
%          Also it makes use of the new primitives of \pdfTeX.
%    \end{itemize}
%
% \subsection{User interface}
%
%    This package \xpackage{embedfile} can be used with both
%    \LaTeX\ and \plainTeX. See \autoref{sec:plain} that
%    explains the use with \plainTeX\ by an example.
%    In \LaTeX\ the package is loaded as usually. There are no options.
%    \begin{quote}
%      \verb|\usepackage{embedfile}|
%    \end{quote}
%
%    \begin{declcs}{embedfile}
%      |[|\meta{options}|]| \M{file}
%    \end{declcs}
%    The macro \cs{embedfile} includes file \meta{file} and
%    attaches it to the PDF document. At the end of the document
%    the sorted list of embedded files are written. Thus you can
%    safely use \cs{embedfile} before \verb|\end{document}|.
%    Embedding files using \cs{AtEndDocument} will only work,
%    if \cs{AtEndDocument} is called before loading the package
%    \xpackage{embedfile}.
%
%    The \meta{options} are give as key value pairs. The following
%    keys are supported:
%    \begin{description}
%    \item[\xoption{filespec}]
%      This allows to override the file name
%      that appears in the PDF file. If you are using other than
%      simple file names (8-bit, path separators, \dots), look into
%      the PDF specification (\cite{pdfspec}).
%      There are rules how these file names must be written/encoded.
%      Avoid 8-bit characters
%      and other special characters, the behaviour is currently undefined.
%      Use option \xoption{ucfilespec} for more funny file names.
%      The string method, see below, is |escape| since version 2.4.
%
%      This name is also used as entry in a name tree
%      (see PDF specification: |/EmbeddedFiles|). Therefore the
%      value for \xoption{filespec} must be unique among all
%      embedded files. Also key \xoption{initialfiles} refers
%      to this name, if the file name and the value of \xoption{filespec}
%      are different.
%    \item[\xoption{ucfilespec}]
%      Since PDF 1.7 the file name may be provided in Unicode.
%      The conversion of the option value into a PDF string
%      is controlled by option \xoption{stringmethod}.
%    \item[\xoption{filesystem}]
%      This sets the entry \verb|/FS| in
%      the file specification dictionary, see PDF specification
%      (\cite{pdfspec}).
%      Example: \verb|filesystem=URL|.
%    \item[\xoption{mimetype}]
%      This sets the mime type (\cite{mime}) of the file,
%      see \autoref{sec:plain} for examples and \cite{types}
%      for a list of officially registered types.
%    \item[\xoption{desc}]
%      The description for the file.
%    \item[\xoption{stringmethod}]
%      The package must convert the values of the keys \xoption{ucfilespec}
%      and \xoption{desc} into a PDF string (before version 2.4: \xoption{filespec}
%      and \xoption{desc}). If \xpackage{hyperref}
%      is found, then its \cs{pdfstringdef} will be used, otherwise
%      \pdfTeX's \cs{pdfescapestring} is used. Value \verb|psd|
%      forces the use of \cs{pdfstringdef}, value \verb|escape|
%      the use of \cs{pdfescapestring}.
%    \item[\xoption{\meta{key}.value}] Sets the value of a collection
%      item property, see section \ref{sec:collection}.
%    \item[\xoption{\meta{key}.prefix}]
%      Sets the prefix of a collection item property,
%      see section \ref{sec:collection}.
%    \item[\xoption{id}]
%      The value must be an unique name. Macros
%      \cs{embedfileifobjectexists} and \cs{embedfilegetobject}
%      are using this name later.
%    \end{description}
%
%    \begin{declcs}{embedfilefinish}
%    \end{declcs}
%    The list of all embedded files must be added as data structure
%    in the PDF file. In case of \LaTeX\ this is automatically done.
%    The package uses \cs{AtEndDocument}. Then the list of all
%    files should be known. However, \plainTeX\ does not know
%    about \cs{AtEndDocument}. Thus the user must call
%    \cs{embedfilefinish} at the end of the document
%    after the last file is embedded.
%
%    \begin{declcs}{embedfilesetup} \M{options}
%    \end{declcs}
%    Options for \cs{embedfile} and collection support
%    can be set in \cs{embedfilesetup}.
%
% \subsection{Collection support (PDF 1.7)}
% \label{sec:collection}
%
%    Since PDF 1.7 the embedded files can form a \emph{collection}
%    (sometimes referred as \emph{package}), the main document
%    is called \emph{cover sheet}. See PDF specification
%    8.2.4 ``Collections'' and 3.10.5 ``Collection items'' \cite{pdfspec}.
%
%    Usually Acrobat Reader 7 or 8 shows the
%    embedded files in a table at the bottom with the following columns:
%    \begin{quote}
%    \begin{tabular}{@{}l|l|l|l@{}}
%       Name & Description & Modified & Size\\
%       \hline
%       \dots & \dots & \dots & \dots\\
%    \end{tabular}
%    \end{quote}
%    Acrobat Reader 10 shows the embeded files in the left panel
%    and adds a new column for the compressed size.
%
%    If the files form a collection, then they are displayed
%    in a table left or top (depending on option \xoption{view},
%    see \cs{embedfilesetup}).
%
%
%    Collection support is enabled automatically, if
%    it is used.
%
%    \begin{declcs}{embedfilesetup} \M{options}
%    \end{declcs}
%    The following options are supported in addition
%    to options for \cs{embedfile}:
%    \begin{description}
%    \item[\xoption{view}]
%      If the PDF file contains a collection, then Acrobat Reader 8
%      shows a line at the top below the menu bar and the toolbar.
%      It shows the current selected file, icons for changing the
%      view mode, an options menu.
%      The initial mode how the collection is presented is
%      set by this option \xoption{view}.
%      The following modes/values are supported, the default is |details|:
%        \begin{description}
%        \item[\texttt{details}] The full collection table is displayed
%          at the top below the collection bar.
%        \item[\texttt{tile}] The files of the collection are shown
%          in tile mode on the left.
%        \item[\texttt{hidden}] The collection table is not shown.
%        \end{description}
%    \item[\xoption{initialfile}]
%      Selects the file that is initially presented. Especially
%      useful for an embedded PDF file that is then shown instead
%      of the cover document. There must be an \cs{embedfile} command
%      somewhere whose value for key \xoption{filespec} is used here.
%      The \cs{embedfile} command can drop option \xoption{filespec}
%      if the file name is not different.
%    \end{description}
%
%    \begin{declcs}{embedfilefield} \M{key} \M{options}
%    \end{declcs}
%    Macro \cs{embedfilefield} defines a column/field in the collection table.
%    The name of the field is \meta{key}.
%    \begin{description}
%    \item[\xoption{type}]
%      sets the type of the field. The supported values are:
%      \begin{description}
%      \item[\texttt{text}] A text field. Its value is set in
%        \cs{embedfile} by option \xoption{\meta{key}.value}.
%      \item[\texttt{date}] A date field. Its value is set in
%        \cs{embedfile} by option \xoption{\meta{key}.value}.
%        A special format is required, see ``3.8.3 Dates'' \cite{pdfspec}.
%      \item[\texttt{number}] A field with an integer or float number.
%        Its value is set in \cs{embedfile} by option
%        \xoption{\meta{key}.value}.
%      \item[\texttt{file}] The file name of the embedded file.
%      \item[\texttt{desc}] The description text of the embedded file. It is
%       set in \cs{embedfile} by option \xoption{desc}.
%      \item[\texttt{moddate}] The modification date of the embedded file.
%      \item[\texttt{size}] The size of the embedded file.
%      \end{description}
%      All types allow the use of a prefix that is disregarded by
%      sorting. The prefix for this field is set in \cs{embedfile} by
%      option \xoption{\meta{key}.prefix}.
%    \item[\xoption{title}] sets the column title.
%    \item[\xoption{visible}] controls whether the column is presented:
%      \begin{description}
%      \item[\texttt{true}] shows the column.
%      \item[\texttt{false}] hides the column.
%      \end{description}
%      Default: \texttt{true}
%    \item[\xoption{edit}]
%      Allows the editing of field values. Does not seem to have an
%      effect for Acrobat Reader.
%      \begin{description}
%      \item[\texttt{true}] enables the feature, if available (depends
%        on the PDF viewer).
%      \item[\texttt{false}] disables the feature.
%      \end{description}
%      Default: \texttt{false}
%    \end{description}
%    The order of \cs{embedfilefield} statements defines the order
%    of the columns.
%
%    \begin{declcs}{embedfilesort} \M{key-sort-list}
%    \end{declcs}
%    The sort order of the embedded files are controlled by
%    macro \cs{embedfilesort}. \meta{key-sort-list} defines
%    the sort order. The key is a field name defined by \cs{embedfilefield}.
%    Its value is either \texttt{ascending} or \texttt{descending}.
%    The default is \texttt{ascending}.
%
% \subsection{Export of object references}
%
%    Caution: This feature is still experimental. It may be even removed
%    in future versions. Therefore feedback would be nice, if someone
%    has a useful application for this feature.
%
%    Object numbers are saved, if \xoption{id} is given in \cs{embedfile}.
%    The following objects are supported:
%    \begin{itemize}
%    \item \texttt{EmbeddedFile}
%    \item \texttt{Filespec}
%    \end{itemize}
%
%    \begin{declcs}{embedfileifobjectexists} \M{id} \M{type}
%      \M{then} \M{else}
%    \end{declcs}
%    Macro \cs{embedfileifobjectexists} tests whether object of \meta{type}
%    is available for the embedded file identified by \meta{id}.
%
%    \begin{declcs}{embedfilegetobject} \M{id} \M{type}
%    \end{declcs}
%    Macro \cs{embedfilegetobject} expands to the full object reference
%    object of \meta{type} for the embedded file identified by \meta{id}.
%
% \subsubsection{Example}
%
%\begin{quote}
%\begin{verbatim}
%\embedfile[id={foo}]{foo.pdf}
%\embedfileifobjectexists{foo}{Filespec}{%
%  \typeout{%
%    FileSpec object for `foo': %
%    \embedfilegetobject{foo}{Filespec}%
%  }%
%}{%
%  \typeout{No Filespec object for `foo'}%
%}
%\end{verbatim}
%\end{quote}
%
% \subsection{Examples}
% \label{sec:examples}
%
% \subsubsection{\plainTeX}
% \label{sec:plain}
%
%    The package can be used with \plainTeX. It can be used
%    with or without help from \xfile{miniltx.tex}.
%
%    If additionally package \xpackage{keyval} (\xpackage{graphicx})
%    is needed, load it first. Then package \xpackage{embedfile} avoids
%    a duplicate loading of package \xpackage{keyval}.
%
%    Because \plainTeX\ does not provide a hook at end of the document,
%    you have to call \cs{embedfilefinish} manually at the end
%    after the last embedded file.
%    \begin{macrocode}
%<*exampleplain>
%<<END
% Load packages
\input miniltx
% \def\Gin@driver{pdftex.def}
% \input graphicx.sty
\input embedfile.sty
\resetatcatcode

% default setting
\embedfilesetup{
  mimetype=text/plain
}

% Embed files
\embedfile[
  filespec=example.tex,
  desc={Source code (plain-TeX) of this example}
]{embedfile-example-plain.tex}

\embedfile[
  desc={Source of package `embedfile'}
]{embedfile.dtx}

\embedfile[
  mimetype=application/pdf,
  desc={Documentation of package `embedfile'}
]{embedfile.pdf}

% Some text
This example document contains three embedded files.

% End of document
\embedfilefinish % don't forget
\bye
%END
%</exampleplain>
%    \end{macrocode}
%
% \subsubsection{Collection example}
%
%    \begin{macrocode}
%<*examplecollection>
%<<END
\NeedsTeXFormat{LaTeX2e}
\documentclass{article}
\usepackage[bookmarks=false]{hyperref}
% provides \pdfstringdef that is then used by `title' and
% other keys.
\usepackage{embedfile}[2011/04/13]
\embedfilesetup{
  view=details,
  initialfile=embedfile.pdf
}
\embedfilefield{file}{
  type=file,
  title={File name}
}
\embedfilefield{description}{
  type=desc,
  title={Description}
}
\embedfilefield{date}{
  type=moddate,
  title={Date}
}
\embedfilefield{size}{
  type=size,
  title={Size}
}
\embedfilefield{type}{
  type=text,
  title={Type},
  visible=false
}
\embedfilesort{
  type,
  date=descending
}
\begin{document}
An example for embedded files as collection.
You need Acrobat Reader 8 or higher.

\embedfile[
  desc={Source file of package `embedfile'},
  description.prefix={Package: },
  type.value={DTX}
]{embedfile.dtx}

\embedfile[
  desc={Documentation of package `embedfile'},
  description.prefix={Package: },
  type.value={PDF}
]{embedfile.pdf}

\embedfile[
  desc={The source for this example},
  description.prefix={Example: },
  type.value={TEX}
]{\jobname.tex}

\end{document}
%END
%</examplecollection>
%    \end{macrocode}
%
% \subsection{Package \xpackage{dtx-attach}}
%
% Package \xpackage{dtx-attach} is just a small application of
% package \xpackage{embedfile}. I am using it for the CTAN
% documentation of my packages in \CTAN{macros/latex/contrib/oberdiek/}.
% It also serves as small example for the use of the package with \LaTeX.
%
%    \begin{macrocode}
%<*dtxattach>
\NeedsTeXFormat{LaTeX2e}
\ProvidesPackage{dtx-attach}
  [2011/04/13 v2.6 Embed \string\jobname.dtx (HO)]%
\RequirePackage{embedfile}[2011/04/13]
\embedfile[%
  stringmethod=escape,%
  mimetype=plain/text,%
  desc={LaTeX docstrip source archive for package `\jobname'}%
]{\jobname.dtx}
%</dtxattach>
%    \end{macrocode}
%
% \StopEventually{
% }
%
% \section{Implementation}
%
%    \begin{macrocode}
%<*package>
%    \end{macrocode}
%
% \subsection{Reload check and package identification}
%    Reload check, especially if the package is not used with \LaTeX.
%    \begin{macrocode}
\begingroup\catcode61\catcode48\catcode32=10\relax%
  \catcode13=5 % ^^M
  \endlinechar=13 %
  \catcode35=6 % #
  \catcode39=12 % '
  \catcode44=12 % ,
  \catcode45=12 % -
  \catcode46=12 % .
  \catcode58=12 % :
  \catcode64=11 % @
  \catcode123=1 % {
  \catcode125=2 % }
  \expandafter\let\expandafter\x\csname ver@embedfile.sty\endcsname
  \ifx\x\relax % plain-TeX, first loading
  \else
    \def\empty{}%
    \ifx\x\empty % LaTeX, first loading,
      % variable is initialized, but \ProvidesPackage not yet seen
    \else
      \expandafter\ifx\csname PackageInfo\endcsname\relax
        \def\x#1#2{%
          \immediate\write-1{Package #1 Info: #2.}%
        }%
      \else
        \def\x#1#2{\PackageInfo{#1}{#2, stopped}}%
      \fi
      \x{embedfile}{The package is already loaded}%
      \aftergroup\endinput
    \fi
  \fi
\endgroup%
%    \end{macrocode}
%    Package identification:
%    \begin{macrocode}
\begingroup\catcode61\catcode48\catcode32=10\relax%
  \catcode13=5 % ^^M
  \endlinechar=13 %
  \catcode35=6 % #
  \catcode39=12 % '
  \catcode40=12 % (
  \catcode41=12 % )
  \catcode44=12 % ,
  \catcode45=12 % -
  \catcode46=12 % .
  \catcode47=12 % /
  \catcode58=12 % :
  \catcode64=11 % @
  \catcode91=12 % [
  \catcode93=12 % ]
  \catcode123=1 % {
  \catcode125=2 % }
  \expandafter\ifx\csname ProvidesPackage\endcsname\relax
    \def\x#1#2#3[#4]{\endgroup
      \immediate\write-1{Package: #3 #4}%
      \xdef#1{#4}%
    }%
  \else
    \def\x#1#2[#3]{\endgroup
      #2[{#3}]%
      \ifx#1\@undefined
        \xdef#1{#3}%
      \fi
      \ifx#1\relax
        \xdef#1{#3}%
      \fi
    }%
  \fi
\expandafter\x\csname ver@embedfile.sty\endcsname
\ProvidesPackage{embedfile}%
  [2011/04/13 v2.6 Embed files into PDF (HO)]%
%    \end{macrocode}
%
% \subsection{Catcodes}
%
%    \begin{macrocode}
\begingroup\catcode61\catcode48\catcode32=10\relax%
  \catcode13=5 % ^^M
  \endlinechar=13 %
  \catcode123=1 % {
  \catcode125=2 % }
  \catcode64=11 % @
  \def\x{\endgroup
    \expandafter\edef\csname EmFi@AtEnd\endcsname{%
      \endlinechar=\the\endlinechar\relax
      \catcode13=\the\catcode13\relax
      \catcode32=\the\catcode32\relax
      \catcode35=\the\catcode35\relax
      \catcode61=\the\catcode61\relax
      \catcode64=\the\catcode64\relax
      \catcode123=\the\catcode123\relax
      \catcode125=\the\catcode125\relax
    }%
  }%
\x\catcode61\catcode48\catcode32=10\relax%
\catcode13=5 % ^^M
\endlinechar=13 %
\catcode35=6 % #
\catcode64=11 % @
\catcode123=1 % {
\catcode125=2 % }
\def\TMP@EnsureCode#1#2{%
  \edef\EmFi@AtEnd{%
    \EmFi@AtEnd
    \catcode#1=\the\catcode#1\relax
  }%
  \catcode#1=#2\relax
}
\TMP@EnsureCode{39}{12}% '
\TMP@EnsureCode{40}{12}% (
\TMP@EnsureCode{41}{12}% )
\TMP@EnsureCode{44}{12}% ,
\TMP@EnsureCode{46}{12}% .
\TMP@EnsureCode{47}{12}% /
\TMP@EnsureCode{58}{12}% :
\TMP@EnsureCode{60}{12}% <
\TMP@EnsureCode{62}{12}% >
\TMP@EnsureCode{91}{12}% [
\TMP@EnsureCode{93}{12}% ]
\TMP@EnsureCode{96}{12}% `
\edef\EmFi@AtEnd{\EmFi@AtEnd\noexpand\endinput}
%    \end{macrocode}
%
% \subsection{Tools}
%
%    \begin{macro}{\EmFi@RequirePackage}
%    \begin{macrocode}
\begingroup\expandafter\expandafter\expandafter\endgroup
\expandafter\ifx\csname RequirePackage\endcsname\relax
  \def\EmFi@RequirePackage#1[#2]{%
    \input #1.sty\relax
  }%
\else
  \let\EmFi@RequirePackage\RequirePackage
\fi
%    \end{macrocode}
%    \end{macro}
%
%    \begin{macro}{\EmFi@Error}
%    \begin{macrocode}
\EmFi@RequirePackage{infwarerr}[2007/09/09]%
\def\EmFi@Error{%
  \@PackageError{embedfile}%
}
%    \end{macrocode}
%    \end{macro}
%
% \subsection{Check for recent \pdfTeX\ in PDF mode}
%
%    Load package \xpackage{ifpdf} and check mode.
%    \begin{macrocode}
\EmFi@RequirePackage{ifpdf}[2007/09/09]
\ifpdf
\else
  \EmFi@Error{%
    Missing pdfTeX in PDF mode%
  }{%
    Currently other drivers are not supported. %
    Package loading is aborted.%
  }%
  \expandafter\EmFi@AtEnd
\fi%
%    \end{macrocode}
%    \begin{macrocode}
\EmFi@RequirePackage{pdftexcmds}[2007/11/11]
\EmFi@RequirePackage{ltxcmds}[2010/03/01]
\EmFi@RequirePackage{kvsetkeys}[2010/03/01]
\EmFi@RequirePackage{kvdefinekeys}[2010/03/01]
%    \end{macrocode}
%    Check version.
%    \begin{macrocode}
\begingroup\expandafter\expandafter\expandafter\endgroup
\expandafter\ifx\csname pdf@filesize\endcsname\relax
  \EmFi@Error{%
    Unsupported pdfTeX version%
  }{%
    At least version 1.30 is necessary. Package loading is aborted.%
  }%
  \expandafter\EmFi@AtEnd
\fi%
%    \end{macrocode}
%
% \subsection{Strings}
%
%    Minimal version of package \xpackage{pdfescape} is 2007/08/27 v1.5
%    because of \cs{EdefSanitize}.
%    \begin{macrocode}
\EmFi@RequirePackage{pdfescape}[2007/11/11]
%    \end{macrocode}
%
%    \begin{macrocode}
\def\EmFi@temp#1{%
  \expandafter\EdefSanitize\csname EmFi@S@#1\endcsname{#1}%
}
%    \end{macrocode}
%    \begin{macro}{\EmFi@details}
%    \begin{macrocode}
\EmFi@temp{details}%
%    \end{macrocode}
%    \end{macro}
%    \begin{macro}{\EmFi@tile}
%    \begin{macrocode}
\EmFi@temp{tile}%
%    \end{macrocode}
%    \end{macro}
%    \begin{macro}{\EmFi@hidden}
%    \begin{macrocode}
\EmFi@temp{hidden}%
%    \end{macrocode}
%    \end{macro}
%    \begin{macro}{\EmFi@S@text}
%    \begin{macrocode}
\EmFi@temp{text}
%    \end{macrocode}
%    \end{macro}
%    \begin{macro}{\EmFi@S@date}
%    \begin{macrocode}
\EmFi@temp{date}
%    \end{macrocode}
%    \end{macro}
%    \begin{macro}{\EmFi@S@number}
%    \begin{macrocode}
\EmFi@temp{number}
%    \end{macrocode}
%    \end{macro}
%    \begin{macro}{\EmFi@S@file}
%    \begin{macrocode}
\EmFi@temp{file}
%    \end{macrocode}
%    \end{macro}
%    \begin{macro}{\EmFi@S@desc}
%    \begin{macrocode}
\EmFi@temp{desc}
%    \end{macrocode}
%    \end{macro}
%    \begin{macro}{\EmFi@S@moddate}
%    \begin{macrocode}
\EmFi@temp{moddate}
%    \end{macrocode}
%    \end{macro}
%    \begin{macro}{\EmFi@S@creationdate}
%    \begin{macrocode}
\EmFi@temp{creationdate}
%    \end{macrocode}
%    \end{macro}
%    \begin{macro}{\EmFi@S@size}
%    \begin{macrocode}
\EmFi@temp{size}
%    \end{macrocode}
%    \end{macro}
%    \begin{macro}{\EmFi@S@ascending}
%    \begin{macrocode}
\EmFi@temp{ascending}
%    \end{macrocode}
%    \end{macro}
%    \begin{macro}{\EmFi@S@descending}
%    \begin{macrocode}
\EmFi@temp{descending}
%    \end{macrocode}
%    \end{macro}
%    \begin{macro}{\EmFi@S@true}
%    \begin{macrocode}
\EmFi@temp{true}
%    \end{macrocode}
%    \end{macro}
%    \begin{macro}{\EmFi@S@false}
%    \begin{macrocode}
\EmFi@temp{false}
%    \end{macrocode}
%    \end{macro}
%
% \subsection{Switches}
%
%    \begin{macro}{\ifEmFi@collection}
%    \begin{macrocode}
\ltx@newif\ifEmFi@collection
%    \end{macrocode}
%    \end{macro}
%    \begin{macro}{\ifEmFi@sort}
%    \begin{macrocode}
\ltx@newif\ifEmFi@sort
%    \end{macrocode}
%    \end{macro}
%    \begin{macro}{\ifEmFi@visible}
%    \begin{macrocode}
\ltx@newif\ifEmFi@visible
%    \end{macrocode}
%    \end{macro}
%    \begin{macro}{\ifEmFi@edit}
%    \begin{macrocode}
\ltx@newif\ifEmFi@edit
%    \end{macrocode}
%    \end{macro}
%    \begin{macro}{\ifEmFi@item}
%    \begin{macrocode}
\ltx@newif\ifEmFi@item
%    \end{macrocode}
%    \end{macro}
%    \begin{macro}{\ifEmFi@finished}
%    \begin{macrocode}
\ltx@newif\ifEmFi@finished
%    \end{macrocode}
%    \end{macro}
%    \begin{macro}{\ifEmFi@id}
%    \begin{macrocode}
\ltx@newif\ifEmFi@id
%    \end{macrocode}
%    \end{macro}
%
% \subsection{Key value definitions}
%
%    \begin{macro}{\EmFi@GlobalKey}
%    \begin{macrocode}
\def\EmFi@GlobalKey#1#2{%
  \global\expandafter\let\csname KV@#1@#2\expandafter\endcsname
                         \csname KV@#1@#2\endcsname
}
%    \end{macrocode}
%    \end{macro}
%    \begin{macro}{\EmFi@GlobalDefaultKey}
%    \begin{macrocode}
\def\EmFi@GlobalDefaultKey#1#2{%
  \EmFi@GlobalKey{#1}{#2}%
  \global\expandafter\let
      \csname KV@#1@#2@default\expandafter\endcsname
      \csname KV@#1@#2@default\endcsname
}
%    \end{macrocode}
%    \end{macro}
%
%    \begin{macro}{\EmFi@DefineKey}
%    \begin{macrocode}
\def\EmFi@DefineKey#1#2{%
  \kv@define@key{EmFi}{#1}{%
    \expandafter\def\csname EmFi@#1\endcsname{##1}%
  }%
  \expandafter\def\csname EmFi@#1\endcsname{#2}%
}
%    \end{macrocode}
%    \end{macro}
%    Subtype of the embedded file (optional).
%    \begin{macrocode}
\EmFi@DefineKey{mimetype}{}
%    \end{macrocode}
%    File specification string.
%    \begin{macrocode}
\EmFi@DefineKey{filespec}{\EmFi@file}
%    \end{macrocode}
%    File specification string in Unicode.
%    \begin{macrocode}
\EmFi@DefineKey{ucfilespec}{}
%    \end{macrocode}
%    File system (optional).
%    \begin{macrocode}
\EmFi@DefineKey{filesystem}{}
%    \end{macrocode}
%    Description (optional).
%    \begin{macrocode}
\EmFi@DefineKey{desc}{}
%    \end{macrocode}
%    Method for converting text to PDF strings.
%    \begin{macrocode}
\EmFi@DefineKey{stringmethod}{%
  \ifx\pdfstringdef\@undefined
    escape%
  \else
    \ifx\pdfstringdef\relax
      escape%
    \else
      psd%
    \fi
  \fi
}
%    \end{macrocode}
%
%    Option \xoption{id} as key for object numbers.
%    \begin{macrocode}
\kv@define@key{EmFi}{id}{%
  \def\EmFi@id{#1}%
  \EmFi@idtrue
}
%    \end{macrocode}
%    \begin{macro}{\EmFi@defobj}
%    \begin{macrocode}
\def\EmFi@defobj#1{%
  \ifEmFi@id
    \expandafter\xdef\csname EmFi@#1@\EmFi@id\endcsname{%
      \the\pdflastobj\ltx@space 0 R%
    }%
  \fi
}
%    \end{macrocode}
%    \end{macro}
%    \begin{macro}{\embedfileifobjectexists}
%    \begin{macrocode}
\def\embedfileifobjectexists#1#2{%
  \expandafter\ifx\csname EmFi@#2@#1\endcsname\relax
    \expandafter\ltx@secondoftwo
  \else
    \expandafter\ltx@firstoftwo
  \fi
}
%    \end{macrocode}
%    \end{macro}
%    \begin{macro}{\embedfilegetobject}
%    \begin{macrocode}
\def\embedfilegetobject#1#2{%
  \embedfileifobjectexists{#1}{#2}{%
    \csname EmFi@#2@#1\endcsname
  }{%
    0 0 R%
  }%
}
%    \end{macrocode}
%    \end{macro}
%
%    Initial view of the collection.
%    \begin{macrocode}
\kv@define@key{EmFi}{view}[]{%
  \EdefSanitize\EmFi@temp{#1}%
  \def\EmFi@next{%
    \global\EmFi@collectiontrue
  }%
  \ifx\EmFi@temp\ltx@empty
    \let\EmFi@view\EmFi@S@details
  \else\ifx\EmFi@temp\EmFi@S@details
    \let\EmFi@view\EmFi@S@details
  \else\ifx\EmFi@temp\EmFi@S@tile
    \let\EmFi@view\EmFi@S@tile
  \else\ifx\EmFi@temp\EmFi@S@hidden
    \let\EmFi@view\EmFi@S@hidden
  \else
    \let\EmFi@next\relax
    \EmFi@Error{%
      Unknown value `\EmFi@temp' for key `view'.\MessageBreak
      Supported values: `details', `tile', `hidden'.%
    }\@ehc
  \fi\fi\fi\fi
  \EmFi@next
}
%    \end{macrocode}
%    \begin{macrocode}
\EmFi@DefineKey{initialfile}{}
%    \end{macrocode}
%
%    \begin{macro}{\embedfilesetup}
%    \begin{macrocode}
\def\embedfilesetup{%
  \ifEmFi@finished
    \def\EmFi@next##1{}%
    \EmFi@Error{%
      \string\embedfilefield\ltx@space after \string\embedfilefinish
    }{%
      The list of embedded files is already written.%
    }%
  \else
    \def\EmFi@next{%
      \kvsetkeys{EmFi}%
    }%
  \fi
  \EmFi@next
}
%    \end{macrocode}
%    \end{macro}
%
%    \begin{macro}{\EmFi@schema}
%    \begin{macrocode}
\def\EmFi@schema{}
%    \end{macrocode}
%    \end{macro}
%    \begin{macro}{\EmFi@order}
%    \begin{macrocode}
\gdef\EmFi@order{0}
%    \end{macrocode}
%    \end{macro}
%    \begin{macro}{\EmFi@@order}
%    \begin{macrocode}
\let\EmFi@@order\relax
%    \end{macrocode}
%    \end{macro}
%    \begin{macro}{\EmFi@fieldlist}
%    \begin{macrocode}
\def\EmFi@fieldlist{}
%    \end{macrocode}
%    \end{macro}
%    \begin{macro}{\EmFi@sortcase}
%    \begin{macrocode}
\def\EmFi@sortcase{0}%
%    \end{macrocode}
%    \end{macro}
%    \begin{macro}{\embedfilefield}
%    \begin{macrocode}
\def\embedfilefield#1#2{%
  \ifEmFi@finished
    \EmFi@Error{%
      \string\embedfilefield\ltx@space after \string\embedfilefinish
    }{%
      The list of embedded files is already written.%
    }%
  \else
    \global\EmFi@collectiontrue
    \EdefSanitize\EmFi@key{#1}%
    \expandafter\ifx\csname KV@EmFi@\EmFi@key.prefix\endcsname\relax
      \begingroup
        \count@=\EmFi@order
        \advance\count@ 1 %
        \xdef\EmFi@order{\the\count@}%
        \let\EmFi@title\EmFi@key
        \let\EmFi@type\EmFi@S@text
        \EmFi@visibletrue
        \EmFi@editfalse
        \kvsetkeys{EmFiFi}{#2}%
        \EmFi@convert\EmFi@title\EmFi@title
        \xdef\EmFi@schema{%
          \EmFi@schema
          /\pdf@escapename{\EmFi@key}<<%
            /Subtype/%
            \ifx\EmFi@type\EmFi@S@date D%
            \else\ifx\EmFi@type\EmFi@S@number N%
            \else\ifx\EmFi@type\EmFi@S@file F%
            \else\ifx\EmFi@type\EmFi@S@desc Desc%
            \else\ifx\EmFi@type\EmFi@S@moddate ModDate%
            \else\ifx\EmFi@type\EmFi@S@creationdate CreationDate%
            \else\ifx\EmFi@type\EmFi@S@size Size%
            \else S%
            \fi\fi\fi\fi\fi\fi\fi
            /N(\EmFi@title)%
            \EmFi@@order{\EmFi@order}%
            \ifEmFi@visible
            \else
              /V false%
            \fi
            \ifEmFi@edit
              /E true%
            \fi
          >>%
        }%
        \let\do\relax
        \xdef\EmFi@fieldlist{%
          \EmFi@fieldlist
          \do{\EmFi@key}%
        }%
        \ifx\EmFi@type\EmFi@S@text
          \kv@define@key{EmFi}{\EmFi@key.value}{%
            \EmFi@itemtrue
            \def\EmFi@temp{##1}%
            \EmFi@convert\EmFi@temp\EmFi@temp
            \expandafter\def\csname EmFi@V@#1%
            \expandafter\endcsname\expandafter{%
              \expandafter(\EmFi@temp)%
            }%
          }%
          \EmFi@GlobalKey{EmFi}{\EmFi@key.value}%
        \else\ifx\EmFi@type\EmFi@S@date
          \kv@define@key{EmFi}{\EmFi@key.value}{%
            \EmFi@itemtrue
            \def\EmFi@temp{##1}%
            \EmFi@convert\EmFi@temp\EmFi@temp
            \expandafter\def\csname EmFi@V@#1%
            \expandafter\endcsname\expandafter{%
              \expandafter(\EmFi@temp)%
            }%
          }%
          \EmFi@GlobalKey{EmFi}{\EmFi@key.value}%
        \else\ifx\EmFi@type\EmFi@S@number
          \kv@define@key{EmFi}{\EmFi@key.value}{%
            \EmFi@itemtrue
            \expandafter\EdefSanitize\csname EmFi@V@#1\endcsname{ ##1}%
          }%
          \EmFi@GlobalKey{EmFi}{\EmFi@key.value}%
        \fi\fi\fi
        \kv@define@key{EmFi}{\EmFi@key.prefix}{%
          \EmFi@itemtrue
          \expandafter\def\csname EmFi@P@#1\endcsname{##1}%
        }%
        \EmFi@GlobalKey{EmFi}{\EmFi@key.prefix}%
        \kv@define@key{EmFiSo}{\EmFi@key}[ascending]{%
          \EdefSanitize\EmFi@temp{##1}%
          \ifx\EmFi@temp\EmFi@S@ascending
            \def\EmFi@temp{true}%
          \else\ifx\EmFi@temp\EmFi@S@descending
            \def\EmFi@temp{false}%
          \else
            \def\EmFi@temp{}%
            \EmFi@Error{%
              Unknown sort order `\EmFi@temp'.\MessageBreak
              Supported values: `\EmFi@S@ascending', %
              `\EmFi@S@descending
            }\@ehc
          \fi\fi
          \ifx\EmFi@temp\ltx@empty
          \else
            \xdef\EmFi@sortkeys{%
              \EmFi@sortkeys
              /\pdf@escapename{#1}%
            }%
            \ifx\EmFi@sortorders\ltx@empty
              \global\let\EmFi@sortorders\EmFi@temp
              \gdef\EmFi@sortcase{1}%
            \else
              \xdef\EmFi@sortorders{%
                \EmFi@sortorders
                \ltx@space
                \EmFi@temp
              }%
              \xdef\EmFi@sortcase{2}%
            \fi
          \fi
        }%
        \EmFi@GlobalDefaultKey{EmFiSo}\EmFi@key
      \endgroup
    \else
      \EmFi@Error{%
        Field `\EmFi@key' is already defined%
      }\@ehc
    \fi
  \fi
}
%    \end{macrocode}
%    \end{macro}
%    \begin{macrocode}
\kv@define@key{EmFiFi}{type}{%
  \EdefSanitize\EmFi@temp{#1}%
  \ifx\EmFi@temp\EmFi@S@text
    \let\EmFi@type\EmFi@temp
  \else\ifx\EmFi@temp\EmFi@S@date
    \let\EmFi@type\EmFi@temp
  \else\ifx\EmFi@temp\EmFi@S@number
    \let\EmFi@type\EmFi@temp
  \else\ifx\EmFi@temp\EmFi@S@file
    \let\EmFi@type\EmFi@temp
  \else\ifx\EmFi@temp\EmFi@S@desc
    \let\EmFi@type\EmFi@temp
  \else\ifx\EmFi@temp\EmFi@S@moddate
    \let\EmFi@type\EmFi@temp
  \else\ifx\EmFi@temp\EmFi@S@creationdate
    \let\EmFi@type\EmFi@temp
  \else\ifx\EmFi@temp\EmFi@S@size
    \let\EmFi@type\EmFi@temp
  \else
    \EmFi@Error{%
      Unknown type `\EmFi@temp'.\MessageBreak
      Supported types: `text', `date', `number', `file',\MessageBreak
      `desc', `moddate', `creationdate', `size'%
    }%
  \fi\fi\fi\fi\fi\fi\fi\fi
}
%    \end{macrocode}
%    \begin{macrocode}
\kv@define@key{EmFiFi}{title}{%
  \def\EmFi@title{#1}%
}
%    \end{macrocode}
%    \begin{macro}{\EmFi@setboolean}
%    \begin{macrocode}
\def\EmFi@setboolean#1#2{%
  \EdefSanitize\EmFi@temp{#2}%
  \ifx\EmFi@temp\EmFi@S@true
    \csname EmFi@#1true\endcsname
  \else
    \ifx\EmFi@temp\EmFi@S@false
      \csname EmFi@#1false\endcsname
    \else
      \EmFi@Error{%
        Unknown value `\EmFi@temp' for key `#1'.\MessageBreak
        Supported values: `true', `false'%
      }\@ehc
    \fi
  \fi
}
%    \end{macrocode}
%    \end{macro}
%    \begin{macrocode}
\kv@define@key{EmFiFi}{visible}[true]{%
  \EmFi@setboolean{visible}{#1}%
}
%    \end{macrocode}
%    \begin{macrocode}
\kv@define@key{EmFiFi}{edit}[true]{%
  \EmFi@setboolean{edit}{#1}%
}
%    \end{macrocode}
%
%    \begin{macro}{\EmFi@sortkeys}
%    \begin{macrocode}
\def\EmFi@sortkeys{}
%    \end{macrocode}
%    \end{macro}
%    \begin{macro}{\EmFi@sortorders}
%    \begin{macrocode}
\def\EmFi@sortorders{}
%    \end{macrocode}
%    \end{macro}
%    \begin{macro}{\embedfilesort}
%    \begin{macrocode}
\def\embedfilesort{%
  \kvsetkeys{EmFiSo}%
}
%    \end{macrocode}
%    \end{macro}
%
% \subsection{Embed the file}
%
%    \begin{macro}{\embedfile}
%    \begin{macrocode}
\def\embedfile{%
  \ltx@ifnextchar[\EmFi@embedfile{\EmFi@embedfile[]}%
}
%    \end{macrocode}
%    \end{macro}
%    \begin{macro}{\EmFi@embedfile}
%    \begin{macrocode}
\def\EmFi@embedfile[#1]#2{%
  \ifEmFi@finished
    \EmFi@Error{%
      \string\embedfile\ltx@space after \string\embedfilefinish
    }{%
      The list of embedded files is already written.%
    }%
  \else
    \begingroup
      \def\EmFi@file{#2}%
      \kvsetkeys{EmFi}{#1}%
      \expandafter\expandafter\expandafter
      \ifx\expandafter\expandafter\expandafter
          \\\pdf@filesize{\EmFi@file}\\%
        \EmFi@Error{%
          File `\EmFi@file' not found%
        }{%
          The unknown file is not embedded.%
        }%
      \else
        \edef\EmFi@@filespec{%
          \pdf@escapestring{\EmFi@filespec}%
        }%
        \ifx\EmFi@ucfilespec\ltx@empty
          \let\EmFi@@ucfilespec\ltx@empty
        \else
          \EmFi@convert\EmFi@ucfilespec\EmFi@@ucfilespec
        \fi
        \ifx\EmFi@desc\ltx@empty
          \let\EmFi@@desc\ltx@empty
        \else
          \EmFi@convert\EmFi@desc\EmFi@@desc
        \fi
        \ifEmFi@item
          \let\do\EmFi@do
          \immediate\pdfobj{%
            <<%
              \EmFi@fieldlist
            >>%
          }%
          \edef\EmFi@ci{\the\pdflastobj}%
        \fi
        \immediate\pdfobj stream attr{%
          /Type/EmbeddedFile%
          \ifx\EmFi@mimetype\ltx@empty
          \else
            /Subtype/\pdf@escapename{\EmFi@mimetype}%
          \fi
          /Params<<%
            /ModDate(\pdf@filemoddate{\EmFi@file})%
            /Size \pdf@filesize{\EmFi@file}%
            /CheckSum<\pdf@filemdfivesum{\EmFi@file}>%
          >>%
        }file{\EmFi@file}\relax
        \EmFi@defobj{EmbeddedFile}%
        \immediate\pdfobj{%
          <<%
            /Type/Filespec%
            \ifx\EmFi@filesystem\ltx@empty
            \else
            /FS/\pdf@escapename{\EmFi@filesystem}%
            \fi
            /F(\EmFi@@filespec)%
            \ifx\EmFi@@ucfilespec\ltx@empty
            \else
              /UF(\EmFi@@ucfilespec)%
            \fi
            \ifx\EmFi@@desc\ltx@empty
            \else
              /Desc(\EmFi@@desc)%
            \fi
            /EF<<%
              /F \the\pdflastobj\ltx@space 0 R%
            >>%
            \ifEmFi@item
              /CI \EmFi@ci\ltx@space 0 R%
            \fi
          >>%
        }%
        \EmFi@defobj{Filespec}%
        \EmFi@add{%
          \EmFi@@filespec
        }{\the\pdflastobj\ltx@space 0 R}%
      \fi
    \endgroup
  \fi
}
%    \end{macrocode}
%    \end{macro}
%    \begin{macro}{\EmFi@do}
%    \begin{macrocode}
\def\EmFi@do#1{%
  \expandafter\ifx\csname EmFi@P@#1\endcsname\relax
    \expandafter\ifx\csname EmFi@V@#1\endcsname\relax
    \else
      /\pdf@escapename{#1}\csname EmFi@V@#1\endcsname
    \fi
  \else
    /\pdf@escapename{#1}<<%
      \expandafter\ifx\csname EmFi@V@#1\endcsname\relax
      \else
        /D\csname EmFi@V@#1\endcsname
      \fi
      /P(\csname EmFi@P@#1\endcsname)%
    >>%
  \fi
}
%    \end{macrocode}
%    \end{macro}
%    \begin{macro}{\EmFi@convert}
%    \begin{macrocode}
\def\EmFi@convert#1#2{%
  \ifnum\pdf@strcmp{\EmFi@stringmethod}{psd}=0 %
    \pdfstringdef\EmFi@temp{#1}%
    \let#2\EmFi@temp
  \else
    \edef#2{\pdf@escapestring{#1}}%
  \fi
}
%    \end{macrocode}
%    \end{macro}
%    \begin{macrocode}
\global\let\EmFi@list\ltx@empty
%    \end{macrocode}
%    \begin{macro}{\EmFi@add}
%    Sorting is done by the insertion sort algorithm.
%    Probably the sorting could be done more reliable.
%    However, the PDF specification is not too clear to me
%    regarding precise sorting rules (how to deal with
%    different encodings, escaped characters, \dots).
%    \begin{macrocode}
\def\EmFi@add#1#2{%
  \begingroup
    \ifx\EmFi@list\ltx@empty
      \xdef\EmFi@list{\noexpand\do{#1}{#2}}%
    \else
      \def\do##1##2{%
        \ifnum\pdf@strcmp{##1}{#1}>0 %
          \edef\x{%
            \toks@{%
              \the\toks@%
              \noexpand\do{#1}{#2}%
              \noexpand\do{##1}{##2}%
            }%
          }%
          \x
          \def\do####1####2{%
            \toks@\expandafter{\the\toks@\do{####1}{####2}}%
          }%
          \def\stop{%
            \xdef\EmFi@list{\the\toks@}%
          }%
        \else
          \toks@\expandafter{\the\toks@\do{##1}{##2}}%
        \fi
      }%
      \def\stop{%
        \xdef\EmFi@list{\the\toks@\noexpand\do{#1}{#2}}%
      }%
      \toks@{}%
      \EmFi@list\stop
    \fi
  \endgroup
}
%    \end{macrocode}
%    \end{macro}
%    \begin{macro}{\embedfilefinish}
%    \begin{macrocode}
\def\embedfilefinish{%
  \ifEmFi@finished
    \EmFi@Error{%
      Too many invocations of \string\embedfilefinish
    }{%
      The list of embedded files is already written.%
    }%
  \else
    \ifx\EmFi@list\ltx@empty
    \else
%    \end{macrocode}
%    Write |/EmbeddedFiles| entry.
%    \begin{macrocode}
      \global\EmFi@finishedtrue
      \begingroup
        \def\do##1##2{%
          (##1)##2%
        }%
        \immediate\pdfobj{%
          <<%
            /Names[\EmFi@list]%
          >>%
        }%
        \pdfnames{%
          /EmbeddedFiles \the\pdflastobj\ltx@space 0 R%
        }%
      \endgroup
%    \end{macrocode}
%    Write collection objects.
%    \begin{macrocode}
      \ifx\EmFi@initialfile\ltx@empty
      \else
        \EmFi@collectiontrue
      \fi
      \ifEmFi@collection
        \ifx\EmFi@initialfile\ltx@empty
          \let\EmFi@@initialfile\ltx@empty
        \else
          \edef\EmFi@@initialfile{%
            \pdf@escapestring{\EmFi@initialfile}%
          }%
        \fi
%    \end{macrocode}
%    Look for initial file among the embedded files.
%    \begin{macrocode}
        \begingroup
          \let\f=N%
          \def\do##1##2{%
            \def\x{##1}%
            \ifx\x\EmFi@@initialfile
              \let\f=Y%
              \let\do\ltx@gobbletwo
            \fi
          }%
          \EmFi@list
        \expandafter\endgroup
        \ifx\f Y%
        \else
          \@PackageWarningNoLine{embedfile}{%
            Missing initial file `\EmFi@initialfile'\MessageBreak
            among the embedded files%
          }%
          \let\EmFi@initialfile\ltx@empty
          \let\EmFi@@initialfile\ltx@empty
        \fi
        \ifcase\EmFi@sortcase
          \def\EmFi@temp{}%
        \or
          \def\EmFi@temp{%
            /S\EmFi@sortkeys
            /A \EmFi@sortorders
          }%
        \else
          \def\EmFi@temp{%
            /S[\EmFi@sortkeys]%
            /A[\EmFi@sortorders]%
          }%
        \fi
        \def\EmFi@@order##1{%
          \ifnum\EmFi@order>1 %
            /O ##1%
          \fi
        }%
        \immediate\pdfobj{%
          <<%
            \ifx\EmFi@schema\ltx@empty
            \else
              /Schema<<\EmFi@schema>>%
            \fi
            \ifx\EmFi@@initialfile\ltx@empty
            \else
              /D(\EmFi@@initialfile)%
            \fi
            \ifx\EmFi@view\EmFi@S@tile
              /View/T%
            \else\ifx\EmFi@view\EmFi@S@hidden
              /View/H%
            \fi\fi
            \ifx\EmFi@temp\ltx@empty
              \EmFi@temp
            \else
              /Sort<<\EmFi@temp>>%
            \fi
          >>%
        }%
        \pdfcatalog{%
          /Collection \the\pdflastobj\ltx@space0 R%
        }%
      \fi
    \fi
  \fi
}
%    \end{macrocode}
%    \end{macro}
%    \begin{macrocode}
\begingroup\expandafter\expandafter\expandafter\endgroup
\expandafter\ifx\csname AtEndDocument\endcsname\relax
\else
  \AtEndDocument{\embedfilefinish}%
\fi
%    \end{macrocode}
%
%    \begin{macrocode}
\EmFi@AtEnd%
%</package>
%    \end{macrocode}
%
% \section{Test}
%
% \subsection{Catcode checks for loading}
%
%    \begin{macrocode}
%<*test1>
%    \end{macrocode}
%    \begin{macrocode}
\catcode`\{=1 %
\catcode`\}=2 %
\catcode`\#=6 %
\catcode`\@=11 %
\expandafter\ifx\csname count@\endcsname\relax
  \countdef\count@=255 %
\fi
\expandafter\ifx\csname @gobble\endcsname\relax
  \long\def\@gobble#1{}%
\fi
\expandafter\ifx\csname @firstofone\endcsname\relax
  \long\def\@firstofone#1{#1}%
\fi
\expandafter\ifx\csname loop\endcsname\relax
  \expandafter\@firstofone
\else
  \expandafter\@gobble
\fi
{%
  \def\loop#1\repeat{%
    \def\body{#1}%
    \iterate
  }%
  \def\iterate{%
    \body
      \let\next\iterate
    \else
      \let\next\relax
    \fi
    \next
  }%
  \let\repeat=\fi
}%
\def\RestoreCatcodes{}
\count@=0 %
\loop
  \edef\RestoreCatcodes{%
    \RestoreCatcodes
    \catcode\the\count@=\the\catcode\count@\relax
  }%
\ifnum\count@<255 %
  \advance\count@ 1 %
\repeat

\def\RangeCatcodeInvalid#1#2{%
  \count@=#1\relax
  \loop
    \catcode\count@=15 %
  \ifnum\count@<#2\relax
    \advance\count@ 1 %
  \repeat
}
\def\RangeCatcodeCheck#1#2#3{%
  \count@=#1\relax
  \loop
    \ifnum#3=\catcode\count@
    \else
      \errmessage{%
        Character \the\count@\space
        with wrong catcode \the\catcode\count@\space
        instead of \number#3%
      }%
    \fi
  \ifnum\count@<#2\relax
    \advance\count@ 1 %
  \repeat
}
\def\space{ }
\expandafter\ifx\csname LoadCommand\endcsname\relax
  \def\LoadCommand{\input embedfile.sty\relax}%
\fi
\def\Test{%
  \RangeCatcodeInvalid{0}{47}%
  \RangeCatcodeInvalid{58}{64}%
  \RangeCatcodeInvalid{91}{96}%
  \RangeCatcodeInvalid{123}{255}%
  \catcode`\@=12 %
  \catcode`\\=0 %
  \catcode`\%=14 %
  \LoadCommand
  \RangeCatcodeCheck{0}{36}{15}%
  \RangeCatcodeCheck{37}{37}{14}%
  \RangeCatcodeCheck{38}{47}{15}%
  \RangeCatcodeCheck{48}{57}{12}%
  \RangeCatcodeCheck{58}{63}{15}%
  \RangeCatcodeCheck{64}{64}{12}%
  \RangeCatcodeCheck{65}{90}{11}%
  \RangeCatcodeCheck{91}{91}{15}%
  \RangeCatcodeCheck{92}{92}{0}%
  \RangeCatcodeCheck{93}{96}{15}%
  \RangeCatcodeCheck{97}{122}{11}%
  \RangeCatcodeCheck{123}{255}{15}%
  \RestoreCatcodes
}
\Test
\csname @@end\endcsname
\end
%    \end{macrocode}
%    \begin{macrocode}
%</test1>
%    \end{macrocode}
%
% \subsection{Simple test}
%
%    \begin{macrocode}
%<*test2>
\input embedfile.sty\relax
\embedfile[%
  stringmethod=escape,%
  mimetype=plain/text,%
  desc={LaTeX docstrip source archive for package `embedfile'},%
  id={embedfile.dtx}%
]{embedfile.dtx}
\nopagenumbers
Test (plain-\TeX): {\tt embedfile.dtx} should be embedded.%

\def\Test#1{%
  \par
  \embedfileifobjectexists{embedfile.dtx}{#1}{%
    Object #1 (embedfile.dtx): %
    \embedfilegetobject{embedfile.dtx}{#1}%
  }{%
    \errmessage{Missing object #1 (embedfile.dtx)}%
  }%
}
\Test{EmbeddedFile}
\Test{Filespec}
\embedfilefinish
\bye
%</test2>
%    \end{macrocode}
%    \begin{macrocode}
%<*test3>
\NeedsTeXFormat{LaTeX2e}
\let\SavedJobname\jobname
\def\jobname{embedfile}
\RequirePackage{dtx-attach}[2011/04/13]
\let\jobname\SavedJobname
\documentclass{minimal}
\begin{document}
  Test (\LaTeX): \texttt{embedfile.dtx} should be embedded.%
\end{document}
%</test3>
%    \end{macrocode}
%
% \subsection{Test for \iniTeX}
%
%    \begin{macrocode}
%<*test4>
\catcode`\{=1 %
\catcode`\}=2 %
\input ifluatex.sty %
\ifluatex
  \directlua{%
    tex.enableprimitives('', {%
      'pdflastobj',%
      'pdfnames',%
      'pdfobj',%
      'pdfoutput'%
    })%
  }%
\fi
\pdfoutput=1 %
\input embedfile.sty %
\shipout\hbox{}
\embedfile[%
  stringmethod=escape,%
  mimetype=plain/text,%
  desc={iniTeX source},%
]{\jobname.tex}
\embedfilefinish
\end
%</test4>
%    \end{macrocode}
%
% \section{Installation}
%
% \subsection{Download}
%
% \paragraph{Package.} This package is available on
% CTAN\footnote{\url{ftp://ftp.ctan.org/tex-archive/}}:
% \begin{description}
% \item[\CTAN{macros/latex/contrib/oberdiek/embedfile.dtx}] The source file.
% \item[\CTAN{macros/latex/contrib/oberdiek/embedfile.pdf}] Documentation.
% \end{description}
%
%
% \paragraph{Bundle.} All the packages of the bundle `oberdiek'
% are also available in a TDS compliant ZIP archive. There
% the packages are already unpacked and the documentation files
% are generated. The files and directories obey the TDS standard.
% \begin{description}
% \item[\CTAN{install/macros/latex/contrib/oberdiek.tds.zip}]
% \end{description}
% \emph{TDS} refers to the standard ``A Directory Structure
% for \TeX\ Files'' (\CTAN{tds/tds.pdf}). Directories
% with \xfile{texmf} in their name are usually organized this way.
%
% \subsection{Bundle installation}
%
% \paragraph{Unpacking.} Unpack the \xfile{oberdiek.tds.zip} in the
% TDS tree (also known as \xfile{texmf} tree) of your choice.
% Example (linux):
% \begin{quote}
%   |unzip oberdiek.tds.zip -d ~/texmf|
% \end{quote}
%
% \paragraph{Script installation.}
% Check the directory \xfile{TDS:scripts/oberdiek/} for
% scripts that need further installation steps.
% Package \xpackage{attachfile2} comes with the Perl script
% \xfile{pdfatfi.pl} that should be installed in such a way
% that it can be called as \texttt{pdfatfi}.
% Example (linux):
% \begin{quote}
%   |chmod +x scripts/oberdiek/pdfatfi.pl|\\
%   |cp scripts/oberdiek/pdfatfi.pl /usr/local/bin/|
% \end{quote}
%
% \subsection{Package installation}
%
% \paragraph{Unpacking.} The \xfile{.dtx} file is a self-extracting
% \docstrip\ archive. The files are extracted by running the
% \xfile{.dtx} through \plainTeX:
% \begin{quote}
%   \verb|tex embedfile.dtx|
% \end{quote}
%
% \paragraph{TDS.} Now the different files must be moved into
% the different directories in your installation TDS tree
% (also known as \xfile{texmf} tree):
% \begin{quote}
% \def\t{^^A
% \begin{tabular}{@{}>{\ttfamily}l@{ $\rightarrow$ }>{\ttfamily}l@{}}
%   embedfile.sty & tex/generic/oberdiek/embedfile.sty\\
%   dtx-attach.sty & tex/generic/oberdiek/dtx-attach.sty\\
%   embedfile.pdf & doc/latex/oberdiek/embedfile.pdf\\
%   embedfile-example-plain.tex & doc/latex/oberdiek/embedfile-example-plain.tex\\
%   embedfile-example-collection.tex & doc/latex/oberdiek/embedfile-example-collection.tex\\
%   test/embedfile-test1.tex & doc/latex/oberdiek/test/embedfile-test1.tex\\
%   test/embedfile-test2.tex & doc/latex/oberdiek/test/embedfile-test2.tex\\
%   test/embedfile-test3.tex & doc/latex/oberdiek/test/embedfile-test3.tex\\
%   test/embedfile-test4.tex & doc/latex/oberdiek/test/embedfile-test4.tex\\
%   embedfile.dtx & source/latex/oberdiek/embedfile.dtx\\
% \end{tabular}^^A
% }^^A
% \sbox0{\t}^^A
% \ifdim\wd0>\linewidth
%   \begingroup
%     \advance\linewidth by\leftmargin
%     \advance\linewidth by\rightmargin
%   \edef\x{\endgroup
%     \def\noexpand\lw{\the\linewidth}^^A
%   }\x
%   \def\lwbox{^^A
%     \leavevmode
%     \hbox to \linewidth{^^A
%       \kern-\leftmargin\relax
%       \hss
%       \usebox0
%       \hss
%       \kern-\rightmargin\relax
%     }^^A
%   }^^A
%   \ifdim\wd0>\lw
%     \sbox0{\small\t}^^A
%     \ifdim\wd0>\linewidth
%       \ifdim\wd0>\lw
%         \sbox0{\footnotesize\t}^^A
%         \ifdim\wd0>\linewidth
%           \ifdim\wd0>\lw
%             \sbox0{\scriptsize\t}^^A
%             \ifdim\wd0>\linewidth
%               \ifdim\wd0>\lw
%                 \sbox0{\tiny\t}^^A
%                 \ifdim\wd0>\linewidth
%                   \lwbox
%                 \else
%                   \usebox0
%                 \fi
%               \else
%                 \lwbox
%               \fi
%             \else
%               \usebox0
%             \fi
%           \else
%             \lwbox
%           \fi
%         \else
%           \usebox0
%         \fi
%       \else
%         \lwbox
%       \fi
%     \else
%       \usebox0
%     \fi
%   \else
%     \lwbox
%   \fi
% \else
%   \usebox0
% \fi
% \end{quote}
% If you have a \xfile{docstrip.cfg} that configures and enables \docstrip's
% TDS installing feature, then some files can already be in the right
% place, see the documentation of \docstrip.
%
% \subsection{Refresh file name databases}
%
% If your \TeX~distribution
% (\teTeX, \mikTeX, \dots) relies on file name databases, you must refresh
% these. For example, \teTeX\ users run \verb|texhash| or
% \verb|mktexlsr|.
%
% \subsection{Some details for the interested}
%
% \paragraph{Attached source.}
%
% The PDF documentation on CTAN also includes the
% \xfile{.dtx} source file. It can be extracted by
% AcrobatReader 6 or higher. Another option is \textsf{pdftk},
% e.g. unpack the file into the current directory:
% \begin{quote}
%   \verb|pdftk embedfile.pdf unpack_files output .|
% \end{quote}
%
% \paragraph{Unpacking with \LaTeX.}
% The \xfile{.dtx} chooses its action depending on the format:
% \begin{description}
% \item[\plainTeX:] Run \docstrip\ and extract the files.
% \item[\LaTeX:] Generate the documentation.
% \end{description}
% If you insist on using \LaTeX\ for \docstrip\ (really,
% \docstrip\ does not need \LaTeX), then inform the autodetect routine
% about your intention:
% \begin{quote}
%   \verb|latex \let\install=y\input{embedfile.dtx}|
% \end{quote}
% Do not forget to quote the argument according to the demands
% of your shell.
%
% \paragraph{Generating the documentation.}
% You can use both the \xfile{.dtx} or the \xfile{.drv} to generate
% the documentation. The process can be configured by the
% configuration file \xfile{ltxdoc.cfg}. For instance, put this
% line into this file, if you want to have A4 as paper format:
% \begin{quote}
%   \verb|\PassOptionsToClass{a4paper}{article}|
% \end{quote}
% An example follows how to generate the
% documentation with pdf\LaTeX:
% \begin{quote}
%\begin{verbatim}
%pdflatex embedfile.dtx
%makeindex -s gind.ist embedfile.idx
%pdflatex embedfile.dtx
%makeindex -s gind.ist embedfile.idx
%pdflatex embedfile.dtx
%\end{verbatim}
% \end{quote}
%
% \section{Catalogue}
%
% The following XML file can be used as source for the
% \href{http://mirror.ctan.org/help/Catalogue/catalogue.html}{\TeX\ Catalogue}.
% The elements \texttt{caption} and \texttt{description} are imported
% from the original XML file from the Catalogue.
% The name of the XML file in the Catalogue is \xfile{embedfile.xml}.
%    \begin{macrocode}
%<*catalogue>
<?xml version='1.0' encoding='us-ascii'?>
<!DOCTYPE entry SYSTEM 'catalogue.dtd'>
<entry datestamp='$Date$' modifier='$Author$' id='embedfile'>
  <name>embedfile</name>
  <caption>Embed files into PDF.</caption>
  <authorref id='auth:oberdiek'/>
  <copyright owner='Heiko Oberdiek' year='2006-2011'/>
  <license type='lppl1.3'/>
  <version number='2.6'/>
  <description>
    This package embeds files in a PDF document, using the PDF
    format's embedding operation (note the contrast with the attach
    operation used by the <xref refid='attachfile'>attachfile</xref>
    and <xref refid='attachfile2'>attachfile2</xref> packages).
    Currently only <xref refid='pdftex'>pdfTeX</xref> &gt;=1.30, in
    PDF mode, is supported.

    <p/>
    The package is part of the <xref refid='oberdiek'>oberdiek</xref>
    bundle.
  </description>
  <documentation details='Package documentation'
      href='ctan:/macros/latex/contrib/oberdiek/embedfile.pdf'/>
  <ctan file='true' path='/macros/latex/contrib/oberdiek/embedfile.dtx'/>
  <miktex location='oberdiek'/>
  <texlive location='oberdiek'/>
  <install path='/macros/latex/contrib/oberdiek/oberdiek.tds.zip'/>
</entry>
%</catalogue>
%    \end{macrocode}
%
% \begin{thebibliography}{9}
% \bibitem{attachfile}
%   Scott Pakin: \textit{The \xpackage{attachfile} package};
%   2005/02/20 v1.2;
%   \CTAN{macros/latex/contrib/attachfile/}.
%
% \bibitem{attachfile2}
%   Heiko Oberdiek: \textit{The \xpackage{attachfile2} package};
%   2006/08/16 v2.2;
%   \CTAN{macros/latex/contrib/oberdiek/attachfile2.pdf}.
%
% \bibitem{pdfspec}
%   Adobe Systems Incorporated:
%   \href{http://www.adobe.com/devnet/acrobat/pdfs/pdf_reference.pdf}%
%       {\textit{PDF Reference, Sixth Edition, Version 1.7}},%
%   Oktober 2006;
%   \url{http://www.adobe.com/devnet/pdf/pdf_reference.html}.
%
% \bibitem{mime}
%   Network Working Group: RFC 2046, \textit{Multipurpose Internet Mail Extensions (MIME) Part Two:
%     Media Types}, November 1996; \url{http://www.rfc-editor.org/}.
%
% \bibitem{types}
%   IANA (Internet Assigned Numbers Authority):
%   \textit{MIME Media Types}, May 2006;
%   \url{http://www.iana.org/assignments/media-types/}.
%
% \end{thebibliography}
%
% \begin{History}
%   \begin{Version}{2006/08/16 v1.0}
%   \item
%     First public version.
%   \end{Version}
%   \begin{Version}{2007/04/11 v1.1}
%   \item
%     Line ends sanitized.
%   \end{Version}
%   \begin{Version}{2007/09/09 v1.2}
%   \item
%     Fixes for plain-TeX, wrapper for package \xpackage{keyval} added.
%   \item
%     Catcode section rewritten.
%   \end{Version}
%   \begin{Version}{2007/10/28 v2.0}
%   \item
%     Collection support added (PDF 1.7).
%   \end{Version}
%   \begin{Version}{2007/10/29 v2.1}
%   \item
%     Export of object references by adding new option \xoption{id} and
%     new macros \cs{embedfileifobjectexists} and \cs{embedfilegetobject}.
%   \end{Version}
%   \begin{Version}{2007/11/11 v2.2}
%   \item
%     Use of package \xpackage{pdftexcmds} for \LuaTeX\ support.
%   \end{Version}
%   \begin{Version}{2007/11/25 v2.3}
%   \item
%     Fix in use of \cs{pdf@filesize}, bug introduced in previous version.
%   \end{Version}
%   \begin{Version}{2009/09/25 v2.4}
%   \item
%     Bug fix: If \xpackage{hyperref} is used with option \xoption{unicode},
%     the Unicode encoded file name causes trouble. Therefore \cs{pdfstringdef}
%     is now never used for option \xoption{filespec}, always method
%     \texttt{escape} is applied (Peter Cibulka).
%   \item
%     Bug fix for \xoption{initialfile}.
%   \item
%     Bug fix for file names in \texttt{/EmbeddedFiles}.
%   \item
%     New option \xoption{ucfilespec} for file name support in Unicode
%     (since PDF 1.7).
%   \end{Version}
%   \begin{Version}{2010/03/01 v2.5}
%   \item
%     Compatibility for \iniTeX.
%   \item
%     Package \xpackage{keyval} replaced by packages \xpackage{kvsetkeys}
%     and \xpackage{kvdefinekeys} because of compatibility for \iniTeX.
%   \item
%     TDS location moved from TDS:tex/latex/oberdiek/embedfile.sty to
%     TDS:tex/generic/oberdiek/embedfile.sty.
%   \end{Version}
%   \begin{Version}{2011/04/13 v2.6}
%   \item
%     Docu fixes (thanks Hans-Martin M\"unch).
%   \end{Version}
% \end{History}
%
% \PrintIndex
%
% \Finale
\endinput

%        (quote the arguments according to the demands of your shell)
%
% Documentation:
%    (a) If embedfile.drv is present:
%           latex embedfile.drv
%    (b) Without embedfile.drv:
%           latex embedfile.dtx; ...
%    The class ltxdoc loads the configuration file ltxdoc.cfg
%    if available. Here you can specify further options, e.g.
%    use A4 as paper format:
%       \PassOptionsToClass{a4paper}{article}
%
%    Programm calls to get the documentation (example):
%       pdflatex embedfile.dtx
%       makeindex -s gind.ist embedfile.idx
%       pdflatex embedfile.dtx
%       makeindex -s gind.ist embedfile.idx
%       pdflatex embedfile.dtx
%
% Installation:
%    TDS:tex/generic/oberdiek/embedfile.sty
%    TDS:tex/generic/oberdiek/dtx-attach.sty
%    TDS:doc/latex/oberdiek/embedfile.pdf
%    TDS:doc/latex/oberdiek/embedfile-example-plain.tex
%    TDS:doc/latex/oberdiek/embedfile-example-collection.tex
%    TDS:doc/latex/oberdiek/test/embedfile-test1.tex
%    TDS:doc/latex/oberdiek/test/embedfile-test2.tex
%    TDS:doc/latex/oberdiek/test/embedfile-test3.tex
%    TDS:doc/latex/oberdiek/test/embedfile-test4.tex
%    TDS:source/latex/oberdiek/embedfile.dtx
%
%<*ignore>
\begingroup
  \catcode123=1 %
  \catcode125=2 %
  \def\x{LaTeX2e}%
\expandafter\endgroup
\ifcase 0\ifx\install y1\fi\expandafter
         \ifx\csname processbatchFile\endcsname\relax\else1\fi
         \ifx\fmtname\x\else 1\fi\relax
\else\csname fi\endcsname
%</ignore>
%<*install>
\input docstrip.tex
\Msg{************************************************************************}
\Msg{* Installation}
\Msg{* Package: embedfile 2011/04/13 v2.6 Embed files into PDF (HO)}
\Msg{************************************************************************}

\keepsilent
\askforoverwritefalse

\let\MetaPrefix\relax
\preamble

This is a generated file.

Project: embedfile
Version: 2011/04/13 v2.6

Copyright (C) 2006-2011 by
   Heiko Oberdiek <heiko.oberdiek at googlemail.com>

This work may be distributed and/or modified under the
conditions of the LaTeX Project Public License, either
version 1.3c of this license or (at your option) any later
version. This version of this license is in
   http://www.latex-project.org/lppl/lppl-1-3c.txt
and the latest version of this license is in
   http://www.latex-project.org/lppl.txt
and version 1.3 or later is part of all distributions of
LaTeX version 2005/12/01 or later.

This work has the LPPL maintenance status "maintained".

This Current Maintainer of this work is Heiko Oberdiek.

The Base Interpreter refers to any `TeX-Format',
because some files are installed in TDS:tex/generic//.

This work consists of the main source file embedfile.dtx
and the derived files
   embedfile.sty, embedfile.pdf, embedfile.ins, embedfile.drv,
   dtx-attach.sty, embedfile-example-plain.tex,
   embedfile-example-collection.tex, embedfile-test1.tex,
   embedfile-test2.tex, embedfile-test3.tex,
   embedfile-test4.tex.

\endpreamble
\let\MetaPrefix\DoubleperCent

\generate{%
  \file{embedfile.ins}{\from{embedfile.dtx}{install}}%
  \file{embedfile.drv}{\from{embedfile.dtx}{driver}}%
  \usedir{tex/generic/oberdiek}%
  \file{embedfile.sty}{\from{embedfile.dtx}{package}}%
  \usedir{tex/latex/oberdiek}%
  \file{dtx-attach.sty}{\from{embedfile.dtx}{dtxattach}}%
  \usedir{doc/latex/oberdiek}%
  \file{embedfile-example-plain.tex}{\from{embedfile.dtx}{exampleplain}}%
  \file{embedfile-example-collection.tex}{\from{embedfile.dtx}{examplecollection}}%
  \usedir{doc/latex/oberdiek/test}%
  \file{embedfile-test1.tex}{\from{embedfile.dtx}{test1}}%
  \file{embedfile-test2.tex}{\from{embedfile.dtx}{test2}}%
  \file{embedfile-test3.tex}{\from{embedfile.dtx}{test3}}%
  \file{embedfile-test4.tex}{\from{embedfile.dtx}{test4}}%
  \nopreamble
  \nopostamble
  \usedir{source/latex/oberdiek/catalogue}%
  \file{embedfile.xml}{\from{embedfile.dtx}{catalogue}}%
}

\catcode32=13\relax% active space
\let =\space%
\Msg{************************************************************************}
\Msg{*}
\Msg{* To finish the installation you have to move the following}
\Msg{* files into a directory searched by TeX:}
\Msg{*}
\Msg{*     embedfile.sty, dtx-attach.sty}
\Msg{*}
\Msg{* To produce the documentation run the file `embedfile.drv'}
\Msg{* through LaTeX.}
\Msg{*}
\Msg{* Happy TeXing!}
\Msg{*}
\Msg{************************************************************************}

\endbatchfile
%</install>
%<*ignore>
\fi
%</ignore>
%<*driver>
\NeedsTeXFormat{LaTeX2e}
\ProvidesFile{embedfile.drv}%
  [2011/04/13 v2.6 Embed files into PDF (HO)]%
\documentclass{ltxdoc}
\usepackage{holtxdoc}[2011/11/22]
\begin{document}
  \DocInput{embedfile.dtx}%
\end{document}
%</driver>
% \fi
%
% \CheckSum{1363}
%
% \CharacterTable
%  {Upper-case    \A\B\C\D\E\F\G\H\I\J\K\L\M\N\O\P\Q\R\S\T\U\V\W\X\Y\Z
%   Lower-case    \a\b\c\d\e\f\g\h\i\j\k\l\m\n\o\p\q\r\s\t\u\v\w\x\y\z
%   Digits        \0\1\2\3\4\5\6\7\8\9
%   Exclamation   \!     Double quote  \"     Hash (number) \#
%   Dollar        \$     Percent       \%     Ampersand     \&
%   Acute accent  \'     Left paren    \(     Right paren   \)
%   Asterisk      \*     Plus          \+     Comma         \,
%   Minus         \-     Point         \.     Solidus       \/
%   Colon         \:     Semicolon     \;     Less than     \<
%   Equals        \=     Greater than  \>     Question mark \?
%   Commercial at \@     Left bracket  \[     Backslash     \\
%   Right bracket \]     Circumflex    \^     Underscore    \_
%   Grave accent  \`     Left brace    \{     Vertical bar  \|
%   Right brace   \}     Tilde         \~}
%
% \GetFileInfo{embedfile.drv}
%
% \title{The \xpackage{embedfile} package}
% \date{2011/04/13 v2.6}
% \author{Heiko Oberdiek\\\xemail{heiko.oberdiek at googlemail.com}}
%
% \maketitle
%
% \begin{abstract}
% This package embeds files to a PDF document.
% Currently the only supported driver is \pdfTeX\ $>=$ 1.30 in PDF mode.
% \end{abstract}
%
% \tableofcontents
%
% \section{Documentation}
%
% \subsection{Introduction}
%
%    The PDF format (\cite{pdfspec}) allows the inclusion of files
%    inside the PDF document. The included files can be bound to an
%    annotation on a page. Or they can be recorded in a sorted
%    list of embedded files. The packages \xpackage{attachfile}
%    or \xpackage{attachfile2} follow the first approach, this
%    package uses the latter method.
%
% \subsubsection{Future development}
%
%    My dream is a large package that merges the features of
%    all these packages meantioned before:
%    \begin{itemize}
%    \item Files can be attached to a page.
%    \item Files can be attached to the document.
%    \item An easy user interface for simple, common tasks and
%          beginners.
%    \item An interface for the advanced users that want to setup
%          every detail.
%    \item Support of many drivers (pdftex, dvips, dvipdfm, \dots).
%    \item \dots
%    \end{itemize}
%    However, I have not managed to take the time for this project.
%    Instead:
%    \begin{itemize}
%    \item First I experimented with package \xpackage{attachfile},
%          adding driver support, fixing bugs, \dots. The result is
%          currently named as \xpackage{attachfile2}. It uses an external
%          script to get file properties (size, date, checksum, \dots).
%    \item In order to avoid an external program for getting basic
%          file properties I provided a patch ``EscapeAndOther'' for
%          pdfTeX that was accepted for version 1.30.
%    \item This package closes a gap left by the packages for attaching
%          files and allows the embedding of files to the document.
%          Also it makes use of the new primitives of \pdfTeX.
%    \end{itemize}
%
% \subsection{User interface}
%
%    This package \xpackage{embedfile} can be used with both
%    \LaTeX\ and \plainTeX. See \autoref{sec:plain} that
%    explains the use with \plainTeX\ by an example.
%    In \LaTeX\ the package is loaded as usually. There are no options.
%    \begin{quote}
%      \verb|\usepackage{embedfile}|
%    \end{quote}
%
%    \begin{declcs}{embedfile}
%      |[|\meta{options}|]| \M{file}
%    \end{declcs}
%    The macro \cs{embedfile} includes file \meta{file} and
%    attaches it to the PDF document. At the end of the document
%    the sorted list of embedded files are written. Thus you can
%    safely use \cs{embedfile} before \verb|\end{document}|.
%    Embedding files using \cs{AtEndDocument} will only work,
%    if \cs{AtEndDocument} is called before loading the package
%    \xpackage{embedfile}.
%
%    The \meta{options} are give as key value pairs. The following
%    keys are supported:
%    \begin{description}
%    \item[\xoption{filespec}]
%      This allows to override the file name
%      that appears in the PDF file. If you are using other than
%      simple file names (8-bit, path separators, \dots), look into
%      the PDF specification (\cite{pdfspec}).
%      There are rules how these file names must be written/encoded.
%      Avoid 8-bit characters
%      and other special characters, the behaviour is currently undefined.
%      Use option \xoption{ucfilespec} for more funny file names.
%      The string method, see below, is |escape| since version 2.4.
%
%      This name is also used as entry in a name tree
%      (see PDF specification: |/EmbeddedFiles|). Therefore the
%      value for \xoption{filespec} must be unique among all
%      embedded files. Also key \xoption{initialfiles} refers
%      to this name, if the file name and the value of \xoption{filespec}
%      are different.
%    \item[\xoption{ucfilespec}]
%      Since PDF 1.7 the file name may be provided in Unicode.
%      The conversion of the option value into a PDF string
%      is controlled by option \xoption{stringmethod}.
%    \item[\xoption{filesystem}]
%      This sets the entry \verb|/FS| in
%      the file specification dictionary, see PDF specification
%      (\cite{pdfspec}).
%      Example: \verb|filesystem=URL|.
%    \item[\xoption{mimetype}]
%      This sets the mime type (\cite{mime}) of the file,
%      see \autoref{sec:plain} for examples and \cite{types}
%      for a list of officially registered types.
%    \item[\xoption{desc}]
%      The description for the file.
%    \item[\xoption{stringmethod}]
%      The package must convert the values of the keys \xoption{ucfilespec}
%      and \xoption{desc} into a PDF string (before version 2.4: \xoption{filespec}
%      and \xoption{desc}). If \xpackage{hyperref}
%      is found, then its \cs{pdfstringdef} will be used, otherwise
%      \pdfTeX's \cs{pdfescapestring} is used. Value \verb|psd|
%      forces the use of \cs{pdfstringdef}, value \verb|escape|
%      the use of \cs{pdfescapestring}.
%    \item[\xoption{\meta{key}.value}] Sets the value of a collection
%      item property, see section \ref{sec:collection}.
%    \item[\xoption{\meta{key}.prefix}]
%      Sets the prefix of a collection item property,
%      see section \ref{sec:collection}.
%    \item[\xoption{id}]
%      The value must be an unique name. Macros
%      \cs{embedfileifobjectexists} and \cs{embedfilegetobject}
%      are using this name later.
%    \end{description}
%
%    \begin{declcs}{embedfilefinish}
%    \end{declcs}
%    The list of all embedded files must be added as data structure
%    in the PDF file. In case of \LaTeX\ this is automatically done.
%    The package uses \cs{AtEndDocument}. Then the list of all
%    files should be known. However, \plainTeX\ does not know
%    about \cs{AtEndDocument}. Thus the user must call
%    \cs{embedfilefinish} at the end of the document
%    after the last file is embedded.
%
%    \begin{declcs}{embedfilesetup} \M{options}
%    \end{declcs}
%    Options for \cs{embedfile} and collection support
%    can be set in \cs{embedfilesetup}.
%
% \subsection{Collection support (PDF 1.7)}
% \label{sec:collection}
%
%    Since PDF 1.7 the embedded files can form a \emph{collection}
%    (sometimes referred as \emph{package}), the main document
%    is called \emph{cover sheet}. See PDF specification
%    8.2.4 ``Collections'' and 3.10.5 ``Collection items'' \cite{pdfspec}.
%
%    Usually Acrobat Reader 7 or 8 shows the
%    embedded files in a table at the bottom with the following columns:
%    \begin{quote}
%    \begin{tabular}{@{}l|l|l|l@{}}
%       Name & Description & Modified & Size\\
%       \hline
%       \dots & \dots & \dots & \dots\\
%    \end{tabular}
%    \end{quote}
%    Acrobat Reader 10 shows the embeded files in the left panel
%    and adds a new column for the compressed size.
%
%    If the files form a collection, then they are displayed
%    in a table left or top (depending on option \xoption{view},
%    see \cs{embedfilesetup}).
%
%
%    Collection support is enabled automatically, if
%    it is used.
%
%    \begin{declcs}{embedfilesetup} \M{options}
%    \end{declcs}
%    The following options are supported in addition
%    to options for \cs{embedfile}:
%    \begin{description}
%    \item[\xoption{view}]
%      If the PDF file contains a collection, then Acrobat Reader 8
%      shows a line at the top below the menu bar and the toolbar.
%      It shows the current selected file, icons for changing the
%      view mode, an options menu.
%      The initial mode how the collection is presented is
%      set by this option \xoption{view}.
%      The following modes/values are supported, the default is |details|:
%        \begin{description}
%        \item[\texttt{details}] The full collection table is displayed
%          at the top below the collection bar.
%        \item[\texttt{tile}] The files of the collection are shown
%          in tile mode on the left.
%        \item[\texttt{hidden}] The collection table is not shown.
%        \end{description}
%    \item[\xoption{initialfile}]
%      Selects the file that is initially presented. Especially
%      useful for an embedded PDF file that is then shown instead
%      of the cover document. There must be an \cs{embedfile} command
%      somewhere whose value for key \xoption{filespec} is used here.
%      The \cs{embedfile} command can drop option \xoption{filespec}
%      if the file name is not different.
%    \end{description}
%
%    \begin{declcs}{embedfilefield} \M{key} \M{options}
%    \end{declcs}
%    Macro \cs{embedfilefield} defines a column/field in the collection table.
%    The name of the field is \meta{key}.
%    \begin{description}
%    \item[\xoption{type}]
%      sets the type of the field. The supported values are:
%      \begin{description}
%      \item[\texttt{text}] A text field. Its value is set in
%        \cs{embedfile} by option \xoption{\meta{key}.value}.
%      \item[\texttt{date}] A date field. Its value is set in
%        \cs{embedfile} by option \xoption{\meta{key}.value}.
%        A special format is required, see ``3.8.3 Dates'' \cite{pdfspec}.
%      \item[\texttt{number}] A field with an integer or float number.
%        Its value is set in \cs{embedfile} by option
%        \xoption{\meta{key}.value}.
%      \item[\texttt{file}] The file name of the embedded file.
%      \item[\texttt{desc}] The description text of the embedded file. It is
%       set in \cs{embedfile} by option \xoption{desc}.
%      \item[\texttt{moddate}] The modification date of the embedded file.
%      \item[\texttt{size}] The size of the embedded file.
%      \end{description}
%      All types allow the use of a prefix that is disregarded by
%      sorting. The prefix for this field is set in \cs{embedfile} by
%      option \xoption{\meta{key}.prefix}.
%    \item[\xoption{title}] sets the column title.
%    \item[\xoption{visible}] controls whether the column is presented:
%      \begin{description}
%      \item[\texttt{true}] shows the column.
%      \item[\texttt{false}] hides the column.
%      \end{description}
%      Default: \texttt{true}
%    \item[\xoption{edit}]
%      Allows the editing of field values. Does not seem to have an
%      effect for Acrobat Reader.
%      \begin{description}
%      \item[\texttt{true}] enables the feature, if available (depends
%        on the PDF viewer).
%      \item[\texttt{false}] disables the feature.
%      \end{description}
%      Default: \texttt{false}
%    \end{description}
%    The order of \cs{embedfilefield} statements defines the order
%    of the columns.
%
%    \begin{declcs}{embedfilesort} \M{key-sort-list}
%    \end{declcs}
%    The sort order of the embedded files are controlled by
%    macro \cs{embedfilesort}. \meta{key-sort-list} defines
%    the sort order. The key is a field name defined by \cs{embedfilefield}.
%    Its value is either \texttt{ascending} or \texttt{descending}.
%    The default is \texttt{ascending}.
%
% \subsection{Export of object references}
%
%    Caution: This feature is still experimental. It may be even removed
%    in future versions. Therefore feedback would be nice, if someone
%    has a useful application for this feature.
%
%    Object numbers are saved, if \xoption{id} is given in \cs{embedfile}.
%    The following objects are supported:
%    \begin{itemize}
%    \item \texttt{EmbeddedFile}
%    \item \texttt{Filespec}
%    \end{itemize}
%
%    \begin{declcs}{embedfileifobjectexists} \M{id} \M{type}
%      \M{then} \M{else}
%    \end{declcs}
%    Macro \cs{embedfileifobjectexists} tests whether object of \meta{type}
%    is available for the embedded file identified by \meta{id}.
%
%    \begin{declcs}{embedfilegetobject} \M{id} \M{type}
%    \end{declcs}
%    Macro \cs{embedfilegetobject} expands to the full object reference
%    object of \meta{type} for the embedded file identified by \meta{id}.
%
% \subsubsection{Example}
%
%\begin{quote}
%\begin{verbatim}
%\embedfile[id={foo}]{foo.pdf}
%\embedfileifobjectexists{foo}{Filespec}{%
%  \typeout{%
%    FileSpec object for `foo': %
%    \embedfilegetobject{foo}{Filespec}%
%  }%
%}{%
%  \typeout{No Filespec object for `foo'}%
%}
%\end{verbatim}
%\end{quote}
%
% \subsection{Examples}
% \label{sec:examples}
%
% \subsubsection{\plainTeX}
% \label{sec:plain}
%
%    The package can be used with \plainTeX. It can be used
%    with or without help from \xfile{miniltx.tex}.
%
%    If additionally package \xpackage{keyval} (\xpackage{graphicx})
%    is needed, load it first. Then package \xpackage{embedfile} avoids
%    a duplicate loading of package \xpackage{keyval}.
%
%    Because \plainTeX\ does not provide a hook at end of the document,
%    you have to call \cs{embedfilefinish} manually at the end
%    after the last embedded file.
%    \begin{macrocode}
%<*exampleplain>
%<<END
% Load packages
\input miniltx
% \def\Gin@driver{pdftex.def}
% \input graphicx.sty
\input embedfile.sty
\resetatcatcode

% default setting
\embedfilesetup{
  mimetype=text/plain
}

% Embed files
\embedfile[
  filespec=example.tex,
  desc={Source code (plain-TeX) of this example}
]{embedfile-example-plain.tex}

\embedfile[
  desc={Source of package `embedfile'}
]{embedfile.dtx}

\embedfile[
  mimetype=application/pdf,
  desc={Documentation of package `embedfile'}
]{embedfile.pdf}

% Some text
This example document contains three embedded files.

% End of document
\embedfilefinish % don't forget
\bye
%END
%</exampleplain>
%    \end{macrocode}
%
% \subsubsection{Collection example}
%
%    \begin{macrocode}
%<*examplecollection>
%<<END
\NeedsTeXFormat{LaTeX2e}
\documentclass{article}
\usepackage[bookmarks=false]{hyperref}
% provides \pdfstringdef that is then used by `title' and
% other keys.
\usepackage{embedfile}[2011/04/13]
\embedfilesetup{
  view=details,
  initialfile=embedfile.pdf
}
\embedfilefield{file}{
  type=file,
  title={File name}
}
\embedfilefield{description}{
  type=desc,
  title={Description}
}
\embedfilefield{date}{
  type=moddate,
  title={Date}
}
\embedfilefield{size}{
  type=size,
  title={Size}
}
\embedfilefield{type}{
  type=text,
  title={Type},
  visible=false
}
\embedfilesort{
  type,
  date=descending
}
\begin{document}
An example for embedded files as collection.
You need Acrobat Reader 8 or higher.

\embedfile[
  desc={Source file of package `embedfile'},
  description.prefix={Package: },
  type.value={DTX}
]{embedfile.dtx}

\embedfile[
  desc={Documentation of package `embedfile'},
  description.prefix={Package: },
  type.value={PDF}
]{embedfile.pdf}

\embedfile[
  desc={The source for this example},
  description.prefix={Example: },
  type.value={TEX}
]{\jobname.tex}

\end{document}
%END
%</examplecollection>
%    \end{macrocode}
%
% \subsection{Package \xpackage{dtx-attach}}
%
% Package \xpackage{dtx-attach} is just a small application of
% package \xpackage{embedfile}. I am using it for the CTAN
% documentation of my packages in \CTAN{macros/latex/contrib/oberdiek/}.
% It also serves as small example for the use of the package with \LaTeX.
%
%    \begin{macrocode}
%<*dtxattach>
\NeedsTeXFormat{LaTeX2e}
\ProvidesPackage{dtx-attach}
  [2011/04/13 v2.6 Embed \string\jobname.dtx (HO)]%
\RequirePackage{embedfile}[2011/04/13]
\embedfile[%
  stringmethod=escape,%
  mimetype=plain/text,%
  desc={LaTeX docstrip source archive for package `\jobname'}%
]{\jobname.dtx}
%</dtxattach>
%    \end{macrocode}
%
% \StopEventually{
% }
%
% \section{Implementation}
%
%    \begin{macrocode}
%<*package>
%    \end{macrocode}
%
% \subsection{Reload check and package identification}
%    Reload check, especially if the package is not used with \LaTeX.
%    \begin{macrocode}
\begingroup\catcode61\catcode48\catcode32=10\relax%
  \catcode13=5 % ^^M
  \endlinechar=13 %
  \catcode35=6 % #
  \catcode39=12 % '
  \catcode44=12 % ,
  \catcode45=12 % -
  \catcode46=12 % .
  \catcode58=12 % :
  \catcode64=11 % @
  \catcode123=1 % {
  \catcode125=2 % }
  \expandafter\let\expandafter\x\csname ver@embedfile.sty\endcsname
  \ifx\x\relax % plain-TeX, first loading
  \else
    \def\empty{}%
    \ifx\x\empty % LaTeX, first loading,
      % variable is initialized, but \ProvidesPackage not yet seen
    \else
      \expandafter\ifx\csname PackageInfo\endcsname\relax
        \def\x#1#2{%
          \immediate\write-1{Package #1 Info: #2.}%
        }%
      \else
        \def\x#1#2{\PackageInfo{#1}{#2, stopped}}%
      \fi
      \x{embedfile}{The package is already loaded}%
      \aftergroup\endinput
    \fi
  \fi
\endgroup%
%    \end{macrocode}
%    Package identification:
%    \begin{macrocode}
\begingroup\catcode61\catcode48\catcode32=10\relax%
  \catcode13=5 % ^^M
  \endlinechar=13 %
  \catcode35=6 % #
  \catcode39=12 % '
  \catcode40=12 % (
  \catcode41=12 % )
  \catcode44=12 % ,
  \catcode45=12 % -
  \catcode46=12 % .
  \catcode47=12 % /
  \catcode58=12 % :
  \catcode64=11 % @
  \catcode91=12 % [
  \catcode93=12 % ]
  \catcode123=1 % {
  \catcode125=2 % }
  \expandafter\ifx\csname ProvidesPackage\endcsname\relax
    \def\x#1#2#3[#4]{\endgroup
      \immediate\write-1{Package: #3 #4}%
      \xdef#1{#4}%
    }%
  \else
    \def\x#1#2[#3]{\endgroup
      #2[{#3}]%
      \ifx#1\@undefined
        \xdef#1{#3}%
      \fi
      \ifx#1\relax
        \xdef#1{#3}%
      \fi
    }%
  \fi
\expandafter\x\csname ver@embedfile.sty\endcsname
\ProvidesPackage{embedfile}%
  [2011/04/13 v2.6 Embed files into PDF (HO)]%
%    \end{macrocode}
%
% \subsection{Catcodes}
%
%    \begin{macrocode}
\begingroup\catcode61\catcode48\catcode32=10\relax%
  \catcode13=5 % ^^M
  \endlinechar=13 %
  \catcode123=1 % {
  \catcode125=2 % }
  \catcode64=11 % @
  \def\x{\endgroup
    \expandafter\edef\csname EmFi@AtEnd\endcsname{%
      \endlinechar=\the\endlinechar\relax
      \catcode13=\the\catcode13\relax
      \catcode32=\the\catcode32\relax
      \catcode35=\the\catcode35\relax
      \catcode61=\the\catcode61\relax
      \catcode64=\the\catcode64\relax
      \catcode123=\the\catcode123\relax
      \catcode125=\the\catcode125\relax
    }%
  }%
\x\catcode61\catcode48\catcode32=10\relax%
\catcode13=5 % ^^M
\endlinechar=13 %
\catcode35=6 % #
\catcode64=11 % @
\catcode123=1 % {
\catcode125=2 % }
\def\TMP@EnsureCode#1#2{%
  \edef\EmFi@AtEnd{%
    \EmFi@AtEnd
    \catcode#1=\the\catcode#1\relax
  }%
  \catcode#1=#2\relax
}
\TMP@EnsureCode{39}{12}% '
\TMP@EnsureCode{40}{12}% (
\TMP@EnsureCode{41}{12}% )
\TMP@EnsureCode{44}{12}% ,
\TMP@EnsureCode{46}{12}% .
\TMP@EnsureCode{47}{12}% /
\TMP@EnsureCode{58}{12}% :
\TMP@EnsureCode{60}{12}% <
\TMP@EnsureCode{62}{12}% >
\TMP@EnsureCode{91}{12}% [
\TMP@EnsureCode{93}{12}% ]
\TMP@EnsureCode{96}{12}% `
\edef\EmFi@AtEnd{\EmFi@AtEnd\noexpand\endinput}
%    \end{macrocode}
%
% \subsection{Tools}
%
%    \begin{macro}{\EmFi@RequirePackage}
%    \begin{macrocode}
\begingroup\expandafter\expandafter\expandafter\endgroup
\expandafter\ifx\csname RequirePackage\endcsname\relax
  \def\EmFi@RequirePackage#1[#2]{%
    \input #1.sty\relax
  }%
\else
  \let\EmFi@RequirePackage\RequirePackage
\fi
%    \end{macrocode}
%    \end{macro}
%
%    \begin{macro}{\EmFi@Error}
%    \begin{macrocode}
\EmFi@RequirePackage{infwarerr}[2007/09/09]%
\def\EmFi@Error{%
  \@PackageError{embedfile}%
}
%    \end{macrocode}
%    \end{macro}
%
% \subsection{Check for recent \pdfTeX\ in PDF mode}
%
%    Load package \xpackage{ifpdf} and check mode.
%    \begin{macrocode}
\EmFi@RequirePackage{ifpdf}[2007/09/09]
\ifpdf
\else
  \EmFi@Error{%
    Missing pdfTeX in PDF mode%
  }{%
    Currently other drivers are not supported. %
    Package loading is aborted.%
  }%
  \expandafter\EmFi@AtEnd
\fi%
%    \end{macrocode}
%    \begin{macrocode}
\EmFi@RequirePackage{pdftexcmds}[2007/11/11]
\EmFi@RequirePackage{ltxcmds}[2010/03/01]
\EmFi@RequirePackage{kvsetkeys}[2010/03/01]
\EmFi@RequirePackage{kvdefinekeys}[2010/03/01]
%    \end{macrocode}
%    Check version.
%    \begin{macrocode}
\begingroup\expandafter\expandafter\expandafter\endgroup
\expandafter\ifx\csname pdf@filesize\endcsname\relax
  \EmFi@Error{%
    Unsupported pdfTeX version%
  }{%
    At least version 1.30 is necessary. Package loading is aborted.%
  }%
  \expandafter\EmFi@AtEnd
\fi%
%    \end{macrocode}
%
% \subsection{Strings}
%
%    Minimal version of package \xpackage{pdfescape} is 2007/08/27 v1.5
%    because of \cs{EdefSanitize}.
%    \begin{macrocode}
\EmFi@RequirePackage{pdfescape}[2007/11/11]
%    \end{macrocode}
%
%    \begin{macrocode}
\def\EmFi@temp#1{%
  \expandafter\EdefSanitize\csname EmFi@S@#1\endcsname{#1}%
}
%    \end{macrocode}
%    \begin{macro}{\EmFi@details}
%    \begin{macrocode}
\EmFi@temp{details}%
%    \end{macrocode}
%    \end{macro}
%    \begin{macro}{\EmFi@tile}
%    \begin{macrocode}
\EmFi@temp{tile}%
%    \end{macrocode}
%    \end{macro}
%    \begin{macro}{\EmFi@hidden}
%    \begin{macrocode}
\EmFi@temp{hidden}%
%    \end{macrocode}
%    \end{macro}
%    \begin{macro}{\EmFi@S@text}
%    \begin{macrocode}
\EmFi@temp{text}
%    \end{macrocode}
%    \end{macro}
%    \begin{macro}{\EmFi@S@date}
%    \begin{macrocode}
\EmFi@temp{date}
%    \end{macrocode}
%    \end{macro}
%    \begin{macro}{\EmFi@S@number}
%    \begin{macrocode}
\EmFi@temp{number}
%    \end{macrocode}
%    \end{macro}
%    \begin{macro}{\EmFi@S@file}
%    \begin{macrocode}
\EmFi@temp{file}
%    \end{macrocode}
%    \end{macro}
%    \begin{macro}{\EmFi@S@desc}
%    \begin{macrocode}
\EmFi@temp{desc}
%    \end{macrocode}
%    \end{macro}
%    \begin{macro}{\EmFi@S@moddate}
%    \begin{macrocode}
\EmFi@temp{moddate}
%    \end{macrocode}
%    \end{macro}
%    \begin{macro}{\EmFi@S@creationdate}
%    \begin{macrocode}
\EmFi@temp{creationdate}
%    \end{macrocode}
%    \end{macro}
%    \begin{macro}{\EmFi@S@size}
%    \begin{macrocode}
\EmFi@temp{size}
%    \end{macrocode}
%    \end{macro}
%    \begin{macro}{\EmFi@S@ascending}
%    \begin{macrocode}
\EmFi@temp{ascending}
%    \end{macrocode}
%    \end{macro}
%    \begin{macro}{\EmFi@S@descending}
%    \begin{macrocode}
\EmFi@temp{descending}
%    \end{macrocode}
%    \end{macro}
%    \begin{macro}{\EmFi@S@true}
%    \begin{macrocode}
\EmFi@temp{true}
%    \end{macrocode}
%    \end{macro}
%    \begin{macro}{\EmFi@S@false}
%    \begin{macrocode}
\EmFi@temp{false}
%    \end{macrocode}
%    \end{macro}
%
% \subsection{Switches}
%
%    \begin{macro}{\ifEmFi@collection}
%    \begin{macrocode}
\ltx@newif\ifEmFi@collection
%    \end{macrocode}
%    \end{macro}
%    \begin{macro}{\ifEmFi@sort}
%    \begin{macrocode}
\ltx@newif\ifEmFi@sort
%    \end{macrocode}
%    \end{macro}
%    \begin{macro}{\ifEmFi@visible}
%    \begin{macrocode}
\ltx@newif\ifEmFi@visible
%    \end{macrocode}
%    \end{macro}
%    \begin{macro}{\ifEmFi@edit}
%    \begin{macrocode}
\ltx@newif\ifEmFi@edit
%    \end{macrocode}
%    \end{macro}
%    \begin{macro}{\ifEmFi@item}
%    \begin{macrocode}
\ltx@newif\ifEmFi@item
%    \end{macrocode}
%    \end{macro}
%    \begin{macro}{\ifEmFi@finished}
%    \begin{macrocode}
\ltx@newif\ifEmFi@finished
%    \end{macrocode}
%    \end{macro}
%    \begin{macro}{\ifEmFi@id}
%    \begin{macrocode}
\ltx@newif\ifEmFi@id
%    \end{macrocode}
%    \end{macro}
%
% \subsection{Key value definitions}
%
%    \begin{macro}{\EmFi@GlobalKey}
%    \begin{macrocode}
\def\EmFi@GlobalKey#1#2{%
  \global\expandafter\let\csname KV@#1@#2\expandafter\endcsname
                         \csname KV@#1@#2\endcsname
}
%    \end{macrocode}
%    \end{macro}
%    \begin{macro}{\EmFi@GlobalDefaultKey}
%    \begin{macrocode}
\def\EmFi@GlobalDefaultKey#1#2{%
  \EmFi@GlobalKey{#1}{#2}%
  \global\expandafter\let
      \csname KV@#1@#2@default\expandafter\endcsname
      \csname KV@#1@#2@default\endcsname
}
%    \end{macrocode}
%    \end{macro}
%
%    \begin{macro}{\EmFi@DefineKey}
%    \begin{macrocode}
\def\EmFi@DefineKey#1#2{%
  \kv@define@key{EmFi}{#1}{%
    \expandafter\def\csname EmFi@#1\endcsname{##1}%
  }%
  \expandafter\def\csname EmFi@#1\endcsname{#2}%
}
%    \end{macrocode}
%    \end{macro}
%    Subtype of the embedded file (optional).
%    \begin{macrocode}
\EmFi@DefineKey{mimetype}{}
%    \end{macrocode}
%    File specification string.
%    \begin{macrocode}
\EmFi@DefineKey{filespec}{\EmFi@file}
%    \end{macrocode}
%    File specification string in Unicode.
%    \begin{macrocode}
\EmFi@DefineKey{ucfilespec}{}
%    \end{macrocode}
%    File system (optional).
%    \begin{macrocode}
\EmFi@DefineKey{filesystem}{}
%    \end{macrocode}
%    Description (optional).
%    \begin{macrocode}
\EmFi@DefineKey{desc}{}
%    \end{macrocode}
%    Method for converting text to PDF strings.
%    \begin{macrocode}
\EmFi@DefineKey{stringmethod}{%
  \ifx\pdfstringdef\@undefined
    escape%
  \else
    \ifx\pdfstringdef\relax
      escape%
    \else
      psd%
    \fi
  \fi
}
%    \end{macrocode}
%
%    Option \xoption{id} as key for object numbers.
%    \begin{macrocode}
\kv@define@key{EmFi}{id}{%
  \def\EmFi@id{#1}%
  \EmFi@idtrue
}
%    \end{macrocode}
%    \begin{macro}{\EmFi@defobj}
%    \begin{macrocode}
\def\EmFi@defobj#1{%
  \ifEmFi@id
    \expandafter\xdef\csname EmFi@#1@\EmFi@id\endcsname{%
      \the\pdflastobj\ltx@space 0 R%
    }%
  \fi
}
%    \end{macrocode}
%    \end{macro}
%    \begin{macro}{\embedfileifobjectexists}
%    \begin{macrocode}
\def\embedfileifobjectexists#1#2{%
  \expandafter\ifx\csname EmFi@#2@#1\endcsname\relax
    \expandafter\ltx@secondoftwo
  \else
    \expandafter\ltx@firstoftwo
  \fi
}
%    \end{macrocode}
%    \end{macro}
%    \begin{macro}{\embedfilegetobject}
%    \begin{macrocode}
\def\embedfilegetobject#1#2{%
  \embedfileifobjectexists{#1}{#2}{%
    \csname EmFi@#2@#1\endcsname
  }{%
    0 0 R%
  }%
}
%    \end{macrocode}
%    \end{macro}
%
%    Initial view of the collection.
%    \begin{macrocode}
\kv@define@key{EmFi}{view}[]{%
  \EdefSanitize\EmFi@temp{#1}%
  \def\EmFi@next{%
    \global\EmFi@collectiontrue
  }%
  \ifx\EmFi@temp\ltx@empty
    \let\EmFi@view\EmFi@S@details
  \else\ifx\EmFi@temp\EmFi@S@details
    \let\EmFi@view\EmFi@S@details
  \else\ifx\EmFi@temp\EmFi@S@tile
    \let\EmFi@view\EmFi@S@tile
  \else\ifx\EmFi@temp\EmFi@S@hidden
    \let\EmFi@view\EmFi@S@hidden
  \else
    \let\EmFi@next\relax
    \EmFi@Error{%
      Unknown value `\EmFi@temp' for key `view'.\MessageBreak
      Supported values: `details', `tile', `hidden'.%
    }\@ehc
  \fi\fi\fi\fi
  \EmFi@next
}
%    \end{macrocode}
%    \begin{macrocode}
\EmFi@DefineKey{initialfile}{}
%    \end{macrocode}
%
%    \begin{macro}{\embedfilesetup}
%    \begin{macrocode}
\def\embedfilesetup{%
  \ifEmFi@finished
    \def\EmFi@next##1{}%
    \EmFi@Error{%
      \string\embedfilefield\ltx@space after \string\embedfilefinish
    }{%
      The list of embedded files is already written.%
    }%
  \else
    \def\EmFi@next{%
      \kvsetkeys{EmFi}%
    }%
  \fi
  \EmFi@next
}
%    \end{macrocode}
%    \end{macro}
%
%    \begin{macro}{\EmFi@schema}
%    \begin{macrocode}
\def\EmFi@schema{}
%    \end{macrocode}
%    \end{macro}
%    \begin{macro}{\EmFi@order}
%    \begin{macrocode}
\gdef\EmFi@order{0}
%    \end{macrocode}
%    \end{macro}
%    \begin{macro}{\EmFi@@order}
%    \begin{macrocode}
\let\EmFi@@order\relax
%    \end{macrocode}
%    \end{macro}
%    \begin{macro}{\EmFi@fieldlist}
%    \begin{macrocode}
\def\EmFi@fieldlist{}
%    \end{macrocode}
%    \end{macro}
%    \begin{macro}{\EmFi@sortcase}
%    \begin{macrocode}
\def\EmFi@sortcase{0}%
%    \end{macrocode}
%    \end{macro}
%    \begin{macro}{\embedfilefield}
%    \begin{macrocode}
\def\embedfilefield#1#2{%
  \ifEmFi@finished
    \EmFi@Error{%
      \string\embedfilefield\ltx@space after \string\embedfilefinish
    }{%
      The list of embedded files is already written.%
    }%
  \else
    \global\EmFi@collectiontrue
    \EdefSanitize\EmFi@key{#1}%
    \expandafter\ifx\csname KV@EmFi@\EmFi@key.prefix\endcsname\relax
      \begingroup
        \count@=\EmFi@order
        \advance\count@ 1 %
        \xdef\EmFi@order{\the\count@}%
        \let\EmFi@title\EmFi@key
        \let\EmFi@type\EmFi@S@text
        \EmFi@visibletrue
        \EmFi@editfalse
        \kvsetkeys{EmFiFi}{#2}%
        \EmFi@convert\EmFi@title\EmFi@title
        \xdef\EmFi@schema{%
          \EmFi@schema
          /\pdf@escapename{\EmFi@key}<<%
            /Subtype/%
            \ifx\EmFi@type\EmFi@S@date D%
            \else\ifx\EmFi@type\EmFi@S@number N%
            \else\ifx\EmFi@type\EmFi@S@file F%
            \else\ifx\EmFi@type\EmFi@S@desc Desc%
            \else\ifx\EmFi@type\EmFi@S@moddate ModDate%
            \else\ifx\EmFi@type\EmFi@S@creationdate CreationDate%
            \else\ifx\EmFi@type\EmFi@S@size Size%
            \else S%
            \fi\fi\fi\fi\fi\fi\fi
            /N(\EmFi@title)%
            \EmFi@@order{\EmFi@order}%
            \ifEmFi@visible
            \else
              /V false%
            \fi
            \ifEmFi@edit
              /E true%
            \fi
          >>%
        }%
        \let\do\relax
        \xdef\EmFi@fieldlist{%
          \EmFi@fieldlist
          \do{\EmFi@key}%
        }%
        \ifx\EmFi@type\EmFi@S@text
          \kv@define@key{EmFi}{\EmFi@key.value}{%
            \EmFi@itemtrue
            \def\EmFi@temp{##1}%
            \EmFi@convert\EmFi@temp\EmFi@temp
            \expandafter\def\csname EmFi@V@#1%
            \expandafter\endcsname\expandafter{%
              \expandafter(\EmFi@temp)%
            }%
          }%
          \EmFi@GlobalKey{EmFi}{\EmFi@key.value}%
        \else\ifx\EmFi@type\EmFi@S@date
          \kv@define@key{EmFi}{\EmFi@key.value}{%
            \EmFi@itemtrue
            \def\EmFi@temp{##1}%
            \EmFi@convert\EmFi@temp\EmFi@temp
            \expandafter\def\csname EmFi@V@#1%
            \expandafter\endcsname\expandafter{%
              \expandafter(\EmFi@temp)%
            }%
          }%
          \EmFi@GlobalKey{EmFi}{\EmFi@key.value}%
        \else\ifx\EmFi@type\EmFi@S@number
          \kv@define@key{EmFi}{\EmFi@key.value}{%
            \EmFi@itemtrue
            \expandafter\EdefSanitize\csname EmFi@V@#1\endcsname{ ##1}%
          }%
          \EmFi@GlobalKey{EmFi}{\EmFi@key.value}%
        \fi\fi\fi
        \kv@define@key{EmFi}{\EmFi@key.prefix}{%
          \EmFi@itemtrue
          \expandafter\def\csname EmFi@P@#1\endcsname{##1}%
        }%
        \EmFi@GlobalKey{EmFi}{\EmFi@key.prefix}%
        \kv@define@key{EmFiSo}{\EmFi@key}[ascending]{%
          \EdefSanitize\EmFi@temp{##1}%
          \ifx\EmFi@temp\EmFi@S@ascending
            \def\EmFi@temp{true}%
          \else\ifx\EmFi@temp\EmFi@S@descending
            \def\EmFi@temp{false}%
          \else
            \def\EmFi@temp{}%
            \EmFi@Error{%
              Unknown sort order `\EmFi@temp'.\MessageBreak
              Supported values: `\EmFi@S@ascending', %
              `\EmFi@S@descending
            }\@ehc
          \fi\fi
          \ifx\EmFi@temp\ltx@empty
          \else
            \xdef\EmFi@sortkeys{%
              \EmFi@sortkeys
              /\pdf@escapename{#1}%
            }%
            \ifx\EmFi@sortorders\ltx@empty
              \global\let\EmFi@sortorders\EmFi@temp
              \gdef\EmFi@sortcase{1}%
            \else
              \xdef\EmFi@sortorders{%
                \EmFi@sortorders
                \ltx@space
                \EmFi@temp
              }%
              \xdef\EmFi@sortcase{2}%
            \fi
          \fi
        }%
        \EmFi@GlobalDefaultKey{EmFiSo}\EmFi@key
      \endgroup
    \else
      \EmFi@Error{%
        Field `\EmFi@key' is already defined%
      }\@ehc
    \fi
  \fi
}
%    \end{macrocode}
%    \end{macro}
%    \begin{macrocode}
\kv@define@key{EmFiFi}{type}{%
  \EdefSanitize\EmFi@temp{#1}%
  \ifx\EmFi@temp\EmFi@S@text
    \let\EmFi@type\EmFi@temp
  \else\ifx\EmFi@temp\EmFi@S@date
    \let\EmFi@type\EmFi@temp
  \else\ifx\EmFi@temp\EmFi@S@number
    \let\EmFi@type\EmFi@temp
  \else\ifx\EmFi@temp\EmFi@S@file
    \let\EmFi@type\EmFi@temp
  \else\ifx\EmFi@temp\EmFi@S@desc
    \let\EmFi@type\EmFi@temp
  \else\ifx\EmFi@temp\EmFi@S@moddate
    \let\EmFi@type\EmFi@temp
  \else\ifx\EmFi@temp\EmFi@S@creationdate
    \let\EmFi@type\EmFi@temp
  \else\ifx\EmFi@temp\EmFi@S@size
    \let\EmFi@type\EmFi@temp
  \else
    \EmFi@Error{%
      Unknown type `\EmFi@temp'.\MessageBreak
      Supported types: `text', `date', `number', `file',\MessageBreak
      `desc', `moddate', `creationdate', `size'%
    }%
  \fi\fi\fi\fi\fi\fi\fi\fi
}
%    \end{macrocode}
%    \begin{macrocode}
\kv@define@key{EmFiFi}{title}{%
  \def\EmFi@title{#1}%
}
%    \end{macrocode}
%    \begin{macro}{\EmFi@setboolean}
%    \begin{macrocode}
\def\EmFi@setboolean#1#2{%
  \EdefSanitize\EmFi@temp{#2}%
  \ifx\EmFi@temp\EmFi@S@true
    \csname EmFi@#1true\endcsname
  \else
    \ifx\EmFi@temp\EmFi@S@false
      \csname EmFi@#1false\endcsname
    \else
      \EmFi@Error{%
        Unknown value `\EmFi@temp' for key `#1'.\MessageBreak
        Supported values: `true', `false'%
      }\@ehc
    \fi
  \fi
}
%    \end{macrocode}
%    \end{macro}
%    \begin{macrocode}
\kv@define@key{EmFiFi}{visible}[true]{%
  \EmFi@setboolean{visible}{#1}%
}
%    \end{macrocode}
%    \begin{macrocode}
\kv@define@key{EmFiFi}{edit}[true]{%
  \EmFi@setboolean{edit}{#1}%
}
%    \end{macrocode}
%
%    \begin{macro}{\EmFi@sortkeys}
%    \begin{macrocode}
\def\EmFi@sortkeys{}
%    \end{macrocode}
%    \end{macro}
%    \begin{macro}{\EmFi@sortorders}
%    \begin{macrocode}
\def\EmFi@sortorders{}
%    \end{macrocode}
%    \end{macro}
%    \begin{macro}{\embedfilesort}
%    \begin{macrocode}
\def\embedfilesort{%
  \kvsetkeys{EmFiSo}%
}
%    \end{macrocode}
%    \end{macro}
%
% \subsection{Embed the file}
%
%    \begin{macro}{\embedfile}
%    \begin{macrocode}
\def\embedfile{%
  \ltx@ifnextchar[\EmFi@embedfile{\EmFi@embedfile[]}%
}
%    \end{macrocode}
%    \end{macro}
%    \begin{macro}{\EmFi@embedfile}
%    \begin{macrocode}
\def\EmFi@embedfile[#1]#2{%
  \ifEmFi@finished
    \EmFi@Error{%
      \string\embedfile\ltx@space after \string\embedfilefinish
    }{%
      The list of embedded files is already written.%
    }%
  \else
    \begingroup
      \def\EmFi@file{#2}%
      \kvsetkeys{EmFi}{#1}%
      \expandafter\expandafter\expandafter
      \ifx\expandafter\expandafter\expandafter
          \\\pdf@filesize{\EmFi@file}\\%
        \EmFi@Error{%
          File `\EmFi@file' not found%
        }{%
          The unknown file is not embedded.%
        }%
      \else
        \edef\EmFi@@filespec{%
          \pdf@escapestring{\EmFi@filespec}%
        }%
        \ifx\EmFi@ucfilespec\ltx@empty
          \let\EmFi@@ucfilespec\ltx@empty
        \else
          \EmFi@convert\EmFi@ucfilespec\EmFi@@ucfilespec
        \fi
        \ifx\EmFi@desc\ltx@empty
          \let\EmFi@@desc\ltx@empty
        \else
          \EmFi@convert\EmFi@desc\EmFi@@desc
        \fi
        \ifEmFi@item
          \let\do\EmFi@do
          \immediate\pdfobj{%
            <<%
              \EmFi@fieldlist
            >>%
          }%
          \edef\EmFi@ci{\the\pdflastobj}%
        \fi
        \immediate\pdfobj stream attr{%
          /Type/EmbeddedFile%
          \ifx\EmFi@mimetype\ltx@empty
          \else
            /Subtype/\pdf@escapename{\EmFi@mimetype}%
          \fi
          /Params<<%
            /ModDate(\pdf@filemoddate{\EmFi@file})%
            /Size \pdf@filesize{\EmFi@file}%
            /CheckSum<\pdf@filemdfivesum{\EmFi@file}>%
          >>%
        }file{\EmFi@file}\relax
        \EmFi@defobj{EmbeddedFile}%
        \immediate\pdfobj{%
          <<%
            /Type/Filespec%
            \ifx\EmFi@filesystem\ltx@empty
            \else
            /FS/\pdf@escapename{\EmFi@filesystem}%
            \fi
            /F(\EmFi@@filespec)%
            \ifx\EmFi@@ucfilespec\ltx@empty
            \else
              /UF(\EmFi@@ucfilespec)%
            \fi
            \ifx\EmFi@@desc\ltx@empty
            \else
              /Desc(\EmFi@@desc)%
            \fi
            /EF<<%
              /F \the\pdflastobj\ltx@space 0 R%
            >>%
            \ifEmFi@item
              /CI \EmFi@ci\ltx@space 0 R%
            \fi
          >>%
        }%
        \EmFi@defobj{Filespec}%
        \EmFi@add{%
          \EmFi@@filespec
        }{\the\pdflastobj\ltx@space 0 R}%
      \fi
    \endgroup
  \fi
}
%    \end{macrocode}
%    \end{macro}
%    \begin{macro}{\EmFi@do}
%    \begin{macrocode}
\def\EmFi@do#1{%
  \expandafter\ifx\csname EmFi@P@#1\endcsname\relax
    \expandafter\ifx\csname EmFi@V@#1\endcsname\relax
    \else
      /\pdf@escapename{#1}\csname EmFi@V@#1\endcsname
    \fi
  \else
    /\pdf@escapename{#1}<<%
      \expandafter\ifx\csname EmFi@V@#1\endcsname\relax
      \else
        /D\csname EmFi@V@#1\endcsname
      \fi
      /P(\csname EmFi@P@#1\endcsname)%
    >>%
  \fi
}
%    \end{macrocode}
%    \end{macro}
%    \begin{macro}{\EmFi@convert}
%    \begin{macrocode}
\def\EmFi@convert#1#2{%
  \ifnum\pdf@strcmp{\EmFi@stringmethod}{psd}=0 %
    \pdfstringdef\EmFi@temp{#1}%
    \let#2\EmFi@temp
  \else
    \edef#2{\pdf@escapestring{#1}}%
  \fi
}
%    \end{macrocode}
%    \end{macro}
%    \begin{macrocode}
\global\let\EmFi@list\ltx@empty
%    \end{macrocode}
%    \begin{macro}{\EmFi@add}
%    Sorting is done by the insertion sort algorithm.
%    Probably the sorting could be done more reliable.
%    However, the PDF specification is not too clear to me
%    regarding precise sorting rules (how to deal with
%    different encodings, escaped characters, \dots).
%    \begin{macrocode}
\def\EmFi@add#1#2{%
  \begingroup
    \ifx\EmFi@list\ltx@empty
      \xdef\EmFi@list{\noexpand\do{#1}{#2}}%
    \else
      \def\do##1##2{%
        \ifnum\pdf@strcmp{##1}{#1}>0 %
          \edef\x{%
            \toks@{%
              \the\toks@%
              \noexpand\do{#1}{#2}%
              \noexpand\do{##1}{##2}%
            }%
          }%
          \x
          \def\do####1####2{%
            \toks@\expandafter{\the\toks@\do{####1}{####2}}%
          }%
          \def\stop{%
            \xdef\EmFi@list{\the\toks@}%
          }%
        \else
          \toks@\expandafter{\the\toks@\do{##1}{##2}}%
        \fi
      }%
      \def\stop{%
        \xdef\EmFi@list{\the\toks@\noexpand\do{#1}{#2}}%
      }%
      \toks@{}%
      \EmFi@list\stop
    \fi
  \endgroup
}
%    \end{macrocode}
%    \end{macro}
%    \begin{macro}{\embedfilefinish}
%    \begin{macrocode}
\def\embedfilefinish{%
  \ifEmFi@finished
    \EmFi@Error{%
      Too many invocations of \string\embedfilefinish
    }{%
      The list of embedded files is already written.%
    }%
  \else
    \ifx\EmFi@list\ltx@empty
    \else
%    \end{macrocode}
%    Write |/EmbeddedFiles| entry.
%    \begin{macrocode}
      \global\EmFi@finishedtrue
      \begingroup
        \def\do##1##2{%
          (##1)##2%
        }%
        \immediate\pdfobj{%
          <<%
            /Names[\EmFi@list]%
          >>%
        }%
        \pdfnames{%
          /EmbeddedFiles \the\pdflastobj\ltx@space 0 R%
        }%
      \endgroup
%    \end{macrocode}
%    Write collection objects.
%    \begin{macrocode}
      \ifx\EmFi@initialfile\ltx@empty
      \else
        \EmFi@collectiontrue
      \fi
      \ifEmFi@collection
        \ifx\EmFi@initialfile\ltx@empty
          \let\EmFi@@initialfile\ltx@empty
        \else
          \edef\EmFi@@initialfile{%
            \pdf@escapestring{\EmFi@initialfile}%
          }%
        \fi
%    \end{macrocode}
%    Look for initial file among the embedded files.
%    \begin{macrocode}
        \begingroup
          \let\f=N%
          \def\do##1##2{%
            \def\x{##1}%
            \ifx\x\EmFi@@initialfile
              \let\f=Y%
              \let\do\ltx@gobbletwo
            \fi
          }%
          \EmFi@list
        \expandafter\endgroup
        \ifx\f Y%
        \else
          \@PackageWarningNoLine{embedfile}{%
            Missing initial file `\EmFi@initialfile'\MessageBreak
            among the embedded files%
          }%
          \let\EmFi@initialfile\ltx@empty
          \let\EmFi@@initialfile\ltx@empty
        \fi
        \ifcase\EmFi@sortcase
          \def\EmFi@temp{}%
        \or
          \def\EmFi@temp{%
            /S\EmFi@sortkeys
            /A \EmFi@sortorders
          }%
        \else
          \def\EmFi@temp{%
            /S[\EmFi@sortkeys]%
            /A[\EmFi@sortorders]%
          }%
        \fi
        \def\EmFi@@order##1{%
          \ifnum\EmFi@order>1 %
            /O ##1%
          \fi
        }%
        \immediate\pdfobj{%
          <<%
            \ifx\EmFi@schema\ltx@empty
            \else
              /Schema<<\EmFi@schema>>%
            \fi
            \ifx\EmFi@@initialfile\ltx@empty
            \else
              /D(\EmFi@@initialfile)%
            \fi
            \ifx\EmFi@view\EmFi@S@tile
              /View/T%
            \else\ifx\EmFi@view\EmFi@S@hidden
              /View/H%
            \fi\fi
            \ifx\EmFi@temp\ltx@empty
              \EmFi@temp
            \else
              /Sort<<\EmFi@temp>>%
            \fi
          >>%
        }%
        \pdfcatalog{%
          /Collection \the\pdflastobj\ltx@space0 R%
        }%
      \fi
    \fi
  \fi
}
%    \end{macrocode}
%    \end{macro}
%    \begin{macrocode}
\begingroup\expandafter\expandafter\expandafter\endgroup
\expandafter\ifx\csname AtEndDocument\endcsname\relax
\else
  \AtEndDocument{\embedfilefinish}%
\fi
%    \end{macrocode}
%
%    \begin{macrocode}
\EmFi@AtEnd%
%</package>
%    \end{macrocode}
%
% \section{Test}
%
% \subsection{Catcode checks for loading}
%
%    \begin{macrocode}
%<*test1>
%    \end{macrocode}
%    \begin{macrocode}
\catcode`\{=1 %
\catcode`\}=2 %
\catcode`\#=6 %
\catcode`\@=11 %
\expandafter\ifx\csname count@\endcsname\relax
  \countdef\count@=255 %
\fi
\expandafter\ifx\csname @gobble\endcsname\relax
  \long\def\@gobble#1{}%
\fi
\expandafter\ifx\csname @firstofone\endcsname\relax
  \long\def\@firstofone#1{#1}%
\fi
\expandafter\ifx\csname loop\endcsname\relax
  \expandafter\@firstofone
\else
  \expandafter\@gobble
\fi
{%
  \def\loop#1\repeat{%
    \def\body{#1}%
    \iterate
  }%
  \def\iterate{%
    \body
      \let\next\iterate
    \else
      \let\next\relax
    \fi
    \next
  }%
  \let\repeat=\fi
}%
\def\RestoreCatcodes{}
\count@=0 %
\loop
  \edef\RestoreCatcodes{%
    \RestoreCatcodes
    \catcode\the\count@=\the\catcode\count@\relax
  }%
\ifnum\count@<255 %
  \advance\count@ 1 %
\repeat

\def\RangeCatcodeInvalid#1#2{%
  \count@=#1\relax
  \loop
    \catcode\count@=15 %
  \ifnum\count@<#2\relax
    \advance\count@ 1 %
  \repeat
}
\def\RangeCatcodeCheck#1#2#3{%
  \count@=#1\relax
  \loop
    \ifnum#3=\catcode\count@
    \else
      \errmessage{%
        Character \the\count@\space
        with wrong catcode \the\catcode\count@\space
        instead of \number#3%
      }%
    \fi
  \ifnum\count@<#2\relax
    \advance\count@ 1 %
  \repeat
}
\def\space{ }
\expandafter\ifx\csname LoadCommand\endcsname\relax
  \def\LoadCommand{\input embedfile.sty\relax}%
\fi
\def\Test{%
  \RangeCatcodeInvalid{0}{47}%
  \RangeCatcodeInvalid{58}{64}%
  \RangeCatcodeInvalid{91}{96}%
  \RangeCatcodeInvalid{123}{255}%
  \catcode`\@=12 %
  \catcode`\\=0 %
  \catcode`\%=14 %
  \LoadCommand
  \RangeCatcodeCheck{0}{36}{15}%
  \RangeCatcodeCheck{37}{37}{14}%
  \RangeCatcodeCheck{38}{47}{15}%
  \RangeCatcodeCheck{48}{57}{12}%
  \RangeCatcodeCheck{58}{63}{15}%
  \RangeCatcodeCheck{64}{64}{12}%
  \RangeCatcodeCheck{65}{90}{11}%
  \RangeCatcodeCheck{91}{91}{15}%
  \RangeCatcodeCheck{92}{92}{0}%
  \RangeCatcodeCheck{93}{96}{15}%
  \RangeCatcodeCheck{97}{122}{11}%
  \RangeCatcodeCheck{123}{255}{15}%
  \RestoreCatcodes
}
\Test
\csname @@end\endcsname
\end
%    \end{macrocode}
%    \begin{macrocode}
%</test1>
%    \end{macrocode}
%
% \subsection{Simple test}
%
%    \begin{macrocode}
%<*test2>
\input embedfile.sty\relax
\embedfile[%
  stringmethod=escape,%
  mimetype=plain/text,%
  desc={LaTeX docstrip source archive for package `embedfile'},%
  id={embedfile.dtx}%
]{embedfile.dtx}
\nopagenumbers
Test (plain-\TeX): {\tt embedfile.dtx} should be embedded.%

\def\Test#1{%
  \par
  \embedfileifobjectexists{embedfile.dtx}{#1}{%
    Object #1 (embedfile.dtx): %
    \embedfilegetobject{embedfile.dtx}{#1}%
  }{%
    \errmessage{Missing object #1 (embedfile.dtx)}%
  }%
}
\Test{EmbeddedFile}
\Test{Filespec}
\embedfilefinish
\bye
%</test2>
%    \end{macrocode}
%    \begin{macrocode}
%<*test3>
\NeedsTeXFormat{LaTeX2e}
\let\SavedJobname\jobname
\def\jobname{embedfile}
\RequirePackage{dtx-attach}[2011/04/13]
\let\jobname\SavedJobname
\documentclass{minimal}
\begin{document}
  Test (\LaTeX): \texttt{embedfile.dtx} should be embedded.%
\end{document}
%</test3>
%    \end{macrocode}
%
% \subsection{Test for \iniTeX}
%
%    \begin{macrocode}
%<*test4>
\catcode`\{=1 %
\catcode`\}=2 %
\input ifluatex.sty %
\ifluatex
  \directlua{%
    tex.enableprimitives('', {%
      'pdflastobj',%
      'pdfnames',%
      'pdfobj',%
      'pdfoutput'%
    })%
  }%
\fi
\pdfoutput=1 %
\input embedfile.sty %
\shipout\hbox{}
\embedfile[%
  stringmethod=escape,%
  mimetype=plain/text,%
  desc={iniTeX source},%
]{\jobname.tex}
\embedfilefinish
\end
%</test4>
%    \end{macrocode}
%
% \section{Installation}
%
% \subsection{Download}
%
% \paragraph{Package.} This package is available on
% CTAN\footnote{\url{ftp://ftp.ctan.org/tex-archive/}}:
% \begin{description}
% \item[\CTAN{macros/latex/contrib/oberdiek/embedfile.dtx}] The source file.
% \item[\CTAN{macros/latex/contrib/oberdiek/embedfile.pdf}] Documentation.
% \end{description}
%
%
% \paragraph{Bundle.} All the packages of the bundle `oberdiek'
% are also available in a TDS compliant ZIP archive. There
% the packages are already unpacked and the documentation files
% are generated. The files and directories obey the TDS standard.
% \begin{description}
% \item[\CTAN{install/macros/latex/contrib/oberdiek.tds.zip}]
% \end{description}
% \emph{TDS} refers to the standard ``A Directory Structure
% for \TeX\ Files'' (\CTAN{tds/tds.pdf}). Directories
% with \xfile{texmf} in their name are usually organized this way.
%
% \subsection{Bundle installation}
%
% \paragraph{Unpacking.} Unpack the \xfile{oberdiek.tds.zip} in the
% TDS tree (also known as \xfile{texmf} tree) of your choice.
% Example (linux):
% \begin{quote}
%   |unzip oberdiek.tds.zip -d ~/texmf|
% \end{quote}
%
% \paragraph{Script installation.}
% Check the directory \xfile{TDS:scripts/oberdiek/} for
% scripts that need further installation steps.
% Package \xpackage{attachfile2} comes with the Perl script
% \xfile{pdfatfi.pl} that should be installed in such a way
% that it can be called as \texttt{pdfatfi}.
% Example (linux):
% \begin{quote}
%   |chmod +x scripts/oberdiek/pdfatfi.pl|\\
%   |cp scripts/oberdiek/pdfatfi.pl /usr/local/bin/|
% \end{quote}
%
% \subsection{Package installation}
%
% \paragraph{Unpacking.} The \xfile{.dtx} file is a self-extracting
% \docstrip\ archive. The files are extracted by running the
% \xfile{.dtx} through \plainTeX:
% \begin{quote}
%   \verb|tex embedfile.dtx|
% \end{quote}
%
% \paragraph{TDS.} Now the different files must be moved into
% the different directories in your installation TDS tree
% (also known as \xfile{texmf} tree):
% \begin{quote}
% \def\t{^^A
% \begin{tabular}{@{}>{\ttfamily}l@{ $\rightarrow$ }>{\ttfamily}l@{}}
%   embedfile.sty & tex/generic/oberdiek/embedfile.sty\\
%   dtx-attach.sty & tex/generic/oberdiek/dtx-attach.sty\\
%   embedfile.pdf & doc/latex/oberdiek/embedfile.pdf\\
%   embedfile-example-plain.tex & doc/latex/oberdiek/embedfile-example-plain.tex\\
%   embedfile-example-collection.tex & doc/latex/oberdiek/embedfile-example-collection.tex\\
%   test/embedfile-test1.tex & doc/latex/oberdiek/test/embedfile-test1.tex\\
%   test/embedfile-test2.tex & doc/latex/oberdiek/test/embedfile-test2.tex\\
%   test/embedfile-test3.tex & doc/latex/oberdiek/test/embedfile-test3.tex\\
%   test/embedfile-test4.tex & doc/latex/oberdiek/test/embedfile-test4.tex\\
%   embedfile.dtx & source/latex/oberdiek/embedfile.dtx\\
% \end{tabular}^^A
% }^^A
% \sbox0{\t}^^A
% \ifdim\wd0>\linewidth
%   \begingroup
%     \advance\linewidth by\leftmargin
%     \advance\linewidth by\rightmargin
%   \edef\x{\endgroup
%     \def\noexpand\lw{\the\linewidth}^^A
%   }\x
%   \def\lwbox{^^A
%     \leavevmode
%     \hbox to \linewidth{^^A
%       \kern-\leftmargin\relax
%       \hss
%       \usebox0
%       \hss
%       \kern-\rightmargin\relax
%     }^^A
%   }^^A
%   \ifdim\wd0>\lw
%     \sbox0{\small\t}^^A
%     \ifdim\wd0>\linewidth
%       \ifdim\wd0>\lw
%         \sbox0{\footnotesize\t}^^A
%         \ifdim\wd0>\linewidth
%           \ifdim\wd0>\lw
%             \sbox0{\scriptsize\t}^^A
%             \ifdim\wd0>\linewidth
%               \ifdim\wd0>\lw
%                 \sbox0{\tiny\t}^^A
%                 \ifdim\wd0>\linewidth
%                   \lwbox
%                 \else
%                   \usebox0
%                 \fi
%               \else
%                 \lwbox
%               \fi
%             \else
%               \usebox0
%             \fi
%           \else
%             \lwbox
%           \fi
%         \else
%           \usebox0
%         \fi
%       \else
%         \lwbox
%       \fi
%     \else
%       \usebox0
%     \fi
%   \else
%     \lwbox
%   \fi
% \else
%   \usebox0
% \fi
% \end{quote}
% If you have a \xfile{docstrip.cfg} that configures and enables \docstrip's
% TDS installing feature, then some files can already be in the right
% place, see the documentation of \docstrip.
%
% \subsection{Refresh file name databases}
%
% If your \TeX~distribution
% (\teTeX, \mikTeX, \dots) relies on file name databases, you must refresh
% these. For example, \teTeX\ users run \verb|texhash| or
% \verb|mktexlsr|.
%
% \subsection{Some details for the interested}
%
% \paragraph{Attached source.}
%
% The PDF documentation on CTAN also includes the
% \xfile{.dtx} source file. It can be extracted by
% AcrobatReader 6 or higher. Another option is \textsf{pdftk},
% e.g. unpack the file into the current directory:
% \begin{quote}
%   \verb|pdftk embedfile.pdf unpack_files output .|
% \end{quote}
%
% \paragraph{Unpacking with \LaTeX.}
% The \xfile{.dtx} chooses its action depending on the format:
% \begin{description}
% \item[\plainTeX:] Run \docstrip\ and extract the files.
% \item[\LaTeX:] Generate the documentation.
% \end{description}
% If you insist on using \LaTeX\ for \docstrip\ (really,
% \docstrip\ does not need \LaTeX), then inform the autodetect routine
% about your intention:
% \begin{quote}
%   \verb|latex \let\install=y% \iffalse meta-comment
%
% File: embedfile.dtx
% Version: 2011/04/13 v2.6
% Info: Embed files into PDF
%
% Copyright (C) 2006-2011 by
%    Heiko Oberdiek <heiko.oberdiek at googlemail.com>
%
% This work may be distributed and/or modified under the
% conditions of the LaTeX Project Public License, either
% version 1.3c of this license or (at your option) any later
% version. This version of this license is in
%    http://www.latex-project.org/lppl/lppl-1-3c.txt
% and the latest version of this license is in
%    http://www.latex-project.org/lppl.txt
% and version 1.3 or later is part of all distributions of
% LaTeX version 2005/12/01 or later.
%
% This work has the LPPL maintenance status "maintained".
%
% This Current Maintainer of this work is Heiko Oberdiek.
%
% The Base Interpreter refers to any `TeX-Format',
% because some files are installed in TDS:tex/generic//.
%
% This work consists of the main source file embedfile.dtx
% and the derived files
%    embedfile.sty, embedfile.pdf, embedfile.ins, embedfile.drv,
%    dtx-attach.sty, embedfile-example-plain.tex,
%    embedfile-example-collection.tex, embedfile-test1.tex,
%    embedfile-test2.tex, embedfile-test3.tex,
%    embedfile-test4.tex.
%
% Distribution:
%    CTAN:macros/latex/contrib/oberdiek/embedfile.dtx
%    CTAN:macros/latex/contrib/oberdiek/embedfile.pdf
%
% Unpacking:
%    (a) If embedfile.ins is present:
%           tex embedfile.ins
%    (b) Without embedfile.ins:
%           tex embedfile.dtx
%    (c) If you insist on using LaTeX
%           latex \let\install=y\input{embedfile.dtx}
%        (quote the arguments according to the demands of your shell)
%
% Documentation:
%    (a) If embedfile.drv is present:
%           latex embedfile.drv
%    (b) Without embedfile.drv:
%           latex embedfile.dtx; ...
%    The class ltxdoc loads the configuration file ltxdoc.cfg
%    if available. Here you can specify further options, e.g.
%    use A4 as paper format:
%       \PassOptionsToClass{a4paper}{article}
%
%    Programm calls to get the documentation (example):
%       pdflatex embedfile.dtx
%       makeindex -s gind.ist embedfile.idx
%       pdflatex embedfile.dtx
%       makeindex -s gind.ist embedfile.idx
%       pdflatex embedfile.dtx
%
% Installation:
%    TDS:tex/generic/oberdiek/embedfile.sty
%    TDS:tex/generic/oberdiek/dtx-attach.sty
%    TDS:doc/latex/oberdiek/embedfile.pdf
%    TDS:doc/latex/oberdiek/embedfile-example-plain.tex
%    TDS:doc/latex/oberdiek/embedfile-example-collection.tex
%    TDS:doc/latex/oberdiek/test/embedfile-test1.tex
%    TDS:doc/latex/oberdiek/test/embedfile-test2.tex
%    TDS:doc/latex/oberdiek/test/embedfile-test3.tex
%    TDS:doc/latex/oberdiek/test/embedfile-test4.tex
%    TDS:source/latex/oberdiek/embedfile.dtx
%
%<*ignore>
\begingroup
  \catcode123=1 %
  \catcode125=2 %
  \def\x{LaTeX2e}%
\expandafter\endgroup
\ifcase 0\ifx\install y1\fi\expandafter
         \ifx\csname processbatchFile\endcsname\relax\else1\fi
         \ifx\fmtname\x\else 1\fi\relax
\else\csname fi\endcsname
%</ignore>
%<*install>
\input docstrip.tex
\Msg{************************************************************************}
\Msg{* Installation}
\Msg{* Package: embedfile 2011/04/13 v2.6 Embed files into PDF (HO)}
\Msg{************************************************************************}

\keepsilent
\askforoverwritefalse

\let\MetaPrefix\relax
\preamble

This is a generated file.

Project: embedfile
Version: 2011/04/13 v2.6

Copyright (C) 2006-2011 by
   Heiko Oberdiek <heiko.oberdiek at googlemail.com>

This work may be distributed and/or modified under the
conditions of the LaTeX Project Public License, either
version 1.3c of this license or (at your option) any later
version. This version of this license is in
   http://www.latex-project.org/lppl/lppl-1-3c.txt
and the latest version of this license is in
   http://www.latex-project.org/lppl.txt
and version 1.3 or later is part of all distributions of
LaTeX version 2005/12/01 or later.

This work has the LPPL maintenance status "maintained".

This Current Maintainer of this work is Heiko Oberdiek.

The Base Interpreter refers to any `TeX-Format',
because some files are installed in TDS:tex/generic//.

This work consists of the main source file embedfile.dtx
and the derived files
   embedfile.sty, embedfile.pdf, embedfile.ins, embedfile.drv,
   dtx-attach.sty, embedfile-example-plain.tex,
   embedfile-example-collection.tex, embedfile-test1.tex,
   embedfile-test2.tex, embedfile-test3.tex,
   embedfile-test4.tex.

\endpreamble
\let\MetaPrefix\DoubleperCent

\generate{%
  \file{embedfile.ins}{\from{embedfile.dtx}{install}}%
  \file{embedfile.drv}{\from{embedfile.dtx}{driver}}%
  \usedir{tex/generic/oberdiek}%
  \file{embedfile.sty}{\from{embedfile.dtx}{package}}%
  \usedir{tex/latex/oberdiek}%
  \file{dtx-attach.sty}{\from{embedfile.dtx}{dtxattach}}%
  \usedir{doc/latex/oberdiek}%
  \file{embedfile-example-plain.tex}{\from{embedfile.dtx}{exampleplain}}%
  \file{embedfile-example-collection.tex}{\from{embedfile.dtx}{examplecollection}}%
  \usedir{doc/latex/oberdiek/test}%
  \file{embedfile-test1.tex}{\from{embedfile.dtx}{test1}}%
  \file{embedfile-test2.tex}{\from{embedfile.dtx}{test2}}%
  \file{embedfile-test3.tex}{\from{embedfile.dtx}{test3}}%
  \file{embedfile-test4.tex}{\from{embedfile.dtx}{test4}}%
  \nopreamble
  \nopostamble
  \usedir{source/latex/oberdiek/catalogue}%
  \file{embedfile.xml}{\from{embedfile.dtx}{catalogue}}%
}

\catcode32=13\relax% active space
\let =\space%
\Msg{************************************************************************}
\Msg{*}
\Msg{* To finish the installation you have to move the following}
\Msg{* files into a directory searched by TeX:}
\Msg{*}
\Msg{*     embedfile.sty, dtx-attach.sty}
\Msg{*}
\Msg{* To produce the documentation run the file `embedfile.drv'}
\Msg{* through LaTeX.}
\Msg{*}
\Msg{* Happy TeXing!}
\Msg{*}
\Msg{************************************************************************}

\endbatchfile
%</install>
%<*ignore>
\fi
%</ignore>
%<*driver>
\NeedsTeXFormat{LaTeX2e}
\ProvidesFile{embedfile.drv}%
  [2011/04/13 v2.6 Embed files into PDF (HO)]%
\documentclass{ltxdoc}
\usepackage{holtxdoc}[2011/11/22]
\begin{document}
  \DocInput{embedfile.dtx}%
\end{document}
%</driver>
% \fi
%
% \CheckSum{1363}
%
% \CharacterTable
%  {Upper-case    \A\B\C\D\E\F\G\H\I\J\K\L\M\N\O\P\Q\R\S\T\U\V\W\X\Y\Z
%   Lower-case    \a\b\c\d\e\f\g\h\i\j\k\l\m\n\o\p\q\r\s\t\u\v\w\x\y\z
%   Digits        \0\1\2\3\4\5\6\7\8\9
%   Exclamation   \!     Double quote  \"     Hash (number) \#
%   Dollar        \$     Percent       \%     Ampersand     \&
%   Acute accent  \'     Left paren    \(     Right paren   \)
%   Asterisk      \*     Plus          \+     Comma         \,
%   Minus         \-     Point         \.     Solidus       \/
%   Colon         \:     Semicolon     \;     Less than     \<
%   Equals        \=     Greater than  \>     Question mark \?
%   Commercial at \@     Left bracket  \[     Backslash     \\
%   Right bracket \]     Circumflex    \^     Underscore    \_
%   Grave accent  \`     Left brace    \{     Vertical bar  \|
%   Right brace   \}     Tilde         \~}
%
% \GetFileInfo{embedfile.drv}
%
% \title{The \xpackage{embedfile} package}
% \date{2011/04/13 v2.6}
% \author{Heiko Oberdiek\\\xemail{heiko.oberdiek at googlemail.com}}
%
% \maketitle
%
% \begin{abstract}
% This package embeds files to a PDF document.
% Currently the only supported driver is \pdfTeX\ $>=$ 1.30 in PDF mode.
% \end{abstract}
%
% \tableofcontents
%
% \section{Documentation}
%
% \subsection{Introduction}
%
%    The PDF format (\cite{pdfspec}) allows the inclusion of files
%    inside the PDF document. The included files can be bound to an
%    annotation on a page. Or they can be recorded in a sorted
%    list of embedded files. The packages \xpackage{attachfile}
%    or \xpackage{attachfile2} follow the first approach, this
%    package uses the latter method.
%
% \subsubsection{Future development}
%
%    My dream is a large package that merges the features of
%    all these packages meantioned before:
%    \begin{itemize}
%    \item Files can be attached to a page.
%    \item Files can be attached to the document.
%    \item An easy user interface for simple, common tasks and
%          beginners.
%    \item An interface for the advanced users that want to setup
%          every detail.
%    \item Support of many drivers (pdftex, dvips, dvipdfm, \dots).
%    \item \dots
%    \end{itemize}
%    However, I have not managed to take the time for this project.
%    Instead:
%    \begin{itemize}
%    \item First I experimented with package \xpackage{attachfile},
%          adding driver support, fixing bugs, \dots. The result is
%          currently named as \xpackage{attachfile2}. It uses an external
%          script to get file properties (size, date, checksum, \dots).
%    \item In order to avoid an external program for getting basic
%          file properties I provided a patch ``EscapeAndOther'' for
%          pdfTeX that was accepted for version 1.30.
%    \item This package closes a gap left by the packages for attaching
%          files and allows the embedding of files to the document.
%          Also it makes use of the new primitives of \pdfTeX.
%    \end{itemize}
%
% \subsection{User interface}
%
%    This package \xpackage{embedfile} can be used with both
%    \LaTeX\ and \plainTeX. See \autoref{sec:plain} that
%    explains the use with \plainTeX\ by an example.
%    In \LaTeX\ the package is loaded as usually. There are no options.
%    \begin{quote}
%      \verb|\usepackage{embedfile}|
%    \end{quote}
%
%    \begin{declcs}{embedfile}
%      |[|\meta{options}|]| \M{file}
%    \end{declcs}
%    The macro \cs{embedfile} includes file \meta{file} and
%    attaches it to the PDF document. At the end of the document
%    the sorted list of embedded files are written. Thus you can
%    safely use \cs{embedfile} before \verb|\end{document}|.
%    Embedding files using \cs{AtEndDocument} will only work,
%    if \cs{AtEndDocument} is called before loading the package
%    \xpackage{embedfile}.
%
%    The \meta{options} are give as key value pairs. The following
%    keys are supported:
%    \begin{description}
%    \item[\xoption{filespec}]
%      This allows to override the file name
%      that appears in the PDF file. If you are using other than
%      simple file names (8-bit, path separators, \dots), look into
%      the PDF specification (\cite{pdfspec}).
%      There are rules how these file names must be written/encoded.
%      Avoid 8-bit characters
%      and other special characters, the behaviour is currently undefined.
%      Use option \xoption{ucfilespec} for more funny file names.
%      The string method, see below, is |escape| since version 2.4.
%
%      This name is also used as entry in a name tree
%      (see PDF specification: |/EmbeddedFiles|). Therefore the
%      value for \xoption{filespec} must be unique among all
%      embedded files. Also key \xoption{initialfiles} refers
%      to this name, if the file name and the value of \xoption{filespec}
%      are different.
%    \item[\xoption{ucfilespec}]
%      Since PDF 1.7 the file name may be provided in Unicode.
%      The conversion of the option value into a PDF string
%      is controlled by option \xoption{stringmethod}.
%    \item[\xoption{filesystem}]
%      This sets the entry \verb|/FS| in
%      the file specification dictionary, see PDF specification
%      (\cite{pdfspec}).
%      Example: \verb|filesystem=URL|.
%    \item[\xoption{mimetype}]
%      This sets the mime type (\cite{mime}) of the file,
%      see \autoref{sec:plain} for examples and \cite{types}
%      for a list of officially registered types.
%    \item[\xoption{desc}]
%      The description for the file.
%    \item[\xoption{stringmethod}]
%      The package must convert the values of the keys \xoption{ucfilespec}
%      and \xoption{desc} into a PDF string (before version 2.4: \xoption{filespec}
%      and \xoption{desc}). If \xpackage{hyperref}
%      is found, then its \cs{pdfstringdef} will be used, otherwise
%      \pdfTeX's \cs{pdfescapestring} is used. Value \verb|psd|
%      forces the use of \cs{pdfstringdef}, value \verb|escape|
%      the use of \cs{pdfescapestring}.
%    \item[\xoption{\meta{key}.value}] Sets the value of a collection
%      item property, see section \ref{sec:collection}.
%    \item[\xoption{\meta{key}.prefix}]
%      Sets the prefix of a collection item property,
%      see section \ref{sec:collection}.
%    \item[\xoption{id}]
%      The value must be an unique name. Macros
%      \cs{embedfileifobjectexists} and \cs{embedfilegetobject}
%      are using this name later.
%    \end{description}
%
%    \begin{declcs}{embedfilefinish}
%    \end{declcs}
%    The list of all embedded files must be added as data structure
%    in the PDF file. In case of \LaTeX\ this is automatically done.
%    The package uses \cs{AtEndDocument}. Then the list of all
%    files should be known. However, \plainTeX\ does not know
%    about \cs{AtEndDocument}. Thus the user must call
%    \cs{embedfilefinish} at the end of the document
%    after the last file is embedded.
%
%    \begin{declcs}{embedfilesetup} \M{options}
%    \end{declcs}
%    Options for \cs{embedfile} and collection support
%    can be set in \cs{embedfilesetup}.
%
% \subsection{Collection support (PDF 1.7)}
% \label{sec:collection}
%
%    Since PDF 1.7 the embedded files can form a \emph{collection}
%    (sometimes referred as \emph{package}), the main document
%    is called \emph{cover sheet}. See PDF specification
%    8.2.4 ``Collections'' and 3.10.5 ``Collection items'' \cite{pdfspec}.
%
%    Usually Acrobat Reader 7 or 8 shows the
%    embedded files in a table at the bottom with the following columns:
%    \begin{quote}
%    \begin{tabular}{@{}l|l|l|l@{}}
%       Name & Description & Modified & Size\\
%       \hline
%       \dots & \dots & \dots & \dots\\
%    \end{tabular}
%    \end{quote}
%    Acrobat Reader 10 shows the embeded files in the left panel
%    and adds a new column for the compressed size.
%
%    If the files form a collection, then they are displayed
%    in a table left or top (depending on option \xoption{view},
%    see \cs{embedfilesetup}).
%
%
%    Collection support is enabled automatically, if
%    it is used.
%
%    \begin{declcs}{embedfilesetup} \M{options}
%    \end{declcs}
%    The following options are supported in addition
%    to options for \cs{embedfile}:
%    \begin{description}
%    \item[\xoption{view}]
%      If the PDF file contains a collection, then Acrobat Reader 8
%      shows a line at the top below the menu bar and the toolbar.
%      It shows the current selected file, icons for changing the
%      view mode, an options menu.
%      The initial mode how the collection is presented is
%      set by this option \xoption{view}.
%      The following modes/values are supported, the default is |details|:
%        \begin{description}
%        \item[\texttt{details}] The full collection table is displayed
%          at the top below the collection bar.
%        \item[\texttt{tile}] The files of the collection are shown
%          in tile mode on the left.
%        \item[\texttt{hidden}] The collection table is not shown.
%        \end{description}
%    \item[\xoption{initialfile}]
%      Selects the file that is initially presented. Especially
%      useful for an embedded PDF file that is then shown instead
%      of the cover document. There must be an \cs{embedfile} command
%      somewhere whose value for key \xoption{filespec} is used here.
%      The \cs{embedfile} command can drop option \xoption{filespec}
%      if the file name is not different.
%    \end{description}
%
%    \begin{declcs}{embedfilefield} \M{key} \M{options}
%    \end{declcs}
%    Macro \cs{embedfilefield} defines a column/field in the collection table.
%    The name of the field is \meta{key}.
%    \begin{description}
%    \item[\xoption{type}]
%      sets the type of the field. The supported values are:
%      \begin{description}
%      \item[\texttt{text}] A text field. Its value is set in
%        \cs{embedfile} by option \xoption{\meta{key}.value}.
%      \item[\texttt{date}] A date field. Its value is set in
%        \cs{embedfile} by option \xoption{\meta{key}.value}.
%        A special format is required, see ``3.8.3 Dates'' \cite{pdfspec}.
%      \item[\texttt{number}] A field with an integer or float number.
%        Its value is set in \cs{embedfile} by option
%        \xoption{\meta{key}.value}.
%      \item[\texttt{file}] The file name of the embedded file.
%      \item[\texttt{desc}] The description text of the embedded file. It is
%       set in \cs{embedfile} by option \xoption{desc}.
%      \item[\texttt{moddate}] The modification date of the embedded file.
%      \item[\texttt{size}] The size of the embedded file.
%      \end{description}
%      All types allow the use of a prefix that is disregarded by
%      sorting. The prefix for this field is set in \cs{embedfile} by
%      option \xoption{\meta{key}.prefix}.
%    \item[\xoption{title}] sets the column title.
%    \item[\xoption{visible}] controls whether the column is presented:
%      \begin{description}
%      \item[\texttt{true}] shows the column.
%      \item[\texttt{false}] hides the column.
%      \end{description}
%      Default: \texttt{true}
%    \item[\xoption{edit}]
%      Allows the editing of field values. Does not seem to have an
%      effect for Acrobat Reader.
%      \begin{description}
%      \item[\texttt{true}] enables the feature, if available (depends
%        on the PDF viewer).
%      \item[\texttt{false}] disables the feature.
%      \end{description}
%      Default: \texttt{false}
%    \end{description}
%    The order of \cs{embedfilefield} statements defines the order
%    of the columns.
%
%    \begin{declcs}{embedfilesort} \M{key-sort-list}
%    \end{declcs}
%    The sort order of the embedded files are controlled by
%    macro \cs{embedfilesort}. \meta{key-sort-list} defines
%    the sort order. The key is a field name defined by \cs{embedfilefield}.
%    Its value is either \texttt{ascending} or \texttt{descending}.
%    The default is \texttt{ascending}.
%
% \subsection{Export of object references}
%
%    Caution: This feature is still experimental. It may be even removed
%    in future versions. Therefore feedback would be nice, if someone
%    has a useful application for this feature.
%
%    Object numbers are saved, if \xoption{id} is given in \cs{embedfile}.
%    The following objects are supported:
%    \begin{itemize}
%    \item \texttt{EmbeddedFile}
%    \item \texttt{Filespec}
%    \end{itemize}
%
%    \begin{declcs}{embedfileifobjectexists} \M{id} \M{type}
%      \M{then} \M{else}
%    \end{declcs}
%    Macro \cs{embedfileifobjectexists} tests whether object of \meta{type}
%    is available for the embedded file identified by \meta{id}.
%
%    \begin{declcs}{embedfilegetobject} \M{id} \M{type}
%    \end{declcs}
%    Macro \cs{embedfilegetobject} expands to the full object reference
%    object of \meta{type} for the embedded file identified by \meta{id}.
%
% \subsubsection{Example}
%
%\begin{quote}
%\begin{verbatim}
%\embedfile[id={foo}]{foo.pdf}
%\embedfileifobjectexists{foo}{Filespec}{%
%  \typeout{%
%    FileSpec object for `foo': %
%    \embedfilegetobject{foo}{Filespec}%
%  }%
%}{%
%  \typeout{No Filespec object for `foo'}%
%}
%\end{verbatim}
%\end{quote}
%
% \subsection{Examples}
% \label{sec:examples}
%
% \subsubsection{\plainTeX}
% \label{sec:plain}
%
%    The package can be used with \plainTeX. It can be used
%    with or without help from \xfile{miniltx.tex}.
%
%    If additionally package \xpackage{keyval} (\xpackage{graphicx})
%    is needed, load it first. Then package \xpackage{embedfile} avoids
%    a duplicate loading of package \xpackage{keyval}.
%
%    Because \plainTeX\ does not provide a hook at end of the document,
%    you have to call \cs{embedfilefinish} manually at the end
%    after the last embedded file.
%    \begin{macrocode}
%<*exampleplain>
%<<END
% Load packages
\input miniltx
% \def\Gin@driver{pdftex.def}
% \input graphicx.sty
\input embedfile.sty
\resetatcatcode

% default setting
\embedfilesetup{
  mimetype=text/plain
}

% Embed files
\embedfile[
  filespec=example.tex,
  desc={Source code (plain-TeX) of this example}
]{embedfile-example-plain.tex}

\embedfile[
  desc={Source of package `embedfile'}
]{embedfile.dtx}

\embedfile[
  mimetype=application/pdf,
  desc={Documentation of package `embedfile'}
]{embedfile.pdf}

% Some text
This example document contains three embedded files.

% End of document
\embedfilefinish % don't forget
\bye
%END
%</exampleplain>
%    \end{macrocode}
%
% \subsubsection{Collection example}
%
%    \begin{macrocode}
%<*examplecollection>
%<<END
\NeedsTeXFormat{LaTeX2e}
\documentclass{article}
\usepackage[bookmarks=false]{hyperref}
% provides \pdfstringdef that is then used by `title' and
% other keys.
\usepackage{embedfile}[2011/04/13]
\embedfilesetup{
  view=details,
  initialfile=embedfile.pdf
}
\embedfilefield{file}{
  type=file,
  title={File name}
}
\embedfilefield{description}{
  type=desc,
  title={Description}
}
\embedfilefield{date}{
  type=moddate,
  title={Date}
}
\embedfilefield{size}{
  type=size,
  title={Size}
}
\embedfilefield{type}{
  type=text,
  title={Type},
  visible=false
}
\embedfilesort{
  type,
  date=descending
}
\begin{document}
An example for embedded files as collection.
You need Acrobat Reader 8 or higher.

\embedfile[
  desc={Source file of package `embedfile'},
  description.prefix={Package: },
  type.value={DTX}
]{embedfile.dtx}

\embedfile[
  desc={Documentation of package `embedfile'},
  description.prefix={Package: },
  type.value={PDF}
]{embedfile.pdf}

\embedfile[
  desc={The source for this example},
  description.prefix={Example: },
  type.value={TEX}
]{\jobname.tex}

\end{document}
%END
%</examplecollection>
%    \end{macrocode}
%
% \subsection{Package \xpackage{dtx-attach}}
%
% Package \xpackage{dtx-attach} is just a small application of
% package \xpackage{embedfile}. I am using it for the CTAN
% documentation of my packages in \CTAN{macros/latex/contrib/oberdiek/}.
% It also serves as small example for the use of the package with \LaTeX.
%
%    \begin{macrocode}
%<*dtxattach>
\NeedsTeXFormat{LaTeX2e}
\ProvidesPackage{dtx-attach}
  [2011/04/13 v2.6 Embed \string\jobname.dtx (HO)]%
\RequirePackage{embedfile}[2011/04/13]
\embedfile[%
  stringmethod=escape,%
  mimetype=plain/text,%
  desc={LaTeX docstrip source archive for package `\jobname'}%
]{\jobname.dtx}
%</dtxattach>
%    \end{macrocode}
%
% \StopEventually{
% }
%
% \section{Implementation}
%
%    \begin{macrocode}
%<*package>
%    \end{macrocode}
%
% \subsection{Reload check and package identification}
%    Reload check, especially if the package is not used with \LaTeX.
%    \begin{macrocode}
\begingroup\catcode61\catcode48\catcode32=10\relax%
  \catcode13=5 % ^^M
  \endlinechar=13 %
  \catcode35=6 % #
  \catcode39=12 % '
  \catcode44=12 % ,
  \catcode45=12 % -
  \catcode46=12 % .
  \catcode58=12 % :
  \catcode64=11 % @
  \catcode123=1 % {
  \catcode125=2 % }
  \expandafter\let\expandafter\x\csname ver@embedfile.sty\endcsname
  \ifx\x\relax % plain-TeX, first loading
  \else
    \def\empty{}%
    \ifx\x\empty % LaTeX, first loading,
      % variable is initialized, but \ProvidesPackage not yet seen
    \else
      \expandafter\ifx\csname PackageInfo\endcsname\relax
        \def\x#1#2{%
          \immediate\write-1{Package #1 Info: #2.}%
        }%
      \else
        \def\x#1#2{\PackageInfo{#1}{#2, stopped}}%
      \fi
      \x{embedfile}{The package is already loaded}%
      \aftergroup\endinput
    \fi
  \fi
\endgroup%
%    \end{macrocode}
%    Package identification:
%    \begin{macrocode}
\begingroup\catcode61\catcode48\catcode32=10\relax%
  \catcode13=5 % ^^M
  \endlinechar=13 %
  \catcode35=6 % #
  \catcode39=12 % '
  \catcode40=12 % (
  \catcode41=12 % )
  \catcode44=12 % ,
  \catcode45=12 % -
  \catcode46=12 % .
  \catcode47=12 % /
  \catcode58=12 % :
  \catcode64=11 % @
  \catcode91=12 % [
  \catcode93=12 % ]
  \catcode123=1 % {
  \catcode125=2 % }
  \expandafter\ifx\csname ProvidesPackage\endcsname\relax
    \def\x#1#2#3[#4]{\endgroup
      \immediate\write-1{Package: #3 #4}%
      \xdef#1{#4}%
    }%
  \else
    \def\x#1#2[#3]{\endgroup
      #2[{#3}]%
      \ifx#1\@undefined
        \xdef#1{#3}%
      \fi
      \ifx#1\relax
        \xdef#1{#3}%
      \fi
    }%
  \fi
\expandafter\x\csname ver@embedfile.sty\endcsname
\ProvidesPackage{embedfile}%
  [2011/04/13 v2.6 Embed files into PDF (HO)]%
%    \end{macrocode}
%
% \subsection{Catcodes}
%
%    \begin{macrocode}
\begingroup\catcode61\catcode48\catcode32=10\relax%
  \catcode13=5 % ^^M
  \endlinechar=13 %
  \catcode123=1 % {
  \catcode125=2 % }
  \catcode64=11 % @
  \def\x{\endgroup
    \expandafter\edef\csname EmFi@AtEnd\endcsname{%
      \endlinechar=\the\endlinechar\relax
      \catcode13=\the\catcode13\relax
      \catcode32=\the\catcode32\relax
      \catcode35=\the\catcode35\relax
      \catcode61=\the\catcode61\relax
      \catcode64=\the\catcode64\relax
      \catcode123=\the\catcode123\relax
      \catcode125=\the\catcode125\relax
    }%
  }%
\x\catcode61\catcode48\catcode32=10\relax%
\catcode13=5 % ^^M
\endlinechar=13 %
\catcode35=6 % #
\catcode64=11 % @
\catcode123=1 % {
\catcode125=2 % }
\def\TMP@EnsureCode#1#2{%
  \edef\EmFi@AtEnd{%
    \EmFi@AtEnd
    \catcode#1=\the\catcode#1\relax
  }%
  \catcode#1=#2\relax
}
\TMP@EnsureCode{39}{12}% '
\TMP@EnsureCode{40}{12}% (
\TMP@EnsureCode{41}{12}% )
\TMP@EnsureCode{44}{12}% ,
\TMP@EnsureCode{46}{12}% .
\TMP@EnsureCode{47}{12}% /
\TMP@EnsureCode{58}{12}% :
\TMP@EnsureCode{60}{12}% <
\TMP@EnsureCode{62}{12}% >
\TMP@EnsureCode{91}{12}% [
\TMP@EnsureCode{93}{12}% ]
\TMP@EnsureCode{96}{12}% `
\edef\EmFi@AtEnd{\EmFi@AtEnd\noexpand\endinput}
%    \end{macrocode}
%
% \subsection{Tools}
%
%    \begin{macro}{\EmFi@RequirePackage}
%    \begin{macrocode}
\begingroup\expandafter\expandafter\expandafter\endgroup
\expandafter\ifx\csname RequirePackage\endcsname\relax
  \def\EmFi@RequirePackage#1[#2]{%
    \input #1.sty\relax
  }%
\else
  \let\EmFi@RequirePackage\RequirePackage
\fi
%    \end{macrocode}
%    \end{macro}
%
%    \begin{macro}{\EmFi@Error}
%    \begin{macrocode}
\EmFi@RequirePackage{infwarerr}[2007/09/09]%
\def\EmFi@Error{%
  \@PackageError{embedfile}%
}
%    \end{macrocode}
%    \end{macro}
%
% \subsection{Check for recent \pdfTeX\ in PDF mode}
%
%    Load package \xpackage{ifpdf} and check mode.
%    \begin{macrocode}
\EmFi@RequirePackage{ifpdf}[2007/09/09]
\ifpdf
\else
  \EmFi@Error{%
    Missing pdfTeX in PDF mode%
  }{%
    Currently other drivers are not supported. %
    Package loading is aborted.%
  }%
  \expandafter\EmFi@AtEnd
\fi%
%    \end{macrocode}
%    \begin{macrocode}
\EmFi@RequirePackage{pdftexcmds}[2007/11/11]
\EmFi@RequirePackage{ltxcmds}[2010/03/01]
\EmFi@RequirePackage{kvsetkeys}[2010/03/01]
\EmFi@RequirePackage{kvdefinekeys}[2010/03/01]
%    \end{macrocode}
%    Check version.
%    \begin{macrocode}
\begingroup\expandafter\expandafter\expandafter\endgroup
\expandafter\ifx\csname pdf@filesize\endcsname\relax
  \EmFi@Error{%
    Unsupported pdfTeX version%
  }{%
    At least version 1.30 is necessary. Package loading is aborted.%
  }%
  \expandafter\EmFi@AtEnd
\fi%
%    \end{macrocode}
%
% \subsection{Strings}
%
%    Minimal version of package \xpackage{pdfescape} is 2007/08/27 v1.5
%    because of \cs{EdefSanitize}.
%    \begin{macrocode}
\EmFi@RequirePackage{pdfescape}[2007/11/11]
%    \end{macrocode}
%
%    \begin{macrocode}
\def\EmFi@temp#1{%
  \expandafter\EdefSanitize\csname EmFi@S@#1\endcsname{#1}%
}
%    \end{macrocode}
%    \begin{macro}{\EmFi@details}
%    \begin{macrocode}
\EmFi@temp{details}%
%    \end{macrocode}
%    \end{macro}
%    \begin{macro}{\EmFi@tile}
%    \begin{macrocode}
\EmFi@temp{tile}%
%    \end{macrocode}
%    \end{macro}
%    \begin{macro}{\EmFi@hidden}
%    \begin{macrocode}
\EmFi@temp{hidden}%
%    \end{macrocode}
%    \end{macro}
%    \begin{macro}{\EmFi@S@text}
%    \begin{macrocode}
\EmFi@temp{text}
%    \end{macrocode}
%    \end{macro}
%    \begin{macro}{\EmFi@S@date}
%    \begin{macrocode}
\EmFi@temp{date}
%    \end{macrocode}
%    \end{macro}
%    \begin{macro}{\EmFi@S@number}
%    \begin{macrocode}
\EmFi@temp{number}
%    \end{macrocode}
%    \end{macro}
%    \begin{macro}{\EmFi@S@file}
%    \begin{macrocode}
\EmFi@temp{file}
%    \end{macrocode}
%    \end{macro}
%    \begin{macro}{\EmFi@S@desc}
%    \begin{macrocode}
\EmFi@temp{desc}
%    \end{macrocode}
%    \end{macro}
%    \begin{macro}{\EmFi@S@moddate}
%    \begin{macrocode}
\EmFi@temp{moddate}
%    \end{macrocode}
%    \end{macro}
%    \begin{macro}{\EmFi@S@creationdate}
%    \begin{macrocode}
\EmFi@temp{creationdate}
%    \end{macrocode}
%    \end{macro}
%    \begin{macro}{\EmFi@S@size}
%    \begin{macrocode}
\EmFi@temp{size}
%    \end{macrocode}
%    \end{macro}
%    \begin{macro}{\EmFi@S@ascending}
%    \begin{macrocode}
\EmFi@temp{ascending}
%    \end{macrocode}
%    \end{macro}
%    \begin{macro}{\EmFi@S@descending}
%    \begin{macrocode}
\EmFi@temp{descending}
%    \end{macrocode}
%    \end{macro}
%    \begin{macro}{\EmFi@S@true}
%    \begin{macrocode}
\EmFi@temp{true}
%    \end{macrocode}
%    \end{macro}
%    \begin{macro}{\EmFi@S@false}
%    \begin{macrocode}
\EmFi@temp{false}
%    \end{macrocode}
%    \end{macro}
%
% \subsection{Switches}
%
%    \begin{macro}{\ifEmFi@collection}
%    \begin{macrocode}
\ltx@newif\ifEmFi@collection
%    \end{macrocode}
%    \end{macro}
%    \begin{macro}{\ifEmFi@sort}
%    \begin{macrocode}
\ltx@newif\ifEmFi@sort
%    \end{macrocode}
%    \end{macro}
%    \begin{macro}{\ifEmFi@visible}
%    \begin{macrocode}
\ltx@newif\ifEmFi@visible
%    \end{macrocode}
%    \end{macro}
%    \begin{macro}{\ifEmFi@edit}
%    \begin{macrocode}
\ltx@newif\ifEmFi@edit
%    \end{macrocode}
%    \end{macro}
%    \begin{macro}{\ifEmFi@item}
%    \begin{macrocode}
\ltx@newif\ifEmFi@item
%    \end{macrocode}
%    \end{macro}
%    \begin{macro}{\ifEmFi@finished}
%    \begin{macrocode}
\ltx@newif\ifEmFi@finished
%    \end{macrocode}
%    \end{macro}
%    \begin{macro}{\ifEmFi@id}
%    \begin{macrocode}
\ltx@newif\ifEmFi@id
%    \end{macrocode}
%    \end{macro}
%
% \subsection{Key value definitions}
%
%    \begin{macro}{\EmFi@GlobalKey}
%    \begin{macrocode}
\def\EmFi@GlobalKey#1#2{%
  \global\expandafter\let\csname KV@#1@#2\expandafter\endcsname
                         \csname KV@#1@#2\endcsname
}
%    \end{macrocode}
%    \end{macro}
%    \begin{macro}{\EmFi@GlobalDefaultKey}
%    \begin{macrocode}
\def\EmFi@GlobalDefaultKey#1#2{%
  \EmFi@GlobalKey{#1}{#2}%
  \global\expandafter\let
      \csname KV@#1@#2@default\expandafter\endcsname
      \csname KV@#1@#2@default\endcsname
}
%    \end{macrocode}
%    \end{macro}
%
%    \begin{macro}{\EmFi@DefineKey}
%    \begin{macrocode}
\def\EmFi@DefineKey#1#2{%
  \kv@define@key{EmFi}{#1}{%
    \expandafter\def\csname EmFi@#1\endcsname{##1}%
  }%
  \expandafter\def\csname EmFi@#1\endcsname{#2}%
}
%    \end{macrocode}
%    \end{macro}
%    Subtype of the embedded file (optional).
%    \begin{macrocode}
\EmFi@DefineKey{mimetype}{}
%    \end{macrocode}
%    File specification string.
%    \begin{macrocode}
\EmFi@DefineKey{filespec}{\EmFi@file}
%    \end{macrocode}
%    File specification string in Unicode.
%    \begin{macrocode}
\EmFi@DefineKey{ucfilespec}{}
%    \end{macrocode}
%    File system (optional).
%    \begin{macrocode}
\EmFi@DefineKey{filesystem}{}
%    \end{macrocode}
%    Description (optional).
%    \begin{macrocode}
\EmFi@DefineKey{desc}{}
%    \end{macrocode}
%    Method for converting text to PDF strings.
%    \begin{macrocode}
\EmFi@DefineKey{stringmethod}{%
  \ifx\pdfstringdef\@undefined
    escape%
  \else
    \ifx\pdfstringdef\relax
      escape%
    \else
      psd%
    \fi
  \fi
}
%    \end{macrocode}
%
%    Option \xoption{id} as key for object numbers.
%    \begin{macrocode}
\kv@define@key{EmFi}{id}{%
  \def\EmFi@id{#1}%
  \EmFi@idtrue
}
%    \end{macrocode}
%    \begin{macro}{\EmFi@defobj}
%    \begin{macrocode}
\def\EmFi@defobj#1{%
  \ifEmFi@id
    \expandafter\xdef\csname EmFi@#1@\EmFi@id\endcsname{%
      \the\pdflastobj\ltx@space 0 R%
    }%
  \fi
}
%    \end{macrocode}
%    \end{macro}
%    \begin{macro}{\embedfileifobjectexists}
%    \begin{macrocode}
\def\embedfileifobjectexists#1#2{%
  \expandafter\ifx\csname EmFi@#2@#1\endcsname\relax
    \expandafter\ltx@secondoftwo
  \else
    \expandafter\ltx@firstoftwo
  \fi
}
%    \end{macrocode}
%    \end{macro}
%    \begin{macro}{\embedfilegetobject}
%    \begin{macrocode}
\def\embedfilegetobject#1#2{%
  \embedfileifobjectexists{#1}{#2}{%
    \csname EmFi@#2@#1\endcsname
  }{%
    0 0 R%
  }%
}
%    \end{macrocode}
%    \end{macro}
%
%    Initial view of the collection.
%    \begin{macrocode}
\kv@define@key{EmFi}{view}[]{%
  \EdefSanitize\EmFi@temp{#1}%
  \def\EmFi@next{%
    \global\EmFi@collectiontrue
  }%
  \ifx\EmFi@temp\ltx@empty
    \let\EmFi@view\EmFi@S@details
  \else\ifx\EmFi@temp\EmFi@S@details
    \let\EmFi@view\EmFi@S@details
  \else\ifx\EmFi@temp\EmFi@S@tile
    \let\EmFi@view\EmFi@S@tile
  \else\ifx\EmFi@temp\EmFi@S@hidden
    \let\EmFi@view\EmFi@S@hidden
  \else
    \let\EmFi@next\relax
    \EmFi@Error{%
      Unknown value `\EmFi@temp' for key `view'.\MessageBreak
      Supported values: `details', `tile', `hidden'.%
    }\@ehc
  \fi\fi\fi\fi
  \EmFi@next
}
%    \end{macrocode}
%    \begin{macrocode}
\EmFi@DefineKey{initialfile}{}
%    \end{macrocode}
%
%    \begin{macro}{\embedfilesetup}
%    \begin{macrocode}
\def\embedfilesetup{%
  \ifEmFi@finished
    \def\EmFi@next##1{}%
    \EmFi@Error{%
      \string\embedfilefield\ltx@space after \string\embedfilefinish
    }{%
      The list of embedded files is already written.%
    }%
  \else
    \def\EmFi@next{%
      \kvsetkeys{EmFi}%
    }%
  \fi
  \EmFi@next
}
%    \end{macrocode}
%    \end{macro}
%
%    \begin{macro}{\EmFi@schema}
%    \begin{macrocode}
\def\EmFi@schema{}
%    \end{macrocode}
%    \end{macro}
%    \begin{macro}{\EmFi@order}
%    \begin{macrocode}
\gdef\EmFi@order{0}
%    \end{macrocode}
%    \end{macro}
%    \begin{macro}{\EmFi@@order}
%    \begin{macrocode}
\let\EmFi@@order\relax
%    \end{macrocode}
%    \end{macro}
%    \begin{macro}{\EmFi@fieldlist}
%    \begin{macrocode}
\def\EmFi@fieldlist{}
%    \end{macrocode}
%    \end{macro}
%    \begin{macro}{\EmFi@sortcase}
%    \begin{macrocode}
\def\EmFi@sortcase{0}%
%    \end{macrocode}
%    \end{macro}
%    \begin{macro}{\embedfilefield}
%    \begin{macrocode}
\def\embedfilefield#1#2{%
  \ifEmFi@finished
    \EmFi@Error{%
      \string\embedfilefield\ltx@space after \string\embedfilefinish
    }{%
      The list of embedded files is already written.%
    }%
  \else
    \global\EmFi@collectiontrue
    \EdefSanitize\EmFi@key{#1}%
    \expandafter\ifx\csname KV@EmFi@\EmFi@key.prefix\endcsname\relax
      \begingroup
        \count@=\EmFi@order
        \advance\count@ 1 %
        \xdef\EmFi@order{\the\count@}%
        \let\EmFi@title\EmFi@key
        \let\EmFi@type\EmFi@S@text
        \EmFi@visibletrue
        \EmFi@editfalse
        \kvsetkeys{EmFiFi}{#2}%
        \EmFi@convert\EmFi@title\EmFi@title
        \xdef\EmFi@schema{%
          \EmFi@schema
          /\pdf@escapename{\EmFi@key}<<%
            /Subtype/%
            \ifx\EmFi@type\EmFi@S@date D%
            \else\ifx\EmFi@type\EmFi@S@number N%
            \else\ifx\EmFi@type\EmFi@S@file F%
            \else\ifx\EmFi@type\EmFi@S@desc Desc%
            \else\ifx\EmFi@type\EmFi@S@moddate ModDate%
            \else\ifx\EmFi@type\EmFi@S@creationdate CreationDate%
            \else\ifx\EmFi@type\EmFi@S@size Size%
            \else S%
            \fi\fi\fi\fi\fi\fi\fi
            /N(\EmFi@title)%
            \EmFi@@order{\EmFi@order}%
            \ifEmFi@visible
            \else
              /V false%
            \fi
            \ifEmFi@edit
              /E true%
            \fi
          >>%
        }%
        \let\do\relax
        \xdef\EmFi@fieldlist{%
          \EmFi@fieldlist
          \do{\EmFi@key}%
        }%
        \ifx\EmFi@type\EmFi@S@text
          \kv@define@key{EmFi}{\EmFi@key.value}{%
            \EmFi@itemtrue
            \def\EmFi@temp{##1}%
            \EmFi@convert\EmFi@temp\EmFi@temp
            \expandafter\def\csname EmFi@V@#1%
            \expandafter\endcsname\expandafter{%
              \expandafter(\EmFi@temp)%
            }%
          }%
          \EmFi@GlobalKey{EmFi}{\EmFi@key.value}%
        \else\ifx\EmFi@type\EmFi@S@date
          \kv@define@key{EmFi}{\EmFi@key.value}{%
            \EmFi@itemtrue
            \def\EmFi@temp{##1}%
            \EmFi@convert\EmFi@temp\EmFi@temp
            \expandafter\def\csname EmFi@V@#1%
            \expandafter\endcsname\expandafter{%
              \expandafter(\EmFi@temp)%
            }%
          }%
          \EmFi@GlobalKey{EmFi}{\EmFi@key.value}%
        \else\ifx\EmFi@type\EmFi@S@number
          \kv@define@key{EmFi}{\EmFi@key.value}{%
            \EmFi@itemtrue
            \expandafter\EdefSanitize\csname EmFi@V@#1\endcsname{ ##1}%
          }%
          \EmFi@GlobalKey{EmFi}{\EmFi@key.value}%
        \fi\fi\fi
        \kv@define@key{EmFi}{\EmFi@key.prefix}{%
          \EmFi@itemtrue
          \expandafter\def\csname EmFi@P@#1\endcsname{##1}%
        }%
        \EmFi@GlobalKey{EmFi}{\EmFi@key.prefix}%
        \kv@define@key{EmFiSo}{\EmFi@key}[ascending]{%
          \EdefSanitize\EmFi@temp{##1}%
          \ifx\EmFi@temp\EmFi@S@ascending
            \def\EmFi@temp{true}%
          \else\ifx\EmFi@temp\EmFi@S@descending
            \def\EmFi@temp{false}%
          \else
            \def\EmFi@temp{}%
            \EmFi@Error{%
              Unknown sort order `\EmFi@temp'.\MessageBreak
              Supported values: `\EmFi@S@ascending', %
              `\EmFi@S@descending
            }\@ehc
          \fi\fi
          \ifx\EmFi@temp\ltx@empty
          \else
            \xdef\EmFi@sortkeys{%
              \EmFi@sortkeys
              /\pdf@escapename{#1}%
            }%
            \ifx\EmFi@sortorders\ltx@empty
              \global\let\EmFi@sortorders\EmFi@temp
              \gdef\EmFi@sortcase{1}%
            \else
              \xdef\EmFi@sortorders{%
                \EmFi@sortorders
                \ltx@space
                \EmFi@temp
              }%
              \xdef\EmFi@sortcase{2}%
            \fi
          \fi
        }%
        \EmFi@GlobalDefaultKey{EmFiSo}\EmFi@key
      \endgroup
    \else
      \EmFi@Error{%
        Field `\EmFi@key' is already defined%
      }\@ehc
    \fi
  \fi
}
%    \end{macrocode}
%    \end{macro}
%    \begin{macrocode}
\kv@define@key{EmFiFi}{type}{%
  \EdefSanitize\EmFi@temp{#1}%
  \ifx\EmFi@temp\EmFi@S@text
    \let\EmFi@type\EmFi@temp
  \else\ifx\EmFi@temp\EmFi@S@date
    \let\EmFi@type\EmFi@temp
  \else\ifx\EmFi@temp\EmFi@S@number
    \let\EmFi@type\EmFi@temp
  \else\ifx\EmFi@temp\EmFi@S@file
    \let\EmFi@type\EmFi@temp
  \else\ifx\EmFi@temp\EmFi@S@desc
    \let\EmFi@type\EmFi@temp
  \else\ifx\EmFi@temp\EmFi@S@moddate
    \let\EmFi@type\EmFi@temp
  \else\ifx\EmFi@temp\EmFi@S@creationdate
    \let\EmFi@type\EmFi@temp
  \else\ifx\EmFi@temp\EmFi@S@size
    \let\EmFi@type\EmFi@temp
  \else
    \EmFi@Error{%
      Unknown type `\EmFi@temp'.\MessageBreak
      Supported types: `text', `date', `number', `file',\MessageBreak
      `desc', `moddate', `creationdate', `size'%
    }%
  \fi\fi\fi\fi\fi\fi\fi\fi
}
%    \end{macrocode}
%    \begin{macrocode}
\kv@define@key{EmFiFi}{title}{%
  \def\EmFi@title{#1}%
}
%    \end{macrocode}
%    \begin{macro}{\EmFi@setboolean}
%    \begin{macrocode}
\def\EmFi@setboolean#1#2{%
  \EdefSanitize\EmFi@temp{#2}%
  \ifx\EmFi@temp\EmFi@S@true
    \csname EmFi@#1true\endcsname
  \else
    \ifx\EmFi@temp\EmFi@S@false
      \csname EmFi@#1false\endcsname
    \else
      \EmFi@Error{%
        Unknown value `\EmFi@temp' for key `#1'.\MessageBreak
        Supported values: `true', `false'%
      }\@ehc
    \fi
  \fi
}
%    \end{macrocode}
%    \end{macro}
%    \begin{macrocode}
\kv@define@key{EmFiFi}{visible}[true]{%
  \EmFi@setboolean{visible}{#1}%
}
%    \end{macrocode}
%    \begin{macrocode}
\kv@define@key{EmFiFi}{edit}[true]{%
  \EmFi@setboolean{edit}{#1}%
}
%    \end{macrocode}
%
%    \begin{macro}{\EmFi@sortkeys}
%    \begin{macrocode}
\def\EmFi@sortkeys{}
%    \end{macrocode}
%    \end{macro}
%    \begin{macro}{\EmFi@sortorders}
%    \begin{macrocode}
\def\EmFi@sortorders{}
%    \end{macrocode}
%    \end{macro}
%    \begin{macro}{\embedfilesort}
%    \begin{macrocode}
\def\embedfilesort{%
  \kvsetkeys{EmFiSo}%
}
%    \end{macrocode}
%    \end{macro}
%
% \subsection{Embed the file}
%
%    \begin{macro}{\embedfile}
%    \begin{macrocode}
\def\embedfile{%
  \ltx@ifnextchar[\EmFi@embedfile{\EmFi@embedfile[]}%
}
%    \end{macrocode}
%    \end{macro}
%    \begin{macro}{\EmFi@embedfile}
%    \begin{macrocode}
\def\EmFi@embedfile[#1]#2{%
  \ifEmFi@finished
    \EmFi@Error{%
      \string\embedfile\ltx@space after \string\embedfilefinish
    }{%
      The list of embedded files is already written.%
    }%
  \else
    \begingroup
      \def\EmFi@file{#2}%
      \kvsetkeys{EmFi}{#1}%
      \expandafter\expandafter\expandafter
      \ifx\expandafter\expandafter\expandafter
          \\\pdf@filesize{\EmFi@file}\\%
        \EmFi@Error{%
          File `\EmFi@file' not found%
        }{%
          The unknown file is not embedded.%
        }%
      \else
        \edef\EmFi@@filespec{%
          \pdf@escapestring{\EmFi@filespec}%
        }%
        \ifx\EmFi@ucfilespec\ltx@empty
          \let\EmFi@@ucfilespec\ltx@empty
        \else
          \EmFi@convert\EmFi@ucfilespec\EmFi@@ucfilespec
        \fi
        \ifx\EmFi@desc\ltx@empty
          \let\EmFi@@desc\ltx@empty
        \else
          \EmFi@convert\EmFi@desc\EmFi@@desc
        \fi
        \ifEmFi@item
          \let\do\EmFi@do
          \immediate\pdfobj{%
            <<%
              \EmFi@fieldlist
            >>%
          }%
          \edef\EmFi@ci{\the\pdflastobj}%
        \fi
        \immediate\pdfobj stream attr{%
          /Type/EmbeddedFile%
          \ifx\EmFi@mimetype\ltx@empty
          \else
            /Subtype/\pdf@escapename{\EmFi@mimetype}%
          \fi
          /Params<<%
            /ModDate(\pdf@filemoddate{\EmFi@file})%
            /Size \pdf@filesize{\EmFi@file}%
            /CheckSum<\pdf@filemdfivesum{\EmFi@file}>%
          >>%
        }file{\EmFi@file}\relax
        \EmFi@defobj{EmbeddedFile}%
        \immediate\pdfobj{%
          <<%
            /Type/Filespec%
            \ifx\EmFi@filesystem\ltx@empty
            \else
            /FS/\pdf@escapename{\EmFi@filesystem}%
            \fi
            /F(\EmFi@@filespec)%
            \ifx\EmFi@@ucfilespec\ltx@empty
            \else
              /UF(\EmFi@@ucfilespec)%
            \fi
            \ifx\EmFi@@desc\ltx@empty
            \else
              /Desc(\EmFi@@desc)%
            \fi
            /EF<<%
              /F \the\pdflastobj\ltx@space 0 R%
            >>%
            \ifEmFi@item
              /CI \EmFi@ci\ltx@space 0 R%
            \fi
          >>%
        }%
        \EmFi@defobj{Filespec}%
        \EmFi@add{%
          \EmFi@@filespec
        }{\the\pdflastobj\ltx@space 0 R}%
      \fi
    \endgroup
  \fi
}
%    \end{macrocode}
%    \end{macro}
%    \begin{macro}{\EmFi@do}
%    \begin{macrocode}
\def\EmFi@do#1{%
  \expandafter\ifx\csname EmFi@P@#1\endcsname\relax
    \expandafter\ifx\csname EmFi@V@#1\endcsname\relax
    \else
      /\pdf@escapename{#1}\csname EmFi@V@#1\endcsname
    \fi
  \else
    /\pdf@escapename{#1}<<%
      \expandafter\ifx\csname EmFi@V@#1\endcsname\relax
      \else
        /D\csname EmFi@V@#1\endcsname
      \fi
      /P(\csname EmFi@P@#1\endcsname)%
    >>%
  \fi
}
%    \end{macrocode}
%    \end{macro}
%    \begin{macro}{\EmFi@convert}
%    \begin{macrocode}
\def\EmFi@convert#1#2{%
  \ifnum\pdf@strcmp{\EmFi@stringmethod}{psd}=0 %
    \pdfstringdef\EmFi@temp{#1}%
    \let#2\EmFi@temp
  \else
    \edef#2{\pdf@escapestring{#1}}%
  \fi
}
%    \end{macrocode}
%    \end{macro}
%    \begin{macrocode}
\global\let\EmFi@list\ltx@empty
%    \end{macrocode}
%    \begin{macro}{\EmFi@add}
%    Sorting is done by the insertion sort algorithm.
%    Probably the sorting could be done more reliable.
%    However, the PDF specification is not too clear to me
%    regarding precise sorting rules (how to deal with
%    different encodings, escaped characters, \dots).
%    \begin{macrocode}
\def\EmFi@add#1#2{%
  \begingroup
    \ifx\EmFi@list\ltx@empty
      \xdef\EmFi@list{\noexpand\do{#1}{#2}}%
    \else
      \def\do##1##2{%
        \ifnum\pdf@strcmp{##1}{#1}>0 %
          \edef\x{%
            \toks@{%
              \the\toks@%
              \noexpand\do{#1}{#2}%
              \noexpand\do{##1}{##2}%
            }%
          }%
          \x
          \def\do####1####2{%
            \toks@\expandafter{\the\toks@\do{####1}{####2}}%
          }%
          \def\stop{%
            \xdef\EmFi@list{\the\toks@}%
          }%
        \else
          \toks@\expandafter{\the\toks@\do{##1}{##2}}%
        \fi
      }%
      \def\stop{%
        \xdef\EmFi@list{\the\toks@\noexpand\do{#1}{#2}}%
      }%
      \toks@{}%
      \EmFi@list\stop
    \fi
  \endgroup
}
%    \end{macrocode}
%    \end{macro}
%    \begin{macro}{\embedfilefinish}
%    \begin{macrocode}
\def\embedfilefinish{%
  \ifEmFi@finished
    \EmFi@Error{%
      Too many invocations of \string\embedfilefinish
    }{%
      The list of embedded files is already written.%
    }%
  \else
    \ifx\EmFi@list\ltx@empty
    \else
%    \end{macrocode}
%    Write |/EmbeddedFiles| entry.
%    \begin{macrocode}
      \global\EmFi@finishedtrue
      \begingroup
        \def\do##1##2{%
          (##1)##2%
        }%
        \immediate\pdfobj{%
          <<%
            /Names[\EmFi@list]%
          >>%
        }%
        \pdfnames{%
          /EmbeddedFiles \the\pdflastobj\ltx@space 0 R%
        }%
      \endgroup
%    \end{macrocode}
%    Write collection objects.
%    \begin{macrocode}
      \ifx\EmFi@initialfile\ltx@empty
      \else
        \EmFi@collectiontrue
      \fi
      \ifEmFi@collection
        \ifx\EmFi@initialfile\ltx@empty
          \let\EmFi@@initialfile\ltx@empty
        \else
          \edef\EmFi@@initialfile{%
            \pdf@escapestring{\EmFi@initialfile}%
          }%
        \fi
%    \end{macrocode}
%    Look for initial file among the embedded files.
%    \begin{macrocode}
        \begingroup
          \let\f=N%
          \def\do##1##2{%
            \def\x{##1}%
            \ifx\x\EmFi@@initialfile
              \let\f=Y%
              \let\do\ltx@gobbletwo
            \fi
          }%
          \EmFi@list
        \expandafter\endgroup
        \ifx\f Y%
        \else
          \@PackageWarningNoLine{embedfile}{%
            Missing initial file `\EmFi@initialfile'\MessageBreak
            among the embedded files%
          }%
          \let\EmFi@initialfile\ltx@empty
          \let\EmFi@@initialfile\ltx@empty
        \fi
        \ifcase\EmFi@sortcase
          \def\EmFi@temp{}%
        \or
          \def\EmFi@temp{%
            /S\EmFi@sortkeys
            /A \EmFi@sortorders
          }%
        \else
          \def\EmFi@temp{%
            /S[\EmFi@sortkeys]%
            /A[\EmFi@sortorders]%
          }%
        \fi
        \def\EmFi@@order##1{%
          \ifnum\EmFi@order>1 %
            /O ##1%
          \fi
        }%
        \immediate\pdfobj{%
          <<%
            \ifx\EmFi@schema\ltx@empty
            \else
              /Schema<<\EmFi@schema>>%
            \fi
            \ifx\EmFi@@initialfile\ltx@empty
            \else
              /D(\EmFi@@initialfile)%
            \fi
            \ifx\EmFi@view\EmFi@S@tile
              /View/T%
            \else\ifx\EmFi@view\EmFi@S@hidden
              /View/H%
            \fi\fi
            \ifx\EmFi@temp\ltx@empty
              \EmFi@temp
            \else
              /Sort<<\EmFi@temp>>%
            \fi
          >>%
        }%
        \pdfcatalog{%
          /Collection \the\pdflastobj\ltx@space0 R%
        }%
      \fi
    \fi
  \fi
}
%    \end{macrocode}
%    \end{macro}
%    \begin{macrocode}
\begingroup\expandafter\expandafter\expandafter\endgroup
\expandafter\ifx\csname AtEndDocument\endcsname\relax
\else
  \AtEndDocument{\embedfilefinish}%
\fi
%    \end{macrocode}
%
%    \begin{macrocode}
\EmFi@AtEnd%
%</package>
%    \end{macrocode}
%
% \section{Test}
%
% \subsection{Catcode checks for loading}
%
%    \begin{macrocode}
%<*test1>
%    \end{macrocode}
%    \begin{macrocode}
\catcode`\{=1 %
\catcode`\}=2 %
\catcode`\#=6 %
\catcode`\@=11 %
\expandafter\ifx\csname count@\endcsname\relax
  \countdef\count@=255 %
\fi
\expandafter\ifx\csname @gobble\endcsname\relax
  \long\def\@gobble#1{}%
\fi
\expandafter\ifx\csname @firstofone\endcsname\relax
  \long\def\@firstofone#1{#1}%
\fi
\expandafter\ifx\csname loop\endcsname\relax
  \expandafter\@firstofone
\else
  \expandafter\@gobble
\fi
{%
  \def\loop#1\repeat{%
    \def\body{#1}%
    \iterate
  }%
  \def\iterate{%
    \body
      \let\next\iterate
    \else
      \let\next\relax
    \fi
    \next
  }%
  \let\repeat=\fi
}%
\def\RestoreCatcodes{}
\count@=0 %
\loop
  \edef\RestoreCatcodes{%
    \RestoreCatcodes
    \catcode\the\count@=\the\catcode\count@\relax
  }%
\ifnum\count@<255 %
  \advance\count@ 1 %
\repeat

\def\RangeCatcodeInvalid#1#2{%
  \count@=#1\relax
  \loop
    \catcode\count@=15 %
  \ifnum\count@<#2\relax
    \advance\count@ 1 %
  \repeat
}
\def\RangeCatcodeCheck#1#2#3{%
  \count@=#1\relax
  \loop
    \ifnum#3=\catcode\count@
    \else
      \errmessage{%
        Character \the\count@\space
        with wrong catcode \the\catcode\count@\space
        instead of \number#3%
      }%
    \fi
  \ifnum\count@<#2\relax
    \advance\count@ 1 %
  \repeat
}
\def\space{ }
\expandafter\ifx\csname LoadCommand\endcsname\relax
  \def\LoadCommand{\input embedfile.sty\relax}%
\fi
\def\Test{%
  \RangeCatcodeInvalid{0}{47}%
  \RangeCatcodeInvalid{58}{64}%
  \RangeCatcodeInvalid{91}{96}%
  \RangeCatcodeInvalid{123}{255}%
  \catcode`\@=12 %
  \catcode`\\=0 %
  \catcode`\%=14 %
  \LoadCommand
  \RangeCatcodeCheck{0}{36}{15}%
  \RangeCatcodeCheck{37}{37}{14}%
  \RangeCatcodeCheck{38}{47}{15}%
  \RangeCatcodeCheck{48}{57}{12}%
  \RangeCatcodeCheck{58}{63}{15}%
  \RangeCatcodeCheck{64}{64}{12}%
  \RangeCatcodeCheck{65}{90}{11}%
  \RangeCatcodeCheck{91}{91}{15}%
  \RangeCatcodeCheck{92}{92}{0}%
  \RangeCatcodeCheck{93}{96}{15}%
  \RangeCatcodeCheck{97}{122}{11}%
  \RangeCatcodeCheck{123}{255}{15}%
  \RestoreCatcodes
}
\Test
\csname @@end\endcsname
\end
%    \end{macrocode}
%    \begin{macrocode}
%</test1>
%    \end{macrocode}
%
% \subsection{Simple test}
%
%    \begin{macrocode}
%<*test2>
\input embedfile.sty\relax
\embedfile[%
  stringmethod=escape,%
  mimetype=plain/text,%
  desc={LaTeX docstrip source archive for package `embedfile'},%
  id={embedfile.dtx}%
]{embedfile.dtx}
\nopagenumbers
Test (plain-\TeX): {\tt embedfile.dtx} should be embedded.%

\def\Test#1{%
  \par
  \embedfileifobjectexists{embedfile.dtx}{#1}{%
    Object #1 (embedfile.dtx): %
    \embedfilegetobject{embedfile.dtx}{#1}%
  }{%
    \errmessage{Missing object #1 (embedfile.dtx)}%
  }%
}
\Test{EmbeddedFile}
\Test{Filespec}
\embedfilefinish
\bye
%</test2>
%    \end{macrocode}
%    \begin{macrocode}
%<*test3>
\NeedsTeXFormat{LaTeX2e}
\let\SavedJobname\jobname
\def\jobname{embedfile}
\RequirePackage{dtx-attach}[2011/04/13]
\let\jobname\SavedJobname
\documentclass{minimal}
\begin{document}
  Test (\LaTeX): \texttt{embedfile.dtx} should be embedded.%
\end{document}
%</test3>
%    \end{macrocode}
%
% \subsection{Test for \iniTeX}
%
%    \begin{macrocode}
%<*test4>
\catcode`\{=1 %
\catcode`\}=2 %
\input ifluatex.sty %
\ifluatex
  \directlua{%
    tex.enableprimitives('', {%
      'pdflastobj',%
      'pdfnames',%
      'pdfobj',%
      'pdfoutput'%
    })%
  }%
\fi
\pdfoutput=1 %
\input embedfile.sty %
\shipout\hbox{}
\embedfile[%
  stringmethod=escape,%
  mimetype=plain/text,%
  desc={iniTeX source},%
]{\jobname.tex}
\embedfilefinish
\end
%</test4>
%    \end{macrocode}
%
% \section{Installation}
%
% \subsection{Download}
%
% \paragraph{Package.} This package is available on
% CTAN\footnote{\url{ftp://ftp.ctan.org/tex-archive/}}:
% \begin{description}
% \item[\CTAN{macros/latex/contrib/oberdiek/embedfile.dtx}] The source file.
% \item[\CTAN{macros/latex/contrib/oberdiek/embedfile.pdf}] Documentation.
% \end{description}
%
%
% \paragraph{Bundle.} All the packages of the bundle `oberdiek'
% are also available in a TDS compliant ZIP archive. There
% the packages are already unpacked and the documentation files
% are generated. The files and directories obey the TDS standard.
% \begin{description}
% \item[\CTAN{install/macros/latex/contrib/oberdiek.tds.zip}]
% \end{description}
% \emph{TDS} refers to the standard ``A Directory Structure
% for \TeX\ Files'' (\CTAN{tds/tds.pdf}). Directories
% with \xfile{texmf} in their name are usually organized this way.
%
% \subsection{Bundle installation}
%
% \paragraph{Unpacking.} Unpack the \xfile{oberdiek.tds.zip} in the
% TDS tree (also known as \xfile{texmf} tree) of your choice.
% Example (linux):
% \begin{quote}
%   |unzip oberdiek.tds.zip -d ~/texmf|
% \end{quote}
%
% \paragraph{Script installation.}
% Check the directory \xfile{TDS:scripts/oberdiek/} for
% scripts that need further installation steps.
% Package \xpackage{attachfile2} comes with the Perl script
% \xfile{pdfatfi.pl} that should be installed in such a way
% that it can be called as \texttt{pdfatfi}.
% Example (linux):
% \begin{quote}
%   |chmod +x scripts/oberdiek/pdfatfi.pl|\\
%   |cp scripts/oberdiek/pdfatfi.pl /usr/local/bin/|
% \end{quote}
%
% \subsection{Package installation}
%
% \paragraph{Unpacking.} The \xfile{.dtx} file is a self-extracting
% \docstrip\ archive. The files are extracted by running the
% \xfile{.dtx} through \plainTeX:
% \begin{quote}
%   \verb|tex embedfile.dtx|
% \end{quote}
%
% \paragraph{TDS.} Now the different files must be moved into
% the different directories in your installation TDS tree
% (also known as \xfile{texmf} tree):
% \begin{quote}
% \def\t{^^A
% \begin{tabular}{@{}>{\ttfamily}l@{ $\rightarrow$ }>{\ttfamily}l@{}}
%   embedfile.sty & tex/generic/oberdiek/embedfile.sty\\
%   dtx-attach.sty & tex/generic/oberdiek/dtx-attach.sty\\
%   embedfile.pdf & doc/latex/oberdiek/embedfile.pdf\\
%   embedfile-example-plain.tex & doc/latex/oberdiek/embedfile-example-plain.tex\\
%   embedfile-example-collection.tex & doc/latex/oberdiek/embedfile-example-collection.tex\\
%   test/embedfile-test1.tex & doc/latex/oberdiek/test/embedfile-test1.tex\\
%   test/embedfile-test2.tex & doc/latex/oberdiek/test/embedfile-test2.tex\\
%   test/embedfile-test3.tex & doc/latex/oberdiek/test/embedfile-test3.tex\\
%   test/embedfile-test4.tex & doc/latex/oberdiek/test/embedfile-test4.tex\\
%   embedfile.dtx & source/latex/oberdiek/embedfile.dtx\\
% \end{tabular}^^A
% }^^A
% \sbox0{\t}^^A
% \ifdim\wd0>\linewidth
%   \begingroup
%     \advance\linewidth by\leftmargin
%     \advance\linewidth by\rightmargin
%   \edef\x{\endgroup
%     \def\noexpand\lw{\the\linewidth}^^A
%   }\x
%   \def\lwbox{^^A
%     \leavevmode
%     \hbox to \linewidth{^^A
%       \kern-\leftmargin\relax
%       \hss
%       \usebox0
%       \hss
%       \kern-\rightmargin\relax
%     }^^A
%   }^^A
%   \ifdim\wd0>\lw
%     \sbox0{\small\t}^^A
%     \ifdim\wd0>\linewidth
%       \ifdim\wd0>\lw
%         \sbox0{\footnotesize\t}^^A
%         \ifdim\wd0>\linewidth
%           \ifdim\wd0>\lw
%             \sbox0{\scriptsize\t}^^A
%             \ifdim\wd0>\linewidth
%               \ifdim\wd0>\lw
%                 \sbox0{\tiny\t}^^A
%                 \ifdim\wd0>\linewidth
%                   \lwbox
%                 \else
%                   \usebox0
%                 \fi
%               \else
%                 \lwbox
%               \fi
%             \else
%               \usebox0
%             \fi
%           \else
%             \lwbox
%           \fi
%         \else
%           \usebox0
%         \fi
%       \else
%         \lwbox
%       \fi
%     \else
%       \usebox0
%     \fi
%   \else
%     \lwbox
%   \fi
% \else
%   \usebox0
% \fi
% \end{quote}
% If you have a \xfile{docstrip.cfg} that configures and enables \docstrip's
% TDS installing feature, then some files can already be in the right
% place, see the documentation of \docstrip.
%
% \subsection{Refresh file name databases}
%
% If your \TeX~distribution
% (\teTeX, \mikTeX, \dots) relies on file name databases, you must refresh
% these. For example, \teTeX\ users run \verb|texhash| or
% \verb|mktexlsr|.
%
% \subsection{Some details for the interested}
%
% \paragraph{Attached source.}
%
% The PDF documentation on CTAN also includes the
% \xfile{.dtx} source file. It can be extracted by
% AcrobatReader 6 or higher. Another option is \textsf{pdftk},
% e.g. unpack the file into the current directory:
% \begin{quote}
%   \verb|pdftk embedfile.pdf unpack_files output .|
% \end{quote}
%
% \paragraph{Unpacking with \LaTeX.}
% The \xfile{.dtx} chooses its action depending on the format:
% \begin{description}
% \item[\plainTeX:] Run \docstrip\ and extract the files.
% \item[\LaTeX:] Generate the documentation.
% \end{description}
% If you insist on using \LaTeX\ for \docstrip\ (really,
% \docstrip\ does not need \LaTeX), then inform the autodetect routine
% about your intention:
% \begin{quote}
%   \verb|latex \let\install=y\input{embedfile.dtx}|
% \end{quote}
% Do not forget to quote the argument according to the demands
% of your shell.
%
% \paragraph{Generating the documentation.}
% You can use both the \xfile{.dtx} or the \xfile{.drv} to generate
% the documentation. The process can be configured by the
% configuration file \xfile{ltxdoc.cfg}. For instance, put this
% line into this file, if you want to have A4 as paper format:
% \begin{quote}
%   \verb|\PassOptionsToClass{a4paper}{article}|
% \end{quote}
% An example follows how to generate the
% documentation with pdf\LaTeX:
% \begin{quote}
%\begin{verbatim}
%pdflatex embedfile.dtx
%makeindex -s gind.ist embedfile.idx
%pdflatex embedfile.dtx
%makeindex -s gind.ist embedfile.idx
%pdflatex embedfile.dtx
%\end{verbatim}
% \end{quote}
%
% \section{Catalogue}
%
% The following XML file can be used as source for the
% \href{http://mirror.ctan.org/help/Catalogue/catalogue.html}{\TeX\ Catalogue}.
% The elements \texttt{caption} and \texttt{description} are imported
% from the original XML file from the Catalogue.
% The name of the XML file in the Catalogue is \xfile{embedfile.xml}.
%    \begin{macrocode}
%<*catalogue>
<?xml version='1.0' encoding='us-ascii'?>
<!DOCTYPE entry SYSTEM 'catalogue.dtd'>
<entry datestamp='$Date$' modifier='$Author$' id='embedfile'>
  <name>embedfile</name>
  <caption>Embed files into PDF.</caption>
  <authorref id='auth:oberdiek'/>
  <copyright owner='Heiko Oberdiek' year='2006-2011'/>
  <license type='lppl1.3'/>
  <version number='2.6'/>
  <description>
    This package embeds files in a PDF document, using the PDF
    format's embedding operation (note the contrast with the attach
    operation used by the <xref refid='attachfile'>attachfile</xref>
    and <xref refid='attachfile2'>attachfile2</xref> packages).
    Currently only <xref refid='pdftex'>pdfTeX</xref> &gt;=1.30, in
    PDF mode, is supported.

    <p/>
    The package is part of the <xref refid='oberdiek'>oberdiek</xref>
    bundle.
  </description>
  <documentation details='Package documentation'
      href='ctan:/macros/latex/contrib/oberdiek/embedfile.pdf'/>
  <ctan file='true' path='/macros/latex/contrib/oberdiek/embedfile.dtx'/>
  <miktex location='oberdiek'/>
  <texlive location='oberdiek'/>
  <install path='/macros/latex/contrib/oberdiek/oberdiek.tds.zip'/>
</entry>
%</catalogue>
%    \end{macrocode}
%
% \begin{thebibliography}{9}
% \bibitem{attachfile}
%   Scott Pakin: \textit{The \xpackage{attachfile} package};
%   2005/02/20 v1.2;
%   \CTAN{macros/latex/contrib/attachfile/}.
%
% \bibitem{attachfile2}
%   Heiko Oberdiek: \textit{The \xpackage{attachfile2} package};
%   2006/08/16 v2.2;
%   \CTAN{macros/latex/contrib/oberdiek/attachfile2.pdf}.
%
% \bibitem{pdfspec}
%   Adobe Systems Incorporated:
%   \href{http://www.adobe.com/devnet/acrobat/pdfs/pdf_reference.pdf}%
%       {\textit{PDF Reference, Sixth Edition, Version 1.7}},%
%   Oktober 2006;
%   \url{http://www.adobe.com/devnet/pdf/pdf_reference.html}.
%
% \bibitem{mime}
%   Network Working Group: RFC 2046, \textit{Multipurpose Internet Mail Extensions (MIME) Part Two:
%     Media Types}, November 1996; \url{http://www.rfc-editor.org/}.
%
% \bibitem{types}
%   IANA (Internet Assigned Numbers Authority):
%   \textit{MIME Media Types}, May 2006;
%   \url{http://www.iana.org/assignments/media-types/}.
%
% \end{thebibliography}
%
% \begin{History}
%   \begin{Version}{2006/08/16 v1.0}
%   \item
%     First public version.
%   \end{Version}
%   \begin{Version}{2007/04/11 v1.1}
%   \item
%     Line ends sanitized.
%   \end{Version}
%   \begin{Version}{2007/09/09 v1.2}
%   \item
%     Fixes for plain-TeX, wrapper for package \xpackage{keyval} added.
%   \item
%     Catcode section rewritten.
%   \end{Version}
%   \begin{Version}{2007/10/28 v2.0}
%   \item
%     Collection support added (PDF 1.7).
%   \end{Version}
%   \begin{Version}{2007/10/29 v2.1}
%   \item
%     Export of object references by adding new option \xoption{id} and
%     new macros \cs{embedfileifobjectexists} and \cs{embedfilegetobject}.
%   \end{Version}
%   \begin{Version}{2007/11/11 v2.2}
%   \item
%     Use of package \xpackage{pdftexcmds} for \LuaTeX\ support.
%   \end{Version}
%   \begin{Version}{2007/11/25 v2.3}
%   \item
%     Fix in use of \cs{pdf@filesize}, bug introduced in previous version.
%   \end{Version}
%   \begin{Version}{2009/09/25 v2.4}
%   \item
%     Bug fix: If \xpackage{hyperref} is used with option \xoption{unicode},
%     the Unicode encoded file name causes trouble. Therefore \cs{pdfstringdef}
%     is now never used for option \xoption{filespec}, always method
%     \texttt{escape} is applied (Peter Cibulka).
%   \item
%     Bug fix for \xoption{initialfile}.
%   \item
%     Bug fix for file names in \texttt{/EmbeddedFiles}.
%   \item
%     New option \xoption{ucfilespec} for file name support in Unicode
%     (since PDF 1.7).
%   \end{Version}
%   \begin{Version}{2010/03/01 v2.5}
%   \item
%     Compatibility for \iniTeX.
%   \item
%     Package \xpackage{keyval} replaced by packages \xpackage{kvsetkeys}
%     and \xpackage{kvdefinekeys} because of compatibility for \iniTeX.
%   \item
%     TDS location moved from TDS:tex/latex/oberdiek/embedfile.sty to
%     TDS:tex/generic/oberdiek/embedfile.sty.
%   \end{Version}
%   \begin{Version}{2011/04/13 v2.6}
%   \item
%     Docu fixes (thanks Hans-Martin M\"unch).
%   \end{Version}
% \end{History}
%
% \PrintIndex
%
% \Finale
\endinput
|
% \end{quote}
% Do not forget to quote the argument according to the demands
% of your shell.
%
% \paragraph{Generating the documentation.}
% You can use both the \xfile{.dtx} or the \xfile{.drv} to generate
% the documentation. The process can be configured by the
% configuration file \xfile{ltxdoc.cfg}. For instance, put this
% line into this file, if you want to have A4 as paper format:
% \begin{quote}
%   \verb|\PassOptionsToClass{a4paper}{article}|
% \end{quote}
% An example follows how to generate the
% documentation with pdf\LaTeX:
% \begin{quote}
%\begin{verbatim}
%pdflatex embedfile.dtx
%makeindex -s gind.ist embedfile.idx
%pdflatex embedfile.dtx
%makeindex -s gind.ist embedfile.idx
%pdflatex embedfile.dtx
%\end{verbatim}
% \end{quote}
%
% \section{Catalogue}
%
% The following XML file can be used as source for the
% \href{http://mirror.ctan.org/help/Catalogue/catalogue.html}{\TeX\ Catalogue}.
% The elements \texttt{caption} and \texttt{description} are imported
% from the original XML file from the Catalogue.
% The name of the XML file in the Catalogue is \xfile{embedfile.xml}.
%    \begin{macrocode}
%<*catalogue>
<?xml version='1.0' encoding='us-ascii'?>
<!DOCTYPE entry SYSTEM 'catalogue.dtd'>
<entry datestamp='$Date$' modifier='$Author$' id='embedfile'>
  <name>embedfile</name>
  <caption>Embed files into PDF.</caption>
  <authorref id='auth:oberdiek'/>
  <copyright owner='Heiko Oberdiek' year='2006-2011'/>
  <license type='lppl1.3'/>
  <version number='2.6'/>
  <description>
    This package embeds files in a PDF document, using the PDF
    format's embedding operation (note the contrast with the attach
    operation used by the <xref refid='attachfile'>attachfile</xref>
    and <xref refid='attachfile2'>attachfile2</xref> packages).
    Currently only <xref refid='pdftex'>pdfTeX</xref> &gt;=1.30, in
    PDF mode, is supported.

    <p/>
    The package is part of the <xref refid='oberdiek'>oberdiek</xref>
    bundle.
  </description>
  <documentation details='Package documentation'
      href='ctan:/macros/latex/contrib/oberdiek/embedfile.pdf'/>
  <ctan file='true' path='/macros/latex/contrib/oberdiek/embedfile.dtx'/>
  <miktex location='oberdiek'/>
  <texlive location='oberdiek'/>
  <install path='/macros/latex/contrib/oberdiek/oberdiek.tds.zip'/>
</entry>
%</catalogue>
%    \end{macrocode}
%
% \begin{thebibliography}{9}
% \bibitem{attachfile}
%   Scott Pakin: \textit{The \xpackage{attachfile} package};
%   2005/02/20 v1.2;
%   \CTAN{macros/latex/contrib/attachfile/}.
%
% \bibitem{attachfile2}
%   Heiko Oberdiek: \textit{The \xpackage{attachfile2} package};
%   2006/08/16 v2.2;
%   \CTAN{macros/latex/contrib/oberdiek/attachfile2.pdf}.
%
% \bibitem{pdfspec}
%   Adobe Systems Incorporated:
%   \href{http://www.adobe.com/devnet/acrobat/pdfs/pdf_reference.pdf}%
%       {\textit{PDF Reference, Sixth Edition, Version 1.7}},%
%   Oktober 2006;
%   \url{http://www.adobe.com/devnet/pdf/pdf_reference.html}.
%
% \bibitem{mime}
%   Network Working Group: RFC 2046, \textit{Multipurpose Internet Mail Extensions (MIME) Part Two:
%     Media Types}, November 1996; \url{http://www.rfc-editor.org/}.
%
% \bibitem{types}
%   IANA (Internet Assigned Numbers Authority):
%   \textit{MIME Media Types}, May 2006;
%   \url{http://www.iana.org/assignments/media-types/}.
%
% \end{thebibliography}
%
% \begin{History}
%   \begin{Version}{2006/08/16 v1.0}
%   \item
%     First public version.
%   \end{Version}
%   \begin{Version}{2007/04/11 v1.1}
%   \item
%     Line ends sanitized.
%   \end{Version}
%   \begin{Version}{2007/09/09 v1.2}
%   \item
%     Fixes for plain-TeX, wrapper for package \xpackage{keyval} added.
%   \item
%     Catcode section rewritten.
%   \end{Version}
%   \begin{Version}{2007/10/28 v2.0}
%   \item
%     Collection support added (PDF 1.7).
%   \end{Version}
%   \begin{Version}{2007/10/29 v2.1}
%   \item
%     Export of object references by adding new option \xoption{id} and
%     new macros \cs{embedfileifobjectexists} and \cs{embedfilegetobject}.
%   \end{Version}
%   \begin{Version}{2007/11/11 v2.2}
%   \item
%     Use of package \xpackage{pdftexcmds} for \LuaTeX\ support.
%   \end{Version}
%   \begin{Version}{2007/11/25 v2.3}
%   \item
%     Fix in use of \cs{pdf@filesize}, bug introduced in previous version.
%   \end{Version}
%   \begin{Version}{2009/09/25 v2.4}
%   \item
%     Bug fix: If \xpackage{hyperref} is used with option \xoption{unicode},
%     the Unicode encoded file name causes trouble. Therefore \cs{pdfstringdef}
%     is now never used for option \xoption{filespec}, always method
%     \texttt{escape} is applied (Peter Cibulka).
%   \item
%     Bug fix for \xoption{initialfile}.
%   \item
%     Bug fix for file names in \texttt{/EmbeddedFiles}.
%   \item
%     New option \xoption{ucfilespec} for file name support in Unicode
%     (since PDF 1.7).
%   \end{Version}
%   \begin{Version}{2010/03/01 v2.5}
%   \item
%     Compatibility for \iniTeX.
%   \item
%     Package \xpackage{keyval} replaced by packages \xpackage{kvsetkeys}
%     and \xpackage{kvdefinekeys} because of compatibility for \iniTeX.
%   \item
%     TDS location moved from TDS:tex/latex/oberdiek/embedfile.sty to
%     TDS:tex/generic/oberdiek/embedfile.sty.
%   \end{Version}
%   \begin{Version}{2011/04/13 v2.6}
%   \item
%     Docu fixes (thanks Hans-Martin M\"unch).
%   \end{Version}
% \end{History}
%
% \PrintIndex
%
% \Finale
\endinput

%        (quote the arguments according to the demands of your shell)
%
% Documentation:
%    (a) If embedfile.drv is present:
%           latex embedfile.drv
%    (b) Without embedfile.drv:
%           latex embedfile.dtx; ...
%    The class ltxdoc loads the configuration file ltxdoc.cfg
%    if available. Here you can specify further options, e.g.
%    use A4 as paper format:
%       \PassOptionsToClass{a4paper}{article}
%
%    Programm calls to get the documentation (example):
%       pdflatex embedfile.dtx
%       makeindex -s gind.ist embedfile.idx
%       pdflatex embedfile.dtx
%       makeindex -s gind.ist embedfile.idx
%       pdflatex embedfile.dtx
%
% Installation:
%    TDS:tex/generic/oberdiek/embedfile.sty
%    TDS:tex/generic/oberdiek/dtx-attach.sty
%    TDS:doc/latex/oberdiek/embedfile.pdf
%    TDS:doc/latex/oberdiek/embedfile-example-plain.tex
%    TDS:doc/latex/oberdiek/embedfile-example-collection.tex
%    TDS:doc/latex/oberdiek/test/embedfile-test1.tex
%    TDS:doc/latex/oberdiek/test/embedfile-test2.tex
%    TDS:doc/latex/oberdiek/test/embedfile-test3.tex
%    TDS:doc/latex/oberdiek/test/embedfile-test4.tex
%    TDS:source/latex/oberdiek/embedfile.dtx
%
%<*ignore>
\begingroup
  \catcode123=1 %
  \catcode125=2 %
  \def\x{LaTeX2e}%
\expandafter\endgroup
\ifcase 0\ifx\install y1\fi\expandafter
         \ifx\csname processbatchFile\endcsname\relax\else1\fi
         \ifx\fmtname\x\else 1\fi\relax
\else\csname fi\endcsname
%</ignore>
%<*install>
\input docstrip.tex
\Msg{************************************************************************}
\Msg{* Installation}
\Msg{* Package: embedfile 2011/04/13 v2.6 Embed files into PDF (HO)}
\Msg{************************************************************************}

\keepsilent
\askforoverwritefalse

\let\MetaPrefix\relax
\preamble

This is a generated file.

Project: embedfile
Version: 2011/04/13 v2.6

Copyright (C) 2006-2011 by
   Heiko Oberdiek <heiko.oberdiek at googlemail.com>

This work may be distributed and/or modified under the
conditions of the LaTeX Project Public License, either
version 1.3c of this license or (at your option) any later
version. This version of this license is in
   http://www.latex-project.org/lppl/lppl-1-3c.txt
and the latest version of this license is in
   http://www.latex-project.org/lppl.txt
and version 1.3 or later is part of all distributions of
LaTeX version 2005/12/01 or later.

This work has the LPPL maintenance status "maintained".

This Current Maintainer of this work is Heiko Oberdiek.

The Base Interpreter refers to any `TeX-Format',
because some files are installed in TDS:tex/generic//.

This work consists of the main source file embedfile.dtx
and the derived files
   embedfile.sty, embedfile.pdf, embedfile.ins, embedfile.drv,
   dtx-attach.sty, embedfile-example-plain.tex,
   embedfile-example-collection.tex, embedfile-test1.tex,
   embedfile-test2.tex, embedfile-test3.tex,
   embedfile-test4.tex.

\endpreamble
\let\MetaPrefix\DoubleperCent

\generate{%
  \file{embedfile.ins}{\from{embedfile.dtx}{install}}%
  \file{embedfile.drv}{\from{embedfile.dtx}{driver}}%
  \usedir{tex/generic/oberdiek}%
  \file{embedfile.sty}{\from{embedfile.dtx}{package}}%
  \usedir{tex/latex/oberdiek}%
  \file{dtx-attach.sty}{\from{embedfile.dtx}{dtxattach}}%
  \usedir{doc/latex/oberdiek}%
  \file{embedfile-example-plain.tex}{\from{embedfile.dtx}{exampleplain}}%
  \file{embedfile-example-collection.tex}{\from{embedfile.dtx}{examplecollection}}%
  \usedir{doc/latex/oberdiek/test}%
  \file{embedfile-test1.tex}{\from{embedfile.dtx}{test1}}%
  \file{embedfile-test2.tex}{\from{embedfile.dtx}{test2}}%
  \file{embedfile-test3.tex}{\from{embedfile.dtx}{test3}}%
  \file{embedfile-test4.tex}{\from{embedfile.dtx}{test4}}%
  \nopreamble
  \nopostamble
  \usedir{source/latex/oberdiek/catalogue}%
  \file{embedfile.xml}{\from{embedfile.dtx}{catalogue}}%
}

\catcode32=13\relax% active space
\let =\space%
\Msg{************************************************************************}
\Msg{*}
\Msg{* To finish the installation you have to move the following}
\Msg{* files into a directory searched by TeX:}
\Msg{*}
\Msg{*     embedfile.sty, dtx-attach.sty}
\Msg{*}
\Msg{* To produce the documentation run the file `embedfile.drv'}
\Msg{* through LaTeX.}
\Msg{*}
\Msg{* Happy TeXing!}
\Msg{*}
\Msg{************************************************************************}

\endbatchfile
%</install>
%<*ignore>
\fi
%</ignore>
%<*driver>
\NeedsTeXFormat{LaTeX2e}
\ProvidesFile{embedfile.drv}%
  [2011/04/13 v2.6 Embed files into PDF (HO)]%
\documentclass{ltxdoc}
\usepackage{holtxdoc}[2011/11/22]
\begin{document}
  \DocInput{embedfile.dtx}%
\end{document}
%</driver>
% \fi
%
% \CheckSum{1363}
%
% \CharacterTable
%  {Upper-case    \A\B\C\D\E\F\G\H\I\J\K\L\M\N\O\P\Q\R\S\T\U\V\W\X\Y\Z
%   Lower-case    \a\b\c\d\e\f\g\h\i\j\k\l\m\n\o\p\q\r\s\t\u\v\w\x\y\z
%   Digits        \0\1\2\3\4\5\6\7\8\9
%   Exclamation   \!     Double quote  \"     Hash (number) \#
%   Dollar        \$     Percent       \%     Ampersand     \&
%   Acute accent  \'     Left paren    \(     Right paren   \)
%   Asterisk      \*     Plus          \+     Comma         \,
%   Minus         \-     Point         \.     Solidus       \/
%   Colon         \:     Semicolon     \;     Less than     \<
%   Equals        \=     Greater than  \>     Question mark \?
%   Commercial at \@     Left bracket  \[     Backslash     \\
%   Right bracket \]     Circumflex    \^     Underscore    \_
%   Grave accent  \`     Left brace    \{     Vertical bar  \|
%   Right brace   \}     Tilde         \~}
%
% \GetFileInfo{embedfile.drv}
%
% \title{The \xpackage{embedfile} package}
% \date{2011/04/13 v2.6}
% \author{Heiko Oberdiek\\\xemail{heiko.oberdiek at googlemail.com}}
%
% \maketitle
%
% \begin{abstract}
% This package embeds files to a PDF document.
% Currently the only supported driver is \pdfTeX\ $>=$ 1.30 in PDF mode.
% \end{abstract}
%
% \tableofcontents
%
% \section{Documentation}
%
% \subsection{Introduction}
%
%    The PDF format (\cite{pdfspec}) allows the inclusion of files
%    inside the PDF document. The included files can be bound to an
%    annotation on a page. Or they can be recorded in a sorted
%    list of embedded files. The packages \xpackage{attachfile}
%    or \xpackage{attachfile2} follow the first approach, this
%    package uses the latter method.
%
% \subsubsection{Future development}
%
%    My dream is a large package that merges the features of
%    all these packages meantioned before:
%    \begin{itemize}
%    \item Files can be attached to a page.
%    \item Files can be attached to the document.
%    \item An easy user interface for simple, common tasks and
%          beginners.
%    \item An interface for the advanced users that want to setup
%          every detail.
%    \item Support of many drivers (pdftex, dvips, dvipdfm, \dots).
%    \item \dots
%    \end{itemize}
%    However, I have not managed to take the time for this project.
%    Instead:
%    \begin{itemize}
%    \item First I experimented with package \xpackage{attachfile},
%          adding driver support, fixing bugs, \dots. The result is
%          currently named as \xpackage{attachfile2}. It uses an external
%          script to get file properties (size, date, checksum, \dots).
%    \item In order to avoid an external program for getting basic
%          file properties I provided a patch ``EscapeAndOther'' for
%          pdfTeX that was accepted for version 1.30.
%    \item This package closes a gap left by the packages for attaching
%          files and allows the embedding of files to the document.
%          Also it makes use of the new primitives of \pdfTeX.
%    \end{itemize}
%
% \subsection{User interface}
%
%    This package \xpackage{embedfile} can be used with both
%    \LaTeX\ and \plainTeX. See \autoref{sec:plain} that
%    explains the use with \plainTeX\ by an example.
%    In \LaTeX\ the package is loaded as usually. There are no options.
%    \begin{quote}
%      \verb|\usepackage{embedfile}|
%    \end{quote}
%
%    \begin{declcs}{embedfile}
%      |[|\meta{options}|]| \M{file}
%    \end{declcs}
%    The macro \cs{embedfile} includes file \meta{file} and
%    attaches it to the PDF document. At the end of the document
%    the sorted list of embedded files are written. Thus you can
%    safely use \cs{embedfile} before \verb|\end{document}|.
%    Embedding files using \cs{AtEndDocument} will only work,
%    if \cs{AtEndDocument} is called before loading the package
%    \xpackage{embedfile}.
%
%    The \meta{options} are give as key value pairs. The following
%    keys are supported:
%    \begin{description}
%    \item[\xoption{filespec}]
%      This allows to override the file name
%      that appears in the PDF file. If you are using other than
%      simple file names (8-bit, path separators, \dots), look into
%      the PDF specification (\cite{pdfspec}).
%      There are rules how these file names must be written/encoded.
%      Avoid 8-bit characters
%      and other special characters, the behaviour is currently undefined.
%      Use option \xoption{ucfilespec} for more funny file names.
%      The string method, see below, is |escape| since version 2.4.
%
%      This name is also used as entry in a name tree
%      (see PDF specification: |/EmbeddedFiles|). Therefore the
%      value for \xoption{filespec} must be unique among all
%      embedded files. Also key \xoption{initialfiles} refers
%      to this name, if the file name and the value of \xoption{filespec}
%      are different.
%    \item[\xoption{ucfilespec}]
%      Since PDF 1.7 the file name may be provided in Unicode.
%      The conversion of the option value into a PDF string
%      is controlled by option \xoption{stringmethod}.
%    \item[\xoption{filesystem}]
%      This sets the entry \verb|/FS| in
%      the file specification dictionary, see PDF specification
%      (\cite{pdfspec}).
%      Example: \verb|filesystem=URL|.
%    \item[\xoption{mimetype}]
%      This sets the mime type (\cite{mime}) of the file,
%      see \autoref{sec:plain} for examples and \cite{types}
%      for a list of officially registered types.
%    \item[\xoption{desc}]
%      The description for the file.
%    \item[\xoption{stringmethod}]
%      The package must convert the values of the keys \xoption{ucfilespec}
%      and \xoption{desc} into a PDF string (before version 2.4: \xoption{filespec}
%      and \xoption{desc}). If \xpackage{hyperref}
%      is found, then its \cs{pdfstringdef} will be used, otherwise
%      \pdfTeX's \cs{pdfescapestring} is used. Value \verb|psd|
%      forces the use of \cs{pdfstringdef}, value \verb|escape|
%      the use of \cs{pdfescapestring}.
%    \item[\xoption{\meta{key}.value}] Sets the value of a collection
%      item property, see section \ref{sec:collection}.
%    \item[\xoption{\meta{key}.prefix}]
%      Sets the prefix of a collection item property,
%      see section \ref{sec:collection}.
%    \item[\xoption{id}]
%      The value must be an unique name. Macros
%      \cs{embedfileifobjectexists} and \cs{embedfilegetobject}
%      are using this name later.
%    \end{description}
%
%    \begin{declcs}{embedfilefinish}
%    \end{declcs}
%    The list of all embedded files must be added as data structure
%    in the PDF file. In case of \LaTeX\ this is automatically done.
%    The package uses \cs{AtEndDocument}. Then the list of all
%    files should be known. However, \plainTeX\ does not know
%    about \cs{AtEndDocument}. Thus the user must call
%    \cs{embedfilefinish} at the end of the document
%    after the last file is embedded.
%
%    \begin{declcs}{embedfilesetup} \M{options}
%    \end{declcs}
%    Options for \cs{embedfile} and collection support
%    can be set in \cs{embedfilesetup}.
%
% \subsection{Collection support (PDF 1.7)}
% \label{sec:collection}
%
%    Since PDF 1.7 the embedded files can form a \emph{collection}
%    (sometimes referred as \emph{package}), the main document
%    is called \emph{cover sheet}. See PDF specification
%    8.2.4 ``Collections'' and 3.10.5 ``Collection items'' \cite{pdfspec}.
%
%    Usually Acrobat Reader 7 or 8 shows the
%    embedded files in a table at the bottom with the following columns:
%    \begin{quote}
%    \begin{tabular}{@{}l|l|l|l@{}}
%       Name & Description & Modified & Size\\
%       \hline
%       \dots & \dots & \dots & \dots\\
%    \end{tabular}
%    \end{quote}
%    Acrobat Reader 10 shows the embeded files in the left panel
%    and adds a new column for the compressed size.
%
%    If the files form a collection, then they are displayed
%    in a table left or top (depending on option \xoption{view},
%    see \cs{embedfilesetup}).
%
%
%    Collection support is enabled automatically, if
%    it is used.
%
%    \begin{declcs}{embedfilesetup} \M{options}
%    \end{declcs}
%    The following options are supported in addition
%    to options for \cs{embedfile}:
%    \begin{description}
%    \item[\xoption{view}]
%      If the PDF file contains a collection, then Acrobat Reader 8
%      shows a line at the top below the menu bar and the toolbar.
%      It shows the current selected file, icons for changing the
%      view mode, an options menu.
%      The initial mode how the collection is presented is
%      set by this option \xoption{view}.
%      The following modes/values are supported, the default is |details|:
%        \begin{description}
%        \item[\texttt{details}] The full collection table is displayed
%          at the top below the collection bar.
%        \item[\texttt{tile}] The files of the collection are shown
%          in tile mode on the left.
%        \item[\texttt{hidden}] The collection table is not shown.
%        \end{description}
%    \item[\xoption{initialfile}]
%      Selects the file that is initially presented. Especially
%      useful for an embedded PDF file that is then shown instead
%      of the cover document. There must be an \cs{embedfile} command
%      somewhere whose value for key \xoption{filespec} is used here.
%      The \cs{embedfile} command can drop option \xoption{filespec}
%      if the file name is not different.
%    \end{description}
%
%    \begin{declcs}{embedfilefield} \M{key} \M{options}
%    \end{declcs}
%    Macro \cs{embedfilefield} defines a column/field in the collection table.
%    The name of the field is \meta{key}.
%    \begin{description}
%    \item[\xoption{type}]
%      sets the type of the field. The supported values are:
%      \begin{description}
%      \item[\texttt{text}] A text field. Its value is set in
%        \cs{embedfile} by option \xoption{\meta{key}.value}.
%      \item[\texttt{date}] A date field. Its value is set in
%        \cs{embedfile} by option \xoption{\meta{key}.value}.
%        A special format is required, see ``3.8.3 Dates'' \cite{pdfspec}.
%      \item[\texttt{number}] A field with an integer or float number.
%        Its value is set in \cs{embedfile} by option
%        \xoption{\meta{key}.value}.
%      \item[\texttt{file}] The file name of the embedded file.
%      \item[\texttt{desc}] The description text of the embedded file. It is
%       set in \cs{embedfile} by option \xoption{desc}.
%      \item[\texttt{moddate}] The modification date of the embedded file.
%      \item[\texttt{size}] The size of the embedded file.
%      \end{description}
%      All types allow the use of a prefix that is disregarded by
%      sorting. The prefix for this field is set in \cs{embedfile} by
%      option \xoption{\meta{key}.prefix}.
%    \item[\xoption{title}] sets the column title.
%    \item[\xoption{visible}] controls whether the column is presented:
%      \begin{description}
%      \item[\texttt{true}] shows the column.
%      \item[\texttt{false}] hides the column.
%      \end{description}
%      Default: \texttt{true}
%    \item[\xoption{edit}]
%      Allows the editing of field values. Does not seem to have an
%      effect for Acrobat Reader.
%      \begin{description}
%      \item[\texttt{true}] enables the feature, if available (depends
%        on the PDF viewer).
%      \item[\texttt{false}] disables the feature.
%      \end{description}
%      Default: \texttt{false}
%    \end{description}
%    The order of \cs{embedfilefield} statements defines the order
%    of the columns.
%
%    \begin{declcs}{embedfilesort} \M{key-sort-list}
%    \end{declcs}
%    The sort order of the embedded files are controlled by
%    macro \cs{embedfilesort}. \meta{key-sort-list} defines
%    the sort order. The key is a field name defined by \cs{embedfilefield}.
%    Its value is either \texttt{ascending} or \texttt{descending}.
%    The default is \texttt{ascending}.
%
% \subsection{Export of object references}
%
%    Caution: This feature is still experimental. It may be even removed
%    in future versions. Therefore feedback would be nice, if someone
%    has a useful application for this feature.
%
%    Object numbers are saved, if \xoption{id} is given in \cs{embedfile}.
%    The following objects are supported:
%    \begin{itemize}
%    \item \texttt{EmbeddedFile}
%    \item \texttt{Filespec}
%    \end{itemize}
%
%    \begin{declcs}{embedfileifobjectexists} \M{id} \M{type}
%      \M{then} \M{else}
%    \end{declcs}
%    Macro \cs{embedfileifobjectexists} tests whether object of \meta{type}
%    is available for the embedded file identified by \meta{id}.
%
%    \begin{declcs}{embedfilegetobject} \M{id} \M{type}
%    \end{declcs}
%    Macro \cs{embedfilegetobject} expands to the full object reference
%    object of \meta{type} for the embedded file identified by \meta{id}.
%
% \subsubsection{Example}
%
%\begin{quote}
%\begin{verbatim}
%\embedfile[id={foo}]{foo.pdf}
%\embedfileifobjectexists{foo}{Filespec}{%
%  \typeout{%
%    FileSpec object for `foo': %
%    \embedfilegetobject{foo}{Filespec}%
%  }%
%}{%
%  \typeout{No Filespec object for `foo'}%
%}
%\end{verbatim}
%\end{quote}
%
% \subsection{Examples}
% \label{sec:examples}
%
% \subsubsection{\plainTeX}
% \label{sec:plain}
%
%    The package can be used with \plainTeX. It can be used
%    with or without help from \xfile{miniltx.tex}.
%
%    If additionally package \xpackage{keyval} (\xpackage{graphicx})
%    is needed, load it first. Then package \xpackage{embedfile} avoids
%    a duplicate loading of package \xpackage{keyval}.
%
%    Because \plainTeX\ does not provide a hook at end of the document,
%    you have to call \cs{embedfilefinish} manually at the end
%    after the last embedded file.
%    \begin{macrocode}
%<*exampleplain>
%<<END
% Load packages
\input miniltx
% \def\Gin@driver{pdftex.def}
% \input graphicx.sty
\input embedfile.sty
\resetatcatcode

% default setting
\embedfilesetup{
  mimetype=text/plain
}

% Embed files
\embedfile[
  filespec=example.tex,
  desc={Source code (plain-TeX) of this example}
]{embedfile-example-plain.tex}

\embedfile[
  desc={Source of package `embedfile'}
]{embedfile.dtx}

\embedfile[
  mimetype=application/pdf,
  desc={Documentation of package `embedfile'}
]{embedfile.pdf}

% Some text
This example document contains three embedded files.

% End of document
\embedfilefinish % don't forget
\bye
%END
%</exampleplain>
%    \end{macrocode}
%
% \subsubsection{Collection example}
%
%    \begin{macrocode}
%<*examplecollection>
%<<END
\NeedsTeXFormat{LaTeX2e}
\documentclass{article}
\usepackage[bookmarks=false]{hyperref}
% provides \pdfstringdef that is then used by `title' and
% other keys.
\usepackage{embedfile}[2011/04/13]
\embedfilesetup{
  view=details,
  initialfile=embedfile.pdf
}
\embedfilefield{file}{
  type=file,
  title={File name}
}
\embedfilefield{description}{
  type=desc,
  title={Description}
}
\embedfilefield{date}{
  type=moddate,
  title={Date}
}
\embedfilefield{size}{
  type=size,
  title={Size}
}
\embedfilefield{type}{
  type=text,
  title={Type},
  visible=false
}
\embedfilesort{
  type,
  date=descending
}
\begin{document}
An example for embedded files as collection.
You need Acrobat Reader 8 or higher.

\embedfile[
  desc={Source file of package `embedfile'},
  description.prefix={Package: },
  type.value={DTX}
]{embedfile.dtx}

\embedfile[
  desc={Documentation of package `embedfile'},
  description.prefix={Package: },
  type.value={PDF}
]{embedfile.pdf}

\embedfile[
  desc={The source for this example},
  description.prefix={Example: },
  type.value={TEX}
]{\jobname.tex}

\end{document}
%END
%</examplecollection>
%    \end{macrocode}
%
% \subsection{Package \xpackage{dtx-attach}}
%
% Package \xpackage{dtx-attach} is just a small application of
% package \xpackage{embedfile}. I am using it for the CTAN
% documentation of my packages in \CTAN{macros/latex/contrib/oberdiek/}.
% It also serves as small example for the use of the package with \LaTeX.
%
%    \begin{macrocode}
%<*dtxattach>
\NeedsTeXFormat{LaTeX2e}
\ProvidesPackage{dtx-attach}
  [2011/04/13 v2.6 Embed \string\jobname.dtx (HO)]%
\RequirePackage{embedfile}[2011/04/13]
\embedfile[%
  stringmethod=escape,%
  mimetype=plain/text,%
  desc={LaTeX docstrip source archive for package `\jobname'}%
]{\jobname.dtx}
%</dtxattach>
%    \end{macrocode}
%
% \StopEventually{
% }
%
% \section{Implementation}
%
%    \begin{macrocode}
%<*package>
%    \end{macrocode}
%
% \subsection{Reload check and package identification}
%    Reload check, especially if the package is not used with \LaTeX.
%    \begin{macrocode}
\begingroup\catcode61\catcode48\catcode32=10\relax%
  \catcode13=5 % ^^M
  \endlinechar=13 %
  \catcode35=6 % #
  \catcode39=12 % '
  \catcode44=12 % ,
  \catcode45=12 % -
  \catcode46=12 % .
  \catcode58=12 % :
  \catcode64=11 % @
  \catcode123=1 % {
  \catcode125=2 % }
  \expandafter\let\expandafter\x\csname ver@embedfile.sty\endcsname
  \ifx\x\relax % plain-TeX, first loading
  \else
    \def\empty{}%
    \ifx\x\empty % LaTeX, first loading,
      % variable is initialized, but \ProvidesPackage not yet seen
    \else
      \expandafter\ifx\csname PackageInfo\endcsname\relax
        \def\x#1#2{%
          \immediate\write-1{Package #1 Info: #2.}%
        }%
      \else
        \def\x#1#2{\PackageInfo{#1}{#2, stopped}}%
      \fi
      \x{embedfile}{The package is already loaded}%
      \aftergroup\endinput
    \fi
  \fi
\endgroup%
%    \end{macrocode}
%    Package identification:
%    \begin{macrocode}
\begingroup\catcode61\catcode48\catcode32=10\relax%
  \catcode13=5 % ^^M
  \endlinechar=13 %
  \catcode35=6 % #
  \catcode39=12 % '
  \catcode40=12 % (
  \catcode41=12 % )
  \catcode44=12 % ,
  \catcode45=12 % -
  \catcode46=12 % .
  \catcode47=12 % /
  \catcode58=12 % :
  \catcode64=11 % @
  \catcode91=12 % [
  \catcode93=12 % ]
  \catcode123=1 % {
  \catcode125=2 % }
  \expandafter\ifx\csname ProvidesPackage\endcsname\relax
    \def\x#1#2#3[#4]{\endgroup
      \immediate\write-1{Package: #3 #4}%
      \xdef#1{#4}%
    }%
  \else
    \def\x#1#2[#3]{\endgroup
      #2[{#3}]%
      \ifx#1\@undefined
        \xdef#1{#3}%
      \fi
      \ifx#1\relax
        \xdef#1{#3}%
      \fi
    }%
  \fi
\expandafter\x\csname ver@embedfile.sty\endcsname
\ProvidesPackage{embedfile}%
  [2011/04/13 v2.6 Embed files into PDF (HO)]%
%    \end{macrocode}
%
% \subsection{Catcodes}
%
%    \begin{macrocode}
\begingroup\catcode61\catcode48\catcode32=10\relax%
  \catcode13=5 % ^^M
  \endlinechar=13 %
  \catcode123=1 % {
  \catcode125=2 % }
  \catcode64=11 % @
  \def\x{\endgroup
    \expandafter\edef\csname EmFi@AtEnd\endcsname{%
      \endlinechar=\the\endlinechar\relax
      \catcode13=\the\catcode13\relax
      \catcode32=\the\catcode32\relax
      \catcode35=\the\catcode35\relax
      \catcode61=\the\catcode61\relax
      \catcode64=\the\catcode64\relax
      \catcode123=\the\catcode123\relax
      \catcode125=\the\catcode125\relax
    }%
  }%
\x\catcode61\catcode48\catcode32=10\relax%
\catcode13=5 % ^^M
\endlinechar=13 %
\catcode35=6 % #
\catcode64=11 % @
\catcode123=1 % {
\catcode125=2 % }
\def\TMP@EnsureCode#1#2{%
  \edef\EmFi@AtEnd{%
    \EmFi@AtEnd
    \catcode#1=\the\catcode#1\relax
  }%
  \catcode#1=#2\relax
}
\TMP@EnsureCode{39}{12}% '
\TMP@EnsureCode{40}{12}% (
\TMP@EnsureCode{41}{12}% )
\TMP@EnsureCode{44}{12}% ,
\TMP@EnsureCode{46}{12}% .
\TMP@EnsureCode{47}{12}% /
\TMP@EnsureCode{58}{12}% :
\TMP@EnsureCode{60}{12}% <
\TMP@EnsureCode{62}{12}% >
\TMP@EnsureCode{91}{12}% [
\TMP@EnsureCode{93}{12}% ]
\TMP@EnsureCode{96}{12}% `
\edef\EmFi@AtEnd{\EmFi@AtEnd\noexpand\endinput}
%    \end{macrocode}
%
% \subsection{Tools}
%
%    \begin{macro}{\EmFi@RequirePackage}
%    \begin{macrocode}
\begingroup\expandafter\expandafter\expandafter\endgroup
\expandafter\ifx\csname RequirePackage\endcsname\relax
  \def\EmFi@RequirePackage#1[#2]{%
    \input #1.sty\relax
  }%
\else
  \let\EmFi@RequirePackage\RequirePackage
\fi
%    \end{macrocode}
%    \end{macro}
%
%    \begin{macro}{\EmFi@Error}
%    \begin{macrocode}
\EmFi@RequirePackage{infwarerr}[2007/09/09]%
\def\EmFi@Error{%
  \@PackageError{embedfile}%
}
%    \end{macrocode}
%    \end{macro}
%
% \subsection{Check for recent \pdfTeX\ in PDF mode}
%
%    Load package \xpackage{ifpdf} and check mode.
%    \begin{macrocode}
\EmFi@RequirePackage{ifpdf}[2007/09/09]
\ifpdf
\else
  \EmFi@Error{%
    Missing pdfTeX in PDF mode%
  }{%
    Currently other drivers are not supported. %
    Package loading is aborted.%
  }%
  \expandafter\EmFi@AtEnd
\fi%
%    \end{macrocode}
%    \begin{macrocode}
\EmFi@RequirePackage{pdftexcmds}[2007/11/11]
\EmFi@RequirePackage{ltxcmds}[2010/03/01]
\EmFi@RequirePackage{kvsetkeys}[2010/03/01]
\EmFi@RequirePackage{kvdefinekeys}[2010/03/01]
%    \end{macrocode}
%    Check version.
%    \begin{macrocode}
\begingroup\expandafter\expandafter\expandafter\endgroup
\expandafter\ifx\csname pdf@filesize\endcsname\relax
  \EmFi@Error{%
    Unsupported pdfTeX version%
  }{%
    At least version 1.30 is necessary. Package loading is aborted.%
  }%
  \expandafter\EmFi@AtEnd
\fi%
%    \end{macrocode}
%
% \subsection{Strings}
%
%    Minimal version of package \xpackage{pdfescape} is 2007/08/27 v1.5
%    because of \cs{EdefSanitize}.
%    \begin{macrocode}
\EmFi@RequirePackage{pdfescape}[2007/11/11]
%    \end{macrocode}
%
%    \begin{macrocode}
\def\EmFi@temp#1{%
  \expandafter\EdefSanitize\csname EmFi@S@#1\endcsname{#1}%
}
%    \end{macrocode}
%    \begin{macro}{\EmFi@details}
%    \begin{macrocode}
\EmFi@temp{details}%
%    \end{macrocode}
%    \end{macro}
%    \begin{macro}{\EmFi@tile}
%    \begin{macrocode}
\EmFi@temp{tile}%
%    \end{macrocode}
%    \end{macro}
%    \begin{macro}{\EmFi@hidden}
%    \begin{macrocode}
\EmFi@temp{hidden}%
%    \end{macrocode}
%    \end{macro}
%    \begin{macro}{\EmFi@S@text}
%    \begin{macrocode}
\EmFi@temp{text}
%    \end{macrocode}
%    \end{macro}
%    \begin{macro}{\EmFi@S@date}
%    \begin{macrocode}
\EmFi@temp{date}
%    \end{macrocode}
%    \end{macro}
%    \begin{macro}{\EmFi@S@number}
%    \begin{macrocode}
\EmFi@temp{number}
%    \end{macrocode}
%    \end{macro}
%    \begin{macro}{\EmFi@S@file}
%    \begin{macrocode}
\EmFi@temp{file}
%    \end{macrocode}
%    \end{macro}
%    \begin{macro}{\EmFi@S@desc}
%    \begin{macrocode}
\EmFi@temp{desc}
%    \end{macrocode}
%    \end{macro}
%    \begin{macro}{\EmFi@S@moddate}
%    \begin{macrocode}
\EmFi@temp{moddate}
%    \end{macrocode}
%    \end{macro}
%    \begin{macro}{\EmFi@S@creationdate}
%    \begin{macrocode}
\EmFi@temp{creationdate}
%    \end{macrocode}
%    \end{macro}
%    \begin{macro}{\EmFi@S@size}
%    \begin{macrocode}
\EmFi@temp{size}
%    \end{macrocode}
%    \end{macro}
%    \begin{macro}{\EmFi@S@ascending}
%    \begin{macrocode}
\EmFi@temp{ascending}
%    \end{macrocode}
%    \end{macro}
%    \begin{macro}{\EmFi@S@descending}
%    \begin{macrocode}
\EmFi@temp{descending}
%    \end{macrocode}
%    \end{macro}
%    \begin{macro}{\EmFi@S@true}
%    \begin{macrocode}
\EmFi@temp{true}
%    \end{macrocode}
%    \end{macro}
%    \begin{macro}{\EmFi@S@false}
%    \begin{macrocode}
\EmFi@temp{false}
%    \end{macrocode}
%    \end{macro}
%
% \subsection{Switches}
%
%    \begin{macro}{\ifEmFi@collection}
%    \begin{macrocode}
\ltx@newif\ifEmFi@collection
%    \end{macrocode}
%    \end{macro}
%    \begin{macro}{\ifEmFi@sort}
%    \begin{macrocode}
\ltx@newif\ifEmFi@sort
%    \end{macrocode}
%    \end{macro}
%    \begin{macro}{\ifEmFi@visible}
%    \begin{macrocode}
\ltx@newif\ifEmFi@visible
%    \end{macrocode}
%    \end{macro}
%    \begin{macro}{\ifEmFi@edit}
%    \begin{macrocode}
\ltx@newif\ifEmFi@edit
%    \end{macrocode}
%    \end{macro}
%    \begin{macro}{\ifEmFi@item}
%    \begin{macrocode}
\ltx@newif\ifEmFi@item
%    \end{macrocode}
%    \end{macro}
%    \begin{macro}{\ifEmFi@finished}
%    \begin{macrocode}
\ltx@newif\ifEmFi@finished
%    \end{macrocode}
%    \end{macro}
%    \begin{macro}{\ifEmFi@id}
%    \begin{macrocode}
\ltx@newif\ifEmFi@id
%    \end{macrocode}
%    \end{macro}
%
% \subsection{Key value definitions}
%
%    \begin{macro}{\EmFi@GlobalKey}
%    \begin{macrocode}
\def\EmFi@GlobalKey#1#2{%
  \global\expandafter\let\csname KV@#1@#2\expandafter\endcsname
                         \csname KV@#1@#2\endcsname
}
%    \end{macrocode}
%    \end{macro}
%    \begin{macro}{\EmFi@GlobalDefaultKey}
%    \begin{macrocode}
\def\EmFi@GlobalDefaultKey#1#2{%
  \EmFi@GlobalKey{#1}{#2}%
  \global\expandafter\let
      \csname KV@#1@#2@default\expandafter\endcsname
      \csname KV@#1@#2@default\endcsname
}
%    \end{macrocode}
%    \end{macro}
%
%    \begin{macro}{\EmFi@DefineKey}
%    \begin{macrocode}
\def\EmFi@DefineKey#1#2{%
  \kv@define@key{EmFi}{#1}{%
    \expandafter\def\csname EmFi@#1\endcsname{##1}%
  }%
  \expandafter\def\csname EmFi@#1\endcsname{#2}%
}
%    \end{macrocode}
%    \end{macro}
%    Subtype of the embedded file (optional).
%    \begin{macrocode}
\EmFi@DefineKey{mimetype}{}
%    \end{macrocode}
%    File specification string.
%    \begin{macrocode}
\EmFi@DefineKey{filespec}{\EmFi@file}
%    \end{macrocode}
%    File specification string in Unicode.
%    \begin{macrocode}
\EmFi@DefineKey{ucfilespec}{}
%    \end{macrocode}
%    File system (optional).
%    \begin{macrocode}
\EmFi@DefineKey{filesystem}{}
%    \end{macrocode}
%    Description (optional).
%    \begin{macrocode}
\EmFi@DefineKey{desc}{}
%    \end{macrocode}
%    Method for converting text to PDF strings.
%    \begin{macrocode}
\EmFi@DefineKey{stringmethod}{%
  \ifx\pdfstringdef\@undefined
    escape%
  \else
    \ifx\pdfstringdef\relax
      escape%
    \else
      psd%
    \fi
  \fi
}
%    \end{macrocode}
%
%    Option \xoption{id} as key for object numbers.
%    \begin{macrocode}
\kv@define@key{EmFi}{id}{%
  \def\EmFi@id{#1}%
  \EmFi@idtrue
}
%    \end{macrocode}
%    \begin{macro}{\EmFi@defobj}
%    \begin{macrocode}
\def\EmFi@defobj#1{%
  \ifEmFi@id
    \expandafter\xdef\csname EmFi@#1@\EmFi@id\endcsname{%
      \the\pdflastobj\ltx@space 0 R%
    }%
  \fi
}
%    \end{macrocode}
%    \end{macro}
%    \begin{macro}{\embedfileifobjectexists}
%    \begin{macrocode}
\def\embedfileifobjectexists#1#2{%
  \expandafter\ifx\csname EmFi@#2@#1\endcsname\relax
    \expandafter\ltx@secondoftwo
  \else
    \expandafter\ltx@firstoftwo
  \fi
}
%    \end{macrocode}
%    \end{macro}
%    \begin{macro}{\embedfilegetobject}
%    \begin{macrocode}
\def\embedfilegetobject#1#2{%
  \embedfileifobjectexists{#1}{#2}{%
    \csname EmFi@#2@#1\endcsname
  }{%
    0 0 R%
  }%
}
%    \end{macrocode}
%    \end{macro}
%
%    Initial view of the collection.
%    \begin{macrocode}
\kv@define@key{EmFi}{view}[]{%
  \EdefSanitize\EmFi@temp{#1}%
  \def\EmFi@next{%
    \global\EmFi@collectiontrue
  }%
  \ifx\EmFi@temp\ltx@empty
    \let\EmFi@view\EmFi@S@details
  \else\ifx\EmFi@temp\EmFi@S@details
    \let\EmFi@view\EmFi@S@details
  \else\ifx\EmFi@temp\EmFi@S@tile
    \let\EmFi@view\EmFi@S@tile
  \else\ifx\EmFi@temp\EmFi@S@hidden
    \let\EmFi@view\EmFi@S@hidden
  \else
    \let\EmFi@next\relax
    \EmFi@Error{%
      Unknown value `\EmFi@temp' for key `view'.\MessageBreak
      Supported values: `details', `tile', `hidden'.%
    }\@ehc
  \fi\fi\fi\fi
  \EmFi@next
}
%    \end{macrocode}
%    \begin{macrocode}
\EmFi@DefineKey{initialfile}{}
%    \end{macrocode}
%
%    \begin{macro}{\embedfilesetup}
%    \begin{macrocode}
\def\embedfilesetup{%
  \ifEmFi@finished
    \def\EmFi@next##1{}%
    \EmFi@Error{%
      \string\embedfilefield\ltx@space after \string\embedfilefinish
    }{%
      The list of embedded files is already written.%
    }%
  \else
    \def\EmFi@next{%
      \kvsetkeys{EmFi}%
    }%
  \fi
  \EmFi@next
}
%    \end{macrocode}
%    \end{macro}
%
%    \begin{macro}{\EmFi@schema}
%    \begin{macrocode}
\def\EmFi@schema{}
%    \end{macrocode}
%    \end{macro}
%    \begin{macro}{\EmFi@order}
%    \begin{macrocode}
\gdef\EmFi@order{0}
%    \end{macrocode}
%    \end{macro}
%    \begin{macro}{\EmFi@@order}
%    \begin{macrocode}
\let\EmFi@@order\relax
%    \end{macrocode}
%    \end{macro}
%    \begin{macro}{\EmFi@fieldlist}
%    \begin{macrocode}
\def\EmFi@fieldlist{}
%    \end{macrocode}
%    \end{macro}
%    \begin{macro}{\EmFi@sortcase}
%    \begin{macrocode}
\def\EmFi@sortcase{0}%
%    \end{macrocode}
%    \end{macro}
%    \begin{macro}{\embedfilefield}
%    \begin{macrocode}
\def\embedfilefield#1#2{%
  \ifEmFi@finished
    \EmFi@Error{%
      \string\embedfilefield\ltx@space after \string\embedfilefinish
    }{%
      The list of embedded files is already written.%
    }%
  \else
    \global\EmFi@collectiontrue
    \EdefSanitize\EmFi@key{#1}%
    \expandafter\ifx\csname KV@EmFi@\EmFi@key.prefix\endcsname\relax
      \begingroup
        \count@=\EmFi@order
        \advance\count@ 1 %
        \xdef\EmFi@order{\the\count@}%
        \let\EmFi@title\EmFi@key
        \let\EmFi@type\EmFi@S@text
        \EmFi@visibletrue
        \EmFi@editfalse
        \kvsetkeys{EmFiFi}{#2}%
        \EmFi@convert\EmFi@title\EmFi@title
        \xdef\EmFi@schema{%
          \EmFi@schema
          /\pdf@escapename{\EmFi@key}<<%
            /Subtype/%
            \ifx\EmFi@type\EmFi@S@date D%
            \else\ifx\EmFi@type\EmFi@S@number N%
            \else\ifx\EmFi@type\EmFi@S@file F%
            \else\ifx\EmFi@type\EmFi@S@desc Desc%
            \else\ifx\EmFi@type\EmFi@S@moddate ModDate%
            \else\ifx\EmFi@type\EmFi@S@creationdate CreationDate%
            \else\ifx\EmFi@type\EmFi@S@size Size%
            \else S%
            \fi\fi\fi\fi\fi\fi\fi
            /N(\EmFi@title)%
            \EmFi@@order{\EmFi@order}%
            \ifEmFi@visible
            \else
              /V false%
            \fi
            \ifEmFi@edit
              /E true%
            \fi
          >>%
        }%
        \let\do\relax
        \xdef\EmFi@fieldlist{%
          \EmFi@fieldlist
          \do{\EmFi@key}%
        }%
        \ifx\EmFi@type\EmFi@S@text
          \kv@define@key{EmFi}{\EmFi@key.value}{%
            \EmFi@itemtrue
            \def\EmFi@temp{##1}%
            \EmFi@convert\EmFi@temp\EmFi@temp
            \expandafter\def\csname EmFi@V@#1%
            \expandafter\endcsname\expandafter{%
              \expandafter(\EmFi@temp)%
            }%
          }%
          \EmFi@GlobalKey{EmFi}{\EmFi@key.value}%
        \else\ifx\EmFi@type\EmFi@S@date
          \kv@define@key{EmFi}{\EmFi@key.value}{%
            \EmFi@itemtrue
            \def\EmFi@temp{##1}%
            \EmFi@convert\EmFi@temp\EmFi@temp
            \expandafter\def\csname EmFi@V@#1%
            \expandafter\endcsname\expandafter{%
              \expandafter(\EmFi@temp)%
            }%
          }%
          \EmFi@GlobalKey{EmFi}{\EmFi@key.value}%
        \else\ifx\EmFi@type\EmFi@S@number
          \kv@define@key{EmFi}{\EmFi@key.value}{%
            \EmFi@itemtrue
            \expandafter\EdefSanitize\csname EmFi@V@#1\endcsname{ ##1}%
          }%
          \EmFi@GlobalKey{EmFi}{\EmFi@key.value}%
        \fi\fi\fi
        \kv@define@key{EmFi}{\EmFi@key.prefix}{%
          \EmFi@itemtrue
          \expandafter\def\csname EmFi@P@#1\endcsname{##1}%
        }%
        \EmFi@GlobalKey{EmFi}{\EmFi@key.prefix}%
        \kv@define@key{EmFiSo}{\EmFi@key}[ascending]{%
          \EdefSanitize\EmFi@temp{##1}%
          \ifx\EmFi@temp\EmFi@S@ascending
            \def\EmFi@temp{true}%
          \else\ifx\EmFi@temp\EmFi@S@descending
            \def\EmFi@temp{false}%
          \else
            \def\EmFi@temp{}%
            \EmFi@Error{%
              Unknown sort order `\EmFi@temp'.\MessageBreak
              Supported values: `\EmFi@S@ascending', %
              `\EmFi@S@descending
            }\@ehc
          \fi\fi
          \ifx\EmFi@temp\ltx@empty
          \else
            \xdef\EmFi@sortkeys{%
              \EmFi@sortkeys
              /\pdf@escapename{#1}%
            }%
            \ifx\EmFi@sortorders\ltx@empty
              \global\let\EmFi@sortorders\EmFi@temp
              \gdef\EmFi@sortcase{1}%
            \else
              \xdef\EmFi@sortorders{%
                \EmFi@sortorders
                \ltx@space
                \EmFi@temp
              }%
              \xdef\EmFi@sortcase{2}%
            \fi
          \fi
        }%
        \EmFi@GlobalDefaultKey{EmFiSo}\EmFi@key
      \endgroup
    \else
      \EmFi@Error{%
        Field `\EmFi@key' is already defined%
      }\@ehc
    \fi
  \fi
}
%    \end{macrocode}
%    \end{macro}
%    \begin{macrocode}
\kv@define@key{EmFiFi}{type}{%
  \EdefSanitize\EmFi@temp{#1}%
  \ifx\EmFi@temp\EmFi@S@text
    \let\EmFi@type\EmFi@temp
  \else\ifx\EmFi@temp\EmFi@S@date
    \let\EmFi@type\EmFi@temp
  \else\ifx\EmFi@temp\EmFi@S@number
    \let\EmFi@type\EmFi@temp
  \else\ifx\EmFi@temp\EmFi@S@file
    \let\EmFi@type\EmFi@temp
  \else\ifx\EmFi@temp\EmFi@S@desc
    \let\EmFi@type\EmFi@temp
  \else\ifx\EmFi@temp\EmFi@S@moddate
    \let\EmFi@type\EmFi@temp
  \else\ifx\EmFi@temp\EmFi@S@creationdate
    \let\EmFi@type\EmFi@temp
  \else\ifx\EmFi@temp\EmFi@S@size
    \let\EmFi@type\EmFi@temp
  \else
    \EmFi@Error{%
      Unknown type `\EmFi@temp'.\MessageBreak
      Supported types: `text', `date', `number', `file',\MessageBreak
      `desc', `moddate', `creationdate', `size'%
    }%
  \fi\fi\fi\fi\fi\fi\fi\fi
}
%    \end{macrocode}
%    \begin{macrocode}
\kv@define@key{EmFiFi}{title}{%
  \def\EmFi@title{#1}%
}
%    \end{macrocode}
%    \begin{macro}{\EmFi@setboolean}
%    \begin{macrocode}
\def\EmFi@setboolean#1#2{%
  \EdefSanitize\EmFi@temp{#2}%
  \ifx\EmFi@temp\EmFi@S@true
    \csname EmFi@#1true\endcsname
  \else
    \ifx\EmFi@temp\EmFi@S@false
      \csname EmFi@#1false\endcsname
    \else
      \EmFi@Error{%
        Unknown value `\EmFi@temp' for key `#1'.\MessageBreak
        Supported values: `true', `false'%
      }\@ehc
    \fi
  \fi
}
%    \end{macrocode}
%    \end{macro}
%    \begin{macrocode}
\kv@define@key{EmFiFi}{visible}[true]{%
  \EmFi@setboolean{visible}{#1}%
}
%    \end{macrocode}
%    \begin{macrocode}
\kv@define@key{EmFiFi}{edit}[true]{%
  \EmFi@setboolean{edit}{#1}%
}
%    \end{macrocode}
%
%    \begin{macro}{\EmFi@sortkeys}
%    \begin{macrocode}
\def\EmFi@sortkeys{}
%    \end{macrocode}
%    \end{macro}
%    \begin{macro}{\EmFi@sortorders}
%    \begin{macrocode}
\def\EmFi@sortorders{}
%    \end{macrocode}
%    \end{macro}
%    \begin{macro}{\embedfilesort}
%    \begin{macrocode}
\def\embedfilesort{%
  \kvsetkeys{EmFiSo}%
}
%    \end{macrocode}
%    \end{macro}
%
% \subsection{Embed the file}
%
%    \begin{macro}{\embedfile}
%    \begin{macrocode}
\def\embedfile{%
  \ltx@ifnextchar[\EmFi@embedfile{\EmFi@embedfile[]}%
}
%    \end{macrocode}
%    \end{macro}
%    \begin{macro}{\EmFi@embedfile}
%    \begin{macrocode}
\def\EmFi@embedfile[#1]#2{%
  \ifEmFi@finished
    \EmFi@Error{%
      \string\embedfile\ltx@space after \string\embedfilefinish
    }{%
      The list of embedded files is already written.%
    }%
  \else
    \begingroup
      \def\EmFi@file{#2}%
      \kvsetkeys{EmFi}{#1}%
      \expandafter\expandafter\expandafter
      \ifx\expandafter\expandafter\expandafter
          \\\pdf@filesize{\EmFi@file}\\%
        \EmFi@Error{%
          File `\EmFi@file' not found%
        }{%
          The unknown file is not embedded.%
        }%
      \else
        \edef\EmFi@@filespec{%
          \pdf@escapestring{\EmFi@filespec}%
        }%
        \ifx\EmFi@ucfilespec\ltx@empty
          \let\EmFi@@ucfilespec\ltx@empty
        \else
          \EmFi@convert\EmFi@ucfilespec\EmFi@@ucfilespec
        \fi
        \ifx\EmFi@desc\ltx@empty
          \let\EmFi@@desc\ltx@empty
        \else
          \EmFi@convert\EmFi@desc\EmFi@@desc
        \fi
        \ifEmFi@item
          \let\do\EmFi@do
          \immediate\pdfobj{%
            <<%
              \EmFi@fieldlist
            >>%
          }%
          \edef\EmFi@ci{\the\pdflastobj}%
        \fi
        \immediate\pdfobj stream attr{%
          /Type/EmbeddedFile%
          \ifx\EmFi@mimetype\ltx@empty
          \else
            /Subtype/\pdf@escapename{\EmFi@mimetype}%
          \fi
          /Params<<%
            /ModDate(\pdf@filemoddate{\EmFi@file})%
            /Size \pdf@filesize{\EmFi@file}%
            /CheckSum<\pdf@filemdfivesum{\EmFi@file}>%
          >>%
        }file{\EmFi@file}\relax
        \EmFi@defobj{EmbeddedFile}%
        \immediate\pdfobj{%
          <<%
            /Type/Filespec%
            \ifx\EmFi@filesystem\ltx@empty
            \else
            /FS/\pdf@escapename{\EmFi@filesystem}%
            \fi
            /F(\EmFi@@filespec)%
            \ifx\EmFi@@ucfilespec\ltx@empty
            \else
              /UF(\EmFi@@ucfilespec)%
            \fi
            \ifx\EmFi@@desc\ltx@empty
            \else
              /Desc(\EmFi@@desc)%
            \fi
            /EF<<%
              /F \the\pdflastobj\ltx@space 0 R%
            >>%
            \ifEmFi@item
              /CI \EmFi@ci\ltx@space 0 R%
            \fi
          >>%
        }%
        \EmFi@defobj{Filespec}%
        \EmFi@add{%
          \EmFi@@filespec
        }{\the\pdflastobj\ltx@space 0 R}%
      \fi
    \endgroup
  \fi
}
%    \end{macrocode}
%    \end{macro}
%    \begin{macro}{\EmFi@do}
%    \begin{macrocode}
\def\EmFi@do#1{%
  \expandafter\ifx\csname EmFi@P@#1\endcsname\relax
    \expandafter\ifx\csname EmFi@V@#1\endcsname\relax
    \else
      /\pdf@escapename{#1}\csname EmFi@V@#1\endcsname
    \fi
  \else
    /\pdf@escapename{#1}<<%
      \expandafter\ifx\csname EmFi@V@#1\endcsname\relax
      \else
        /D\csname EmFi@V@#1\endcsname
      \fi
      /P(\csname EmFi@P@#1\endcsname)%
    >>%
  \fi
}
%    \end{macrocode}
%    \end{macro}
%    \begin{macro}{\EmFi@convert}
%    \begin{macrocode}
\def\EmFi@convert#1#2{%
  \ifnum\pdf@strcmp{\EmFi@stringmethod}{psd}=0 %
    \pdfstringdef\EmFi@temp{#1}%
    \let#2\EmFi@temp
  \else
    \edef#2{\pdf@escapestring{#1}}%
  \fi
}
%    \end{macrocode}
%    \end{macro}
%    \begin{macrocode}
\global\let\EmFi@list\ltx@empty
%    \end{macrocode}
%    \begin{macro}{\EmFi@add}
%    Sorting is done by the insertion sort algorithm.
%    Probably the sorting could be done more reliable.
%    However, the PDF specification is not too clear to me
%    regarding precise sorting rules (how to deal with
%    different encodings, escaped characters, \dots).
%    \begin{macrocode}
\def\EmFi@add#1#2{%
  \begingroup
    \ifx\EmFi@list\ltx@empty
      \xdef\EmFi@list{\noexpand\do{#1}{#2}}%
    \else
      \def\do##1##2{%
        \ifnum\pdf@strcmp{##1}{#1}>0 %
          \edef\x{%
            \toks@{%
              \the\toks@%
              \noexpand\do{#1}{#2}%
              \noexpand\do{##1}{##2}%
            }%
          }%
          \x
          \def\do####1####2{%
            \toks@\expandafter{\the\toks@\do{####1}{####2}}%
          }%
          \def\stop{%
            \xdef\EmFi@list{\the\toks@}%
          }%
        \else
          \toks@\expandafter{\the\toks@\do{##1}{##2}}%
        \fi
      }%
      \def\stop{%
        \xdef\EmFi@list{\the\toks@\noexpand\do{#1}{#2}}%
      }%
      \toks@{}%
      \EmFi@list\stop
    \fi
  \endgroup
}
%    \end{macrocode}
%    \end{macro}
%    \begin{macro}{\embedfilefinish}
%    \begin{macrocode}
\def\embedfilefinish{%
  \ifEmFi@finished
    \EmFi@Error{%
      Too many invocations of \string\embedfilefinish
    }{%
      The list of embedded files is already written.%
    }%
  \else
    \ifx\EmFi@list\ltx@empty
    \else
%    \end{macrocode}
%    Write |/EmbeddedFiles| entry.
%    \begin{macrocode}
      \global\EmFi@finishedtrue
      \begingroup
        \def\do##1##2{%
          (##1)##2%
        }%
        \immediate\pdfobj{%
          <<%
            /Names[\EmFi@list]%
          >>%
        }%
        \pdfnames{%
          /EmbeddedFiles \the\pdflastobj\ltx@space 0 R%
        }%
      \endgroup
%    \end{macrocode}
%    Write collection objects.
%    \begin{macrocode}
      \ifx\EmFi@initialfile\ltx@empty
      \else
        \EmFi@collectiontrue
      \fi
      \ifEmFi@collection
        \ifx\EmFi@initialfile\ltx@empty
          \let\EmFi@@initialfile\ltx@empty
        \else
          \edef\EmFi@@initialfile{%
            \pdf@escapestring{\EmFi@initialfile}%
          }%
        \fi
%    \end{macrocode}
%    Look for initial file among the embedded files.
%    \begin{macrocode}
        \begingroup
          \let\f=N%
          \def\do##1##2{%
            \def\x{##1}%
            \ifx\x\EmFi@@initialfile
              \let\f=Y%
              \let\do\ltx@gobbletwo
            \fi
          }%
          \EmFi@list
        \expandafter\endgroup
        \ifx\f Y%
        \else
          \@PackageWarningNoLine{embedfile}{%
            Missing initial file `\EmFi@initialfile'\MessageBreak
            among the embedded files%
          }%
          \let\EmFi@initialfile\ltx@empty
          \let\EmFi@@initialfile\ltx@empty
        \fi
        \ifcase\EmFi@sortcase
          \def\EmFi@temp{}%
        \or
          \def\EmFi@temp{%
            /S\EmFi@sortkeys
            /A \EmFi@sortorders
          }%
        \else
          \def\EmFi@temp{%
            /S[\EmFi@sortkeys]%
            /A[\EmFi@sortorders]%
          }%
        \fi
        \def\EmFi@@order##1{%
          \ifnum\EmFi@order>1 %
            /O ##1%
          \fi
        }%
        \immediate\pdfobj{%
          <<%
            \ifx\EmFi@schema\ltx@empty
            \else
              /Schema<<\EmFi@schema>>%
            \fi
            \ifx\EmFi@@initialfile\ltx@empty
            \else
              /D(\EmFi@@initialfile)%
            \fi
            \ifx\EmFi@view\EmFi@S@tile
              /View/T%
            \else\ifx\EmFi@view\EmFi@S@hidden
              /View/H%
            \fi\fi
            \ifx\EmFi@temp\ltx@empty
              \EmFi@temp
            \else
              /Sort<<\EmFi@temp>>%
            \fi
          >>%
        }%
        \pdfcatalog{%
          /Collection \the\pdflastobj\ltx@space0 R%
        }%
      \fi
    \fi
  \fi
}
%    \end{macrocode}
%    \end{macro}
%    \begin{macrocode}
\begingroup\expandafter\expandafter\expandafter\endgroup
\expandafter\ifx\csname AtEndDocument\endcsname\relax
\else
  \AtEndDocument{\embedfilefinish}%
\fi
%    \end{macrocode}
%
%    \begin{macrocode}
\EmFi@AtEnd%
%</package>
%    \end{macrocode}
%
% \section{Test}
%
% \subsection{Catcode checks for loading}
%
%    \begin{macrocode}
%<*test1>
%    \end{macrocode}
%    \begin{macrocode}
\catcode`\{=1 %
\catcode`\}=2 %
\catcode`\#=6 %
\catcode`\@=11 %
\expandafter\ifx\csname count@\endcsname\relax
  \countdef\count@=255 %
\fi
\expandafter\ifx\csname @gobble\endcsname\relax
  \long\def\@gobble#1{}%
\fi
\expandafter\ifx\csname @firstofone\endcsname\relax
  \long\def\@firstofone#1{#1}%
\fi
\expandafter\ifx\csname loop\endcsname\relax
  \expandafter\@firstofone
\else
  \expandafter\@gobble
\fi
{%
  \def\loop#1\repeat{%
    \def\body{#1}%
    \iterate
  }%
  \def\iterate{%
    \body
      \let\next\iterate
    \else
      \let\next\relax
    \fi
    \next
  }%
  \let\repeat=\fi
}%
\def\RestoreCatcodes{}
\count@=0 %
\loop
  \edef\RestoreCatcodes{%
    \RestoreCatcodes
    \catcode\the\count@=\the\catcode\count@\relax
  }%
\ifnum\count@<255 %
  \advance\count@ 1 %
\repeat

\def\RangeCatcodeInvalid#1#2{%
  \count@=#1\relax
  \loop
    \catcode\count@=15 %
  \ifnum\count@<#2\relax
    \advance\count@ 1 %
  \repeat
}
\def\RangeCatcodeCheck#1#2#3{%
  \count@=#1\relax
  \loop
    \ifnum#3=\catcode\count@
    \else
      \errmessage{%
        Character \the\count@\space
        with wrong catcode \the\catcode\count@\space
        instead of \number#3%
      }%
    \fi
  \ifnum\count@<#2\relax
    \advance\count@ 1 %
  \repeat
}
\def\space{ }
\expandafter\ifx\csname LoadCommand\endcsname\relax
  \def\LoadCommand{\input embedfile.sty\relax}%
\fi
\def\Test{%
  \RangeCatcodeInvalid{0}{47}%
  \RangeCatcodeInvalid{58}{64}%
  \RangeCatcodeInvalid{91}{96}%
  \RangeCatcodeInvalid{123}{255}%
  \catcode`\@=12 %
  \catcode`\\=0 %
  \catcode`\%=14 %
  \LoadCommand
  \RangeCatcodeCheck{0}{36}{15}%
  \RangeCatcodeCheck{37}{37}{14}%
  \RangeCatcodeCheck{38}{47}{15}%
  \RangeCatcodeCheck{48}{57}{12}%
  \RangeCatcodeCheck{58}{63}{15}%
  \RangeCatcodeCheck{64}{64}{12}%
  \RangeCatcodeCheck{65}{90}{11}%
  \RangeCatcodeCheck{91}{91}{15}%
  \RangeCatcodeCheck{92}{92}{0}%
  \RangeCatcodeCheck{93}{96}{15}%
  \RangeCatcodeCheck{97}{122}{11}%
  \RangeCatcodeCheck{123}{255}{15}%
  \RestoreCatcodes
}
\Test
\csname @@end\endcsname
\end
%    \end{macrocode}
%    \begin{macrocode}
%</test1>
%    \end{macrocode}
%
% \subsection{Simple test}
%
%    \begin{macrocode}
%<*test2>
\input embedfile.sty\relax
\embedfile[%
  stringmethod=escape,%
  mimetype=plain/text,%
  desc={LaTeX docstrip source archive for package `embedfile'},%
  id={embedfile.dtx}%
]{embedfile.dtx}
\nopagenumbers
Test (plain-\TeX): {\tt embedfile.dtx} should be embedded.%

\def\Test#1{%
  \par
  \embedfileifobjectexists{embedfile.dtx}{#1}{%
    Object #1 (embedfile.dtx): %
    \embedfilegetobject{embedfile.dtx}{#1}%
  }{%
    \errmessage{Missing object #1 (embedfile.dtx)}%
  }%
}
\Test{EmbeddedFile}
\Test{Filespec}
\embedfilefinish
\bye
%</test2>
%    \end{macrocode}
%    \begin{macrocode}
%<*test3>
\NeedsTeXFormat{LaTeX2e}
\let\SavedJobname\jobname
\def\jobname{embedfile}
\RequirePackage{dtx-attach}[2011/04/13]
\let\jobname\SavedJobname
\documentclass{minimal}
\begin{document}
  Test (\LaTeX): \texttt{embedfile.dtx} should be embedded.%
\end{document}
%</test3>
%    \end{macrocode}
%
% \subsection{Test for \iniTeX}
%
%    \begin{macrocode}
%<*test4>
\catcode`\{=1 %
\catcode`\}=2 %
\input ifluatex.sty %
\ifluatex
  \directlua{%
    tex.enableprimitives('', {%
      'pdflastobj',%
      'pdfnames',%
      'pdfobj',%
      'pdfoutput'%
    })%
  }%
\fi
\pdfoutput=1 %
\input embedfile.sty %
\shipout\hbox{}
\embedfile[%
  stringmethod=escape,%
  mimetype=plain/text,%
  desc={iniTeX source},%
]{\jobname.tex}
\embedfilefinish
\end
%</test4>
%    \end{macrocode}
%
% \section{Installation}
%
% \subsection{Download}
%
% \paragraph{Package.} This package is available on
% CTAN\footnote{\url{ftp://ftp.ctan.org/tex-archive/}}:
% \begin{description}
% \item[\CTAN{macros/latex/contrib/oberdiek/embedfile.dtx}] The source file.
% \item[\CTAN{macros/latex/contrib/oberdiek/embedfile.pdf}] Documentation.
% \end{description}
%
%
% \paragraph{Bundle.} All the packages of the bundle `oberdiek'
% are also available in a TDS compliant ZIP archive. There
% the packages are already unpacked and the documentation files
% are generated. The files and directories obey the TDS standard.
% \begin{description}
% \item[\CTAN{install/macros/latex/contrib/oberdiek.tds.zip}]
% \end{description}
% \emph{TDS} refers to the standard ``A Directory Structure
% for \TeX\ Files'' (\CTAN{tds/tds.pdf}). Directories
% with \xfile{texmf} in their name are usually organized this way.
%
% \subsection{Bundle installation}
%
% \paragraph{Unpacking.} Unpack the \xfile{oberdiek.tds.zip} in the
% TDS tree (also known as \xfile{texmf} tree) of your choice.
% Example (linux):
% \begin{quote}
%   |unzip oberdiek.tds.zip -d ~/texmf|
% \end{quote}
%
% \paragraph{Script installation.}
% Check the directory \xfile{TDS:scripts/oberdiek/} for
% scripts that need further installation steps.
% Package \xpackage{attachfile2} comes with the Perl script
% \xfile{pdfatfi.pl} that should be installed in such a way
% that it can be called as \texttt{pdfatfi}.
% Example (linux):
% \begin{quote}
%   |chmod +x scripts/oberdiek/pdfatfi.pl|\\
%   |cp scripts/oberdiek/pdfatfi.pl /usr/local/bin/|
% \end{quote}
%
% \subsection{Package installation}
%
% \paragraph{Unpacking.} The \xfile{.dtx} file is a self-extracting
% \docstrip\ archive. The files are extracted by running the
% \xfile{.dtx} through \plainTeX:
% \begin{quote}
%   \verb|tex embedfile.dtx|
% \end{quote}
%
% \paragraph{TDS.} Now the different files must be moved into
% the different directories in your installation TDS tree
% (also known as \xfile{texmf} tree):
% \begin{quote}
% \def\t{^^A
% \begin{tabular}{@{}>{\ttfamily}l@{ $\rightarrow$ }>{\ttfamily}l@{}}
%   embedfile.sty & tex/generic/oberdiek/embedfile.sty\\
%   dtx-attach.sty & tex/generic/oberdiek/dtx-attach.sty\\
%   embedfile.pdf & doc/latex/oberdiek/embedfile.pdf\\
%   embedfile-example-plain.tex & doc/latex/oberdiek/embedfile-example-plain.tex\\
%   embedfile-example-collection.tex & doc/latex/oberdiek/embedfile-example-collection.tex\\
%   test/embedfile-test1.tex & doc/latex/oberdiek/test/embedfile-test1.tex\\
%   test/embedfile-test2.tex & doc/latex/oberdiek/test/embedfile-test2.tex\\
%   test/embedfile-test3.tex & doc/latex/oberdiek/test/embedfile-test3.tex\\
%   test/embedfile-test4.tex & doc/latex/oberdiek/test/embedfile-test4.tex\\
%   embedfile.dtx & source/latex/oberdiek/embedfile.dtx\\
% \end{tabular}^^A
% }^^A
% \sbox0{\t}^^A
% \ifdim\wd0>\linewidth
%   \begingroup
%     \advance\linewidth by\leftmargin
%     \advance\linewidth by\rightmargin
%   \edef\x{\endgroup
%     \def\noexpand\lw{\the\linewidth}^^A
%   }\x
%   \def\lwbox{^^A
%     \leavevmode
%     \hbox to \linewidth{^^A
%       \kern-\leftmargin\relax
%       \hss
%       \usebox0
%       \hss
%       \kern-\rightmargin\relax
%     }^^A
%   }^^A
%   \ifdim\wd0>\lw
%     \sbox0{\small\t}^^A
%     \ifdim\wd0>\linewidth
%       \ifdim\wd0>\lw
%         \sbox0{\footnotesize\t}^^A
%         \ifdim\wd0>\linewidth
%           \ifdim\wd0>\lw
%             \sbox0{\scriptsize\t}^^A
%             \ifdim\wd0>\linewidth
%               \ifdim\wd0>\lw
%                 \sbox0{\tiny\t}^^A
%                 \ifdim\wd0>\linewidth
%                   \lwbox
%                 \else
%                   \usebox0
%                 \fi
%               \else
%                 \lwbox
%               \fi
%             \else
%               \usebox0
%             \fi
%           \else
%             \lwbox
%           \fi
%         \else
%           \usebox0
%         \fi
%       \else
%         \lwbox
%       \fi
%     \else
%       \usebox0
%     \fi
%   \else
%     \lwbox
%   \fi
% \else
%   \usebox0
% \fi
% \end{quote}
% If you have a \xfile{docstrip.cfg} that configures and enables \docstrip's
% TDS installing feature, then some files can already be in the right
% place, see the documentation of \docstrip.
%
% \subsection{Refresh file name databases}
%
% If your \TeX~distribution
% (\teTeX, \mikTeX, \dots) relies on file name databases, you must refresh
% these. For example, \teTeX\ users run \verb|texhash| or
% \verb|mktexlsr|.
%
% \subsection{Some details for the interested}
%
% \paragraph{Attached source.}
%
% The PDF documentation on CTAN also includes the
% \xfile{.dtx} source file. It can be extracted by
% AcrobatReader 6 or higher. Another option is \textsf{pdftk},
% e.g. unpack the file into the current directory:
% \begin{quote}
%   \verb|pdftk embedfile.pdf unpack_files output .|
% \end{quote}
%
% \paragraph{Unpacking with \LaTeX.}
% The \xfile{.dtx} chooses its action depending on the format:
% \begin{description}
% \item[\plainTeX:] Run \docstrip\ and extract the files.
% \item[\LaTeX:] Generate the documentation.
% \end{description}
% If you insist on using \LaTeX\ for \docstrip\ (really,
% \docstrip\ does not need \LaTeX), then inform the autodetect routine
% about your intention:
% \begin{quote}
%   \verb|latex \let\install=y% \iffalse meta-comment
%
% File: embedfile.dtx
% Version: 2011/04/13 v2.6
% Info: Embed files into PDF
%
% Copyright (C) 2006-2011 by
%    Heiko Oberdiek <heiko.oberdiek at googlemail.com>
%
% This work may be distributed and/or modified under the
% conditions of the LaTeX Project Public License, either
% version 1.3c of this license or (at your option) any later
% version. This version of this license is in
%    http://www.latex-project.org/lppl/lppl-1-3c.txt
% and the latest version of this license is in
%    http://www.latex-project.org/lppl.txt
% and version 1.3 or later is part of all distributions of
% LaTeX version 2005/12/01 or later.
%
% This work has the LPPL maintenance status "maintained".
%
% This Current Maintainer of this work is Heiko Oberdiek.
%
% The Base Interpreter refers to any `TeX-Format',
% because some files are installed in TDS:tex/generic//.
%
% This work consists of the main source file embedfile.dtx
% and the derived files
%    embedfile.sty, embedfile.pdf, embedfile.ins, embedfile.drv,
%    dtx-attach.sty, embedfile-example-plain.tex,
%    embedfile-example-collection.tex, embedfile-test1.tex,
%    embedfile-test2.tex, embedfile-test3.tex,
%    embedfile-test4.tex.
%
% Distribution:
%    CTAN:macros/latex/contrib/oberdiek/embedfile.dtx
%    CTAN:macros/latex/contrib/oberdiek/embedfile.pdf
%
% Unpacking:
%    (a) If embedfile.ins is present:
%           tex embedfile.ins
%    (b) Without embedfile.ins:
%           tex embedfile.dtx
%    (c) If you insist on using LaTeX
%           latex \let\install=y% \iffalse meta-comment
%
% File: embedfile.dtx
% Version: 2011/04/13 v2.6
% Info: Embed files into PDF
%
% Copyright (C) 2006-2011 by
%    Heiko Oberdiek <heiko.oberdiek at googlemail.com>
%
% This work may be distributed and/or modified under the
% conditions of the LaTeX Project Public License, either
% version 1.3c of this license or (at your option) any later
% version. This version of this license is in
%    http://www.latex-project.org/lppl/lppl-1-3c.txt
% and the latest version of this license is in
%    http://www.latex-project.org/lppl.txt
% and version 1.3 or later is part of all distributions of
% LaTeX version 2005/12/01 or later.
%
% This work has the LPPL maintenance status "maintained".
%
% This Current Maintainer of this work is Heiko Oberdiek.
%
% The Base Interpreter refers to any `TeX-Format',
% because some files are installed in TDS:tex/generic//.
%
% This work consists of the main source file embedfile.dtx
% and the derived files
%    embedfile.sty, embedfile.pdf, embedfile.ins, embedfile.drv,
%    dtx-attach.sty, embedfile-example-plain.tex,
%    embedfile-example-collection.tex, embedfile-test1.tex,
%    embedfile-test2.tex, embedfile-test3.tex,
%    embedfile-test4.tex.
%
% Distribution:
%    CTAN:macros/latex/contrib/oberdiek/embedfile.dtx
%    CTAN:macros/latex/contrib/oberdiek/embedfile.pdf
%
% Unpacking:
%    (a) If embedfile.ins is present:
%           tex embedfile.ins
%    (b) Without embedfile.ins:
%           tex embedfile.dtx
%    (c) If you insist on using LaTeX
%           latex \let\install=y\input{embedfile.dtx}
%        (quote the arguments according to the demands of your shell)
%
% Documentation:
%    (a) If embedfile.drv is present:
%           latex embedfile.drv
%    (b) Without embedfile.drv:
%           latex embedfile.dtx; ...
%    The class ltxdoc loads the configuration file ltxdoc.cfg
%    if available. Here you can specify further options, e.g.
%    use A4 as paper format:
%       \PassOptionsToClass{a4paper}{article}
%
%    Programm calls to get the documentation (example):
%       pdflatex embedfile.dtx
%       makeindex -s gind.ist embedfile.idx
%       pdflatex embedfile.dtx
%       makeindex -s gind.ist embedfile.idx
%       pdflatex embedfile.dtx
%
% Installation:
%    TDS:tex/generic/oberdiek/embedfile.sty
%    TDS:tex/generic/oberdiek/dtx-attach.sty
%    TDS:doc/latex/oberdiek/embedfile.pdf
%    TDS:doc/latex/oberdiek/embedfile-example-plain.tex
%    TDS:doc/latex/oberdiek/embedfile-example-collection.tex
%    TDS:doc/latex/oberdiek/test/embedfile-test1.tex
%    TDS:doc/latex/oberdiek/test/embedfile-test2.tex
%    TDS:doc/latex/oberdiek/test/embedfile-test3.tex
%    TDS:doc/latex/oberdiek/test/embedfile-test4.tex
%    TDS:source/latex/oberdiek/embedfile.dtx
%
%<*ignore>
\begingroup
  \catcode123=1 %
  \catcode125=2 %
  \def\x{LaTeX2e}%
\expandafter\endgroup
\ifcase 0\ifx\install y1\fi\expandafter
         \ifx\csname processbatchFile\endcsname\relax\else1\fi
         \ifx\fmtname\x\else 1\fi\relax
\else\csname fi\endcsname
%</ignore>
%<*install>
\input docstrip.tex
\Msg{************************************************************************}
\Msg{* Installation}
\Msg{* Package: embedfile 2011/04/13 v2.6 Embed files into PDF (HO)}
\Msg{************************************************************************}

\keepsilent
\askforoverwritefalse

\let\MetaPrefix\relax
\preamble

This is a generated file.

Project: embedfile
Version: 2011/04/13 v2.6

Copyright (C) 2006-2011 by
   Heiko Oberdiek <heiko.oberdiek at googlemail.com>

This work may be distributed and/or modified under the
conditions of the LaTeX Project Public License, either
version 1.3c of this license or (at your option) any later
version. This version of this license is in
   http://www.latex-project.org/lppl/lppl-1-3c.txt
and the latest version of this license is in
   http://www.latex-project.org/lppl.txt
and version 1.3 or later is part of all distributions of
LaTeX version 2005/12/01 or later.

This work has the LPPL maintenance status "maintained".

This Current Maintainer of this work is Heiko Oberdiek.

The Base Interpreter refers to any `TeX-Format',
because some files are installed in TDS:tex/generic//.

This work consists of the main source file embedfile.dtx
and the derived files
   embedfile.sty, embedfile.pdf, embedfile.ins, embedfile.drv,
   dtx-attach.sty, embedfile-example-plain.tex,
   embedfile-example-collection.tex, embedfile-test1.tex,
   embedfile-test2.tex, embedfile-test3.tex,
   embedfile-test4.tex.

\endpreamble
\let\MetaPrefix\DoubleperCent

\generate{%
  \file{embedfile.ins}{\from{embedfile.dtx}{install}}%
  \file{embedfile.drv}{\from{embedfile.dtx}{driver}}%
  \usedir{tex/generic/oberdiek}%
  \file{embedfile.sty}{\from{embedfile.dtx}{package}}%
  \usedir{tex/latex/oberdiek}%
  \file{dtx-attach.sty}{\from{embedfile.dtx}{dtxattach}}%
  \usedir{doc/latex/oberdiek}%
  \file{embedfile-example-plain.tex}{\from{embedfile.dtx}{exampleplain}}%
  \file{embedfile-example-collection.tex}{\from{embedfile.dtx}{examplecollection}}%
  \usedir{doc/latex/oberdiek/test}%
  \file{embedfile-test1.tex}{\from{embedfile.dtx}{test1}}%
  \file{embedfile-test2.tex}{\from{embedfile.dtx}{test2}}%
  \file{embedfile-test3.tex}{\from{embedfile.dtx}{test3}}%
  \file{embedfile-test4.tex}{\from{embedfile.dtx}{test4}}%
  \nopreamble
  \nopostamble
  \usedir{source/latex/oberdiek/catalogue}%
  \file{embedfile.xml}{\from{embedfile.dtx}{catalogue}}%
}

\catcode32=13\relax% active space
\let =\space%
\Msg{************************************************************************}
\Msg{*}
\Msg{* To finish the installation you have to move the following}
\Msg{* files into a directory searched by TeX:}
\Msg{*}
\Msg{*     embedfile.sty, dtx-attach.sty}
\Msg{*}
\Msg{* To produce the documentation run the file `embedfile.drv'}
\Msg{* through LaTeX.}
\Msg{*}
\Msg{* Happy TeXing!}
\Msg{*}
\Msg{************************************************************************}

\endbatchfile
%</install>
%<*ignore>
\fi
%</ignore>
%<*driver>
\NeedsTeXFormat{LaTeX2e}
\ProvidesFile{embedfile.drv}%
  [2011/04/13 v2.6 Embed files into PDF (HO)]%
\documentclass{ltxdoc}
\usepackage{holtxdoc}[2011/11/22]
\begin{document}
  \DocInput{embedfile.dtx}%
\end{document}
%</driver>
% \fi
%
% \CheckSum{1363}
%
% \CharacterTable
%  {Upper-case    \A\B\C\D\E\F\G\H\I\J\K\L\M\N\O\P\Q\R\S\T\U\V\W\X\Y\Z
%   Lower-case    \a\b\c\d\e\f\g\h\i\j\k\l\m\n\o\p\q\r\s\t\u\v\w\x\y\z
%   Digits        \0\1\2\3\4\5\6\7\8\9
%   Exclamation   \!     Double quote  \"     Hash (number) \#
%   Dollar        \$     Percent       \%     Ampersand     \&
%   Acute accent  \'     Left paren    \(     Right paren   \)
%   Asterisk      \*     Plus          \+     Comma         \,
%   Minus         \-     Point         \.     Solidus       \/
%   Colon         \:     Semicolon     \;     Less than     \<
%   Equals        \=     Greater than  \>     Question mark \?
%   Commercial at \@     Left bracket  \[     Backslash     \\
%   Right bracket \]     Circumflex    \^     Underscore    \_
%   Grave accent  \`     Left brace    \{     Vertical bar  \|
%   Right brace   \}     Tilde         \~}
%
% \GetFileInfo{embedfile.drv}
%
% \title{The \xpackage{embedfile} package}
% \date{2011/04/13 v2.6}
% \author{Heiko Oberdiek\\\xemail{heiko.oberdiek at googlemail.com}}
%
% \maketitle
%
% \begin{abstract}
% This package embeds files to a PDF document.
% Currently the only supported driver is \pdfTeX\ $>=$ 1.30 in PDF mode.
% \end{abstract}
%
% \tableofcontents
%
% \section{Documentation}
%
% \subsection{Introduction}
%
%    The PDF format (\cite{pdfspec}) allows the inclusion of files
%    inside the PDF document. The included files can be bound to an
%    annotation on a page. Or they can be recorded in a sorted
%    list of embedded files. The packages \xpackage{attachfile}
%    or \xpackage{attachfile2} follow the first approach, this
%    package uses the latter method.
%
% \subsubsection{Future development}
%
%    My dream is a large package that merges the features of
%    all these packages meantioned before:
%    \begin{itemize}
%    \item Files can be attached to a page.
%    \item Files can be attached to the document.
%    \item An easy user interface for simple, common tasks and
%          beginners.
%    \item An interface for the advanced users that want to setup
%          every detail.
%    \item Support of many drivers (pdftex, dvips, dvipdfm, \dots).
%    \item \dots
%    \end{itemize}
%    However, I have not managed to take the time for this project.
%    Instead:
%    \begin{itemize}
%    \item First I experimented with package \xpackage{attachfile},
%          adding driver support, fixing bugs, \dots. The result is
%          currently named as \xpackage{attachfile2}. It uses an external
%          script to get file properties (size, date, checksum, \dots).
%    \item In order to avoid an external program for getting basic
%          file properties I provided a patch ``EscapeAndOther'' for
%          pdfTeX that was accepted for version 1.30.
%    \item This package closes a gap left by the packages for attaching
%          files and allows the embedding of files to the document.
%          Also it makes use of the new primitives of \pdfTeX.
%    \end{itemize}
%
% \subsection{User interface}
%
%    This package \xpackage{embedfile} can be used with both
%    \LaTeX\ and \plainTeX. See \autoref{sec:plain} that
%    explains the use with \plainTeX\ by an example.
%    In \LaTeX\ the package is loaded as usually. There are no options.
%    \begin{quote}
%      \verb|\usepackage{embedfile}|
%    \end{quote}
%
%    \begin{declcs}{embedfile}
%      |[|\meta{options}|]| \M{file}
%    \end{declcs}
%    The macro \cs{embedfile} includes file \meta{file} and
%    attaches it to the PDF document. At the end of the document
%    the sorted list of embedded files are written. Thus you can
%    safely use \cs{embedfile} before \verb|\end{document}|.
%    Embedding files using \cs{AtEndDocument} will only work,
%    if \cs{AtEndDocument} is called before loading the package
%    \xpackage{embedfile}.
%
%    The \meta{options} are give as key value pairs. The following
%    keys are supported:
%    \begin{description}
%    \item[\xoption{filespec}]
%      This allows to override the file name
%      that appears in the PDF file. If you are using other than
%      simple file names (8-bit, path separators, \dots), look into
%      the PDF specification (\cite{pdfspec}).
%      There are rules how these file names must be written/encoded.
%      Avoid 8-bit characters
%      and other special characters, the behaviour is currently undefined.
%      Use option \xoption{ucfilespec} for more funny file names.
%      The string method, see below, is |escape| since version 2.4.
%
%      This name is also used as entry in a name tree
%      (see PDF specification: |/EmbeddedFiles|). Therefore the
%      value for \xoption{filespec} must be unique among all
%      embedded files. Also key \xoption{initialfiles} refers
%      to this name, if the file name and the value of \xoption{filespec}
%      are different.
%    \item[\xoption{ucfilespec}]
%      Since PDF 1.7 the file name may be provided in Unicode.
%      The conversion of the option value into a PDF string
%      is controlled by option \xoption{stringmethod}.
%    \item[\xoption{filesystem}]
%      This sets the entry \verb|/FS| in
%      the file specification dictionary, see PDF specification
%      (\cite{pdfspec}).
%      Example: \verb|filesystem=URL|.
%    \item[\xoption{mimetype}]
%      This sets the mime type (\cite{mime}) of the file,
%      see \autoref{sec:plain} for examples and \cite{types}
%      for a list of officially registered types.
%    \item[\xoption{desc}]
%      The description for the file.
%    \item[\xoption{stringmethod}]
%      The package must convert the values of the keys \xoption{ucfilespec}
%      and \xoption{desc} into a PDF string (before version 2.4: \xoption{filespec}
%      and \xoption{desc}). If \xpackage{hyperref}
%      is found, then its \cs{pdfstringdef} will be used, otherwise
%      \pdfTeX's \cs{pdfescapestring} is used. Value \verb|psd|
%      forces the use of \cs{pdfstringdef}, value \verb|escape|
%      the use of \cs{pdfescapestring}.
%    \item[\xoption{\meta{key}.value}] Sets the value of a collection
%      item property, see section \ref{sec:collection}.
%    \item[\xoption{\meta{key}.prefix}]
%      Sets the prefix of a collection item property,
%      see section \ref{sec:collection}.
%    \item[\xoption{id}]
%      The value must be an unique name. Macros
%      \cs{embedfileifobjectexists} and \cs{embedfilegetobject}
%      are using this name later.
%    \end{description}
%
%    \begin{declcs}{embedfilefinish}
%    \end{declcs}
%    The list of all embedded files must be added as data structure
%    in the PDF file. In case of \LaTeX\ this is automatically done.
%    The package uses \cs{AtEndDocument}. Then the list of all
%    files should be known. However, \plainTeX\ does not know
%    about \cs{AtEndDocument}. Thus the user must call
%    \cs{embedfilefinish} at the end of the document
%    after the last file is embedded.
%
%    \begin{declcs}{embedfilesetup} \M{options}
%    \end{declcs}
%    Options for \cs{embedfile} and collection support
%    can be set in \cs{embedfilesetup}.
%
% \subsection{Collection support (PDF 1.7)}
% \label{sec:collection}
%
%    Since PDF 1.7 the embedded files can form a \emph{collection}
%    (sometimes referred as \emph{package}), the main document
%    is called \emph{cover sheet}. See PDF specification
%    8.2.4 ``Collections'' and 3.10.5 ``Collection items'' \cite{pdfspec}.
%
%    Usually Acrobat Reader 7 or 8 shows the
%    embedded files in a table at the bottom with the following columns:
%    \begin{quote}
%    \begin{tabular}{@{}l|l|l|l@{}}
%       Name & Description & Modified & Size\\
%       \hline
%       \dots & \dots & \dots & \dots\\
%    \end{tabular}
%    \end{quote}
%    Acrobat Reader 10 shows the embeded files in the left panel
%    and adds a new column for the compressed size.
%
%    If the files form a collection, then they are displayed
%    in a table left or top (depending on option \xoption{view},
%    see \cs{embedfilesetup}).
%
%
%    Collection support is enabled automatically, if
%    it is used.
%
%    \begin{declcs}{embedfilesetup} \M{options}
%    \end{declcs}
%    The following options are supported in addition
%    to options for \cs{embedfile}:
%    \begin{description}
%    \item[\xoption{view}]
%      If the PDF file contains a collection, then Acrobat Reader 8
%      shows a line at the top below the menu bar and the toolbar.
%      It shows the current selected file, icons for changing the
%      view mode, an options menu.
%      The initial mode how the collection is presented is
%      set by this option \xoption{view}.
%      The following modes/values are supported, the default is |details|:
%        \begin{description}
%        \item[\texttt{details}] The full collection table is displayed
%          at the top below the collection bar.
%        \item[\texttt{tile}] The files of the collection are shown
%          in tile mode on the left.
%        \item[\texttt{hidden}] The collection table is not shown.
%        \end{description}
%    \item[\xoption{initialfile}]
%      Selects the file that is initially presented. Especially
%      useful for an embedded PDF file that is then shown instead
%      of the cover document. There must be an \cs{embedfile} command
%      somewhere whose value for key \xoption{filespec} is used here.
%      The \cs{embedfile} command can drop option \xoption{filespec}
%      if the file name is not different.
%    \end{description}
%
%    \begin{declcs}{embedfilefield} \M{key} \M{options}
%    \end{declcs}
%    Macro \cs{embedfilefield} defines a column/field in the collection table.
%    The name of the field is \meta{key}.
%    \begin{description}
%    \item[\xoption{type}]
%      sets the type of the field. The supported values are:
%      \begin{description}
%      \item[\texttt{text}] A text field. Its value is set in
%        \cs{embedfile} by option \xoption{\meta{key}.value}.
%      \item[\texttt{date}] A date field. Its value is set in
%        \cs{embedfile} by option \xoption{\meta{key}.value}.
%        A special format is required, see ``3.8.3 Dates'' \cite{pdfspec}.
%      \item[\texttt{number}] A field with an integer or float number.
%        Its value is set in \cs{embedfile} by option
%        \xoption{\meta{key}.value}.
%      \item[\texttt{file}] The file name of the embedded file.
%      \item[\texttt{desc}] The description text of the embedded file. It is
%       set in \cs{embedfile} by option \xoption{desc}.
%      \item[\texttt{moddate}] The modification date of the embedded file.
%      \item[\texttt{size}] The size of the embedded file.
%      \end{description}
%      All types allow the use of a prefix that is disregarded by
%      sorting. The prefix for this field is set in \cs{embedfile} by
%      option \xoption{\meta{key}.prefix}.
%    \item[\xoption{title}] sets the column title.
%    \item[\xoption{visible}] controls whether the column is presented:
%      \begin{description}
%      \item[\texttt{true}] shows the column.
%      \item[\texttt{false}] hides the column.
%      \end{description}
%      Default: \texttt{true}
%    \item[\xoption{edit}]
%      Allows the editing of field values. Does not seem to have an
%      effect for Acrobat Reader.
%      \begin{description}
%      \item[\texttt{true}] enables the feature, if available (depends
%        on the PDF viewer).
%      \item[\texttt{false}] disables the feature.
%      \end{description}
%      Default: \texttt{false}
%    \end{description}
%    The order of \cs{embedfilefield} statements defines the order
%    of the columns.
%
%    \begin{declcs}{embedfilesort} \M{key-sort-list}
%    \end{declcs}
%    The sort order of the embedded files are controlled by
%    macro \cs{embedfilesort}. \meta{key-sort-list} defines
%    the sort order. The key is a field name defined by \cs{embedfilefield}.
%    Its value is either \texttt{ascending} or \texttt{descending}.
%    The default is \texttt{ascending}.
%
% \subsection{Export of object references}
%
%    Caution: This feature is still experimental. It may be even removed
%    in future versions. Therefore feedback would be nice, if someone
%    has a useful application for this feature.
%
%    Object numbers are saved, if \xoption{id} is given in \cs{embedfile}.
%    The following objects are supported:
%    \begin{itemize}
%    \item \texttt{EmbeddedFile}
%    \item \texttt{Filespec}
%    \end{itemize}
%
%    \begin{declcs}{embedfileifobjectexists} \M{id} \M{type}
%      \M{then} \M{else}
%    \end{declcs}
%    Macro \cs{embedfileifobjectexists} tests whether object of \meta{type}
%    is available for the embedded file identified by \meta{id}.
%
%    \begin{declcs}{embedfilegetobject} \M{id} \M{type}
%    \end{declcs}
%    Macro \cs{embedfilegetobject} expands to the full object reference
%    object of \meta{type} for the embedded file identified by \meta{id}.
%
% \subsubsection{Example}
%
%\begin{quote}
%\begin{verbatim}
%\embedfile[id={foo}]{foo.pdf}
%\embedfileifobjectexists{foo}{Filespec}{%
%  \typeout{%
%    FileSpec object for `foo': %
%    \embedfilegetobject{foo}{Filespec}%
%  }%
%}{%
%  \typeout{No Filespec object for `foo'}%
%}
%\end{verbatim}
%\end{quote}
%
% \subsection{Examples}
% \label{sec:examples}
%
% \subsubsection{\plainTeX}
% \label{sec:plain}
%
%    The package can be used with \plainTeX. It can be used
%    with or without help from \xfile{miniltx.tex}.
%
%    If additionally package \xpackage{keyval} (\xpackage{graphicx})
%    is needed, load it first. Then package \xpackage{embedfile} avoids
%    a duplicate loading of package \xpackage{keyval}.
%
%    Because \plainTeX\ does not provide a hook at end of the document,
%    you have to call \cs{embedfilefinish} manually at the end
%    after the last embedded file.
%    \begin{macrocode}
%<*exampleplain>
%<<END
% Load packages
\input miniltx
% \def\Gin@driver{pdftex.def}
% \input graphicx.sty
\input embedfile.sty
\resetatcatcode

% default setting
\embedfilesetup{
  mimetype=text/plain
}

% Embed files
\embedfile[
  filespec=example.tex,
  desc={Source code (plain-TeX) of this example}
]{embedfile-example-plain.tex}

\embedfile[
  desc={Source of package `embedfile'}
]{embedfile.dtx}

\embedfile[
  mimetype=application/pdf,
  desc={Documentation of package `embedfile'}
]{embedfile.pdf}

% Some text
This example document contains three embedded files.

% End of document
\embedfilefinish % don't forget
\bye
%END
%</exampleplain>
%    \end{macrocode}
%
% \subsubsection{Collection example}
%
%    \begin{macrocode}
%<*examplecollection>
%<<END
\NeedsTeXFormat{LaTeX2e}
\documentclass{article}
\usepackage[bookmarks=false]{hyperref}
% provides \pdfstringdef that is then used by `title' and
% other keys.
\usepackage{embedfile}[2011/04/13]
\embedfilesetup{
  view=details,
  initialfile=embedfile.pdf
}
\embedfilefield{file}{
  type=file,
  title={File name}
}
\embedfilefield{description}{
  type=desc,
  title={Description}
}
\embedfilefield{date}{
  type=moddate,
  title={Date}
}
\embedfilefield{size}{
  type=size,
  title={Size}
}
\embedfilefield{type}{
  type=text,
  title={Type},
  visible=false
}
\embedfilesort{
  type,
  date=descending
}
\begin{document}
An example for embedded files as collection.
You need Acrobat Reader 8 or higher.

\embedfile[
  desc={Source file of package `embedfile'},
  description.prefix={Package: },
  type.value={DTX}
]{embedfile.dtx}

\embedfile[
  desc={Documentation of package `embedfile'},
  description.prefix={Package: },
  type.value={PDF}
]{embedfile.pdf}

\embedfile[
  desc={The source for this example},
  description.prefix={Example: },
  type.value={TEX}
]{\jobname.tex}

\end{document}
%END
%</examplecollection>
%    \end{macrocode}
%
% \subsection{Package \xpackage{dtx-attach}}
%
% Package \xpackage{dtx-attach} is just a small application of
% package \xpackage{embedfile}. I am using it for the CTAN
% documentation of my packages in \CTAN{macros/latex/contrib/oberdiek/}.
% It also serves as small example for the use of the package with \LaTeX.
%
%    \begin{macrocode}
%<*dtxattach>
\NeedsTeXFormat{LaTeX2e}
\ProvidesPackage{dtx-attach}
  [2011/04/13 v2.6 Embed \string\jobname.dtx (HO)]%
\RequirePackage{embedfile}[2011/04/13]
\embedfile[%
  stringmethod=escape,%
  mimetype=plain/text,%
  desc={LaTeX docstrip source archive for package `\jobname'}%
]{\jobname.dtx}
%</dtxattach>
%    \end{macrocode}
%
% \StopEventually{
% }
%
% \section{Implementation}
%
%    \begin{macrocode}
%<*package>
%    \end{macrocode}
%
% \subsection{Reload check and package identification}
%    Reload check, especially if the package is not used with \LaTeX.
%    \begin{macrocode}
\begingroup\catcode61\catcode48\catcode32=10\relax%
  \catcode13=5 % ^^M
  \endlinechar=13 %
  \catcode35=6 % #
  \catcode39=12 % '
  \catcode44=12 % ,
  \catcode45=12 % -
  \catcode46=12 % .
  \catcode58=12 % :
  \catcode64=11 % @
  \catcode123=1 % {
  \catcode125=2 % }
  \expandafter\let\expandafter\x\csname ver@embedfile.sty\endcsname
  \ifx\x\relax % plain-TeX, first loading
  \else
    \def\empty{}%
    \ifx\x\empty % LaTeX, first loading,
      % variable is initialized, but \ProvidesPackage not yet seen
    \else
      \expandafter\ifx\csname PackageInfo\endcsname\relax
        \def\x#1#2{%
          \immediate\write-1{Package #1 Info: #2.}%
        }%
      \else
        \def\x#1#2{\PackageInfo{#1}{#2, stopped}}%
      \fi
      \x{embedfile}{The package is already loaded}%
      \aftergroup\endinput
    \fi
  \fi
\endgroup%
%    \end{macrocode}
%    Package identification:
%    \begin{macrocode}
\begingroup\catcode61\catcode48\catcode32=10\relax%
  \catcode13=5 % ^^M
  \endlinechar=13 %
  \catcode35=6 % #
  \catcode39=12 % '
  \catcode40=12 % (
  \catcode41=12 % )
  \catcode44=12 % ,
  \catcode45=12 % -
  \catcode46=12 % .
  \catcode47=12 % /
  \catcode58=12 % :
  \catcode64=11 % @
  \catcode91=12 % [
  \catcode93=12 % ]
  \catcode123=1 % {
  \catcode125=2 % }
  \expandafter\ifx\csname ProvidesPackage\endcsname\relax
    \def\x#1#2#3[#4]{\endgroup
      \immediate\write-1{Package: #3 #4}%
      \xdef#1{#4}%
    }%
  \else
    \def\x#1#2[#3]{\endgroup
      #2[{#3}]%
      \ifx#1\@undefined
        \xdef#1{#3}%
      \fi
      \ifx#1\relax
        \xdef#1{#3}%
      \fi
    }%
  \fi
\expandafter\x\csname ver@embedfile.sty\endcsname
\ProvidesPackage{embedfile}%
  [2011/04/13 v2.6 Embed files into PDF (HO)]%
%    \end{macrocode}
%
% \subsection{Catcodes}
%
%    \begin{macrocode}
\begingroup\catcode61\catcode48\catcode32=10\relax%
  \catcode13=5 % ^^M
  \endlinechar=13 %
  \catcode123=1 % {
  \catcode125=2 % }
  \catcode64=11 % @
  \def\x{\endgroup
    \expandafter\edef\csname EmFi@AtEnd\endcsname{%
      \endlinechar=\the\endlinechar\relax
      \catcode13=\the\catcode13\relax
      \catcode32=\the\catcode32\relax
      \catcode35=\the\catcode35\relax
      \catcode61=\the\catcode61\relax
      \catcode64=\the\catcode64\relax
      \catcode123=\the\catcode123\relax
      \catcode125=\the\catcode125\relax
    }%
  }%
\x\catcode61\catcode48\catcode32=10\relax%
\catcode13=5 % ^^M
\endlinechar=13 %
\catcode35=6 % #
\catcode64=11 % @
\catcode123=1 % {
\catcode125=2 % }
\def\TMP@EnsureCode#1#2{%
  \edef\EmFi@AtEnd{%
    \EmFi@AtEnd
    \catcode#1=\the\catcode#1\relax
  }%
  \catcode#1=#2\relax
}
\TMP@EnsureCode{39}{12}% '
\TMP@EnsureCode{40}{12}% (
\TMP@EnsureCode{41}{12}% )
\TMP@EnsureCode{44}{12}% ,
\TMP@EnsureCode{46}{12}% .
\TMP@EnsureCode{47}{12}% /
\TMP@EnsureCode{58}{12}% :
\TMP@EnsureCode{60}{12}% <
\TMP@EnsureCode{62}{12}% >
\TMP@EnsureCode{91}{12}% [
\TMP@EnsureCode{93}{12}% ]
\TMP@EnsureCode{96}{12}% `
\edef\EmFi@AtEnd{\EmFi@AtEnd\noexpand\endinput}
%    \end{macrocode}
%
% \subsection{Tools}
%
%    \begin{macro}{\EmFi@RequirePackage}
%    \begin{macrocode}
\begingroup\expandafter\expandafter\expandafter\endgroup
\expandafter\ifx\csname RequirePackage\endcsname\relax
  \def\EmFi@RequirePackage#1[#2]{%
    \input #1.sty\relax
  }%
\else
  \let\EmFi@RequirePackage\RequirePackage
\fi
%    \end{macrocode}
%    \end{macro}
%
%    \begin{macro}{\EmFi@Error}
%    \begin{macrocode}
\EmFi@RequirePackage{infwarerr}[2007/09/09]%
\def\EmFi@Error{%
  \@PackageError{embedfile}%
}
%    \end{macrocode}
%    \end{macro}
%
% \subsection{Check for recent \pdfTeX\ in PDF mode}
%
%    Load package \xpackage{ifpdf} and check mode.
%    \begin{macrocode}
\EmFi@RequirePackage{ifpdf}[2007/09/09]
\ifpdf
\else
  \EmFi@Error{%
    Missing pdfTeX in PDF mode%
  }{%
    Currently other drivers are not supported. %
    Package loading is aborted.%
  }%
  \expandafter\EmFi@AtEnd
\fi%
%    \end{macrocode}
%    \begin{macrocode}
\EmFi@RequirePackage{pdftexcmds}[2007/11/11]
\EmFi@RequirePackage{ltxcmds}[2010/03/01]
\EmFi@RequirePackage{kvsetkeys}[2010/03/01]
\EmFi@RequirePackage{kvdefinekeys}[2010/03/01]
%    \end{macrocode}
%    Check version.
%    \begin{macrocode}
\begingroup\expandafter\expandafter\expandafter\endgroup
\expandafter\ifx\csname pdf@filesize\endcsname\relax
  \EmFi@Error{%
    Unsupported pdfTeX version%
  }{%
    At least version 1.30 is necessary. Package loading is aborted.%
  }%
  \expandafter\EmFi@AtEnd
\fi%
%    \end{macrocode}
%
% \subsection{Strings}
%
%    Minimal version of package \xpackage{pdfescape} is 2007/08/27 v1.5
%    because of \cs{EdefSanitize}.
%    \begin{macrocode}
\EmFi@RequirePackage{pdfescape}[2007/11/11]
%    \end{macrocode}
%
%    \begin{macrocode}
\def\EmFi@temp#1{%
  \expandafter\EdefSanitize\csname EmFi@S@#1\endcsname{#1}%
}
%    \end{macrocode}
%    \begin{macro}{\EmFi@details}
%    \begin{macrocode}
\EmFi@temp{details}%
%    \end{macrocode}
%    \end{macro}
%    \begin{macro}{\EmFi@tile}
%    \begin{macrocode}
\EmFi@temp{tile}%
%    \end{macrocode}
%    \end{macro}
%    \begin{macro}{\EmFi@hidden}
%    \begin{macrocode}
\EmFi@temp{hidden}%
%    \end{macrocode}
%    \end{macro}
%    \begin{macro}{\EmFi@S@text}
%    \begin{macrocode}
\EmFi@temp{text}
%    \end{macrocode}
%    \end{macro}
%    \begin{macro}{\EmFi@S@date}
%    \begin{macrocode}
\EmFi@temp{date}
%    \end{macrocode}
%    \end{macro}
%    \begin{macro}{\EmFi@S@number}
%    \begin{macrocode}
\EmFi@temp{number}
%    \end{macrocode}
%    \end{macro}
%    \begin{macro}{\EmFi@S@file}
%    \begin{macrocode}
\EmFi@temp{file}
%    \end{macrocode}
%    \end{macro}
%    \begin{macro}{\EmFi@S@desc}
%    \begin{macrocode}
\EmFi@temp{desc}
%    \end{macrocode}
%    \end{macro}
%    \begin{macro}{\EmFi@S@moddate}
%    \begin{macrocode}
\EmFi@temp{moddate}
%    \end{macrocode}
%    \end{macro}
%    \begin{macro}{\EmFi@S@creationdate}
%    \begin{macrocode}
\EmFi@temp{creationdate}
%    \end{macrocode}
%    \end{macro}
%    \begin{macro}{\EmFi@S@size}
%    \begin{macrocode}
\EmFi@temp{size}
%    \end{macrocode}
%    \end{macro}
%    \begin{macro}{\EmFi@S@ascending}
%    \begin{macrocode}
\EmFi@temp{ascending}
%    \end{macrocode}
%    \end{macro}
%    \begin{macro}{\EmFi@S@descending}
%    \begin{macrocode}
\EmFi@temp{descending}
%    \end{macrocode}
%    \end{macro}
%    \begin{macro}{\EmFi@S@true}
%    \begin{macrocode}
\EmFi@temp{true}
%    \end{macrocode}
%    \end{macro}
%    \begin{macro}{\EmFi@S@false}
%    \begin{macrocode}
\EmFi@temp{false}
%    \end{macrocode}
%    \end{macro}
%
% \subsection{Switches}
%
%    \begin{macro}{\ifEmFi@collection}
%    \begin{macrocode}
\ltx@newif\ifEmFi@collection
%    \end{macrocode}
%    \end{macro}
%    \begin{macro}{\ifEmFi@sort}
%    \begin{macrocode}
\ltx@newif\ifEmFi@sort
%    \end{macrocode}
%    \end{macro}
%    \begin{macro}{\ifEmFi@visible}
%    \begin{macrocode}
\ltx@newif\ifEmFi@visible
%    \end{macrocode}
%    \end{macro}
%    \begin{macro}{\ifEmFi@edit}
%    \begin{macrocode}
\ltx@newif\ifEmFi@edit
%    \end{macrocode}
%    \end{macro}
%    \begin{macro}{\ifEmFi@item}
%    \begin{macrocode}
\ltx@newif\ifEmFi@item
%    \end{macrocode}
%    \end{macro}
%    \begin{macro}{\ifEmFi@finished}
%    \begin{macrocode}
\ltx@newif\ifEmFi@finished
%    \end{macrocode}
%    \end{macro}
%    \begin{macro}{\ifEmFi@id}
%    \begin{macrocode}
\ltx@newif\ifEmFi@id
%    \end{macrocode}
%    \end{macro}
%
% \subsection{Key value definitions}
%
%    \begin{macro}{\EmFi@GlobalKey}
%    \begin{macrocode}
\def\EmFi@GlobalKey#1#2{%
  \global\expandafter\let\csname KV@#1@#2\expandafter\endcsname
                         \csname KV@#1@#2\endcsname
}
%    \end{macrocode}
%    \end{macro}
%    \begin{macro}{\EmFi@GlobalDefaultKey}
%    \begin{macrocode}
\def\EmFi@GlobalDefaultKey#1#2{%
  \EmFi@GlobalKey{#1}{#2}%
  \global\expandafter\let
      \csname KV@#1@#2@default\expandafter\endcsname
      \csname KV@#1@#2@default\endcsname
}
%    \end{macrocode}
%    \end{macro}
%
%    \begin{macro}{\EmFi@DefineKey}
%    \begin{macrocode}
\def\EmFi@DefineKey#1#2{%
  \kv@define@key{EmFi}{#1}{%
    \expandafter\def\csname EmFi@#1\endcsname{##1}%
  }%
  \expandafter\def\csname EmFi@#1\endcsname{#2}%
}
%    \end{macrocode}
%    \end{macro}
%    Subtype of the embedded file (optional).
%    \begin{macrocode}
\EmFi@DefineKey{mimetype}{}
%    \end{macrocode}
%    File specification string.
%    \begin{macrocode}
\EmFi@DefineKey{filespec}{\EmFi@file}
%    \end{macrocode}
%    File specification string in Unicode.
%    \begin{macrocode}
\EmFi@DefineKey{ucfilespec}{}
%    \end{macrocode}
%    File system (optional).
%    \begin{macrocode}
\EmFi@DefineKey{filesystem}{}
%    \end{macrocode}
%    Description (optional).
%    \begin{macrocode}
\EmFi@DefineKey{desc}{}
%    \end{macrocode}
%    Method for converting text to PDF strings.
%    \begin{macrocode}
\EmFi@DefineKey{stringmethod}{%
  \ifx\pdfstringdef\@undefined
    escape%
  \else
    \ifx\pdfstringdef\relax
      escape%
    \else
      psd%
    \fi
  \fi
}
%    \end{macrocode}
%
%    Option \xoption{id} as key for object numbers.
%    \begin{macrocode}
\kv@define@key{EmFi}{id}{%
  \def\EmFi@id{#1}%
  \EmFi@idtrue
}
%    \end{macrocode}
%    \begin{macro}{\EmFi@defobj}
%    \begin{macrocode}
\def\EmFi@defobj#1{%
  \ifEmFi@id
    \expandafter\xdef\csname EmFi@#1@\EmFi@id\endcsname{%
      \the\pdflastobj\ltx@space 0 R%
    }%
  \fi
}
%    \end{macrocode}
%    \end{macro}
%    \begin{macro}{\embedfileifobjectexists}
%    \begin{macrocode}
\def\embedfileifobjectexists#1#2{%
  \expandafter\ifx\csname EmFi@#2@#1\endcsname\relax
    \expandafter\ltx@secondoftwo
  \else
    \expandafter\ltx@firstoftwo
  \fi
}
%    \end{macrocode}
%    \end{macro}
%    \begin{macro}{\embedfilegetobject}
%    \begin{macrocode}
\def\embedfilegetobject#1#2{%
  \embedfileifobjectexists{#1}{#2}{%
    \csname EmFi@#2@#1\endcsname
  }{%
    0 0 R%
  }%
}
%    \end{macrocode}
%    \end{macro}
%
%    Initial view of the collection.
%    \begin{macrocode}
\kv@define@key{EmFi}{view}[]{%
  \EdefSanitize\EmFi@temp{#1}%
  \def\EmFi@next{%
    \global\EmFi@collectiontrue
  }%
  \ifx\EmFi@temp\ltx@empty
    \let\EmFi@view\EmFi@S@details
  \else\ifx\EmFi@temp\EmFi@S@details
    \let\EmFi@view\EmFi@S@details
  \else\ifx\EmFi@temp\EmFi@S@tile
    \let\EmFi@view\EmFi@S@tile
  \else\ifx\EmFi@temp\EmFi@S@hidden
    \let\EmFi@view\EmFi@S@hidden
  \else
    \let\EmFi@next\relax
    \EmFi@Error{%
      Unknown value `\EmFi@temp' for key `view'.\MessageBreak
      Supported values: `details', `tile', `hidden'.%
    }\@ehc
  \fi\fi\fi\fi
  \EmFi@next
}
%    \end{macrocode}
%    \begin{macrocode}
\EmFi@DefineKey{initialfile}{}
%    \end{macrocode}
%
%    \begin{macro}{\embedfilesetup}
%    \begin{macrocode}
\def\embedfilesetup{%
  \ifEmFi@finished
    \def\EmFi@next##1{}%
    \EmFi@Error{%
      \string\embedfilefield\ltx@space after \string\embedfilefinish
    }{%
      The list of embedded files is already written.%
    }%
  \else
    \def\EmFi@next{%
      \kvsetkeys{EmFi}%
    }%
  \fi
  \EmFi@next
}
%    \end{macrocode}
%    \end{macro}
%
%    \begin{macro}{\EmFi@schema}
%    \begin{macrocode}
\def\EmFi@schema{}
%    \end{macrocode}
%    \end{macro}
%    \begin{macro}{\EmFi@order}
%    \begin{macrocode}
\gdef\EmFi@order{0}
%    \end{macrocode}
%    \end{macro}
%    \begin{macro}{\EmFi@@order}
%    \begin{macrocode}
\let\EmFi@@order\relax
%    \end{macrocode}
%    \end{macro}
%    \begin{macro}{\EmFi@fieldlist}
%    \begin{macrocode}
\def\EmFi@fieldlist{}
%    \end{macrocode}
%    \end{macro}
%    \begin{macro}{\EmFi@sortcase}
%    \begin{macrocode}
\def\EmFi@sortcase{0}%
%    \end{macrocode}
%    \end{macro}
%    \begin{macro}{\embedfilefield}
%    \begin{macrocode}
\def\embedfilefield#1#2{%
  \ifEmFi@finished
    \EmFi@Error{%
      \string\embedfilefield\ltx@space after \string\embedfilefinish
    }{%
      The list of embedded files is already written.%
    }%
  \else
    \global\EmFi@collectiontrue
    \EdefSanitize\EmFi@key{#1}%
    \expandafter\ifx\csname KV@EmFi@\EmFi@key.prefix\endcsname\relax
      \begingroup
        \count@=\EmFi@order
        \advance\count@ 1 %
        \xdef\EmFi@order{\the\count@}%
        \let\EmFi@title\EmFi@key
        \let\EmFi@type\EmFi@S@text
        \EmFi@visibletrue
        \EmFi@editfalse
        \kvsetkeys{EmFiFi}{#2}%
        \EmFi@convert\EmFi@title\EmFi@title
        \xdef\EmFi@schema{%
          \EmFi@schema
          /\pdf@escapename{\EmFi@key}<<%
            /Subtype/%
            \ifx\EmFi@type\EmFi@S@date D%
            \else\ifx\EmFi@type\EmFi@S@number N%
            \else\ifx\EmFi@type\EmFi@S@file F%
            \else\ifx\EmFi@type\EmFi@S@desc Desc%
            \else\ifx\EmFi@type\EmFi@S@moddate ModDate%
            \else\ifx\EmFi@type\EmFi@S@creationdate CreationDate%
            \else\ifx\EmFi@type\EmFi@S@size Size%
            \else S%
            \fi\fi\fi\fi\fi\fi\fi
            /N(\EmFi@title)%
            \EmFi@@order{\EmFi@order}%
            \ifEmFi@visible
            \else
              /V false%
            \fi
            \ifEmFi@edit
              /E true%
            \fi
          >>%
        }%
        \let\do\relax
        \xdef\EmFi@fieldlist{%
          \EmFi@fieldlist
          \do{\EmFi@key}%
        }%
        \ifx\EmFi@type\EmFi@S@text
          \kv@define@key{EmFi}{\EmFi@key.value}{%
            \EmFi@itemtrue
            \def\EmFi@temp{##1}%
            \EmFi@convert\EmFi@temp\EmFi@temp
            \expandafter\def\csname EmFi@V@#1%
            \expandafter\endcsname\expandafter{%
              \expandafter(\EmFi@temp)%
            }%
          }%
          \EmFi@GlobalKey{EmFi}{\EmFi@key.value}%
        \else\ifx\EmFi@type\EmFi@S@date
          \kv@define@key{EmFi}{\EmFi@key.value}{%
            \EmFi@itemtrue
            \def\EmFi@temp{##1}%
            \EmFi@convert\EmFi@temp\EmFi@temp
            \expandafter\def\csname EmFi@V@#1%
            \expandafter\endcsname\expandafter{%
              \expandafter(\EmFi@temp)%
            }%
          }%
          \EmFi@GlobalKey{EmFi}{\EmFi@key.value}%
        \else\ifx\EmFi@type\EmFi@S@number
          \kv@define@key{EmFi}{\EmFi@key.value}{%
            \EmFi@itemtrue
            \expandafter\EdefSanitize\csname EmFi@V@#1\endcsname{ ##1}%
          }%
          \EmFi@GlobalKey{EmFi}{\EmFi@key.value}%
        \fi\fi\fi
        \kv@define@key{EmFi}{\EmFi@key.prefix}{%
          \EmFi@itemtrue
          \expandafter\def\csname EmFi@P@#1\endcsname{##1}%
        }%
        \EmFi@GlobalKey{EmFi}{\EmFi@key.prefix}%
        \kv@define@key{EmFiSo}{\EmFi@key}[ascending]{%
          \EdefSanitize\EmFi@temp{##1}%
          \ifx\EmFi@temp\EmFi@S@ascending
            \def\EmFi@temp{true}%
          \else\ifx\EmFi@temp\EmFi@S@descending
            \def\EmFi@temp{false}%
          \else
            \def\EmFi@temp{}%
            \EmFi@Error{%
              Unknown sort order `\EmFi@temp'.\MessageBreak
              Supported values: `\EmFi@S@ascending', %
              `\EmFi@S@descending
            }\@ehc
          \fi\fi
          \ifx\EmFi@temp\ltx@empty
          \else
            \xdef\EmFi@sortkeys{%
              \EmFi@sortkeys
              /\pdf@escapename{#1}%
            }%
            \ifx\EmFi@sortorders\ltx@empty
              \global\let\EmFi@sortorders\EmFi@temp
              \gdef\EmFi@sortcase{1}%
            \else
              \xdef\EmFi@sortorders{%
                \EmFi@sortorders
                \ltx@space
                \EmFi@temp
              }%
              \xdef\EmFi@sortcase{2}%
            \fi
          \fi
        }%
        \EmFi@GlobalDefaultKey{EmFiSo}\EmFi@key
      \endgroup
    \else
      \EmFi@Error{%
        Field `\EmFi@key' is already defined%
      }\@ehc
    \fi
  \fi
}
%    \end{macrocode}
%    \end{macro}
%    \begin{macrocode}
\kv@define@key{EmFiFi}{type}{%
  \EdefSanitize\EmFi@temp{#1}%
  \ifx\EmFi@temp\EmFi@S@text
    \let\EmFi@type\EmFi@temp
  \else\ifx\EmFi@temp\EmFi@S@date
    \let\EmFi@type\EmFi@temp
  \else\ifx\EmFi@temp\EmFi@S@number
    \let\EmFi@type\EmFi@temp
  \else\ifx\EmFi@temp\EmFi@S@file
    \let\EmFi@type\EmFi@temp
  \else\ifx\EmFi@temp\EmFi@S@desc
    \let\EmFi@type\EmFi@temp
  \else\ifx\EmFi@temp\EmFi@S@moddate
    \let\EmFi@type\EmFi@temp
  \else\ifx\EmFi@temp\EmFi@S@creationdate
    \let\EmFi@type\EmFi@temp
  \else\ifx\EmFi@temp\EmFi@S@size
    \let\EmFi@type\EmFi@temp
  \else
    \EmFi@Error{%
      Unknown type `\EmFi@temp'.\MessageBreak
      Supported types: `text', `date', `number', `file',\MessageBreak
      `desc', `moddate', `creationdate', `size'%
    }%
  \fi\fi\fi\fi\fi\fi\fi\fi
}
%    \end{macrocode}
%    \begin{macrocode}
\kv@define@key{EmFiFi}{title}{%
  \def\EmFi@title{#1}%
}
%    \end{macrocode}
%    \begin{macro}{\EmFi@setboolean}
%    \begin{macrocode}
\def\EmFi@setboolean#1#2{%
  \EdefSanitize\EmFi@temp{#2}%
  \ifx\EmFi@temp\EmFi@S@true
    \csname EmFi@#1true\endcsname
  \else
    \ifx\EmFi@temp\EmFi@S@false
      \csname EmFi@#1false\endcsname
    \else
      \EmFi@Error{%
        Unknown value `\EmFi@temp' for key `#1'.\MessageBreak
        Supported values: `true', `false'%
      }\@ehc
    \fi
  \fi
}
%    \end{macrocode}
%    \end{macro}
%    \begin{macrocode}
\kv@define@key{EmFiFi}{visible}[true]{%
  \EmFi@setboolean{visible}{#1}%
}
%    \end{macrocode}
%    \begin{macrocode}
\kv@define@key{EmFiFi}{edit}[true]{%
  \EmFi@setboolean{edit}{#1}%
}
%    \end{macrocode}
%
%    \begin{macro}{\EmFi@sortkeys}
%    \begin{macrocode}
\def\EmFi@sortkeys{}
%    \end{macrocode}
%    \end{macro}
%    \begin{macro}{\EmFi@sortorders}
%    \begin{macrocode}
\def\EmFi@sortorders{}
%    \end{macrocode}
%    \end{macro}
%    \begin{macro}{\embedfilesort}
%    \begin{macrocode}
\def\embedfilesort{%
  \kvsetkeys{EmFiSo}%
}
%    \end{macrocode}
%    \end{macro}
%
% \subsection{Embed the file}
%
%    \begin{macro}{\embedfile}
%    \begin{macrocode}
\def\embedfile{%
  \ltx@ifnextchar[\EmFi@embedfile{\EmFi@embedfile[]}%
}
%    \end{macrocode}
%    \end{macro}
%    \begin{macro}{\EmFi@embedfile}
%    \begin{macrocode}
\def\EmFi@embedfile[#1]#2{%
  \ifEmFi@finished
    \EmFi@Error{%
      \string\embedfile\ltx@space after \string\embedfilefinish
    }{%
      The list of embedded files is already written.%
    }%
  \else
    \begingroup
      \def\EmFi@file{#2}%
      \kvsetkeys{EmFi}{#1}%
      \expandafter\expandafter\expandafter
      \ifx\expandafter\expandafter\expandafter
          \\\pdf@filesize{\EmFi@file}\\%
        \EmFi@Error{%
          File `\EmFi@file' not found%
        }{%
          The unknown file is not embedded.%
        }%
      \else
        \edef\EmFi@@filespec{%
          \pdf@escapestring{\EmFi@filespec}%
        }%
        \ifx\EmFi@ucfilespec\ltx@empty
          \let\EmFi@@ucfilespec\ltx@empty
        \else
          \EmFi@convert\EmFi@ucfilespec\EmFi@@ucfilespec
        \fi
        \ifx\EmFi@desc\ltx@empty
          \let\EmFi@@desc\ltx@empty
        \else
          \EmFi@convert\EmFi@desc\EmFi@@desc
        \fi
        \ifEmFi@item
          \let\do\EmFi@do
          \immediate\pdfobj{%
            <<%
              \EmFi@fieldlist
            >>%
          }%
          \edef\EmFi@ci{\the\pdflastobj}%
        \fi
        \immediate\pdfobj stream attr{%
          /Type/EmbeddedFile%
          \ifx\EmFi@mimetype\ltx@empty
          \else
            /Subtype/\pdf@escapename{\EmFi@mimetype}%
          \fi
          /Params<<%
            /ModDate(\pdf@filemoddate{\EmFi@file})%
            /Size \pdf@filesize{\EmFi@file}%
            /CheckSum<\pdf@filemdfivesum{\EmFi@file}>%
          >>%
        }file{\EmFi@file}\relax
        \EmFi@defobj{EmbeddedFile}%
        \immediate\pdfobj{%
          <<%
            /Type/Filespec%
            \ifx\EmFi@filesystem\ltx@empty
            \else
            /FS/\pdf@escapename{\EmFi@filesystem}%
            \fi
            /F(\EmFi@@filespec)%
            \ifx\EmFi@@ucfilespec\ltx@empty
            \else
              /UF(\EmFi@@ucfilespec)%
            \fi
            \ifx\EmFi@@desc\ltx@empty
            \else
              /Desc(\EmFi@@desc)%
            \fi
            /EF<<%
              /F \the\pdflastobj\ltx@space 0 R%
            >>%
            \ifEmFi@item
              /CI \EmFi@ci\ltx@space 0 R%
            \fi
          >>%
        }%
        \EmFi@defobj{Filespec}%
        \EmFi@add{%
          \EmFi@@filespec
        }{\the\pdflastobj\ltx@space 0 R}%
      \fi
    \endgroup
  \fi
}
%    \end{macrocode}
%    \end{macro}
%    \begin{macro}{\EmFi@do}
%    \begin{macrocode}
\def\EmFi@do#1{%
  \expandafter\ifx\csname EmFi@P@#1\endcsname\relax
    \expandafter\ifx\csname EmFi@V@#1\endcsname\relax
    \else
      /\pdf@escapename{#1}\csname EmFi@V@#1\endcsname
    \fi
  \else
    /\pdf@escapename{#1}<<%
      \expandafter\ifx\csname EmFi@V@#1\endcsname\relax
      \else
        /D\csname EmFi@V@#1\endcsname
      \fi
      /P(\csname EmFi@P@#1\endcsname)%
    >>%
  \fi
}
%    \end{macrocode}
%    \end{macro}
%    \begin{macro}{\EmFi@convert}
%    \begin{macrocode}
\def\EmFi@convert#1#2{%
  \ifnum\pdf@strcmp{\EmFi@stringmethod}{psd}=0 %
    \pdfstringdef\EmFi@temp{#1}%
    \let#2\EmFi@temp
  \else
    \edef#2{\pdf@escapestring{#1}}%
  \fi
}
%    \end{macrocode}
%    \end{macro}
%    \begin{macrocode}
\global\let\EmFi@list\ltx@empty
%    \end{macrocode}
%    \begin{macro}{\EmFi@add}
%    Sorting is done by the insertion sort algorithm.
%    Probably the sorting could be done more reliable.
%    However, the PDF specification is not too clear to me
%    regarding precise sorting rules (how to deal with
%    different encodings, escaped characters, \dots).
%    \begin{macrocode}
\def\EmFi@add#1#2{%
  \begingroup
    \ifx\EmFi@list\ltx@empty
      \xdef\EmFi@list{\noexpand\do{#1}{#2}}%
    \else
      \def\do##1##2{%
        \ifnum\pdf@strcmp{##1}{#1}>0 %
          \edef\x{%
            \toks@{%
              \the\toks@%
              \noexpand\do{#1}{#2}%
              \noexpand\do{##1}{##2}%
            }%
          }%
          \x
          \def\do####1####2{%
            \toks@\expandafter{\the\toks@\do{####1}{####2}}%
          }%
          \def\stop{%
            \xdef\EmFi@list{\the\toks@}%
          }%
        \else
          \toks@\expandafter{\the\toks@\do{##1}{##2}}%
        \fi
      }%
      \def\stop{%
        \xdef\EmFi@list{\the\toks@\noexpand\do{#1}{#2}}%
      }%
      \toks@{}%
      \EmFi@list\stop
    \fi
  \endgroup
}
%    \end{macrocode}
%    \end{macro}
%    \begin{macro}{\embedfilefinish}
%    \begin{macrocode}
\def\embedfilefinish{%
  \ifEmFi@finished
    \EmFi@Error{%
      Too many invocations of \string\embedfilefinish
    }{%
      The list of embedded files is already written.%
    }%
  \else
    \ifx\EmFi@list\ltx@empty
    \else
%    \end{macrocode}
%    Write |/EmbeddedFiles| entry.
%    \begin{macrocode}
      \global\EmFi@finishedtrue
      \begingroup
        \def\do##1##2{%
          (##1)##2%
        }%
        \immediate\pdfobj{%
          <<%
            /Names[\EmFi@list]%
          >>%
        }%
        \pdfnames{%
          /EmbeddedFiles \the\pdflastobj\ltx@space 0 R%
        }%
      \endgroup
%    \end{macrocode}
%    Write collection objects.
%    \begin{macrocode}
      \ifx\EmFi@initialfile\ltx@empty
      \else
        \EmFi@collectiontrue
      \fi
      \ifEmFi@collection
        \ifx\EmFi@initialfile\ltx@empty
          \let\EmFi@@initialfile\ltx@empty
        \else
          \edef\EmFi@@initialfile{%
            \pdf@escapestring{\EmFi@initialfile}%
          }%
        \fi
%    \end{macrocode}
%    Look for initial file among the embedded files.
%    \begin{macrocode}
        \begingroup
          \let\f=N%
          \def\do##1##2{%
            \def\x{##1}%
            \ifx\x\EmFi@@initialfile
              \let\f=Y%
              \let\do\ltx@gobbletwo
            \fi
          }%
          \EmFi@list
        \expandafter\endgroup
        \ifx\f Y%
        \else
          \@PackageWarningNoLine{embedfile}{%
            Missing initial file `\EmFi@initialfile'\MessageBreak
            among the embedded files%
          }%
          \let\EmFi@initialfile\ltx@empty
          \let\EmFi@@initialfile\ltx@empty
        \fi
        \ifcase\EmFi@sortcase
          \def\EmFi@temp{}%
        \or
          \def\EmFi@temp{%
            /S\EmFi@sortkeys
            /A \EmFi@sortorders
          }%
        \else
          \def\EmFi@temp{%
            /S[\EmFi@sortkeys]%
            /A[\EmFi@sortorders]%
          }%
        \fi
        \def\EmFi@@order##1{%
          \ifnum\EmFi@order>1 %
            /O ##1%
          \fi
        }%
        \immediate\pdfobj{%
          <<%
            \ifx\EmFi@schema\ltx@empty
            \else
              /Schema<<\EmFi@schema>>%
            \fi
            \ifx\EmFi@@initialfile\ltx@empty
            \else
              /D(\EmFi@@initialfile)%
            \fi
            \ifx\EmFi@view\EmFi@S@tile
              /View/T%
            \else\ifx\EmFi@view\EmFi@S@hidden
              /View/H%
            \fi\fi
            \ifx\EmFi@temp\ltx@empty
              \EmFi@temp
            \else
              /Sort<<\EmFi@temp>>%
            \fi
          >>%
        }%
        \pdfcatalog{%
          /Collection \the\pdflastobj\ltx@space0 R%
        }%
      \fi
    \fi
  \fi
}
%    \end{macrocode}
%    \end{macro}
%    \begin{macrocode}
\begingroup\expandafter\expandafter\expandafter\endgroup
\expandafter\ifx\csname AtEndDocument\endcsname\relax
\else
  \AtEndDocument{\embedfilefinish}%
\fi
%    \end{macrocode}
%
%    \begin{macrocode}
\EmFi@AtEnd%
%</package>
%    \end{macrocode}
%
% \section{Test}
%
% \subsection{Catcode checks for loading}
%
%    \begin{macrocode}
%<*test1>
%    \end{macrocode}
%    \begin{macrocode}
\catcode`\{=1 %
\catcode`\}=2 %
\catcode`\#=6 %
\catcode`\@=11 %
\expandafter\ifx\csname count@\endcsname\relax
  \countdef\count@=255 %
\fi
\expandafter\ifx\csname @gobble\endcsname\relax
  \long\def\@gobble#1{}%
\fi
\expandafter\ifx\csname @firstofone\endcsname\relax
  \long\def\@firstofone#1{#1}%
\fi
\expandafter\ifx\csname loop\endcsname\relax
  \expandafter\@firstofone
\else
  \expandafter\@gobble
\fi
{%
  \def\loop#1\repeat{%
    \def\body{#1}%
    \iterate
  }%
  \def\iterate{%
    \body
      \let\next\iterate
    \else
      \let\next\relax
    \fi
    \next
  }%
  \let\repeat=\fi
}%
\def\RestoreCatcodes{}
\count@=0 %
\loop
  \edef\RestoreCatcodes{%
    \RestoreCatcodes
    \catcode\the\count@=\the\catcode\count@\relax
  }%
\ifnum\count@<255 %
  \advance\count@ 1 %
\repeat

\def\RangeCatcodeInvalid#1#2{%
  \count@=#1\relax
  \loop
    \catcode\count@=15 %
  \ifnum\count@<#2\relax
    \advance\count@ 1 %
  \repeat
}
\def\RangeCatcodeCheck#1#2#3{%
  \count@=#1\relax
  \loop
    \ifnum#3=\catcode\count@
    \else
      \errmessage{%
        Character \the\count@\space
        with wrong catcode \the\catcode\count@\space
        instead of \number#3%
      }%
    \fi
  \ifnum\count@<#2\relax
    \advance\count@ 1 %
  \repeat
}
\def\space{ }
\expandafter\ifx\csname LoadCommand\endcsname\relax
  \def\LoadCommand{\input embedfile.sty\relax}%
\fi
\def\Test{%
  \RangeCatcodeInvalid{0}{47}%
  \RangeCatcodeInvalid{58}{64}%
  \RangeCatcodeInvalid{91}{96}%
  \RangeCatcodeInvalid{123}{255}%
  \catcode`\@=12 %
  \catcode`\\=0 %
  \catcode`\%=14 %
  \LoadCommand
  \RangeCatcodeCheck{0}{36}{15}%
  \RangeCatcodeCheck{37}{37}{14}%
  \RangeCatcodeCheck{38}{47}{15}%
  \RangeCatcodeCheck{48}{57}{12}%
  \RangeCatcodeCheck{58}{63}{15}%
  \RangeCatcodeCheck{64}{64}{12}%
  \RangeCatcodeCheck{65}{90}{11}%
  \RangeCatcodeCheck{91}{91}{15}%
  \RangeCatcodeCheck{92}{92}{0}%
  \RangeCatcodeCheck{93}{96}{15}%
  \RangeCatcodeCheck{97}{122}{11}%
  \RangeCatcodeCheck{123}{255}{15}%
  \RestoreCatcodes
}
\Test
\csname @@end\endcsname
\end
%    \end{macrocode}
%    \begin{macrocode}
%</test1>
%    \end{macrocode}
%
% \subsection{Simple test}
%
%    \begin{macrocode}
%<*test2>
\input embedfile.sty\relax
\embedfile[%
  stringmethod=escape,%
  mimetype=plain/text,%
  desc={LaTeX docstrip source archive for package `embedfile'},%
  id={embedfile.dtx}%
]{embedfile.dtx}
\nopagenumbers
Test (plain-\TeX): {\tt embedfile.dtx} should be embedded.%

\def\Test#1{%
  \par
  \embedfileifobjectexists{embedfile.dtx}{#1}{%
    Object #1 (embedfile.dtx): %
    \embedfilegetobject{embedfile.dtx}{#1}%
  }{%
    \errmessage{Missing object #1 (embedfile.dtx)}%
  }%
}
\Test{EmbeddedFile}
\Test{Filespec}
\embedfilefinish
\bye
%</test2>
%    \end{macrocode}
%    \begin{macrocode}
%<*test3>
\NeedsTeXFormat{LaTeX2e}
\let\SavedJobname\jobname
\def\jobname{embedfile}
\RequirePackage{dtx-attach}[2011/04/13]
\let\jobname\SavedJobname
\documentclass{minimal}
\begin{document}
  Test (\LaTeX): \texttt{embedfile.dtx} should be embedded.%
\end{document}
%</test3>
%    \end{macrocode}
%
% \subsection{Test for \iniTeX}
%
%    \begin{macrocode}
%<*test4>
\catcode`\{=1 %
\catcode`\}=2 %
\input ifluatex.sty %
\ifluatex
  \directlua{%
    tex.enableprimitives('', {%
      'pdflastobj',%
      'pdfnames',%
      'pdfobj',%
      'pdfoutput'%
    })%
  }%
\fi
\pdfoutput=1 %
\input embedfile.sty %
\shipout\hbox{}
\embedfile[%
  stringmethod=escape,%
  mimetype=plain/text,%
  desc={iniTeX source},%
]{\jobname.tex}
\embedfilefinish
\end
%</test4>
%    \end{macrocode}
%
% \section{Installation}
%
% \subsection{Download}
%
% \paragraph{Package.} This package is available on
% CTAN\footnote{\url{ftp://ftp.ctan.org/tex-archive/}}:
% \begin{description}
% \item[\CTAN{macros/latex/contrib/oberdiek/embedfile.dtx}] The source file.
% \item[\CTAN{macros/latex/contrib/oberdiek/embedfile.pdf}] Documentation.
% \end{description}
%
%
% \paragraph{Bundle.} All the packages of the bundle `oberdiek'
% are also available in a TDS compliant ZIP archive. There
% the packages are already unpacked and the documentation files
% are generated. The files and directories obey the TDS standard.
% \begin{description}
% \item[\CTAN{install/macros/latex/contrib/oberdiek.tds.zip}]
% \end{description}
% \emph{TDS} refers to the standard ``A Directory Structure
% for \TeX\ Files'' (\CTAN{tds/tds.pdf}). Directories
% with \xfile{texmf} in their name are usually organized this way.
%
% \subsection{Bundle installation}
%
% \paragraph{Unpacking.} Unpack the \xfile{oberdiek.tds.zip} in the
% TDS tree (also known as \xfile{texmf} tree) of your choice.
% Example (linux):
% \begin{quote}
%   |unzip oberdiek.tds.zip -d ~/texmf|
% \end{quote}
%
% \paragraph{Script installation.}
% Check the directory \xfile{TDS:scripts/oberdiek/} for
% scripts that need further installation steps.
% Package \xpackage{attachfile2} comes with the Perl script
% \xfile{pdfatfi.pl} that should be installed in such a way
% that it can be called as \texttt{pdfatfi}.
% Example (linux):
% \begin{quote}
%   |chmod +x scripts/oberdiek/pdfatfi.pl|\\
%   |cp scripts/oberdiek/pdfatfi.pl /usr/local/bin/|
% \end{quote}
%
% \subsection{Package installation}
%
% \paragraph{Unpacking.} The \xfile{.dtx} file is a self-extracting
% \docstrip\ archive. The files are extracted by running the
% \xfile{.dtx} through \plainTeX:
% \begin{quote}
%   \verb|tex embedfile.dtx|
% \end{quote}
%
% \paragraph{TDS.} Now the different files must be moved into
% the different directories in your installation TDS tree
% (also known as \xfile{texmf} tree):
% \begin{quote}
% \def\t{^^A
% \begin{tabular}{@{}>{\ttfamily}l@{ $\rightarrow$ }>{\ttfamily}l@{}}
%   embedfile.sty & tex/generic/oberdiek/embedfile.sty\\
%   dtx-attach.sty & tex/generic/oberdiek/dtx-attach.sty\\
%   embedfile.pdf & doc/latex/oberdiek/embedfile.pdf\\
%   embedfile-example-plain.tex & doc/latex/oberdiek/embedfile-example-plain.tex\\
%   embedfile-example-collection.tex & doc/latex/oberdiek/embedfile-example-collection.tex\\
%   test/embedfile-test1.tex & doc/latex/oberdiek/test/embedfile-test1.tex\\
%   test/embedfile-test2.tex & doc/latex/oberdiek/test/embedfile-test2.tex\\
%   test/embedfile-test3.tex & doc/latex/oberdiek/test/embedfile-test3.tex\\
%   test/embedfile-test4.tex & doc/latex/oberdiek/test/embedfile-test4.tex\\
%   embedfile.dtx & source/latex/oberdiek/embedfile.dtx\\
% \end{tabular}^^A
% }^^A
% \sbox0{\t}^^A
% \ifdim\wd0>\linewidth
%   \begingroup
%     \advance\linewidth by\leftmargin
%     \advance\linewidth by\rightmargin
%   \edef\x{\endgroup
%     \def\noexpand\lw{\the\linewidth}^^A
%   }\x
%   \def\lwbox{^^A
%     \leavevmode
%     \hbox to \linewidth{^^A
%       \kern-\leftmargin\relax
%       \hss
%       \usebox0
%       \hss
%       \kern-\rightmargin\relax
%     }^^A
%   }^^A
%   \ifdim\wd0>\lw
%     \sbox0{\small\t}^^A
%     \ifdim\wd0>\linewidth
%       \ifdim\wd0>\lw
%         \sbox0{\footnotesize\t}^^A
%         \ifdim\wd0>\linewidth
%           \ifdim\wd0>\lw
%             \sbox0{\scriptsize\t}^^A
%             \ifdim\wd0>\linewidth
%               \ifdim\wd0>\lw
%                 \sbox0{\tiny\t}^^A
%                 \ifdim\wd0>\linewidth
%                   \lwbox
%                 \else
%                   \usebox0
%                 \fi
%               \else
%                 \lwbox
%               \fi
%             \else
%               \usebox0
%             \fi
%           \else
%             \lwbox
%           \fi
%         \else
%           \usebox0
%         \fi
%       \else
%         \lwbox
%       \fi
%     \else
%       \usebox0
%     \fi
%   \else
%     \lwbox
%   \fi
% \else
%   \usebox0
% \fi
% \end{quote}
% If you have a \xfile{docstrip.cfg} that configures and enables \docstrip's
% TDS installing feature, then some files can already be in the right
% place, see the documentation of \docstrip.
%
% \subsection{Refresh file name databases}
%
% If your \TeX~distribution
% (\teTeX, \mikTeX, \dots) relies on file name databases, you must refresh
% these. For example, \teTeX\ users run \verb|texhash| or
% \verb|mktexlsr|.
%
% \subsection{Some details for the interested}
%
% \paragraph{Attached source.}
%
% The PDF documentation on CTAN also includes the
% \xfile{.dtx} source file. It can be extracted by
% AcrobatReader 6 or higher. Another option is \textsf{pdftk},
% e.g. unpack the file into the current directory:
% \begin{quote}
%   \verb|pdftk embedfile.pdf unpack_files output .|
% \end{quote}
%
% \paragraph{Unpacking with \LaTeX.}
% The \xfile{.dtx} chooses its action depending on the format:
% \begin{description}
% \item[\plainTeX:] Run \docstrip\ and extract the files.
% \item[\LaTeX:] Generate the documentation.
% \end{description}
% If you insist on using \LaTeX\ for \docstrip\ (really,
% \docstrip\ does not need \LaTeX), then inform the autodetect routine
% about your intention:
% \begin{quote}
%   \verb|latex \let\install=y\input{embedfile.dtx}|
% \end{quote}
% Do not forget to quote the argument according to the demands
% of your shell.
%
% \paragraph{Generating the documentation.}
% You can use both the \xfile{.dtx} or the \xfile{.drv} to generate
% the documentation. The process can be configured by the
% configuration file \xfile{ltxdoc.cfg}. For instance, put this
% line into this file, if you want to have A4 as paper format:
% \begin{quote}
%   \verb|\PassOptionsToClass{a4paper}{article}|
% \end{quote}
% An example follows how to generate the
% documentation with pdf\LaTeX:
% \begin{quote}
%\begin{verbatim}
%pdflatex embedfile.dtx
%makeindex -s gind.ist embedfile.idx
%pdflatex embedfile.dtx
%makeindex -s gind.ist embedfile.idx
%pdflatex embedfile.dtx
%\end{verbatim}
% \end{quote}
%
% \section{Catalogue}
%
% The following XML file can be used as source for the
% \href{http://mirror.ctan.org/help/Catalogue/catalogue.html}{\TeX\ Catalogue}.
% The elements \texttt{caption} and \texttt{description} are imported
% from the original XML file from the Catalogue.
% The name of the XML file in the Catalogue is \xfile{embedfile.xml}.
%    \begin{macrocode}
%<*catalogue>
<?xml version='1.0' encoding='us-ascii'?>
<!DOCTYPE entry SYSTEM 'catalogue.dtd'>
<entry datestamp='$Date$' modifier='$Author$' id='embedfile'>
  <name>embedfile</name>
  <caption>Embed files into PDF.</caption>
  <authorref id='auth:oberdiek'/>
  <copyright owner='Heiko Oberdiek' year='2006-2011'/>
  <license type='lppl1.3'/>
  <version number='2.6'/>
  <description>
    This package embeds files in a PDF document, using the PDF
    format's embedding operation (note the contrast with the attach
    operation used by the <xref refid='attachfile'>attachfile</xref>
    and <xref refid='attachfile2'>attachfile2</xref> packages).
    Currently only <xref refid='pdftex'>pdfTeX</xref> &gt;=1.30, in
    PDF mode, is supported.

    <p/>
    The package is part of the <xref refid='oberdiek'>oberdiek</xref>
    bundle.
  </description>
  <documentation details='Package documentation'
      href='ctan:/macros/latex/contrib/oberdiek/embedfile.pdf'/>
  <ctan file='true' path='/macros/latex/contrib/oberdiek/embedfile.dtx'/>
  <miktex location='oberdiek'/>
  <texlive location='oberdiek'/>
  <install path='/macros/latex/contrib/oberdiek/oberdiek.tds.zip'/>
</entry>
%</catalogue>
%    \end{macrocode}
%
% \begin{thebibliography}{9}
% \bibitem{attachfile}
%   Scott Pakin: \textit{The \xpackage{attachfile} package};
%   2005/02/20 v1.2;
%   \CTAN{macros/latex/contrib/attachfile/}.
%
% \bibitem{attachfile2}
%   Heiko Oberdiek: \textit{The \xpackage{attachfile2} package};
%   2006/08/16 v2.2;
%   \CTAN{macros/latex/contrib/oberdiek/attachfile2.pdf}.
%
% \bibitem{pdfspec}
%   Adobe Systems Incorporated:
%   \href{http://www.adobe.com/devnet/acrobat/pdfs/pdf_reference.pdf}%
%       {\textit{PDF Reference, Sixth Edition, Version 1.7}},%
%   Oktober 2006;
%   \url{http://www.adobe.com/devnet/pdf/pdf_reference.html}.
%
% \bibitem{mime}
%   Network Working Group: RFC 2046, \textit{Multipurpose Internet Mail Extensions (MIME) Part Two:
%     Media Types}, November 1996; \url{http://www.rfc-editor.org/}.
%
% \bibitem{types}
%   IANA (Internet Assigned Numbers Authority):
%   \textit{MIME Media Types}, May 2006;
%   \url{http://www.iana.org/assignments/media-types/}.
%
% \end{thebibliography}
%
% \begin{History}
%   \begin{Version}{2006/08/16 v1.0}
%   \item
%     First public version.
%   \end{Version}
%   \begin{Version}{2007/04/11 v1.1}
%   \item
%     Line ends sanitized.
%   \end{Version}
%   \begin{Version}{2007/09/09 v1.2}
%   \item
%     Fixes for plain-TeX, wrapper for package \xpackage{keyval} added.
%   \item
%     Catcode section rewritten.
%   \end{Version}
%   \begin{Version}{2007/10/28 v2.0}
%   \item
%     Collection support added (PDF 1.7).
%   \end{Version}
%   \begin{Version}{2007/10/29 v2.1}
%   \item
%     Export of object references by adding new option \xoption{id} and
%     new macros \cs{embedfileifobjectexists} and \cs{embedfilegetobject}.
%   \end{Version}
%   \begin{Version}{2007/11/11 v2.2}
%   \item
%     Use of package \xpackage{pdftexcmds} for \LuaTeX\ support.
%   \end{Version}
%   \begin{Version}{2007/11/25 v2.3}
%   \item
%     Fix in use of \cs{pdf@filesize}, bug introduced in previous version.
%   \end{Version}
%   \begin{Version}{2009/09/25 v2.4}
%   \item
%     Bug fix: If \xpackage{hyperref} is used with option \xoption{unicode},
%     the Unicode encoded file name causes trouble. Therefore \cs{pdfstringdef}
%     is now never used for option \xoption{filespec}, always method
%     \texttt{escape} is applied (Peter Cibulka).
%   \item
%     Bug fix for \xoption{initialfile}.
%   \item
%     Bug fix for file names in \texttt{/EmbeddedFiles}.
%   \item
%     New option \xoption{ucfilespec} for file name support in Unicode
%     (since PDF 1.7).
%   \end{Version}
%   \begin{Version}{2010/03/01 v2.5}
%   \item
%     Compatibility for \iniTeX.
%   \item
%     Package \xpackage{keyval} replaced by packages \xpackage{kvsetkeys}
%     and \xpackage{kvdefinekeys} because of compatibility for \iniTeX.
%   \item
%     TDS location moved from TDS:tex/latex/oberdiek/embedfile.sty to
%     TDS:tex/generic/oberdiek/embedfile.sty.
%   \end{Version}
%   \begin{Version}{2011/04/13 v2.6}
%   \item
%     Docu fixes (thanks Hans-Martin M\"unch).
%   \end{Version}
% \end{History}
%
% \PrintIndex
%
% \Finale
\endinput

%        (quote the arguments according to the demands of your shell)
%
% Documentation:
%    (a) If embedfile.drv is present:
%           latex embedfile.drv
%    (b) Without embedfile.drv:
%           latex embedfile.dtx; ...
%    The class ltxdoc loads the configuration file ltxdoc.cfg
%    if available. Here you can specify further options, e.g.
%    use A4 as paper format:
%       \PassOptionsToClass{a4paper}{article}
%
%    Programm calls to get the documentation (example):
%       pdflatex embedfile.dtx
%       makeindex -s gind.ist embedfile.idx
%       pdflatex embedfile.dtx
%       makeindex -s gind.ist embedfile.idx
%       pdflatex embedfile.dtx
%
% Installation:
%    TDS:tex/generic/oberdiek/embedfile.sty
%    TDS:tex/generic/oberdiek/dtx-attach.sty
%    TDS:doc/latex/oberdiek/embedfile.pdf
%    TDS:doc/latex/oberdiek/embedfile-example-plain.tex
%    TDS:doc/latex/oberdiek/embedfile-example-collection.tex
%    TDS:doc/latex/oberdiek/test/embedfile-test1.tex
%    TDS:doc/latex/oberdiek/test/embedfile-test2.tex
%    TDS:doc/latex/oberdiek/test/embedfile-test3.tex
%    TDS:doc/latex/oberdiek/test/embedfile-test4.tex
%    TDS:source/latex/oberdiek/embedfile.dtx
%
%<*ignore>
\begingroup
  \catcode123=1 %
  \catcode125=2 %
  \def\x{LaTeX2e}%
\expandafter\endgroup
\ifcase 0\ifx\install y1\fi\expandafter
         \ifx\csname processbatchFile\endcsname\relax\else1\fi
         \ifx\fmtname\x\else 1\fi\relax
\else\csname fi\endcsname
%</ignore>
%<*install>
\input docstrip.tex
\Msg{************************************************************************}
\Msg{* Installation}
\Msg{* Package: embedfile 2011/04/13 v2.6 Embed files into PDF (HO)}
\Msg{************************************************************************}

\keepsilent
\askforoverwritefalse

\let\MetaPrefix\relax
\preamble

This is a generated file.

Project: embedfile
Version: 2011/04/13 v2.6

Copyright (C) 2006-2011 by
   Heiko Oberdiek <heiko.oberdiek at googlemail.com>

This work may be distributed and/or modified under the
conditions of the LaTeX Project Public License, either
version 1.3c of this license or (at your option) any later
version. This version of this license is in
   http://www.latex-project.org/lppl/lppl-1-3c.txt
and the latest version of this license is in
   http://www.latex-project.org/lppl.txt
and version 1.3 or later is part of all distributions of
LaTeX version 2005/12/01 or later.

This work has the LPPL maintenance status "maintained".

This Current Maintainer of this work is Heiko Oberdiek.

The Base Interpreter refers to any `TeX-Format',
because some files are installed in TDS:tex/generic//.

This work consists of the main source file embedfile.dtx
and the derived files
   embedfile.sty, embedfile.pdf, embedfile.ins, embedfile.drv,
   dtx-attach.sty, embedfile-example-plain.tex,
   embedfile-example-collection.tex, embedfile-test1.tex,
   embedfile-test2.tex, embedfile-test3.tex,
   embedfile-test4.tex.

\endpreamble
\let\MetaPrefix\DoubleperCent

\generate{%
  \file{embedfile.ins}{\from{embedfile.dtx}{install}}%
  \file{embedfile.drv}{\from{embedfile.dtx}{driver}}%
  \usedir{tex/generic/oberdiek}%
  \file{embedfile.sty}{\from{embedfile.dtx}{package}}%
  \usedir{tex/latex/oberdiek}%
  \file{dtx-attach.sty}{\from{embedfile.dtx}{dtxattach}}%
  \usedir{doc/latex/oberdiek}%
  \file{embedfile-example-plain.tex}{\from{embedfile.dtx}{exampleplain}}%
  \file{embedfile-example-collection.tex}{\from{embedfile.dtx}{examplecollection}}%
  \usedir{doc/latex/oberdiek/test}%
  \file{embedfile-test1.tex}{\from{embedfile.dtx}{test1}}%
  \file{embedfile-test2.tex}{\from{embedfile.dtx}{test2}}%
  \file{embedfile-test3.tex}{\from{embedfile.dtx}{test3}}%
  \file{embedfile-test4.tex}{\from{embedfile.dtx}{test4}}%
  \nopreamble
  \nopostamble
  \usedir{source/latex/oberdiek/catalogue}%
  \file{embedfile.xml}{\from{embedfile.dtx}{catalogue}}%
}

\catcode32=13\relax% active space
\let =\space%
\Msg{************************************************************************}
\Msg{*}
\Msg{* To finish the installation you have to move the following}
\Msg{* files into a directory searched by TeX:}
\Msg{*}
\Msg{*     embedfile.sty, dtx-attach.sty}
\Msg{*}
\Msg{* To produce the documentation run the file `embedfile.drv'}
\Msg{* through LaTeX.}
\Msg{*}
\Msg{* Happy TeXing!}
\Msg{*}
\Msg{************************************************************************}

\endbatchfile
%</install>
%<*ignore>
\fi
%</ignore>
%<*driver>
\NeedsTeXFormat{LaTeX2e}
\ProvidesFile{embedfile.drv}%
  [2011/04/13 v2.6 Embed files into PDF (HO)]%
\documentclass{ltxdoc}
\usepackage{holtxdoc}[2011/11/22]
\begin{document}
  \DocInput{embedfile.dtx}%
\end{document}
%</driver>
% \fi
%
% \CheckSum{1363}
%
% \CharacterTable
%  {Upper-case    \A\B\C\D\E\F\G\H\I\J\K\L\M\N\O\P\Q\R\S\T\U\V\W\X\Y\Z
%   Lower-case    \a\b\c\d\e\f\g\h\i\j\k\l\m\n\o\p\q\r\s\t\u\v\w\x\y\z
%   Digits        \0\1\2\3\4\5\6\7\8\9
%   Exclamation   \!     Double quote  \"     Hash (number) \#
%   Dollar        \$     Percent       \%     Ampersand     \&
%   Acute accent  \'     Left paren    \(     Right paren   \)
%   Asterisk      \*     Plus          \+     Comma         \,
%   Minus         \-     Point         \.     Solidus       \/
%   Colon         \:     Semicolon     \;     Less than     \<
%   Equals        \=     Greater than  \>     Question mark \?
%   Commercial at \@     Left bracket  \[     Backslash     \\
%   Right bracket \]     Circumflex    \^     Underscore    \_
%   Grave accent  \`     Left brace    \{     Vertical bar  \|
%   Right brace   \}     Tilde         \~}
%
% \GetFileInfo{embedfile.drv}
%
% \title{The \xpackage{embedfile} package}
% \date{2011/04/13 v2.6}
% \author{Heiko Oberdiek\\\xemail{heiko.oberdiek at googlemail.com}}
%
% \maketitle
%
% \begin{abstract}
% This package embeds files to a PDF document.
% Currently the only supported driver is \pdfTeX\ $>=$ 1.30 in PDF mode.
% \end{abstract}
%
% \tableofcontents
%
% \section{Documentation}
%
% \subsection{Introduction}
%
%    The PDF format (\cite{pdfspec}) allows the inclusion of files
%    inside the PDF document. The included files can be bound to an
%    annotation on a page. Or they can be recorded in a sorted
%    list of embedded files. The packages \xpackage{attachfile}
%    or \xpackage{attachfile2} follow the first approach, this
%    package uses the latter method.
%
% \subsubsection{Future development}
%
%    My dream is a large package that merges the features of
%    all these packages meantioned before:
%    \begin{itemize}
%    \item Files can be attached to a page.
%    \item Files can be attached to the document.
%    \item An easy user interface for simple, common tasks and
%          beginners.
%    \item An interface for the advanced users that want to setup
%          every detail.
%    \item Support of many drivers (pdftex, dvips, dvipdfm, \dots).
%    \item \dots
%    \end{itemize}
%    However, I have not managed to take the time for this project.
%    Instead:
%    \begin{itemize}
%    \item First I experimented with package \xpackage{attachfile},
%          adding driver support, fixing bugs, \dots. The result is
%          currently named as \xpackage{attachfile2}. It uses an external
%          script to get file properties (size, date, checksum, \dots).
%    \item In order to avoid an external program for getting basic
%          file properties I provided a patch ``EscapeAndOther'' for
%          pdfTeX that was accepted for version 1.30.
%    \item This package closes a gap left by the packages for attaching
%          files and allows the embedding of files to the document.
%          Also it makes use of the new primitives of \pdfTeX.
%    \end{itemize}
%
% \subsection{User interface}
%
%    This package \xpackage{embedfile} can be used with both
%    \LaTeX\ and \plainTeX. See \autoref{sec:plain} that
%    explains the use with \plainTeX\ by an example.
%    In \LaTeX\ the package is loaded as usually. There are no options.
%    \begin{quote}
%      \verb|\usepackage{embedfile}|
%    \end{quote}
%
%    \begin{declcs}{embedfile}
%      |[|\meta{options}|]| \M{file}
%    \end{declcs}
%    The macro \cs{embedfile} includes file \meta{file} and
%    attaches it to the PDF document. At the end of the document
%    the sorted list of embedded files are written. Thus you can
%    safely use \cs{embedfile} before \verb|\end{document}|.
%    Embedding files using \cs{AtEndDocument} will only work,
%    if \cs{AtEndDocument} is called before loading the package
%    \xpackage{embedfile}.
%
%    The \meta{options} are give as key value pairs. The following
%    keys are supported:
%    \begin{description}
%    \item[\xoption{filespec}]
%      This allows to override the file name
%      that appears in the PDF file. If you are using other than
%      simple file names (8-bit, path separators, \dots), look into
%      the PDF specification (\cite{pdfspec}).
%      There are rules how these file names must be written/encoded.
%      Avoid 8-bit characters
%      and other special characters, the behaviour is currently undefined.
%      Use option \xoption{ucfilespec} for more funny file names.
%      The string method, see below, is |escape| since version 2.4.
%
%      This name is also used as entry in a name tree
%      (see PDF specification: |/EmbeddedFiles|). Therefore the
%      value for \xoption{filespec} must be unique among all
%      embedded files. Also key \xoption{initialfiles} refers
%      to this name, if the file name and the value of \xoption{filespec}
%      are different.
%    \item[\xoption{ucfilespec}]
%      Since PDF 1.7 the file name may be provided in Unicode.
%      The conversion of the option value into a PDF string
%      is controlled by option \xoption{stringmethod}.
%    \item[\xoption{filesystem}]
%      This sets the entry \verb|/FS| in
%      the file specification dictionary, see PDF specification
%      (\cite{pdfspec}).
%      Example: \verb|filesystem=URL|.
%    \item[\xoption{mimetype}]
%      This sets the mime type (\cite{mime}) of the file,
%      see \autoref{sec:plain} for examples and \cite{types}
%      for a list of officially registered types.
%    \item[\xoption{desc}]
%      The description for the file.
%    \item[\xoption{stringmethod}]
%      The package must convert the values of the keys \xoption{ucfilespec}
%      and \xoption{desc} into a PDF string (before version 2.4: \xoption{filespec}
%      and \xoption{desc}). If \xpackage{hyperref}
%      is found, then its \cs{pdfstringdef} will be used, otherwise
%      \pdfTeX's \cs{pdfescapestring} is used. Value \verb|psd|
%      forces the use of \cs{pdfstringdef}, value \verb|escape|
%      the use of \cs{pdfescapestring}.
%    \item[\xoption{\meta{key}.value}] Sets the value of a collection
%      item property, see section \ref{sec:collection}.
%    \item[\xoption{\meta{key}.prefix}]
%      Sets the prefix of a collection item property,
%      see section \ref{sec:collection}.
%    \item[\xoption{id}]
%      The value must be an unique name. Macros
%      \cs{embedfileifobjectexists} and \cs{embedfilegetobject}
%      are using this name later.
%    \end{description}
%
%    \begin{declcs}{embedfilefinish}
%    \end{declcs}
%    The list of all embedded files must be added as data structure
%    in the PDF file. In case of \LaTeX\ this is automatically done.
%    The package uses \cs{AtEndDocument}. Then the list of all
%    files should be known. However, \plainTeX\ does not know
%    about \cs{AtEndDocument}. Thus the user must call
%    \cs{embedfilefinish} at the end of the document
%    after the last file is embedded.
%
%    \begin{declcs}{embedfilesetup} \M{options}
%    \end{declcs}
%    Options for \cs{embedfile} and collection support
%    can be set in \cs{embedfilesetup}.
%
% \subsection{Collection support (PDF 1.7)}
% \label{sec:collection}
%
%    Since PDF 1.7 the embedded files can form a \emph{collection}
%    (sometimes referred as \emph{package}), the main document
%    is called \emph{cover sheet}. See PDF specification
%    8.2.4 ``Collections'' and 3.10.5 ``Collection items'' \cite{pdfspec}.
%
%    Usually Acrobat Reader 7 or 8 shows the
%    embedded files in a table at the bottom with the following columns:
%    \begin{quote}
%    \begin{tabular}{@{}l|l|l|l@{}}
%       Name & Description & Modified & Size\\
%       \hline
%       \dots & \dots & \dots & \dots\\
%    \end{tabular}
%    \end{quote}
%    Acrobat Reader 10 shows the embeded files in the left panel
%    and adds a new column for the compressed size.
%
%    If the files form a collection, then they are displayed
%    in a table left or top (depending on option \xoption{view},
%    see \cs{embedfilesetup}).
%
%
%    Collection support is enabled automatically, if
%    it is used.
%
%    \begin{declcs}{embedfilesetup} \M{options}
%    \end{declcs}
%    The following options are supported in addition
%    to options for \cs{embedfile}:
%    \begin{description}
%    \item[\xoption{view}]
%      If the PDF file contains a collection, then Acrobat Reader 8
%      shows a line at the top below the menu bar and the toolbar.
%      It shows the current selected file, icons for changing the
%      view mode, an options menu.
%      The initial mode how the collection is presented is
%      set by this option \xoption{view}.
%      The following modes/values are supported, the default is |details|:
%        \begin{description}
%        \item[\texttt{details}] The full collection table is displayed
%          at the top below the collection bar.
%        \item[\texttt{tile}] The files of the collection are shown
%          in tile mode on the left.
%        \item[\texttt{hidden}] The collection table is not shown.
%        \end{description}
%    \item[\xoption{initialfile}]
%      Selects the file that is initially presented. Especially
%      useful for an embedded PDF file that is then shown instead
%      of the cover document. There must be an \cs{embedfile} command
%      somewhere whose value for key \xoption{filespec} is used here.
%      The \cs{embedfile} command can drop option \xoption{filespec}
%      if the file name is not different.
%    \end{description}
%
%    \begin{declcs}{embedfilefield} \M{key} \M{options}
%    \end{declcs}
%    Macro \cs{embedfilefield} defines a column/field in the collection table.
%    The name of the field is \meta{key}.
%    \begin{description}
%    \item[\xoption{type}]
%      sets the type of the field. The supported values are:
%      \begin{description}
%      \item[\texttt{text}] A text field. Its value is set in
%        \cs{embedfile} by option \xoption{\meta{key}.value}.
%      \item[\texttt{date}] A date field. Its value is set in
%        \cs{embedfile} by option \xoption{\meta{key}.value}.
%        A special format is required, see ``3.8.3 Dates'' \cite{pdfspec}.
%      \item[\texttt{number}] A field with an integer or float number.
%        Its value is set in \cs{embedfile} by option
%        \xoption{\meta{key}.value}.
%      \item[\texttt{file}] The file name of the embedded file.
%      \item[\texttt{desc}] The description text of the embedded file. It is
%       set in \cs{embedfile} by option \xoption{desc}.
%      \item[\texttt{moddate}] The modification date of the embedded file.
%      \item[\texttt{size}] The size of the embedded file.
%      \end{description}
%      All types allow the use of a prefix that is disregarded by
%      sorting. The prefix for this field is set in \cs{embedfile} by
%      option \xoption{\meta{key}.prefix}.
%    \item[\xoption{title}] sets the column title.
%    \item[\xoption{visible}] controls whether the column is presented:
%      \begin{description}
%      \item[\texttt{true}] shows the column.
%      \item[\texttt{false}] hides the column.
%      \end{description}
%      Default: \texttt{true}
%    \item[\xoption{edit}]
%      Allows the editing of field values. Does not seem to have an
%      effect for Acrobat Reader.
%      \begin{description}
%      \item[\texttt{true}] enables the feature, if available (depends
%        on the PDF viewer).
%      \item[\texttt{false}] disables the feature.
%      \end{description}
%      Default: \texttt{false}
%    \end{description}
%    The order of \cs{embedfilefield} statements defines the order
%    of the columns.
%
%    \begin{declcs}{embedfilesort} \M{key-sort-list}
%    \end{declcs}
%    The sort order of the embedded files are controlled by
%    macro \cs{embedfilesort}. \meta{key-sort-list} defines
%    the sort order. The key is a field name defined by \cs{embedfilefield}.
%    Its value is either \texttt{ascending} or \texttt{descending}.
%    The default is \texttt{ascending}.
%
% \subsection{Export of object references}
%
%    Caution: This feature is still experimental. It may be even removed
%    in future versions. Therefore feedback would be nice, if someone
%    has a useful application for this feature.
%
%    Object numbers are saved, if \xoption{id} is given in \cs{embedfile}.
%    The following objects are supported:
%    \begin{itemize}
%    \item \texttt{EmbeddedFile}
%    \item \texttt{Filespec}
%    \end{itemize}
%
%    \begin{declcs}{embedfileifobjectexists} \M{id} \M{type}
%      \M{then} \M{else}
%    \end{declcs}
%    Macro \cs{embedfileifobjectexists} tests whether object of \meta{type}
%    is available for the embedded file identified by \meta{id}.
%
%    \begin{declcs}{embedfilegetobject} \M{id} \M{type}
%    \end{declcs}
%    Macro \cs{embedfilegetobject} expands to the full object reference
%    object of \meta{type} for the embedded file identified by \meta{id}.
%
% \subsubsection{Example}
%
%\begin{quote}
%\begin{verbatim}
%\embedfile[id={foo}]{foo.pdf}
%\embedfileifobjectexists{foo}{Filespec}{%
%  \typeout{%
%    FileSpec object for `foo': %
%    \embedfilegetobject{foo}{Filespec}%
%  }%
%}{%
%  \typeout{No Filespec object for `foo'}%
%}
%\end{verbatim}
%\end{quote}
%
% \subsection{Examples}
% \label{sec:examples}
%
% \subsubsection{\plainTeX}
% \label{sec:plain}
%
%    The package can be used with \plainTeX. It can be used
%    with or without help from \xfile{miniltx.tex}.
%
%    If additionally package \xpackage{keyval} (\xpackage{graphicx})
%    is needed, load it first. Then package \xpackage{embedfile} avoids
%    a duplicate loading of package \xpackage{keyval}.
%
%    Because \plainTeX\ does not provide a hook at end of the document,
%    you have to call \cs{embedfilefinish} manually at the end
%    after the last embedded file.
%    \begin{macrocode}
%<*exampleplain>
%<<END
% Load packages
\input miniltx
% \def\Gin@driver{pdftex.def}
% \input graphicx.sty
\input embedfile.sty
\resetatcatcode

% default setting
\embedfilesetup{
  mimetype=text/plain
}

% Embed files
\embedfile[
  filespec=example.tex,
  desc={Source code (plain-TeX) of this example}
]{embedfile-example-plain.tex}

\embedfile[
  desc={Source of package `embedfile'}
]{embedfile.dtx}

\embedfile[
  mimetype=application/pdf,
  desc={Documentation of package `embedfile'}
]{embedfile.pdf}

% Some text
This example document contains three embedded files.

% End of document
\embedfilefinish % don't forget
\bye
%END
%</exampleplain>
%    \end{macrocode}
%
% \subsubsection{Collection example}
%
%    \begin{macrocode}
%<*examplecollection>
%<<END
\NeedsTeXFormat{LaTeX2e}
\documentclass{article}
\usepackage[bookmarks=false]{hyperref}
% provides \pdfstringdef that is then used by `title' and
% other keys.
\usepackage{embedfile}[2011/04/13]
\embedfilesetup{
  view=details,
  initialfile=embedfile.pdf
}
\embedfilefield{file}{
  type=file,
  title={File name}
}
\embedfilefield{description}{
  type=desc,
  title={Description}
}
\embedfilefield{date}{
  type=moddate,
  title={Date}
}
\embedfilefield{size}{
  type=size,
  title={Size}
}
\embedfilefield{type}{
  type=text,
  title={Type},
  visible=false
}
\embedfilesort{
  type,
  date=descending
}
\begin{document}
An example for embedded files as collection.
You need Acrobat Reader 8 or higher.

\embedfile[
  desc={Source file of package `embedfile'},
  description.prefix={Package: },
  type.value={DTX}
]{embedfile.dtx}

\embedfile[
  desc={Documentation of package `embedfile'},
  description.prefix={Package: },
  type.value={PDF}
]{embedfile.pdf}

\embedfile[
  desc={The source for this example},
  description.prefix={Example: },
  type.value={TEX}
]{\jobname.tex}

\end{document}
%END
%</examplecollection>
%    \end{macrocode}
%
% \subsection{Package \xpackage{dtx-attach}}
%
% Package \xpackage{dtx-attach} is just a small application of
% package \xpackage{embedfile}. I am using it for the CTAN
% documentation of my packages in \CTAN{macros/latex/contrib/oberdiek/}.
% It also serves as small example for the use of the package with \LaTeX.
%
%    \begin{macrocode}
%<*dtxattach>
\NeedsTeXFormat{LaTeX2e}
\ProvidesPackage{dtx-attach}
  [2011/04/13 v2.6 Embed \string\jobname.dtx (HO)]%
\RequirePackage{embedfile}[2011/04/13]
\embedfile[%
  stringmethod=escape,%
  mimetype=plain/text,%
  desc={LaTeX docstrip source archive for package `\jobname'}%
]{\jobname.dtx}
%</dtxattach>
%    \end{macrocode}
%
% \StopEventually{
% }
%
% \section{Implementation}
%
%    \begin{macrocode}
%<*package>
%    \end{macrocode}
%
% \subsection{Reload check and package identification}
%    Reload check, especially if the package is not used with \LaTeX.
%    \begin{macrocode}
\begingroup\catcode61\catcode48\catcode32=10\relax%
  \catcode13=5 % ^^M
  \endlinechar=13 %
  \catcode35=6 % #
  \catcode39=12 % '
  \catcode44=12 % ,
  \catcode45=12 % -
  \catcode46=12 % .
  \catcode58=12 % :
  \catcode64=11 % @
  \catcode123=1 % {
  \catcode125=2 % }
  \expandafter\let\expandafter\x\csname ver@embedfile.sty\endcsname
  \ifx\x\relax % plain-TeX, first loading
  \else
    \def\empty{}%
    \ifx\x\empty % LaTeX, first loading,
      % variable is initialized, but \ProvidesPackage not yet seen
    \else
      \expandafter\ifx\csname PackageInfo\endcsname\relax
        \def\x#1#2{%
          \immediate\write-1{Package #1 Info: #2.}%
        }%
      \else
        \def\x#1#2{\PackageInfo{#1}{#2, stopped}}%
      \fi
      \x{embedfile}{The package is already loaded}%
      \aftergroup\endinput
    \fi
  \fi
\endgroup%
%    \end{macrocode}
%    Package identification:
%    \begin{macrocode}
\begingroup\catcode61\catcode48\catcode32=10\relax%
  \catcode13=5 % ^^M
  \endlinechar=13 %
  \catcode35=6 % #
  \catcode39=12 % '
  \catcode40=12 % (
  \catcode41=12 % )
  \catcode44=12 % ,
  \catcode45=12 % -
  \catcode46=12 % .
  \catcode47=12 % /
  \catcode58=12 % :
  \catcode64=11 % @
  \catcode91=12 % [
  \catcode93=12 % ]
  \catcode123=1 % {
  \catcode125=2 % }
  \expandafter\ifx\csname ProvidesPackage\endcsname\relax
    \def\x#1#2#3[#4]{\endgroup
      \immediate\write-1{Package: #3 #4}%
      \xdef#1{#4}%
    }%
  \else
    \def\x#1#2[#3]{\endgroup
      #2[{#3}]%
      \ifx#1\@undefined
        \xdef#1{#3}%
      \fi
      \ifx#1\relax
        \xdef#1{#3}%
      \fi
    }%
  \fi
\expandafter\x\csname ver@embedfile.sty\endcsname
\ProvidesPackage{embedfile}%
  [2011/04/13 v2.6 Embed files into PDF (HO)]%
%    \end{macrocode}
%
% \subsection{Catcodes}
%
%    \begin{macrocode}
\begingroup\catcode61\catcode48\catcode32=10\relax%
  \catcode13=5 % ^^M
  \endlinechar=13 %
  \catcode123=1 % {
  \catcode125=2 % }
  \catcode64=11 % @
  \def\x{\endgroup
    \expandafter\edef\csname EmFi@AtEnd\endcsname{%
      \endlinechar=\the\endlinechar\relax
      \catcode13=\the\catcode13\relax
      \catcode32=\the\catcode32\relax
      \catcode35=\the\catcode35\relax
      \catcode61=\the\catcode61\relax
      \catcode64=\the\catcode64\relax
      \catcode123=\the\catcode123\relax
      \catcode125=\the\catcode125\relax
    }%
  }%
\x\catcode61\catcode48\catcode32=10\relax%
\catcode13=5 % ^^M
\endlinechar=13 %
\catcode35=6 % #
\catcode64=11 % @
\catcode123=1 % {
\catcode125=2 % }
\def\TMP@EnsureCode#1#2{%
  \edef\EmFi@AtEnd{%
    \EmFi@AtEnd
    \catcode#1=\the\catcode#1\relax
  }%
  \catcode#1=#2\relax
}
\TMP@EnsureCode{39}{12}% '
\TMP@EnsureCode{40}{12}% (
\TMP@EnsureCode{41}{12}% )
\TMP@EnsureCode{44}{12}% ,
\TMP@EnsureCode{46}{12}% .
\TMP@EnsureCode{47}{12}% /
\TMP@EnsureCode{58}{12}% :
\TMP@EnsureCode{60}{12}% <
\TMP@EnsureCode{62}{12}% >
\TMP@EnsureCode{91}{12}% [
\TMP@EnsureCode{93}{12}% ]
\TMP@EnsureCode{96}{12}% `
\edef\EmFi@AtEnd{\EmFi@AtEnd\noexpand\endinput}
%    \end{macrocode}
%
% \subsection{Tools}
%
%    \begin{macro}{\EmFi@RequirePackage}
%    \begin{macrocode}
\begingroup\expandafter\expandafter\expandafter\endgroup
\expandafter\ifx\csname RequirePackage\endcsname\relax
  \def\EmFi@RequirePackage#1[#2]{%
    \input #1.sty\relax
  }%
\else
  \let\EmFi@RequirePackage\RequirePackage
\fi
%    \end{macrocode}
%    \end{macro}
%
%    \begin{macro}{\EmFi@Error}
%    \begin{macrocode}
\EmFi@RequirePackage{infwarerr}[2007/09/09]%
\def\EmFi@Error{%
  \@PackageError{embedfile}%
}
%    \end{macrocode}
%    \end{macro}
%
% \subsection{Check for recent \pdfTeX\ in PDF mode}
%
%    Load package \xpackage{ifpdf} and check mode.
%    \begin{macrocode}
\EmFi@RequirePackage{ifpdf}[2007/09/09]
\ifpdf
\else
  \EmFi@Error{%
    Missing pdfTeX in PDF mode%
  }{%
    Currently other drivers are not supported. %
    Package loading is aborted.%
  }%
  \expandafter\EmFi@AtEnd
\fi%
%    \end{macrocode}
%    \begin{macrocode}
\EmFi@RequirePackage{pdftexcmds}[2007/11/11]
\EmFi@RequirePackage{ltxcmds}[2010/03/01]
\EmFi@RequirePackage{kvsetkeys}[2010/03/01]
\EmFi@RequirePackage{kvdefinekeys}[2010/03/01]
%    \end{macrocode}
%    Check version.
%    \begin{macrocode}
\begingroup\expandafter\expandafter\expandafter\endgroup
\expandafter\ifx\csname pdf@filesize\endcsname\relax
  \EmFi@Error{%
    Unsupported pdfTeX version%
  }{%
    At least version 1.30 is necessary. Package loading is aborted.%
  }%
  \expandafter\EmFi@AtEnd
\fi%
%    \end{macrocode}
%
% \subsection{Strings}
%
%    Minimal version of package \xpackage{pdfescape} is 2007/08/27 v1.5
%    because of \cs{EdefSanitize}.
%    \begin{macrocode}
\EmFi@RequirePackage{pdfescape}[2007/11/11]
%    \end{macrocode}
%
%    \begin{macrocode}
\def\EmFi@temp#1{%
  \expandafter\EdefSanitize\csname EmFi@S@#1\endcsname{#1}%
}
%    \end{macrocode}
%    \begin{macro}{\EmFi@details}
%    \begin{macrocode}
\EmFi@temp{details}%
%    \end{macrocode}
%    \end{macro}
%    \begin{macro}{\EmFi@tile}
%    \begin{macrocode}
\EmFi@temp{tile}%
%    \end{macrocode}
%    \end{macro}
%    \begin{macro}{\EmFi@hidden}
%    \begin{macrocode}
\EmFi@temp{hidden}%
%    \end{macrocode}
%    \end{macro}
%    \begin{macro}{\EmFi@S@text}
%    \begin{macrocode}
\EmFi@temp{text}
%    \end{macrocode}
%    \end{macro}
%    \begin{macro}{\EmFi@S@date}
%    \begin{macrocode}
\EmFi@temp{date}
%    \end{macrocode}
%    \end{macro}
%    \begin{macro}{\EmFi@S@number}
%    \begin{macrocode}
\EmFi@temp{number}
%    \end{macrocode}
%    \end{macro}
%    \begin{macro}{\EmFi@S@file}
%    \begin{macrocode}
\EmFi@temp{file}
%    \end{macrocode}
%    \end{macro}
%    \begin{macro}{\EmFi@S@desc}
%    \begin{macrocode}
\EmFi@temp{desc}
%    \end{macrocode}
%    \end{macro}
%    \begin{macro}{\EmFi@S@moddate}
%    \begin{macrocode}
\EmFi@temp{moddate}
%    \end{macrocode}
%    \end{macro}
%    \begin{macro}{\EmFi@S@creationdate}
%    \begin{macrocode}
\EmFi@temp{creationdate}
%    \end{macrocode}
%    \end{macro}
%    \begin{macro}{\EmFi@S@size}
%    \begin{macrocode}
\EmFi@temp{size}
%    \end{macrocode}
%    \end{macro}
%    \begin{macro}{\EmFi@S@ascending}
%    \begin{macrocode}
\EmFi@temp{ascending}
%    \end{macrocode}
%    \end{macro}
%    \begin{macro}{\EmFi@S@descending}
%    \begin{macrocode}
\EmFi@temp{descending}
%    \end{macrocode}
%    \end{macro}
%    \begin{macro}{\EmFi@S@true}
%    \begin{macrocode}
\EmFi@temp{true}
%    \end{macrocode}
%    \end{macro}
%    \begin{macro}{\EmFi@S@false}
%    \begin{macrocode}
\EmFi@temp{false}
%    \end{macrocode}
%    \end{macro}
%
% \subsection{Switches}
%
%    \begin{macro}{\ifEmFi@collection}
%    \begin{macrocode}
\ltx@newif\ifEmFi@collection
%    \end{macrocode}
%    \end{macro}
%    \begin{macro}{\ifEmFi@sort}
%    \begin{macrocode}
\ltx@newif\ifEmFi@sort
%    \end{macrocode}
%    \end{macro}
%    \begin{macro}{\ifEmFi@visible}
%    \begin{macrocode}
\ltx@newif\ifEmFi@visible
%    \end{macrocode}
%    \end{macro}
%    \begin{macro}{\ifEmFi@edit}
%    \begin{macrocode}
\ltx@newif\ifEmFi@edit
%    \end{macrocode}
%    \end{macro}
%    \begin{macro}{\ifEmFi@item}
%    \begin{macrocode}
\ltx@newif\ifEmFi@item
%    \end{macrocode}
%    \end{macro}
%    \begin{macro}{\ifEmFi@finished}
%    \begin{macrocode}
\ltx@newif\ifEmFi@finished
%    \end{macrocode}
%    \end{macro}
%    \begin{macro}{\ifEmFi@id}
%    \begin{macrocode}
\ltx@newif\ifEmFi@id
%    \end{macrocode}
%    \end{macro}
%
% \subsection{Key value definitions}
%
%    \begin{macro}{\EmFi@GlobalKey}
%    \begin{macrocode}
\def\EmFi@GlobalKey#1#2{%
  \global\expandafter\let\csname KV@#1@#2\expandafter\endcsname
                         \csname KV@#1@#2\endcsname
}
%    \end{macrocode}
%    \end{macro}
%    \begin{macro}{\EmFi@GlobalDefaultKey}
%    \begin{macrocode}
\def\EmFi@GlobalDefaultKey#1#2{%
  \EmFi@GlobalKey{#1}{#2}%
  \global\expandafter\let
      \csname KV@#1@#2@default\expandafter\endcsname
      \csname KV@#1@#2@default\endcsname
}
%    \end{macrocode}
%    \end{macro}
%
%    \begin{macro}{\EmFi@DefineKey}
%    \begin{macrocode}
\def\EmFi@DefineKey#1#2{%
  \kv@define@key{EmFi}{#1}{%
    \expandafter\def\csname EmFi@#1\endcsname{##1}%
  }%
  \expandafter\def\csname EmFi@#1\endcsname{#2}%
}
%    \end{macrocode}
%    \end{macro}
%    Subtype of the embedded file (optional).
%    \begin{macrocode}
\EmFi@DefineKey{mimetype}{}
%    \end{macrocode}
%    File specification string.
%    \begin{macrocode}
\EmFi@DefineKey{filespec}{\EmFi@file}
%    \end{macrocode}
%    File specification string in Unicode.
%    \begin{macrocode}
\EmFi@DefineKey{ucfilespec}{}
%    \end{macrocode}
%    File system (optional).
%    \begin{macrocode}
\EmFi@DefineKey{filesystem}{}
%    \end{macrocode}
%    Description (optional).
%    \begin{macrocode}
\EmFi@DefineKey{desc}{}
%    \end{macrocode}
%    Method for converting text to PDF strings.
%    \begin{macrocode}
\EmFi@DefineKey{stringmethod}{%
  \ifx\pdfstringdef\@undefined
    escape%
  \else
    \ifx\pdfstringdef\relax
      escape%
    \else
      psd%
    \fi
  \fi
}
%    \end{macrocode}
%
%    Option \xoption{id} as key for object numbers.
%    \begin{macrocode}
\kv@define@key{EmFi}{id}{%
  \def\EmFi@id{#1}%
  \EmFi@idtrue
}
%    \end{macrocode}
%    \begin{macro}{\EmFi@defobj}
%    \begin{macrocode}
\def\EmFi@defobj#1{%
  \ifEmFi@id
    \expandafter\xdef\csname EmFi@#1@\EmFi@id\endcsname{%
      \the\pdflastobj\ltx@space 0 R%
    }%
  \fi
}
%    \end{macrocode}
%    \end{macro}
%    \begin{macro}{\embedfileifobjectexists}
%    \begin{macrocode}
\def\embedfileifobjectexists#1#2{%
  \expandafter\ifx\csname EmFi@#2@#1\endcsname\relax
    \expandafter\ltx@secondoftwo
  \else
    \expandafter\ltx@firstoftwo
  \fi
}
%    \end{macrocode}
%    \end{macro}
%    \begin{macro}{\embedfilegetobject}
%    \begin{macrocode}
\def\embedfilegetobject#1#2{%
  \embedfileifobjectexists{#1}{#2}{%
    \csname EmFi@#2@#1\endcsname
  }{%
    0 0 R%
  }%
}
%    \end{macrocode}
%    \end{macro}
%
%    Initial view of the collection.
%    \begin{macrocode}
\kv@define@key{EmFi}{view}[]{%
  \EdefSanitize\EmFi@temp{#1}%
  \def\EmFi@next{%
    \global\EmFi@collectiontrue
  }%
  \ifx\EmFi@temp\ltx@empty
    \let\EmFi@view\EmFi@S@details
  \else\ifx\EmFi@temp\EmFi@S@details
    \let\EmFi@view\EmFi@S@details
  \else\ifx\EmFi@temp\EmFi@S@tile
    \let\EmFi@view\EmFi@S@tile
  \else\ifx\EmFi@temp\EmFi@S@hidden
    \let\EmFi@view\EmFi@S@hidden
  \else
    \let\EmFi@next\relax
    \EmFi@Error{%
      Unknown value `\EmFi@temp' for key `view'.\MessageBreak
      Supported values: `details', `tile', `hidden'.%
    }\@ehc
  \fi\fi\fi\fi
  \EmFi@next
}
%    \end{macrocode}
%    \begin{macrocode}
\EmFi@DefineKey{initialfile}{}
%    \end{macrocode}
%
%    \begin{macro}{\embedfilesetup}
%    \begin{macrocode}
\def\embedfilesetup{%
  \ifEmFi@finished
    \def\EmFi@next##1{}%
    \EmFi@Error{%
      \string\embedfilefield\ltx@space after \string\embedfilefinish
    }{%
      The list of embedded files is already written.%
    }%
  \else
    \def\EmFi@next{%
      \kvsetkeys{EmFi}%
    }%
  \fi
  \EmFi@next
}
%    \end{macrocode}
%    \end{macro}
%
%    \begin{macro}{\EmFi@schema}
%    \begin{macrocode}
\def\EmFi@schema{}
%    \end{macrocode}
%    \end{macro}
%    \begin{macro}{\EmFi@order}
%    \begin{macrocode}
\gdef\EmFi@order{0}
%    \end{macrocode}
%    \end{macro}
%    \begin{macro}{\EmFi@@order}
%    \begin{macrocode}
\let\EmFi@@order\relax
%    \end{macrocode}
%    \end{macro}
%    \begin{macro}{\EmFi@fieldlist}
%    \begin{macrocode}
\def\EmFi@fieldlist{}
%    \end{macrocode}
%    \end{macro}
%    \begin{macro}{\EmFi@sortcase}
%    \begin{macrocode}
\def\EmFi@sortcase{0}%
%    \end{macrocode}
%    \end{macro}
%    \begin{macro}{\embedfilefield}
%    \begin{macrocode}
\def\embedfilefield#1#2{%
  \ifEmFi@finished
    \EmFi@Error{%
      \string\embedfilefield\ltx@space after \string\embedfilefinish
    }{%
      The list of embedded files is already written.%
    }%
  \else
    \global\EmFi@collectiontrue
    \EdefSanitize\EmFi@key{#1}%
    \expandafter\ifx\csname KV@EmFi@\EmFi@key.prefix\endcsname\relax
      \begingroup
        \count@=\EmFi@order
        \advance\count@ 1 %
        \xdef\EmFi@order{\the\count@}%
        \let\EmFi@title\EmFi@key
        \let\EmFi@type\EmFi@S@text
        \EmFi@visibletrue
        \EmFi@editfalse
        \kvsetkeys{EmFiFi}{#2}%
        \EmFi@convert\EmFi@title\EmFi@title
        \xdef\EmFi@schema{%
          \EmFi@schema
          /\pdf@escapename{\EmFi@key}<<%
            /Subtype/%
            \ifx\EmFi@type\EmFi@S@date D%
            \else\ifx\EmFi@type\EmFi@S@number N%
            \else\ifx\EmFi@type\EmFi@S@file F%
            \else\ifx\EmFi@type\EmFi@S@desc Desc%
            \else\ifx\EmFi@type\EmFi@S@moddate ModDate%
            \else\ifx\EmFi@type\EmFi@S@creationdate CreationDate%
            \else\ifx\EmFi@type\EmFi@S@size Size%
            \else S%
            \fi\fi\fi\fi\fi\fi\fi
            /N(\EmFi@title)%
            \EmFi@@order{\EmFi@order}%
            \ifEmFi@visible
            \else
              /V false%
            \fi
            \ifEmFi@edit
              /E true%
            \fi
          >>%
        }%
        \let\do\relax
        \xdef\EmFi@fieldlist{%
          \EmFi@fieldlist
          \do{\EmFi@key}%
        }%
        \ifx\EmFi@type\EmFi@S@text
          \kv@define@key{EmFi}{\EmFi@key.value}{%
            \EmFi@itemtrue
            \def\EmFi@temp{##1}%
            \EmFi@convert\EmFi@temp\EmFi@temp
            \expandafter\def\csname EmFi@V@#1%
            \expandafter\endcsname\expandafter{%
              \expandafter(\EmFi@temp)%
            }%
          }%
          \EmFi@GlobalKey{EmFi}{\EmFi@key.value}%
        \else\ifx\EmFi@type\EmFi@S@date
          \kv@define@key{EmFi}{\EmFi@key.value}{%
            \EmFi@itemtrue
            \def\EmFi@temp{##1}%
            \EmFi@convert\EmFi@temp\EmFi@temp
            \expandafter\def\csname EmFi@V@#1%
            \expandafter\endcsname\expandafter{%
              \expandafter(\EmFi@temp)%
            }%
          }%
          \EmFi@GlobalKey{EmFi}{\EmFi@key.value}%
        \else\ifx\EmFi@type\EmFi@S@number
          \kv@define@key{EmFi}{\EmFi@key.value}{%
            \EmFi@itemtrue
            \expandafter\EdefSanitize\csname EmFi@V@#1\endcsname{ ##1}%
          }%
          \EmFi@GlobalKey{EmFi}{\EmFi@key.value}%
        \fi\fi\fi
        \kv@define@key{EmFi}{\EmFi@key.prefix}{%
          \EmFi@itemtrue
          \expandafter\def\csname EmFi@P@#1\endcsname{##1}%
        }%
        \EmFi@GlobalKey{EmFi}{\EmFi@key.prefix}%
        \kv@define@key{EmFiSo}{\EmFi@key}[ascending]{%
          \EdefSanitize\EmFi@temp{##1}%
          \ifx\EmFi@temp\EmFi@S@ascending
            \def\EmFi@temp{true}%
          \else\ifx\EmFi@temp\EmFi@S@descending
            \def\EmFi@temp{false}%
          \else
            \def\EmFi@temp{}%
            \EmFi@Error{%
              Unknown sort order `\EmFi@temp'.\MessageBreak
              Supported values: `\EmFi@S@ascending', %
              `\EmFi@S@descending
            }\@ehc
          \fi\fi
          \ifx\EmFi@temp\ltx@empty
          \else
            \xdef\EmFi@sortkeys{%
              \EmFi@sortkeys
              /\pdf@escapename{#1}%
            }%
            \ifx\EmFi@sortorders\ltx@empty
              \global\let\EmFi@sortorders\EmFi@temp
              \gdef\EmFi@sortcase{1}%
            \else
              \xdef\EmFi@sortorders{%
                \EmFi@sortorders
                \ltx@space
                \EmFi@temp
              }%
              \xdef\EmFi@sortcase{2}%
            \fi
          \fi
        }%
        \EmFi@GlobalDefaultKey{EmFiSo}\EmFi@key
      \endgroup
    \else
      \EmFi@Error{%
        Field `\EmFi@key' is already defined%
      }\@ehc
    \fi
  \fi
}
%    \end{macrocode}
%    \end{macro}
%    \begin{macrocode}
\kv@define@key{EmFiFi}{type}{%
  \EdefSanitize\EmFi@temp{#1}%
  \ifx\EmFi@temp\EmFi@S@text
    \let\EmFi@type\EmFi@temp
  \else\ifx\EmFi@temp\EmFi@S@date
    \let\EmFi@type\EmFi@temp
  \else\ifx\EmFi@temp\EmFi@S@number
    \let\EmFi@type\EmFi@temp
  \else\ifx\EmFi@temp\EmFi@S@file
    \let\EmFi@type\EmFi@temp
  \else\ifx\EmFi@temp\EmFi@S@desc
    \let\EmFi@type\EmFi@temp
  \else\ifx\EmFi@temp\EmFi@S@moddate
    \let\EmFi@type\EmFi@temp
  \else\ifx\EmFi@temp\EmFi@S@creationdate
    \let\EmFi@type\EmFi@temp
  \else\ifx\EmFi@temp\EmFi@S@size
    \let\EmFi@type\EmFi@temp
  \else
    \EmFi@Error{%
      Unknown type `\EmFi@temp'.\MessageBreak
      Supported types: `text', `date', `number', `file',\MessageBreak
      `desc', `moddate', `creationdate', `size'%
    }%
  \fi\fi\fi\fi\fi\fi\fi\fi
}
%    \end{macrocode}
%    \begin{macrocode}
\kv@define@key{EmFiFi}{title}{%
  \def\EmFi@title{#1}%
}
%    \end{macrocode}
%    \begin{macro}{\EmFi@setboolean}
%    \begin{macrocode}
\def\EmFi@setboolean#1#2{%
  \EdefSanitize\EmFi@temp{#2}%
  \ifx\EmFi@temp\EmFi@S@true
    \csname EmFi@#1true\endcsname
  \else
    \ifx\EmFi@temp\EmFi@S@false
      \csname EmFi@#1false\endcsname
    \else
      \EmFi@Error{%
        Unknown value `\EmFi@temp' for key `#1'.\MessageBreak
        Supported values: `true', `false'%
      }\@ehc
    \fi
  \fi
}
%    \end{macrocode}
%    \end{macro}
%    \begin{macrocode}
\kv@define@key{EmFiFi}{visible}[true]{%
  \EmFi@setboolean{visible}{#1}%
}
%    \end{macrocode}
%    \begin{macrocode}
\kv@define@key{EmFiFi}{edit}[true]{%
  \EmFi@setboolean{edit}{#1}%
}
%    \end{macrocode}
%
%    \begin{macro}{\EmFi@sortkeys}
%    \begin{macrocode}
\def\EmFi@sortkeys{}
%    \end{macrocode}
%    \end{macro}
%    \begin{macro}{\EmFi@sortorders}
%    \begin{macrocode}
\def\EmFi@sortorders{}
%    \end{macrocode}
%    \end{macro}
%    \begin{macro}{\embedfilesort}
%    \begin{macrocode}
\def\embedfilesort{%
  \kvsetkeys{EmFiSo}%
}
%    \end{macrocode}
%    \end{macro}
%
% \subsection{Embed the file}
%
%    \begin{macro}{\embedfile}
%    \begin{macrocode}
\def\embedfile{%
  \ltx@ifnextchar[\EmFi@embedfile{\EmFi@embedfile[]}%
}
%    \end{macrocode}
%    \end{macro}
%    \begin{macro}{\EmFi@embedfile}
%    \begin{macrocode}
\def\EmFi@embedfile[#1]#2{%
  \ifEmFi@finished
    \EmFi@Error{%
      \string\embedfile\ltx@space after \string\embedfilefinish
    }{%
      The list of embedded files is already written.%
    }%
  \else
    \begingroup
      \def\EmFi@file{#2}%
      \kvsetkeys{EmFi}{#1}%
      \expandafter\expandafter\expandafter
      \ifx\expandafter\expandafter\expandafter
          \\\pdf@filesize{\EmFi@file}\\%
        \EmFi@Error{%
          File `\EmFi@file' not found%
        }{%
          The unknown file is not embedded.%
        }%
      \else
        \edef\EmFi@@filespec{%
          \pdf@escapestring{\EmFi@filespec}%
        }%
        \ifx\EmFi@ucfilespec\ltx@empty
          \let\EmFi@@ucfilespec\ltx@empty
        \else
          \EmFi@convert\EmFi@ucfilespec\EmFi@@ucfilespec
        \fi
        \ifx\EmFi@desc\ltx@empty
          \let\EmFi@@desc\ltx@empty
        \else
          \EmFi@convert\EmFi@desc\EmFi@@desc
        \fi
        \ifEmFi@item
          \let\do\EmFi@do
          \immediate\pdfobj{%
            <<%
              \EmFi@fieldlist
            >>%
          }%
          \edef\EmFi@ci{\the\pdflastobj}%
        \fi
        \immediate\pdfobj stream attr{%
          /Type/EmbeddedFile%
          \ifx\EmFi@mimetype\ltx@empty
          \else
            /Subtype/\pdf@escapename{\EmFi@mimetype}%
          \fi
          /Params<<%
            /ModDate(\pdf@filemoddate{\EmFi@file})%
            /Size \pdf@filesize{\EmFi@file}%
            /CheckSum<\pdf@filemdfivesum{\EmFi@file}>%
          >>%
        }file{\EmFi@file}\relax
        \EmFi@defobj{EmbeddedFile}%
        \immediate\pdfobj{%
          <<%
            /Type/Filespec%
            \ifx\EmFi@filesystem\ltx@empty
            \else
            /FS/\pdf@escapename{\EmFi@filesystem}%
            \fi
            /F(\EmFi@@filespec)%
            \ifx\EmFi@@ucfilespec\ltx@empty
            \else
              /UF(\EmFi@@ucfilespec)%
            \fi
            \ifx\EmFi@@desc\ltx@empty
            \else
              /Desc(\EmFi@@desc)%
            \fi
            /EF<<%
              /F \the\pdflastobj\ltx@space 0 R%
            >>%
            \ifEmFi@item
              /CI \EmFi@ci\ltx@space 0 R%
            \fi
          >>%
        }%
        \EmFi@defobj{Filespec}%
        \EmFi@add{%
          \EmFi@@filespec
        }{\the\pdflastobj\ltx@space 0 R}%
      \fi
    \endgroup
  \fi
}
%    \end{macrocode}
%    \end{macro}
%    \begin{macro}{\EmFi@do}
%    \begin{macrocode}
\def\EmFi@do#1{%
  \expandafter\ifx\csname EmFi@P@#1\endcsname\relax
    \expandafter\ifx\csname EmFi@V@#1\endcsname\relax
    \else
      /\pdf@escapename{#1}\csname EmFi@V@#1\endcsname
    \fi
  \else
    /\pdf@escapename{#1}<<%
      \expandafter\ifx\csname EmFi@V@#1\endcsname\relax
      \else
        /D\csname EmFi@V@#1\endcsname
      \fi
      /P(\csname EmFi@P@#1\endcsname)%
    >>%
  \fi
}
%    \end{macrocode}
%    \end{macro}
%    \begin{macro}{\EmFi@convert}
%    \begin{macrocode}
\def\EmFi@convert#1#2{%
  \ifnum\pdf@strcmp{\EmFi@stringmethod}{psd}=0 %
    \pdfstringdef\EmFi@temp{#1}%
    \let#2\EmFi@temp
  \else
    \edef#2{\pdf@escapestring{#1}}%
  \fi
}
%    \end{macrocode}
%    \end{macro}
%    \begin{macrocode}
\global\let\EmFi@list\ltx@empty
%    \end{macrocode}
%    \begin{macro}{\EmFi@add}
%    Sorting is done by the insertion sort algorithm.
%    Probably the sorting could be done more reliable.
%    However, the PDF specification is not too clear to me
%    regarding precise sorting rules (how to deal with
%    different encodings, escaped characters, \dots).
%    \begin{macrocode}
\def\EmFi@add#1#2{%
  \begingroup
    \ifx\EmFi@list\ltx@empty
      \xdef\EmFi@list{\noexpand\do{#1}{#2}}%
    \else
      \def\do##1##2{%
        \ifnum\pdf@strcmp{##1}{#1}>0 %
          \edef\x{%
            \toks@{%
              \the\toks@%
              \noexpand\do{#1}{#2}%
              \noexpand\do{##1}{##2}%
            }%
          }%
          \x
          \def\do####1####2{%
            \toks@\expandafter{\the\toks@\do{####1}{####2}}%
          }%
          \def\stop{%
            \xdef\EmFi@list{\the\toks@}%
          }%
        \else
          \toks@\expandafter{\the\toks@\do{##1}{##2}}%
        \fi
      }%
      \def\stop{%
        \xdef\EmFi@list{\the\toks@\noexpand\do{#1}{#2}}%
      }%
      \toks@{}%
      \EmFi@list\stop
    \fi
  \endgroup
}
%    \end{macrocode}
%    \end{macro}
%    \begin{macro}{\embedfilefinish}
%    \begin{macrocode}
\def\embedfilefinish{%
  \ifEmFi@finished
    \EmFi@Error{%
      Too many invocations of \string\embedfilefinish
    }{%
      The list of embedded files is already written.%
    }%
  \else
    \ifx\EmFi@list\ltx@empty
    \else
%    \end{macrocode}
%    Write |/EmbeddedFiles| entry.
%    \begin{macrocode}
      \global\EmFi@finishedtrue
      \begingroup
        \def\do##1##2{%
          (##1)##2%
        }%
        \immediate\pdfobj{%
          <<%
            /Names[\EmFi@list]%
          >>%
        }%
        \pdfnames{%
          /EmbeddedFiles \the\pdflastobj\ltx@space 0 R%
        }%
      \endgroup
%    \end{macrocode}
%    Write collection objects.
%    \begin{macrocode}
      \ifx\EmFi@initialfile\ltx@empty
      \else
        \EmFi@collectiontrue
      \fi
      \ifEmFi@collection
        \ifx\EmFi@initialfile\ltx@empty
          \let\EmFi@@initialfile\ltx@empty
        \else
          \edef\EmFi@@initialfile{%
            \pdf@escapestring{\EmFi@initialfile}%
          }%
        \fi
%    \end{macrocode}
%    Look for initial file among the embedded files.
%    \begin{macrocode}
        \begingroup
          \let\f=N%
          \def\do##1##2{%
            \def\x{##1}%
            \ifx\x\EmFi@@initialfile
              \let\f=Y%
              \let\do\ltx@gobbletwo
            \fi
          }%
          \EmFi@list
        \expandafter\endgroup
        \ifx\f Y%
        \else
          \@PackageWarningNoLine{embedfile}{%
            Missing initial file `\EmFi@initialfile'\MessageBreak
            among the embedded files%
          }%
          \let\EmFi@initialfile\ltx@empty
          \let\EmFi@@initialfile\ltx@empty
        \fi
        \ifcase\EmFi@sortcase
          \def\EmFi@temp{}%
        \or
          \def\EmFi@temp{%
            /S\EmFi@sortkeys
            /A \EmFi@sortorders
          }%
        \else
          \def\EmFi@temp{%
            /S[\EmFi@sortkeys]%
            /A[\EmFi@sortorders]%
          }%
        \fi
        \def\EmFi@@order##1{%
          \ifnum\EmFi@order>1 %
            /O ##1%
          \fi
        }%
        \immediate\pdfobj{%
          <<%
            \ifx\EmFi@schema\ltx@empty
            \else
              /Schema<<\EmFi@schema>>%
            \fi
            \ifx\EmFi@@initialfile\ltx@empty
            \else
              /D(\EmFi@@initialfile)%
            \fi
            \ifx\EmFi@view\EmFi@S@tile
              /View/T%
            \else\ifx\EmFi@view\EmFi@S@hidden
              /View/H%
            \fi\fi
            \ifx\EmFi@temp\ltx@empty
              \EmFi@temp
            \else
              /Sort<<\EmFi@temp>>%
            \fi
          >>%
        }%
        \pdfcatalog{%
          /Collection \the\pdflastobj\ltx@space0 R%
        }%
      \fi
    \fi
  \fi
}
%    \end{macrocode}
%    \end{macro}
%    \begin{macrocode}
\begingroup\expandafter\expandafter\expandafter\endgroup
\expandafter\ifx\csname AtEndDocument\endcsname\relax
\else
  \AtEndDocument{\embedfilefinish}%
\fi
%    \end{macrocode}
%
%    \begin{macrocode}
\EmFi@AtEnd%
%</package>
%    \end{macrocode}
%
% \section{Test}
%
% \subsection{Catcode checks for loading}
%
%    \begin{macrocode}
%<*test1>
%    \end{macrocode}
%    \begin{macrocode}
\catcode`\{=1 %
\catcode`\}=2 %
\catcode`\#=6 %
\catcode`\@=11 %
\expandafter\ifx\csname count@\endcsname\relax
  \countdef\count@=255 %
\fi
\expandafter\ifx\csname @gobble\endcsname\relax
  \long\def\@gobble#1{}%
\fi
\expandafter\ifx\csname @firstofone\endcsname\relax
  \long\def\@firstofone#1{#1}%
\fi
\expandafter\ifx\csname loop\endcsname\relax
  \expandafter\@firstofone
\else
  \expandafter\@gobble
\fi
{%
  \def\loop#1\repeat{%
    \def\body{#1}%
    \iterate
  }%
  \def\iterate{%
    \body
      \let\next\iterate
    \else
      \let\next\relax
    \fi
    \next
  }%
  \let\repeat=\fi
}%
\def\RestoreCatcodes{}
\count@=0 %
\loop
  \edef\RestoreCatcodes{%
    \RestoreCatcodes
    \catcode\the\count@=\the\catcode\count@\relax
  }%
\ifnum\count@<255 %
  \advance\count@ 1 %
\repeat

\def\RangeCatcodeInvalid#1#2{%
  \count@=#1\relax
  \loop
    \catcode\count@=15 %
  \ifnum\count@<#2\relax
    \advance\count@ 1 %
  \repeat
}
\def\RangeCatcodeCheck#1#2#3{%
  \count@=#1\relax
  \loop
    \ifnum#3=\catcode\count@
    \else
      \errmessage{%
        Character \the\count@\space
        with wrong catcode \the\catcode\count@\space
        instead of \number#3%
      }%
    \fi
  \ifnum\count@<#2\relax
    \advance\count@ 1 %
  \repeat
}
\def\space{ }
\expandafter\ifx\csname LoadCommand\endcsname\relax
  \def\LoadCommand{\input embedfile.sty\relax}%
\fi
\def\Test{%
  \RangeCatcodeInvalid{0}{47}%
  \RangeCatcodeInvalid{58}{64}%
  \RangeCatcodeInvalid{91}{96}%
  \RangeCatcodeInvalid{123}{255}%
  \catcode`\@=12 %
  \catcode`\\=0 %
  \catcode`\%=14 %
  \LoadCommand
  \RangeCatcodeCheck{0}{36}{15}%
  \RangeCatcodeCheck{37}{37}{14}%
  \RangeCatcodeCheck{38}{47}{15}%
  \RangeCatcodeCheck{48}{57}{12}%
  \RangeCatcodeCheck{58}{63}{15}%
  \RangeCatcodeCheck{64}{64}{12}%
  \RangeCatcodeCheck{65}{90}{11}%
  \RangeCatcodeCheck{91}{91}{15}%
  \RangeCatcodeCheck{92}{92}{0}%
  \RangeCatcodeCheck{93}{96}{15}%
  \RangeCatcodeCheck{97}{122}{11}%
  \RangeCatcodeCheck{123}{255}{15}%
  \RestoreCatcodes
}
\Test
\csname @@end\endcsname
\end
%    \end{macrocode}
%    \begin{macrocode}
%</test1>
%    \end{macrocode}
%
% \subsection{Simple test}
%
%    \begin{macrocode}
%<*test2>
\input embedfile.sty\relax
\embedfile[%
  stringmethod=escape,%
  mimetype=plain/text,%
  desc={LaTeX docstrip source archive for package `embedfile'},%
  id={embedfile.dtx}%
]{embedfile.dtx}
\nopagenumbers
Test (plain-\TeX): {\tt embedfile.dtx} should be embedded.%

\def\Test#1{%
  \par
  \embedfileifobjectexists{embedfile.dtx}{#1}{%
    Object #1 (embedfile.dtx): %
    \embedfilegetobject{embedfile.dtx}{#1}%
  }{%
    \errmessage{Missing object #1 (embedfile.dtx)}%
  }%
}
\Test{EmbeddedFile}
\Test{Filespec}
\embedfilefinish
\bye
%</test2>
%    \end{macrocode}
%    \begin{macrocode}
%<*test3>
\NeedsTeXFormat{LaTeX2e}
\let\SavedJobname\jobname
\def\jobname{embedfile}
\RequirePackage{dtx-attach}[2011/04/13]
\let\jobname\SavedJobname
\documentclass{minimal}
\begin{document}
  Test (\LaTeX): \texttt{embedfile.dtx} should be embedded.%
\end{document}
%</test3>
%    \end{macrocode}
%
% \subsection{Test for \iniTeX}
%
%    \begin{macrocode}
%<*test4>
\catcode`\{=1 %
\catcode`\}=2 %
\input ifluatex.sty %
\ifluatex
  \directlua{%
    tex.enableprimitives('', {%
      'pdflastobj',%
      'pdfnames',%
      'pdfobj',%
      'pdfoutput'%
    })%
  }%
\fi
\pdfoutput=1 %
\input embedfile.sty %
\shipout\hbox{}
\embedfile[%
  stringmethod=escape,%
  mimetype=plain/text,%
  desc={iniTeX source},%
]{\jobname.tex}
\embedfilefinish
\end
%</test4>
%    \end{macrocode}
%
% \section{Installation}
%
% \subsection{Download}
%
% \paragraph{Package.} This package is available on
% CTAN\footnote{\url{ftp://ftp.ctan.org/tex-archive/}}:
% \begin{description}
% \item[\CTAN{macros/latex/contrib/oberdiek/embedfile.dtx}] The source file.
% \item[\CTAN{macros/latex/contrib/oberdiek/embedfile.pdf}] Documentation.
% \end{description}
%
%
% \paragraph{Bundle.} All the packages of the bundle `oberdiek'
% are also available in a TDS compliant ZIP archive. There
% the packages are already unpacked and the documentation files
% are generated. The files and directories obey the TDS standard.
% \begin{description}
% \item[\CTAN{install/macros/latex/contrib/oberdiek.tds.zip}]
% \end{description}
% \emph{TDS} refers to the standard ``A Directory Structure
% for \TeX\ Files'' (\CTAN{tds/tds.pdf}). Directories
% with \xfile{texmf} in their name are usually organized this way.
%
% \subsection{Bundle installation}
%
% \paragraph{Unpacking.} Unpack the \xfile{oberdiek.tds.zip} in the
% TDS tree (also known as \xfile{texmf} tree) of your choice.
% Example (linux):
% \begin{quote}
%   |unzip oberdiek.tds.zip -d ~/texmf|
% \end{quote}
%
% \paragraph{Script installation.}
% Check the directory \xfile{TDS:scripts/oberdiek/} for
% scripts that need further installation steps.
% Package \xpackage{attachfile2} comes with the Perl script
% \xfile{pdfatfi.pl} that should be installed in such a way
% that it can be called as \texttt{pdfatfi}.
% Example (linux):
% \begin{quote}
%   |chmod +x scripts/oberdiek/pdfatfi.pl|\\
%   |cp scripts/oberdiek/pdfatfi.pl /usr/local/bin/|
% \end{quote}
%
% \subsection{Package installation}
%
% \paragraph{Unpacking.} The \xfile{.dtx} file is a self-extracting
% \docstrip\ archive. The files are extracted by running the
% \xfile{.dtx} through \plainTeX:
% \begin{quote}
%   \verb|tex embedfile.dtx|
% \end{quote}
%
% \paragraph{TDS.} Now the different files must be moved into
% the different directories in your installation TDS tree
% (also known as \xfile{texmf} tree):
% \begin{quote}
% \def\t{^^A
% \begin{tabular}{@{}>{\ttfamily}l@{ $\rightarrow$ }>{\ttfamily}l@{}}
%   embedfile.sty & tex/generic/oberdiek/embedfile.sty\\
%   dtx-attach.sty & tex/generic/oberdiek/dtx-attach.sty\\
%   embedfile.pdf & doc/latex/oberdiek/embedfile.pdf\\
%   embedfile-example-plain.tex & doc/latex/oberdiek/embedfile-example-plain.tex\\
%   embedfile-example-collection.tex & doc/latex/oberdiek/embedfile-example-collection.tex\\
%   test/embedfile-test1.tex & doc/latex/oberdiek/test/embedfile-test1.tex\\
%   test/embedfile-test2.tex & doc/latex/oberdiek/test/embedfile-test2.tex\\
%   test/embedfile-test3.tex & doc/latex/oberdiek/test/embedfile-test3.tex\\
%   test/embedfile-test4.tex & doc/latex/oberdiek/test/embedfile-test4.tex\\
%   embedfile.dtx & source/latex/oberdiek/embedfile.dtx\\
% \end{tabular}^^A
% }^^A
% \sbox0{\t}^^A
% \ifdim\wd0>\linewidth
%   \begingroup
%     \advance\linewidth by\leftmargin
%     \advance\linewidth by\rightmargin
%   \edef\x{\endgroup
%     \def\noexpand\lw{\the\linewidth}^^A
%   }\x
%   \def\lwbox{^^A
%     \leavevmode
%     \hbox to \linewidth{^^A
%       \kern-\leftmargin\relax
%       \hss
%       \usebox0
%       \hss
%       \kern-\rightmargin\relax
%     }^^A
%   }^^A
%   \ifdim\wd0>\lw
%     \sbox0{\small\t}^^A
%     \ifdim\wd0>\linewidth
%       \ifdim\wd0>\lw
%         \sbox0{\footnotesize\t}^^A
%         \ifdim\wd0>\linewidth
%           \ifdim\wd0>\lw
%             \sbox0{\scriptsize\t}^^A
%             \ifdim\wd0>\linewidth
%               \ifdim\wd0>\lw
%                 \sbox0{\tiny\t}^^A
%                 \ifdim\wd0>\linewidth
%                   \lwbox
%                 \else
%                   \usebox0
%                 \fi
%               \else
%                 \lwbox
%               \fi
%             \else
%               \usebox0
%             \fi
%           \else
%             \lwbox
%           \fi
%         \else
%           \usebox0
%         \fi
%       \else
%         \lwbox
%       \fi
%     \else
%       \usebox0
%     \fi
%   \else
%     \lwbox
%   \fi
% \else
%   \usebox0
% \fi
% \end{quote}
% If you have a \xfile{docstrip.cfg} that configures and enables \docstrip's
% TDS installing feature, then some files can already be in the right
% place, see the documentation of \docstrip.
%
% \subsection{Refresh file name databases}
%
% If your \TeX~distribution
% (\teTeX, \mikTeX, \dots) relies on file name databases, you must refresh
% these. For example, \teTeX\ users run \verb|texhash| or
% \verb|mktexlsr|.
%
% \subsection{Some details for the interested}
%
% \paragraph{Attached source.}
%
% The PDF documentation on CTAN also includes the
% \xfile{.dtx} source file. It can be extracted by
% AcrobatReader 6 or higher. Another option is \textsf{pdftk},
% e.g. unpack the file into the current directory:
% \begin{quote}
%   \verb|pdftk embedfile.pdf unpack_files output .|
% \end{quote}
%
% \paragraph{Unpacking with \LaTeX.}
% The \xfile{.dtx} chooses its action depending on the format:
% \begin{description}
% \item[\plainTeX:] Run \docstrip\ and extract the files.
% \item[\LaTeX:] Generate the documentation.
% \end{description}
% If you insist on using \LaTeX\ for \docstrip\ (really,
% \docstrip\ does not need \LaTeX), then inform the autodetect routine
% about your intention:
% \begin{quote}
%   \verb|latex \let\install=y% \iffalse meta-comment
%
% File: embedfile.dtx
% Version: 2011/04/13 v2.6
% Info: Embed files into PDF
%
% Copyright (C) 2006-2011 by
%    Heiko Oberdiek <heiko.oberdiek at googlemail.com>
%
% This work may be distributed and/or modified under the
% conditions of the LaTeX Project Public License, either
% version 1.3c of this license or (at your option) any later
% version. This version of this license is in
%    http://www.latex-project.org/lppl/lppl-1-3c.txt
% and the latest version of this license is in
%    http://www.latex-project.org/lppl.txt
% and version 1.3 or later is part of all distributions of
% LaTeX version 2005/12/01 or later.
%
% This work has the LPPL maintenance status "maintained".
%
% This Current Maintainer of this work is Heiko Oberdiek.
%
% The Base Interpreter refers to any `TeX-Format',
% because some files are installed in TDS:tex/generic//.
%
% This work consists of the main source file embedfile.dtx
% and the derived files
%    embedfile.sty, embedfile.pdf, embedfile.ins, embedfile.drv,
%    dtx-attach.sty, embedfile-example-plain.tex,
%    embedfile-example-collection.tex, embedfile-test1.tex,
%    embedfile-test2.tex, embedfile-test3.tex,
%    embedfile-test4.tex.
%
% Distribution:
%    CTAN:macros/latex/contrib/oberdiek/embedfile.dtx
%    CTAN:macros/latex/contrib/oberdiek/embedfile.pdf
%
% Unpacking:
%    (a) If embedfile.ins is present:
%           tex embedfile.ins
%    (b) Without embedfile.ins:
%           tex embedfile.dtx
%    (c) If you insist on using LaTeX
%           latex \let\install=y\input{embedfile.dtx}
%        (quote the arguments according to the demands of your shell)
%
% Documentation:
%    (a) If embedfile.drv is present:
%           latex embedfile.drv
%    (b) Without embedfile.drv:
%           latex embedfile.dtx; ...
%    The class ltxdoc loads the configuration file ltxdoc.cfg
%    if available. Here you can specify further options, e.g.
%    use A4 as paper format:
%       \PassOptionsToClass{a4paper}{article}
%
%    Programm calls to get the documentation (example):
%       pdflatex embedfile.dtx
%       makeindex -s gind.ist embedfile.idx
%       pdflatex embedfile.dtx
%       makeindex -s gind.ist embedfile.idx
%       pdflatex embedfile.dtx
%
% Installation:
%    TDS:tex/generic/oberdiek/embedfile.sty
%    TDS:tex/generic/oberdiek/dtx-attach.sty
%    TDS:doc/latex/oberdiek/embedfile.pdf
%    TDS:doc/latex/oberdiek/embedfile-example-plain.tex
%    TDS:doc/latex/oberdiek/embedfile-example-collection.tex
%    TDS:doc/latex/oberdiek/test/embedfile-test1.tex
%    TDS:doc/latex/oberdiek/test/embedfile-test2.tex
%    TDS:doc/latex/oberdiek/test/embedfile-test3.tex
%    TDS:doc/latex/oberdiek/test/embedfile-test4.tex
%    TDS:source/latex/oberdiek/embedfile.dtx
%
%<*ignore>
\begingroup
  \catcode123=1 %
  \catcode125=2 %
  \def\x{LaTeX2e}%
\expandafter\endgroup
\ifcase 0\ifx\install y1\fi\expandafter
         \ifx\csname processbatchFile\endcsname\relax\else1\fi
         \ifx\fmtname\x\else 1\fi\relax
\else\csname fi\endcsname
%</ignore>
%<*install>
\input docstrip.tex
\Msg{************************************************************************}
\Msg{* Installation}
\Msg{* Package: embedfile 2011/04/13 v2.6 Embed files into PDF (HO)}
\Msg{************************************************************************}

\keepsilent
\askforoverwritefalse

\let\MetaPrefix\relax
\preamble

This is a generated file.

Project: embedfile
Version: 2011/04/13 v2.6

Copyright (C) 2006-2011 by
   Heiko Oberdiek <heiko.oberdiek at googlemail.com>

This work may be distributed and/or modified under the
conditions of the LaTeX Project Public License, either
version 1.3c of this license or (at your option) any later
version. This version of this license is in
   http://www.latex-project.org/lppl/lppl-1-3c.txt
and the latest version of this license is in
   http://www.latex-project.org/lppl.txt
and version 1.3 or later is part of all distributions of
LaTeX version 2005/12/01 or later.

This work has the LPPL maintenance status "maintained".

This Current Maintainer of this work is Heiko Oberdiek.

The Base Interpreter refers to any `TeX-Format',
because some files are installed in TDS:tex/generic//.

This work consists of the main source file embedfile.dtx
and the derived files
   embedfile.sty, embedfile.pdf, embedfile.ins, embedfile.drv,
   dtx-attach.sty, embedfile-example-plain.tex,
   embedfile-example-collection.tex, embedfile-test1.tex,
   embedfile-test2.tex, embedfile-test3.tex,
   embedfile-test4.tex.

\endpreamble
\let\MetaPrefix\DoubleperCent

\generate{%
  \file{embedfile.ins}{\from{embedfile.dtx}{install}}%
  \file{embedfile.drv}{\from{embedfile.dtx}{driver}}%
  \usedir{tex/generic/oberdiek}%
  \file{embedfile.sty}{\from{embedfile.dtx}{package}}%
  \usedir{tex/latex/oberdiek}%
  \file{dtx-attach.sty}{\from{embedfile.dtx}{dtxattach}}%
  \usedir{doc/latex/oberdiek}%
  \file{embedfile-example-plain.tex}{\from{embedfile.dtx}{exampleplain}}%
  \file{embedfile-example-collection.tex}{\from{embedfile.dtx}{examplecollection}}%
  \usedir{doc/latex/oberdiek/test}%
  \file{embedfile-test1.tex}{\from{embedfile.dtx}{test1}}%
  \file{embedfile-test2.tex}{\from{embedfile.dtx}{test2}}%
  \file{embedfile-test3.tex}{\from{embedfile.dtx}{test3}}%
  \file{embedfile-test4.tex}{\from{embedfile.dtx}{test4}}%
  \nopreamble
  \nopostamble
  \usedir{source/latex/oberdiek/catalogue}%
  \file{embedfile.xml}{\from{embedfile.dtx}{catalogue}}%
}

\catcode32=13\relax% active space
\let =\space%
\Msg{************************************************************************}
\Msg{*}
\Msg{* To finish the installation you have to move the following}
\Msg{* files into a directory searched by TeX:}
\Msg{*}
\Msg{*     embedfile.sty, dtx-attach.sty}
\Msg{*}
\Msg{* To produce the documentation run the file `embedfile.drv'}
\Msg{* through LaTeX.}
\Msg{*}
\Msg{* Happy TeXing!}
\Msg{*}
\Msg{************************************************************************}

\endbatchfile
%</install>
%<*ignore>
\fi
%</ignore>
%<*driver>
\NeedsTeXFormat{LaTeX2e}
\ProvidesFile{embedfile.drv}%
  [2011/04/13 v2.6 Embed files into PDF (HO)]%
\documentclass{ltxdoc}
\usepackage{holtxdoc}[2011/11/22]
\begin{document}
  \DocInput{embedfile.dtx}%
\end{document}
%</driver>
% \fi
%
% \CheckSum{1363}
%
% \CharacterTable
%  {Upper-case    \A\B\C\D\E\F\G\H\I\J\K\L\M\N\O\P\Q\R\S\T\U\V\W\X\Y\Z
%   Lower-case    \a\b\c\d\e\f\g\h\i\j\k\l\m\n\o\p\q\r\s\t\u\v\w\x\y\z
%   Digits        \0\1\2\3\4\5\6\7\8\9
%   Exclamation   \!     Double quote  \"     Hash (number) \#
%   Dollar        \$     Percent       \%     Ampersand     \&
%   Acute accent  \'     Left paren    \(     Right paren   \)
%   Asterisk      \*     Plus          \+     Comma         \,
%   Minus         \-     Point         \.     Solidus       \/
%   Colon         \:     Semicolon     \;     Less than     \<
%   Equals        \=     Greater than  \>     Question mark \?
%   Commercial at \@     Left bracket  \[     Backslash     \\
%   Right bracket \]     Circumflex    \^     Underscore    \_
%   Grave accent  \`     Left brace    \{     Vertical bar  \|
%   Right brace   \}     Tilde         \~}
%
% \GetFileInfo{embedfile.drv}
%
% \title{The \xpackage{embedfile} package}
% \date{2011/04/13 v2.6}
% \author{Heiko Oberdiek\\\xemail{heiko.oberdiek at googlemail.com}}
%
% \maketitle
%
% \begin{abstract}
% This package embeds files to a PDF document.
% Currently the only supported driver is \pdfTeX\ $>=$ 1.30 in PDF mode.
% \end{abstract}
%
% \tableofcontents
%
% \section{Documentation}
%
% \subsection{Introduction}
%
%    The PDF format (\cite{pdfspec}) allows the inclusion of files
%    inside the PDF document. The included files can be bound to an
%    annotation on a page. Or they can be recorded in a sorted
%    list of embedded files. The packages \xpackage{attachfile}
%    or \xpackage{attachfile2} follow the first approach, this
%    package uses the latter method.
%
% \subsubsection{Future development}
%
%    My dream is a large package that merges the features of
%    all these packages meantioned before:
%    \begin{itemize}
%    \item Files can be attached to a page.
%    \item Files can be attached to the document.
%    \item An easy user interface for simple, common tasks and
%          beginners.
%    \item An interface for the advanced users that want to setup
%          every detail.
%    \item Support of many drivers (pdftex, dvips, dvipdfm, \dots).
%    \item \dots
%    \end{itemize}
%    However, I have not managed to take the time for this project.
%    Instead:
%    \begin{itemize}
%    \item First I experimented with package \xpackage{attachfile},
%          adding driver support, fixing bugs, \dots. The result is
%          currently named as \xpackage{attachfile2}. It uses an external
%          script to get file properties (size, date, checksum, \dots).
%    \item In order to avoid an external program for getting basic
%          file properties I provided a patch ``EscapeAndOther'' for
%          pdfTeX that was accepted for version 1.30.
%    \item This package closes a gap left by the packages for attaching
%          files and allows the embedding of files to the document.
%          Also it makes use of the new primitives of \pdfTeX.
%    \end{itemize}
%
% \subsection{User interface}
%
%    This package \xpackage{embedfile} can be used with both
%    \LaTeX\ and \plainTeX. See \autoref{sec:plain} that
%    explains the use with \plainTeX\ by an example.
%    In \LaTeX\ the package is loaded as usually. There are no options.
%    \begin{quote}
%      \verb|\usepackage{embedfile}|
%    \end{quote}
%
%    \begin{declcs}{embedfile}
%      |[|\meta{options}|]| \M{file}
%    \end{declcs}
%    The macro \cs{embedfile} includes file \meta{file} and
%    attaches it to the PDF document. At the end of the document
%    the sorted list of embedded files are written. Thus you can
%    safely use \cs{embedfile} before \verb|\end{document}|.
%    Embedding files using \cs{AtEndDocument} will only work,
%    if \cs{AtEndDocument} is called before loading the package
%    \xpackage{embedfile}.
%
%    The \meta{options} are give as key value pairs. The following
%    keys are supported:
%    \begin{description}
%    \item[\xoption{filespec}]
%      This allows to override the file name
%      that appears in the PDF file. If you are using other than
%      simple file names (8-bit, path separators, \dots), look into
%      the PDF specification (\cite{pdfspec}).
%      There are rules how these file names must be written/encoded.
%      Avoid 8-bit characters
%      and other special characters, the behaviour is currently undefined.
%      Use option \xoption{ucfilespec} for more funny file names.
%      The string method, see below, is |escape| since version 2.4.
%
%      This name is also used as entry in a name tree
%      (see PDF specification: |/EmbeddedFiles|). Therefore the
%      value for \xoption{filespec} must be unique among all
%      embedded files. Also key \xoption{initialfiles} refers
%      to this name, if the file name and the value of \xoption{filespec}
%      are different.
%    \item[\xoption{ucfilespec}]
%      Since PDF 1.7 the file name may be provided in Unicode.
%      The conversion of the option value into a PDF string
%      is controlled by option \xoption{stringmethod}.
%    \item[\xoption{filesystem}]
%      This sets the entry \verb|/FS| in
%      the file specification dictionary, see PDF specification
%      (\cite{pdfspec}).
%      Example: \verb|filesystem=URL|.
%    \item[\xoption{mimetype}]
%      This sets the mime type (\cite{mime}) of the file,
%      see \autoref{sec:plain} for examples and \cite{types}
%      for a list of officially registered types.
%    \item[\xoption{desc}]
%      The description for the file.
%    \item[\xoption{stringmethod}]
%      The package must convert the values of the keys \xoption{ucfilespec}
%      and \xoption{desc} into a PDF string (before version 2.4: \xoption{filespec}
%      and \xoption{desc}). If \xpackage{hyperref}
%      is found, then its \cs{pdfstringdef} will be used, otherwise
%      \pdfTeX's \cs{pdfescapestring} is used. Value \verb|psd|
%      forces the use of \cs{pdfstringdef}, value \verb|escape|
%      the use of \cs{pdfescapestring}.
%    \item[\xoption{\meta{key}.value}] Sets the value of a collection
%      item property, see section \ref{sec:collection}.
%    \item[\xoption{\meta{key}.prefix}]
%      Sets the prefix of a collection item property,
%      see section \ref{sec:collection}.
%    \item[\xoption{id}]
%      The value must be an unique name. Macros
%      \cs{embedfileifobjectexists} and \cs{embedfilegetobject}
%      are using this name later.
%    \end{description}
%
%    \begin{declcs}{embedfilefinish}
%    \end{declcs}
%    The list of all embedded files must be added as data structure
%    in the PDF file. In case of \LaTeX\ this is automatically done.
%    The package uses \cs{AtEndDocument}. Then the list of all
%    files should be known. However, \plainTeX\ does not know
%    about \cs{AtEndDocument}. Thus the user must call
%    \cs{embedfilefinish} at the end of the document
%    after the last file is embedded.
%
%    \begin{declcs}{embedfilesetup} \M{options}
%    \end{declcs}
%    Options for \cs{embedfile} and collection support
%    can be set in \cs{embedfilesetup}.
%
% \subsection{Collection support (PDF 1.7)}
% \label{sec:collection}
%
%    Since PDF 1.7 the embedded files can form a \emph{collection}
%    (sometimes referred as \emph{package}), the main document
%    is called \emph{cover sheet}. See PDF specification
%    8.2.4 ``Collections'' and 3.10.5 ``Collection items'' \cite{pdfspec}.
%
%    Usually Acrobat Reader 7 or 8 shows the
%    embedded files in a table at the bottom with the following columns:
%    \begin{quote}
%    \begin{tabular}{@{}l|l|l|l@{}}
%       Name & Description & Modified & Size\\
%       \hline
%       \dots & \dots & \dots & \dots\\
%    \end{tabular}
%    \end{quote}
%    Acrobat Reader 10 shows the embeded files in the left panel
%    and adds a new column for the compressed size.
%
%    If the files form a collection, then they are displayed
%    in a table left or top (depending on option \xoption{view},
%    see \cs{embedfilesetup}).
%
%
%    Collection support is enabled automatically, if
%    it is used.
%
%    \begin{declcs}{embedfilesetup} \M{options}
%    \end{declcs}
%    The following options are supported in addition
%    to options for \cs{embedfile}:
%    \begin{description}
%    \item[\xoption{view}]
%      If the PDF file contains a collection, then Acrobat Reader 8
%      shows a line at the top below the menu bar and the toolbar.
%      It shows the current selected file, icons for changing the
%      view mode, an options menu.
%      The initial mode how the collection is presented is
%      set by this option \xoption{view}.
%      The following modes/values are supported, the default is |details|:
%        \begin{description}
%        \item[\texttt{details}] The full collection table is displayed
%          at the top below the collection bar.
%        \item[\texttt{tile}] The files of the collection are shown
%          in tile mode on the left.
%        \item[\texttt{hidden}] The collection table is not shown.
%        \end{description}
%    \item[\xoption{initialfile}]
%      Selects the file that is initially presented. Especially
%      useful for an embedded PDF file that is then shown instead
%      of the cover document. There must be an \cs{embedfile} command
%      somewhere whose value for key \xoption{filespec} is used here.
%      The \cs{embedfile} command can drop option \xoption{filespec}
%      if the file name is not different.
%    \end{description}
%
%    \begin{declcs}{embedfilefield} \M{key} \M{options}
%    \end{declcs}
%    Macro \cs{embedfilefield} defines a column/field in the collection table.
%    The name of the field is \meta{key}.
%    \begin{description}
%    \item[\xoption{type}]
%      sets the type of the field. The supported values are:
%      \begin{description}
%      \item[\texttt{text}] A text field. Its value is set in
%        \cs{embedfile} by option \xoption{\meta{key}.value}.
%      \item[\texttt{date}] A date field. Its value is set in
%        \cs{embedfile} by option \xoption{\meta{key}.value}.
%        A special format is required, see ``3.8.3 Dates'' \cite{pdfspec}.
%      \item[\texttt{number}] A field with an integer or float number.
%        Its value is set in \cs{embedfile} by option
%        \xoption{\meta{key}.value}.
%      \item[\texttt{file}] The file name of the embedded file.
%      \item[\texttt{desc}] The description text of the embedded file. It is
%       set in \cs{embedfile} by option \xoption{desc}.
%      \item[\texttt{moddate}] The modification date of the embedded file.
%      \item[\texttt{size}] The size of the embedded file.
%      \end{description}
%      All types allow the use of a prefix that is disregarded by
%      sorting. The prefix for this field is set in \cs{embedfile} by
%      option \xoption{\meta{key}.prefix}.
%    \item[\xoption{title}] sets the column title.
%    \item[\xoption{visible}] controls whether the column is presented:
%      \begin{description}
%      \item[\texttt{true}] shows the column.
%      \item[\texttt{false}] hides the column.
%      \end{description}
%      Default: \texttt{true}
%    \item[\xoption{edit}]
%      Allows the editing of field values. Does not seem to have an
%      effect for Acrobat Reader.
%      \begin{description}
%      \item[\texttt{true}] enables the feature, if available (depends
%        on the PDF viewer).
%      \item[\texttt{false}] disables the feature.
%      \end{description}
%      Default: \texttt{false}
%    \end{description}
%    The order of \cs{embedfilefield} statements defines the order
%    of the columns.
%
%    \begin{declcs}{embedfilesort} \M{key-sort-list}
%    \end{declcs}
%    The sort order of the embedded files are controlled by
%    macro \cs{embedfilesort}. \meta{key-sort-list} defines
%    the sort order. The key is a field name defined by \cs{embedfilefield}.
%    Its value is either \texttt{ascending} or \texttt{descending}.
%    The default is \texttt{ascending}.
%
% \subsection{Export of object references}
%
%    Caution: This feature is still experimental. It may be even removed
%    in future versions. Therefore feedback would be nice, if someone
%    has a useful application for this feature.
%
%    Object numbers are saved, if \xoption{id} is given in \cs{embedfile}.
%    The following objects are supported:
%    \begin{itemize}
%    \item \texttt{EmbeddedFile}
%    \item \texttt{Filespec}
%    \end{itemize}
%
%    \begin{declcs}{embedfileifobjectexists} \M{id} \M{type}
%      \M{then} \M{else}
%    \end{declcs}
%    Macro \cs{embedfileifobjectexists} tests whether object of \meta{type}
%    is available for the embedded file identified by \meta{id}.
%
%    \begin{declcs}{embedfilegetobject} \M{id} \M{type}
%    \end{declcs}
%    Macro \cs{embedfilegetobject} expands to the full object reference
%    object of \meta{type} for the embedded file identified by \meta{id}.
%
% \subsubsection{Example}
%
%\begin{quote}
%\begin{verbatim}
%\embedfile[id={foo}]{foo.pdf}
%\embedfileifobjectexists{foo}{Filespec}{%
%  \typeout{%
%    FileSpec object for `foo': %
%    \embedfilegetobject{foo}{Filespec}%
%  }%
%}{%
%  \typeout{No Filespec object for `foo'}%
%}
%\end{verbatim}
%\end{quote}
%
% \subsection{Examples}
% \label{sec:examples}
%
% \subsubsection{\plainTeX}
% \label{sec:plain}
%
%    The package can be used with \plainTeX. It can be used
%    with or without help from \xfile{miniltx.tex}.
%
%    If additionally package \xpackage{keyval} (\xpackage{graphicx})
%    is needed, load it first. Then package \xpackage{embedfile} avoids
%    a duplicate loading of package \xpackage{keyval}.
%
%    Because \plainTeX\ does not provide a hook at end of the document,
%    you have to call \cs{embedfilefinish} manually at the end
%    after the last embedded file.
%    \begin{macrocode}
%<*exampleplain>
%<<END
% Load packages
\input miniltx
% \def\Gin@driver{pdftex.def}
% \input graphicx.sty
\input embedfile.sty
\resetatcatcode

% default setting
\embedfilesetup{
  mimetype=text/plain
}

% Embed files
\embedfile[
  filespec=example.tex,
  desc={Source code (plain-TeX) of this example}
]{embedfile-example-plain.tex}

\embedfile[
  desc={Source of package `embedfile'}
]{embedfile.dtx}

\embedfile[
  mimetype=application/pdf,
  desc={Documentation of package `embedfile'}
]{embedfile.pdf}

% Some text
This example document contains three embedded files.

% End of document
\embedfilefinish % don't forget
\bye
%END
%</exampleplain>
%    \end{macrocode}
%
% \subsubsection{Collection example}
%
%    \begin{macrocode}
%<*examplecollection>
%<<END
\NeedsTeXFormat{LaTeX2e}
\documentclass{article}
\usepackage[bookmarks=false]{hyperref}
% provides \pdfstringdef that is then used by `title' and
% other keys.
\usepackage{embedfile}[2011/04/13]
\embedfilesetup{
  view=details,
  initialfile=embedfile.pdf
}
\embedfilefield{file}{
  type=file,
  title={File name}
}
\embedfilefield{description}{
  type=desc,
  title={Description}
}
\embedfilefield{date}{
  type=moddate,
  title={Date}
}
\embedfilefield{size}{
  type=size,
  title={Size}
}
\embedfilefield{type}{
  type=text,
  title={Type},
  visible=false
}
\embedfilesort{
  type,
  date=descending
}
\begin{document}
An example for embedded files as collection.
You need Acrobat Reader 8 or higher.

\embedfile[
  desc={Source file of package `embedfile'},
  description.prefix={Package: },
  type.value={DTX}
]{embedfile.dtx}

\embedfile[
  desc={Documentation of package `embedfile'},
  description.prefix={Package: },
  type.value={PDF}
]{embedfile.pdf}

\embedfile[
  desc={The source for this example},
  description.prefix={Example: },
  type.value={TEX}
]{\jobname.tex}

\end{document}
%END
%</examplecollection>
%    \end{macrocode}
%
% \subsection{Package \xpackage{dtx-attach}}
%
% Package \xpackage{dtx-attach} is just a small application of
% package \xpackage{embedfile}. I am using it for the CTAN
% documentation of my packages in \CTAN{macros/latex/contrib/oberdiek/}.
% It also serves as small example for the use of the package with \LaTeX.
%
%    \begin{macrocode}
%<*dtxattach>
\NeedsTeXFormat{LaTeX2e}
\ProvidesPackage{dtx-attach}
  [2011/04/13 v2.6 Embed \string\jobname.dtx (HO)]%
\RequirePackage{embedfile}[2011/04/13]
\embedfile[%
  stringmethod=escape,%
  mimetype=plain/text,%
  desc={LaTeX docstrip source archive for package `\jobname'}%
]{\jobname.dtx}
%</dtxattach>
%    \end{macrocode}
%
% \StopEventually{
% }
%
% \section{Implementation}
%
%    \begin{macrocode}
%<*package>
%    \end{macrocode}
%
% \subsection{Reload check and package identification}
%    Reload check, especially if the package is not used with \LaTeX.
%    \begin{macrocode}
\begingroup\catcode61\catcode48\catcode32=10\relax%
  \catcode13=5 % ^^M
  \endlinechar=13 %
  \catcode35=6 % #
  \catcode39=12 % '
  \catcode44=12 % ,
  \catcode45=12 % -
  \catcode46=12 % .
  \catcode58=12 % :
  \catcode64=11 % @
  \catcode123=1 % {
  \catcode125=2 % }
  \expandafter\let\expandafter\x\csname ver@embedfile.sty\endcsname
  \ifx\x\relax % plain-TeX, first loading
  \else
    \def\empty{}%
    \ifx\x\empty % LaTeX, first loading,
      % variable is initialized, but \ProvidesPackage not yet seen
    \else
      \expandafter\ifx\csname PackageInfo\endcsname\relax
        \def\x#1#2{%
          \immediate\write-1{Package #1 Info: #2.}%
        }%
      \else
        \def\x#1#2{\PackageInfo{#1}{#2, stopped}}%
      \fi
      \x{embedfile}{The package is already loaded}%
      \aftergroup\endinput
    \fi
  \fi
\endgroup%
%    \end{macrocode}
%    Package identification:
%    \begin{macrocode}
\begingroup\catcode61\catcode48\catcode32=10\relax%
  \catcode13=5 % ^^M
  \endlinechar=13 %
  \catcode35=6 % #
  \catcode39=12 % '
  \catcode40=12 % (
  \catcode41=12 % )
  \catcode44=12 % ,
  \catcode45=12 % -
  \catcode46=12 % .
  \catcode47=12 % /
  \catcode58=12 % :
  \catcode64=11 % @
  \catcode91=12 % [
  \catcode93=12 % ]
  \catcode123=1 % {
  \catcode125=2 % }
  \expandafter\ifx\csname ProvidesPackage\endcsname\relax
    \def\x#1#2#3[#4]{\endgroup
      \immediate\write-1{Package: #3 #4}%
      \xdef#1{#4}%
    }%
  \else
    \def\x#1#2[#3]{\endgroup
      #2[{#3}]%
      \ifx#1\@undefined
        \xdef#1{#3}%
      \fi
      \ifx#1\relax
        \xdef#1{#3}%
      \fi
    }%
  \fi
\expandafter\x\csname ver@embedfile.sty\endcsname
\ProvidesPackage{embedfile}%
  [2011/04/13 v2.6 Embed files into PDF (HO)]%
%    \end{macrocode}
%
% \subsection{Catcodes}
%
%    \begin{macrocode}
\begingroup\catcode61\catcode48\catcode32=10\relax%
  \catcode13=5 % ^^M
  \endlinechar=13 %
  \catcode123=1 % {
  \catcode125=2 % }
  \catcode64=11 % @
  \def\x{\endgroup
    \expandafter\edef\csname EmFi@AtEnd\endcsname{%
      \endlinechar=\the\endlinechar\relax
      \catcode13=\the\catcode13\relax
      \catcode32=\the\catcode32\relax
      \catcode35=\the\catcode35\relax
      \catcode61=\the\catcode61\relax
      \catcode64=\the\catcode64\relax
      \catcode123=\the\catcode123\relax
      \catcode125=\the\catcode125\relax
    }%
  }%
\x\catcode61\catcode48\catcode32=10\relax%
\catcode13=5 % ^^M
\endlinechar=13 %
\catcode35=6 % #
\catcode64=11 % @
\catcode123=1 % {
\catcode125=2 % }
\def\TMP@EnsureCode#1#2{%
  \edef\EmFi@AtEnd{%
    \EmFi@AtEnd
    \catcode#1=\the\catcode#1\relax
  }%
  \catcode#1=#2\relax
}
\TMP@EnsureCode{39}{12}% '
\TMP@EnsureCode{40}{12}% (
\TMP@EnsureCode{41}{12}% )
\TMP@EnsureCode{44}{12}% ,
\TMP@EnsureCode{46}{12}% .
\TMP@EnsureCode{47}{12}% /
\TMP@EnsureCode{58}{12}% :
\TMP@EnsureCode{60}{12}% <
\TMP@EnsureCode{62}{12}% >
\TMP@EnsureCode{91}{12}% [
\TMP@EnsureCode{93}{12}% ]
\TMP@EnsureCode{96}{12}% `
\edef\EmFi@AtEnd{\EmFi@AtEnd\noexpand\endinput}
%    \end{macrocode}
%
% \subsection{Tools}
%
%    \begin{macro}{\EmFi@RequirePackage}
%    \begin{macrocode}
\begingroup\expandafter\expandafter\expandafter\endgroup
\expandafter\ifx\csname RequirePackage\endcsname\relax
  \def\EmFi@RequirePackage#1[#2]{%
    \input #1.sty\relax
  }%
\else
  \let\EmFi@RequirePackage\RequirePackage
\fi
%    \end{macrocode}
%    \end{macro}
%
%    \begin{macro}{\EmFi@Error}
%    \begin{macrocode}
\EmFi@RequirePackage{infwarerr}[2007/09/09]%
\def\EmFi@Error{%
  \@PackageError{embedfile}%
}
%    \end{macrocode}
%    \end{macro}
%
% \subsection{Check for recent \pdfTeX\ in PDF mode}
%
%    Load package \xpackage{ifpdf} and check mode.
%    \begin{macrocode}
\EmFi@RequirePackage{ifpdf}[2007/09/09]
\ifpdf
\else
  \EmFi@Error{%
    Missing pdfTeX in PDF mode%
  }{%
    Currently other drivers are not supported. %
    Package loading is aborted.%
  }%
  \expandafter\EmFi@AtEnd
\fi%
%    \end{macrocode}
%    \begin{macrocode}
\EmFi@RequirePackage{pdftexcmds}[2007/11/11]
\EmFi@RequirePackage{ltxcmds}[2010/03/01]
\EmFi@RequirePackage{kvsetkeys}[2010/03/01]
\EmFi@RequirePackage{kvdefinekeys}[2010/03/01]
%    \end{macrocode}
%    Check version.
%    \begin{macrocode}
\begingroup\expandafter\expandafter\expandafter\endgroup
\expandafter\ifx\csname pdf@filesize\endcsname\relax
  \EmFi@Error{%
    Unsupported pdfTeX version%
  }{%
    At least version 1.30 is necessary. Package loading is aborted.%
  }%
  \expandafter\EmFi@AtEnd
\fi%
%    \end{macrocode}
%
% \subsection{Strings}
%
%    Minimal version of package \xpackage{pdfescape} is 2007/08/27 v1.5
%    because of \cs{EdefSanitize}.
%    \begin{macrocode}
\EmFi@RequirePackage{pdfescape}[2007/11/11]
%    \end{macrocode}
%
%    \begin{macrocode}
\def\EmFi@temp#1{%
  \expandafter\EdefSanitize\csname EmFi@S@#1\endcsname{#1}%
}
%    \end{macrocode}
%    \begin{macro}{\EmFi@details}
%    \begin{macrocode}
\EmFi@temp{details}%
%    \end{macrocode}
%    \end{macro}
%    \begin{macro}{\EmFi@tile}
%    \begin{macrocode}
\EmFi@temp{tile}%
%    \end{macrocode}
%    \end{macro}
%    \begin{macro}{\EmFi@hidden}
%    \begin{macrocode}
\EmFi@temp{hidden}%
%    \end{macrocode}
%    \end{macro}
%    \begin{macro}{\EmFi@S@text}
%    \begin{macrocode}
\EmFi@temp{text}
%    \end{macrocode}
%    \end{macro}
%    \begin{macro}{\EmFi@S@date}
%    \begin{macrocode}
\EmFi@temp{date}
%    \end{macrocode}
%    \end{macro}
%    \begin{macro}{\EmFi@S@number}
%    \begin{macrocode}
\EmFi@temp{number}
%    \end{macrocode}
%    \end{macro}
%    \begin{macro}{\EmFi@S@file}
%    \begin{macrocode}
\EmFi@temp{file}
%    \end{macrocode}
%    \end{macro}
%    \begin{macro}{\EmFi@S@desc}
%    \begin{macrocode}
\EmFi@temp{desc}
%    \end{macrocode}
%    \end{macro}
%    \begin{macro}{\EmFi@S@moddate}
%    \begin{macrocode}
\EmFi@temp{moddate}
%    \end{macrocode}
%    \end{macro}
%    \begin{macro}{\EmFi@S@creationdate}
%    \begin{macrocode}
\EmFi@temp{creationdate}
%    \end{macrocode}
%    \end{macro}
%    \begin{macro}{\EmFi@S@size}
%    \begin{macrocode}
\EmFi@temp{size}
%    \end{macrocode}
%    \end{macro}
%    \begin{macro}{\EmFi@S@ascending}
%    \begin{macrocode}
\EmFi@temp{ascending}
%    \end{macrocode}
%    \end{macro}
%    \begin{macro}{\EmFi@S@descending}
%    \begin{macrocode}
\EmFi@temp{descending}
%    \end{macrocode}
%    \end{macro}
%    \begin{macro}{\EmFi@S@true}
%    \begin{macrocode}
\EmFi@temp{true}
%    \end{macrocode}
%    \end{macro}
%    \begin{macro}{\EmFi@S@false}
%    \begin{macrocode}
\EmFi@temp{false}
%    \end{macrocode}
%    \end{macro}
%
% \subsection{Switches}
%
%    \begin{macro}{\ifEmFi@collection}
%    \begin{macrocode}
\ltx@newif\ifEmFi@collection
%    \end{macrocode}
%    \end{macro}
%    \begin{macro}{\ifEmFi@sort}
%    \begin{macrocode}
\ltx@newif\ifEmFi@sort
%    \end{macrocode}
%    \end{macro}
%    \begin{macro}{\ifEmFi@visible}
%    \begin{macrocode}
\ltx@newif\ifEmFi@visible
%    \end{macrocode}
%    \end{macro}
%    \begin{macro}{\ifEmFi@edit}
%    \begin{macrocode}
\ltx@newif\ifEmFi@edit
%    \end{macrocode}
%    \end{macro}
%    \begin{macro}{\ifEmFi@item}
%    \begin{macrocode}
\ltx@newif\ifEmFi@item
%    \end{macrocode}
%    \end{macro}
%    \begin{macro}{\ifEmFi@finished}
%    \begin{macrocode}
\ltx@newif\ifEmFi@finished
%    \end{macrocode}
%    \end{macro}
%    \begin{macro}{\ifEmFi@id}
%    \begin{macrocode}
\ltx@newif\ifEmFi@id
%    \end{macrocode}
%    \end{macro}
%
% \subsection{Key value definitions}
%
%    \begin{macro}{\EmFi@GlobalKey}
%    \begin{macrocode}
\def\EmFi@GlobalKey#1#2{%
  \global\expandafter\let\csname KV@#1@#2\expandafter\endcsname
                         \csname KV@#1@#2\endcsname
}
%    \end{macrocode}
%    \end{macro}
%    \begin{macro}{\EmFi@GlobalDefaultKey}
%    \begin{macrocode}
\def\EmFi@GlobalDefaultKey#1#2{%
  \EmFi@GlobalKey{#1}{#2}%
  \global\expandafter\let
      \csname KV@#1@#2@default\expandafter\endcsname
      \csname KV@#1@#2@default\endcsname
}
%    \end{macrocode}
%    \end{macro}
%
%    \begin{macro}{\EmFi@DefineKey}
%    \begin{macrocode}
\def\EmFi@DefineKey#1#2{%
  \kv@define@key{EmFi}{#1}{%
    \expandafter\def\csname EmFi@#1\endcsname{##1}%
  }%
  \expandafter\def\csname EmFi@#1\endcsname{#2}%
}
%    \end{macrocode}
%    \end{macro}
%    Subtype of the embedded file (optional).
%    \begin{macrocode}
\EmFi@DefineKey{mimetype}{}
%    \end{macrocode}
%    File specification string.
%    \begin{macrocode}
\EmFi@DefineKey{filespec}{\EmFi@file}
%    \end{macrocode}
%    File specification string in Unicode.
%    \begin{macrocode}
\EmFi@DefineKey{ucfilespec}{}
%    \end{macrocode}
%    File system (optional).
%    \begin{macrocode}
\EmFi@DefineKey{filesystem}{}
%    \end{macrocode}
%    Description (optional).
%    \begin{macrocode}
\EmFi@DefineKey{desc}{}
%    \end{macrocode}
%    Method for converting text to PDF strings.
%    \begin{macrocode}
\EmFi@DefineKey{stringmethod}{%
  \ifx\pdfstringdef\@undefined
    escape%
  \else
    \ifx\pdfstringdef\relax
      escape%
    \else
      psd%
    \fi
  \fi
}
%    \end{macrocode}
%
%    Option \xoption{id} as key for object numbers.
%    \begin{macrocode}
\kv@define@key{EmFi}{id}{%
  \def\EmFi@id{#1}%
  \EmFi@idtrue
}
%    \end{macrocode}
%    \begin{macro}{\EmFi@defobj}
%    \begin{macrocode}
\def\EmFi@defobj#1{%
  \ifEmFi@id
    \expandafter\xdef\csname EmFi@#1@\EmFi@id\endcsname{%
      \the\pdflastobj\ltx@space 0 R%
    }%
  \fi
}
%    \end{macrocode}
%    \end{macro}
%    \begin{macro}{\embedfileifobjectexists}
%    \begin{macrocode}
\def\embedfileifobjectexists#1#2{%
  \expandafter\ifx\csname EmFi@#2@#1\endcsname\relax
    \expandafter\ltx@secondoftwo
  \else
    \expandafter\ltx@firstoftwo
  \fi
}
%    \end{macrocode}
%    \end{macro}
%    \begin{macro}{\embedfilegetobject}
%    \begin{macrocode}
\def\embedfilegetobject#1#2{%
  \embedfileifobjectexists{#1}{#2}{%
    \csname EmFi@#2@#1\endcsname
  }{%
    0 0 R%
  }%
}
%    \end{macrocode}
%    \end{macro}
%
%    Initial view of the collection.
%    \begin{macrocode}
\kv@define@key{EmFi}{view}[]{%
  \EdefSanitize\EmFi@temp{#1}%
  \def\EmFi@next{%
    \global\EmFi@collectiontrue
  }%
  \ifx\EmFi@temp\ltx@empty
    \let\EmFi@view\EmFi@S@details
  \else\ifx\EmFi@temp\EmFi@S@details
    \let\EmFi@view\EmFi@S@details
  \else\ifx\EmFi@temp\EmFi@S@tile
    \let\EmFi@view\EmFi@S@tile
  \else\ifx\EmFi@temp\EmFi@S@hidden
    \let\EmFi@view\EmFi@S@hidden
  \else
    \let\EmFi@next\relax
    \EmFi@Error{%
      Unknown value `\EmFi@temp' for key `view'.\MessageBreak
      Supported values: `details', `tile', `hidden'.%
    }\@ehc
  \fi\fi\fi\fi
  \EmFi@next
}
%    \end{macrocode}
%    \begin{macrocode}
\EmFi@DefineKey{initialfile}{}
%    \end{macrocode}
%
%    \begin{macro}{\embedfilesetup}
%    \begin{macrocode}
\def\embedfilesetup{%
  \ifEmFi@finished
    \def\EmFi@next##1{}%
    \EmFi@Error{%
      \string\embedfilefield\ltx@space after \string\embedfilefinish
    }{%
      The list of embedded files is already written.%
    }%
  \else
    \def\EmFi@next{%
      \kvsetkeys{EmFi}%
    }%
  \fi
  \EmFi@next
}
%    \end{macrocode}
%    \end{macro}
%
%    \begin{macro}{\EmFi@schema}
%    \begin{macrocode}
\def\EmFi@schema{}
%    \end{macrocode}
%    \end{macro}
%    \begin{macro}{\EmFi@order}
%    \begin{macrocode}
\gdef\EmFi@order{0}
%    \end{macrocode}
%    \end{macro}
%    \begin{macro}{\EmFi@@order}
%    \begin{macrocode}
\let\EmFi@@order\relax
%    \end{macrocode}
%    \end{macro}
%    \begin{macro}{\EmFi@fieldlist}
%    \begin{macrocode}
\def\EmFi@fieldlist{}
%    \end{macrocode}
%    \end{macro}
%    \begin{macro}{\EmFi@sortcase}
%    \begin{macrocode}
\def\EmFi@sortcase{0}%
%    \end{macrocode}
%    \end{macro}
%    \begin{macro}{\embedfilefield}
%    \begin{macrocode}
\def\embedfilefield#1#2{%
  \ifEmFi@finished
    \EmFi@Error{%
      \string\embedfilefield\ltx@space after \string\embedfilefinish
    }{%
      The list of embedded files is already written.%
    }%
  \else
    \global\EmFi@collectiontrue
    \EdefSanitize\EmFi@key{#1}%
    \expandafter\ifx\csname KV@EmFi@\EmFi@key.prefix\endcsname\relax
      \begingroup
        \count@=\EmFi@order
        \advance\count@ 1 %
        \xdef\EmFi@order{\the\count@}%
        \let\EmFi@title\EmFi@key
        \let\EmFi@type\EmFi@S@text
        \EmFi@visibletrue
        \EmFi@editfalse
        \kvsetkeys{EmFiFi}{#2}%
        \EmFi@convert\EmFi@title\EmFi@title
        \xdef\EmFi@schema{%
          \EmFi@schema
          /\pdf@escapename{\EmFi@key}<<%
            /Subtype/%
            \ifx\EmFi@type\EmFi@S@date D%
            \else\ifx\EmFi@type\EmFi@S@number N%
            \else\ifx\EmFi@type\EmFi@S@file F%
            \else\ifx\EmFi@type\EmFi@S@desc Desc%
            \else\ifx\EmFi@type\EmFi@S@moddate ModDate%
            \else\ifx\EmFi@type\EmFi@S@creationdate CreationDate%
            \else\ifx\EmFi@type\EmFi@S@size Size%
            \else S%
            \fi\fi\fi\fi\fi\fi\fi
            /N(\EmFi@title)%
            \EmFi@@order{\EmFi@order}%
            \ifEmFi@visible
            \else
              /V false%
            \fi
            \ifEmFi@edit
              /E true%
            \fi
          >>%
        }%
        \let\do\relax
        \xdef\EmFi@fieldlist{%
          \EmFi@fieldlist
          \do{\EmFi@key}%
        }%
        \ifx\EmFi@type\EmFi@S@text
          \kv@define@key{EmFi}{\EmFi@key.value}{%
            \EmFi@itemtrue
            \def\EmFi@temp{##1}%
            \EmFi@convert\EmFi@temp\EmFi@temp
            \expandafter\def\csname EmFi@V@#1%
            \expandafter\endcsname\expandafter{%
              \expandafter(\EmFi@temp)%
            }%
          }%
          \EmFi@GlobalKey{EmFi}{\EmFi@key.value}%
        \else\ifx\EmFi@type\EmFi@S@date
          \kv@define@key{EmFi}{\EmFi@key.value}{%
            \EmFi@itemtrue
            \def\EmFi@temp{##1}%
            \EmFi@convert\EmFi@temp\EmFi@temp
            \expandafter\def\csname EmFi@V@#1%
            \expandafter\endcsname\expandafter{%
              \expandafter(\EmFi@temp)%
            }%
          }%
          \EmFi@GlobalKey{EmFi}{\EmFi@key.value}%
        \else\ifx\EmFi@type\EmFi@S@number
          \kv@define@key{EmFi}{\EmFi@key.value}{%
            \EmFi@itemtrue
            \expandafter\EdefSanitize\csname EmFi@V@#1\endcsname{ ##1}%
          }%
          \EmFi@GlobalKey{EmFi}{\EmFi@key.value}%
        \fi\fi\fi
        \kv@define@key{EmFi}{\EmFi@key.prefix}{%
          \EmFi@itemtrue
          \expandafter\def\csname EmFi@P@#1\endcsname{##1}%
        }%
        \EmFi@GlobalKey{EmFi}{\EmFi@key.prefix}%
        \kv@define@key{EmFiSo}{\EmFi@key}[ascending]{%
          \EdefSanitize\EmFi@temp{##1}%
          \ifx\EmFi@temp\EmFi@S@ascending
            \def\EmFi@temp{true}%
          \else\ifx\EmFi@temp\EmFi@S@descending
            \def\EmFi@temp{false}%
          \else
            \def\EmFi@temp{}%
            \EmFi@Error{%
              Unknown sort order `\EmFi@temp'.\MessageBreak
              Supported values: `\EmFi@S@ascending', %
              `\EmFi@S@descending
            }\@ehc
          \fi\fi
          \ifx\EmFi@temp\ltx@empty
          \else
            \xdef\EmFi@sortkeys{%
              \EmFi@sortkeys
              /\pdf@escapename{#1}%
            }%
            \ifx\EmFi@sortorders\ltx@empty
              \global\let\EmFi@sortorders\EmFi@temp
              \gdef\EmFi@sortcase{1}%
            \else
              \xdef\EmFi@sortorders{%
                \EmFi@sortorders
                \ltx@space
                \EmFi@temp
              }%
              \xdef\EmFi@sortcase{2}%
            \fi
          \fi
        }%
        \EmFi@GlobalDefaultKey{EmFiSo}\EmFi@key
      \endgroup
    \else
      \EmFi@Error{%
        Field `\EmFi@key' is already defined%
      }\@ehc
    \fi
  \fi
}
%    \end{macrocode}
%    \end{macro}
%    \begin{macrocode}
\kv@define@key{EmFiFi}{type}{%
  \EdefSanitize\EmFi@temp{#1}%
  \ifx\EmFi@temp\EmFi@S@text
    \let\EmFi@type\EmFi@temp
  \else\ifx\EmFi@temp\EmFi@S@date
    \let\EmFi@type\EmFi@temp
  \else\ifx\EmFi@temp\EmFi@S@number
    \let\EmFi@type\EmFi@temp
  \else\ifx\EmFi@temp\EmFi@S@file
    \let\EmFi@type\EmFi@temp
  \else\ifx\EmFi@temp\EmFi@S@desc
    \let\EmFi@type\EmFi@temp
  \else\ifx\EmFi@temp\EmFi@S@moddate
    \let\EmFi@type\EmFi@temp
  \else\ifx\EmFi@temp\EmFi@S@creationdate
    \let\EmFi@type\EmFi@temp
  \else\ifx\EmFi@temp\EmFi@S@size
    \let\EmFi@type\EmFi@temp
  \else
    \EmFi@Error{%
      Unknown type `\EmFi@temp'.\MessageBreak
      Supported types: `text', `date', `number', `file',\MessageBreak
      `desc', `moddate', `creationdate', `size'%
    }%
  \fi\fi\fi\fi\fi\fi\fi\fi
}
%    \end{macrocode}
%    \begin{macrocode}
\kv@define@key{EmFiFi}{title}{%
  \def\EmFi@title{#1}%
}
%    \end{macrocode}
%    \begin{macro}{\EmFi@setboolean}
%    \begin{macrocode}
\def\EmFi@setboolean#1#2{%
  \EdefSanitize\EmFi@temp{#2}%
  \ifx\EmFi@temp\EmFi@S@true
    \csname EmFi@#1true\endcsname
  \else
    \ifx\EmFi@temp\EmFi@S@false
      \csname EmFi@#1false\endcsname
    \else
      \EmFi@Error{%
        Unknown value `\EmFi@temp' for key `#1'.\MessageBreak
        Supported values: `true', `false'%
      }\@ehc
    \fi
  \fi
}
%    \end{macrocode}
%    \end{macro}
%    \begin{macrocode}
\kv@define@key{EmFiFi}{visible}[true]{%
  \EmFi@setboolean{visible}{#1}%
}
%    \end{macrocode}
%    \begin{macrocode}
\kv@define@key{EmFiFi}{edit}[true]{%
  \EmFi@setboolean{edit}{#1}%
}
%    \end{macrocode}
%
%    \begin{macro}{\EmFi@sortkeys}
%    \begin{macrocode}
\def\EmFi@sortkeys{}
%    \end{macrocode}
%    \end{macro}
%    \begin{macro}{\EmFi@sortorders}
%    \begin{macrocode}
\def\EmFi@sortorders{}
%    \end{macrocode}
%    \end{macro}
%    \begin{macro}{\embedfilesort}
%    \begin{macrocode}
\def\embedfilesort{%
  \kvsetkeys{EmFiSo}%
}
%    \end{macrocode}
%    \end{macro}
%
% \subsection{Embed the file}
%
%    \begin{macro}{\embedfile}
%    \begin{macrocode}
\def\embedfile{%
  \ltx@ifnextchar[\EmFi@embedfile{\EmFi@embedfile[]}%
}
%    \end{macrocode}
%    \end{macro}
%    \begin{macro}{\EmFi@embedfile}
%    \begin{macrocode}
\def\EmFi@embedfile[#1]#2{%
  \ifEmFi@finished
    \EmFi@Error{%
      \string\embedfile\ltx@space after \string\embedfilefinish
    }{%
      The list of embedded files is already written.%
    }%
  \else
    \begingroup
      \def\EmFi@file{#2}%
      \kvsetkeys{EmFi}{#1}%
      \expandafter\expandafter\expandafter
      \ifx\expandafter\expandafter\expandafter
          \\\pdf@filesize{\EmFi@file}\\%
        \EmFi@Error{%
          File `\EmFi@file' not found%
        }{%
          The unknown file is not embedded.%
        }%
      \else
        \edef\EmFi@@filespec{%
          \pdf@escapestring{\EmFi@filespec}%
        }%
        \ifx\EmFi@ucfilespec\ltx@empty
          \let\EmFi@@ucfilespec\ltx@empty
        \else
          \EmFi@convert\EmFi@ucfilespec\EmFi@@ucfilespec
        \fi
        \ifx\EmFi@desc\ltx@empty
          \let\EmFi@@desc\ltx@empty
        \else
          \EmFi@convert\EmFi@desc\EmFi@@desc
        \fi
        \ifEmFi@item
          \let\do\EmFi@do
          \immediate\pdfobj{%
            <<%
              \EmFi@fieldlist
            >>%
          }%
          \edef\EmFi@ci{\the\pdflastobj}%
        \fi
        \immediate\pdfobj stream attr{%
          /Type/EmbeddedFile%
          \ifx\EmFi@mimetype\ltx@empty
          \else
            /Subtype/\pdf@escapename{\EmFi@mimetype}%
          \fi
          /Params<<%
            /ModDate(\pdf@filemoddate{\EmFi@file})%
            /Size \pdf@filesize{\EmFi@file}%
            /CheckSum<\pdf@filemdfivesum{\EmFi@file}>%
          >>%
        }file{\EmFi@file}\relax
        \EmFi@defobj{EmbeddedFile}%
        \immediate\pdfobj{%
          <<%
            /Type/Filespec%
            \ifx\EmFi@filesystem\ltx@empty
            \else
            /FS/\pdf@escapename{\EmFi@filesystem}%
            \fi
            /F(\EmFi@@filespec)%
            \ifx\EmFi@@ucfilespec\ltx@empty
            \else
              /UF(\EmFi@@ucfilespec)%
            \fi
            \ifx\EmFi@@desc\ltx@empty
            \else
              /Desc(\EmFi@@desc)%
            \fi
            /EF<<%
              /F \the\pdflastobj\ltx@space 0 R%
            >>%
            \ifEmFi@item
              /CI \EmFi@ci\ltx@space 0 R%
            \fi
          >>%
        }%
        \EmFi@defobj{Filespec}%
        \EmFi@add{%
          \EmFi@@filespec
        }{\the\pdflastobj\ltx@space 0 R}%
      \fi
    \endgroup
  \fi
}
%    \end{macrocode}
%    \end{macro}
%    \begin{macro}{\EmFi@do}
%    \begin{macrocode}
\def\EmFi@do#1{%
  \expandafter\ifx\csname EmFi@P@#1\endcsname\relax
    \expandafter\ifx\csname EmFi@V@#1\endcsname\relax
    \else
      /\pdf@escapename{#1}\csname EmFi@V@#1\endcsname
    \fi
  \else
    /\pdf@escapename{#1}<<%
      \expandafter\ifx\csname EmFi@V@#1\endcsname\relax
      \else
        /D\csname EmFi@V@#1\endcsname
      \fi
      /P(\csname EmFi@P@#1\endcsname)%
    >>%
  \fi
}
%    \end{macrocode}
%    \end{macro}
%    \begin{macro}{\EmFi@convert}
%    \begin{macrocode}
\def\EmFi@convert#1#2{%
  \ifnum\pdf@strcmp{\EmFi@stringmethod}{psd}=0 %
    \pdfstringdef\EmFi@temp{#1}%
    \let#2\EmFi@temp
  \else
    \edef#2{\pdf@escapestring{#1}}%
  \fi
}
%    \end{macrocode}
%    \end{macro}
%    \begin{macrocode}
\global\let\EmFi@list\ltx@empty
%    \end{macrocode}
%    \begin{macro}{\EmFi@add}
%    Sorting is done by the insertion sort algorithm.
%    Probably the sorting could be done more reliable.
%    However, the PDF specification is not too clear to me
%    regarding precise sorting rules (how to deal with
%    different encodings, escaped characters, \dots).
%    \begin{macrocode}
\def\EmFi@add#1#2{%
  \begingroup
    \ifx\EmFi@list\ltx@empty
      \xdef\EmFi@list{\noexpand\do{#1}{#2}}%
    \else
      \def\do##1##2{%
        \ifnum\pdf@strcmp{##1}{#1}>0 %
          \edef\x{%
            \toks@{%
              \the\toks@%
              \noexpand\do{#1}{#2}%
              \noexpand\do{##1}{##2}%
            }%
          }%
          \x
          \def\do####1####2{%
            \toks@\expandafter{\the\toks@\do{####1}{####2}}%
          }%
          \def\stop{%
            \xdef\EmFi@list{\the\toks@}%
          }%
        \else
          \toks@\expandafter{\the\toks@\do{##1}{##2}}%
        \fi
      }%
      \def\stop{%
        \xdef\EmFi@list{\the\toks@\noexpand\do{#1}{#2}}%
      }%
      \toks@{}%
      \EmFi@list\stop
    \fi
  \endgroup
}
%    \end{macrocode}
%    \end{macro}
%    \begin{macro}{\embedfilefinish}
%    \begin{macrocode}
\def\embedfilefinish{%
  \ifEmFi@finished
    \EmFi@Error{%
      Too many invocations of \string\embedfilefinish
    }{%
      The list of embedded files is already written.%
    }%
  \else
    \ifx\EmFi@list\ltx@empty
    \else
%    \end{macrocode}
%    Write |/EmbeddedFiles| entry.
%    \begin{macrocode}
      \global\EmFi@finishedtrue
      \begingroup
        \def\do##1##2{%
          (##1)##2%
        }%
        \immediate\pdfobj{%
          <<%
            /Names[\EmFi@list]%
          >>%
        }%
        \pdfnames{%
          /EmbeddedFiles \the\pdflastobj\ltx@space 0 R%
        }%
      \endgroup
%    \end{macrocode}
%    Write collection objects.
%    \begin{macrocode}
      \ifx\EmFi@initialfile\ltx@empty
      \else
        \EmFi@collectiontrue
      \fi
      \ifEmFi@collection
        \ifx\EmFi@initialfile\ltx@empty
          \let\EmFi@@initialfile\ltx@empty
        \else
          \edef\EmFi@@initialfile{%
            \pdf@escapestring{\EmFi@initialfile}%
          }%
        \fi
%    \end{macrocode}
%    Look for initial file among the embedded files.
%    \begin{macrocode}
        \begingroup
          \let\f=N%
          \def\do##1##2{%
            \def\x{##1}%
            \ifx\x\EmFi@@initialfile
              \let\f=Y%
              \let\do\ltx@gobbletwo
            \fi
          }%
          \EmFi@list
        \expandafter\endgroup
        \ifx\f Y%
        \else
          \@PackageWarningNoLine{embedfile}{%
            Missing initial file `\EmFi@initialfile'\MessageBreak
            among the embedded files%
          }%
          \let\EmFi@initialfile\ltx@empty
          \let\EmFi@@initialfile\ltx@empty
        \fi
        \ifcase\EmFi@sortcase
          \def\EmFi@temp{}%
        \or
          \def\EmFi@temp{%
            /S\EmFi@sortkeys
            /A \EmFi@sortorders
          }%
        \else
          \def\EmFi@temp{%
            /S[\EmFi@sortkeys]%
            /A[\EmFi@sortorders]%
          }%
        \fi
        \def\EmFi@@order##1{%
          \ifnum\EmFi@order>1 %
            /O ##1%
          \fi
        }%
        \immediate\pdfobj{%
          <<%
            \ifx\EmFi@schema\ltx@empty
            \else
              /Schema<<\EmFi@schema>>%
            \fi
            \ifx\EmFi@@initialfile\ltx@empty
            \else
              /D(\EmFi@@initialfile)%
            \fi
            \ifx\EmFi@view\EmFi@S@tile
              /View/T%
            \else\ifx\EmFi@view\EmFi@S@hidden
              /View/H%
            \fi\fi
            \ifx\EmFi@temp\ltx@empty
              \EmFi@temp
            \else
              /Sort<<\EmFi@temp>>%
            \fi
          >>%
        }%
        \pdfcatalog{%
          /Collection \the\pdflastobj\ltx@space0 R%
        }%
      \fi
    \fi
  \fi
}
%    \end{macrocode}
%    \end{macro}
%    \begin{macrocode}
\begingroup\expandafter\expandafter\expandafter\endgroup
\expandafter\ifx\csname AtEndDocument\endcsname\relax
\else
  \AtEndDocument{\embedfilefinish}%
\fi
%    \end{macrocode}
%
%    \begin{macrocode}
\EmFi@AtEnd%
%</package>
%    \end{macrocode}
%
% \section{Test}
%
% \subsection{Catcode checks for loading}
%
%    \begin{macrocode}
%<*test1>
%    \end{macrocode}
%    \begin{macrocode}
\catcode`\{=1 %
\catcode`\}=2 %
\catcode`\#=6 %
\catcode`\@=11 %
\expandafter\ifx\csname count@\endcsname\relax
  \countdef\count@=255 %
\fi
\expandafter\ifx\csname @gobble\endcsname\relax
  \long\def\@gobble#1{}%
\fi
\expandafter\ifx\csname @firstofone\endcsname\relax
  \long\def\@firstofone#1{#1}%
\fi
\expandafter\ifx\csname loop\endcsname\relax
  \expandafter\@firstofone
\else
  \expandafter\@gobble
\fi
{%
  \def\loop#1\repeat{%
    \def\body{#1}%
    \iterate
  }%
  \def\iterate{%
    \body
      \let\next\iterate
    \else
      \let\next\relax
    \fi
    \next
  }%
  \let\repeat=\fi
}%
\def\RestoreCatcodes{}
\count@=0 %
\loop
  \edef\RestoreCatcodes{%
    \RestoreCatcodes
    \catcode\the\count@=\the\catcode\count@\relax
  }%
\ifnum\count@<255 %
  \advance\count@ 1 %
\repeat

\def\RangeCatcodeInvalid#1#2{%
  \count@=#1\relax
  \loop
    \catcode\count@=15 %
  \ifnum\count@<#2\relax
    \advance\count@ 1 %
  \repeat
}
\def\RangeCatcodeCheck#1#2#3{%
  \count@=#1\relax
  \loop
    \ifnum#3=\catcode\count@
    \else
      \errmessage{%
        Character \the\count@\space
        with wrong catcode \the\catcode\count@\space
        instead of \number#3%
      }%
    \fi
  \ifnum\count@<#2\relax
    \advance\count@ 1 %
  \repeat
}
\def\space{ }
\expandafter\ifx\csname LoadCommand\endcsname\relax
  \def\LoadCommand{\input embedfile.sty\relax}%
\fi
\def\Test{%
  \RangeCatcodeInvalid{0}{47}%
  \RangeCatcodeInvalid{58}{64}%
  \RangeCatcodeInvalid{91}{96}%
  \RangeCatcodeInvalid{123}{255}%
  \catcode`\@=12 %
  \catcode`\\=0 %
  \catcode`\%=14 %
  \LoadCommand
  \RangeCatcodeCheck{0}{36}{15}%
  \RangeCatcodeCheck{37}{37}{14}%
  \RangeCatcodeCheck{38}{47}{15}%
  \RangeCatcodeCheck{48}{57}{12}%
  \RangeCatcodeCheck{58}{63}{15}%
  \RangeCatcodeCheck{64}{64}{12}%
  \RangeCatcodeCheck{65}{90}{11}%
  \RangeCatcodeCheck{91}{91}{15}%
  \RangeCatcodeCheck{92}{92}{0}%
  \RangeCatcodeCheck{93}{96}{15}%
  \RangeCatcodeCheck{97}{122}{11}%
  \RangeCatcodeCheck{123}{255}{15}%
  \RestoreCatcodes
}
\Test
\csname @@end\endcsname
\end
%    \end{macrocode}
%    \begin{macrocode}
%</test1>
%    \end{macrocode}
%
% \subsection{Simple test}
%
%    \begin{macrocode}
%<*test2>
\input embedfile.sty\relax
\embedfile[%
  stringmethod=escape,%
  mimetype=plain/text,%
  desc={LaTeX docstrip source archive for package `embedfile'},%
  id={embedfile.dtx}%
]{embedfile.dtx}
\nopagenumbers
Test (plain-\TeX): {\tt embedfile.dtx} should be embedded.%

\def\Test#1{%
  \par
  \embedfileifobjectexists{embedfile.dtx}{#1}{%
    Object #1 (embedfile.dtx): %
    \embedfilegetobject{embedfile.dtx}{#1}%
  }{%
    \errmessage{Missing object #1 (embedfile.dtx)}%
  }%
}
\Test{EmbeddedFile}
\Test{Filespec}
\embedfilefinish
\bye
%</test2>
%    \end{macrocode}
%    \begin{macrocode}
%<*test3>
\NeedsTeXFormat{LaTeX2e}
\let\SavedJobname\jobname
\def\jobname{embedfile}
\RequirePackage{dtx-attach}[2011/04/13]
\let\jobname\SavedJobname
\documentclass{minimal}
\begin{document}
  Test (\LaTeX): \texttt{embedfile.dtx} should be embedded.%
\end{document}
%</test3>
%    \end{macrocode}
%
% \subsection{Test for \iniTeX}
%
%    \begin{macrocode}
%<*test4>
\catcode`\{=1 %
\catcode`\}=2 %
\input ifluatex.sty %
\ifluatex
  \directlua{%
    tex.enableprimitives('', {%
      'pdflastobj',%
      'pdfnames',%
      'pdfobj',%
      'pdfoutput'%
    })%
  }%
\fi
\pdfoutput=1 %
\input embedfile.sty %
\shipout\hbox{}
\embedfile[%
  stringmethod=escape,%
  mimetype=plain/text,%
  desc={iniTeX source},%
]{\jobname.tex}
\embedfilefinish
\end
%</test4>
%    \end{macrocode}
%
% \section{Installation}
%
% \subsection{Download}
%
% \paragraph{Package.} This package is available on
% CTAN\footnote{\url{ftp://ftp.ctan.org/tex-archive/}}:
% \begin{description}
% \item[\CTAN{macros/latex/contrib/oberdiek/embedfile.dtx}] The source file.
% \item[\CTAN{macros/latex/contrib/oberdiek/embedfile.pdf}] Documentation.
% \end{description}
%
%
% \paragraph{Bundle.} All the packages of the bundle `oberdiek'
% are also available in a TDS compliant ZIP archive. There
% the packages are already unpacked and the documentation files
% are generated. The files and directories obey the TDS standard.
% \begin{description}
% \item[\CTAN{install/macros/latex/contrib/oberdiek.tds.zip}]
% \end{description}
% \emph{TDS} refers to the standard ``A Directory Structure
% for \TeX\ Files'' (\CTAN{tds/tds.pdf}). Directories
% with \xfile{texmf} in their name are usually organized this way.
%
% \subsection{Bundle installation}
%
% \paragraph{Unpacking.} Unpack the \xfile{oberdiek.tds.zip} in the
% TDS tree (also known as \xfile{texmf} tree) of your choice.
% Example (linux):
% \begin{quote}
%   |unzip oberdiek.tds.zip -d ~/texmf|
% \end{quote}
%
% \paragraph{Script installation.}
% Check the directory \xfile{TDS:scripts/oberdiek/} for
% scripts that need further installation steps.
% Package \xpackage{attachfile2} comes with the Perl script
% \xfile{pdfatfi.pl} that should be installed in such a way
% that it can be called as \texttt{pdfatfi}.
% Example (linux):
% \begin{quote}
%   |chmod +x scripts/oberdiek/pdfatfi.pl|\\
%   |cp scripts/oberdiek/pdfatfi.pl /usr/local/bin/|
% \end{quote}
%
% \subsection{Package installation}
%
% \paragraph{Unpacking.} The \xfile{.dtx} file is a self-extracting
% \docstrip\ archive. The files are extracted by running the
% \xfile{.dtx} through \plainTeX:
% \begin{quote}
%   \verb|tex embedfile.dtx|
% \end{quote}
%
% \paragraph{TDS.} Now the different files must be moved into
% the different directories in your installation TDS tree
% (also known as \xfile{texmf} tree):
% \begin{quote}
% \def\t{^^A
% \begin{tabular}{@{}>{\ttfamily}l@{ $\rightarrow$ }>{\ttfamily}l@{}}
%   embedfile.sty & tex/generic/oberdiek/embedfile.sty\\
%   dtx-attach.sty & tex/generic/oberdiek/dtx-attach.sty\\
%   embedfile.pdf & doc/latex/oberdiek/embedfile.pdf\\
%   embedfile-example-plain.tex & doc/latex/oberdiek/embedfile-example-plain.tex\\
%   embedfile-example-collection.tex & doc/latex/oberdiek/embedfile-example-collection.tex\\
%   test/embedfile-test1.tex & doc/latex/oberdiek/test/embedfile-test1.tex\\
%   test/embedfile-test2.tex & doc/latex/oberdiek/test/embedfile-test2.tex\\
%   test/embedfile-test3.tex & doc/latex/oberdiek/test/embedfile-test3.tex\\
%   test/embedfile-test4.tex & doc/latex/oberdiek/test/embedfile-test4.tex\\
%   embedfile.dtx & source/latex/oberdiek/embedfile.dtx\\
% \end{tabular}^^A
% }^^A
% \sbox0{\t}^^A
% \ifdim\wd0>\linewidth
%   \begingroup
%     \advance\linewidth by\leftmargin
%     \advance\linewidth by\rightmargin
%   \edef\x{\endgroup
%     \def\noexpand\lw{\the\linewidth}^^A
%   }\x
%   \def\lwbox{^^A
%     \leavevmode
%     \hbox to \linewidth{^^A
%       \kern-\leftmargin\relax
%       \hss
%       \usebox0
%       \hss
%       \kern-\rightmargin\relax
%     }^^A
%   }^^A
%   \ifdim\wd0>\lw
%     \sbox0{\small\t}^^A
%     \ifdim\wd0>\linewidth
%       \ifdim\wd0>\lw
%         \sbox0{\footnotesize\t}^^A
%         \ifdim\wd0>\linewidth
%           \ifdim\wd0>\lw
%             \sbox0{\scriptsize\t}^^A
%             \ifdim\wd0>\linewidth
%               \ifdim\wd0>\lw
%                 \sbox0{\tiny\t}^^A
%                 \ifdim\wd0>\linewidth
%                   \lwbox
%                 \else
%                   \usebox0
%                 \fi
%               \else
%                 \lwbox
%               \fi
%             \else
%               \usebox0
%             \fi
%           \else
%             \lwbox
%           \fi
%         \else
%           \usebox0
%         \fi
%       \else
%         \lwbox
%       \fi
%     \else
%       \usebox0
%     \fi
%   \else
%     \lwbox
%   \fi
% \else
%   \usebox0
% \fi
% \end{quote}
% If you have a \xfile{docstrip.cfg} that configures and enables \docstrip's
% TDS installing feature, then some files can already be in the right
% place, see the documentation of \docstrip.
%
% \subsection{Refresh file name databases}
%
% If your \TeX~distribution
% (\teTeX, \mikTeX, \dots) relies on file name databases, you must refresh
% these. For example, \teTeX\ users run \verb|texhash| or
% \verb|mktexlsr|.
%
% \subsection{Some details for the interested}
%
% \paragraph{Attached source.}
%
% The PDF documentation on CTAN also includes the
% \xfile{.dtx} source file. It can be extracted by
% AcrobatReader 6 or higher. Another option is \textsf{pdftk},
% e.g. unpack the file into the current directory:
% \begin{quote}
%   \verb|pdftk embedfile.pdf unpack_files output .|
% \end{quote}
%
% \paragraph{Unpacking with \LaTeX.}
% The \xfile{.dtx} chooses its action depending on the format:
% \begin{description}
% \item[\plainTeX:] Run \docstrip\ and extract the files.
% \item[\LaTeX:] Generate the documentation.
% \end{description}
% If you insist on using \LaTeX\ for \docstrip\ (really,
% \docstrip\ does not need \LaTeX), then inform the autodetect routine
% about your intention:
% \begin{quote}
%   \verb|latex \let\install=y\input{embedfile.dtx}|
% \end{quote}
% Do not forget to quote the argument according to the demands
% of your shell.
%
% \paragraph{Generating the documentation.}
% You can use both the \xfile{.dtx} or the \xfile{.drv} to generate
% the documentation. The process can be configured by the
% configuration file \xfile{ltxdoc.cfg}. For instance, put this
% line into this file, if you want to have A4 as paper format:
% \begin{quote}
%   \verb|\PassOptionsToClass{a4paper}{article}|
% \end{quote}
% An example follows how to generate the
% documentation with pdf\LaTeX:
% \begin{quote}
%\begin{verbatim}
%pdflatex embedfile.dtx
%makeindex -s gind.ist embedfile.idx
%pdflatex embedfile.dtx
%makeindex -s gind.ist embedfile.idx
%pdflatex embedfile.dtx
%\end{verbatim}
% \end{quote}
%
% \section{Catalogue}
%
% The following XML file can be used as source for the
% \href{http://mirror.ctan.org/help/Catalogue/catalogue.html}{\TeX\ Catalogue}.
% The elements \texttt{caption} and \texttt{description} are imported
% from the original XML file from the Catalogue.
% The name of the XML file in the Catalogue is \xfile{embedfile.xml}.
%    \begin{macrocode}
%<*catalogue>
<?xml version='1.0' encoding='us-ascii'?>
<!DOCTYPE entry SYSTEM 'catalogue.dtd'>
<entry datestamp='$Date$' modifier='$Author$' id='embedfile'>
  <name>embedfile</name>
  <caption>Embed files into PDF.</caption>
  <authorref id='auth:oberdiek'/>
  <copyright owner='Heiko Oberdiek' year='2006-2011'/>
  <license type='lppl1.3'/>
  <version number='2.6'/>
  <description>
    This package embeds files in a PDF document, using the PDF
    format's embedding operation (note the contrast with the attach
    operation used by the <xref refid='attachfile'>attachfile</xref>
    and <xref refid='attachfile2'>attachfile2</xref> packages).
    Currently only <xref refid='pdftex'>pdfTeX</xref> &gt;=1.30, in
    PDF mode, is supported.

    <p/>
    The package is part of the <xref refid='oberdiek'>oberdiek</xref>
    bundle.
  </description>
  <documentation details='Package documentation'
      href='ctan:/macros/latex/contrib/oberdiek/embedfile.pdf'/>
  <ctan file='true' path='/macros/latex/contrib/oberdiek/embedfile.dtx'/>
  <miktex location='oberdiek'/>
  <texlive location='oberdiek'/>
  <install path='/macros/latex/contrib/oberdiek/oberdiek.tds.zip'/>
</entry>
%</catalogue>
%    \end{macrocode}
%
% \begin{thebibliography}{9}
% \bibitem{attachfile}
%   Scott Pakin: \textit{The \xpackage{attachfile} package};
%   2005/02/20 v1.2;
%   \CTAN{macros/latex/contrib/attachfile/}.
%
% \bibitem{attachfile2}
%   Heiko Oberdiek: \textit{The \xpackage{attachfile2} package};
%   2006/08/16 v2.2;
%   \CTAN{macros/latex/contrib/oberdiek/attachfile2.pdf}.
%
% \bibitem{pdfspec}
%   Adobe Systems Incorporated:
%   \href{http://www.adobe.com/devnet/acrobat/pdfs/pdf_reference.pdf}%
%       {\textit{PDF Reference, Sixth Edition, Version 1.7}},%
%   Oktober 2006;
%   \url{http://www.adobe.com/devnet/pdf/pdf_reference.html}.
%
% \bibitem{mime}
%   Network Working Group: RFC 2046, \textit{Multipurpose Internet Mail Extensions (MIME) Part Two:
%     Media Types}, November 1996; \url{http://www.rfc-editor.org/}.
%
% \bibitem{types}
%   IANA (Internet Assigned Numbers Authority):
%   \textit{MIME Media Types}, May 2006;
%   \url{http://www.iana.org/assignments/media-types/}.
%
% \end{thebibliography}
%
% \begin{History}
%   \begin{Version}{2006/08/16 v1.0}
%   \item
%     First public version.
%   \end{Version}
%   \begin{Version}{2007/04/11 v1.1}
%   \item
%     Line ends sanitized.
%   \end{Version}
%   \begin{Version}{2007/09/09 v1.2}
%   \item
%     Fixes for plain-TeX, wrapper for package \xpackage{keyval} added.
%   \item
%     Catcode section rewritten.
%   \end{Version}
%   \begin{Version}{2007/10/28 v2.0}
%   \item
%     Collection support added (PDF 1.7).
%   \end{Version}
%   \begin{Version}{2007/10/29 v2.1}
%   \item
%     Export of object references by adding new option \xoption{id} and
%     new macros \cs{embedfileifobjectexists} and \cs{embedfilegetobject}.
%   \end{Version}
%   \begin{Version}{2007/11/11 v2.2}
%   \item
%     Use of package \xpackage{pdftexcmds} for \LuaTeX\ support.
%   \end{Version}
%   \begin{Version}{2007/11/25 v2.3}
%   \item
%     Fix in use of \cs{pdf@filesize}, bug introduced in previous version.
%   \end{Version}
%   \begin{Version}{2009/09/25 v2.4}
%   \item
%     Bug fix: If \xpackage{hyperref} is used with option \xoption{unicode},
%     the Unicode encoded file name causes trouble. Therefore \cs{pdfstringdef}
%     is now never used for option \xoption{filespec}, always method
%     \texttt{escape} is applied (Peter Cibulka).
%   \item
%     Bug fix for \xoption{initialfile}.
%   \item
%     Bug fix for file names in \texttt{/EmbeddedFiles}.
%   \item
%     New option \xoption{ucfilespec} for file name support in Unicode
%     (since PDF 1.7).
%   \end{Version}
%   \begin{Version}{2010/03/01 v2.5}
%   \item
%     Compatibility for \iniTeX.
%   \item
%     Package \xpackage{keyval} replaced by packages \xpackage{kvsetkeys}
%     and \xpackage{kvdefinekeys} because of compatibility for \iniTeX.
%   \item
%     TDS location moved from TDS:tex/latex/oberdiek/embedfile.sty to
%     TDS:tex/generic/oberdiek/embedfile.sty.
%   \end{Version}
%   \begin{Version}{2011/04/13 v2.6}
%   \item
%     Docu fixes (thanks Hans-Martin M\"unch).
%   \end{Version}
% \end{History}
%
% \PrintIndex
%
% \Finale
\endinput
|
% \end{quote}
% Do not forget to quote the argument according to the demands
% of your shell.
%
% \paragraph{Generating the documentation.}
% You can use both the \xfile{.dtx} or the \xfile{.drv} to generate
% the documentation. The process can be configured by the
% configuration file \xfile{ltxdoc.cfg}. For instance, put this
% line into this file, if you want to have A4 as paper format:
% \begin{quote}
%   \verb|\PassOptionsToClass{a4paper}{article}|
% \end{quote}
% An example follows how to generate the
% documentation with pdf\LaTeX:
% \begin{quote}
%\begin{verbatim}
%pdflatex embedfile.dtx
%makeindex -s gind.ist embedfile.idx
%pdflatex embedfile.dtx
%makeindex -s gind.ist embedfile.idx
%pdflatex embedfile.dtx
%\end{verbatim}
% \end{quote}
%
% \section{Catalogue}
%
% The following XML file can be used as source for the
% \href{http://mirror.ctan.org/help/Catalogue/catalogue.html}{\TeX\ Catalogue}.
% The elements \texttt{caption} and \texttt{description} are imported
% from the original XML file from the Catalogue.
% The name of the XML file in the Catalogue is \xfile{embedfile.xml}.
%    \begin{macrocode}
%<*catalogue>
<?xml version='1.0' encoding='us-ascii'?>
<!DOCTYPE entry SYSTEM 'catalogue.dtd'>
<entry datestamp='$Date$' modifier='$Author$' id='embedfile'>
  <name>embedfile</name>
  <caption>Embed files into PDF.</caption>
  <authorref id='auth:oberdiek'/>
  <copyright owner='Heiko Oberdiek' year='2006-2011'/>
  <license type='lppl1.3'/>
  <version number='2.6'/>
  <description>
    This package embeds files in a PDF document, using the PDF
    format's embedding operation (note the contrast with the attach
    operation used by the <xref refid='attachfile'>attachfile</xref>
    and <xref refid='attachfile2'>attachfile2</xref> packages).
    Currently only <xref refid='pdftex'>pdfTeX</xref> &gt;=1.30, in
    PDF mode, is supported.

    <p/>
    The package is part of the <xref refid='oberdiek'>oberdiek</xref>
    bundle.
  </description>
  <documentation details='Package documentation'
      href='ctan:/macros/latex/contrib/oberdiek/embedfile.pdf'/>
  <ctan file='true' path='/macros/latex/contrib/oberdiek/embedfile.dtx'/>
  <miktex location='oberdiek'/>
  <texlive location='oberdiek'/>
  <install path='/macros/latex/contrib/oberdiek/oberdiek.tds.zip'/>
</entry>
%</catalogue>
%    \end{macrocode}
%
% \begin{thebibliography}{9}
% \bibitem{attachfile}
%   Scott Pakin: \textit{The \xpackage{attachfile} package};
%   2005/02/20 v1.2;
%   \CTAN{macros/latex/contrib/attachfile/}.
%
% \bibitem{attachfile2}
%   Heiko Oberdiek: \textit{The \xpackage{attachfile2} package};
%   2006/08/16 v2.2;
%   \CTAN{macros/latex/contrib/oberdiek/attachfile2.pdf}.
%
% \bibitem{pdfspec}
%   Adobe Systems Incorporated:
%   \href{http://www.adobe.com/devnet/acrobat/pdfs/pdf_reference.pdf}%
%       {\textit{PDF Reference, Sixth Edition, Version 1.7}},%
%   Oktober 2006;
%   \url{http://www.adobe.com/devnet/pdf/pdf_reference.html}.
%
% \bibitem{mime}
%   Network Working Group: RFC 2046, \textit{Multipurpose Internet Mail Extensions (MIME) Part Two:
%     Media Types}, November 1996; \url{http://www.rfc-editor.org/}.
%
% \bibitem{types}
%   IANA (Internet Assigned Numbers Authority):
%   \textit{MIME Media Types}, May 2006;
%   \url{http://www.iana.org/assignments/media-types/}.
%
% \end{thebibliography}
%
% \begin{History}
%   \begin{Version}{2006/08/16 v1.0}
%   \item
%     First public version.
%   \end{Version}
%   \begin{Version}{2007/04/11 v1.1}
%   \item
%     Line ends sanitized.
%   \end{Version}
%   \begin{Version}{2007/09/09 v1.2}
%   \item
%     Fixes for plain-TeX, wrapper for package \xpackage{keyval} added.
%   \item
%     Catcode section rewritten.
%   \end{Version}
%   \begin{Version}{2007/10/28 v2.0}
%   \item
%     Collection support added (PDF 1.7).
%   \end{Version}
%   \begin{Version}{2007/10/29 v2.1}
%   \item
%     Export of object references by adding new option \xoption{id} and
%     new macros \cs{embedfileifobjectexists} and \cs{embedfilegetobject}.
%   \end{Version}
%   \begin{Version}{2007/11/11 v2.2}
%   \item
%     Use of package \xpackage{pdftexcmds} for \LuaTeX\ support.
%   \end{Version}
%   \begin{Version}{2007/11/25 v2.3}
%   \item
%     Fix in use of \cs{pdf@filesize}, bug introduced in previous version.
%   \end{Version}
%   \begin{Version}{2009/09/25 v2.4}
%   \item
%     Bug fix: If \xpackage{hyperref} is used with option \xoption{unicode},
%     the Unicode encoded file name causes trouble. Therefore \cs{pdfstringdef}
%     is now never used for option \xoption{filespec}, always method
%     \texttt{escape} is applied (Peter Cibulka).
%   \item
%     Bug fix for \xoption{initialfile}.
%   \item
%     Bug fix for file names in \texttt{/EmbeddedFiles}.
%   \item
%     New option \xoption{ucfilespec} for file name support in Unicode
%     (since PDF 1.7).
%   \end{Version}
%   \begin{Version}{2010/03/01 v2.5}
%   \item
%     Compatibility for \iniTeX.
%   \item
%     Package \xpackage{keyval} replaced by packages \xpackage{kvsetkeys}
%     and \xpackage{kvdefinekeys} because of compatibility for \iniTeX.
%   \item
%     TDS location moved from TDS:tex/latex/oberdiek/embedfile.sty to
%     TDS:tex/generic/oberdiek/embedfile.sty.
%   \end{Version}
%   \begin{Version}{2011/04/13 v2.6}
%   \item
%     Docu fixes (thanks Hans-Martin M\"unch).
%   \end{Version}
% \end{History}
%
% \PrintIndex
%
% \Finale
\endinput
|
% \end{quote}
% Do not forget to quote the argument according to the demands
% of your shell.
%
% \paragraph{Generating the documentation.}
% You can use both the \xfile{.dtx} or the \xfile{.drv} to generate
% the documentation. The process can be configured by the
% configuration file \xfile{ltxdoc.cfg}. For instance, put this
% line into this file, if you want to have A4 as paper format:
% \begin{quote}
%   \verb|\PassOptionsToClass{a4paper}{article}|
% \end{quote}
% An example follows how to generate the
% documentation with pdf\LaTeX:
% \begin{quote}
%\begin{verbatim}
%pdflatex embedfile.dtx
%makeindex -s gind.ist embedfile.idx
%pdflatex embedfile.dtx
%makeindex -s gind.ist embedfile.idx
%pdflatex embedfile.dtx
%\end{verbatim}
% \end{quote}
%
% \section{Catalogue}
%
% The following XML file can be used as source for the
% \href{http://mirror.ctan.org/help/Catalogue/catalogue.html}{\TeX\ Catalogue}.
% The elements \texttt{caption} and \texttt{description} are imported
% from the original XML file from the Catalogue.
% The name of the XML file in the Catalogue is \xfile{embedfile.xml}.
%    \begin{macrocode}
%<*catalogue>
<?xml version='1.0' encoding='us-ascii'?>
<!DOCTYPE entry SYSTEM 'catalogue.dtd'>
<entry datestamp='$Date$' modifier='$Author$' id='embedfile'>
  <name>embedfile</name>
  <caption>Embed files into PDF.</caption>
  <authorref id='auth:oberdiek'/>
  <copyright owner='Heiko Oberdiek' year='2006-2011'/>
  <license type='lppl1.3'/>
  <version number='2.6'/>
  <description>
    This package embeds files in a PDF document, using the PDF
    format's embedding operation (note the contrast with the attach
    operation used by the <xref refid='attachfile'>attachfile</xref>
    and <xref refid='attachfile2'>attachfile2</xref> packages).
    Currently only <xref refid='pdftex'>pdfTeX</xref> &gt;=1.30, in
    PDF mode, is supported.

    <p/>
    The package is part of the <xref refid='oberdiek'>oberdiek</xref>
    bundle.
  </description>
  <documentation details='Package documentation'
      href='ctan:/macros/latex/contrib/oberdiek/embedfile.pdf'/>
  <ctan file='true' path='/macros/latex/contrib/oberdiek/embedfile.dtx'/>
  <miktex location='oberdiek'/>
  <texlive location='oberdiek'/>
  <install path='/macros/latex/contrib/oberdiek/oberdiek.tds.zip'/>
</entry>
%</catalogue>
%    \end{macrocode}
%
% \begin{thebibliography}{9}
% \bibitem{attachfile}
%   Scott Pakin: \textit{The \xpackage{attachfile} package};
%   2005/02/20 v1.2;
%   \CTAN{macros/latex/contrib/attachfile/}.
%
% \bibitem{attachfile2}
%   Heiko Oberdiek: \textit{The \xpackage{attachfile2} package};
%   2006/08/16 v2.2;
%   \CTAN{macros/latex/contrib/oberdiek/attachfile2.pdf}.
%
% \bibitem{pdfspec}
%   Adobe Systems Incorporated:
%   \href{http://www.adobe.com/devnet/acrobat/pdfs/pdf_reference.pdf}%
%       {\textit{PDF Reference, Sixth Edition, Version 1.7}},%
%   Oktober 2006;
%   \url{http://www.adobe.com/devnet/pdf/pdf_reference.html}.
%
% \bibitem{mime}
%   Network Working Group: RFC 2046, \textit{Multipurpose Internet Mail Extensions (MIME) Part Two:
%     Media Types}, November 1996; \url{http://www.rfc-editor.org/}.
%
% \bibitem{types}
%   IANA (Internet Assigned Numbers Authority):
%   \textit{MIME Media Types}, May 2006;
%   \url{http://www.iana.org/assignments/media-types/}.
%
% \end{thebibliography}
%
% \begin{History}
%   \begin{Version}{2006/08/16 v1.0}
%   \item
%     First public version.
%   \end{Version}
%   \begin{Version}{2007/04/11 v1.1}
%   \item
%     Line ends sanitized.
%   \end{Version}
%   \begin{Version}{2007/09/09 v1.2}
%   \item
%     Fixes for plain-TeX, wrapper for package \xpackage{keyval} added.
%   \item
%     Catcode section rewritten.
%   \end{Version}
%   \begin{Version}{2007/10/28 v2.0}
%   \item
%     Collection support added (PDF 1.7).
%   \end{Version}
%   \begin{Version}{2007/10/29 v2.1}
%   \item
%     Export of object references by adding new option \xoption{id} and
%     new macros \cs{embedfileifobjectexists} and \cs{embedfilegetobject}.
%   \end{Version}
%   \begin{Version}{2007/11/11 v2.2}
%   \item
%     Use of package \xpackage{pdftexcmds} for \LuaTeX\ support.
%   \end{Version}
%   \begin{Version}{2007/11/25 v2.3}
%   \item
%     Fix in use of \cs{pdf@filesize}, bug introduced in previous version.
%   \end{Version}
%   \begin{Version}{2009/09/25 v2.4}
%   \item
%     Bug fix: If \xpackage{hyperref} is used with option \xoption{unicode},
%     the Unicode encoded file name causes trouble. Therefore \cs{pdfstringdef}
%     is now never used for option \xoption{filespec}, always method
%     \texttt{escape} is applied (Peter Cibulka).
%   \item
%     Bug fix for \xoption{initialfile}.
%   \item
%     Bug fix for file names in \texttt{/EmbeddedFiles}.
%   \item
%     New option \xoption{ucfilespec} for file name support in Unicode
%     (since PDF 1.7).
%   \end{Version}
%   \begin{Version}{2010/03/01 v2.5}
%   \item
%     Compatibility for \iniTeX.
%   \item
%     Package \xpackage{keyval} replaced by packages \xpackage{kvsetkeys}
%     and \xpackage{kvdefinekeys} because of compatibility for \iniTeX.
%   \item
%     TDS location moved from TDS:tex/latex/oberdiek/embedfile.sty to
%     TDS:tex/generic/oberdiek/embedfile.sty.
%   \end{Version}
%   \begin{Version}{2011/04/13 v2.6}
%   \item
%     Docu fixes (thanks Hans-Martin M\"unch).
%   \end{Version}
% \end{History}
%
% \PrintIndex
%
% \Finale
\endinput

%        (quote the arguments according to the demands of your shell)
%
% Documentation:
%    (a) If embedfile.drv is present:
%           latex embedfile.drv
%    (b) Without embedfile.drv:
%           latex embedfile.dtx; ...
%    The class ltxdoc loads the configuration file ltxdoc.cfg
%    if available. Here you can specify further options, e.g.
%    use A4 as paper format:
%       \PassOptionsToClass{a4paper}{article}
%
%    Programm calls to get the documentation (example):
%       pdflatex embedfile.dtx
%       makeindex -s gind.ist embedfile.idx
%       pdflatex embedfile.dtx
%       makeindex -s gind.ist embedfile.idx
%       pdflatex embedfile.dtx
%
% Installation:
%    TDS:tex/generic/oberdiek/embedfile.sty
%    TDS:tex/generic/oberdiek/dtx-attach.sty
%    TDS:doc/latex/oberdiek/embedfile.pdf
%    TDS:doc/latex/oberdiek/embedfile-example-plain.tex
%    TDS:doc/latex/oberdiek/embedfile-example-collection.tex
%    TDS:doc/latex/oberdiek/test/embedfile-test1.tex
%    TDS:doc/latex/oberdiek/test/embedfile-test2.tex
%    TDS:doc/latex/oberdiek/test/embedfile-test3.tex
%    TDS:doc/latex/oberdiek/test/embedfile-test4.tex
%    TDS:source/latex/oberdiek/embedfile.dtx
%
%<*ignore>
\begingroup
  \catcode123=1 %
  \catcode125=2 %
  \def\x{LaTeX2e}%
\expandafter\endgroup
\ifcase 0\ifx\install y1\fi\expandafter
         \ifx\csname processbatchFile\endcsname\relax\else1\fi
         \ifx\fmtname\x\else 1\fi\relax
\else\csname fi\endcsname
%</ignore>
%<*install>
\input docstrip.tex
\Msg{************************************************************************}
\Msg{* Installation}
\Msg{* Package: embedfile 2011/04/13 v2.6 Embed files into PDF (HO)}
\Msg{************************************************************************}

\keepsilent
\askforoverwritefalse

\let\MetaPrefix\relax
\preamble

This is a generated file.

Project: embedfile
Version: 2011/04/13 v2.6

Copyright (C) 2006-2011 by
   Heiko Oberdiek <heiko.oberdiek at googlemail.com>

This work may be distributed and/or modified under the
conditions of the LaTeX Project Public License, either
version 1.3c of this license or (at your option) any later
version. This version of this license is in
   http://www.latex-project.org/lppl/lppl-1-3c.txt
and the latest version of this license is in
   http://www.latex-project.org/lppl.txt
and version 1.3 or later is part of all distributions of
LaTeX version 2005/12/01 or later.

This work has the LPPL maintenance status "maintained".

This Current Maintainer of this work is Heiko Oberdiek.

The Base Interpreter refers to any `TeX-Format',
because some files are installed in TDS:tex/generic//.

This work consists of the main source file embedfile.dtx
and the derived files
   embedfile.sty, embedfile.pdf, embedfile.ins, embedfile.drv,
   dtx-attach.sty, embedfile-example-plain.tex,
   embedfile-example-collection.tex, embedfile-test1.tex,
   embedfile-test2.tex, embedfile-test3.tex,
   embedfile-test4.tex.

\endpreamble
\let\MetaPrefix\DoubleperCent

\generate{%
  \file{embedfile.ins}{\from{embedfile.dtx}{install}}%
  \file{embedfile.drv}{\from{embedfile.dtx}{driver}}%
  \usedir{tex/generic/oberdiek}%
  \file{embedfile.sty}{\from{embedfile.dtx}{package}}%
  \usedir{tex/latex/oberdiek}%
  \file{dtx-attach.sty}{\from{embedfile.dtx}{dtxattach}}%
  \usedir{doc/latex/oberdiek}%
  \file{embedfile-example-plain.tex}{\from{embedfile.dtx}{exampleplain}}%
  \file{embedfile-example-collection.tex}{\from{embedfile.dtx}{examplecollection}}%
  \usedir{doc/latex/oberdiek/test}%
  \file{embedfile-test1.tex}{\from{embedfile.dtx}{test1}}%
  \file{embedfile-test2.tex}{\from{embedfile.dtx}{test2}}%
  \file{embedfile-test3.tex}{\from{embedfile.dtx}{test3}}%
  \file{embedfile-test4.tex}{\from{embedfile.dtx}{test4}}%
  \nopreamble
  \nopostamble
  \usedir{source/latex/oberdiek/catalogue}%
  \file{embedfile.xml}{\from{embedfile.dtx}{catalogue}}%
}

\catcode32=13\relax% active space
\let =\space%
\Msg{************************************************************************}
\Msg{*}
\Msg{* To finish the installation you have to move the following}
\Msg{* files into a directory searched by TeX:}
\Msg{*}
\Msg{*     embedfile.sty, dtx-attach.sty}
\Msg{*}
\Msg{* To produce the documentation run the file `embedfile.drv'}
\Msg{* through LaTeX.}
\Msg{*}
\Msg{* Happy TeXing!}
\Msg{*}
\Msg{************************************************************************}

\endbatchfile
%</install>
%<*ignore>
\fi
%</ignore>
%<*driver>
\NeedsTeXFormat{LaTeX2e}
\ProvidesFile{embedfile.drv}%
  [2011/04/13 v2.6 Embed files into PDF (HO)]%
\documentclass{ltxdoc}
\usepackage{holtxdoc}[2011/11/22]
\begin{document}
  \DocInput{embedfile.dtx}%
\end{document}
%</driver>
% \fi
%
% \CheckSum{1363}
%
% \CharacterTable
%  {Upper-case    \A\B\C\D\E\F\G\H\I\J\K\L\M\N\O\P\Q\R\S\T\U\V\W\X\Y\Z
%   Lower-case    \a\b\c\d\e\f\g\h\i\j\k\l\m\n\o\p\q\r\s\t\u\v\w\x\y\z
%   Digits        \0\1\2\3\4\5\6\7\8\9
%   Exclamation   \!     Double quote  \"     Hash (number) \#
%   Dollar        \$     Percent       \%     Ampersand     \&
%   Acute accent  \'     Left paren    \(     Right paren   \)
%   Asterisk      \*     Plus          \+     Comma         \,
%   Minus         \-     Point         \.     Solidus       \/
%   Colon         \:     Semicolon     \;     Less than     \<
%   Equals        \=     Greater than  \>     Question mark \?
%   Commercial at \@     Left bracket  \[     Backslash     \\
%   Right bracket \]     Circumflex    \^     Underscore    \_
%   Grave accent  \`     Left brace    \{     Vertical bar  \|
%   Right brace   \}     Tilde         \~}
%
% \GetFileInfo{embedfile.drv}
%
% \title{The \xpackage{embedfile} package}
% \date{2011/04/13 v2.6}
% \author{Heiko Oberdiek\\\xemail{heiko.oberdiek at googlemail.com}}
%
% \maketitle
%
% \begin{abstract}
% This package embeds files to a PDF document.
% Currently the only supported driver is \pdfTeX\ $>=$ 1.30 in PDF mode.
% \end{abstract}
%
% \tableofcontents
%
% \section{Documentation}
%
% \subsection{Introduction}
%
%    The PDF format (\cite{pdfspec}) allows the inclusion of files
%    inside the PDF document. The included files can be bound to an
%    annotation on a page. Or they can be recorded in a sorted
%    list of embedded files. The packages \xpackage{attachfile}
%    or \xpackage{attachfile2} follow the first approach, this
%    package uses the latter method.
%
% \subsubsection{Future development}
%
%    My dream is a large package that merges the features of
%    all these packages meantioned before:
%    \begin{itemize}
%    \item Files can be attached to a page.
%    \item Files can be attached to the document.
%    \item An easy user interface for simple, common tasks and
%          beginners.
%    \item An interface for the advanced users that want to setup
%          every detail.
%    \item Support of many drivers (pdftex, dvips, dvipdfm, \dots).
%    \item \dots
%    \end{itemize}
%    However, I have not managed to take the time for this project.
%    Instead:
%    \begin{itemize}
%    \item First I experimented with package \xpackage{attachfile},
%          adding driver support, fixing bugs, \dots. The result is
%          currently named as \xpackage{attachfile2}. It uses an external
%          script to get file properties (size, date, checksum, \dots).
%    \item In order to avoid an external program for getting basic
%          file properties I provided a patch ``EscapeAndOther'' for
%          pdfTeX that was accepted for version 1.30.
%    \item This package closes a gap left by the packages for attaching
%          files and allows the embedding of files to the document.
%          Also it makes use of the new primitives of \pdfTeX.
%    \end{itemize}
%
% \subsection{User interface}
%
%    This package \xpackage{embedfile} can be used with both
%    \LaTeX\ and \plainTeX. See \autoref{sec:plain} that
%    explains the use with \plainTeX\ by an example.
%    In \LaTeX\ the package is loaded as usually. There are no options.
%    \begin{quote}
%      \verb|\usepackage{embedfile}|
%    \end{quote}
%
%    \begin{declcs}{embedfile}
%      |[|\meta{options}|]| \M{file}
%    \end{declcs}
%    The macro \cs{embedfile} includes file \meta{file} and
%    attaches it to the PDF document. At the end of the document
%    the sorted list of embedded files are written. Thus you can
%    safely use \cs{embedfile} before \verb|\end{document}|.
%    Embedding files using \cs{AtEndDocument} will only work,
%    if \cs{AtEndDocument} is called before loading the package
%    \xpackage{embedfile}.
%
%    The \meta{options} are give as key value pairs. The following
%    keys are supported:
%    \begin{description}
%    \item[\xoption{filespec}]
%      This allows to override the file name
%      that appears in the PDF file. If you are using other than
%      simple file names (8-bit, path separators, \dots), look into
%      the PDF specification (\cite{pdfspec}).
%      There are rules how these file names must be written/encoded.
%      Avoid 8-bit characters
%      and other special characters, the behaviour is currently undefined.
%      Use option \xoption{ucfilespec} for more funny file names.
%      The string method, see below, is |escape| since version 2.4.
%
%      This name is also used as entry in a name tree
%      (see PDF specification: |/EmbeddedFiles|). Therefore the
%      value for \xoption{filespec} must be unique among all
%      embedded files. Also key \xoption{initialfiles} refers
%      to this name, if the file name and the value of \xoption{filespec}
%      are different.
%    \item[\xoption{ucfilespec}]
%      Since PDF 1.7 the file name may be provided in Unicode.
%      The conversion of the option value into a PDF string
%      is controlled by option \xoption{stringmethod}.
%    \item[\xoption{filesystem}]
%      This sets the entry \verb|/FS| in
%      the file specification dictionary, see PDF specification
%      (\cite{pdfspec}).
%      Example: \verb|filesystem=URL|.
%    \item[\xoption{mimetype}]
%      This sets the mime type (\cite{mime}) of the file,
%      see \autoref{sec:plain} for examples and \cite{types}
%      for a list of officially registered types.
%    \item[\xoption{desc}]
%      The description for the file.
%    \item[\xoption{stringmethod}]
%      The package must convert the values of the keys \xoption{ucfilespec}
%      and \xoption{desc} into a PDF string (before version 2.4: \xoption{filespec}
%      and \xoption{desc}). If \xpackage{hyperref}
%      is found, then its \cs{pdfstringdef} will be used, otherwise
%      \pdfTeX's \cs{pdfescapestring} is used. Value \verb|psd|
%      forces the use of \cs{pdfstringdef}, value \verb|escape|
%      the use of \cs{pdfescapestring}.
%    \item[\xoption{\meta{key}.value}] Sets the value of a collection
%      item property, see section \ref{sec:collection}.
%    \item[\xoption{\meta{key}.prefix}]
%      Sets the prefix of a collection item property,
%      see section \ref{sec:collection}.
%    \item[\xoption{id}]
%      The value must be an unique name. Macros
%      \cs{embedfileifobjectexists} and \cs{embedfilegetobject}
%      are using this name later.
%    \end{description}
%
%    \begin{declcs}{embedfilefinish}
%    \end{declcs}
%    The list of all embedded files must be added as data structure
%    in the PDF file. In case of \LaTeX\ this is automatically done.
%    The package uses \cs{AtEndDocument}. Then the list of all
%    files should be known. However, \plainTeX\ does not know
%    about \cs{AtEndDocument}. Thus the user must call
%    \cs{embedfilefinish} at the end of the document
%    after the last file is embedded.
%
%    \begin{declcs}{embedfilesetup} \M{options}
%    \end{declcs}
%    Options for \cs{embedfile} and collection support
%    can be set in \cs{embedfilesetup}.
%
% \subsection{Collection support (PDF 1.7)}
% \label{sec:collection}
%
%    Since PDF 1.7 the embedded files can form a \emph{collection}
%    (sometimes referred as \emph{package}), the main document
%    is called \emph{cover sheet}. See PDF specification
%    8.2.4 ``Collections'' and 3.10.5 ``Collection items'' \cite{pdfspec}.
%
%    Usually Acrobat Reader 7 or 8 shows the
%    embedded files in a table at the bottom with the following columns:
%    \begin{quote}
%    \begin{tabular}{@{}l|l|l|l@{}}
%       Name & Description & Modified & Size\\
%       \hline
%       \dots & \dots & \dots & \dots\\
%    \end{tabular}
%    \end{quote}
%    Acrobat Reader 10 shows the embeded files in the left panel
%    and adds a new column for the compressed size.
%
%    If the files form a collection, then they are displayed
%    in a table left or top (depending on option \xoption{view},
%    see \cs{embedfilesetup}).
%
%
%    Collection support is enabled automatically, if
%    it is used.
%
%    \begin{declcs}{embedfilesetup} \M{options}
%    \end{declcs}
%    The following options are supported in addition
%    to options for \cs{embedfile}:
%    \begin{description}
%    \item[\xoption{view}]
%      If the PDF file contains a collection, then Acrobat Reader 8
%      shows a line at the top below the menu bar and the toolbar.
%      It shows the current selected file, icons for changing the
%      view mode, an options menu.
%      The initial mode how the collection is presented is
%      set by this option \xoption{view}.
%      The following modes/values are supported, the default is |details|:
%        \begin{description}
%        \item[\texttt{details}] The full collection table is displayed
%          at the top below the collection bar.
%        \item[\texttt{tile}] The files of the collection are shown
%          in tile mode on the left.
%        \item[\texttt{hidden}] The collection table is not shown.
%        \end{description}
%    \item[\xoption{initialfile}]
%      Selects the file that is initially presented. Especially
%      useful for an embedded PDF file that is then shown instead
%      of the cover document. There must be an \cs{embedfile} command
%      somewhere whose value for key \xoption{filespec} is used here.
%      The \cs{embedfile} command can drop option \xoption{filespec}
%      if the file name is not different.
%    \end{description}
%
%    \begin{declcs}{embedfilefield} \M{key} \M{options}
%    \end{declcs}
%    Macro \cs{embedfilefield} defines a column/field in the collection table.
%    The name of the field is \meta{key}.
%    \begin{description}
%    \item[\xoption{type}]
%      sets the type of the field. The supported values are:
%      \begin{description}
%      \item[\texttt{text}] A text field. Its value is set in
%        \cs{embedfile} by option \xoption{\meta{key}.value}.
%      \item[\texttt{date}] A date field. Its value is set in
%        \cs{embedfile} by option \xoption{\meta{key}.value}.
%        A special format is required, see ``3.8.3 Dates'' \cite{pdfspec}.
%      \item[\texttt{number}] A field with an integer or float number.
%        Its value is set in \cs{embedfile} by option
%        \xoption{\meta{key}.value}.
%      \item[\texttt{file}] The file name of the embedded file.
%      \item[\texttt{desc}] The description text of the embedded file. It is
%       set in \cs{embedfile} by option \xoption{desc}.
%      \item[\texttt{moddate}] The modification date of the embedded file.
%      \item[\texttt{size}] The size of the embedded file.
%      \end{description}
%      All types allow the use of a prefix that is disregarded by
%      sorting. The prefix for this field is set in \cs{embedfile} by
%      option \xoption{\meta{key}.prefix}.
%    \item[\xoption{title}] sets the column title.
%    \item[\xoption{visible}] controls whether the column is presented:
%      \begin{description}
%      \item[\texttt{true}] shows the column.
%      \item[\texttt{false}] hides the column.
%      \end{description}
%      Default: \texttt{true}
%    \item[\xoption{edit}]
%      Allows the editing of field values. Does not seem to have an
%      effect for Acrobat Reader.
%      \begin{description}
%      \item[\texttt{true}] enables the feature, if available (depends
%        on the PDF viewer).
%      \item[\texttt{false}] disables the feature.
%      \end{description}
%      Default: \texttt{false}
%    \end{description}
%    The order of \cs{embedfilefield} statements defines the order
%    of the columns.
%
%    \begin{declcs}{embedfilesort} \M{key-sort-list}
%    \end{declcs}
%    The sort order of the embedded files are controlled by
%    macro \cs{embedfilesort}. \meta{key-sort-list} defines
%    the sort order. The key is a field name defined by \cs{embedfilefield}.
%    Its value is either \texttt{ascending} or \texttt{descending}.
%    The default is \texttt{ascending}.
%
% \subsection{Export of object references}
%
%    Caution: This feature is still experimental. It may be even removed
%    in future versions. Therefore feedback would be nice, if someone
%    has a useful application for this feature.
%
%    Object numbers are saved, if \xoption{id} is given in \cs{embedfile}.
%    The following objects are supported:
%    \begin{itemize}
%    \item \texttt{EmbeddedFile}
%    \item \texttt{Filespec}
%    \end{itemize}
%
%    \begin{declcs}{embedfileifobjectexists} \M{id} \M{type}
%      \M{then} \M{else}
%    \end{declcs}
%    Macro \cs{embedfileifobjectexists} tests whether object of \meta{type}
%    is available for the embedded file identified by \meta{id}.
%
%    \begin{declcs}{embedfilegetobject} \M{id} \M{type}
%    \end{declcs}
%    Macro \cs{embedfilegetobject} expands to the full object reference
%    object of \meta{type} for the embedded file identified by \meta{id}.
%
% \subsubsection{Example}
%
%\begin{quote}
%\begin{verbatim}
%\embedfile[id={foo}]{foo.pdf}
%\embedfileifobjectexists{foo}{Filespec}{%
%  \typeout{%
%    FileSpec object for `foo': %
%    \embedfilegetobject{foo}{Filespec}%
%  }%
%}{%
%  \typeout{No Filespec object for `foo'}%
%}
%\end{verbatim}
%\end{quote}
%
% \subsection{Examples}
% \label{sec:examples}
%
% \subsubsection{\plainTeX}
% \label{sec:plain}
%
%    The package can be used with \plainTeX. It can be used
%    with or without help from \xfile{miniltx.tex}.
%
%    If additionally package \xpackage{keyval} (\xpackage{graphicx})
%    is needed, load it first. Then package \xpackage{embedfile} avoids
%    a duplicate loading of package \xpackage{keyval}.
%
%    Because \plainTeX\ does not provide a hook at end of the document,
%    you have to call \cs{embedfilefinish} manually at the end
%    after the last embedded file.
%    \begin{macrocode}
%<*exampleplain>
%<<END
% Load packages
\input miniltx
% \def\Gin@driver{pdftex.def}
% \input graphicx.sty
\input embedfile.sty
\resetatcatcode

% default setting
\embedfilesetup{
  mimetype=text/plain
}

% Embed files
\embedfile[
  filespec=example.tex,
  desc={Source code (plain-TeX) of this example}
]{embedfile-example-plain.tex}

\embedfile[
  desc={Source of package `embedfile'}
]{embedfile.dtx}

\embedfile[
  mimetype=application/pdf,
  desc={Documentation of package `embedfile'}
]{embedfile.pdf}

% Some text
This example document contains three embedded files.

% End of document
\embedfilefinish % don't forget
\bye
%END
%</exampleplain>
%    \end{macrocode}
%
% \subsubsection{Collection example}
%
%    \begin{macrocode}
%<*examplecollection>
%<<END
\NeedsTeXFormat{LaTeX2e}
\documentclass{article}
\usepackage[bookmarks=false]{hyperref}
% provides \pdfstringdef that is then used by `title' and
% other keys.
\usepackage{embedfile}[2011/04/13]
\embedfilesetup{
  view=details,
  initialfile=embedfile.pdf
}
\embedfilefield{file}{
  type=file,
  title={File name}
}
\embedfilefield{description}{
  type=desc,
  title={Description}
}
\embedfilefield{date}{
  type=moddate,
  title={Date}
}
\embedfilefield{size}{
  type=size,
  title={Size}
}
\embedfilefield{type}{
  type=text,
  title={Type},
  visible=false
}
\embedfilesort{
  type,
  date=descending
}
\begin{document}
An example for embedded files as collection.
You need Acrobat Reader 8 or higher.

\embedfile[
  desc={Source file of package `embedfile'},
  description.prefix={Package: },
  type.value={DTX}
]{embedfile.dtx}

\embedfile[
  desc={Documentation of package `embedfile'},
  description.prefix={Package: },
  type.value={PDF}
]{embedfile.pdf}

\embedfile[
  desc={The source for this example},
  description.prefix={Example: },
  type.value={TEX}
]{\jobname.tex}

\end{document}
%END
%</examplecollection>
%    \end{macrocode}
%
% \subsection{Package \xpackage{dtx-attach}}
%
% Package \xpackage{dtx-attach} is just a small application of
% package \xpackage{embedfile}. I am using it for the CTAN
% documentation of my packages in \CTAN{macros/latex/contrib/oberdiek/}.
% It also serves as small example for the use of the package with \LaTeX.
%
%    \begin{macrocode}
%<*dtxattach>
\NeedsTeXFormat{LaTeX2e}
\ProvidesPackage{dtx-attach}
  [2011/04/13 v2.6 Embed \string\jobname.dtx (HO)]%
\RequirePackage{embedfile}[2011/04/13]
\embedfile[%
  stringmethod=escape,%
  mimetype=plain/text,%
  desc={LaTeX docstrip source archive for package `\jobname'}%
]{\jobname.dtx}
%</dtxattach>
%    \end{macrocode}
%
% \StopEventually{
% }
%
% \section{Implementation}
%
%    \begin{macrocode}
%<*package>
%    \end{macrocode}
%
% \subsection{Reload check and package identification}
%    Reload check, especially if the package is not used with \LaTeX.
%    \begin{macrocode}
\begingroup\catcode61\catcode48\catcode32=10\relax%
  \catcode13=5 % ^^M
  \endlinechar=13 %
  \catcode35=6 % #
  \catcode39=12 % '
  \catcode44=12 % ,
  \catcode45=12 % -
  \catcode46=12 % .
  \catcode58=12 % :
  \catcode64=11 % @
  \catcode123=1 % {
  \catcode125=2 % }
  \expandafter\let\expandafter\x\csname ver@embedfile.sty\endcsname
  \ifx\x\relax % plain-TeX, first loading
  \else
    \def\empty{}%
    \ifx\x\empty % LaTeX, first loading,
      % variable is initialized, but \ProvidesPackage not yet seen
    \else
      \expandafter\ifx\csname PackageInfo\endcsname\relax
        \def\x#1#2{%
          \immediate\write-1{Package #1 Info: #2.}%
        }%
      \else
        \def\x#1#2{\PackageInfo{#1}{#2, stopped}}%
      \fi
      \x{embedfile}{The package is already loaded}%
      \aftergroup\endinput
    \fi
  \fi
\endgroup%
%    \end{macrocode}
%    Package identification:
%    \begin{macrocode}
\begingroup\catcode61\catcode48\catcode32=10\relax%
  \catcode13=5 % ^^M
  \endlinechar=13 %
  \catcode35=6 % #
  \catcode39=12 % '
  \catcode40=12 % (
  \catcode41=12 % )
  \catcode44=12 % ,
  \catcode45=12 % -
  \catcode46=12 % .
  \catcode47=12 % /
  \catcode58=12 % :
  \catcode64=11 % @
  \catcode91=12 % [
  \catcode93=12 % ]
  \catcode123=1 % {
  \catcode125=2 % }
  \expandafter\ifx\csname ProvidesPackage\endcsname\relax
    \def\x#1#2#3[#4]{\endgroup
      \immediate\write-1{Package: #3 #4}%
      \xdef#1{#4}%
    }%
  \else
    \def\x#1#2[#3]{\endgroup
      #2[{#3}]%
      \ifx#1\@undefined
        \xdef#1{#3}%
      \fi
      \ifx#1\relax
        \xdef#1{#3}%
      \fi
    }%
  \fi
\expandafter\x\csname ver@embedfile.sty\endcsname
\ProvidesPackage{embedfile}%
  [2011/04/13 v2.6 Embed files into PDF (HO)]%
%    \end{macrocode}
%
% \subsection{Catcodes}
%
%    \begin{macrocode}
\begingroup\catcode61\catcode48\catcode32=10\relax%
  \catcode13=5 % ^^M
  \endlinechar=13 %
  \catcode123=1 % {
  \catcode125=2 % }
  \catcode64=11 % @
  \def\x{\endgroup
    \expandafter\edef\csname EmFi@AtEnd\endcsname{%
      \endlinechar=\the\endlinechar\relax
      \catcode13=\the\catcode13\relax
      \catcode32=\the\catcode32\relax
      \catcode35=\the\catcode35\relax
      \catcode61=\the\catcode61\relax
      \catcode64=\the\catcode64\relax
      \catcode123=\the\catcode123\relax
      \catcode125=\the\catcode125\relax
    }%
  }%
\x\catcode61\catcode48\catcode32=10\relax%
\catcode13=5 % ^^M
\endlinechar=13 %
\catcode35=6 % #
\catcode64=11 % @
\catcode123=1 % {
\catcode125=2 % }
\def\TMP@EnsureCode#1#2{%
  \edef\EmFi@AtEnd{%
    \EmFi@AtEnd
    \catcode#1=\the\catcode#1\relax
  }%
  \catcode#1=#2\relax
}
\TMP@EnsureCode{39}{12}% '
\TMP@EnsureCode{40}{12}% (
\TMP@EnsureCode{41}{12}% )
\TMP@EnsureCode{44}{12}% ,
\TMP@EnsureCode{46}{12}% .
\TMP@EnsureCode{47}{12}% /
\TMP@EnsureCode{58}{12}% :
\TMP@EnsureCode{60}{12}% <
\TMP@EnsureCode{62}{12}% >
\TMP@EnsureCode{91}{12}% [
\TMP@EnsureCode{93}{12}% ]
\TMP@EnsureCode{96}{12}% `
\edef\EmFi@AtEnd{\EmFi@AtEnd\noexpand\endinput}
%    \end{macrocode}
%
% \subsection{Tools}
%
%    \begin{macro}{\EmFi@RequirePackage}
%    \begin{macrocode}
\begingroup\expandafter\expandafter\expandafter\endgroup
\expandafter\ifx\csname RequirePackage\endcsname\relax
  \def\EmFi@RequirePackage#1[#2]{%
    \input #1.sty\relax
  }%
\else
  \let\EmFi@RequirePackage\RequirePackage
\fi
%    \end{macrocode}
%    \end{macro}
%
%    \begin{macro}{\EmFi@Error}
%    \begin{macrocode}
\EmFi@RequirePackage{infwarerr}[2007/09/09]%
\def\EmFi@Error{%
  \@PackageError{embedfile}%
}
%    \end{macrocode}
%    \end{macro}
%
% \subsection{Check for recent \pdfTeX\ in PDF mode}
%
%    Load package \xpackage{ifpdf} and check mode.
%    \begin{macrocode}
\EmFi@RequirePackage{ifpdf}[2007/09/09]
\ifpdf
\else
  \EmFi@Error{%
    Missing pdfTeX in PDF mode%
  }{%
    Currently other drivers are not supported. %
    Package loading is aborted.%
  }%
  \expandafter\EmFi@AtEnd
\fi%
%    \end{macrocode}
%    \begin{macrocode}
\EmFi@RequirePackage{pdftexcmds}[2007/11/11]
\EmFi@RequirePackage{ltxcmds}[2010/03/01]
\EmFi@RequirePackage{kvsetkeys}[2010/03/01]
\EmFi@RequirePackage{kvdefinekeys}[2010/03/01]
%    \end{macrocode}
%    Check version.
%    \begin{macrocode}
\begingroup\expandafter\expandafter\expandafter\endgroup
\expandafter\ifx\csname pdf@filesize\endcsname\relax
  \EmFi@Error{%
    Unsupported pdfTeX version%
  }{%
    At least version 1.30 is necessary. Package loading is aborted.%
  }%
  \expandafter\EmFi@AtEnd
\fi%
%    \end{macrocode}
%
% \subsection{Strings}
%
%    Minimal version of package \xpackage{pdfescape} is 2007/08/27 v1.5
%    because of \cs{EdefSanitize}.
%    \begin{macrocode}
\EmFi@RequirePackage{pdfescape}[2007/11/11]
%    \end{macrocode}
%
%    \begin{macrocode}
\def\EmFi@temp#1{%
  \expandafter\EdefSanitize\csname EmFi@S@#1\endcsname{#1}%
}
%    \end{macrocode}
%    \begin{macro}{\EmFi@details}
%    \begin{macrocode}
\EmFi@temp{details}%
%    \end{macrocode}
%    \end{macro}
%    \begin{macro}{\EmFi@tile}
%    \begin{macrocode}
\EmFi@temp{tile}%
%    \end{macrocode}
%    \end{macro}
%    \begin{macro}{\EmFi@hidden}
%    \begin{macrocode}
\EmFi@temp{hidden}%
%    \end{macrocode}
%    \end{macro}
%    \begin{macro}{\EmFi@S@text}
%    \begin{macrocode}
\EmFi@temp{text}
%    \end{macrocode}
%    \end{macro}
%    \begin{macro}{\EmFi@S@date}
%    \begin{macrocode}
\EmFi@temp{date}
%    \end{macrocode}
%    \end{macro}
%    \begin{macro}{\EmFi@S@number}
%    \begin{macrocode}
\EmFi@temp{number}
%    \end{macrocode}
%    \end{macro}
%    \begin{macro}{\EmFi@S@file}
%    \begin{macrocode}
\EmFi@temp{file}
%    \end{macrocode}
%    \end{macro}
%    \begin{macro}{\EmFi@S@desc}
%    \begin{macrocode}
\EmFi@temp{desc}
%    \end{macrocode}
%    \end{macro}
%    \begin{macro}{\EmFi@S@moddate}
%    \begin{macrocode}
\EmFi@temp{moddate}
%    \end{macrocode}
%    \end{macro}
%    \begin{macro}{\EmFi@S@creationdate}
%    \begin{macrocode}
\EmFi@temp{creationdate}
%    \end{macrocode}
%    \end{macro}
%    \begin{macro}{\EmFi@S@size}
%    \begin{macrocode}
\EmFi@temp{size}
%    \end{macrocode}
%    \end{macro}
%    \begin{macro}{\EmFi@S@ascending}
%    \begin{macrocode}
\EmFi@temp{ascending}
%    \end{macrocode}
%    \end{macro}
%    \begin{macro}{\EmFi@S@descending}
%    \begin{macrocode}
\EmFi@temp{descending}
%    \end{macrocode}
%    \end{macro}
%    \begin{macro}{\EmFi@S@true}
%    \begin{macrocode}
\EmFi@temp{true}
%    \end{macrocode}
%    \end{macro}
%    \begin{macro}{\EmFi@S@false}
%    \begin{macrocode}
\EmFi@temp{false}
%    \end{macrocode}
%    \end{macro}
%
% \subsection{Switches}
%
%    \begin{macro}{\ifEmFi@collection}
%    \begin{macrocode}
\ltx@newif\ifEmFi@collection
%    \end{macrocode}
%    \end{macro}
%    \begin{macro}{\ifEmFi@sort}
%    \begin{macrocode}
\ltx@newif\ifEmFi@sort
%    \end{macrocode}
%    \end{macro}
%    \begin{macro}{\ifEmFi@visible}
%    \begin{macrocode}
\ltx@newif\ifEmFi@visible
%    \end{macrocode}
%    \end{macro}
%    \begin{macro}{\ifEmFi@edit}
%    \begin{macrocode}
\ltx@newif\ifEmFi@edit
%    \end{macrocode}
%    \end{macro}
%    \begin{macro}{\ifEmFi@item}
%    \begin{macrocode}
\ltx@newif\ifEmFi@item
%    \end{macrocode}
%    \end{macro}
%    \begin{macro}{\ifEmFi@finished}
%    \begin{macrocode}
\ltx@newif\ifEmFi@finished
%    \end{macrocode}
%    \end{macro}
%    \begin{macro}{\ifEmFi@id}
%    \begin{macrocode}
\ltx@newif\ifEmFi@id
%    \end{macrocode}
%    \end{macro}
%
% \subsection{Key value definitions}
%
%    \begin{macro}{\EmFi@GlobalKey}
%    \begin{macrocode}
\def\EmFi@GlobalKey#1#2{%
  \global\expandafter\let\csname KV@#1@#2\expandafter\endcsname
                         \csname KV@#1@#2\endcsname
}
%    \end{macrocode}
%    \end{macro}
%    \begin{macro}{\EmFi@GlobalDefaultKey}
%    \begin{macrocode}
\def\EmFi@GlobalDefaultKey#1#2{%
  \EmFi@GlobalKey{#1}{#2}%
  \global\expandafter\let
      \csname KV@#1@#2@default\expandafter\endcsname
      \csname KV@#1@#2@default\endcsname
}
%    \end{macrocode}
%    \end{macro}
%
%    \begin{macro}{\EmFi@DefineKey}
%    \begin{macrocode}
\def\EmFi@DefineKey#1#2{%
  \kv@define@key{EmFi}{#1}{%
    \expandafter\def\csname EmFi@#1\endcsname{##1}%
  }%
  \expandafter\def\csname EmFi@#1\endcsname{#2}%
}
%    \end{macrocode}
%    \end{macro}
%    Subtype of the embedded file (optional).
%    \begin{macrocode}
\EmFi@DefineKey{mimetype}{}
%    \end{macrocode}
%    File specification string.
%    \begin{macrocode}
\EmFi@DefineKey{filespec}{\EmFi@file}
%    \end{macrocode}
%    File specification string in Unicode.
%    \begin{macrocode}
\EmFi@DefineKey{ucfilespec}{}
%    \end{macrocode}
%    File system (optional).
%    \begin{macrocode}
\EmFi@DefineKey{filesystem}{}
%    \end{macrocode}
%    Description (optional).
%    \begin{macrocode}
\EmFi@DefineKey{desc}{}
%    \end{macrocode}
%    Method for converting text to PDF strings.
%    \begin{macrocode}
\EmFi@DefineKey{stringmethod}{%
  \ifx\pdfstringdef\@undefined
    escape%
  \else
    \ifx\pdfstringdef\relax
      escape%
    \else
      psd%
    \fi
  \fi
}
%    \end{macrocode}
%
%    Option \xoption{id} as key for object numbers.
%    \begin{macrocode}
\kv@define@key{EmFi}{id}{%
  \def\EmFi@id{#1}%
  \EmFi@idtrue
}
%    \end{macrocode}
%    \begin{macro}{\EmFi@defobj}
%    \begin{macrocode}
\def\EmFi@defobj#1{%
  \ifEmFi@id
    \expandafter\xdef\csname EmFi@#1@\EmFi@id\endcsname{%
      \the\pdflastobj\ltx@space 0 R%
    }%
  \fi
}
%    \end{macrocode}
%    \end{macro}
%    \begin{macro}{\embedfileifobjectexists}
%    \begin{macrocode}
\def\embedfileifobjectexists#1#2{%
  \expandafter\ifx\csname EmFi@#2@#1\endcsname\relax
    \expandafter\ltx@secondoftwo
  \else
    \expandafter\ltx@firstoftwo
  \fi
}
%    \end{macrocode}
%    \end{macro}
%    \begin{macro}{\embedfilegetobject}
%    \begin{macrocode}
\def\embedfilegetobject#1#2{%
  \embedfileifobjectexists{#1}{#2}{%
    \csname EmFi@#2@#1\endcsname
  }{%
    0 0 R%
  }%
}
%    \end{macrocode}
%    \end{macro}
%
%    Initial view of the collection.
%    \begin{macrocode}
\kv@define@key{EmFi}{view}[]{%
  \EdefSanitize\EmFi@temp{#1}%
  \def\EmFi@next{%
    \global\EmFi@collectiontrue
  }%
  \ifx\EmFi@temp\ltx@empty
    \let\EmFi@view\EmFi@S@details
  \else\ifx\EmFi@temp\EmFi@S@details
    \let\EmFi@view\EmFi@S@details
  \else\ifx\EmFi@temp\EmFi@S@tile
    \let\EmFi@view\EmFi@S@tile
  \else\ifx\EmFi@temp\EmFi@S@hidden
    \let\EmFi@view\EmFi@S@hidden
  \else
    \let\EmFi@next\relax
    \EmFi@Error{%
      Unknown value `\EmFi@temp' for key `view'.\MessageBreak
      Supported values: `details', `tile', `hidden'.%
    }\@ehc
  \fi\fi\fi\fi
  \EmFi@next
}
%    \end{macrocode}
%    \begin{macrocode}
\EmFi@DefineKey{initialfile}{}
%    \end{macrocode}
%
%    \begin{macro}{\embedfilesetup}
%    \begin{macrocode}
\def\embedfilesetup{%
  \ifEmFi@finished
    \def\EmFi@next##1{}%
    \EmFi@Error{%
      \string\embedfilefield\ltx@space after \string\embedfilefinish
    }{%
      The list of embedded files is already written.%
    }%
  \else
    \def\EmFi@next{%
      \kvsetkeys{EmFi}%
    }%
  \fi
  \EmFi@next
}
%    \end{macrocode}
%    \end{macro}
%
%    \begin{macro}{\EmFi@schema}
%    \begin{macrocode}
\def\EmFi@schema{}
%    \end{macrocode}
%    \end{macro}
%    \begin{macro}{\EmFi@order}
%    \begin{macrocode}
\gdef\EmFi@order{0}
%    \end{macrocode}
%    \end{macro}
%    \begin{macro}{\EmFi@@order}
%    \begin{macrocode}
\let\EmFi@@order\relax
%    \end{macrocode}
%    \end{macro}
%    \begin{macro}{\EmFi@fieldlist}
%    \begin{macrocode}
\def\EmFi@fieldlist{}
%    \end{macrocode}
%    \end{macro}
%    \begin{macro}{\EmFi@sortcase}
%    \begin{macrocode}
\def\EmFi@sortcase{0}%
%    \end{macrocode}
%    \end{macro}
%    \begin{macro}{\embedfilefield}
%    \begin{macrocode}
\def\embedfilefield#1#2{%
  \ifEmFi@finished
    \EmFi@Error{%
      \string\embedfilefield\ltx@space after \string\embedfilefinish
    }{%
      The list of embedded files is already written.%
    }%
  \else
    \global\EmFi@collectiontrue
    \EdefSanitize\EmFi@key{#1}%
    \expandafter\ifx\csname KV@EmFi@\EmFi@key.prefix\endcsname\relax
      \begingroup
        \count@=\EmFi@order
        \advance\count@ 1 %
        \xdef\EmFi@order{\the\count@}%
        \let\EmFi@title\EmFi@key
        \let\EmFi@type\EmFi@S@text
        \EmFi@visibletrue
        \EmFi@editfalse
        \kvsetkeys{EmFiFi}{#2}%
        \EmFi@convert\EmFi@title\EmFi@title
        \xdef\EmFi@schema{%
          \EmFi@schema
          /\pdf@escapename{\EmFi@key}<<%
            /Subtype/%
            \ifx\EmFi@type\EmFi@S@date D%
            \else\ifx\EmFi@type\EmFi@S@number N%
            \else\ifx\EmFi@type\EmFi@S@file F%
            \else\ifx\EmFi@type\EmFi@S@desc Desc%
            \else\ifx\EmFi@type\EmFi@S@moddate ModDate%
            \else\ifx\EmFi@type\EmFi@S@creationdate CreationDate%
            \else\ifx\EmFi@type\EmFi@S@size Size%
            \else S%
            \fi\fi\fi\fi\fi\fi\fi
            /N(\EmFi@title)%
            \EmFi@@order{\EmFi@order}%
            \ifEmFi@visible
            \else
              /V false%
            \fi
            \ifEmFi@edit
              /E true%
            \fi
          >>%
        }%
        \let\do\relax
        \xdef\EmFi@fieldlist{%
          \EmFi@fieldlist
          \do{\EmFi@key}%
        }%
        \ifx\EmFi@type\EmFi@S@text
          \kv@define@key{EmFi}{\EmFi@key.value}{%
            \EmFi@itemtrue
            \def\EmFi@temp{##1}%
            \EmFi@convert\EmFi@temp\EmFi@temp
            \expandafter\def\csname EmFi@V@#1%
            \expandafter\endcsname\expandafter{%
              \expandafter(\EmFi@temp)%
            }%
          }%
          \EmFi@GlobalKey{EmFi}{\EmFi@key.value}%
        \else\ifx\EmFi@type\EmFi@S@date
          \kv@define@key{EmFi}{\EmFi@key.value}{%
            \EmFi@itemtrue
            \def\EmFi@temp{##1}%
            \EmFi@convert\EmFi@temp\EmFi@temp
            \expandafter\def\csname EmFi@V@#1%
            \expandafter\endcsname\expandafter{%
              \expandafter(\EmFi@temp)%
            }%
          }%
          \EmFi@GlobalKey{EmFi}{\EmFi@key.value}%
        \else\ifx\EmFi@type\EmFi@S@number
          \kv@define@key{EmFi}{\EmFi@key.value}{%
            \EmFi@itemtrue
            \expandafter\EdefSanitize\csname EmFi@V@#1\endcsname{ ##1}%
          }%
          \EmFi@GlobalKey{EmFi}{\EmFi@key.value}%
        \fi\fi\fi
        \kv@define@key{EmFi}{\EmFi@key.prefix}{%
          \EmFi@itemtrue
          \expandafter\def\csname EmFi@P@#1\endcsname{##1}%
        }%
        \EmFi@GlobalKey{EmFi}{\EmFi@key.prefix}%
        \kv@define@key{EmFiSo}{\EmFi@key}[ascending]{%
          \EdefSanitize\EmFi@temp{##1}%
          \ifx\EmFi@temp\EmFi@S@ascending
            \def\EmFi@temp{true}%
          \else\ifx\EmFi@temp\EmFi@S@descending
            \def\EmFi@temp{false}%
          \else
            \def\EmFi@temp{}%
            \EmFi@Error{%
              Unknown sort order `\EmFi@temp'.\MessageBreak
              Supported values: `\EmFi@S@ascending', %
              `\EmFi@S@descending
            }\@ehc
          \fi\fi
          \ifx\EmFi@temp\ltx@empty
          \else
            \xdef\EmFi@sortkeys{%
              \EmFi@sortkeys
              /\pdf@escapename{#1}%
            }%
            \ifx\EmFi@sortorders\ltx@empty
              \global\let\EmFi@sortorders\EmFi@temp
              \gdef\EmFi@sortcase{1}%
            \else
              \xdef\EmFi@sortorders{%
                \EmFi@sortorders
                \ltx@space
                \EmFi@temp
              }%
              \xdef\EmFi@sortcase{2}%
            \fi
          \fi
        }%
        \EmFi@GlobalDefaultKey{EmFiSo}\EmFi@key
      \endgroup
    \else
      \EmFi@Error{%
        Field `\EmFi@key' is already defined%
      }\@ehc
    \fi
  \fi
}
%    \end{macrocode}
%    \end{macro}
%    \begin{macrocode}
\kv@define@key{EmFiFi}{type}{%
  \EdefSanitize\EmFi@temp{#1}%
  \ifx\EmFi@temp\EmFi@S@text
    \let\EmFi@type\EmFi@temp
  \else\ifx\EmFi@temp\EmFi@S@date
    \let\EmFi@type\EmFi@temp
  \else\ifx\EmFi@temp\EmFi@S@number
    \let\EmFi@type\EmFi@temp
  \else\ifx\EmFi@temp\EmFi@S@file
    \let\EmFi@type\EmFi@temp
  \else\ifx\EmFi@temp\EmFi@S@desc
    \let\EmFi@type\EmFi@temp
  \else\ifx\EmFi@temp\EmFi@S@moddate
    \let\EmFi@type\EmFi@temp
  \else\ifx\EmFi@temp\EmFi@S@creationdate
    \let\EmFi@type\EmFi@temp
  \else\ifx\EmFi@temp\EmFi@S@size
    \let\EmFi@type\EmFi@temp
  \else
    \EmFi@Error{%
      Unknown type `\EmFi@temp'.\MessageBreak
      Supported types: `text', `date', `number', `file',\MessageBreak
      `desc', `moddate', `creationdate', `size'%
    }%
  \fi\fi\fi\fi\fi\fi\fi\fi
}
%    \end{macrocode}
%    \begin{macrocode}
\kv@define@key{EmFiFi}{title}{%
  \def\EmFi@title{#1}%
}
%    \end{macrocode}
%    \begin{macro}{\EmFi@setboolean}
%    \begin{macrocode}
\def\EmFi@setboolean#1#2{%
  \EdefSanitize\EmFi@temp{#2}%
  \ifx\EmFi@temp\EmFi@S@true
    \csname EmFi@#1true\endcsname
  \else
    \ifx\EmFi@temp\EmFi@S@false
      \csname EmFi@#1false\endcsname
    \else
      \EmFi@Error{%
        Unknown value `\EmFi@temp' for key `#1'.\MessageBreak
        Supported values: `true', `false'%
      }\@ehc
    \fi
  \fi
}
%    \end{macrocode}
%    \end{macro}
%    \begin{macrocode}
\kv@define@key{EmFiFi}{visible}[true]{%
  \EmFi@setboolean{visible}{#1}%
}
%    \end{macrocode}
%    \begin{macrocode}
\kv@define@key{EmFiFi}{edit}[true]{%
  \EmFi@setboolean{edit}{#1}%
}
%    \end{macrocode}
%
%    \begin{macro}{\EmFi@sortkeys}
%    \begin{macrocode}
\def\EmFi@sortkeys{}
%    \end{macrocode}
%    \end{macro}
%    \begin{macro}{\EmFi@sortorders}
%    \begin{macrocode}
\def\EmFi@sortorders{}
%    \end{macrocode}
%    \end{macro}
%    \begin{macro}{\embedfilesort}
%    \begin{macrocode}
\def\embedfilesort{%
  \kvsetkeys{EmFiSo}%
}
%    \end{macrocode}
%    \end{macro}
%
% \subsection{Embed the file}
%
%    \begin{macro}{\embedfile}
%    \begin{macrocode}
\def\embedfile{%
  \ltx@ifnextchar[\EmFi@embedfile{\EmFi@embedfile[]}%
}
%    \end{macrocode}
%    \end{macro}
%    \begin{macro}{\EmFi@embedfile}
%    \begin{macrocode}
\def\EmFi@embedfile[#1]#2{%
  \ifEmFi@finished
    \EmFi@Error{%
      \string\embedfile\ltx@space after \string\embedfilefinish
    }{%
      The list of embedded files is already written.%
    }%
  \else
    \begingroup
      \def\EmFi@file{#2}%
      \kvsetkeys{EmFi}{#1}%
      \expandafter\expandafter\expandafter
      \ifx\expandafter\expandafter\expandafter
          \\\pdf@filesize{\EmFi@file}\\%
        \EmFi@Error{%
          File `\EmFi@file' not found%
        }{%
          The unknown file is not embedded.%
        }%
      \else
        \edef\EmFi@@filespec{%
          \pdf@escapestring{\EmFi@filespec}%
        }%
        \ifx\EmFi@ucfilespec\ltx@empty
          \let\EmFi@@ucfilespec\ltx@empty
        \else
          \EmFi@convert\EmFi@ucfilespec\EmFi@@ucfilespec
        \fi
        \ifx\EmFi@desc\ltx@empty
          \let\EmFi@@desc\ltx@empty
        \else
          \EmFi@convert\EmFi@desc\EmFi@@desc
        \fi
        \ifEmFi@item
          \let\do\EmFi@do
          \immediate\pdfobj{%
            <<%
              \EmFi@fieldlist
            >>%
          }%
          \edef\EmFi@ci{\the\pdflastobj}%
        \fi
        \immediate\pdfobj stream attr{%
          /Type/EmbeddedFile%
          \ifx\EmFi@mimetype\ltx@empty
          \else
            /Subtype/\pdf@escapename{\EmFi@mimetype}%
          \fi
          /Params<<%
            /ModDate(\pdf@filemoddate{\EmFi@file})%
            /Size \pdf@filesize{\EmFi@file}%
            /CheckSum<\pdf@filemdfivesum{\EmFi@file}>%
          >>%
        }file{\EmFi@file}\relax
        \EmFi@defobj{EmbeddedFile}%
        \immediate\pdfobj{%
          <<%
            /Type/Filespec%
            \ifx\EmFi@filesystem\ltx@empty
            \else
            /FS/\pdf@escapename{\EmFi@filesystem}%
            \fi
            /F(\EmFi@@filespec)%
            \ifx\EmFi@@ucfilespec\ltx@empty
            \else
              /UF(\EmFi@@ucfilespec)%
            \fi
            \ifx\EmFi@@desc\ltx@empty
            \else
              /Desc(\EmFi@@desc)%
            \fi
            /EF<<%
              /F \the\pdflastobj\ltx@space 0 R%
            >>%
            \ifEmFi@item
              /CI \EmFi@ci\ltx@space 0 R%
            \fi
          >>%
        }%
        \EmFi@defobj{Filespec}%
        \EmFi@add{%
          \EmFi@@filespec
        }{\the\pdflastobj\ltx@space 0 R}%
      \fi
    \endgroup
  \fi
}
%    \end{macrocode}
%    \end{macro}
%    \begin{macro}{\EmFi@do}
%    \begin{macrocode}
\def\EmFi@do#1{%
  \expandafter\ifx\csname EmFi@P@#1\endcsname\relax
    \expandafter\ifx\csname EmFi@V@#1\endcsname\relax
    \else
      /\pdf@escapename{#1}\csname EmFi@V@#1\endcsname
    \fi
  \else
    /\pdf@escapename{#1}<<%
      \expandafter\ifx\csname EmFi@V@#1\endcsname\relax
      \else
        /D\csname EmFi@V@#1\endcsname
      \fi
      /P(\csname EmFi@P@#1\endcsname)%
    >>%
  \fi
}
%    \end{macrocode}
%    \end{macro}
%    \begin{macro}{\EmFi@convert}
%    \begin{macrocode}
\def\EmFi@convert#1#2{%
  \ifnum\pdf@strcmp{\EmFi@stringmethod}{psd}=0 %
    \pdfstringdef\EmFi@temp{#1}%
    \let#2\EmFi@temp
  \else
    \edef#2{\pdf@escapestring{#1}}%
  \fi
}
%    \end{macrocode}
%    \end{macro}
%    \begin{macrocode}
\global\let\EmFi@list\ltx@empty
%    \end{macrocode}
%    \begin{macro}{\EmFi@add}
%    Sorting is done by the insertion sort algorithm.
%    Probably the sorting could be done more reliable.
%    However, the PDF specification is not too clear to me
%    regarding precise sorting rules (how to deal with
%    different encodings, escaped characters, \dots).
%    \begin{macrocode}
\def\EmFi@add#1#2{%
  \begingroup
    \ifx\EmFi@list\ltx@empty
      \xdef\EmFi@list{\noexpand\do{#1}{#2}}%
    \else
      \def\do##1##2{%
        \ifnum\pdf@strcmp{##1}{#1}>0 %
          \edef\x{%
            \toks@{%
              \the\toks@%
              \noexpand\do{#1}{#2}%
              \noexpand\do{##1}{##2}%
            }%
          }%
          \x
          \def\do####1####2{%
            \toks@\expandafter{\the\toks@\do{####1}{####2}}%
          }%
          \def\stop{%
            \xdef\EmFi@list{\the\toks@}%
          }%
        \else
          \toks@\expandafter{\the\toks@\do{##1}{##2}}%
        \fi
      }%
      \def\stop{%
        \xdef\EmFi@list{\the\toks@\noexpand\do{#1}{#2}}%
      }%
      \toks@{}%
      \EmFi@list\stop
    \fi
  \endgroup
}
%    \end{macrocode}
%    \end{macro}
%    \begin{macro}{\embedfilefinish}
%    \begin{macrocode}
\def\embedfilefinish{%
  \ifEmFi@finished
    \EmFi@Error{%
      Too many invocations of \string\embedfilefinish
    }{%
      The list of embedded files is already written.%
    }%
  \else
    \ifx\EmFi@list\ltx@empty
    \else
%    \end{macrocode}
%    Write |/EmbeddedFiles| entry.
%    \begin{macrocode}
      \global\EmFi@finishedtrue
      \begingroup
        \def\do##1##2{%
          (##1)##2%
        }%
        \immediate\pdfobj{%
          <<%
            /Names[\EmFi@list]%
          >>%
        }%
        \pdfnames{%
          /EmbeddedFiles \the\pdflastobj\ltx@space 0 R%
        }%
      \endgroup
%    \end{macrocode}
%    Write collection objects.
%    \begin{macrocode}
      \ifx\EmFi@initialfile\ltx@empty
      \else
        \EmFi@collectiontrue
      \fi
      \ifEmFi@collection
        \ifx\EmFi@initialfile\ltx@empty
          \let\EmFi@@initialfile\ltx@empty
        \else
          \edef\EmFi@@initialfile{%
            \pdf@escapestring{\EmFi@initialfile}%
          }%
        \fi
%    \end{macrocode}
%    Look for initial file among the embedded files.
%    \begin{macrocode}
        \begingroup
          \let\f=N%
          \def\do##1##2{%
            \def\x{##1}%
            \ifx\x\EmFi@@initialfile
              \let\f=Y%
              \let\do\ltx@gobbletwo
            \fi
          }%
          \EmFi@list
        \expandafter\endgroup
        \ifx\f Y%
        \else
          \@PackageWarningNoLine{embedfile}{%
            Missing initial file `\EmFi@initialfile'\MessageBreak
            among the embedded files%
          }%
          \let\EmFi@initialfile\ltx@empty
          \let\EmFi@@initialfile\ltx@empty
        \fi
        \ifcase\EmFi@sortcase
          \def\EmFi@temp{}%
        \or
          \def\EmFi@temp{%
            /S\EmFi@sortkeys
            /A \EmFi@sortorders
          }%
        \else
          \def\EmFi@temp{%
            /S[\EmFi@sortkeys]%
            /A[\EmFi@sortorders]%
          }%
        \fi
        \def\EmFi@@order##1{%
          \ifnum\EmFi@order>1 %
            /O ##1%
          \fi
        }%
        \immediate\pdfobj{%
          <<%
            \ifx\EmFi@schema\ltx@empty
            \else
              /Schema<<\EmFi@schema>>%
            \fi
            \ifx\EmFi@@initialfile\ltx@empty
            \else
              /D(\EmFi@@initialfile)%
            \fi
            \ifx\EmFi@view\EmFi@S@tile
              /View/T%
            \else\ifx\EmFi@view\EmFi@S@hidden
              /View/H%
            \fi\fi
            \ifx\EmFi@temp\ltx@empty
              \EmFi@temp
            \else
              /Sort<<\EmFi@temp>>%
            \fi
          >>%
        }%
        \pdfcatalog{%
          /Collection \the\pdflastobj\ltx@space0 R%
        }%
      \fi
    \fi
  \fi
}
%    \end{macrocode}
%    \end{macro}
%    \begin{macrocode}
\begingroup\expandafter\expandafter\expandafter\endgroup
\expandafter\ifx\csname AtEndDocument\endcsname\relax
\else
  \AtEndDocument{\embedfilefinish}%
\fi
%    \end{macrocode}
%
%    \begin{macrocode}
\EmFi@AtEnd%
%</package>
%    \end{macrocode}
%
% \section{Test}
%
% \subsection{Catcode checks for loading}
%
%    \begin{macrocode}
%<*test1>
%    \end{macrocode}
%    \begin{macrocode}
\catcode`\{=1 %
\catcode`\}=2 %
\catcode`\#=6 %
\catcode`\@=11 %
\expandafter\ifx\csname count@\endcsname\relax
  \countdef\count@=255 %
\fi
\expandafter\ifx\csname @gobble\endcsname\relax
  \long\def\@gobble#1{}%
\fi
\expandafter\ifx\csname @firstofone\endcsname\relax
  \long\def\@firstofone#1{#1}%
\fi
\expandafter\ifx\csname loop\endcsname\relax
  \expandafter\@firstofone
\else
  \expandafter\@gobble
\fi
{%
  \def\loop#1\repeat{%
    \def\body{#1}%
    \iterate
  }%
  \def\iterate{%
    \body
      \let\next\iterate
    \else
      \let\next\relax
    \fi
    \next
  }%
  \let\repeat=\fi
}%
\def\RestoreCatcodes{}
\count@=0 %
\loop
  \edef\RestoreCatcodes{%
    \RestoreCatcodes
    \catcode\the\count@=\the\catcode\count@\relax
  }%
\ifnum\count@<255 %
  \advance\count@ 1 %
\repeat

\def\RangeCatcodeInvalid#1#2{%
  \count@=#1\relax
  \loop
    \catcode\count@=15 %
  \ifnum\count@<#2\relax
    \advance\count@ 1 %
  \repeat
}
\def\RangeCatcodeCheck#1#2#3{%
  \count@=#1\relax
  \loop
    \ifnum#3=\catcode\count@
    \else
      \errmessage{%
        Character \the\count@\space
        with wrong catcode \the\catcode\count@\space
        instead of \number#3%
      }%
    \fi
  \ifnum\count@<#2\relax
    \advance\count@ 1 %
  \repeat
}
\def\space{ }
\expandafter\ifx\csname LoadCommand\endcsname\relax
  \def\LoadCommand{\input embedfile.sty\relax}%
\fi
\def\Test{%
  \RangeCatcodeInvalid{0}{47}%
  \RangeCatcodeInvalid{58}{64}%
  \RangeCatcodeInvalid{91}{96}%
  \RangeCatcodeInvalid{123}{255}%
  \catcode`\@=12 %
  \catcode`\\=0 %
  \catcode`\%=14 %
  \LoadCommand
  \RangeCatcodeCheck{0}{36}{15}%
  \RangeCatcodeCheck{37}{37}{14}%
  \RangeCatcodeCheck{38}{47}{15}%
  \RangeCatcodeCheck{48}{57}{12}%
  \RangeCatcodeCheck{58}{63}{15}%
  \RangeCatcodeCheck{64}{64}{12}%
  \RangeCatcodeCheck{65}{90}{11}%
  \RangeCatcodeCheck{91}{91}{15}%
  \RangeCatcodeCheck{92}{92}{0}%
  \RangeCatcodeCheck{93}{96}{15}%
  \RangeCatcodeCheck{97}{122}{11}%
  \RangeCatcodeCheck{123}{255}{15}%
  \RestoreCatcodes
}
\Test
\csname @@end\endcsname
\end
%    \end{macrocode}
%    \begin{macrocode}
%</test1>
%    \end{macrocode}
%
% \subsection{Simple test}
%
%    \begin{macrocode}
%<*test2>
\input embedfile.sty\relax
\embedfile[%
  stringmethod=escape,%
  mimetype=plain/text,%
  desc={LaTeX docstrip source archive for package `embedfile'},%
  id={embedfile.dtx}%
]{embedfile.dtx}
\nopagenumbers
Test (plain-\TeX): {\tt embedfile.dtx} should be embedded.%

\def\Test#1{%
  \par
  \embedfileifobjectexists{embedfile.dtx}{#1}{%
    Object #1 (embedfile.dtx): %
    \embedfilegetobject{embedfile.dtx}{#1}%
  }{%
    \errmessage{Missing object #1 (embedfile.dtx)}%
  }%
}
\Test{EmbeddedFile}
\Test{Filespec}
\embedfilefinish
\bye
%</test2>
%    \end{macrocode}
%    \begin{macrocode}
%<*test3>
\NeedsTeXFormat{LaTeX2e}
\let\SavedJobname\jobname
\def\jobname{embedfile}
\RequirePackage{dtx-attach}[2011/04/13]
\let\jobname\SavedJobname
\documentclass{minimal}
\begin{document}
  Test (\LaTeX): \texttt{embedfile.dtx} should be embedded.%
\end{document}
%</test3>
%    \end{macrocode}
%
% \subsection{Test for \iniTeX}
%
%    \begin{macrocode}
%<*test4>
\catcode`\{=1 %
\catcode`\}=2 %
\input ifluatex.sty %
\ifluatex
  \directlua{%
    tex.enableprimitives('', {%
      'pdflastobj',%
      'pdfnames',%
      'pdfobj',%
      'pdfoutput'%
    })%
  }%
\fi
\pdfoutput=1 %
\input embedfile.sty %
\shipout\hbox{}
\embedfile[%
  stringmethod=escape,%
  mimetype=plain/text,%
  desc={iniTeX source},%
]{\jobname.tex}
\embedfilefinish
\end
%</test4>
%    \end{macrocode}
%
% \section{Installation}
%
% \subsection{Download}
%
% \paragraph{Package.} This package is available on
% CTAN\footnote{\url{ftp://ftp.ctan.org/tex-archive/}}:
% \begin{description}
% \item[\CTAN{macros/latex/contrib/oberdiek/embedfile.dtx}] The source file.
% \item[\CTAN{macros/latex/contrib/oberdiek/embedfile.pdf}] Documentation.
% \end{description}
%
%
% \paragraph{Bundle.} All the packages of the bundle `oberdiek'
% are also available in a TDS compliant ZIP archive. There
% the packages are already unpacked and the documentation files
% are generated. The files and directories obey the TDS standard.
% \begin{description}
% \item[\CTAN{install/macros/latex/contrib/oberdiek.tds.zip}]
% \end{description}
% \emph{TDS} refers to the standard ``A Directory Structure
% for \TeX\ Files'' (\CTAN{tds/tds.pdf}). Directories
% with \xfile{texmf} in their name are usually organized this way.
%
% \subsection{Bundle installation}
%
% \paragraph{Unpacking.} Unpack the \xfile{oberdiek.tds.zip} in the
% TDS tree (also known as \xfile{texmf} tree) of your choice.
% Example (linux):
% \begin{quote}
%   |unzip oberdiek.tds.zip -d ~/texmf|
% \end{quote}
%
% \paragraph{Script installation.}
% Check the directory \xfile{TDS:scripts/oberdiek/} for
% scripts that need further installation steps.
% Package \xpackage{attachfile2} comes with the Perl script
% \xfile{pdfatfi.pl} that should be installed in such a way
% that it can be called as \texttt{pdfatfi}.
% Example (linux):
% \begin{quote}
%   |chmod +x scripts/oberdiek/pdfatfi.pl|\\
%   |cp scripts/oberdiek/pdfatfi.pl /usr/local/bin/|
% \end{quote}
%
% \subsection{Package installation}
%
% \paragraph{Unpacking.} The \xfile{.dtx} file is a self-extracting
% \docstrip\ archive. The files are extracted by running the
% \xfile{.dtx} through \plainTeX:
% \begin{quote}
%   \verb|tex embedfile.dtx|
% \end{quote}
%
% \paragraph{TDS.} Now the different files must be moved into
% the different directories in your installation TDS tree
% (also known as \xfile{texmf} tree):
% \begin{quote}
% \def\t{^^A
% \begin{tabular}{@{}>{\ttfamily}l@{ $\rightarrow$ }>{\ttfamily}l@{}}
%   embedfile.sty & tex/generic/oberdiek/embedfile.sty\\
%   dtx-attach.sty & tex/generic/oberdiek/dtx-attach.sty\\
%   embedfile.pdf & doc/latex/oberdiek/embedfile.pdf\\
%   embedfile-example-plain.tex & doc/latex/oberdiek/embedfile-example-plain.tex\\
%   embedfile-example-collection.tex & doc/latex/oberdiek/embedfile-example-collection.tex\\
%   test/embedfile-test1.tex & doc/latex/oberdiek/test/embedfile-test1.tex\\
%   test/embedfile-test2.tex & doc/latex/oberdiek/test/embedfile-test2.tex\\
%   test/embedfile-test3.tex & doc/latex/oberdiek/test/embedfile-test3.tex\\
%   test/embedfile-test4.tex & doc/latex/oberdiek/test/embedfile-test4.tex\\
%   embedfile.dtx & source/latex/oberdiek/embedfile.dtx\\
% \end{tabular}^^A
% }^^A
% \sbox0{\t}^^A
% \ifdim\wd0>\linewidth
%   \begingroup
%     \advance\linewidth by\leftmargin
%     \advance\linewidth by\rightmargin
%   \edef\x{\endgroup
%     \def\noexpand\lw{\the\linewidth}^^A
%   }\x
%   \def\lwbox{^^A
%     \leavevmode
%     \hbox to \linewidth{^^A
%       \kern-\leftmargin\relax
%       \hss
%       \usebox0
%       \hss
%       \kern-\rightmargin\relax
%     }^^A
%   }^^A
%   \ifdim\wd0>\lw
%     \sbox0{\small\t}^^A
%     \ifdim\wd0>\linewidth
%       \ifdim\wd0>\lw
%         \sbox0{\footnotesize\t}^^A
%         \ifdim\wd0>\linewidth
%           \ifdim\wd0>\lw
%             \sbox0{\scriptsize\t}^^A
%             \ifdim\wd0>\linewidth
%               \ifdim\wd0>\lw
%                 \sbox0{\tiny\t}^^A
%                 \ifdim\wd0>\linewidth
%                   \lwbox
%                 \else
%                   \usebox0
%                 \fi
%               \else
%                 \lwbox
%               \fi
%             \else
%               \usebox0
%             \fi
%           \else
%             \lwbox
%           \fi
%         \else
%           \usebox0
%         \fi
%       \else
%         \lwbox
%       \fi
%     \else
%       \usebox0
%     \fi
%   \else
%     \lwbox
%   \fi
% \else
%   \usebox0
% \fi
% \end{quote}
% If you have a \xfile{docstrip.cfg} that configures and enables \docstrip's
% TDS installing feature, then some files can already be in the right
% place, see the documentation of \docstrip.
%
% \subsection{Refresh file name databases}
%
% If your \TeX~distribution
% (\teTeX, \mikTeX, \dots) relies on file name databases, you must refresh
% these. For example, \teTeX\ users run \verb|texhash| or
% \verb|mktexlsr|.
%
% \subsection{Some details for the interested}
%
% \paragraph{Attached source.}
%
% The PDF documentation on CTAN also includes the
% \xfile{.dtx} source file. It can be extracted by
% AcrobatReader 6 or higher. Another option is \textsf{pdftk},
% e.g. unpack the file into the current directory:
% \begin{quote}
%   \verb|pdftk embedfile.pdf unpack_files output .|
% \end{quote}
%
% \paragraph{Unpacking with \LaTeX.}
% The \xfile{.dtx} chooses its action depending on the format:
% \begin{description}
% \item[\plainTeX:] Run \docstrip\ and extract the files.
% \item[\LaTeX:] Generate the documentation.
% \end{description}
% If you insist on using \LaTeX\ for \docstrip\ (really,
% \docstrip\ does not need \LaTeX), then inform the autodetect routine
% about your intention:
% \begin{quote}
%   \verb|latex \let\install=y% \iffalse meta-comment
%
% File: embedfile.dtx
% Version: 2011/04/13 v2.6
% Info: Embed files into PDF
%
% Copyright (C) 2006-2011 by
%    Heiko Oberdiek <heiko.oberdiek at googlemail.com>
%
% This work may be distributed and/or modified under the
% conditions of the LaTeX Project Public License, either
% version 1.3c of this license or (at your option) any later
% version. This version of this license is in
%    http://www.latex-project.org/lppl/lppl-1-3c.txt
% and the latest version of this license is in
%    http://www.latex-project.org/lppl.txt
% and version 1.3 or later is part of all distributions of
% LaTeX version 2005/12/01 or later.
%
% This work has the LPPL maintenance status "maintained".
%
% This Current Maintainer of this work is Heiko Oberdiek.
%
% The Base Interpreter refers to any `TeX-Format',
% because some files are installed in TDS:tex/generic//.
%
% This work consists of the main source file embedfile.dtx
% and the derived files
%    embedfile.sty, embedfile.pdf, embedfile.ins, embedfile.drv,
%    dtx-attach.sty, embedfile-example-plain.tex,
%    embedfile-example-collection.tex, embedfile-test1.tex,
%    embedfile-test2.tex, embedfile-test3.tex,
%    embedfile-test4.tex.
%
% Distribution:
%    CTAN:macros/latex/contrib/oberdiek/embedfile.dtx
%    CTAN:macros/latex/contrib/oberdiek/embedfile.pdf
%
% Unpacking:
%    (a) If embedfile.ins is present:
%           tex embedfile.ins
%    (b) Without embedfile.ins:
%           tex embedfile.dtx
%    (c) If you insist on using LaTeX
%           latex \let\install=y% \iffalse meta-comment
%
% File: embedfile.dtx
% Version: 2011/04/13 v2.6
% Info: Embed files into PDF
%
% Copyright (C) 2006-2011 by
%    Heiko Oberdiek <heiko.oberdiek at googlemail.com>
%
% This work may be distributed and/or modified under the
% conditions of the LaTeX Project Public License, either
% version 1.3c of this license or (at your option) any later
% version. This version of this license is in
%    http://www.latex-project.org/lppl/lppl-1-3c.txt
% and the latest version of this license is in
%    http://www.latex-project.org/lppl.txt
% and version 1.3 or later is part of all distributions of
% LaTeX version 2005/12/01 or later.
%
% This work has the LPPL maintenance status "maintained".
%
% This Current Maintainer of this work is Heiko Oberdiek.
%
% The Base Interpreter refers to any `TeX-Format',
% because some files are installed in TDS:tex/generic//.
%
% This work consists of the main source file embedfile.dtx
% and the derived files
%    embedfile.sty, embedfile.pdf, embedfile.ins, embedfile.drv,
%    dtx-attach.sty, embedfile-example-plain.tex,
%    embedfile-example-collection.tex, embedfile-test1.tex,
%    embedfile-test2.tex, embedfile-test3.tex,
%    embedfile-test4.tex.
%
% Distribution:
%    CTAN:macros/latex/contrib/oberdiek/embedfile.dtx
%    CTAN:macros/latex/contrib/oberdiek/embedfile.pdf
%
% Unpacking:
%    (a) If embedfile.ins is present:
%           tex embedfile.ins
%    (b) Without embedfile.ins:
%           tex embedfile.dtx
%    (c) If you insist on using LaTeX
%           latex \let\install=y% \iffalse meta-comment
%
% File: embedfile.dtx
% Version: 2011/04/13 v2.6
% Info: Embed files into PDF
%
% Copyright (C) 2006-2011 by
%    Heiko Oberdiek <heiko.oberdiek at googlemail.com>
%
% This work may be distributed and/or modified under the
% conditions of the LaTeX Project Public License, either
% version 1.3c of this license or (at your option) any later
% version. This version of this license is in
%    http://www.latex-project.org/lppl/lppl-1-3c.txt
% and the latest version of this license is in
%    http://www.latex-project.org/lppl.txt
% and version 1.3 or later is part of all distributions of
% LaTeX version 2005/12/01 or later.
%
% This work has the LPPL maintenance status "maintained".
%
% This Current Maintainer of this work is Heiko Oberdiek.
%
% The Base Interpreter refers to any `TeX-Format',
% because some files are installed in TDS:tex/generic//.
%
% This work consists of the main source file embedfile.dtx
% and the derived files
%    embedfile.sty, embedfile.pdf, embedfile.ins, embedfile.drv,
%    dtx-attach.sty, embedfile-example-plain.tex,
%    embedfile-example-collection.tex, embedfile-test1.tex,
%    embedfile-test2.tex, embedfile-test3.tex,
%    embedfile-test4.tex.
%
% Distribution:
%    CTAN:macros/latex/contrib/oberdiek/embedfile.dtx
%    CTAN:macros/latex/contrib/oberdiek/embedfile.pdf
%
% Unpacking:
%    (a) If embedfile.ins is present:
%           tex embedfile.ins
%    (b) Without embedfile.ins:
%           tex embedfile.dtx
%    (c) If you insist on using LaTeX
%           latex \let\install=y\input{embedfile.dtx}
%        (quote the arguments according to the demands of your shell)
%
% Documentation:
%    (a) If embedfile.drv is present:
%           latex embedfile.drv
%    (b) Without embedfile.drv:
%           latex embedfile.dtx; ...
%    The class ltxdoc loads the configuration file ltxdoc.cfg
%    if available. Here you can specify further options, e.g.
%    use A4 as paper format:
%       \PassOptionsToClass{a4paper}{article}
%
%    Programm calls to get the documentation (example):
%       pdflatex embedfile.dtx
%       makeindex -s gind.ist embedfile.idx
%       pdflatex embedfile.dtx
%       makeindex -s gind.ist embedfile.idx
%       pdflatex embedfile.dtx
%
% Installation:
%    TDS:tex/generic/oberdiek/embedfile.sty
%    TDS:tex/generic/oberdiek/dtx-attach.sty
%    TDS:doc/latex/oberdiek/embedfile.pdf
%    TDS:doc/latex/oberdiek/embedfile-example-plain.tex
%    TDS:doc/latex/oberdiek/embedfile-example-collection.tex
%    TDS:doc/latex/oberdiek/test/embedfile-test1.tex
%    TDS:doc/latex/oberdiek/test/embedfile-test2.tex
%    TDS:doc/latex/oberdiek/test/embedfile-test3.tex
%    TDS:doc/latex/oberdiek/test/embedfile-test4.tex
%    TDS:source/latex/oberdiek/embedfile.dtx
%
%<*ignore>
\begingroup
  \catcode123=1 %
  \catcode125=2 %
  \def\x{LaTeX2e}%
\expandafter\endgroup
\ifcase 0\ifx\install y1\fi\expandafter
         \ifx\csname processbatchFile\endcsname\relax\else1\fi
         \ifx\fmtname\x\else 1\fi\relax
\else\csname fi\endcsname
%</ignore>
%<*install>
\input docstrip.tex
\Msg{************************************************************************}
\Msg{* Installation}
\Msg{* Package: embedfile 2011/04/13 v2.6 Embed files into PDF (HO)}
\Msg{************************************************************************}

\keepsilent
\askforoverwritefalse

\let\MetaPrefix\relax
\preamble

This is a generated file.

Project: embedfile
Version: 2011/04/13 v2.6

Copyright (C) 2006-2011 by
   Heiko Oberdiek <heiko.oberdiek at googlemail.com>

This work may be distributed and/or modified under the
conditions of the LaTeX Project Public License, either
version 1.3c of this license or (at your option) any later
version. This version of this license is in
   http://www.latex-project.org/lppl/lppl-1-3c.txt
and the latest version of this license is in
   http://www.latex-project.org/lppl.txt
and version 1.3 or later is part of all distributions of
LaTeX version 2005/12/01 or later.

This work has the LPPL maintenance status "maintained".

This Current Maintainer of this work is Heiko Oberdiek.

The Base Interpreter refers to any `TeX-Format',
because some files are installed in TDS:tex/generic//.

This work consists of the main source file embedfile.dtx
and the derived files
   embedfile.sty, embedfile.pdf, embedfile.ins, embedfile.drv,
   dtx-attach.sty, embedfile-example-plain.tex,
   embedfile-example-collection.tex, embedfile-test1.tex,
   embedfile-test2.tex, embedfile-test3.tex,
   embedfile-test4.tex.

\endpreamble
\let\MetaPrefix\DoubleperCent

\generate{%
  \file{embedfile.ins}{\from{embedfile.dtx}{install}}%
  \file{embedfile.drv}{\from{embedfile.dtx}{driver}}%
  \usedir{tex/generic/oberdiek}%
  \file{embedfile.sty}{\from{embedfile.dtx}{package}}%
  \usedir{tex/latex/oberdiek}%
  \file{dtx-attach.sty}{\from{embedfile.dtx}{dtxattach}}%
  \usedir{doc/latex/oberdiek}%
  \file{embedfile-example-plain.tex}{\from{embedfile.dtx}{exampleplain}}%
  \file{embedfile-example-collection.tex}{\from{embedfile.dtx}{examplecollection}}%
  \usedir{doc/latex/oberdiek/test}%
  \file{embedfile-test1.tex}{\from{embedfile.dtx}{test1}}%
  \file{embedfile-test2.tex}{\from{embedfile.dtx}{test2}}%
  \file{embedfile-test3.tex}{\from{embedfile.dtx}{test3}}%
  \file{embedfile-test4.tex}{\from{embedfile.dtx}{test4}}%
  \nopreamble
  \nopostamble
  \usedir{source/latex/oberdiek/catalogue}%
  \file{embedfile.xml}{\from{embedfile.dtx}{catalogue}}%
}

\catcode32=13\relax% active space
\let =\space%
\Msg{************************************************************************}
\Msg{*}
\Msg{* To finish the installation you have to move the following}
\Msg{* files into a directory searched by TeX:}
\Msg{*}
\Msg{*     embedfile.sty, dtx-attach.sty}
\Msg{*}
\Msg{* To produce the documentation run the file `embedfile.drv'}
\Msg{* through LaTeX.}
\Msg{*}
\Msg{* Happy TeXing!}
\Msg{*}
\Msg{************************************************************************}

\endbatchfile
%</install>
%<*ignore>
\fi
%</ignore>
%<*driver>
\NeedsTeXFormat{LaTeX2e}
\ProvidesFile{embedfile.drv}%
  [2011/04/13 v2.6 Embed files into PDF (HO)]%
\documentclass{ltxdoc}
\usepackage{holtxdoc}[2011/11/22]
\begin{document}
  \DocInput{embedfile.dtx}%
\end{document}
%</driver>
% \fi
%
% \CheckSum{1363}
%
% \CharacterTable
%  {Upper-case    \A\B\C\D\E\F\G\H\I\J\K\L\M\N\O\P\Q\R\S\T\U\V\W\X\Y\Z
%   Lower-case    \a\b\c\d\e\f\g\h\i\j\k\l\m\n\o\p\q\r\s\t\u\v\w\x\y\z
%   Digits        \0\1\2\3\4\5\6\7\8\9
%   Exclamation   \!     Double quote  \"     Hash (number) \#
%   Dollar        \$     Percent       \%     Ampersand     \&
%   Acute accent  \'     Left paren    \(     Right paren   \)
%   Asterisk      \*     Plus          \+     Comma         \,
%   Minus         \-     Point         \.     Solidus       \/
%   Colon         \:     Semicolon     \;     Less than     \<
%   Equals        \=     Greater than  \>     Question mark \?
%   Commercial at \@     Left bracket  \[     Backslash     \\
%   Right bracket \]     Circumflex    \^     Underscore    \_
%   Grave accent  \`     Left brace    \{     Vertical bar  \|
%   Right brace   \}     Tilde         \~}
%
% \GetFileInfo{embedfile.drv}
%
% \title{The \xpackage{embedfile} package}
% \date{2011/04/13 v2.6}
% \author{Heiko Oberdiek\\\xemail{heiko.oberdiek at googlemail.com}}
%
% \maketitle
%
% \begin{abstract}
% This package embeds files to a PDF document.
% Currently the only supported driver is \pdfTeX\ $>=$ 1.30 in PDF mode.
% \end{abstract}
%
% \tableofcontents
%
% \section{Documentation}
%
% \subsection{Introduction}
%
%    The PDF format (\cite{pdfspec}) allows the inclusion of files
%    inside the PDF document. The included files can be bound to an
%    annotation on a page. Or they can be recorded in a sorted
%    list of embedded files. The packages \xpackage{attachfile}
%    or \xpackage{attachfile2} follow the first approach, this
%    package uses the latter method.
%
% \subsubsection{Future development}
%
%    My dream is a large package that merges the features of
%    all these packages meantioned before:
%    \begin{itemize}
%    \item Files can be attached to a page.
%    \item Files can be attached to the document.
%    \item An easy user interface for simple, common tasks and
%          beginners.
%    \item An interface for the advanced users that want to setup
%          every detail.
%    \item Support of many drivers (pdftex, dvips, dvipdfm, \dots).
%    \item \dots
%    \end{itemize}
%    However, I have not managed to take the time for this project.
%    Instead:
%    \begin{itemize}
%    \item First I experimented with package \xpackage{attachfile},
%          adding driver support, fixing bugs, \dots. The result is
%          currently named as \xpackage{attachfile2}. It uses an external
%          script to get file properties (size, date, checksum, \dots).
%    \item In order to avoid an external program for getting basic
%          file properties I provided a patch ``EscapeAndOther'' for
%          pdfTeX that was accepted for version 1.30.
%    \item This package closes a gap left by the packages for attaching
%          files and allows the embedding of files to the document.
%          Also it makes use of the new primitives of \pdfTeX.
%    \end{itemize}
%
% \subsection{User interface}
%
%    This package \xpackage{embedfile} can be used with both
%    \LaTeX\ and \plainTeX. See \autoref{sec:plain} that
%    explains the use with \plainTeX\ by an example.
%    In \LaTeX\ the package is loaded as usually. There are no options.
%    \begin{quote}
%      \verb|\usepackage{embedfile}|
%    \end{quote}
%
%    \begin{declcs}{embedfile}
%      |[|\meta{options}|]| \M{file}
%    \end{declcs}
%    The macro \cs{embedfile} includes file \meta{file} and
%    attaches it to the PDF document. At the end of the document
%    the sorted list of embedded files are written. Thus you can
%    safely use \cs{embedfile} before \verb|\end{document}|.
%    Embedding files using \cs{AtEndDocument} will only work,
%    if \cs{AtEndDocument} is called before loading the package
%    \xpackage{embedfile}.
%
%    The \meta{options} are give as key value pairs. The following
%    keys are supported:
%    \begin{description}
%    \item[\xoption{filespec}]
%      This allows to override the file name
%      that appears in the PDF file. If you are using other than
%      simple file names (8-bit, path separators, \dots), look into
%      the PDF specification (\cite{pdfspec}).
%      There are rules how these file names must be written/encoded.
%      Avoid 8-bit characters
%      and other special characters, the behaviour is currently undefined.
%      Use option \xoption{ucfilespec} for more funny file names.
%      The string method, see below, is |escape| since version 2.4.
%
%      This name is also used as entry in a name tree
%      (see PDF specification: |/EmbeddedFiles|). Therefore the
%      value for \xoption{filespec} must be unique among all
%      embedded files. Also key \xoption{initialfiles} refers
%      to this name, if the file name and the value of \xoption{filespec}
%      are different.
%    \item[\xoption{ucfilespec}]
%      Since PDF 1.7 the file name may be provided in Unicode.
%      The conversion of the option value into a PDF string
%      is controlled by option \xoption{stringmethod}.
%    \item[\xoption{filesystem}]
%      This sets the entry \verb|/FS| in
%      the file specification dictionary, see PDF specification
%      (\cite{pdfspec}).
%      Example: \verb|filesystem=URL|.
%    \item[\xoption{mimetype}]
%      This sets the mime type (\cite{mime}) of the file,
%      see \autoref{sec:plain} for examples and \cite{types}
%      for a list of officially registered types.
%    \item[\xoption{desc}]
%      The description for the file.
%    \item[\xoption{stringmethod}]
%      The package must convert the values of the keys \xoption{ucfilespec}
%      and \xoption{desc} into a PDF string (before version 2.4: \xoption{filespec}
%      and \xoption{desc}). If \xpackage{hyperref}
%      is found, then its \cs{pdfstringdef} will be used, otherwise
%      \pdfTeX's \cs{pdfescapestring} is used. Value \verb|psd|
%      forces the use of \cs{pdfstringdef}, value \verb|escape|
%      the use of \cs{pdfescapestring}.
%    \item[\xoption{\meta{key}.value}] Sets the value of a collection
%      item property, see section \ref{sec:collection}.
%    \item[\xoption{\meta{key}.prefix}]
%      Sets the prefix of a collection item property,
%      see section \ref{sec:collection}.
%    \item[\xoption{id}]
%      The value must be an unique name. Macros
%      \cs{embedfileifobjectexists} and \cs{embedfilegetobject}
%      are using this name later.
%    \end{description}
%
%    \begin{declcs}{embedfilefinish}
%    \end{declcs}
%    The list of all embedded files must be added as data structure
%    in the PDF file. In case of \LaTeX\ this is automatically done.
%    The package uses \cs{AtEndDocument}. Then the list of all
%    files should be known. However, \plainTeX\ does not know
%    about \cs{AtEndDocument}. Thus the user must call
%    \cs{embedfilefinish} at the end of the document
%    after the last file is embedded.
%
%    \begin{declcs}{embedfilesetup} \M{options}
%    \end{declcs}
%    Options for \cs{embedfile} and collection support
%    can be set in \cs{embedfilesetup}.
%
% \subsection{Collection support (PDF 1.7)}
% \label{sec:collection}
%
%    Since PDF 1.7 the embedded files can form a \emph{collection}
%    (sometimes referred as \emph{package}), the main document
%    is called \emph{cover sheet}. See PDF specification
%    8.2.4 ``Collections'' and 3.10.5 ``Collection items'' \cite{pdfspec}.
%
%    Usually Acrobat Reader 7 or 8 shows the
%    embedded files in a table at the bottom with the following columns:
%    \begin{quote}
%    \begin{tabular}{@{}l|l|l|l@{}}
%       Name & Description & Modified & Size\\
%       \hline
%       \dots & \dots & \dots & \dots\\
%    \end{tabular}
%    \end{quote}
%    Acrobat Reader 10 shows the embeded files in the left panel
%    and adds a new column for the compressed size.
%
%    If the files form a collection, then they are displayed
%    in a table left or top (depending on option \xoption{view},
%    see \cs{embedfilesetup}).
%
%
%    Collection support is enabled automatically, if
%    it is used.
%
%    \begin{declcs}{embedfilesetup} \M{options}
%    \end{declcs}
%    The following options are supported in addition
%    to options for \cs{embedfile}:
%    \begin{description}
%    \item[\xoption{view}]
%      If the PDF file contains a collection, then Acrobat Reader 8
%      shows a line at the top below the menu bar and the toolbar.
%      It shows the current selected file, icons for changing the
%      view mode, an options menu.
%      The initial mode how the collection is presented is
%      set by this option \xoption{view}.
%      The following modes/values are supported, the default is |details|:
%        \begin{description}
%        \item[\texttt{details}] The full collection table is displayed
%          at the top below the collection bar.
%        \item[\texttt{tile}] The files of the collection are shown
%          in tile mode on the left.
%        \item[\texttt{hidden}] The collection table is not shown.
%        \end{description}
%    \item[\xoption{initialfile}]
%      Selects the file that is initially presented. Especially
%      useful for an embedded PDF file that is then shown instead
%      of the cover document. There must be an \cs{embedfile} command
%      somewhere whose value for key \xoption{filespec} is used here.
%      The \cs{embedfile} command can drop option \xoption{filespec}
%      if the file name is not different.
%    \end{description}
%
%    \begin{declcs}{embedfilefield} \M{key} \M{options}
%    \end{declcs}
%    Macro \cs{embedfilefield} defines a column/field in the collection table.
%    The name of the field is \meta{key}.
%    \begin{description}
%    \item[\xoption{type}]
%      sets the type of the field. The supported values are:
%      \begin{description}
%      \item[\texttt{text}] A text field. Its value is set in
%        \cs{embedfile} by option \xoption{\meta{key}.value}.
%      \item[\texttt{date}] A date field. Its value is set in
%        \cs{embedfile} by option \xoption{\meta{key}.value}.
%        A special format is required, see ``3.8.3 Dates'' \cite{pdfspec}.
%      \item[\texttt{number}] A field with an integer or float number.
%        Its value is set in \cs{embedfile} by option
%        \xoption{\meta{key}.value}.
%      \item[\texttt{file}] The file name of the embedded file.
%      \item[\texttt{desc}] The description text of the embedded file. It is
%       set in \cs{embedfile} by option \xoption{desc}.
%      \item[\texttt{moddate}] The modification date of the embedded file.
%      \item[\texttt{size}] The size of the embedded file.
%      \end{description}
%      All types allow the use of a prefix that is disregarded by
%      sorting. The prefix for this field is set in \cs{embedfile} by
%      option \xoption{\meta{key}.prefix}.
%    \item[\xoption{title}] sets the column title.
%    \item[\xoption{visible}] controls whether the column is presented:
%      \begin{description}
%      \item[\texttt{true}] shows the column.
%      \item[\texttt{false}] hides the column.
%      \end{description}
%      Default: \texttt{true}
%    \item[\xoption{edit}]
%      Allows the editing of field values. Does not seem to have an
%      effect for Acrobat Reader.
%      \begin{description}
%      \item[\texttt{true}] enables the feature, if available (depends
%        on the PDF viewer).
%      \item[\texttt{false}] disables the feature.
%      \end{description}
%      Default: \texttt{false}
%    \end{description}
%    The order of \cs{embedfilefield} statements defines the order
%    of the columns.
%
%    \begin{declcs}{embedfilesort} \M{key-sort-list}
%    \end{declcs}
%    The sort order of the embedded files are controlled by
%    macro \cs{embedfilesort}. \meta{key-sort-list} defines
%    the sort order. The key is a field name defined by \cs{embedfilefield}.
%    Its value is either \texttt{ascending} or \texttt{descending}.
%    The default is \texttt{ascending}.
%
% \subsection{Export of object references}
%
%    Caution: This feature is still experimental. It may be even removed
%    in future versions. Therefore feedback would be nice, if someone
%    has a useful application for this feature.
%
%    Object numbers are saved, if \xoption{id} is given in \cs{embedfile}.
%    The following objects are supported:
%    \begin{itemize}
%    \item \texttt{EmbeddedFile}
%    \item \texttt{Filespec}
%    \end{itemize}
%
%    \begin{declcs}{embedfileifobjectexists} \M{id} \M{type}
%      \M{then} \M{else}
%    \end{declcs}
%    Macro \cs{embedfileifobjectexists} tests whether object of \meta{type}
%    is available for the embedded file identified by \meta{id}.
%
%    \begin{declcs}{embedfilegetobject} \M{id} \M{type}
%    \end{declcs}
%    Macro \cs{embedfilegetobject} expands to the full object reference
%    object of \meta{type} for the embedded file identified by \meta{id}.
%
% \subsubsection{Example}
%
%\begin{quote}
%\begin{verbatim}
%\embedfile[id={foo}]{foo.pdf}
%\embedfileifobjectexists{foo}{Filespec}{%
%  \typeout{%
%    FileSpec object for `foo': %
%    \embedfilegetobject{foo}{Filespec}%
%  }%
%}{%
%  \typeout{No Filespec object for `foo'}%
%}
%\end{verbatim}
%\end{quote}
%
% \subsection{Examples}
% \label{sec:examples}
%
% \subsubsection{\plainTeX}
% \label{sec:plain}
%
%    The package can be used with \plainTeX. It can be used
%    with or without help from \xfile{miniltx.tex}.
%
%    If additionally package \xpackage{keyval} (\xpackage{graphicx})
%    is needed, load it first. Then package \xpackage{embedfile} avoids
%    a duplicate loading of package \xpackage{keyval}.
%
%    Because \plainTeX\ does not provide a hook at end of the document,
%    you have to call \cs{embedfilefinish} manually at the end
%    after the last embedded file.
%    \begin{macrocode}
%<*exampleplain>
%<<END
% Load packages
\input miniltx
% \def\Gin@driver{pdftex.def}
% \input graphicx.sty
\input embedfile.sty
\resetatcatcode

% default setting
\embedfilesetup{
  mimetype=text/plain
}

% Embed files
\embedfile[
  filespec=example.tex,
  desc={Source code (plain-TeX) of this example}
]{embedfile-example-plain.tex}

\embedfile[
  desc={Source of package `embedfile'}
]{embedfile.dtx}

\embedfile[
  mimetype=application/pdf,
  desc={Documentation of package `embedfile'}
]{embedfile.pdf}

% Some text
This example document contains three embedded files.

% End of document
\embedfilefinish % don't forget
\bye
%END
%</exampleplain>
%    \end{macrocode}
%
% \subsubsection{Collection example}
%
%    \begin{macrocode}
%<*examplecollection>
%<<END
\NeedsTeXFormat{LaTeX2e}
\documentclass{article}
\usepackage[bookmarks=false]{hyperref}
% provides \pdfstringdef that is then used by `title' and
% other keys.
\usepackage{embedfile}[2011/04/13]
\embedfilesetup{
  view=details,
  initialfile=embedfile.pdf
}
\embedfilefield{file}{
  type=file,
  title={File name}
}
\embedfilefield{description}{
  type=desc,
  title={Description}
}
\embedfilefield{date}{
  type=moddate,
  title={Date}
}
\embedfilefield{size}{
  type=size,
  title={Size}
}
\embedfilefield{type}{
  type=text,
  title={Type},
  visible=false
}
\embedfilesort{
  type,
  date=descending
}
\begin{document}
An example for embedded files as collection.
You need Acrobat Reader 8 or higher.

\embedfile[
  desc={Source file of package `embedfile'},
  description.prefix={Package: },
  type.value={DTX}
]{embedfile.dtx}

\embedfile[
  desc={Documentation of package `embedfile'},
  description.prefix={Package: },
  type.value={PDF}
]{embedfile.pdf}

\embedfile[
  desc={The source for this example},
  description.prefix={Example: },
  type.value={TEX}
]{\jobname.tex}

\end{document}
%END
%</examplecollection>
%    \end{macrocode}
%
% \subsection{Package \xpackage{dtx-attach}}
%
% Package \xpackage{dtx-attach} is just a small application of
% package \xpackage{embedfile}. I am using it for the CTAN
% documentation of my packages in \CTAN{macros/latex/contrib/oberdiek/}.
% It also serves as small example for the use of the package with \LaTeX.
%
%    \begin{macrocode}
%<*dtxattach>
\NeedsTeXFormat{LaTeX2e}
\ProvidesPackage{dtx-attach}
  [2011/04/13 v2.6 Embed \string\jobname.dtx (HO)]%
\RequirePackage{embedfile}[2011/04/13]
\embedfile[%
  stringmethod=escape,%
  mimetype=plain/text,%
  desc={LaTeX docstrip source archive for package `\jobname'}%
]{\jobname.dtx}
%</dtxattach>
%    \end{macrocode}
%
% \StopEventually{
% }
%
% \section{Implementation}
%
%    \begin{macrocode}
%<*package>
%    \end{macrocode}
%
% \subsection{Reload check and package identification}
%    Reload check, especially if the package is not used with \LaTeX.
%    \begin{macrocode}
\begingroup\catcode61\catcode48\catcode32=10\relax%
  \catcode13=5 % ^^M
  \endlinechar=13 %
  \catcode35=6 % #
  \catcode39=12 % '
  \catcode44=12 % ,
  \catcode45=12 % -
  \catcode46=12 % .
  \catcode58=12 % :
  \catcode64=11 % @
  \catcode123=1 % {
  \catcode125=2 % }
  \expandafter\let\expandafter\x\csname ver@embedfile.sty\endcsname
  \ifx\x\relax % plain-TeX, first loading
  \else
    \def\empty{}%
    \ifx\x\empty % LaTeX, first loading,
      % variable is initialized, but \ProvidesPackage not yet seen
    \else
      \expandafter\ifx\csname PackageInfo\endcsname\relax
        \def\x#1#2{%
          \immediate\write-1{Package #1 Info: #2.}%
        }%
      \else
        \def\x#1#2{\PackageInfo{#1}{#2, stopped}}%
      \fi
      \x{embedfile}{The package is already loaded}%
      \aftergroup\endinput
    \fi
  \fi
\endgroup%
%    \end{macrocode}
%    Package identification:
%    \begin{macrocode}
\begingroup\catcode61\catcode48\catcode32=10\relax%
  \catcode13=5 % ^^M
  \endlinechar=13 %
  \catcode35=6 % #
  \catcode39=12 % '
  \catcode40=12 % (
  \catcode41=12 % )
  \catcode44=12 % ,
  \catcode45=12 % -
  \catcode46=12 % .
  \catcode47=12 % /
  \catcode58=12 % :
  \catcode64=11 % @
  \catcode91=12 % [
  \catcode93=12 % ]
  \catcode123=1 % {
  \catcode125=2 % }
  \expandafter\ifx\csname ProvidesPackage\endcsname\relax
    \def\x#1#2#3[#4]{\endgroup
      \immediate\write-1{Package: #3 #4}%
      \xdef#1{#4}%
    }%
  \else
    \def\x#1#2[#3]{\endgroup
      #2[{#3}]%
      \ifx#1\@undefined
        \xdef#1{#3}%
      \fi
      \ifx#1\relax
        \xdef#1{#3}%
      \fi
    }%
  \fi
\expandafter\x\csname ver@embedfile.sty\endcsname
\ProvidesPackage{embedfile}%
  [2011/04/13 v2.6 Embed files into PDF (HO)]%
%    \end{macrocode}
%
% \subsection{Catcodes}
%
%    \begin{macrocode}
\begingroup\catcode61\catcode48\catcode32=10\relax%
  \catcode13=5 % ^^M
  \endlinechar=13 %
  \catcode123=1 % {
  \catcode125=2 % }
  \catcode64=11 % @
  \def\x{\endgroup
    \expandafter\edef\csname EmFi@AtEnd\endcsname{%
      \endlinechar=\the\endlinechar\relax
      \catcode13=\the\catcode13\relax
      \catcode32=\the\catcode32\relax
      \catcode35=\the\catcode35\relax
      \catcode61=\the\catcode61\relax
      \catcode64=\the\catcode64\relax
      \catcode123=\the\catcode123\relax
      \catcode125=\the\catcode125\relax
    }%
  }%
\x\catcode61\catcode48\catcode32=10\relax%
\catcode13=5 % ^^M
\endlinechar=13 %
\catcode35=6 % #
\catcode64=11 % @
\catcode123=1 % {
\catcode125=2 % }
\def\TMP@EnsureCode#1#2{%
  \edef\EmFi@AtEnd{%
    \EmFi@AtEnd
    \catcode#1=\the\catcode#1\relax
  }%
  \catcode#1=#2\relax
}
\TMP@EnsureCode{39}{12}% '
\TMP@EnsureCode{40}{12}% (
\TMP@EnsureCode{41}{12}% )
\TMP@EnsureCode{44}{12}% ,
\TMP@EnsureCode{46}{12}% .
\TMP@EnsureCode{47}{12}% /
\TMP@EnsureCode{58}{12}% :
\TMP@EnsureCode{60}{12}% <
\TMP@EnsureCode{62}{12}% >
\TMP@EnsureCode{91}{12}% [
\TMP@EnsureCode{93}{12}% ]
\TMP@EnsureCode{96}{12}% `
\edef\EmFi@AtEnd{\EmFi@AtEnd\noexpand\endinput}
%    \end{macrocode}
%
% \subsection{Tools}
%
%    \begin{macro}{\EmFi@RequirePackage}
%    \begin{macrocode}
\begingroup\expandafter\expandafter\expandafter\endgroup
\expandafter\ifx\csname RequirePackage\endcsname\relax
  \def\EmFi@RequirePackage#1[#2]{%
    \input #1.sty\relax
  }%
\else
  \let\EmFi@RequirePackage\RequirePackage
\fi
%    \end{macrocode}
%    \end{macro}
%
%    \begin{macro}{\EmFi@Error}
%    \begin{macrocode}
\EmFi@RequirePackage{infwarerr}[2007/09/09]%
\def\EmFi@Error{%
  \@PackageError{embedfile}%
}
%    \end{macrocode}
%    \end{macro}
%
% \subsection{Check for recent \pdfTeX\ in PDF mode}
%
%    Load package \xpackage{ifpdf} and check mode.
%    \begin{macrocode}
\EmFi@RequirePackage{ifpdf}[2007/09/09]
\ifpdf
\else
  \EmFi@Error{%
    Missing pdfTeX in PDF mode%
  }{%
    Currently other drivers are not supported. %
    Package loading is aborted.%
  }%
  \expandafter\EmFi@AtEnd
\fi%
%    \end{macrocode}
%    \begin{macrocode}
\EmFi@RequirePackage{pdftexcmds}[2007/11/11]
\EmFi@RequirePackage{ltxcmds}[2010/03/01]
\EmFi@RequirePackage{kvsetkeys}[2010/03/01]
\EmFi@RequirePackage{kvdefinekeys}[2010/03/01]
%    \end{macrocode}
%    Check version.
%    \begin{macrocode}
\begingroup\expandafter\expandafter\expandafter\endgroup
\expandafter\ifx\csname pdf@filesize\endcsname\relax
  \EmFi@Error{%
    Unsupported pdfTeX version%
  }{%
    At least version 1.30 is necessary. Package loading is aborted.%
  }%
  \expandafter\EmFi@AtEnd
\fi%
%    \end{macrocode}
%
% \subsection{Strings}
%
%    Minimal version of package \xpackage{pdfescape} is 2007/08/27 v1.5
%    because of \cs{EdefSanitize}.
%    \begin{macrocode}
\EmFi@RequirePackage{pdfescape}[2007/11/11]
%    \end{macrocode}
%
%    \begin{macrocode}
\def\EmFi@temp#1{%
  \expandafter\EdefSanitize\csname EmFi@S@#1\endcsname{#1}%
}
%    \end{macrocode}
%    \begin{macro}{\EmFi@details}
%    \begin{macrocode}
\EmFi@temp{details}%
%    \end{macrocode}
%    \end{macro}
%    \begin{macro}{\EmFi@tile}
%    \begin{macrocode}
\EmFi@temp{tile}%
%    \end{macrocode}
%    \end{macro}
%    \begin{macro}{\EmFi@hidden}
%    \begin{macrocode}
\EmFi@temp{hidden}%
%    \end{macrocode}
%    \end{macro}
%    \begin{macro}{\EmFi@S@text}
%    \begin{macrocode}
\EmFi@temp{text}
%    \end{macrocode}
%    \end{macro}
%    \begin{macro}{\EmFi@S@date}
%    \begin{macrocode}
\EmFi@temp{date}
%    \end{macrocode}
%    \end{macro}
%    \begin{macro}{\EmFi@S@number}
%    \begin{macrocode}
\EmFi@temp{number}
%    \end{macrocode}
%    \end{macro}
%    \begin{macro}{\EmFi@S@file}
%    \begin{macrocode}
\EmFi@temp{file}
%    \end{macrocode}
%    \end{macro}
%    \begin{macro}{\EmFi@S@desc}
%    \begin{macrocode}
\EmFi@temp{desc}
%    \end{macrocode}
%    \end{macro}
%    \begin{macro}{\EmFi@S@moddate}
%    \begin{macrocode}
\EmFi@temp{moddate}
%    \end{macrocode}
%    \end{macro}
%    \begin{macro}{\EmFi@S@creationdate}
%    \begin{macrocode}
\EmFi@temp{creationdate}
%    \end{macrocode}
%    \end{macro}
%    \begin{macro}{\EmFi@S@size}
%    \begin{macrocode}
\EmFi@temp{size}
%    \end{macrocode}
%    \end{macro}
%    \begin{macro}{\EmFi@S@ascending}
%    \begin{macrocode}
\EmFi@temp{ascending}
%    \end{macrocode}
%    \end{macro}
%    \begin{macro}{\EmFi@S@descending}
%    \begin{macrocode}
\EmFi@temp{descending}
%    \end{macrocode}
%    \end{macro}
%    \begin{macro}{\EmFi@S@true}
%    \begin{macrocode}
\EmFi@temp{true}
%    \end{macrocode}
%    \end{macro}
%    \begin{macro}{\EmFi@S@false}
%    \begin{macrocode}
\EmFi@temp{false}
%    \end{macrocode}
%    \end{macro}
%
% \subsection{Switches}
%
%    \begin{macro}{\ifEmFi@collection}
%    \begin{macrocode}
\ltx@newif\ifEmFi@collection
%    \end{macrocode}
%    \end{macro}
%    \begin{macro}{\ifEmFi@sort}
%    \begin{macrocode}
\ltx@newif\ifEmFi@sort
%    \end{macrocode}
%    \end{macro}
%    \begin{macro}{\ifEmFi@visible}
%    \begin{macrocode}
\ltx@newif\ifEmFi@visible
%    \end{macrocode}
%    \end{macro}
%    \begin{macro}{\ifEmFi@edit}
%    \begin{macrocode}
\ltx@newif\ifEmFi@edit
%    \end{macrocode}
%    \end{macro}
%    \begin{macro}{\ifEmFi@item}
%    \begin{macrocode}
\ltx@newif\ifEmFi@item
%    \end{macrocode}
%    \end{macro}
%    \begin{macro}{\ifEmFi@finished}
%    \begin{macrocode}
\ltx@newif\ifEmFi@finished
%    \end{macrocode}
%    \end{macro}
%    \begin{macro}{\ifEmFi@id}
%    \begin{macrocode}
\ltx@newif\ifEmFi@id
%    \end{macrocode}
%    \end{macro}
%
% \subsection{Key value definitions}
%
%    \begin{macro}{\EmFi@GlobalKey}
%    \begin{macrocode}
\def\EmFi@GlobalKey#1#2{%
  \global\expandafter\let\csname KV@#1@#2\expandafter\endcsname
                         \csname KV@#1@#2\endcsname
}
%    \end{macrocode}
%    \end{macro}
%    \begin{macro}{\EmFi@GlobalDefaultKey}
%    \begin{macrocode}
\def\EmFi@GlobalDefaultKey#1#2{%
  \EmFi@GlobalKey{#1}{#2}%
  \global\expandafter\let
      \csname KV@#1@#2@default\expandafter\endcsname
      \csname KV@#1@#2@default\endcsname
}
%    \end{macrocode}
%    \end{macro}
%
%    \begin{macro}{\EmFi@DefineKey}
%    \begin{macrocode}
\def\EmFi@DefineKey#1#2{%
  \kv@define@key{EmFi}{#1}{%
    \expandafter\def\csname EmFi@#1\endcsname{##1}%
  }%
  \expandafter\def\csname EmFi@#1\endcsname{#2}%
}
%    \end{macrocode}
%    \end{macro}
%    Subtype of the embedded file (optional).
%    \begin{macrocode}
\EmFi@DefineKey{mimetype}{}
%    \end{macrocode}
%    File specification string.
%    \begin{macrocode}
\EmFi@DefineKey{filespec}{\EmFi@file}
%    \end{macrocode}
%    File specification string in Unicode.
%    \begin{macrocode}
\EmFi@DefineKey{ucfilespec}{}
%    \end{macrocode}
%    File system (optional).
%    \begin{macrocode}
\EmFi@DefineKey{filesystem}{}
%    \end{macrocode}
%    Description (optional).
%    \begin{macrocode}
\EmFi@DefineKey{desc}{}
%    \end{macrocode}
%    Method for converting text to PDF strings.
%    \begin{macrocode}
\EmFi@DefineKey{stringmethod}{%
  \ifx\pdfstringdef\@undefined
    escape%
  \else
    \ifx\pdfstringdef\relax
      escape%
    \else
      psd%
    \fi
  \fi
}
%    \end{macrocode}
%
%    Option \xoption{id} as key for object numbers.
%    \begin{macrocode}
\kv@define@key{EmFi}{id}{%
  \def\EmFi@id{#1}%
  \EmFi@idtrue
}
%    \end{macrocode}
%    \begin{macro}{\EmFi@defobj}
%    \begin{macrocode}
\def\EmFi@defobj#1{%
  \ifEmFi@id
    \expandafter\xdef\csname EmFi@#1@\EmFi@id\endcsname{%
      \the\pdflastobj\ltx@space 0 R%
    }%
  \fi
}
%    \end{macrocode}
%    \end{macro}
%    \begin{macro}{\embedfileifobjectexists}
%    \begin{macrocode}
\def\embedfileifobjectexists#1#2{%
  \expandafter\ifx\csname EmFi@#2@#1\endcsname\relax
    \expandafter\ltx@secondoftwo
  \else
    \expandafter\ltx@firstoftwo
  \fi
}
%    \end{macrocode}
%    \end{macro}
%    \begin{macro}{\embedfilegetobject}
%    \begin{macrocode}
\def\embedfilegetobject#1#2{%
  \embedfileifobjectexists{#1}{#2}{%
    \csname EmFi@#2@#1\endcsname
  }{%
    0 0 R%
  }%
}
%    \end{macrocode}
%    \end{macro}
%
%    Initial view of the collection.
%    \begin{macrocode}
\kv@define@key{EmFi}{view}[]{%
  \EdefSanitize\EmFi@temp{#1}%
  \def\EmFi@next{%
    \global\EmFi@collectiontrue
  }%
  \ifx\EmFi@temp\ltx@empty
    \let\EmFi@view\EmFi@S@details
  \else\ifx\EmFi@temp\EmFi@S@details
    \let\EmFi@view\EmFi@S@details
  \else\ifx\EmFi@temp\EmFi@S@tile
    \let\EmFi@view\EmFi@S@tile
  \else\ifx\EmFi@temp\EmFi@S@hidden
    \let\EmFi@view\EmFi@S@hidden
  \else
    \let\EmFi@next\relax
    \EmFi@Error{%
      Unknown value `\EmFi@temp' for key `view'.\MessageBreak
      Supported values: `details', `tile', `hidden'.%
    }\@ehc
  \fi\fi\fi\fi
  \EmFi@next
}
%    \end{macrocode}
%    \begin{macrocode}
\EmFi@DefineKey{initialfile}{}
%    \end{macrocode}
%
%    \begin{macro}{\embedfilesetup}
%    \begin{macrocode}
\def\embedfilesetup{%
  \ifEmFi@finished
    \def\EmFi@next##1{}%
    \EmFi@Error{%
      \string\embedfilefield\ltx@space after \string\embedfilefinish
    }{%
      The list of embedded files is already written.%
    }%
  \else
    \def\EmFi@next{%
      \kvsetkeys{EmFi}%
    }%
  \fi
  \EmFi@next
}
%    \end{macrocode}
%    \end{macro}
%
%    \begin{macro}{\EmFi@schema}
%    \begin{macrocode}
\def\EmFi@schema{}
%    \end{macrocode}
%    \end{macro}
%    \begin{macro}{\EmFi@order}
%    \begin{macrocode}
\gdef\EmFi@order{0}
%    \end{macrocode}
%    \end{macro}
%    \begin{macro}{\EmFi@@order}
%    \begin{macrocode}
\let\EmFi@@order\relax
%    \end{macrocode}
%    \end{macro}
%    \begin{macro}{\EmFi@fieldlist}
%    \begin{macrocode}
\def\EmFi@fieldlist{}
%    \end{macrocode}
%    \end{macro}
%    \begin{macro}{\EmFi@sortcase}
%    \begin{macrocode}
\def\EmFi@sortcase{0}%
%    \end{macrocode}
%    \end{macro}
%    \begin{macro}{\embedfilefield}
%    \begin{macrocode}
\def\embedfilefield#1#2{%
  \ifEmFi@finished
    \EmFi@Error{%
      \string\embedfilefield\ltx@space after \string\embedfilefinish
    }{%
      The list of embedded files is already written.%
    }%
  \else
    \global\EmFi@collectiontrue
    \EdefSanitize\EmFi@key{#1}%
    \expandafter\ifx\csname KV@EmFi@\EmFi@key.prefix\endcsname\relax
      \begingroup
        \count@=\EmFi@order
        \advance\count@ 1 %
        \xdef\EmFi@order{\the\count@}%
        \let\EmFi@title\EmFi@key
        \let\EmFi@type\EmFi@S@text
        \EmFi@visibletrue
        \EmFi@editfalse
        \kvsetkeys{EmFiFi}{#2}%
        \EmFi@convert\EmFi@title\EmFi@title
        \xdef\EmFi@schema{%
          \EmFi@schema
          /\pdf@escapename{\EmFi@key}<<%
            /Subtype/%
            \ifx\EmFi@type\EmFi@S@date D%
            \else\ifx\EmFi@type\EmFi@S@number N%
            \else\ifx\EmFi@type\EmFi@S@file F%
            \else\ifx\EmFi@type\EmFi@S@desc Desc%
            \else\ifx\EmFi@type\EmFi@S@moddate ModDate%
            \else\ifx\EmFi@type\EmFi@S@creationdate CreationDate%
            \else\ifx\EmFi@type\EmFi@S@size Size%
            \else S%
            \fi\fi\fi\fi\fi\fi\fi
            /N(\EmFi@title)%
            \EmFi@@order{\EmFi@order}%
            \ifEmFi@visible
            \else
              /V false%
            \fi
            \ifEmFi@edit
              /E true%
            \fi
          >>%
        }%
        \let\do\relax
        \xdef\EmFi@fieldlist{%
          \EmFi@fieldlist
          \do{\EmFi@key}%
        }%
        \ifx\EmFi@type\EmFi@S@text
          \kv@define@key{EmFi}{\EmFi@key.value}{%
            \EmFi@itemtrue
            \def\EmFi@temp{##1}%
            \EmFi@convert\EmFi@temp\EmFi@temp
            \expandafter\def\csname EmFi@V@#1%
            \expandafter\endcsname\expandafter{%
              \expandafter(\EmFi@temp)%
            }%
          }%
          \EmFi@GlobalKey{EmFi}{\EmFi@key.value}%
        \else\ifx\EmFi@type\EmFi@S@date
          \kv@define@key{EmFi}{\EmFi@key.value}{%
            \EmFi@itemtrue
            \def\EmFi@temp{##1}%
            \EmFi@convert\EmFi@temp\EmFi@temp
            \expandafter\def\csname EmFi@V@#1%
            \expandafter\endcsname\expandafter{%
              \expandafter(\EmFi@temp)%
            }%
          }%
          \EmFi@GlobalKey{EmFi}{\EmFi@key.value}%
        \else\ifx\EmFi@type\EmFi@S@number
          \kv@define@key{EmFi}{\EmFi@key.value}{%
            \EmFi@itemtrue
            \expandafter\EdefSanitize\csname EmFi@V@#1\endcsname{ ##1}%
          }%
          \EmFi@GlobalKey{EmFi}{\EmFi@key.value}%
        \fi\fi\fi
        \kv@define@key{EmFi}{\EmFi@key.prefix}{%
          \EmFi@itemtrue
          \expandafter\def\csname EmFi@P@#1\endcsname{##1}%
        }%
        \EmFi@GlobalKey{EmFi}{\EmFi@key.prefix}%
        \kv@define@key{EmFiSo}{\EmFi@key}[ascending]{%
          \EdefSanitize\EmFi@temp{##1}%
          \ifx\EmFi@temp\EmFi@S@ascending
            \def\EmFi@temp{true}%
          \else\ifx\EmFi@temp\EmFi@S@descending
            \def\EmFi@temp{false}%
          \else
            \def\EmFi@temp{}%
            \EmFi@Error{%
              Unknown sort order `\EmFi@temp'.\MessageBreak
              Supported values: `\EmFi@S@ascending', %
              `\EmFi@S@descending
            }\@ehc
          \fi\fi
          \ifx\EmFi@temp\ltx@empty
          \else
            \xdef\EmFi@sortkeys{%
              \EmFi@sortkeys
              /\pdf@escapename{#1}%
            }%
            \ifx\EmFi@sortorders\ltx@empty
              \global\let\EmFi@sortorders\EmFi@temp
              \gdef\EmFi@sortcase{1}%
            \else
              \xdef\EmFi@sortorders{%
                \EmFi@sortorders
                \ltx@space
                \EmFi@temp
              }%
              \xdef\EmFi@sortcase{2}%
            \fi
          \fi
        }%
        \EmFi@GlobalDefaultKey{EmFiSo}\EmFi@key
      \endgroup
    \else
      \EmFi@Error{%
        Field `\EmFi@key' is already defined%
      }\@ehc
    \fi
  \fi
}
%    \end{macrocode}
%    \end{macro}
%    \begin{macrocode}
\kv@define@key{EmFiFi}{type}{%
  \EdefSanitize\EmFi@temp{#1}%
  \ifx\EmFi@temp\EmFi@S@text
    \let\EmFi@type\EmFi@temp
  \else\ifx\EmFi@temp\EmFi@S@date
    \let\EmFi@type\EmFi@temp
  \else\ifx\EmFi@temp\EmFi@S@number
    \let\EmFi@type\EmFi@temp
  \else\ifx\EmFi@temp\EmFi@S@file
    \let\EmFi@type\EmFi@temp
  \else\ifx\EmFi@temp\EmFi@S@desc
    \let\EmFi@type\EmFi@temp
  \else\ifx\EmFi@temp\EmFi@S@moddate
    \let\EmFi@type\EmFi@temp
  \else\ifx\EmFi@temp\EmFi@S@creationdate
    \let\EmFi@type\EmFi@temp
  \else\ifx\EmFi@temp\EmFi@S@size
    \let\EmFi@type\EmFi@temp
  \else
    \EmFi@Error{%
      Unknown type `\EmFi@temp'.\MessageBreak
      Supported types: `text', `date', `number', `file',\MessageBreak
      `desc', `moddate', `creationdate', `size'%
    }%
  \fi\fi\fi\fi\fi\fi\fi\fi
}
%    \end{macrocode}
%    \begin{macrocode}
\kv@define@key{EmFiFi}{title}{%
  \def\EmFi@title{#1}%
}
%    \end{macrocode}
%    \begin{macro}{\EmFi@setboolean}
%    \begin{macrocode}
\def\EmFi@setboolean#1#2{%
  \EdefSanitize\EmFi@temp{#2}%
  \ifx\EmFi@temp\EmFi@S@true
    \csname EmFi@#1true\endcsname
  \else
    \ifx\EmFi@temp\EmFi@S@false
      \csname EmFi@#1false\endcsname
    \else
      \EmFi@Error{%
        Unknown value `\EmFi@temp' for key `#1'.\MessageBreak
        Supported values: `true', `false'%
      }\@ehc
    \fi
  \fi
}
%    \end{macrocode}
%    \end{macro}
%    \begin{macrocode}
\kv@define@key{EmFiFi}{visible}[true]{%
  \EmFi@setboolean{visible}{#1}%
}
%    \end{macrocode}
%    \begin{macrocode}
\kv@define@key{EmFiFi}{edit}[true]{%
  \EmFi@setboolean{edit}{#1}%
}
%    \end{macrocode}
%
%    \begin{macro}{\EmFi@sortkeys}
%    \begin{macrocode}
\def\EmFi@sortkeys{}
%    \end{macrocode}
%    \end{macro}
%    \begin{macro}{\EmFi@sortorders}
%    \begin{macrocode}
\def\EmFi@sortorders{}
%    \end{macrocode}
%    \end{macro}
%    \begin{macro}{\embedfilesort}
%    \begin{macrocode}
\def\embedfilesort{%
  \kvsetkeys{EmFiSo}%
}
%    \end{macrocode}
%    \end{macro}
%
% \subsection{Embed the file}
%
%    \begin{macro}{\embedfile}
%    \begin{macrocode}
\def\embedfile{%
  \ltx@ifnextchar[\EmFi@embedfile{\EmFi@embedfile[]}%
}
%    \end{macrocode}
%    \end{macro}
%    \begin{macro}{\EmFi@embedfile}
%    \begin{macrocode}
\def\EmFi@embedfile[#1]#2{%
  \ifEmFi@finished
    \EmFi@Error{%
      \string\embedfile\ltx@space after \string\embedfilefinish
    }{%
      The list of embedded files is already written.%
    }%
  \else
    \begingroup
      \def\EmFi@file{#2}%
      \kvsetkeys{EmFi}{#1}%
      \expandafter\expandafter\expandafter
      \ifx\expandafter\expandafter\expandafter
          \\\pdf@filesize{\EmFi@file}\\%
        \EmFi@Error{%
          File `\EmFi@file' not found%
        }{%
          The unknown file is not embedded.%
        }%
      \else
        \edef\EmFi@@filespec{%
          \pdf@escapestring{\EmFi@filespec}%
        }%
        \ifx\EmFi@ucfilespec\ltx@empty
          \let\EmFi@@ucfilespec\ltx@empty
        \else
          \EmFi@convert\EmFi@ucfilespec\EmFi@@ucfilespec
        \fi
        \ifx\EmFi@desc\ltx@empty
          \let\EmFi@@desc\ltx@empty
        \else
          \EmFi@convert\EmFi@desc\EmFi@@desc
        \fi
        \ifEmFi@item
          \let\do\EmFi@do
          \immediate\pdfobj{%
            <<%
              \EmFi@fieldlist
            >>%
          }%
          \edef\EmFi@ci{\the\pdflastobj}%
        \fi
        \immediate\pdfobj stream attr{%
          /Type/EmbeddedFile%
          \ifx\EmFi@mimetype\ltx@empty
          \else
            /Subtype/\pdf@escapename{\EmFi@mimetype}%
          \fi
          /Params<<%
            /ModDate(\pdf@filemoddate{\EmFi@file})%
            /Size \pdf@filesize{\EmFi@file}%
            /CheckSum<\pdf@filemdfivesum{\EmFi@file}>%
          >>%
        }file{\EmFi@file}\relax
        \EmFi@defobj{EmbeddedFile}%
        \immediate\pdfobj{%
          <<%
            /Type/Filespec%
            \ifx\EmFi@filesystem\ltx@empty
            \else
            /FS/\pdf@escapename{\EmFi@filesystem}%
            \fi
            /F(\EmFi@@filespec)%
            \ifx\EmFi@@ucfilespec\ltx@empty
            \else
              /UF(\EmFi@@ucfilespec)%
            \fi
            \ifx\EmFi@@desc\ltx@empty
            \else
              /Desc(\EmFi@@desc)%
            \fi
            /EF<<%
              /F \the\pdflastobj\ltx@space 0 R%
            >>%
            \ifEmFi@item
              /CI \EmFi@ci\ltx@space 0 R%
            \fi
          >>%
        }%
        \EmFi@defobj{Filespec}%
        \EmFi@add{%
          \EmFi@@filespec
        }{\the\pdflastobj\ltx@space 0 R}%
      \fi
    \endgroup
  \fi
}
%    \end{macrocode}
%    \end{macro}
%    \begin{macro}{\EmFi@do}
%    \begin{macrocode}
\def\EmFi@do#1{%
  \expandafter\ifx\csname EmFi@P@#1\endcsname\relax
    \expandafter\ifx\csname EmFi@V@#1\endcsname\relax
    \else
      /\pdf@escapename{#1}\csname EmFi@V@#1\endcsname
    \fi
  \else
    /\pdf@escapename{#1}<<%
      \expandafter\ifx\csname EmFi@V@#1\endcsname\relax
      \else
        /D\csname EmFi@V@#1\endcsname
      \fi
      /P(\csname EmFi@P@#1\endcsname)%
    >>%
  \fi
}
%    \end{macrocode}
%    \end{macro}
%    \begin{macro}{\EmFi@convert}
%    \begin{macrocode}
\def\EmFi@convert#1#2{%
  \ifnum\pdf@strcmp{\EmFi@stringmethod}{psd}=0 %
    \pdfstringdef\EmFi@temp{#1}%
    \let#2\EmFi@temp
  \else
    \edef#2{\pdf@escapestring{#1}}%
  \fi
}
%    \end{macrocode}
%    \end{macro}
%    \begin{macrocode}
\global\let\EmFi@list\ltx@empty
%    \end{macrocode}
%    \begin{macro}{\EmFi@add}
%    Sorting is done by the insertion sort algorithm.
%    Probably the sorting could be done more reliable.
%    However, the PDF specification is not too clear to me
%    regarding precise sorting rules (how to deal with
%    different encodings, escaped characters, \dots).
%    \begin{macrocode}
\def\EmFi@add#1#2{%
  \begingroup
    \ifx\EmFi@list\ltx@empty
      \xdef\EmFi@list{\noexpand\do{#1}{#2}}%
    \else
      \def\do##1##2{%
        \ifnum\pdf@strcmp{##1}{#1}>0 %
          \edef\x{%
            \toks@{%
              \the\toks@%
              \noexpand\do{#1}{#2}%
              \noexpand\do{##1}{##2}%
            }%
          }%
          \x
          \def\do####1####2{%
            \toks@\expandafter{\the\toks@\do{####1}{####2}}%
          }%
          \def\stop{%
            \xdef\EmFi@list{\the\toks@}%
          }%
        \else
          \toks@\expandafter{\the\toks@\do{##1}{##2}}%
        \fi
      }%
      \def\stop{%
        \xdef\EmFi@list{\the\toks@\noexpand\do{#1}{#2}}%
      }%
      \toks@{}%
      \EmFi@list\stop
    \fi
  \endgroup
}
%    \end{macrocode}
%    \end{macro}
%    \begin{macro}{\embedfilefinish}
%    \begin{macrocode}
\def\embedfilefinish{%
  \ifEmFi@finished
    \EmFi@Error{%
      Too many invocations of \string\embedfilefinish
    }{%
      The list of embedded files is already written.%
    }%
  \else
    \ifx\EmFi@list\ltx@empty
    \else
%    \end{macrocode}
%    Write |/EmbeddedFiles| entry.
%    \begin{macrocode}
      \global\EmFi@finishedtrue
      \begingroup
        \def\do##1##2{%
          (##1)##2%
        }%
        \immediate\pdfobj{%
          <<%
            /Names[\EmFi@list]%
          >>%
        }%
        \pdfnames{%
          /EmbeddedFiles \the\pdflastobj\ltx@space 0 R%
        }%
      \endgroup
%    \end{macrocode}
%    Write collection objects.
%    \begin{macrocode}
      \ifx\EmFi@initialfile\ltx@empty
      \else
        \EmFi@collectiontrue
      \fi
      \ifEmFi@collection
        \ifx\EmFi@initialfile\ltx@empty
          \let\EmFi@@initialfile\ltx@empty
        \else
          \edef\EmFi@@initialfile{%
            \pdf@escapestring{\EmFi@initialfile}%
          }%
        \fi
%    \end{macrocode}
%    Look for initial file among the embedded files.
%    \begin{macrocode}
        \begingroup
          \let\f=N%
          \def\do##1##2{%
            \def\x{##1}%
            \ifx\x\EmFi@@initialfile
              \let\f=Y%
              \let\do\ltx@gobbletwo
            \fi
          }%
          \EmFi@list
        \expandafter\endgroup
        \ifx\f Y%
        \else
          \@PackageWarningNoLine{embedfile}{%
            Missing initial file `\EmFi@initialfile'\MessageBreak
            among the embedded files%
          }%
          \let\EmFi@initialfile\ltx@empty
          \let\EmFi@@initialfile\ltx@empty
        \fi
        \ifcase\EmFi@sortcase
          \def\EmFi@temp{}%
        \or
          \def\EmFi@temp{%
            /S\EmFi@sortkeys
            /A \EmFi@sortorders
          }%
        \else
          \def\EmFi@temp{%
            /S[\EmFi@sortkeys]%
            /A[\EmFi@sortorders]%
          }%
        \fi
        \def\EmFi@@order##1{%
          \ifnum\EmFi@order>1 %
            /O ##1%
          \fi
        }%
        \immediate\pdfobj{%
          <<%
            \ifx\EmFi@schema\ltx@empty
            \else
              /Schema<<\EmFi@schema>>%
            \fi
            \ifx\EmFi@@initialfile\ltx@empty
            \else
              /D(\EmFi@@initialfile)%
            \fi
            \ifx\EmFi@view\EmFi@S@tile
              /View/T%
            \else\ifx\EmFi@view\EmFi@S@hidden
              /View/H%
            \fi\fi
            \ifx\EmFi@temp\ltx@empty
              \EmFi@temp
            \else
              /Sort<<\EmFi@temp>>%
            \fi
          >>%
        }%
        \pdfcatalog{%
          /Collection \the\pdflastobj\ltx@space0 R%
        }%
      \fi
    \fi
  \fi
}
%    \end{macrocode}
%    \end{macro}
%    \begin{macrocode}
\begingroup\expandafter\expandafter\expandafter\endgroup
\expandafter\ifx\csname AtEndDocument\endcsname\relax
\else
  \AtEndDocument{\embedfilefinish}%
\fi
%    \end{macrocode}
%
%    \begin{macrocode}
\EmFi@AtEnd%
%</package>
%    \end{macrocode}
%
% \section{Test}
%
% \subsection{Catcode checks for loading}
%
%    \begin{macrocode}
%<*test1>
%    \end{macrocode}
%    \begin{macrocode}
\catcode`\{=1 %
\catcode`\}=2 %
\catcode`\#=6 %
\catcode`\@=11 %
\expandafter\ifx\csname count@\endcsname\relax
  \countdef\count@=255 %
\fi
\expandafter\ifx\csname @gobble\endcsname\relax
  \long\def\@gobble#1{}%
\fi
\expandafter\ifx\csname @firstofone\endcsname\relax
  \long\def\@firstofone#1{#1}%
\fi
\expandafter\ifx\csname loop\endcsname\relax
  \expandafter\@firstofone
\else
  \expandafter\@gobble
\fi
{%
  \def\loop#1\repeat{%
    \def\body{#1}%
    \iterate
  }%
  \def\iterate{%
    \body
      \let\next\iterate
    \else
      \let\next\relax
    \fi
    \next
  }%
  \let\repeat=\fi
}%
\def\RestoreCatcodes{}
\count@=0 %
\loop
  \edef\RestoreCatcodes{%
    \RestoreCatcodes
    \catcode\the\count@=\the\catcode\count@\relax
  }%
\ifnum\count@<255 %
  \advance\count@ 1 %
\repeat

\def\RangeCatcodeInvalid#1#2{%
  \count@=#1\relax
  \loop
    \catcode\count@=15 %
  \ifnum\count@<#2\relax
    \advance\count@ 1 %
  \repeat
}
\def\RangeCatcodeCheck#1#2#3{%
  \count@=#1\relax
  \loop
    \ifnum#3=\catcode\count@
    \else
      \errmessage{%
        Character \the\count@\space
        with wrong catcode \the\catcode\count@\space
        instead of \number#3%
      }%
    \fi
  \ifnum\count@<#2\relax
    \advance\count@ 1 %
  \repeat
}
\def\space{ }
\expandafter\ifx\csname LoadCommand\endcsname\relax
  \def\LoadCommand{\input embedfile.sty\relax}%
\fi
\def\Test{%
  \RangeCatcodeInvalid{0}{47}%
  \RangeCatcodeInvalid{58}{64}%
  \RangeCatcodeInvalid{91}{96}%
  \RangeCatcodeInvalid{123}{255}%
  \catcode`\@=12 %
  \catcode`\\=0 %
  \catcode`\%=14 %
  \LoadCommand
  \RangeCatcodeCheck{0}{36}{15}%
  \RangeCatcodeCheck{37}{37}{14}%
  \RangeCatcodeCheck{38}{47}{15}%
  \RangeCatcodeCheck{48}{57}{12}%
  \RangeCatcodeCheck{58}{63}{15}%
  \RangeCatcodeCheck{64}{64}{12}%
  \RangeCatcodeCheck{65}{90}{11}%
  \RangeCatcodeCheck{91}{91}{15}%
  \RangeCatcodeCheck{92}{92}{0}%
  \RangeCatcodeCheck{93}{96}{15}%
  \RangeCatcodeCheck{97}{122}{11}%
  \RangeCatcodeCheck{123}{255}{15}%
  \RestoreCatcodes
}
\Test
\csname @@end\endcsname
\end
%    \end{macrocode}
%    \begin{macrocode}
%</test1>
%    \end{macrocode}
%
% \subsection{Simple test}
%
%    \begin{macrocode}
%<*test2>
\input embedfile.sty\relax
\embedfile[%
  stringmethod=escape,%
  mimetype=plain/text,%
  desc={LaTeX docstrip source archive for package `embedfile'},%
  id={embedfile.dtx}%
]{embedfile.dtx}
\nopagenumbers
Test (plain-\TeX): {\tt embedfile.dtx} should be embedded.%

\def\Test#1{%
  \par
  \embedfileifobjectexists{embedfile.dtx}{#1}{%
    Object #1 (embedfile.dtx): %
    \embedfilegetobject{embedfile.dtx}{#1}%
  }{%
    \errmessage{Missing object #1 (embedfile.dtx)}%
  }%
}
\Test{EmbeddedFile}
\Test{Filespec}
\embedfilefinish
\bye
%</test2>
%    \end{macrocode}
%    \begin{macrocode}
%<*test3>
\NeedsTeXFormat{LaTeX2e}
\let\SavedJobname\jobname
\def\jobname{embedfile}
\RequirePackage{dtx-attach}[2011/04/13]
\let\jobname\SavedJobname
\documentclass{minimal}
\begin{document}
  Test (\LaTeX): \texttt{embedfile.dtx} should be embedded.%
\end{document}
%</test3>
%    \end{macrocode}
%
% \subsection{Test for \iniTeX}
%
%    \begin{macrocode}
%<*test4>
\catcode`\{=1 %
\catcode`\}=2 %
\input ifluatex.sty %
\ifluatex
  \directlua{%
    tex.enableprimitives('', {%
      'pdflastobj',%
      'pdfnames',%
      'pdfobj',%
      'pdfoutput'%
    })%
  }%
\fi
\pdfoutput=1 %
\input embedfile.sty %
\shipout\hbox{}
\embedfile[%
  stringmethod=escape,%
  mimetype=plain/text,%
  desc={iniTeX source},%
]{\jobname.tex}
\embedfilefinish
\end
%</test4>
%    \end{macrocode}
%
% \section{Installation}
%
% \subsection{Download}
%
% \paragraph{Package.} This package is available on
% CTAN\footnote{\url{ftp://ftp.ctan.org/tex-archive/}}:
% \begin{description}
% \item[\CTAN{macros/latex/contrib/oberdiek/embedfile.dtx}] The source file.
% \item[\CTAN{macros/latex/contrib/oberdiek/embedfile.pdf}] Documentation.
% \end{description}
%
%
% \paragraph{Bundle.} All the packages of the bundle `oberdiek'
% are also available in a TDS compliant ZIP archive. There
% the packages are already unpacked and the documentation files
% are generated. The files and directories obey the TDS standard.
% \begin{description}
% \item[\CTAN{install/macros/latex/contrib/oberdiek.tds.zip}]
% \end{description}
% \emph{TDS} refers to the standard ``A Directory Structure
% for \TeX\ Files'' (\CTAN{tds/tds.pdf}). Directories
% with \xfile{texmf} in their name are usually organized this way.
%
% \subsection{Bundle installation}
%
% \paragraph{Unpacking.} Unpack the \xfile{oberdiek.tds.zip} in the
% TDS tree (also known as \xfile{texmf} tree) of your choice.
% Example (linux):
% \begin{quote}
%   |unzip oberdiek.tds.zip -d ~/texmf|
% \end{quote}
%
% \paragraph{Script installation.}
% Check the directory \xfile{TDS:scripts/oberdiek/} for
% scripts that need further installation steps.
% Package \xpackage{attachfile2} comes with the Perl script
% \xfile{pdfatfi.pl} that should be installed in such a way
% that it can be called as \texttt{pdfatfi}.
% Example (linux):
% \begin{quote}
%   |chmod +x scripts/oberdiek/pdfatfi.pl|\\
%   |cp scripts/oberdiek/pdfatfi.pl /usr/local/bin/|
% \end{quote}
%
% \subsection{Package installation}
%
% \paragraph{Unpacking.} The \xfile{.dtx} file is a self-extracting
% \docstrip\ archive. The files are extracted by running the
% \xfile{.dtx} through \plainTeX:
% \begin{quote}
%   \verb|tex embedfile.dtx|
% \end{quote}
%
% \paragraph{TDS.} Now the different files must be moved into
% the different directories in your installation TDS tree
% (also known as \xfile{texmf} tree):
% \begin{quote}
% \def\t{^^A
% \begin{tabular}{@{}>{\ttfamily}l@{ $\rightarrow$ }>{\ttfamily}l@{}}
%   embedfile.sty & tex/generic/oberdiek/embedfile.sty\\
%   dtx-attach.sty & tex/generic/oberdiek/dtx-attach.sty\\
%   embedfile.pdf & doc/latex/oberdiek/embedfile.pdf\\
%   embedfile-example-plain.tex & doc/latex/oberdiek/embedfile-example-plain.tex\\
%   embedfile-example-collection.tex & doc/latex/oberdiek/embedfile-example-collection.tex\\
%   test/embedfile-test1.tex & doc/latex/oberdiek/test/embedfile-test1.tex\\
%   test/embedfile-test2.tex & doc/latex/oberdiek/test/embedfile-test2.tex\\
%   test/embedfile-test3.tex & doc/latex/oberdiek/test/embedfile-test3.tex\\
%   test/embedfile-test4.tex & doc/latex/oberdiek/test/embedfile-test4.tex\\
%   embedfile.dtx & source/latex/oberdiek/embedfile.dtx\\
% \end{tabular}^^A
% }^^A
% \sbox0{\t}^^A
% \ifdim\wd0>\linewidth
%   \begingroup
%     \advance\linewidth by\leftmargin
%     \advance\linewidth by\rightmargin
%   \edef\x{\endgroup
%     \def\noexpand\lw{\the\linewidth}^^A
%   }\x
%   \def\lwbox{^^A
%     \leavevmode
%     \hbox to \linewidth{^^A
%       \kern-\leftmargin\relax
%       \hss
%       \usebox0
%       \hss
%       \kern-\rightmargin\relax
%     }^^A
%   }^^A
%   \ifdim\wd0>\lw
%     \sbox0{\small\t}^^A
%     \ifdim\wd0>\linewidth
%       \ifdim\wd0>\lw
%         \sbox0{\footnotesize\t}^^A
%         \ifdim\wd0>\linewidth
%           \ifdim\wd0>\lw
%             \sbox0{\scriptsize\t}^^A
%             \ifdim\wd0>\linewidth
%               \ifdim\wd0>\lw
%                 \sbox0{\tiny\t}^^A
%                 \ifdim\wd0>\linewidth
%                   \lwbox
%                 \else
%                   \usebox0
%                 \fi
%               \else
%                 \lwbox
%               \fi
%             \else
%               \usebox0
%             \fi
%           \else
%             \lwbox
%           \fi
%         \else
%           \usebox0
%         \fi
%       \else
%         \lwbox
%       \fi
%     \else
%       \usebox0
%     \fi
%   \else
%     \lwbox
%   \fi
% \else
%   \usebox0
% \fi
% \end{quote}
% If you have a \xfile{docstrip.cfg} that configures and enables \docstrip's
% TDS installing feature, then some files can already be in the right
% place, see the documentation of \docstrip.
%
% \subsection{Refresh file name databases}
%
% If your \TeX~distribution
% (\teTeX, \mikTeX, \dots) relies on file name databases, you must refresh
% these. For example, \teTeX\ users run \verb|texhash| or
% \verb|mktexlsr|.
%
% \subsection{Some details for the interested}
%
% \paragraph{Attached source.}
%
% The PDF documentation on CTAN also includes the
% \xfile{.dtx} source file. It can be extracted by
% AcrobatReader 6 or higher. Another option is \textsf{pdftk},
% e.g. unpack the file into the current directory:
% \begin{quote}
%   \verb|pdftk embedfile.pdf unpack_files output .|
% \end{quote}
%
% \paragraph{Unpacking with \LaTeX.}
% The \xfile{.dtx} chooses its action depending on the format:
% \begin{description}
% \item[\plainTeX:] Run \docstrip\ and extract the files.
% \item[\LaTeX:] Generate the documentation.
% \end{description}
% If you insist on using \LaTeX\ for \docstrip\ (really,
% \docstrip\ does not need \LaTeX), then inform the autodetect routine
% about your intention:
% \begin{quote}
%   \verb|latex \let\install=y\input{embedfile.dtx}|
% \end{quote}
% Do not forget to quote the argument according to the demands
% of your shell.
%
% \paragraph{Generating the documentation.}
% You can use both the \xfile{.dtx} or the \xfile{.drv} to generate
% the documentation. The process can be configured by the
% configuration file \xfile{ltxdoc.cfg}. For instance, put this
% line into this file, if you want to have A4 as paper format:
% \begin{quote}
%   \verb|\PassOptionsToClass{a4paper}{article}|
% \end{quote}
% An example follows how to generate the
% documentation with pdf\LaTeX:
% \begin{quote}
%\begin{verbatim}
%pdflatex embedfile.dtx
%makeindex -s gind.ist embedfile.idx
%pdflatex embedfile.dtx
%makeindex -s gind.ist embedfile.idx
%pdflatex embedfile.dtx
%\end{verbatim}
% \end{quote}
%
% \section{Catalogue}
%
% The following XML file can be used as source for the
% \href{http://mirror.ctan.org/help/Catalogue/catalogue.html}{\TeX\ Catalogue}.
% The elements \texttt{caption} and \texttt{description} are imported
% from the original XML file from the Catalogue.
% The name of the XML file in the Catalogue is \xfile{embedfile.xml}.
%    \begin{macrocode}
%<*catalogue>
<?xml version='1.0' encoding='us-ascii'?>
<!DOCTYPE entry SYSTEM 'catalogue.dtd'>
<entry datestamp='$Date$' modifier='$Author$' id='embedfile'>
  <name>embedfile</name>
  <caption>Embed files into PDF.</caption>
  <authorref id='auth:oberdiek'/>
  <copyright owner='Heiko Oberdiek' year='2006-2011'/>
  <license type='lppl1.3'/>
  <version number='2.6'/>
  <description>
    This package embeds files in a PDF document, using the PDF
    format's embedding operation (note the contrast with the attach
    operation used by the <xref refid='attachfile'>attachfile</xref>
    and <xref refid='attachfile2'>attachfile2</xref> packages).
    Currently only <xref refid='pdftex'>pdfTeX</xref> &gt;=1.30, in
    PDF mode, is supported.

    <p/>
    The package is part of the <xref refid='oberdiek'>oberdiek</xref>
    bundle.
  </description>
  <documentation details='Package documentation'
      href='ctan:/macros/latex/contrib/oberdiek/embedfile.pdf'/>
  <ctan file='true' path='/macros/latex/contrib/oberdiek/embedfile.dtx'/>
  <miktex location='oberdiek'/>
  <texlive location='oberdiek'/>
  <install path='/macros/latex/contrib/oberdiek/oberdiek.tds.zip'/>
</entry>
%</catalogue>
%    \end{macrocode}
%
% \begin{thebibliography}{9}
% \bibitem{attachfile}
%   Scott Pakin: \textit{The \xpackage{attachfile} package};
%   2005/02/20 v1.2;
%   \CTAN{macros/latex/contrib/attachfile/}.
%
% \bibitem{attachfile2}
%   Heiko Oberdiek: \textit{The \xpackage{attachfile2} package};
%   2006/08/16 v2.2;
%   \CTAN{macros/latex/contrib/oberdiek/attachfile2.pdf}.
%
% \bibitem{pdfspec}
%   Adobe Systems Incorporated:
%   \href{http://www.adobe.com/devnet/acrobat/pdfs/pdf_reference.pdf}%
%       {\textit{PDF Reference, Sixth Edition, Version 1.7}},%
%   Oktober 2006;
%   \url{http://www.adobe.com/devnet/pdf/pdf_reference.html}.
%
% \bibitem{mime}
%   Network Working Group: RFC 2046, \textit{Multipurpose Internet Mail Extensions (MIME) Part Two:
%     Media Types}, November 1996; \url{http://www.rfc-editor.org/}.
%
% \bibitem{types}
%   IANA (Internet Assigned Numbers Authority):
%   \textit{MIME Media Types}, May 2006;
%   \url{http://www.iana.org/assignments/media-types/}.
%
% \end{thebibliography}
%
% \begin{History}
%   \begin{Version}{2006/08/16 v1.0}
%   \item
%     First public version.
%   \end{Version}
%   \begin{Version}{2007/04/11 v1.1}
%   \item
%     Line ends sanitized.
%   \end{Version}
%   \begin{Version}{2007/09/09 v1.2}
%   \item
%     Fixes for plain-TeX, wrapper for package \xpackage{keyval} added.
%   \item
%     Catcode section rewritten.
%   \end{Version}
%   \begin{Version}{2007/10/28 v2.0}
%   \item
%     Collection support added (PDF 1.7).
%   \end{Version}
%   \begin{Version}{2007/10/29 v2.1}
%   \item
%     Export of object references by adding new option \xoption{id} and
%     new macros \cs{embedfileifobjectexists} and \cs{embedfilegetobject}.
%   \end{Version}
%   \begin{Version}{2007/11/11 v2.2}
%   \item
%     Use of package \xpackage{pdftexcmds} for \LuaTeX\ support.
%   \end{Version}
%   \begin{Version}{2007/11/25 v2.3}
%   \item
%     Fix in use of \cs{pdf@filesize}, bug introduced in previous version.
%   \end{Version}
%   \begin{Version}{2009/09/25 v2.4}
%   \item
%     Bug fix: If \xpackage{hyperref} is used with option \xoption{unicode},
%     the Unicode encoded file name causes trouble. Therefore \cs{pdfstringdef}
%     is now never used for option \xoption{filespec}, always method
%     \texttt{escape} is applied (Peter Cibulka).
%   \item
%     Bug fix for \xoption{initialfile}.
%   \item
%     Bug fix for file names in \texttt{/EmbeddedFiles}.
%   \item
%     New option \xoption{ucfilespec} for file name support in Unicode
%     (since PDF 1.7).
%   \end{Version}
%   \begin{Version}{2010/03/01 v2.5}
%   \item
%     Compatibility for \iniTeX.
%   \item
%     Package \xpackage{keyval} replaced by packages \xpackage{kvsetkeys}
%     and \xpackage{kvdefinekeys} because of compatibility for \iniTeX.
%   \item
%     TDS location moved from TDS:tex/latex/oberdiek/embedfile.sty to
%     TDS:tex/generic/oberdiek/embedfile.sty.
%   \end{Version}
%   \begin{Version}{2011/04/13 v2.6}
%   \item
%     Docu fixes (thanks Hans-Martin M\"unch).
%   \end{Version}
% \end{History}
%
% \PrintIndex
%
% \Finale
\endinput

%        (quote the arguments according to the demands of your shell)
%
% Documentation:
%    (a) If embedfile.drv is present:
%           latex embedfile.drv
%    (b) Without embedfile.drv:
%           latex embedfile.dtx; ...
%    The class ltxdoc loads the configuration file ltxdoc.cfg
%    if available. Here you can specify further options, e.g.
%    use A4 as paper format:
%       \PassOptionsToClass{a4paper}{article}
%
%    Programm calls to get the documentation (example):
%       pdflatex embedfile.dtx
%       makeindex -s gind.ist embedfile.idx
%       pdflatex embedfile.dtx
%       makeindex -s gind.ist embedfile.idx
%       pdflatex embedfile.dtx
%
% Installation:
%    TDS:tex/generic/oberdiek/embedfile.sty
%    TDS:tex/generic/oberdiek/dtx-attach.sty
%    TDS:doc/latex/oberdiek/embedfile.pdf
%    TDS:doc/latex/oberdiek/embedfile-example-plain.tex
%    TDS:doc/latex/oberdiek/embedfile-example-collection.tex
%    TDS:doc/latex/oberdiek/test/embedfile-test1.tex
%    TDS:doc/latex/oberdiek/test/embedfile-test2.tex
%    TDS:doc/latex/oberdiek/test/embedfile-test3.tex
%    TDS:doc/latex/oberdiek/test/embedfile-test4.tex
%    TDS:source/latex/oberdiek/embedfile.dtx
%
%<*ignore>
\begingroup
  \catcode123=1 %
  \catcode125=2 %
  \def\x{LaTeX2e}%
\expandafter\endgroup
\ifcase 0\ifx\install y1\fi\expandafter
         \ifx\csname processbatchFile\endcsname\relax\else1\fi
         \ifx\fmtname\x\else 1\fi\relax
\else\csname fi\endcsname
%</ignore>
%<*install>
\input docstrip.tex
\Msg{************************************************************************}
\Msg{* Installation}
\Msg{* Package: embedfile 2011/04/13 v2.6 Embed files into PDF (HO)}
\Msg{************************************************************************}

\keepsilent
\askforoverwritefalse

\let\MetaPrefix\relax
\preamble

This is a generated file.

Project: embedfile
Version: 2011/04/13 v2.6

Copyright (C) 2006-2011 by
   Heiko Oberdiek <heiko.oberdiek at googlemail.com>

This work may be distributed and/or modified under the
conditions of the LaTeX Project Public License, either
version 1.3c of this license or (at your option) any later
version. This version of this license is in
   http://www.latex-project.org/lppl/lppl-1-3c.txt
and the latest version of this license is in
   http://www.latex-project.org/lppl.txt
and version 1.3 or later is part of all distributions of
LaTeX version 2005/12/01 or later.

This work has the LPPL maintenance status "maintained".

This Current Maintainer of this work is Heiko Oberdiek.

The Base Interpreter refers to any `TeX-Format',
because some files are installed in TDS:tex/generic//.

This work consists of the main source file embedfile.dtx
and the derived files
   embedfile.sty, embedfile.pdf, embedfile.ins, embedfile.drv,
   dtx-attach.sty, embedfile-example-plain.tex,
   embedfile-example-collection.tex, embedfile-test1.tex,
   embedfile-test2.tex, embedfile-test3.tex,
   embedfile-test4.tex.

\endpreamble
\let\MetaPrefix\DoubleperCent

\generate{%
  \file{embedfile.ins}{\from{embedfile.dtx}{install}}%
  \file{embedfile.drv}{\from{embedfile.dtx}{driver}}%
  \usedir{tex/generic/oberdiek}%
  \file{embedfile.sty}{\from{embedfile.dtx}{package}}%
  \usedir{tex/latex/oberdiek}%
  \file{dtx-attach.sty}{\from{embedfile.dtx}{dtxattach}}%
  \usedir{doc/latex/oberdiek}%
  \file{embedfile-example-plain.tex}{\from{embedfile.dtx}{exampleplain}}%
  \file{embedfile-example-collection.tex}{\from{embedfile.dtx}{examplecollection}}%
  \usedir{doc/latex/oberdiek/test}%
  \file{embedfile-test1.tex}{\from{embedfile.dtx}{test1}}%
  \file{embedfile-test2.tex}{\from{embedfile.dtx}{test2}}%
  \file{embedfile-test3.tex}{\from{embedfile.dtx}{test3}}%
  \file{embedfile-test4.tex}{\from{embedfile.dtx}{test4}}%
  \nopreamble
  \nopostamble
  \usedir{source/latex/oberdiek/catalogue}%
  \file{embedfile.xml}{\from{embedfile.dtx}{catalogue}}%
}

\catcode32=13\relax% active space
\let =\space%
\Msg{************************************************************************}
\Msg{*}
\Msg{* To finish the installation you have to move the following}
\Msg{* files into a directory searched by TeX:}
\Msg{*}
\Msg{*     embedfile.sty, dtx-attach.sty}
\Msg{*}
\Msg{* To produce the documentation run the file `embedfile.drv'}
\Msg{* through LaTeX.}
\Msg{*}
\Msg{* Happy TeXing!}
\Msg{*}
\Msg{************************************************************************}

\endbatchfile
%</install>
%<*ignore>
\fi
%</ignore>
%<*driver>
\NeedsTeXFormat{LaTeX2e}
\ProvidesFile{embedfile.drv}%
  [2011/04/13 v2.6 Embed files into PDF (HO)]%
\documentclass{ltxdoc}
\usepackage{holtxdoc}[2011/11/22]
\begin{document}
  \DocInput{embedfile.dtx}%
\end{document}
%</driver>
% \fi
%
% \CheckSum{1363}
%
% \CharacterTable
%  {Upper-case    \A\B\C\D\E\F\G\H\I\J\K\L\M\N\O\P\Q\R\S\T\U\V\W\X\Y\Z
%   Lower-case    \a\b\c\d\e\f\g\h\i\j\k\l\m\n\o\p\q\r\s\t\u\v\w\x\y\z
%   Digits        \0\1\2\3\4\5\6\7\8\9
%   Exclamation   \!     Double quote  \"     Hash (number) \#
%   Dollar        \$     Percent       \%     Ampersand     \&
%   Acute accent  \'     Left paren    \(     Right paren   \)
%   Asterisk      \*     Plus          \+     Comma         \,
%   Minus         \-     Point         \.     Solidus       \/
%   Colon         \:     Semicolon     \;     Less than     \<
%   Equals        \=     Greater than  \>     Question mark \?
%   Commercial at \@     Left bracket  \[     Backslash     \\
%   Right bracket \]     Circumflex    \^     Underscore    \_
%   Grave accent  \`     Left brace    \{     Vertical bar  \|
%   Right brace   \}     Tilde         \~}
%
% \GetFileInfo{embedfile.drv}
%
% \title{The \xpackage{embedfile} package}
% \date{2011/04/13 v2.6}
% \author{Heiko Oberdiek\\\xemail{heiko.oberdiek at googlemail.com}}
%
% \maketitle
%
% \begin{abstract}
% This package embeds files to a PDF document.
% Currently the only supported driver is \pdfTeX\ $>=$ 1.30 in PDF mode.
% \end{abstract}
%
% \tableofcontents
%
% \section{Documentation}
%
% \subsection{Introduction}
%
%    The PDF format (\cite{pdfspec}) allows the inclusion of files
%    inside the PDF document. The included files can be bound to an
%    annotation on a page. Or they can be recorded in a sorted
%    list of embedded files. The packages \xpackage{attachfile}
%    or \xpackage{attachfile2} follow the first approach, this
%    package uses the latter method.
%
% \subsubsection{Future development}
%
%    My dream is a large package that merges the features of
%    all these packages meantioned before:
%    \begin{itemize}
%    \item Files can be attached to a page.
%    \item Files can be attached to the document.
%    \item An easy user interface for simple, common tasks and
%          beginners.
%    \item An interface for the advanced users that want to setup
%          every detail.
%    \item Support of many drivers (pdftex, dvips, dvipdfm, \dots).
%    \item \dots
%    \end{itemize}
%    However, I have not managed to take the time for this project.
%    Instead:
%    \begin{itemize}
%    \item First I experimented with package \xpackage{attachfile},
%          adding driver support, fixing bugs, \dots. The result is
%          currently named as \xpackage{attachfile2}. It uses an external
%          script to get file properties (size, date, checksum, \dots).
%    \item In order to avoid an external program for getting basic
%          file properties I provided a patch ``EscapeAndOther'' for
%          pdfTeX that was accepted for version 1.30.
%    \item This package closes a gap left by the packages for attaching
%          files and allows the embedding of files to the document.
%          Also it makes use of the new primitives of \pdfTeX.
%    \end{itemize}
%
% \subsection{User interface}
%
%    This package \xpackage{embedfile} can be used with both
%    \LaTeX\ and \plainTeX. See \autoref{sec:plain} that
%    explains the use with \plainTeX\ by an example.
%    In \LaTeX\ the package is loaded as usually. There are no options.
%    \begin{quote}
%      \verb|\usepackage{embedfile}|
%    \end{quote}
%
%    \begin{declcs}{embedfile}
%      |[|\meta{options}|]| \M{file}
%    \end{declcs}
%    The macro \cs{embedfile} includes file \meta{file} and
%    attaches it to the PDF document. At the end of the document
%    the sorted list of embedded files are written. Thus you can
%    safely use \cs{embedfile} before \verb|\end{document}|.
%    Embedding files using \cs{AtEndDocument} will only work,
%    if \cs{AtEndDocument} is called before loading the package
%    \xpackage{embedfile}.
%
%    The \meta{options} are give as key value pairs. The following
%    keys are supported:
%    \begin{description}
%    \item[\xoption{filespec}]
%      This allows to override the file name
%      that appears in the PDF file. If you are using other than
%      simple file names (8-bit, path separators, \dots), look into
%      the PDF specification (\cite{pdfspec}).
%      There are rules how these file names must be written/encoded.
%      Avoid 8-bit characters
%      and other special characters, the behaviour is currently undefined.
%      Use option \xoption{ucfilespec} for more funny file names.
%      The string method, see below, is |escape| since version 2.4.
%
%      This name is also used as entry in a name tree
%      (see PDF specification: |/EmbeddedFiles|). Therefore the
%      value for \xoption{filespec} must be unique among all
%      embedded files. Also key \xoption{initialfiles} refers
%      to this name, if the file name and the value of \xoption{filespec}
%      are different.
%    \item[\xoption{ucfilespec}]
%      Since PDF 1.7 the file name may be provided in Unicode.
%      The conversion of the option value into a PDF string
%      is controlled by option \xoption{stringmethod}.
%    \item[\xoption{filesystem}]
%      This sets the entry \verb|/FS| in
%      the file specification dictionary, see PDF specification
%      (\cite{pdfspec}).
%      Example: \verb|filesystem=URL|.
%    \item[\xoption{mimetype}]
%      This sets the mime type (\cite{mime}) of the file,
%      see \autoref{sec:plain} for examples and \cite{types}
%      for a list of officially registered types.
%    \item[\xoption{desc}]
%      The description for the file.
%    \item[\xoption{stringmethod}]
%      The package must convert the values of the keys \xoption{ucfilespec}
%      and \xoption{desc} into a PDF string (before version 2.4: \xoption{filespec}
%      and \xoption{desc}). If \xpackage{hyperref}
%      is found, then its \cs{pdfstringdef} will be used, otherwise
%      \pdfTeX's \cs{pdfescapestring} is used. Value \verb|psd|
%      forces the use of \cs{pdfstringdef}, value \verb|escape|
%      the use of \cs{pdfescapestring}.
%    \item[\xoption{\meta{key}.value}] Sets the value of a collection
%      item property, see section \ref{sec:collection}.
%    \item[\xoption{\meta{key}.prefix}]
%      Sets the prefix of a collection item property,
%      see section \ref{sec:collection}.
%    \item[\xoption{id}]
%      The value must be an unique name. Macros
%      \cs{embedfileifobjectexists} and \cs{embedfilegetobject}
%      are using this name later.
%    \end{description}
%
%    \begin{declcs}{embedfilefinish}
%    \end{declcs}
%    The list of all embedded files must be added as data structure
%    in the PDF file. In case of \LaTeX\ this is automatically done.
%    The package uses \cs{AtEndDocument}. Then the list of all
%    files should be known. However, \plainTeX\ does not know
%    about \cs{AtEndDocument}. Thus the user must call
%    \cs{embedfilefinish} at the end of the document
%    after the last file is embedded.
%
%    \begin{declcs}{embedfilesetup} \M{options}
%    \end{declcs}
%    Options for \cs{embedfile} and collection support
%    can be set in \cs{embedfilesetup}.
%
% \subsection{Collection support (PDF 1.7)}
% \label{sec:collection}
%
%    Since PDF 1.7 the embedded files can form a \emph{collection}
%    (sometimes referred as \emph{package}), the main document
%    is called \emph{cover sheet}. See PDF specification
%    8.2.4 ``Collections'' and 3.10.5 ``Collection items'' \cite{pdfspec}.
%
%    Usually Acrobat Reader 7 or 8 shows the
%    embedded files in a table at the bottom with the following columns:
%    \begin{quote}
%    \begin{tabular}{@{}l|l|l|l@{}}
%       Name & Description & Modified & Size\\
%       \hline
%       \dots & \dots & \dots & \dots\\
%    \end{tabular}
%    \end{quote}
%    Acrobat Reader 10 shows the embeded files in the left panel
%    and adds a new column for the compressed size.
%
%    If the files form a collection, then they are displayed
%    in a table left or top (depending on option \xoption{view},
%    see \cs{embedfilesetup}).
%
%
%    Collection support is enabled automatically, if
%    it is used.
%
%    \begin{declcs}{embedfilesetup} \M{options}
%    \end{declcs}
%    The following options are supported in addition
%    to options for \cs{embedfile}:
%    \begin{description}
%    \item[\xoption{view}]
%      If the PDF file contains a collection, then Acrobat Reader 8
%      shows a line at the top below the menu bar and the toolbar.
%      It shows the current selected file, icons for changing the
%      view mode, an options menu.
%      The initial mode how the collection is presented is
%      set by this option \xoption{view}.
%      The following modes/values are supported, the default is |details|:
%        \begin{description}
%        \item[\texttt{details}] The full collection table is displayed
%          at the top below the collection bar.
%        \item[\texttt{tile}] The files of the collection are shown
%          in tile mode on the left.
%        \item[\texttt{hidden}] The collection table is not shown.
%        \end{description}
%    \item[\xoption{initialfile}]
%      Selects the file that is initially presented. Especially
%      useful for an embedded PDF file that is then shown instead
%      of the cover document. There must be an \cs{embedfile} command
%      somewhere whose value for key \xoption{filespec} is used here.
%      The \cs{embedfile} command can drop option \xoption{filespec}
%      if the file name is not different.
%    \end{description}
%
%    \begin{declcs}{embedfilefield} \M{key} \M{options}
%    \end{declcs}
%    Macro \cs{embedfilefield} defines a column/field in the collection table.
%    The name of the field is \meta{key}.
%    \begin{description}
%    \item[\xoption{type}]
%      sets the type of the field. The supported values are:
%      \begin{description}
%      \item[\texttt{text}] A text field. Its value is set in
%        \cs{embedfile} by option \xoption{\meta{key}.value}.
%      \item[\texttt{date}] A date field. Its value is set in
%        \cs{embedfile} by option \xoption{\meta{key}.value}.
%        A special format is required, see ``3.8.3 Dates'' \cite{pdfspec}.
%      \item[\texttt{number}] A field with an integer or float number.
%        Its value is set in \cs{embedfile} by option
%        \xoption{\meta{key}.value}.
%      \item[\texttt{file}] The file name of the embedded file.
%      \item[\texttt{desc}] The description text of the embedded file. It is
%       set in \cs{embedfile} by option \xoption{desc}.
%      \item[\texttt{moddate}] The modification date of the embedded file.
%      \item[\texttt{size}] The size of the embedded file.
%      \end{description}
%      All types allow the use of a prefix that is disregarded by
%      sorting. The prefix for this field is set in \cs{embedfile} by
%      option \xoption{\meta{key}.prefix}.
%    \item[\xoption{title}] sets the column title.
%    \item[\xoption{visible}] controls whether the column is presented:
%      \begin{description}
%      \item[\texttt{true}] shows the column.
%      \item[\texttt{false}] hides the column.
%      \end{description}
%      Default: \texttt{true}
%    \item[\xoption{edit}]
%      Allows the editing of field values. Does not seem to have an
%      effect for Acrobat Reader.
%      \begin{description}
%      \item[\texttt{true}] enables the feature, if available (depends
%        on the PDF viewer).
%      \item[\texttt{false}] disables the feature.
%      \end{description}
%      Default: \texttt{false}
%    \end{description}
%    The order of \cs{embedfilefield} statements defines the order
%    of the columns.
%
%    \begin{declcs}{embedfilesort} \M{key-sort-list}
%    \end{declcs}
%    The sort order of the embedded files are controlled by
%    macro \cs{embedfilesort}. \meta{key-sort-list} defines
%    the sort order. The key is a field name defined by \cs{embedfilefield}.
%    Its value is either \texttt{ascending} or \texttt{descending}.
%    The default is \texttt{ascending}.
%
% \subsection{Export of object references}
%
%    Caution: This feature is still experimental. It may be even removed
%    in future versions. Therefore feedback would be nice, if someone
%    has a useful application for this feature.
%
%    Object numbers are saved, if \xoption{id} is given in \cs{embedfile}.
%    The following objects are supported:
%    \begin{itemize}
%    \item \texttt{EmbeddedFile}
%    \item \texttt{Filespec}
%    \end{itemize}
%
%    \begin{declcs}{embedfileifobjectexists} \M{id} \M{type}
%      \M{then} \M{else}
%    \end{declcs}
%    Macro \cs{embedfileifobjectexists} tests whether object of \meta{type}
%    is available for the embedded file identified by \meta{id}.
%
%    \begin{declcs}{embedfilegetobject} \M{id} \M{type}
%    \end{declcs}
%    Macro \cs{embedfilegetobject} expands to the full object reference
%    object of \meta{type} for the embedded file identified by \meta{id}.
%
% \subsubsection{Example}
%
%\begin{quote}
%\begin{verbatim}
%\embedfile[id={foo}]{foo.pdf}
%\embedfileifobjectexists{foo}{Filespec}{%
%  \typeout{%
%    FileSpec object for `foo': %
%    \embedfilegetobject{foo}{Filespec}%
%  }%
%}{%
%  \typeout{No Filespec object for `foo'}%
%}
%\end{verbatim}
%\end{quote}
%
% \subsection{Examples}
% \label{sec:examples}
%
% \subsubsection{\plainTeX}
% \label{sec:plain}
%
%    The package can be used with \plainTeX. It can be used
%    with or without help from \xfile{miniltx.tex}.
%
%    If additionally package \xpackage{keyval} (\xpackage{graphicx})
%    is needed, load it first. Then package \xpackage{embedfile} avoids
%    a duplicate loading of package \xpackage{keyval}.
%
%    Because \plainTeX\ does not provide a hook at end of the document,
%    you have to call \cs{embedfilefinish} manually at the end
%    after the last embedded file.
%    \begin{macrocode}
%<*exampleplain>
%<<END
% Load packages
\input miniltx
% \def\Gin@driver{pdftex.def}
% \input graphicx.sty
\input embedfile.sty
\resetatcatcode

% default setting
\embedfilesetup{
  mimetype=text/plain
}

% Embed files
\embedfile[
  filespec=example.tex,
  desc={Source code (plain-TeX) of this example}
]{embedfile-example-plain.tex}

\embedfile[
  desc={Source of package `embedfile'}
]{embedfile.dtx}

\embedfile[
  mimetype=application/pdf,
  desc={Documentation of package `embedfile'}
]{embedfile.pdf}

% Some text
This example document contains three embedded files.

% End of document
\embedfilefinish % don't forget
\bye
%END
%</exampleplain>
%    \end{macrocode}
%
% \subsubsection{Collection example}
%
%    \begin{macrocode}
%<*examplecollection>
%<<END
\NeedsTeXFormat{LaTeX2e}
\documentclass{article}
\usepackage[bookmarks=false]{hyperref}
% provides \pdfstringdef that is then used by `title' and
% other keys.
\usepackage{embedfile}[2011/04/13]
\embedfilesetup{
  view=details,
  initialfile=embedfile.pdf
}
\embedfilefield{file}{
  type=file,
  title={File name}
}
\embedfilefield{description}{
  type=desc,
  title={Description}
}
\embedfilefield{date}{
  type=moddate,
  title={Date}
}
\embedfilefield{size}{
  type=size,
  title={Size}
}
\embedfilefield{type}{
  type=text,
  title={Type},
  visible=false
}
\embedfilesort{
  type,
  date=descending
}
\begin{document}
An example for embedded files as collection.
You need Acrobat Reader 8 or higher.

\embedfile[
  desc={Source file of package `embedfile'},
  description.prefix={Package: },
  type.value={DTX}
]{embedfile.dtx}

\embedfile[
  desc={Documentation of package `embedfile'},
  description.prefix={Package: },
  type.value={PDF}
]{embedfile.pdf}

\embedfile[
  desc={The source for this example},
  description.prefix={Example: },
  type.value={TEX}
]{\jobname.tex}

\end{document}
%END
%</examplecollection>
%    \end{macrocode}
%
% \subsection{Package \xpackage{dtx-attach}}
%
% Package \xpackage{dtx-attach} is just a small application of
% package \xpackage{embedfile}. I am using it for the CTAN
% documentation of my packages in \CTAN{macros/latex/contrib/oberdiek/}.
% It also serves as small example for the use of the package with \LaTeX.
%
%    \begin{macrocode}
%<*dtxattach>
\NeedsTeXFormat{LaTeX2e}
\ProvidesPackage{dtx-attach}
  [2011/04/13 v2.6 Embed \string\jobname.dtx (HO)]%
\RequirePackage{embedfile}[2011/04/13]
\embedfile[%
  stringmethod=escape,%
  mimetype=plain/text,%
  desc={LaTeX docstrip source archive for package `\jobname'}%
]{\jobname.dtx}
%</dtxattach>
%    \end{macrocode}
%
% \StopEventually{
% }
%
% \section{Implementation}
%
%    \begin{macrocode}
%<*package>
%    \end{macrocode}
%
% \subsection{Reload check and package identification}
%    Reload check, especially if the package is not used with \LaTeX.
%    \begin{macrocode}
\begingroup\catcode61\catcode48\catcode32=10\relax%
  \catcode13=5 % ^^M
  \endlinechar=13 %
  \catcode35=6 % #
  \catcode39=12 % '
  \catcode44=12 % ,
  \catcode45=12 % -
  \catcode46=12 % .
  \catcode58=12 % :
  \catcode64=11 % @
  \catcode123=1 % {
  \catcode125=2 % }
  \expandafter\let\expandafter\x\csname ver@embedfile.sty\endcsname
  \ifx\x\relax % plain-TeX, first loading
  \else
    \def\empty{}%
    \ifx\x\empty % LaTeX, first loading,
      % variable is initialized, but \ProvidesPackage not yet seen
    \else
      \expandafter\ifx\csname PackageInfo\endcsname\relax
        \def\x#1#2{%
          \immediate\write-1{Package #1 Info: #2.}%
        }%
      \else
        \def\x#1#2{\PackageInfo{#1}{#2, stopped}}%
      \fi
      \x{embedfile}{The package is already loaded}%
      \aftergroup\endinput
    \fi
  \fi
\endgroup%
%    \end{macrocode}
%    Package identification:
%    \begin{macrocode}
\begingroup\catcode61\catcode48\catcode32=10\relax%
  \catcode13=5 % ^^M
  \endlinechar=13 %
  \catcode35=6 % #
  \catcode39=12 % '
  \catcode40=12 % (
  \catcode41=12 % )
  \catcode44=12 % ,
  \catcode45=12 % -
  \catcode46=12 % .
  \catcode47=12 % /
  \catcode58=12 % :
  \catcode64=11 % @
  \catcode91=12 % [
  \catcode93=12 % ]
  \catcode123=1 % {
  \catcode125=2 % }
  \expandafter\ifx\csname ProvidesPackage\endcsname\relax
    \def\x#1#2#3[#4]{\endgroup
      \immediate\write-1{Package: #3 #4}%
      \xdef#1{#4}%
    }%
  \else
    \def\x#1#2[#3]{\endgroup
      #2[{#3}]%
      \ifx#1\@undefined
        \xdef#1{#3}%
      \fi
      \ifx#1\relax
        \xdef#1{#3}%
      \fi
    }%
  \fi
\expandafter\x\csname ver@embedfile.sty\endcsname
\ProvidesPackage{embedfile}%
  [2011/04/13 v2.6 Embed files into PDF (HO)]%
%    \end{macrocode}
%
% \subsection{Catcodes}
%
%    \begin{macrocode}
\begingroup\catcode61\catcode48\catcode32=10\relax%
  \catcode13=5 % ^^M
  \endlinechar=13 %
  \catcode123=1 % {
  \catcode125=2 % }
  \catcode64=11 % @
  \def\x{\endgroup
    \expandafter\edef\csname EmFi@AtEnd\endcsname{%
      \endlinechar=\the\endlinechar\relax
      \catcode13=\the\catcode13\relax
      \catcode32=\the\catcode32\relax
      \catcode35=\the\catcode35\relax
      \catcode61=\the\catcode61\relax
      \catcode64=\the\catcode64\relax
      \catcode123=\the\catcode123\relax
      \catcode125=\the\catcode125\relax
    }%
  }%
\x\catcode61\catcode48\catcode32=10\relax%
\catcode13=5 % ^^M
\endlinechar=13 %
\catcode35=6 % #
\catcode64=11 % @
\catcode123=1 % {
\catcode125=2 % }
\def\TMP@EnsureCode#1#2{%
  \edef\EmFi@AtEnd{%
    \EmFi@AtEnd
    \catcode#1=\the\catcode#1\relax
  }%
  \catcode#1=#2\relax
}
\TMP@EnsureCode{39}{12}% '
\TMP@EnsureCode{40}{12}% (
\TMP@EnsureCode{41}{12}% )
\TMP@EnsureCode{44}{12}% ,
\TMP@EnsureCode{46}{12}% .
\TMP@EnsureCode{47}{12}% /
\TMP@EnsureCode{58}{12}% :
\TMP@EnsureCode{60}{12}% <
\TMP@EnsureCode{62}{12}% >
\TMP@EnsureCode{91}{12}% [
\TMP@EnsureCode{93}{12}% ]
\TMP@EnsureCode{96}{12}% `
\edef\EmFi@AtEnd{\EmFi@AtEnd\noexpand\endinput}
%    \end{macrocode}
%
% \subsection{Tools}
%
%    \begin{macro}{\EmFi@RequirePackage}
%    \begin{macrocode}
\begingroup\expandafter\expandafter\expandafter\endgroup
\expandafter\ifx\csname RequirePackage\endcsname\relax
  \def\EmFi@RequirePackage#1[#2]{%
    \input #1.sty\relax
  }%
\else
  \let\EmFi@RequirePackage\RequirePackage
\fi
%    \end{macrocode}
%    \end{macro}
%
%    \begin{macro}{\EmFi@Error}
%    \begin{macrocode}
\EmFi@RequirePackage{infwarerr}[2007/09/09]%
\def\EmFi@Error{%
  \@PackageError{embedfile}%
}
%    \end{macrocode}
%    \end{macro}
%
% \subsection{Check for recent \pdfTeX\ in PDF mode}
%
%    Load package \xpackage{ifpdf} and check mode.
%    \begin{macrocode}
\EmFi@RequirePackage{ifpdf}[2007/09/09]
\ifpdf
\else
  \EmFi@Error{%
    Missing pdfTeX in PDF mode%
  }{%
    Currently other drivers are not supported. %
    Package loading is aborted.%
  }%
  \expandafter\EmFi@AtEnd
\fi%
%    \end{macrocode}
%    \begin{macrocode}
\EmFi@RequirePackage{pdftexcmds}[2007/11/11]
\EmFi@RequirePackage{ltxcmds}[2010/03/01]
\EmFi@RequirePackage{kvsetkeys}[2010/03/01]
\EmFi@RequirePackage{kvdefinekeys}[2010/03/01]
%    \end{macrocode}
%    Check version.
%    \begin{macrocode}
\begingroup\expandafter\expandafter\expandafter\endgroup
\expandafter\ifx\csname pdf@filesize\endcsname\relax
  \EmFi@Error{%
    Unsupported pdfTeX version%
  }{%
    At least version 1.30 is necessary. Package loading is aborted.%
  }%
  \expandafter\EmFi@AtEnd
\fi%
%    \end{macrocode}
%
% \subsection{Strings}
%
%    Minimal version of package \xpackage{pdfescape} is 2007/08/27 v1.5
%    because of \cs{EdefSanitize}.
%    \begin{macrocode}
\EmFi@RequirePackage{pdfescape}[2007/11/11]
%    \end{macrocode}
%
%    \begin{macrocode}
\def\EmFi@temp#1{%
  \expandafter\EdefSanitize\csname EmFi@S@#1\endcsname{#1}%
}
%    \end{macrocode}
%    \begin{macro}{\EmFi@details}
%    \begin{macrocode}
\EmFi@temp{details}%
%    \end{macrocode}
%    \end{macro}
%    \begin{macro}{\EmFi@tile}
%    \begin{macrocode}
\EmFi@temp{tile}%
%    \end{macrocode}
%    \end{macro}
%    \begin{macro}{\EmFi@hidden}
%    \begin{macrocode}
\EmFi@temp{hidden}%
%    \end{macrocode}
%    \end{macro}
%    \begin{macro}{\EmFi@S@text}
%    \begin{macrocode}
\EmFi@temp{text}
%    \end{macrocode}
%    \end{macro}
%    \begin{macro}{\EmFi@S@date}
%    \begin{macrocode}
\EmFi@temp{date}
%    \end{macrocode}
%    \end{macro}
%    \begin{macro}{\EmFi@S@number}
%    \begin{macrocode}
\EmFi@temp{number}
%    \end{macrocode}
%    \end{macro}
%    \begin{macro}{\EmFi@S@file}
%    \begin{macrocode}
\EmFi@temp{file}
%    \end{macrocode}
%    \end{macro}
%    \begin{macro}{\EmFi@S@desc}
%    \begin{macrocode}
\EmFi@temp{desc}
%    \end{macrocode}
%    \end{macro}
%    \begin{macro}{\EmFi@S@moddate}
%    \begin{macrocode}
\EmFi@temp{moddate}
%    \end{macrocode}
%    \end{macro}
%    \begin{macro}{\EmFi@S@creationdate}
%    \begin{macrocode}
\EmFi@temp{creationdate}
%    \end{macrocode}
%    \end{macro}
%    \begin{macro}{\EmFi@S@size}
%    \begin{macrocode}
\EmFi@temp{size}
%    \end{macrocode}
%    \end{macro}
%    \begin{macro}{\EmFi@S@ascending}
%    \begin{macrocode}
\EmFi@temp{ascending}
%    \end{macrocode}
%    \end{macro}
%    \begin{macro}{\EmFi@S@descending}
%    \begin{macrocode}
\EmFi@temp{descending}
%    \end{macrocode}
%    \end{macro}
%    \begin{macro}{\EmFi@S@true}
%    \begin{macrocode}
\EmFi@temp{true}
%    \end{macrocode}
%    \end{macro}
%    \begin{macro}{\EmFi@S@false}
%    \begin{macrocode}
\EmFi@temp{false}
%    \end{macrocode}
%    \end{macro}
%
% \subsection{Switches}
%
%    \begin{macro}{\ifEmFi@collection}
%    \begin{macrocode}
\ltx@newif\ifEmFi@collection
%    \end{macrocode}
%    \end{macro}
%    \begin{macro}{\ifEmFi@sort}
%    \begin{macrocode}
\ltx@newif\ifEmFi@sort
%    \end{macrocode}
%    \end{macro}
%    \begin{macro}{\ifEmFi@visible}
%    \begin{macrocode}
\ltx@newif\ifEmFi@visible
%    \end{macrocode}
%    \end{macro}
%    \begin{macro}{\ifEmFi@edit}
%    \begin{macrocode}
\ltx@newif\ifEmFi@edit
%    \end{macrocode}
%    \end{macro}
%    \begin{macro}{\ifEmFi@item}
%    \begin{macrocode}
\ltx@newif\ifEmFi@item
%    \end{macrocode}
%    \end{macro}
%    \begin{macro}{\ifEmFi@finished}
%    \begin{macrocode}
\ltx@newif\ifEmFi@finished
%    \end{macrocode}
%    \end{macro}
%    \begin{macro}{\ifEmFi@id}
%    \begin{macrocode}
\ltx@newif\ifEmFi@id
%    \end{macrocode}
%    \end{macro}
%
% \subsection{Key value definitions}
%
%    \begin{macro}{\EmFi@GlobalKey}
%    \begin{macrocode}
\def\EmFi@GlobalKey#1#2{%
  \global\expandafter\let\csname KV@#1@#2\expandafter\endcsname
                         \csname KV@#1@#2\endcsname
}
%    \end{macrocode}
%    \end{macro}
%    \begin{macro}{\EmFi@GlobalDefaultKey}
%    \begin{macrocode}
\def\EmFi@GlobalDefaultKey#1#2{%
  \EmFi@GlobalKey{#1}{#2}%
  \global\expandafter\let
      \csname KV@#1@#2@default\expandafter\endcsname
      \csname KV@#1@#2@default\endcsname
}
%    \end{macrocode}
%    \end{macro}
%
%    \begin{macro}{\EmFi@DefineKey}
%    \begin{macrocode}
\def\EmFi@DefineKey#1#2{%
  \kv@define@key{EmFi}{#1}{%
    \expandafter\def\csname EmFi@#1\endcsname{##1}%
  }%
  \expandafter\def\csname EmFi@#1\endcsname{#2}%
}
%    \end{macrocode}
%    \end{macro}
%    Subtype of the embedded file (optional).
%    \begin{macrocode}
\EmFi@DefineKey{mimetype}{}
%    \end{macrocode}
%    File specification string.
%    \begin{macrocode}
\EmFi@DefineKey{filespec}{\EmFi@file}
%    \end{macrocode}
%    File specification string in Unicode.
%    \begin{macrocode}
\EmFi@DefineKey{ucfilespec}{}
%    \end{macrocode}
%    File system (optional).
%    \begin{macrocode}
\EmFi@DefineKey{filesystem}{}
%    \end{macrocode}
%    Description (optional).
%    \begin{macrocode}
\EmFi@DefineKey{desc}{}
%    \end{macrocode}
%    Method for converting text to PDF strings.
%    \begin{macrocode}
\EmFi@DefineKey{stringmethod}{%
  \ifx\pdfstringdef\@undefined
    escape%
  \else
    \ifx\pdfstringdef\relax
      escape%
    \else
      psd%
    \fi
  \fi
}
%    \end{macrocode}
%
%    Option \xoption{id} as key for object numbers.
%    \begin{macrocode}
\kv@define@key{EmFi}{id}{%
  \def\EmFi@id{#1}%
  \EmFi@idtrue
}
%    \end{macrocode}
%    \begin{macro}{\EmFi@defobj}
%    \begin{macrocode}
\def\EmFi@defobj#1{%
  \ifEmFi@id
    \expandafter\xdef\csname EmFi@#1@\EmFi@id\endcsname{%
      \the\pdflastobj\ltx@space 0 R%
    }%
  \fi
}
%    \end{macrocode}
%    \end{macro}
%    \begin{macro}{\embedfileifobjectexists}
%    \begin{macrocode}
\def\embedfileifobjectexists#1#2{%
  \expandafter\ifx\csname EmFi@#2@#1\endcsname\relax
    \expandafter\ltx@secondoftwo
  \else
    \expandafter\ltx@firstoftwo
  \fi
}
%    \end{macrocode}
%    \end{macro}
%    \begin{macro}{\embedfilegetobject}
%    \begin{macrocode}
\def\embedfilegetobject#1#2{%
  \embedfileifobjectexists{#1}{#2}{%
    \csname EmFi@#2@#1\endcsname
  }{%
    0 0 R%
  }%
}
%    \end{macrocode}
%    \end{macro}
%
%    Initial view of the collection.
%    \begin{macrocode}
\kv@define@key{EmFi}{view}[]{%
  \EdefSanitize\EmFi@temp{#1}%
  \def\EmFi@next{%
    \global\EmFi@collectiontrue
  }%
  \ifx\EmFi@temp\ltx@empty
    \let\EmFi@view\EmFi@S@details
  \else\ifx\EmFi@temp\EmFi@S@details
    \let\EmFi@view\EmFi@S@details
  \else\ifx\EmFi@temp\EmFi@S@tile
    \let\EmFi@view\EmFi@S@tile
  \else\ifx\EmFi@temp\EmFi@S@hidden
    \let\EmFi@view\EmFi@S@hidden
  \else
    \let\EmFi@next\relax
    \EmFi@Error{%
      Unknown value `\EmFi@temp' for key `view'.\MessageBreak
      Supported values: `details', `tile', `hidden'.%
    }\@ehc
  \fi\fi\fi\fi
  \EmFi@next
}
%    \end{macrocode}
%    \begin{macrocode}
\EmFi@DefineKey{initialfile}{}
%    \end{macrocode}
%
%    \begin{macro}{\embedfilesetup}
%    \begin{macrocode}
\def\embedfilesetup{%
  \ifEmFi@finished
    \def\EmFi@next##1{}%
    \EmFi@Error{%
      \string\embedfilefield\ltx@space after \string\embedfilefinish
    }{%
      The list of embedded files is already written.%
    }%
  \else
    \def\EmFi@next{%
      \kvsetkeys{EmFi}%
    }%
  \fi
  \EmFi@next
}
%    \end{macrocode}
%    \end{macro}
%
%    \begin{macro}{\EmFi@schema}
%    \begin{macrocode}
\def\EmFi@schema{}
%    \end{macrocode}
%    \end{macro}
%    \begin{macro}{\EmFi@order}
%    \begin{macrocode}
\gdef\EmFi@order{0}
%    \end{macrocode}
%    \end{macro}
%    \begin{macro}{\EmFi@@order}
%    \begin{macrocode}
\let\EmFi@@order\relax
%    \end{macrocode}
%    \end{macro}
%    \begin{macro}{\EmFi@fieldlist}
%    \begin{macrocode}
\def\EmFi@fieldlist{}
%    \end{macrocode}
%    \end{macro}
%    \begin{macro}{\EmFi@sortcase}
%    \begin{macrocode}
\def\EmFi@sortcase{0}%
%    \end{macrocode}
%    \end{macro}
%    \begin{macro}{\embedfilefield}
%    \begin{macrocode}
\def\embedfilefield#1#2{%
  \ifEmFi@finished
    \EmFi@Error{%
      \string\embedfilefield\ltx@space after \string\embedfilefinish
    }{%
      The list of embedded files is already written.%
    }%
  \else
    \global\EmFi@collectiontrue
    \EdefSanitize\EmFi@key{#1}%
    \expandafter\ifx\csname KV@EmFi@\EmFi@key.prefix\endcsname\relax
      \begingroup
        \count@=\EmFi@order
        \advance\count@ 1 %
        \xdef\EmFi@order{\the\count@}%
        \let\EmFi@title\EmFi@key
        \let\EmFi@type\EmFi@S@text
        \EmFi@visibletrue
        \EmFi@editfalse
        \kvsetkeys{EmFiFi}{#2}%
        \EmFi@convert\EmFi@title\EmFi@title
        \xdef\EmFi@schema{%
          \EmFi@schema
          /\pdf@escapename{\EmFi@key}<<%
            /Subtype/%
            \ifx\EmFi@type\EmFi@S@date D%
            \else\ifx\EmFi@type\EmFi@S@number N%
            \else\ifx\EmFi@type\EmFi@S@file F%
            \else\ifx\EmFi@type\EmFi@S@desc Desc%
            \else\ifx\EmFi@type\EmFi@S@moddate ModDate%
            \else\ifx\EmFi@type\EmFi@S@creationdate CreationDate%
            \else\ifx\EmFi@type\EmFi@S@size Size%
            \else S%
            \fi\fi\fi\fi\fi\fi\fi
            /N(\EmFi@title)%
            \EmFi@@order{\EmFi@order}%
            \ifEmFi@visible
            \else
              /V false%
            \fi
            \ifEmFi@edit
              /E true%
            \fi
          >>%
        }%
        \let\do\relax
        \xdef\EmFi@fieldlist{%
          \EmFi@fieldlist
          \do{\EmFi@key}%
        }%
        \ifx\EmFi@type\EmFi@S@text
          \kv@define@key{EmFi}{\EmFi@key.value}{%
            \EmFi@itemtrue
            \def\EmFi@temp{##1}%
            \EmFi@convert\EmFi@temp\EmFi@temp
            \expandafter\def\csname EmFi@V@#1%
            \expandafter\endcsname\expandafter{%
              \expandafter(\EmFi@temp)%
            }%
          }%
          \EmFi@GlobalKey{EmFi}{\EmFi@key.value}%
        \else\ifx\EmFi@type\EmFi@S@date
          \kv@define@key{EmFi}{\EmFi@key.value}{%
            \EmFi@itemtrue
            \def\EmFi@temp{##1}%
            \EmFi@convert\EmFi@temp\EmFi@temp
            \expandafter\def\csname EmFi@V@#1%
            \expandafter\endcsname\expandafter{%
              \expandafter(\EmFi@temp)%
            }%
          }%
          \EmFi@GlobalKey{EmFi}{\EmFi@key.value}%
        \else\ifx\EmFi@type\EmFi@S@number
          \kv@define@key{EmFi}{\EmFi@key.value}{%
            \EmFi@itemtrue
            \expandafter\EdefSanitize\csname EmFi@V@#1\endcsname{ ##1}%
          }%
          \EmFi@GlobalKey{EmFi}{\EmFi@key.value}%
        \fi\fi\fi
        \kv@define@key{EmFi}{\EmFi@key.prefix}{%
          \EmFi@itemtrue
          \expandafter\def\csname EmFi@P@#1\endcsname{##1}%
        }%
        \EmFi@GlobalKey{EmFi}{\EmFi@key.prefix}%
        \kv@define@key{EmFiSo}{\EmFi@key}[ascending]{%
          \EdefSanitize\EmFi@temp{##1}%
          \ifx\EmFi@temp\EmFi@S@ascending
            \def\EmFi@temp{true}%
          \else\ifx\EmFi@temp\EmFi@S@descending
            \def\EmFi@temp{false}%
          \else
            \def\EmFi@temp{}%
            \EmFi@Error{%
              Unknown sort order `\EmFi@temp'.\MessageBreak
              Supported values: `\EmFi@S@ascending', %
              `\EmFi@S@descending
            }\@ehc
          \fi\fi
          \ifx\EmFi@temp\ltx@empty
          \else
            \xdef\EmFi@sortkeys{%
              \EmFi@sortkeys
              /\pdf@escapename{#1}%
            }%
            \ifx\EmFi@sortorders\ltx@empty
              \global\let\EmFi@sortorders\EmFi@temp
              \gdef\EmFi@sortcase{1}%
            \else
              \xdef\EmFi@sortorders{%
                \EmFi@sortorders
                \ltx@space
                \EmFi@temp
              }%
              \xdef\EmFi@sortcase{2}%
            \fi
          \fi
        }%
        \EmFi@GlobalDefaultKey{EmFiSo}\EmFi@key
      \endgroup
    \else
      \EmFi@Error{%
        Field `\EmFi@key' is already defined%
      }\@ehc
    \fi
  \fi
}
%    \end{macrocode}
%    \end{macro}
%    \begin{macrocode}
\kv@define@key{EmFiFi}{type}{%
  \EdefSanitize\EmFi@temp{#1}%
  \ifx\EmFi@temp\EmFi@S@text
    \let\EmFi@type\EmFi@temp
  \else\ifx\EmFi@temp\EmFi@S@date
    \let\EmFi@type\EmFi@temp
  \else\ifx\EmFi@temp\EmFi@S@number
    \let\EmFi@type\EmFi@temp
  \else\ifx\EmFi@temp\EmFi@S@file
    \let\EmFi@type\EmFi@temp
  \else\ifx\EmFi@temp\EmFi@S@desc
    \let\EmFi@type\EmFi@temp
  \else\ifx\EmFi@temp\EmFi@S@moddate
    \let\EmFi@type\EmFi@temp
  \else\ifx\EmFi@temp\EmFi@S@creationdate
    \let\EmFi@type\EmFi@temp
  \else\ifx\EmFi@temp\EmFi@S@size
    \let\EmFi@type\EmFi@temp
  \else
    \EmFi@Error{%
      Unknown type `\EmFi@temp'.\MessageBreak
      Supported types: `text', `date', `number', `file',\MessageBreak
      `desc', `moddate', `creationdate', `size'%
    }%
  \fi\fi\fi\fi\fi\fi\fi\fi
}
%    \end{macrocode}
%    \begin{macrocode}
\kv@define@key{EmFiFi}{title}{%
  \def\EmFi@title{#1}%
}
%    \end{macrocode}
%    \begin{macro}{\EmFi@setboolean}
%    \begin{macrocode}
\def\EmFi@setboolean#1#2{%
  \EdefSanitize\EmFi@temp{#2}%
  \ifx\EmFi@temp\EmFi@S@true
    \csname EmFi@#1true\endcsname
  \else
    \ifx\EmFi@temp\EmFi@S@false
      \csname EmFi@#1false\endcsname
    \else
      \EmFi@Error{%
        Unknown value `\EmFi@temp' for key `#1'.\MessageBreak
        Supported values: `true', `false'%
      }\@ehc
    \fi
  \fi
}
%    \end{macrocode}
%    \end{macro}
%    \begin{macrocode}
\kv@define@key{EmFiFi}{visible}[true]{%
  \EmFi@setboolean{visible}{#1}%
}
%    \end{macrocode}
%    \begin{macrocode}
\kv@define@key{EmFiFi}{edit}[true]{%
  \EmFi@setboolean{edit}{#1}%
}
%    \end{macrocode}
%
%    \begin{macro}{\EmFi@sortkeys}
%    \begin{macrocode}
\def\EmFi@sortkeys{}
%    \end{macrocode}
%    \end{macro}
%    \begin{macro}{\EmFi@sortorders}
%    \begin{macrocode}
\def\EmFi@sortorders{}
%    \end{macrocode}
%    \end{macro}
%    \begin{macro}{\embedfilesort}
%    \begin{macrocode}
\def\embedfilesort{%
  \kvsetkeys{EmFiSo}%
}
%    \end{macrocode}
%    \end{macro}
%
% \subsection{Embed the file}
%
%    \begin{macro}{\embedfile}
%    \begin{macrocode}
\def\embedfile{%
  \ltx@ifnextchar[\EmFi@embedfile{\EmFi@embedfile[]}%
}
%    \end{macrocode}
%    \end{macro}
%    \begin{macro}{\EmFi@embedfile}
%    \begin{macrocode}
\def\EmFi@embedfile[#1]#2{%
  \ifEmFi@finished
    \EmFi@Error{%
      \string\embedfile\ltx@space after \string\embedfilefinish
    }{%
      The list of embedded files is already written.%
    }%
  \else
    \begingroup
      \def\EmFi@file{#2}%
      \kvsetkeys{EmFi}{#1}%
      \expandafter\expandafter\expandafter
      \ifx\expandafter\expandafter\expandafter
          \\\pdf@filesize{\EmFi@file}\\%
        \EmFi@Error{%
          File `\EmFi@file' not found%
        }{%
          The unknown file is not embedded.%
        }%
      \else
        \edef\EmFi@@filespec{%
          \pdf@escapestring{\EmFi@filespec}%
        }%
        \ifx\EmFi@ucfilespec\ltx@empty
          \let\EmFi@@ucfilespec\ltx@empty
        \else
          \EmFi@convert\EmFi@ucfilespec\EmFi@@ucfilespec
        \fi
        \ifx\EmFi@desc\ltx@empty
          \let\EmFi@@desc\ltx@empty
        \else
          \EmFi@convert\EmFi@desc\EmFi@@desc
        \fi
        \ifEmFi@item
          \let\do\EmFi@do
          \immediate\pdfobj{%
            <<%
              \EmFi@fieldlist
            >>%
          }%
          \edef\EmFi@ci{\the\pdflastobj}%
        \fi
        \immediate\pdfobj stream attr{%
          /Type/EmbeddedFile%
          \ifx\EmFi@mimetype\ltx@empty
          \else
            /Subtype/\pdf@escapename{\EmFi@mimetype}%
          \fi
          /Params<<%
            /ModDate(\pdf@filemoddate{\EmFi@file})%
            /Size \pdf@filesize{\EmFi@file}%
            /CheckSum<\pdf@filemdfivesum{\EmFi@file}>%
          >>%
        }file{\EmFi@file}\relax
        \EmFi@defobj{EmbeddedFile}%
        \immediate\pdfobj{%
          <<%
            /Type/Filespec%
            \ifx\EmFi@filesystem\ltx@empty
            \else
            /FS/\pdf@escapename{\EmFi@filesystem}%
            \fi
            /F(\EmFi@@filespec)%
            \ifx\EmFi@@ucfilespec\ltx@empty
            \else
              /UF(\EmFi@@ucfilespec)%
            \fi
            \ifx\EmFi@@desc\ltx@empty
            \else
              /Desc(\EmFi@@desc)%
            \fi
            /EF<<%
              /F \the\pdflastobj\ltx@space 0 R%
            >>%
            \ifEmFi@item
              /CI \EmFi@ci\ltx@space 0 R%
            \fi
          >>%
        }%
        \EmFi@defobj{Filespec}%
        \EmFi@add{%
          \EmFi@@filespec
        }{\the\pdflastobj\ltx@space 0 R}%
      \fi
    \endgroup
  \fi
}
%    \end{macrocode}
%    \end{macro}
%    \begin{macro}{\EmFi@do}
%    \begin{macrocode}
\def\EmFi@do#1{%
  \expandafter\ifx\csname EmFi@P@#1\endcsname\relax
    \expandafter\ifx\csname EmFi@V@#1\endcsname\relax
    \else
      /\pdf@escapename{#1}\csname EmFi@V@#1\endcsname
    \fi
  \else
    /\pdf@escapename{#1}<<%
      \expandafter\ifx\csname EmFi@V@#1\endcsname\relax
      \else
        /D\csname EmFi@V@#1\endcsname
      \fi
      /P(\csname EmFi@P@#1\endcsname)%
    >>%
  \fi
}
%    \end{macrocode}
%    \end{macro}
%    \begin{macro}{\EmFi@convert}
%    \begin{macrocode}
\def\EmFi@convert#1#2{%
  \ifnum\pdf@strcmp{\EmFi@stringmethod}{psd}=0 %
    \pdfstringdef\EmFi@temp{#1}%
    \let#2\EmFi@temp
  \else
    \edef#2{\pdf@escapestring{#1}}%
  \fi
}
%    \end{macrocode}
%    \end{macro}
%    \begin{macrocode}
\global\let\EmFi@list\ltx@empty
%    \end{macrocode}
%    \begin{macro}{\EmFi@add}
%    Sorting is done by the insertion sort algorithm.
%    Probably the sorting could be done more reliable.
%    However, the PDF specification is not too clear to me
%    regarding precise sorting rules (how to deal with
%    different encodings, escaped characters, \dots).
%    \begin{macrocode}
\def\EmFi@add#1#2{%
  \begingroup
    \ifx\EmFi@list\ltx@empty
      \xdef\EmFi@list{\noexpand\do{#1}{#2}}%
    \else
      \def\do##1##2{%
        \ifnum\pdf@strcmp{##1}{#1}>0 %
          \edef\x{%
            \toks@{%
              \the\toks@%
              \noexpand\do{#1}{#2}%
              \noexpand\do{##1}{##2}%
            }%
          }%
          \x
          \def\do####1####2{%
            \toks@\expandafter{\the\toks@\do{####1}{####2}}%
          }%
          \def\stop{%
            \xdef\EmFi@list{\the\toks@}%
          }%
        \else
          \toks@\expandafter{\the\toks@\do{##1}{##2}}%
        \fi
      }%
      \def\stop{%
        \xdef\EmFi@list{\the\toks@\noexpand\do{#1}{#2}}%
      }%
      \toks@{}%
      \EmFi@list\stop
    \fi
  \endgroup
}
%    \end{macrocode}
%    \end{macro}
%    \begin{macro}{\embedfilefinish}
%    \begin{macrocode}
\def\embedfilefinish{%
  \ifEmFi@finished
    \EmFi@Error{%
      Too many invocations of \string\embedfilefinish
    }{%
      The list of embedded files is already written.%
    }%
  \else
    \ifx\EmFi@list\ltx@empty
    \else
%    \end{macrocode}
%    Write |/EmbeddedFiles| entry.
%    \begin{macrocode}
      \global\EmFi@finishedtrue
      \begingroup
        \def\do##1##2{%
          (##1)##2%
        }%
        \immediate\pdfobj{%
          <<%
            /Names[\EmFi@list]%
          >>%
        }%
        \pdfnames{%
          /EmbeddedFiles \the\pdflastobj\ltx@space 0 R%
        }%
      \endgroup
%    \end{macrocode}
%    Write collection objects.
%    \begin{macrocode}
      \ifx\EmFi@initialfile\ltx@empty
      \else
        \EmFi@collectiontrue
      \fi
      \ifEmFi@collection
        \ifx\EmFi@initialfile\ltx@empty
          \let\EmFi@@initialfile\ltx@empty
        \else
          \edef\EmFi@@initialfile{%
            \pdf@escapestring{\EmFi@initialfile}%
          }%
        \fi
%    \end{macrocode}
%    Look for initial file among the embedded files.
%    \begin{macrocode}
        \begingroup
          \let\f=N%
          \def\do##1##2{%
            \def\x{##1}%
            \ifx\x\EmFi@@initialfile
              \let\f=Y%
              \let\do\ltx@gobbletwo
            \fi
          }%
          \EmFi@list
        \expandafter\endgroup
        \ifx\f Y%
        \else
          \@PackageWarningNoLine{embedfile}{%
            Missing initial file `\EmFi@initialfile'\MessageBreak
            among the embedded files%
          }%
          \let\EmFi@initialfile\ltx@empty
          \let\EmFi@@initialfile\ltx@empty
        \fi
        \ifcase\EmFi@sortcase
          \def\EmFi@temp{}%
        \or
          \def\EmFi@temp{%
            /S\EmFi@sortkeys
            /A \EmFi@sortorders
          }%
        \else
          \def\EmFi@temp{%
            /S[\EmFi@sortkeys]%
            /A[\EmFi@sortorders]%
          }%
        \fi
        \def\EmFi@@order##1{%
          \ifnum\EmFi@order>1 %
            /O ##1%
          \fi
        }%
        \immediate\pdfobj{%
          <<%
            \ifx\EmFi@schema\ltx@empty
            \else
              /Schema<<\EmFi@schema>>%
            \fi
            \ifx\EmFi@@initialfile\ltx@empty
            \else
              /D(\EmFi@@initialfile)%
            \fi
            \ifx\EmFi@view\EmFi@S@tile
              /View/T%
            \else\ifx\EmFi@view\EmFi@S@hidden
              /View/H%
            \fi\fi
            \ifx\EmFi@temp\ltx@empty
              \EmFi@temp
            \else
              /Sort<<\EmFi@temp>>%
            \fi
          >>%
        }%
        \pdfcatalog{%
          /Collection \the\pdflastobj\ltx@space0 R%
        }%
      \fi
    \fi
  \fi
}
%    \end{macrocode}
%    \end{macro}
%    \begin{macrocode}
\begingroup\expandafter\expandafter\expandafter\endgroup
\expandafter\ifx\csname AtEndDocument\endcsname\relax
\else
  \AtEndDocument{\embedfilefinish}%
\fi
%    \end{macrocode}
%
%    \begin{macrocode}
\EmFi@AtEnd%
%</package>
%    \end{macrocode}
%
% \section{Test}
%
% \subsection{Catcode checks for loading}
%
%    \begin{macrocode}
%<*test1>
%    \end{macrocode}
%    \begin{macrocode}
\catcode`\{=1 %
\catcode`\}=2 %
\catcode`\#=6 %
\catcode`\@=11 %
\expandafter\ifx\csname count@\endcsname\relax
  \countdef\count@=255 %
\fi
\expandafter\ifx\csname @gobble\endcsname\relax
  \long\def\@gobble#1{}%
\fi
\expandafter\ifx\csname @firstofone\endcsname\relax
  \long\def\@firstofone#1{#1}%
\fi
\expandafter\ifx\csname loop\endcsname\relax
  \expandafter\@firstofone
\else
  \expandafter\@gobble
\fi
{%
  \def\loop#1\repeat{%
    \def\body{#1}%
    \iterate
  }%
  \def\iterate{%
    \body
      \let\next\iterate
    \else
      \let\next\relax
    \fi
    \next
  }%
  \let\repeat=\fi
}%
\def\RestoreCatcodes{}
\count@=0 %
\loop
  \edef\RestoreCatcodes{%
    \RestoreCatcodes
    \catcode\the\count@=\the\catcode\count@\relax
  }%
\ifnum\count@<255 %
  \advance\count@ 1 %
\repeat

\def\RangeCatcodeInvalid#1#2{%
  \count@=#1\relax
  \loop
    \catcode\count@=15 %
  \ifnum\count@<#2\relax
    \advance\count@ 1 %
  \repeat
}
\def\RangeCatcodeCheck#1#2#3{%
  \count@=#1\relax
  \loop
    \ifnum#3=\catcode\count@
    \else
      \errmessage{%
        Character \the\count@\space
        with wrong catcode \the\catcode\count@\space
        instead of \number#3%
      }%
    \fi
  \ifnum\count@<#2\relax
    \advance\count@ 1 %
  \repeat
}
\def\space{ }
\expandafter\ifx\csname LoadCommand\endcsname\relax
  \def\LoadCommand{\input embedfile.sty\relax}%
\fi
\def\Test{%
  \RangeCatcodeInvalid{0}{47}%
  \RangeCatcodeInvalid{58}{64}%
  \RangeCatcodeInvalid{91}{96}%
  \RangeCatcodeInvalid{123}{255}%
  \catcode`\@=12 %
  \catcode`\\=0 %
  \catcode`\%=14 %
  \LoadCommand
  \RangeCatcodeCheck{0}{36}{15}%
  \RangeCatcodeCheck{37}{37}{14}%
  \RangeCatcodeCheck{38}{47}{15}%
  \RangeCatcodeCheck{48}{57}{12}%
  \RangeCatcodeCheck{58}{63}{15}%
  \RangeCatcodeCheck{64}{64}{12}%
  \RangeCatcodeCheck{65}{90}{11}%
  \RangeCatcodeCheck{91}{91}{15}%
  \RangeCatcodeCheck{92}{92}{0}%
  \RangeCatcodeCheck{93}{96}{15}%
  \RangeCatcodeCheck{97}{122}{11}%
  \RangeCatcodeCheck{123}{255}{15}%
  \RestoreCatcodes
}
\Test
\csname @@end\endcsname
\end
%    \end{macrocode}
%    \begin{macrocode}
%</test1>
%    \end{macrocode}
%
% \subsection{Simple test}
%
%    \begin{macrocode}
%<*test2>
\input embedfile.sty\relax
\embedfile[%
  stringmethod=escape,%
  mimetype=plain/text,%
  desc={LaTeX docstrip source archive for package `embedfile'},%
  id={embedfile.dtx}%
]{embedfile.dtx}
\nopagenumbers
Test (plain-\TeX): {\tt embedfile.dtx} should be embedded.%

\def\Test#1{%
  \par
  \embedfileifobjectexists{embedfile.dtx}{#1}{%
    Object #1 (embedfile.dtx): %
    \embedfilegetobject{embedfile.dtx}{#1}%
  }{%
    \errmessage{Missing object #1 (embedfile.dtx)}%
  }%
}
\Test{EmbeddedFile}
\Test{Filespec}
\embedfilefinish
\bye
%</test2>
%    \end{macrocode}
%    \begin{macrocode}
%<*test3>
\NeedsTeXFormat{LaTeX2e}
\let\SavedJobname\jobname
\def\jobname{embedfile}
\RequirePackage{dtx-attach}[2011/04/13]
\let\jobname\SavedJobname
\documentclass{minimal}
\begin{document}
  Test (\LaTeX): \texttt{embedfile.dtx} should be embedded.%
\end{document}
%</test3>
%    \end{macrocode}
%
% \subsection{Test for \iniTeX}
%
%    \begin{macrocode}
%<*test4>
\catcode`\{=1 %
\catcode`\}=2 %
\input ifluatex.sty %
\ifluatex
  \directlua{%
    tex.enableprimitives('', {%
      'pdflastobj',%
      'pdfnames',%
      'pdfobj',%
      'pdfoutput'%
    })%
  }%
\fi
\pdfoutput=1 %
\input embedfile.sty %
\shipout\hbox{}
\embedfile[%
  stringmethod=escape,%
  mimetype=plain/text,%
  desc={iniTeX source},%
]{\jobname.tex}
\embedfilefinish
\end
%</test4>
%    \end{macrocode}
%
% \section{Installation}
%
% \subsection{Download}
%
% \paragraph{Package.} This package is available on
% CTAN\footnote{\url{ftp://ftp.ctan.org/tex-archive/}}:
% \begin{description}
% \item[\CTAN{macros/latex/contrib/oberdiek/embedfile.dtx}] The source file.
% \item[\CTAN{macros/latex/contrib/oberdiek/embedfile.pdf}] Documentation.
% \end{description}
%
%
% \paragraph{Bundle.} All the packages of the bundle `oberdiek'
% are also available in a TDS compliant ZIP archive. There
% the packages are already unpacked and the documentation files
% are generated. The files and directories obey the TDS standard.
% \begin{description}
% \item[\CTAN{install/macros/latex/contrib/oberdiek.tds.zip}]
% \end{description}
% \emph{TDS} refers to the standard ``A Directory Structure
% for \TeX\ Files'' (\CTAN{tds/tds.pdf}). Directories
% with \xfile{texmf} in their name are usually organized this way.
%
% \subsection{Bundle installation}
%
% \paragraph{Unpacking.} Unpack the \xfile{oberdiek.tds.zip} in the
% TDS tree (also known as \xfile{texmf} tree) of your choice.
% Example (linux):
% \begin{quote}
%   |unzip oberdiek.tds.zip -d ~/texmf|
% \end{quote}
%
% \paragraph{Script installation.}
% Check the directory \xfile{TDS:scripts/oberdiek/} for
% scripts that need further installation steps.
% Package \xpackage{attachfile2} comes with the Perl script
% \xfile{pdfatfi.pl} that should be installed in such a way
% that it can be called as \texttt{pdfatfi}.
% Example (linux):
% \begin{quote}
%   |chmod +x scripts/oberdiek/pdfatfi.pl|\\
%   |cp scripts/oberdiek/pdfatfi.pl /usr/local/bin/|
% \end{quote}
%
% \subsection{Package installation}
%
% \paragraph{Unpacking.} The \xfile{.dtx} file is a self-extracting
% \docstrip\ archive. The files are extracted by running the
% \xfile{.dtx} through \plainTeX:
% \begin{quote}
%   \verb|tex embedfile.dtx|
% \end{quote}
%
% \paragraph{TDS.} Now the different files must be moved into
% the different directories in your installation TDS tree
% (also known as \xfile{texmf} tree):
% \begin{quote}
% \def\t{^^A
% \begin{tabular}{@{}>{\ttfamily}l@{ $\rightarrow$ }>{\ttfamily}l@{}}
%   embedfile.sty & tex/generic/oberdiek/embedfile.sty\\
%   dtx-attach.sty & tex/generic/oberdiek/dtx-attach.sty\\
%   embedfile.pdf & doc/latex/oberdiek/embedfile.pdf\\
%   embedfile-example-plain.tex & doc/latex/oberdiek/embedfile-example-plain.tex\\
%   embedfile-example-collection.tex & doc/latex/oberdiek/embedfile-example-collection.tex\\
%   test/embedfile-test1.tex & doc/latex/oberdiek/test/embedfile-test1.tex\\
%   test/embedfile-test2.tex & doc/latex/oberdiek/test/embedfile-test2.tex\\
%   test/embedfile-test3.tex & doc/latex/oberdiek/test/embedfile-test3.tex\\
%   test/embedfile-test4.tex & doc/latex/oberdiek/test/embedfile-test4.tex\\
%   embedfile.dtx & source/latex/oberdiek/embedfile.dtx\\
% \end{tabular}^^A
% }^^A
% \sbox0{\t}^^A
% \ifdim\wd0>\linewidth
%   \begingroup
%     \advance\linewidth by\leftmargin
%     \advance\linewidth by\rightmargin
%   \edef\x{\endgroup
%     \def\noexpand\lw{\the\linewidth}^^A
%   }\x
%   \def\lwbox{^^A
%     \leavevmode
%     \hbox to \linewidth{^^A
%       \kern-\leftmargin\relax
%       \hss
%       \usebox0
%       \hss
%       \kern-\rightmargin\relax
%     }^^A
%   }^^A
%   \ifdim\wd0>\lw
%     \sbox0{\small\t}^^A
%     \ifdim\wd0>\linewidth
%       \ifdim\wd0>\lw
%         \sbox0{\footnotesize\t}^^A
%         \ifdim\wd0>\linewidth
%           \ifdim\wd0>\lw
%             \sbox0{\scriptsize\t}^^A
%             \ifdim\wd0>\linewidth
%               \ifdim\wd0>\lw
%                 \sbox0{\tiny\t}^^A
%                 \ifdim\wd0>\linewidth
%                   \lwbox
%                 \else
%                   \usebox0
%                 \fi
%               \else
%                 \lwbox
%               \fi
%             \else
%               \usebox0
%             \fi
%           \else
%             \lwbox
%           \fi
%         \else
%           \usebox0
%         \fi
%       \else
%         \lwbox
%       \fi
%     \else
%       \usebox0
%     \fi
%   \else
%     \lwbox
%   \fi
% \else
%   \usebox0
% \fi
% \end{quote}
% If you have a \xfile{docstrip.cfg} that configures and enables \docstrip's
% TDS installing feature, then some files can already be in the right
% place, see the documentation of \docstrip.
%
% \subsection{Refresh file name databases}
%
% If your \TeX~distribution
% (\teTeX, \mikTeX, \dots) relies on file name databases, you must refresh
% these. For example, \teTeX\ users run \verb|texhash| or
% \verb|mktexlsr|.
%
% \subsection{Some details for the interested}
%
% \paragraph{Attached source.}
%
% The PDF documentation on CTAN also includes the
% \xfile{.dtx} source file. It can be extracted by
% AcrobatReader 6 or higher. Another option is \textsf{pdftk},
% e.g. unpack the file into the current directory:
% \begin{quote}
%   \verb|pdftk embedfile.pdf unpack_files output .|
% \end{quote}
%
% \paragraph{Unpacking with \LaTeX.}
% The \xfile{.dtx} chooses its action depending on the format:
% \begin{description}
% \item[\plainTeX:] Run \docstrip\ and extract the files.
% \item[\LaTeX:] Generate the documentation.
% \end{description}
% If you insist on using \LaTeX\ for \docstrip\ (really,
% \docstrip\ does not need \LaTeX), then inform the autodetect routine
% about your intention:
% \begin{quote}
%   \verb|latex \let\install=y% \iffalse meta-comment
%
% File: embedfile.dtx
% Version: 2011/04/13 v2.6
% Info: Embed files into PDF
%
% Copyright (C) 2006-2011 by
%    Heiko Oberdiek <heiko.oberdiek at googlemail.com>
%
% This work may be distributed and/or modified under the
% conditions of the LaTeX Project Public License, either
% version 1.3c of this license or (at your option) any later
% version. This version of this license is in
%    http://www.latex-project.org/lppl/lppl-1-3c.txt
% and the latest version of this license is in
%    http://www.latex-project.org/lppl.txt
% and version 1.3 or later is part of all distributions of
% LaTeX version 2005/12/01 or later.
%
% This work has the LPPL maintenance status "maintained".
%
% This Current Maintainer of this work is Heiko Oberdiek.
%
% The Base Interpreter refers to any `TeX-Format',
% because some files are installed in TDS:tex/generic//.
%
% This work consists of the main source file embedfile.dtx
% and the derived files
%    embedfile.sty, embedfile.pdf, embedfile.ins, embedfile.drv,
%    dtx-attach.sty, embedfile-example-plain.tex,
%    embedfile-example-collection.tex, embedfile-test1.tex,
%    embedfile-test2.tex, embedfile-test3.tex,
%    embedfile-test4.tex.
%
% Distribution:
%    CTAN:macros/latex/contrib/oberdiek/embedfile.dtx
%    CTAN:macros/latex/contrib/oberdiek/embedfile.pdf
%
% Unpacking:
%    (a) If embedfile.ins is present:
%           tex embedfile.ins
%    (b) Without embedfile.ins:
%           tex embedfile.dtx
%    (c) If you insist on using LaTeX
%           latex \let\install=y\input{embedfile.dtx}
%        (quote the arguments according to the demands of your shell)
%
% Documentation:
%    (a) If embedfile.drv is present:
%           latex embedfile.drv
%    (b) Without embedfile.drv:
%           latex embedfile.dtx; ...
%    The class ltxdoc loads the configuration file ltxdoc.cfg
%    if available. Here you can specify further options, e.g.
%    use A4 as paper format:
%       \PassOptionsToClass{a4paper}{article}
%
%    Programm calls to get the documentation (example):
%       pdflatex embedfile.dtx
%       makeindex -s gind.ist embedfile.idx
%       pdflatex embedfile.dtx
%       makeindex -s gind.ist embedfile.idx
%       pdflatex embedfile.dtx
%
% Installation:
%    TDS:tex/generic/oberdiek/embedfile.sty
%    TDS:tex/generic/oberdiek/dtx-attach.sty
%    TDS:doc/latex/oberdiek/embedfile.pdf
%    TDS:doc/latex/oberdiek/embedfile-example-plain.tex
%    TDS:doc/latex/oberdiek/embedfile-example-collection.tex
%    TDS:doc/latex/oberdiek/test/embedfile-test1.tex
%    TDS:doc/latex/oberdiek/test/embedfile-test2.tex
%    TDS:doc/latex/oberdiek/test/embedfile-test3.tex
%    TDS:doc/latex/oberdiek/test/embedfile-test4.tex
%    TDS:source/latex/oberdiek/embedfile.dtx
%
%<*ignore>
\begingroup
  \catcode123=1 %
  \catcode125=2 %
  \def\x{LaTeX2e}%
\expandafter\endgroup
\ifcase 0\ifx\install y1\fi\expandafter
         \ifx\csname processbatchFile\endcsname\relax\else1\fi
         \ifx\fmtname\x\else 1\fi\relax
\else\csname fi\endcsname
%</ignore>
%<*install>
\input docstrip.tex
\Msg{************************************************************************}
\Msg{* Installation}
\Msg{* Package: embedfile 2011/04/13 v2.6 Embed files into PDF (HO)}
\Msg{************************************************************************}

\keepsilent
\askforoverwritefalse

\let\MetaPrefix\relax
\preamble

This is a generated file.

Project: embedfile
Version: 2011/04/13 v2.6

Copyright (C) 2006-2011 by
   Heiko Oberdiek <heiko.oberdiek at googlemail.com>

This work may be distributed and/or modified under the
conditions of the LaTeX Project Public License, either
version 1.3c of this license or (at your option) any later
version. This version of this license is in
   http://www.latex-project.org/lppl/lppl-1-3c.txt
and the latest version of this license is in
   http://www.latex-project.org/lppl.txt
and version 1.3 or later is part of all distributions of
LaTeX version 2005/12/01 or later.

This work has the LPPL maintenance status "maintained".

This Current Maintainer of this work is Heiko Oberdiek.

The Base Interpreter refers to any `TeX-Format',
because some files are installed in TDS:tex/generic//.

This work consists of the main source file embedfile.dtx
and the derived files
   embedfile.sty, embedfile.pdf, embedfile.ins, embedfile.drv,
   dtx-attach.sty, embedfile-example-plain.tex,
   embedfile-example-collection.tex, embedfile-test1.tex,
   embedfile-test2.tex, embedfile-test3.tex,
   embedfile-test4.tex.

\endpreamble
\let\MetaPrefix\DoubleperCent

\generate{%
  \file{embedfile.ins}{\from{embedfile.dtx}{install}}%
  \file{embedfile.drv}{\from{embedfile.dtx}{driver}}%
  \usedir{tex/generic/oberdiek}%
  \file{embedfile.sty}{\from{embedfile.dtx}{package}}%
  \usedir{tex/latex/oberdiek}%
  \file{dtx-attach.sty}{\from{embedfile.dtx}{dtxattach}}%
  \usedir{doc/latex/oberdiek}%
  \file{embedfile-example-plain.tex}{\from{embedfile.dtx}{exampleplain}}%
  \file{embedfile-example-collection.tex}{\from{embedfile.dtx}{examplecollection}}%
  \usedir{doc/latex/oberdiek/test}%
  \file{embedfile-test1.tex}{\from{embedfile.dtx}{test1}}%
  \file{embedfile-test2.tex}{\from{embedfile.dtx}{test2}}%
  \file{embedfile-test3.tex}{\from{embedfile.dtx}{test3}}%
  \file{embedfile-test4.tex}{\from{embedfile.dtx}{test4}}%
  \nopreamble
  \nopostamble
  \usedir{source/latex/oberdiek/catalogue}%
  \file{embedfile.xml}{\from{embedfile.dtx}{catalogue}}%
}

\catcode32=13\relax% active space
\let =\space%
\Msg{************************************************************************}
\Msg{*}
\Msg{* To finish the installation you have to move the following}
\Msg{* files into a directory searched by TeX:}
\Msg{*}
\Msg{*     embedfile.sty, dtx-attach.sty}
\Msg{*}
\Msg{* To produce the documentation run the file `embedfile.drv'}
\Msg{* through LaTeX.}
\Msg{*}
\Msg{* Happy TeXing!}
\Msg{*}
\Msg{************************************************************************}

\endbatchfile
%</install>
%<*ignore>
\fi
%</ignore>
%<*driver>
\NeedsTeXFormat{LaTeX2e}
\ProvidesFile{embedfile.drv}%
  [2011/04/13 v2.6 Embed files into PDF (HO)]%
\documentclass{ltxdoc}
\usepackage{holtxdoc}[2011/11/22]
\begin{document}
  \DocInput{embedfile.dtx}%
\end{document}
%</driver>
% \fi
%
% \CheckSum{1363}
%
% \CharacterTable
%  {Upper-case    \A\B\C\D\E\F\G\H\I\J\K\L\M\N\O\P\Q\R\S\T\U\V\W\X\Y\Z
%   Lower-case    \a\b\c\d\e\f\g\h\i\j\k\l\m\n\o\p\q\r\s\t\u\v\w\x\y\z
%   Digits        \0\1\2\3\4\5\6\7\8\9
%   Exclamation   \!     Double quote  \"     Hash (number) \#
%   Dollar        \$     Percent       \%     Ampersand     \&
%   Acute accent  \'     Left paren    \(     Right paren   \)
%   Asterisk      \*     Plus          \+     Comma         \,
%   Minus         \-     Point         \.     Solidus       \/
%   Colon         \:     Semicolon     \;     Less than     \<
%   Equals        \=     Greater than  \>     Question mark \?
%   Commercial at \@     Left bracket  \[     Backslash     \\
%   Right bracket \]     Circumflex    \^     Underscore    \_
%   Grave accent  \`     Left brace    \{     Vertical bar  \|
%   Right brace   \}     Tilde         \~}
%
% \GetFileInfo{embedfile.drv}
%
% \title{The \xpackage{embedfile} package}
% \date{2011/04/13 v2.6}
% \author{Heiko Oberdiek\\\xemail{heiko.oberdiek at googlemail.com}}
%
% \maketitle
%
% \begin{abstract}
% This package embeds files to a PDF document.
% Currently the only supported driver is \pdfTeX\ $>=$ 1.30 in PDF mode.
% \end{abstract}
%
% \tableofcontents
%
% \section{Documentation}
%
% \subsection{Introduction}
%
%    The PDF format (\cite{pdfspec}) allows the inclusion of files
%    inside the PDF document. The included files can be bound to an
%    annotation on a page. Or they can be recorded in a sorted
%    list of embedded files. The packages \xpackage{attachfile}
%    or \xpackage{attachfile2} follow the first approach, this
%    package uses the latter method.
%
% \subsubsection{Future development}
%
%    My dream is a large package that merges the features of
%    all these packages meantioned before:
%    \begin{itemize}
%    \item Files can be attached to a page.
%    \item Files can be attached to the document.
%    \item An easy user interface for simple, common tasks and
%          beginners.
%    \item An interface for the advanced users that want to setup
%          every detail.
%    \item Support of many drivers (pdftex, dvips, dvipdfm, \dots).
%    \item \dots
%    \end{itemize}
%    However, I have not managed to take the time for this project.
%    Instead:
%    \begin{itemize}
%    \item First I experimented with package \xpackage{attachfile},
%          adding driver support, fixing bugs, \dots. The result is
%          currently named as \xpackage{attachfile2}. It uses an external
%          script to get file properties (size, date, checksum, \dots).
%    \item In order to avoid an external program for getting basic
%          file properties I provided a patch ``EscapeAndOther'' for
%          pdfTeX that was accepted for version 1.30.
%    \item This package closes a gap left by the packages for attaching
%          files and allows the embedding of files to the document.
%          Also it makes use of the new primitives of \pdfTeX.
%    \end{itemize}
%
% \subsection{User interface}
%
%    This package \xpackage{embedfile} can be used with both
%    \LaTeX\ and \plainTeX. See \autoref{sec:plain} that
%    explains the use with \plainTeX\ by an example.
%    In \LaTeX\ the package is loaded as usually. There are no options.
%    \begin{quote}
%      \verb|\usepackage{embedfile}|
%    \end{quote}
%
%    \begin{declcs}{embedfile}
%      |[|\meta{options}|]| \M{file}
%    \end{declcs}
%    The macro \cs{embedfile} includes file \meta{file} and
%    attaches it to the PDF document. At the end of the document
%    the sorted list of embedded files are written. Thus you can
%    safely use \cs{embedfile} before \verb|\end{document}|.
%    Embedding files using \cs{AtEndDocument} will only work,
%    if \cs{AtEndDocument} is called before loading the package
%    \xpackage{embedfile}.
%
%    The \meta{options} are give as key value pairs. The following
%    keys are supported:
%    \begin{description}
%    \item[\xoption{filespec}]
%      This allows to override the file name
%      that appears in the PDF file. If you are using other than
%      simple file names (8-bit, path separators, \dots), look into
%      the PDF specification (\cite{pdfspec}).
%      There are rules how these file names must be written/encoded.
%      Avoid 8-bit characters
%      and other special characters, the behaviour is currently undefined.
%      Use option \xoption{ucfilespec} for more funny file names.
%      The string method, see below, is |escape| since version 2.4.
%
%      This name is also used as entry in a name tree
%      (see PDF specification: |/EmbeddedFiles|). Therefore the
%      value for \xoption{filespec} must be unique among all
%      embedded files. Also key \xoption{initialfiles} refers
%      to this name, if the file name and the value of \xoption{filespec}
%      are different.
%    \item[\xoption{ucfilespec}]
%      Since PDF 1.7 the file name may be provided in Unicode.
%      The conversion of the option value into a PDF string
%      is controlled by option \xoption{stringmethod}.
%    \item[\xoption{filesystem}]
%      This sets the entry \verb|/FS| in
%      the file specification dictionary, see PDF specification
%      (\cite{pdfspec}).
%      Example: \verb|filesystem=URL|.
%    \item[\xoption{mimetype}]
%      This sets the mime type (\cite{mime}) of the file,
%      see \autoref{sec:plain} for examples and \cite{types}
%      for a list of officially registered types.
%    \item[\xoption{desc}]
%      The description for the file.
%    \item[\xoption{stringmethod}]
%      The package must convert the values of the keys \xoption{ucfilespec}
%      and \xoption{desc} into a PDF string (before version 2.4: \xoption{filespec}
%      and \xoption{desc}). If \xpackage{hyperref}
%      is found, then its \cs{pdfstringdef} will be used, otherwise
%      \pdfTeX's \cs{pdfescapestring} is used. Value \verb|psd|
%      forces the use of \cs{pdfstringdef}, value \verb|escape|
%      the use of \cs{pdfescapestring}.
%    \item[\xoption{\meta{key}.value}] Sets the value of a collection
%      item property, see section \ref{sec:collection}.
%    \item[\xoption{\meta{key}.prefix}]
%      Sets the prefix of a collection item property,
%      see section \ref{sec:collection}.
%    \item[\xoption{id}]
%      The value must be an unique name. Macros
%      \cs{embedfileifobjectexists} and \cs{embedfilegetobject}
%      are using this name later.
%    \end{description}
%
%    \begin{declcs}{embedfilefinish}
%    \end{declcs}
%    The list of all embedded files must be added as data structure
%    in the PDF file. In case of \LaTeX\ this is automatically done.
%    The package uses \cs{AtEndDocument}. Then the list of all
%    files should be known. However, \plainTeX\ does not know
%    about \cs{AtEndDocument}. Thus the user must call
%    \cs{embedfilefinish} at the end of the document
%    after the last file is embedded.
%
%    \begin{declcs}{embedfilesetup} \M{options}
%    \end{declcs}
%    Options for \cs{embedfile} and collection support
%    can be set in \cs{embedfilesetup}.
%
% \subsection{Collection support (PDF 1.7)}
% \label{sec:collection}
%
%    Since PDF 1.7 the embedded files can form a \emph{collection}
%    (sometimes referred as \emph{package}), the main document
%    is called \emph{cover sheet}. See PDF specification
%    8.2.4 ``Collections'' and 3.10.5 ``Collection items'' \cite{pdfspec}.
%
%    Usually Acrobat Reader 7 or 8 shows the
%    embedded files in a table at the bottom with the following columns:
%    \begin{quote}
%    \begin{tabular}{@{}l|l|l|l@{}}
%       Name & Description & Modified & Size\\
%       \hline
%       \dots & \dots & \dots & \dots\\
%    \end{tabular}
%    \end{quote}
%    Acrobat Reader 10 shows the embeded files in the left panel
%    and adds a new column for the compressed size.
%
%    If the files form a collection, then they are displayed
%    in a table left or top (depending on option \xoption{view},
%    see \cs{embedfilesetup}).
%
%
%    Collection support is enabled automatically, if
%    it is used.
%
%    \begin{declcs}{embedfilesetup} \M{options}
%    \end{declcs}
%    The following options are supported in addition
%    to options for \cs{embedfile}:
%    \begin{description}
%    \item[\xoption{view}]
%      If the PDF file contains a collection, then Acrobat Reader 8
%      shows a line at the top below the menu bar and the toolbar.
%      It shows the current selected file, icons for changing the
%      view mode, an options menu.
%      The initial mode how the collection is presented is
%      set by this option \xoption{view}.
%      The following modes/values are supported, the default is |details|:
%        \begin{description}
%        \item[\texttt{details}] The full collection table is displayed
%          at the top below the collection bar.
%        \item[\texttt{tile}] The files of the collection are shown
%          in tile mode on the left.
%        \item[\texttt{hidden}] The collection table is not shown.
%        \end{description}
%    \item[\xoption{initialfile}]
%      Selects the file that is initially presented. Especially
%      useful for an embedded PDF file that is then shown instead
%      of the cover document. There must be an \cs{embedfile} command
%      somewhere whose value for key \xoption{filespec} is used here.
%      The \cs{embedfile} command can drop option \xoption{filespec}
%      if the file name is not different.
%    \end{description}
%
%    \begin{declcs}{embedfilefield} \M{key} \M{options}
%    \end{declcs}
%    Macro \cs{embedfilefield} defines a column/field in the collection table.
%    The name of the field is \meta{key}.
%    \begin{description}
%    \item[\xoption{type}]
%      sets the type of the field. The supported values are:
%      \begin{description}
%      \item[\texttt{text}] A text field. Its value is set in
%        \cs{embedfile} by option \xoption{\meta{key}.value}.
%      \item[\texttt{date}] A date field. Its value is set in
%        \cs{embedfile} by option \xoption{\meta{key}.value}.
%        A special format is required, see ``3.8.3 Dates'' \cite{pdfspec}.
%      \item[\texttt{number}] A field with an integer or float number.
%        Its value is set in \cs{embedfile} by option
%        \xoption{\meta{key}.value}.
%      \item[\texttt{file}] The file name of the embedded file.
%      \item[\texttt{desc}] The description text of the embedded file. It is
%       set in \cs{embedfile} by option \xoption{desc}.
%      \item[\texttt{moddate}] The modification date of the embedded file.
%      \item[\texttt{size}] The size of the embedded file.
%      \end{description}
%      All types allow the use of a prefix that is disregarded by
%      sorting. The prefix for this field is set in \cs{embedfile} by
%      option \xoption{\meta{key}.prefix}.
%    \item[\xoption{title}] sets the column title.
%    \item[\xoption{visible}] controls whether the column is presented:
%      \begin{description}
%      \item[\texttt{true}] shows the column.
%      \item[\texttt{false}] hides the column.
%      \end{description}
%      Default: \texttt{true}
%    \item[\xoption{edit}]
%      Allows the editing of field values. Does not seem to have an
%      effect for Acrobat Reader.
%      \begin{description}
%      \item[\texttt{true}] enables the feature, if available (depends
%        on the PDF viewer).
%      \item[\texttt{false}] disables the feature.
%      \end{description}
%      Default: \texttt{false}
%    \end{description}
%    The order of \cs{embedfilefield} statements defines the order
%    of the columns.
%
%    \begin{declcs}{embedfilesort} \M{key-sort-list}
%    \end{declcs}
%    The sort order of the embedded files are controlled by
%    macro \cs{embedfilesort}. \meta{key-sort-list} defines
%    the sort order. The key is a field name defined by \cs{embedfilefield}.
%    Its value is either \texttt{ascending} or \texttt{descending}.
%    The default is \texttt{ascending}.
%
% \subsection{Export of object references}
%
%    Caution: This feature is still experimental. It may be even removed
%    in future versions. Therefore feedback would be nice, if someone
%    has a useful application for this feature.
%
%    Object numbers are saved, if \xoption{id} is given in \cs{embedfile}.
%    The following objects are supported:
%    \begin{itemize}
%    \item \texttt{EmbeddedFile}
%    \item \texttt{Filespec}
%    \end{itemize}
%
%    \begin{declcs}{embedfileifobjectexists} \M{id} \M{type}
%      \M{then} \M{else}
%    \end{declcs}
%    Macro \cs{embedfileifobjectexists} tests whether object of \meta{type}
%    is available for the embedded file identified by \meta{id}.
%
%    \begin{declcs}{embedfilegetobject} \M{id} \M{type}
%    \end{declcs}
%    Macro \cs{embedfilegetobject} expands to the full object reference
%    object of \meta{type} for the embedded file identified by \meta{id}.
%
% \subsubsection{Example}
%
%\begin{quote}
%\begin{verbatim}
%\embedfile[id={foo}]{foo.pdf}
%\embedfileifobjectexists{foo}{Filespec}{%
%  \typeout{%
%    FileSpec object for `foo': %
%    \embedfilegetobject{foo}{Filespec}%
%  }%
%}{%
%  \typeout{No Filespec object for `foo'}%
%}
%\end{verbatim}
%\end{quote}
%
% \subsection{Examples}
% \label{sec:examples}
%
% \subsubsection{\plainTeX}
% \label{sec:plain}
%
%    The package can be used with \plainTeX. It can be used
%    with or without help from \xfile{miniltx.tex}.
%
%    If additionally package \xpackage{keyval} (\xpackage{graphicx})
%    is needed, load it first. Then package \xpackage{embedfile} avoids
%    a duplicate loading of package \xpackage{keyval}.
%
%    Because \plainTeX\ does not provide a hook at end of the document,
%    you have to call \cs{embedfilefinish} manually at the end
%    after the last embedded file.
%    \begin{macrocode}
%<*exampleplain>
%<<END
% Load packages
\input miniltx
% \def\Gin@driver{pdftex.def}
% \input graphicx.sty
\input embedfile.sty
\resetatcatcode

% default setting
\embedfilesetup{
  mimetype=text/plain
}

% Embed files
\embedfile[
  filespec=example.tex,
  desc={Source code (plain-TeX) of this example}
]{embedfile-example-plain.tex}

\embedfile[
  desc={Source of package `embedfile'}
]{embedfile.dtx}

\embedfile[
  mimetype=application/pdf,
  desc={Documentation of package `embedfile'}
]{embedfile.pdf}

% Some text
This example document contains three embedded files.

% End of document
\embedfilefinish % don't forget
\bye
%END
%</exampleplain>
%    \end{macrocode}
%
% \subsubsection{Collection example}
%
%    \begin{macrocode}
%<*examplecollection>
%<<END
\NeedsTeXFormat{LaTeX2e}
\documentclass{article}
\usepackage[bookmarks=false]{hyperref}
% provides \pdfstringdef that is then used by `title' and
% other keys.
\usepackage{embedfile}[2011/04/13]
\embedfilesetup{
  view=details,
  initialfile=embedfile.pdf
}
\embedfilefield{file}{
  type=file,
  title={File name}
}
\embedfilefield{description}{
  type=desc,
  title={Description}
}
\embedfilefield{date}{
  type=moddate,
  title={Date}
}
\embedfilefield{size}{
  type=size,
  title={Size}
}
\embedfilefield{type}{
  type=text,
  title={Type},
  visible=false
}
\embedfilesort{
  type,
  date=descending
}
\begin{document}
An example for embedded files as collection.
You need Acrobat Reader 8 or higher.

\embedfile[
  desc={Source file of package `embedfile'},
  description.prefix={Package: },
  type.value={DTX}
]{embedfile.dtx}

\embedfile[
  desc={Documentation of package `embedfile'},
  description.prefix={Package: },
  type.value={PDF}
]{embedfile.pdf}

\embedfile[
  desc={The source for this example},
  description.prefix={Example: },
  type.value={TEX}
]{\jobname.tex}

\end{document}
%END
%</examplecollection>
%    \end{macrocode}
%
% \subsection{Package \xpackage{dtx-attach}}
%
% Package \xpackage{dtx-attach} is just a small application of
% package \xpackage{embedfile}. I am using it for the CTAN
% documentation of my packages in \CTAN{macros/latex/contrib/oberdiek/}.
% It also serves as small example for the use of the package with \LaTeX.
%
%    \begin{macrocode}
%<*dtxattach>
\NeedsTeXFormat{LaTeX2e}
\ProvidesPackage{dtx-attach}
  [2011/04/13 v2.6 Embed \string\jobname.dtx (HO)]%
\RequirePackage{embedfile}[2011/04/13]
\embedfile[%
  stringmethod=escape,%
  mimetype=plain/text,%
  desc={LaTeX docstrip source archive for package `\jobname'}%
]{\jobname.dtx}
%</dtxattach>
%    \end{macrocode}
%
% \StopEventually{
% }
%
% \section{Implementation}
%
%    \begin{macrocode}
%<*package>
%    \end{macrocode}
%
% \subsection{Reload check and package identification}
%    Reload check, especially if the package is not used with \LaTeX.
%    \begin{macrocode}
\begingroup\catcode61\catcode48\catcode32=10\relax%
  \catcode13=5 % ^^M
  \endlinechar=13 %
  \catcode35=6 % #
  \catcode39=12 % '
  \catcode44=12 % ,
  \catcode45=12 % -
  \catcode46=12 % .
  \catcode58=12 % :
  \catcode64=11 % @
  \catcode123=1 % {
  \catcode125=2 % }
  \expandafter\let\expandafter\x\csname ver@embedfile.sty\endcsname
  \ifx\x\relax % plain-TeX, first loading
  \else
    \def\empty{}%
    \ifx\x\empty % LaTeX, first loading,
      % variable is initialized, but \ProvidesPackage not yet seen
    \else
      \expandafter\ifx\csname PackageInfo\endcsname\relax
        \def\x#1#2{%
          \immediate\write-1{Package #1 Info: #2.}%
        }%
      \else
        \def\x#1#2{\PackageInfo{#1}{#2, stopped}}%
      \fi
      \x{embedfile}{The package is already loaded}%
      \aftergroup\endinput
    \fi
  \fi
\endgroup%
%    \end{macrocode}
%    Package identification:
%    \begin{macrocode}
\begingroup\catcode61\catcode48\catcode32=10\relax%
  \catcode13=5 % ^^M
  \endlinechar=13 %
  \catcode35=6 % #
  \catcode39=12 % '
  \catcode40=12 % (
  \catcode41=12 % )
  \catcode44=12 % ,
  \catcode45=12 % -
  \catcode46=12 % .
  \catcode47=12 % /
  \catcode58=12 % :
  \catcode64=11 % @
  \catcode91=12 % [
  \catcode93=12 % ]
  \catcode123=1 % {
  \catcode125=2 % }
  \expandafter\ifx\csname ProvidesPackage\endcsname\relax
    \def\x#1#2#3[#4]{\endgroup
      \immediate\write-1{Package: #3 #4}%
      \xdef#1{#4}%
    }%
  \else
    \def\x#1#2[#3]{\endgroup
      #2[{#3}]%
      \ifx#1\@undefined
        \xdef#1{#3}%
      \fi
      \ifx#1\relax
        \xdef#1{#3}%
      \fi
    }%
  \fi
\expandafter\x\csname ver@embedfile.sty\endcsname
\ProvidesPackage{embedfile}%
  [2011/04/13 v2.6 Embed files into PDF (HO)]%
%    \end{macrocode}
%
% \subsection{Catcodes}
%
%    \begin{macrocode}
\begingroup\catcode61\catcode48\catcode32=10\relax%
  \catcode13=5 % ^^M
  \endlinechar=13 %
  \catcode123=1 % {
  \catcode125=2 % }
  \catcode64=11 % @
  \def\x{\endgroup
    \expandafter\edef\csname EmFi@AtEnd\endcsname{%
      \endlinechar=\the\endlinechar\relax
      \catcode13=\the\catcode13\relax
      \catcode32=\the\catcode32\relax
      \catcode35=\the\catcode35\relax
      \catcode61=\the\catcode61\relax
      \catcode64=\the\catcode64\relax
      \catcode123=\the\catcode123\relax
      \catcode125=\the\catcode125\relax
    }%
  }%
\x\catcode61\catcode48\catcode32=10\relax%
\catcode13=5 % ^^M
\endlinechar=13 %
\catcode35=6 % #
\catcode64=11 % @
\catcode123=1 % {
\catcode125=2 % }
\def\TMP@EnsureCode#1#2{%
  \edef\EmFi@AtEnd{%
    \EmFi@AtEnd
    \catcode#1=\the\catcode#1\relax
  }%
  \catcode#1=#2\relax
}
\TMP@EnsureCode{39}{12}% '
\TMP@EnsureCode{40}{12}% (
\TMP@EnsureCode{41}{12}% )
\TMP@EnsureCode{44}{12}% ,
\TMP@EnsureCode{46}{12}% .
\TMP@EnsureCode{47}{12}% /
\TMP@EnsureCode{58}{12}% :
\TMP@EnsureCode{60}{12}% <
\TMP@EnsureCode{62}{12}% >
\TMP@EnsureCode{91}{12}% [
\TMP@EnsureCode{93}{12}% ]
\TMP@EnsureCode{96}{12}% `
\edef\EmFi@AtEnd{\EmFi@AtEnd\noexpand\endinput}
%    \end{macrocode}
%
% \subsection{Tools}
%
%    \begin{macro}{\EmFi@RequirePackage}
%    \begin{macrocode}
\begingroup\expandafter\expandafter\expandafter\endgroup
\expandafter\ifx\csname RequirePackage\endcsname\relax
  \def\EmFi@RequirePackage#1[#2]{%
    \input #1.sty\relax
  }%
\else
  \let\EmFi@RequirePackage\RequirePackage
\fi
%    \end{macrocode}
%    \end{macro}
%
%    \begin{macro}{\EmFi@Error}
%    \begin{macrocode}
\EmFi@RequirePackage{infwarerr}[2007/09/09]%
\def\EmFi@Error{%
  \@PackageError{embedfile}%
}
%    \end{macrocode}
%    \end{macro}
%
% \subsection{Check for recent \pdfTeX\ in PDF mode}
%
%    Load package \xpackage{ifpdf} and check mode.
%    \begin{macrocode}
\EmFi@RequirePackage{ifpdf}[2007/09/09]
\ifpdf
\else
  \EmFi@Error{%
    Missing pdfTeX in PDF mode%
  }{%
    Currently other drivers are not supported. %
    Package loading is aborted.%
  }%
  \expandafter\EmFi@AtEnd
\fi%
%    \end{macrocode}
%    \begin{macrocode}
\EmFi@RequirePackage{pdftexcmds}[2007/11/11]
\EmFi@RequirePackage{ltxcmds}[2010/03/01]
\EmFi@RequirePackage{kvsetkeys}[2010/03/01]
\EmFi@RequirePackage{kvdefinekeys}[2010/03/01]
%    \end{macrocode}
%    Check version.
%    \begin{macrocode}
\begingroup\expandafter\expandafter\expandafter\endgroup
\expandafter\ifx\csname pdf@filesize\endcsname\relax
  \EmFi@Error{%
    Unsupported pdfTeX version%
  }{%
    At least version 1.30 is necessary. Package loading is aborted.%
  }%
  \expandafter\EmFi@AtEnd
\fi%
%    \end{macrocode}
%
% \subsection{Strings}
%
%    Minimal version of package \xpackage{pdfescape} is 2007/08/27 v1.5
%    because of \cs{EdefSanitize}.
%    \begin{macrocode}
\EmFi@RequirePackage{pdfescape}[2007/11/11]
%    \end{macrocode}
%
%    \begin{macrocode}
\def\EmFi@temp#1{%
  \expandafter\EdefSanitize\csname EmFi@S@#1\endcsname{#1}%
}
%    \end{macrocode}
%    \begin{macro}{\EmFi@details}
%    \begin{macrocode}
\EmFi@temp{details}%
%    \end{macrocode}
%    \end{macro}
%    \begin{macro}{\EmFi@tile}
%    \begin{macrocode}
\EmFi@temp{tile}%
%    \end{macrocode}
%    \end{macro}
%    \begin{macro}{\EmFi@hidden}
%    \begin{macrocode}
\EmFi@temp{hidden}%
%    \end{macrocode}
%    \end{macro}
%    \begin{macro}{\EmFi@S@text}
%    \begin{macrocode}
\EmFi@temp{text}
%    \end{macrocode}
%    \end{macro}
%    \begin{macro}{\EmFi@S@date}
%    \begin{macrocode}
\EmFi@temp{date}
%    \end{macrocode}
%    \end{macro}
%    \begin{macro}{\EmFi@S@number}
%    \begin{macrocode}
\EmFi@temp{number}
%    \end{macrocode}
%    \end{macro}
%    \begin{macro}{\EmFi@S@file}
%    \begin{macrocode}
\EmFi@temp{file}
%    \end{macrocode}
%    \end{macro}
%    \begin{macro}{\EmFi@S@desc}
%    \begin{macrocode}
\EmFi@temp{desc}
%    \end{macrocode}
%    \end{macro}
%    \begin{macro}{\EmFi@S@moddate}
%    \begin{macrocode}
\EmFi@temp{moddate}
%    \end{macrocode}
%    \end{macro}
%    \begin{macro}{\EmFi@S@creationdate}
%    \begin{macrocode}
\EmFi@temp{creationdate}
%    \end{macrocode}
%    \end{macro}
%    \begin{macro}{\EmFi@S@size}
%    \begin{macrocode}
\EmFi@temp{size}
%    \end{macrocode}
%    \end{macro}
%    \begin{macro}{\EmFi@S@ascending}
%    \begin{macrocode}
\EmFi@temp{ascending}
%    \end{macrocode}
%    \end{macro}
%    \begin{macro}{\EmFi@S@descending}
%    \begin{macrocode}
\EmFi@temp{descending}
%    \end{macrocode}
%    \end{macro}
%    \begin{macro}{\EmFi@S@true}
%    \begin{macrocode}
\EmFi@temp{true}
%    \end{macrocode}
%    \end{macro}
%    \begin{macro}{\EmFi@S@false}
%    \begin{macrocode}
\EmFi@temp{false}
%    \end{macrocode}
%    \end{macro}
%
% \subsection{Switches}
%
%    \begin{macro}{\ifEmFi@collection}
%    \begin{macrocode}
\ltx@newif\ifEmFi@collection
%    \end{macrocode}
%    \end{macro}
%    \begin{macro}{\ifEmFi@sort}
%    \begin{macrocode}
\ltx@newif\ifEmFi@sort
%    \end{macrocode}
%    \end{macro}
%    \begin{macro}{\ifEmFi@visible}
%    \begin{macrocode}
\ltx@newif\ifEmFi@visible
%    \end{macrocode}
%    \end{macro}
%    \begin{macro}{\ifEmFi@edit}
%    \begin{macrocode}
\ltx@newif\ifEmFi@edit
%    \end{macrocode}
%    \end{macro}
%    \begin{macro}{\ifEmFi@item}
%    \begin{macrocode}
\ltx@newif\ifEmFi@item
%    \end{macrocode}
%    \end{macro}
%    \begin{macro}{\ifEmFi@finished}
%    \begin{macrocode}
\ltx@newif\ifEmFi@finished
%    \end{macrocode}
%    \end{macro}
%    \begin{macro}{\ifEmFi@id}
%    \begin{macrocode}
\ltx@newif\ifEmFi@id
%    \end{macrocode}
%    \end{macro}
%
% \subsection{Key value definitions}
%
%    \begin{macro}{\EmFi@GlobalKey}
%    \begin{macrocode}
\def\EmFi@GlobalKey#1#2{%
  \global\expandafter\let\csname KV@#1@#2\expandafter\endcsname
                         \csname KV@#1@#2\endcsname
}
%    \end{macrocode}
%    \end{macro}
%    \begin{macro}{\EmFi@GlobalDefaultKey}
%    \begin{macrocode}
\def\EmFi@GlobalDefaultKey#1#2{%
  \EmFi@GlobalKey{#1}{#2}%
  \global\expandafter\let
      \csname KV@#1@#2@default\expandafter\endcsname
      \csname KV@#1@#2@default\endcsname
}
%    \end{macrocode}
%    \end{macro}
%
%    \begin{macro}{\EmFi@DefineKey}
%    \begin{macrocode}
\def\EmFi@DefineKey#1#2{%
  \kv@define@key{EmFi}{#1}{%
    \expandafter\def\csname EmFi@#1\endcsname{##1}%
  }%
  \expandafter\def\csname EmFi@#1\endcsname{#2}%
}
%    \end{macrocode}
%    \end{macro}
%    Subtype of the embedded file (optional).
%    \begin{macrocode}
\EmFi@DefineKey{mimetype}{}
%    \end{macrocode}
%    File specification string.
%    \begin{macrocode}
\EmFi@DefineKey{filespec}{\EmFi@file}
%    \end{macrocode}
%    File specification string in Unicode.
%    \begin{macrocode}
\EmFi@DefineKey{ucfilespec}{}
%    \end{macrocode}
%    File system (optional).
%    \begin{macrocode}
\EmFi@DefineKey{filesystem}{}
%    \end{macrocode}
%    Description (optional).
%    \begin{macrocode}
\EmFi@DefineKey{desc}{}
%    \end{macrocode}
%    Method for converting text to PDF strings.
%    \begin{macrocode}
\EmFi@DefineKey{stringmethod}{%
  \ifx\pdfstringdef\@undefined
    escape%
  \else
    \ifx\pdfstringdef\relax
      escape%
    \else
      psd%
    \fi
  \fi
}
%    \end{macrocode}
%
%    Option \xoption{id} as key for object numbers.
%    \begin{macrocode}
\kv@define@key{EmFi}{id}{%
  \def\EmFi@id{#1}%
  \EmFi@idtrue
}
%    \end{macrocode}
%    \begin{macro}{\EmFi@defobj}
%    \begin{macrocode}
\def\EmFi@defobj#1{%
  \ifEmFi@id
    \expandafter\xdef\csname EmFi@#1@\EmFi@id\endcsname{%
      \the\pdflastobj\ltx@space 0 R%
    }%
  \fi
}
%    \end{macrocode}
%    \end{macro}
%    \begin{macro}{\embedfileifobjectexists}
%    \begin{macrocode}
\def\embedfileifobjectexists#1#2{%
  \expandafter\ifx\csname EmFi@#2@#1\endcsname\relax
    \expandafter\ltx@secondoftwo
  \else
    \expandafter\ltx@firstoftwo
  \fi
}
%    \end{macrocode}
%    \end{macro}
%    \begin{macro}{\embedfilegetobject}
%    \begin{macrocode}
\def\embedfilegetobject#1#2{%
  \embedfileifobjectexists{#1}{#2}{%
    \csname EmFi@#2@#1\endcsname
  }{%
    0 0 R%
  }%
}
%    \end{macrocode}
%    \end{macro}
%
%    Initial view of the collection.
%    \begin{macrocode}
\kv@define@key{EmFi}{view}[]{%
  \EdefSanitize\EmFi@temp{#1}%
  \def\EmFi@next{%
    \global\EmFi@collectiontrue
  }%
  \ifx\EmFi@temp\ltx@empty
    \let\EmFi@view\EmFi@S@details
  \else\ifx\EmFi@temp\EmFi@S@details
    \let\EmFi@view\EmFi@S@details
  \else\ifx\EmFi@temp\EmFi@S@tile
    \let\EmFi@view\EmFi@S@tile
  \else\ifx\EmFi@temp\EmFi@S@hidden
    \let\EmFi@view\EmFi@S@hidden
  \else
    \let\EmFi@next\relax
    \EmFi@Error{%
      Unknown value `\EmFi@temp' for key `view'.\MessageBreak
      Supported values: `details', `tile', `hidden'.%
    }\@ehc
  \fi\fi\fi\fi
  \EmFi@next
}
%    \end{macrocode}
%    \begin{macrocode}
\EmFi@DefineKey{initialfile}{}
%    \end{macrocode}
%
%    \begin{macro}{\embedfilesetup}
%    \begin{macrocode}
\def\embedfilesetup{%
  \ifEmFi@finished
    \def\EmFi@next##1{}%
    \EmFi@Error{%
      \string\embedfilefield\ltx@space after \string\embedfilefinish
    }{%
      The list of embedded files is already written.%
    }%
  \else
    \def\EmFi@next{%
      \kvsetkeys{EmFi}%
    }%
  \fi
  \EmFi@next
}
%    \end{macrocode}
%    \end{macro}
%
%    \begin{macro}{\EmFi@schema}
%    \begin{macrocode}
\def\EmFi@schema{}
%    \end{macrocode}
%    \end{macro}
%    \begin{macro}{\EmFi@order}
%    \begin{macrocode}
\gdef\EmFi@order{0}
%    \end{macrocode}
%    \end{macro}
%    \begin{macro}{\EmFi@@order}
%    \begin{macrocode}
\let\EmFi@@order\relax
%    \end{macrocode}
%    \end{macro}
%    \begin{macro}{\EmFi@fieldlist}
%    \begin{macrocode}
\def\EmFi@fieldlist{}
%    \end{macrocode}
%    \end{macro}
%    \begin{macro}{\EmFi@sortcase}
%    \begin{macrocode}
\def\EmFi@sortcase{0}%
%    \end{macrocode}
%    \end{macro}
%    \begin{macro}{\embedfilefield}
%    \begin{macrocode}
\def\embedfilefield#1#2{%
  \ifEmFi@finished
    \EmFi@Error{%
      \string\embedfilefield\ltx@space after \string\embedfilefinish
    }{%
      The list of embedded files is already written.%
    }%
  \else
    \global\EmFi@collectiontrue
    \EdefSanitize\EmFi@key{#1}%
    \expandafter\ifx\csname KV@EmFi@\EmFi@key.prefix\endcsname\relax
      \begingroup
        \count@=\EmFi@order
        \advance\count@ 1 %
        \xdef\EmFi@order{\the\count@}%
        \let\EmFi@title\EmFi@key
        \let\EmFi@type\EmFi@S@text
        \EmFi@visibletrue
        \EmFi@editfalse
        \kvsetkeys{EmFiFi}{#2}%
        \EmFi@convert\EmFi@title\EmFi@title
        \xdef\EmFi@schema{%
          \EmFi@schema
          /\pdf@escapename{\EmFi@key}<<%
            /Subtype/%
            \ifx\EmFi@type\EmFi@S@date D%
            \else\ifx\EmFi@type\EmFi@S@number N%
            \else\ifx\EmFi@type\EmFi@S@file F%
            \else\ifx\EmFi@type\EmFi@S@desc Desc%
            \else\ifx\EmFi@type\EmFi@S@moddate ModDate%
            \else\ifx\EmFi@type\EmFi@S@creationdate CreationDate%
            \else\ifx\EmFi@type\EmFi@S@size Size%
            \else S%
            \fi\fi\fi\fi\fi\fi\fi
            /N(\EmFi@title)%
            \EmFi@@order{\EmFi@order}%
            \ifEmFi@visible
            \else
              /V false%
            \fi
            \ifEmFi@edit
              /E true%
            \fi
          >>%
        }%
        \let\do\relax
        \xdef\EmFi@fieldlist{%
          \EmFi@fieldlist
          \do{\EmFi@key}%
        }%
        \ifx\EmFi@type\EmFi@S@text
          \kv@define@key{EmFi}{\EmFi@key.value}{%
            \EmFi@itemtrue
            \def\EmFi@temp{##1}%
            \EmFi@convert\EmFi@temp\EmFi@temp
            \expandafter\def\csname EmFi@V@#1%
            \expandafter\endcsname\expandafter{%
              \expandafter(\EmFi@temp)%
            }%
          }%
          \EmFi@GlobalKey{EmFi}{\EmFi@key.value}%
        \else\ifx\EmFi@type\EmFi@S@date
          \kv@define@key{EmFi}{\EmFi@key.value}{%
            \EmFi@itemtrue
            \def\EmFi@temp{##1}%
            \EmFi@convert\EmFi@temp\EmFi@temp
            \expandafter\def\csname EmFi@V@#1%
            \expandafter\endcsname\expandafter{%
              \expandafter(\EmFi@temp)%
            }%
          }%
          \EmFi@GlobalKey{EmFi}{\EmFi@key.value}%
        \else\ifx\EmFi@type\EmFi@S@number
          \kv@define@key{EmFi}{\EmFi@key.value}{%
            \EmFi@itemtrue
            \expandafter\EdefSanitize\csname EmFi@V@#1\endcsname{ ##1}%
          }%
          \EmFi@GlobalKey{EmFi}{\EmFi@key.value}%
        \fi\fi\fi
        \kv@define@key{EmFi}{\EmFi@key.prefix}{%
          \EmFi@itemtrue
          \expandafter\def\csname EmFi@P@#1\endcsname{##1}%
        }%
        \EmFi@GlobalKey{EmFi}{\EmFi@key.prefix}%
        \kv@define@key{EmFiSo}{\EmFi@key}[ascending]{%
          \EdefSanitize\EmFi@temp{##1}%
          \ifx\EmFi@temp\EmFi@S@ascending
            \def\EmFi@temp{true}%
          \else\ifx\EmFi@temp\EmFi@S@descending
            \def\EmFi@temp{false}%
          \else
            \def\EmFi@temp{}%
            \EmFi@Error{%
              Unknown sort order `\EmFi@temp'.\MessageBreak
              Supported values: `\EmFi@S@ascending', %
              `\EmFi@S@descending
            }\@ehc
          \fi\fi
          \ifx\EmFi@temp\ltx@empty
          \else
            \xdef\EmFi@sortkeys{%
              \EmFi@sortkeys
              /\pdf@escapename{#1}%
            }%
            \ifx\EmFi@sortorders\ltx@empty
              \global\let\EmFi@sortorders\EmFi@temp
              \gdef\EmFi@sortcase{1}%
            \else
              \xdef\EmFi@sortorders{%
                \EmFi@sortorders
                \ltx@space
                \EmFi@temp
              }%
              \xdef\EmFi@sortcase{2}%
            \fi
          \fi
        }%
        \EmFi@GlobalDefaultKey{EmFiSo}\EmFi@key
      \endgroup
    \else
      \EmFi@Error{%
        Field `\EmFi@key' is already defined%
      }\@ehc
    \fi
  \fi
}
%    \end{macrocode}
%    \end{macro}
%    \begin{macrocode}
\kv@define@key{EmFiFi}{type}{%
  \EdefSanitize\EmFi@temp{#1}%
  \ifx\EmFi@temp\EmFi@S@text
    \let\EmFi@type\EmFi@temp
  \else\ifx\EmFi@temp\EmFi@S@date
    \let\EmFi@type\EmFi@temp
  \else\ifx\EmFi@temp\EmFi@S@number
    \let\EmFi@type\EmFi@temp
  \else\ifx\EmFi@temp\EmFi@S@file
    \let\EmFi@type\EmFi@temp
  \else\ifx\EmFi@temp\EmFi@S@desc
    \let\EmFi@type\EmFi@temp
  \else\ifx\EmFi@temp\EmFi@S@moddate
    \let\EmFi@type\EmFi@temp
  \else\ifx\EmFi@temp\EmFi@S@creationdate
    \let\EmFi@type\EmFi@temp
  \else\ifx\EmFi@temp\EmFi@S@size
    \let\EmFi@type\EmFi@temp
  \else
    \EmFi@Error{%
      Unknown type `\EmFi@temp'.\MessageBreak
      Supported types: `text', `date', `number', `file',\MessageBreak
      `desc', `moddate', `creationdate', `size'%
    }%
  \fi\fi\fi\fi\fi\fi\fi\fi
}
%    \end{macrocode}
%    \begin{macrocode}
\kv@define@key{EmFiFi}{title}{%
  \def\EmFi@title{#1}%
}
%    \end{macrocode}
%    \begin{macro}{\EmFi@setboolean}
%    \begin{macrocode}
\def\EmFi@setboolean#1#2{%
  \EdefSanitize\EmFi@temp{#2}%
  \ifx\EmFi@temp\EmFi@S@true
    \csname EmFi@#1true\endcsname
  \else
    \ifx\EmFi@temp\EmFi@S@false
      \csname EmFi@#1false\endcsname
    \else
      \EmFi@Error{%
        Unknown value `\EmFi@temp' for key `#1'.\MessageBreak
        Supported values: `true', `false'%
      }\@ehc
    \fi
  \fi
}
%    \end{macrocode}
%    \end{macro}
%    \begin{macrocode}
\kv@define@key{EmFiFi}{visible}[true]{%
  \EmFi@setboolean{visible}{#1}%
}
%    \end{macrocode}
%    \begin{macrocode}
\kv@define@key{EmFiFi}{edit}[true]{%
  \EmFi@setboolean{edit}{#1}%
}
%    \end{macrocode}
%
%    \begin{macro}{\EmFi@sortkeys}
%    \begin{macrocode}
\def\EmFi@sortkeys{}
%    \end{macrocode}
%    \end{macro}
%    \begin{macro}{\EmFi@sortorders}
%    \begin{macrocode}
\def\EmFi@sortorders{}
%    \end{macrocode}
%    \end{macro}
%    \begin{macro}{\embedfilesort}
%    \begin{macrocode}
\def\embedfilesort{%
  \kvsetkeys{EmFiSo}%
}
%    \end{macrocode}
%    \end{macro}
%
% \subsection{Embed the file}
%
%    \begin{macro}{\embedfile}
%    \begin{macrocode}
\def\embedfile{%
  \ltx@ifnextchar[\EmFi@embedfile{\EmFi@embedfile[]}%
}
%    \end{macrocode}
%    \end{macro}
%    \begin{macro}{\EmFi@embedfile}
%    \begin{macrocode}
\def\EmFi@embedfile[#1]#2{%
  \ifEmFi@finished
    \EmFi@Error{%
      \string\embedfile\ltx@space after \string\embedfilefinish
    }{%
      The list of embedded files is already written.%
    }%
  \else
    \begingroup
      \def\EmFi@file{#2}%
      \kvsetkeys{EmFi}{#1}%
      \expandafter\expandafter\expandafter
      \ifx\expandafter\expandafter\expandafter
          \\\pdf@filesize{\EmFi@file}\\%
        \EmFi@Error{%
          File `\EmFi@file' not found%
        }{%
          The unknown file is not embedded.%
        }%
      \else
        \edef\EmFi@@filespec{%
          \pdf@escapestring{\EmFi@filespec}%
        }%
        \ifx\EmFi@ucfilespec\ltx@empty
          \let\EmFi@@ucfilespec\ltx@empty
        \else
          \EmFi@convert\EmFi@ucfilespec\EmFi@@ucfilespec
        \fi
        \ifx\EmFi@desc\ltx@empty
          \let\EmFi@@desc\ltx@empty
        \else
          \EmFi@convert\EmFi@desc\EmFi@@desc
        \fi
        \ifEmFi@item
          \let\do\EmFi@do
          \immediate\pdfobj{%
            <<%
              \EmFi@fieldlist
            >>%
          }%
          \edef\EmFi@ci{\the\pdflastobj}%
        \fi
        \immediate\pdfobj stream attr{%
          /Type/EmbeddedFile%
          \ifx\EmFi@mimetype\ltx@empty
          \else
            /Subtype/\pdf@escapename{\EmFi@mimetype}%
          \fi
          /Params<<%
            /ModDate(\pdf@filemoddate{\EmFi@file})%
            /Size \pdf@filesize{\EmFi@file}%
            /CheckSum<\pdf@filemdfivesum{\EmFi@file}>%
          >>%
        }file{\EmFi@file}\relax
        \EmFi@defobj{EmbeddedFile}%
        \immediate\pdfobj{%
          <<%
            /Type/Filespec%
            \ifx\EmFi@filesystem\ltx@empty
            \else
            /FS/\pdf@escapename{\EmFi@filesystem}%
            \fi
            /F(\EmFi@@filespec)%
            \ifx\EmFi@@ucfilespec\ltx@empty
            \else
              /UF(\EmFi@@ucfilespec)%
            \fi
            \ifx\EmFi@@desc\ltx@empty
            \else
              /Desc(\EmFi@@desc)%
            \fi
            /EF<<%
              /F \the\pdflastobj\ltx@space 0 R%
            >>%
            \ifEmFi@item
              /CI \EmFi@ci\ltx@space 0 R%
            \fi
          >>%
        }%
        \EmFi@defobj{Filespec}%
        \EmFi@add{%
          \EmFi@@filespec
        }{\the\pdflastobj\ltx@space 0 R}%
      \fi
    \endgroup
  \fi
}
%    \end{macrocode}
%    \end{macro}
%    \begin{macro}{\EmFi@do}
%    \begin{macrocode}
\def\EmFi@do#1{%
  \expandafter\ifx\csname EmFi@P@#1\endcsname\relax
    \expandafter\ifx\csname EmFi@V@#1\endcsname\relax
    \else
      /\pdf@escapename{#1}\csname EmFi@V@#1\endcsname
    \fi
  \else
    /\pdf@escapename{#1}<<%
      \expandafter\ifx\csname EmFi@V@#1\endcsname\relax
      \else
        /D\csname EmFi@V@#1\endcsname
      \fi
      /P(\csname EmFi@P@#1\endcsname)%
    >>%
  \fi
}
%    \end{macrocode}
%    \end{macro}
%    \begin{macro}{\EmFi@convert}
%    \begin{macrocode}
\def\EmFi@convert#1#2{%
  \ifnum\pdf@strcmp{\EmFi@stringmethod}{psd}=0 %
    \pdfstringdef\EmFi@temp{#1}%
    \let#2\EmFi@temp
  \else
    \edef#2{\pdf@escapestring{#1}}%
  \fi
}
%    \end{macrocode}
%    \end{macro}
%    \begin{macrocode}
\global\let\EmFi@list\ltx@empty
%    \end{macrocode}
%    \begin{macro}{\EmFi@add}
%    Sorting is done by the insertion sort algorithm.
%    Probably the sorting could be done more reliable.
%    However, the PDF specification is not too clear to me
%    regarding precise sorting rules (how to deal with
%    different encodings, escaped characters, \dots).
%    \begin{macrocode}
\def\EmFi@add#1#2{%
  \begingroup
    \ifx\EmFi@list\ltx@empty
      \xdef\EmFi@list{\noexpand\do{#1}{#2}}%
    \else
      \def\do##1##2{%
        \ifnum\pdf@strcmp{##1}{#1}>0 %
          \edef\x{%
            \toks@{%
              \the\toks@%
              \noexpand\do{#1}{#2}%
              \noexpand\do{##1}{##2}%
            }%
          }%
          \x
          \def\do####1####2{%
            \toks@\expandafter{\the\toks@\do{####1}{####2}}%
          }%
          \def\stop{%
            \xdef\EmFi@list{\the\toks@}%
          }%
        \else
          \toks@\expandafter{\the\toks@\do{##1}{##2}}%
        \fi
      }%
      \def\stop{%
        \xdef\EmFi@list{\the\toks@\noexpand\do{#1}{#2}}%
      }%
      \toks@{}%
      \EmFi@list\stop
    \fi
  \endgroup
}
%    \end{macrocode}
%    \end{macro}
%    \begin{macro}{\embedfilefinish}
%    \begin{macrocode}
\def\embedfilefinish{%
  \ifEmFi@finished
    \EmFi@Error{%
      Too many invocations of \string\embedfilefinish
    }{%
      The list of embedded files is already written.%
    }%
  \else
    \ifx\EmFi@list\ltx@empty
    \else
%    \end{macrocode}
%    Write |/EmbeddedFiles| entry.
%    \begin{macrocode}
      \global\EmFi@finishedtrue
      \begingroup
        \def\do##1##2{%
          (##1)##2%
        }%
        \immediate\pdfobj{%
          <<%
            /Names[\EmFi@list]%
          >>%
        }%
        \pdfnames{%
          /EmbeddedFiles \the\pdflastobj\ltx@space 0 R%
        }%
      \endgroup
%    \end{macrocode}
%    Write collection objects.
%    \begin{macrocode}
      \ifx\EmFi@initialfile\ltx@empty
      \else
        \EmFi@collectiontrue
      \fi
      \ifEmFi@collection
        \ifx\EmFi@initialfile\ltx@empty
          \let\EmFi@@initialfile\ltx@empty
        \else
          \edef\EmFi@@initialfile{%
            \pdf@escapestring{\EmFi@initialfile}%
          }%
        \fi
%    \end{macrocode}
%    Look for initial file among the embedded files.
%    \begin{macrocode}
        \begingroup
          \let\f=N%
          \def\do##1##2{%
            \def\x{##1}%
            \ifx\x\EmFi@@initialfile
              \let\f=Y%
              \let\do\ltx@gobbletwo
            \fi
          }%
          \EmFi@list
        \expandafter\endgroup
        \ifx\f Y%
        \else
          \@PackageWarningNoLine{embedfile}{%
            Missing initial file `\EmFi@initialfile'\MessageBreak
            among the embedded files%
          }%
          \let\EmFi@initialfile\ltx@empty
          \let\EmFi@@initialfile\ltx@empty
        \fi
        \ifcase\EmFi@sortcase
          \def\EmFi@temp{}%
        \or
          \def\EmFi@temp{%
            /S\EmFi@sortkeys
            /A \EmFi@sortorders
          }%
        \else
          \def\EmFi@temp{%
            /S[\EmFi@sortkeys]%
            /A[\EmFi@sortorders]%
          }%
        \fi
        \def\EmFi@@order##1{%
          \ifnum\EmFi@order>1 %
            /O ##1%
          \fi
        }%
        \immediate\pdfobj{%
          <<%
            \ifx\EmFi@schema\ltx@empty
            \else
              /Schema<<\EmFi@schema>>%
            \fi
            \ifx\EmFi@@initialfile\ltx@empty
            \else
              /D(\EmFi@@initialfile)%
            \fi
            \ifx\EmFi@view\EmFi@S@tile
              /View/T%
            \else\ifx\EmFi@view\EmFi@S@hidden
              /View/H%
            \fi\fi
            \ifx\EmFi@temp\ltx@empty
              \EmFi@temp
            \else
              /Sort<<\EmFi@temp>>%
            \fi
          >>%
        }%
        \pdfcatalog{%
          /Collection \the\pdflastobj\ltx@space0 R%
        }%
      \fi
    \fi
  \fi
}
%    \end{macrocode}
%    \end{macro}
%    \begin{macrocode}
\begingroup\expandafter\expandafter\expandafter\endgroup
\expandafter\ifx\csname AtEndDocument\endcsname\relax
\else
  \AtEndDocument{\embedfilefinish}%
\fi
%    \end{macrocode}
%
%    \begin{macrocode}
\EmFi@AtEnd%
%</package>
%    \end{macrocode}
%
% \section{Test}
%
% \subsection{Catcode checks for loading}
%
%    \begin{macrocode}
%<*test1>
%    \end{macrocode}
%    \begin{macrocode}
\catcode`\{=1 %
\catcode`\}=2 %
\catcode`\#=6 %
\catcode`\@=11 %
\expandafter\ifx\csname count@\endcsname\relax
  \countdef\count@=255 %
\fi
\expandafter\ifx\csname @gobble\endcsname\relax
  \long\def\@gobble#1{}%
\fi
\expandafter\ifx\csname @firstofone\endcsname\relax
  \long\def\@firstofone#1{#1}%
\fi
\expandafter\ifx\csname loop\endcsname\relax
  \expandafter\@firstofone
\else
  \expandafter\@gobble
\fi
{%
  \def\loop#1\repeat{%
    \def\body{#1}%
    \iterate
  }%
  \def\iterate{%
    \body
      \let\next\iterate
    \else
      \let\next\relax
    \fi
    \next
  }%
  \let\repeat=\fi
}%
\def\RestoreCatcodes{}
\count@=0 %
\loop
  \edef\RestoreCatcodes{%
    \RestoreCatcodes
    \catcode\the\count@=\the\catcode\count@\relax
  }%
\ifnum\count@<255 %
  \advance\count@ 1 %
\repeat

\def\RangeCatcodeInvalid#1#2{%
  \count@=#1\relax
  \loop
    \catcode\count@=15 %
  \ifnum\count@<#2\relax
    \advance\count@ 1 %
  \repeat
}
\def\RangeCatcodeCheck#1#2#3{%
  \count@=#1\relax
  \loop
    \ifnum#3=\catcode\count@
    \else
      \errmessage{%
        Character \the\count@\space
        with wrong catcode \the\catcode\count@\space
        instead of \number#3%
      }%
    \fi
  \ifnum\count@<#2\relax
    \advance\count@ 1 %
  \repeat
}
\def\space{ }
\expandafter\ifx\csname LoadCommand\endcsname\relax
  \def\LoadCommand{\input embedfile.sty\relax}%
\fi
\def\Test{%
  \RangeCatcodeInvalid{0}{47}%
  \RangeCatcodeInvalid{58}{64}%
  \RangeCatcodeInvalid{91}{96}%
  \RangeCatcodeInvalid{123}{255}%
  \catcode`\@=12 %
  \catcode`\\=0 %
  \catcode`\%=14 %
  \LoadCommand
  \RangeCatcodeCheck{0}{36}{15}%
  \RangeCatcodeCheck{37}{37}{14}%
  \RangeCatcodeCheck{38}{47}{15}%
  \RangeCatcodeCheck{48}{57}{12}%
  \RangeCatcodeCheck{58}{63}{15}%
  \RangeCatcodeCheck{64}{64}{12}%
  \RangeCatcodeCheck{65}{90}{11}%
  \RangeCatcodeCheck{91}{91}{15}%
  \RangeCatcodeCheck{92}{92}{0}%
  \RangeCatcodeCheck{93}{96}{15}%
  \RangeCatcodeCheck{97}{122}{11}%
  \RangeCatcodeCheck{123}{255}{15}%
  \RestoreCatcodes
}
\Test
\csname @@end\endcsname
\end
%    \end{macrocode}
%    \begin{macrocode}
%</test1>
%    \end{macrocode}
%
% \subsection{Simple test}
%
%    \begin{macrocode}
%<*test2>
\input embedfile.sty\relax
\embedfile[%
  stringmethod=escape,%
  mimetype=plain/text,%
  desc={LaTeX docstrip source archive for package `embedfile'},%
  id={embedfile.dtx}%
]{embedfile.dtx}
\nopagenumbers
Test (plain-\TeX): {\tt embedfile.dtx} should be embedded.%

\def\Test#1{%
  \par
  \embedfileifobjectexists{embedfile.dtx}{#1}{%
    Object #1 (embedfile.dtx): %
    \embedfilegetobject{embedfile.dtx}{#1}%
  }{%
    \errmessage{Missing object #1 (embedfile.dtx)}%
  }%
}
\Test{EmbeddedFile}
\Test{Filespec}
\embedfilefinish
\bye
%</test2>
%    \end{macrocode}
%    \begin{macrocode}
%<*test3>
\NeedsTeXFormat{LaTeX2e}
\let\SavedJobname\jobname
\def\jobname{embedfile}
\RequirePackage{dtx-attach}[2011/04/13]
\let\jobname\SavedJobname
\documentclass{minimal}
\begin{document}
  Test (\LaTeX): \texttt{embedfile.dtx} should be embedded.%
\end{document}
%</test3>
%    \end{macrocode}
%
% \subsection{Test for \iniTeX}
%
%    \begin{macrocode}
%<*test4>
\catcode`\{=1 %
\catcode`\}=2 %
\input ifluatex.sty %
\ifluatex
  \directlua{%
    tex.enableprimitives('', {%
      'pdflastobj',%
      'pdfnames',%
      'pdfobj',%
      'pdfoutput'%
    })%
  }%
\fi
\pdfoutput=1 %
\input embedfile.sty %
\shipout\hbox{}
\embedfile[%
  stringmethod=escape,%
  mimetype=plain/text,%
  desc={iniTeX source},%
]{\jobname.tex}
\embedfilefinish
\end
%</test4>
%    \end{macrocode}
%
% \section{Installation}
%
% \subsection{Download}
%
% \paragraph{Package.} This package is available on
% CTAN\footnote{\url{ftp://ftp.ctan.org/tex-archive/}}:
% \begin{description}
% \item[\CTAN{macros/latex/contrib/oberdiek/embedfile.dtx}] The source file.
% \item[\CTAN{macros/latex/contrib/oberdiek/embedfile.pdf}] Documentation.
% \end{description}
%
%
% \paragraph{Bundle.} All the packages of the bundle `oberdiek'
% are also available in a TDS compliant ZIP archive. There
% the packages are already unpacked and the documentation files
% are generated. The files and directories obey the TDS standard.
% \begin{description}
% \item[\CTAN{install/macros/latex/contrib/oberdiek.tds.zip}]
% \end{description}
% \emph{TDS} refers to the standard ``A Directory Structure
% for \TeX\ Files'' (\CTAN{tds/tds.pdf}). Directories
% with \xfile{texmf} in their name are usually organized this way.
%
% \subsection{Bundle installation}
%
% \paragraph{Unpacking.} Unpack the \xfile{oberdiek.tds.zip} in the
% TDS tree (also known as \xfile{texmf} tree) of your choice.
% Example (linux):
% \begin{quote}
%   |unzip oberdiek.tds.zip -d ~/texmf|
% \end{quote}
%
% \paragraph{Script installation.}
% Check the directory \xfile{TDS:scripts/oberdiek/} for
% scripts that need further installation steps.
% Package \xpackage{attachfile2} comes with the Perl script
% \xfile{pdfatfi.pl} that should be installed in such a way
% that it can be called as \texttt{pdfatfi}.
% Example (linux):
% \begin{quote}
%   |chmod +x scripts/oberdiek/pdfatfi.pl|\\
%   |cp scripts/oberdiek/pdfatfi.pl /usr/local/bin/|
% \end{quote}
%
% \subsection{Package installation}
%
% \paragraph{Unpacking.} The \xfile{.dtx} file is a self-extracting
% \docstrip\ archive. The files are extracted by running the
% \xfile{.dtx} through \plainTeX:
% \begin{quote}
%   \verb|tex embedfile.dtx|
% \end{quote}
%
% \paragraph{TDS.} Now the different files must be moved into
% the different directories in your installation TDS tree
% (also known as \xfile{texmf} tree):
% \begin{quote}
% \def\t{^^A
% \begin{tabular}{@{}>{\ttfamily}l@{ $\rightarrow$ }>{\ttfamily}l@{}}
%   embedfile.sty & tex/generic/oberdiek/embedfile.sty\\
%   dtx-attach.sty & tex/generic/oberdiek/dtx-attach.sty\\
%   embedfile.pdf & doc/latex/oberdiek/embedfile.pdf\\
%   embedfile-example-plain.tex & doc/latex/oberdiek/embedfile-example-plain.tex\\
%   embedfile-example-collection.tex & doc/latex/oberdiek/embedfile-example-collection.tex\\
%   test/embedfile-test1.tex & doc/latex/oberdiek/test/embedfile-test1.tex\\
%   test/embedfile-test2.tex & doc/latex/oberdiek/test/embedfile-test2.tex\\
%   test/embedfile-test3.tex & doc/latex/oberdiek/test/embedfile-test3.tex\\
%   test/embedfile-test4.tex & doc/latex/oberdiek/test/embedfile-test4.tex\\
%   embedfile.dtx & source/latex/oberdiek/embedfile.dtx\\
% \end{tabular}^^A
% }^^A
% \sbox0{\t}^^A
% \ifdim\wd0>\linewidth
%   \begingroup
%     \advance\linewidth by\leftmargin
%     \advance\linewidth by\rightmargin
%   \edef\x{\endgroup
%     \def\noexpand\lw{\the\linewidth}^^A
%   }\x
%   \def\lwbox{^^A
%     \leavevmode
%     \hbox to \linewidth{^^A
%       \kern-\leftmargin\relax
%       \hss
%       \usebox0
%       \hss
%       \kern-\rightmargin\relax
%     }^^A
%   }^^A
%   \ifdim\wd0>\lw
%     \sbox0{\small\t}^^A
%     \ifdim\wd0>\linewidth
%       \ifdim\wd0>\lw
%         \sbox0{\footnotesize\t}^^A
%         \ifdim\wd0>\linewidth
%           \ifdim\wd0>\lw
%             \sbox0{\scriptsize\t}^^A
%             \ifdim\wd0>\linewidth
%               \ifdim\wd0>\lw
%                 \sbox0{\tiny\t}^^A
%                 \ifdim\wd0>\linewidth
%                   \lwbox
%                 \else
%                   \usebox0
%                 \fi
%               \else
%                 \lwbox
%               \fi
%             \else
%               \usebox0
%             \fi
%           \else
%             \lwbox
%           \fi
%         \else
%           \usebox0
%         \fi
%       \else
%         \lwbox
%       \fi
%     \else
%       \usebox0
%     \fi
%   \else
%     \lwbox
%   \fi
% \else
%   \usebox0
% \fi
% \end{quote}
% If you have a \xfile{docstrip.cfg} that configures and enables \docstrip's
% TDS installing feature, then some files can already be in the right
% place, see the documentation of \docstrip.
%
% \subsection{Refresh file name databases}
%
% If your \TeX~distribution
% (\teTeX, \mikTeX, \dots) relies on file name databases, you must refresh
% these. For example, \teTeX\ users run \verb|texhash| or
% \verb|mktexlsr|.
%
% \subsection{Some details for the interested}
%
% \paragraph{Attached source.}
%
% The PDF documentation on CTAN also includes the
% \xfile{.dtx} source file. It can be extracted by
% AcrobatReader 6 or higher. Another option is \textsf{pdftk},
% e.g. unpack the file into the current directory:
% \begin{quote}
%   \verb|pdftk embedfile.pdf unpack_files output .|
% \end{quote}
%
% \paragraph{Unpacking with \LaTeX.}
% The \xfile{.dtx} chooses its action depending on the format:
% \begin{description}
% \item[\plainTeX:] Run \docstrip\ and extract the files.
% \item[\LaTeX:] Generate the documentation.
% \end{description}
% If you insist on using \LaTeX\ for \docstrip\ (really,
% \docstrip\ does not need \LaTeX), then inform the autodetect routine
% about your intention:
% \begin{quote}
%   \verb|latex \let\install=y\input{embedfile.dtx}|
% \end{quote}
% Do not forget to quote the argument according to the demands
% of your shell.
%
% \paragraph{Generating the documentation.}
% You can use both the \xfile{.dtx} or the \xfile{.drv} to generate
% the documentation. The process can be configured by the
% configuration file \xfile{ltxdoc.cfg}. For instance, put this
% line into this file, if you want to have A4 as paper format:
% \begin{quote}
%   \verb|\PassOptionsToClass{a4paper}{article}|
% \end{quote}
% An example follows how to generate the
% documentation with pdf\LaTeX:
% \begin{quote}
%\begin{verbatim}
%pdflatex embedfile.dtx
%makeindex -s gind.ist embedfile.idx
%pdflatex embedfile.dtx
%makeindex -s gind.ist embedfile.idx
%pdflatex embedfile.dtx
%\end{verbatim}
% \end{quote}
%
% \section{Catalogue}
%
% The following XML file can be used as source for the
% \href{http://mirror.ctan.org/help/Catalogue/catalogue.html}{\TeX\ Catalogue}.
% The elements \texttt{caption} and \texttt{description} are imported
% from the original XML file from the Catalogue.
% The name of the XML file in the Catalogue is \xfile{embedfile.xml}.
%    \begin{macrocode}
%<*catalogue>
<?xml version='1.0' encoding='us-ascii'?>
<!DOCTYPE entry SYSTEM 'catalogue.dtd'>
<entry datestamp='$Date$' modifier='$Author$' id='embedfile'>
  <name>embedfile</name>
  <caption>Embed files into PDF.</caption>
  <authorref id='auth:oberdiek'/>
  <copyright owner='Heiko Oberdiek' year='2006-2011'/>
  <license type='lppl1.3'/>
  <version number='2.6'/>
  <description>
    This package embeds files in a PDF document, using the PDF
    format's embedding operation (note the contrast with the attach
    operation used by the <xref refid='attachfile'>attachfile</xref>
    and <xref refid='attachfile2'>attachfile2</xref> packages).
    Currently only <xref refid='pdftex'>pdfTeX</xref> &gt;=1.30, in
    PDF mode, is supported.

    <p/>
    The package is part of the <xref refid='oberdiek'>oberdiek</xref>
    bundle.
  </description>
  <documentation details='Package documentation'
      href='ctan:/macros/latex/contrib/oberdiek/embedfile.pdf'/>
  <ctan file='true' path='/macros/latex/contrib/oberdiek/embedfile.dtx'/>
  <miktex location='oberdiek'/>
  <texlive location='oberdiek'/>
  <install path='/macros/latex/contrib/oberdiek/oberdiek.tds.zip'/>
</entry>
%</catalogue>
%    \end{macrocode}
%
% \begin{thebibliography}{9}
% \bibitem{attachfile}
%   Scott Pakin: \textit{The \xpackage{attachfile} package};
%   2005/02/20 v1.2;
%   \CTAN{macros/latex/contrib/attachfile/}.
%
% \bibitem{attachfile2}
%   Heiko Oberdiek: \textit{The \xpackage{attachfile2} package};
%   2006/08/16 v2.2;
%   \CTAN{macros/latex/contrib/oberdiek/attachfile2.pdf}.
%
% \bibitem{pdfspec}
%   Adobe Systems Incorporated:
%   \href{http://www.adobe.com/devnet/acrobat/pdfs/pdf_reference.pdf}%
%       {\textit{PDF Reference, Sixth Edition, Version 1.7}},%
%   Oktober 2006;
%   \url{http://www.adobe.com/devnet/pdf/pdf_reference.html}.
%
% \bibitem{mime}
%   Network Working Group: RFC 2046, \textit{Multipurpose Internet Mail Extensions (MIME) Part Two:
%     Media Types}, November 1996; \url{http://www.rfc-editor.org/}.
%
% \bibitem{types}
%   IANA (Internet Assigned Numbers Authority):
%   \textit{MIME Media Types}, May 2006;
%   \url{http://www.iana.org/assignments/media-types/}.
%
% \end{thebibliography}
%
% \begin{History}
%   \begin{Version}{2006/08/16 v1.0}
%   \item
%     First public version.
%   \end{Version}
%   \begin{Version}{2007/04/11 v1.1}
%   \item
%     Line ends sanitized.
%   \end{Version}
%   \begin{Version}{2007/09/09 v1.2}
%   \item
%     Fixes for plain-TeX, wrapper for package \xpackage{keyval} added.
%   \item
%     Catcode section rewritten.
%   \end{Version}
%   \begin{Version}{2007/10/28 v2.0}
%   \item
%     Collection support added (PDF 1.7).
%   \end{Version}
%   \begin{Version}{2007/10/29 v2.1}
%   \item
%     Export of object references by adding new option \xoption{id} and
%     new macros \cs{embedfileifobjectexists} and \cs{embedfilegetobject}.
%   \end{Version}
%   \begin{Version}{2007/11/11 v2.2}
%   \item
%     Use of package \xpackage{pdftexcmds} for \LuaTeX\ support.
%   \end{Version}
%   \begin{Version}{2007/11/25 v2.3}
%   \item
%     Fix in use of \cs{pdf@filesize}, bug introduced in previous version.
%   \end{Version}
%   \begin{Version}{2009/09/25 v2.4}
%   \item
%     Bug fix: If \xpackage{hyperref} is used with option \xoption{unicode},
%     the Unicode encoded file name causes trouble. Therefore \cs{pdfstringdef}
%     is now never used for option \xoption{filespec}, always method
%     \texttt{escape} is applied (Peter Cibulka).
%   \item
%     Bug fix for \xoption{initialfile}.
%   \item
%     Bug fix for file names in \texttt{/EmbeddedFiles}.
%   \item
%     New option \xoption{ucfilespec} for file name support in Unicode
%     (since PDF 1.7).
%   \end{Version}
%   \begin{Version}{2010/03/01 v2.5}
%   \item
%     Compatibility for \iniTeX.
%   \item
%     Package \xpackage{keyval} replaced by packages \xpackage{kvsetkeys}
%     and \xpackage{kvdefinekeys} because of compatibility for \iniTeX.
%   \item
%     TDS location moved from TDS:tex/latex/oberdiek/embedfile.sty to
%     TDS:tex/generic/oberdiek/embedfile.sty.
%   \end{Version}
%   \begin{Version}{2011/04/13 v2.6}
%   \item
%     Docu fixes (thanks Hans-Martin M\"unch).
%   \end{Version}
% \end{History}
%
% \PrintIndex
%
% \Finale
\endinput
|
% \end{quote}
% Do not forget to quote the argument according to the demands
% of your shell.
%
% \paragraph{Generating the documentation.}
% You can use both the \xfile{.dtx} or the \xfile{.drv} to generate
% the documentation. The process can be configured by the
% configuration file \xfile{ltxdoc.cfg}. For instance, put this
% line into this file, if you want to have A4 as paper format:
% \begin{quote}
%   \verb|\PassOptionsToClass{a4paper}{article}|
% \end{quote}
% An example follows how to generate the
% documentation with pdf\LaTeX:
% \begin{quote}
%\begin{verbatim}
%pdflatex embedfile.dtx
%makeindex -s gind.ist embedfile.idx
%pdflatex embedfile.dtx
%makeindex -s gind.ist embedfile.idx
%pdflatex embedfile.dtx
%\end{verbatim}
% \end{quote}
%
% \section{Catalogue}
%
% The following XML file can be used as source for the
% \href{http://mirror.ctan.org/help/Catalogue/catalogue.html}{\TeX\ Catalogue}.
% The elements \texttt{caption} and \texttt{description} are imported
% from the original XML file from the Catalogue.
% The name of the XML file in the Catalogue is \xfile{embedfile.xml}.
%    \begin{macrocode}
%<*catalogue>
<?xml version='1.0' encoding='us-ascii'?>
<!DOCTYPE entry SYSTEM 'catalogue.dtd'>
<entry datestamp='$Date$' modifier='$Author$' id='embedfile'>
  <name>embedfile</name>
  <caption>Embed files into PDF.</caption>
  <authorref id='auth:oberdiek'/>
  <copyright owner='Heiko Oberdiek' year='2006-2011'/>
  <license type='lppl1.3'/>
  <version number='2.6'/>
  <description>
    This package embeds files in a PDF document, using the PDF
    format's embedding operation (note the contrast with the attach
    operation used by the <xref refid='attachfile'>attachfile</xref>
    and <xref refid='attachfile2'>attachfile2</xref> packages).
    Currently only <xref refid='pdftex'>pdfTeX</xref> &gt;=1.30, in
    PDF mode, is supported.

    <p/>
    The package is part of the <xref refid='oberdiek'>oberdiek</xref>
    bundle.
  </description>
  <documentation details='Package documentation'
      href='ctan:/macros/latex/contrib/oberdiek/embedfile.pdf'/>
  <ctan file='true' path='/macros/latex/contrib/oberdiek/embedfile.dtx'/>
  <miktex location='oberdiek'/>
  <texlive location='oberdiek'/>
  <install path='/macros/latex/contrib/oberdiek/oberdiek.tds.zip'/>
</entry>
%</catalogue>
%    \end{macrocode}
%
% \begin{thebibliography}{9}
% \bibitem{attachfile}
%   Scott Pakin: \textit{The \xpackage{attachfile} package};
%   2005/02/20 v1.2;
%   \CTAN{macros/latex/contrib/attachfile/}.
%
% \bibitem{attachfile2}
%   Heiko Oberdiek: \textit{The \xpackage{attachfile2} package};
%   2006/08/16 v2.2;
%   \CTAN{macros/latex/contrib/oberdiek/attachfile2.pdf}.
%
% \bibitem{pdfspec}
%   Adobe Systems Incorporated:
%   \href{http://www.adobe.com/devnet/acrobat/pdfs/pdf_reference.pdf}%
%       {\textit{PDF Reference, Sixth Edition, Version 1.7}},%
%   Oktober 2006;
%   \url{http://www.adobe.com/devnet/pdf/pdf_reference.html}.
%
% \bibitem{mime}
%   Network Working Group: RFC 2046, \textit{Multipurpose Internet Mail Extensions (MIME) Part Two:
%     Media Types}, November 1996; \url{http://www.rfc-editor.org/}.
%
% \bibitem{types}
%   IANA (Internet Assigned Numbers Authority):
%   \textit{MIME Media Types}, May 2006;
%   \url{http://www.iana.org/assignments/media-types/}.
%
% \end{thebibliography}
%
% \begin{History}
%   \begin{Version}{2006/08/16 v1.0}
%   \item
%     First public version.
%   \end{Version}
%   \begin{Version}{2007/04/11 v1.1}
%   \item
%     Line ends sanitized.
%   \end{Version}
%   \begin{Version}{2007/09/09 v1.2}
%   \item
%     Fixes for plain-TeX, wrapper for package \xpackage{keyval} added.
%   \item
%     Catcode section rewritten.
%   \end{Version}
%   \begin{Version}{2007/10/28 v2.0}
%   \item
%     Collection support added (PDF 1.7).
%   \end{Version}
%   \begin{Version}{2007/10/29 v2.1}
%   \item
%     Export of object references by adding new option \xoption{id} and
%     new macros \cs{embedfileifobjectexists} and \cs{embedfilegetobject}.
%   \end{Version}
%   \begin{Version}{2007/11/11 v2.2}
%   \item
%     Use of package \xpackage{pdftexcmds} for \LuaTeX\ support.
%   \end{Version}
%   \begin{Version}{2007/11/25 v2.3}
%   \item
%     Fix in use of \cs{pdf@filesize}, bug introduced in previous version.
%   \end{Version}
%   \begin{Version}{2009/09/25 v2.4}
%   \item
%     Bug fix: If \xpackage{hyperref} is used with option \xoption{unicode},
%     the Unicode encoded file name causes trouble. Therefore \cs{pdfstringdef}
%     is now never used for option \xoption{filespec}, always method
%     \texttt{escape} is applied (Peter Cibulka).
%   \item
%     Bug fix for \xoption{initialfile}.
%   \item
%     Bug fix for file names in \texttt{/EmbeddedFiles}.
%   \item
%     New option \xoption{ucfilespec} for file name support in Unicode
%     (since PDF 1.7).
%   \end{Version}
%   \begin{Version}{2010/03/01 v2.5}
%   \item
%     Compatibility for \iniTeX.
%   \item
%     Package \xpackage{keyval} replaced by packages \xpackage{kvsetkeys}
%     and \xpackage{kvdefinekeys} because of compatibility for \iniTeX.
%   \item
%     TDS location moved from TDS:tex/latex/oberdiek/embedfile.sty to
%     TDS:tex/generic/oberdiek/embedfile.sty.
%   \end{Version}
%   \begin{Version}{2011/04/13 v2.6}
%   \item
%     Docu fixes (thanks Hans-Martin M\"unch).
%   \end{Version}
% \end{History}
%
% \PrintIndex
%
% \Finale
\endinput

%        (quote the arguments according to the demands of your shell)
%
% Documentation:
%    (a) If embedfile.drv is present:
%           latex embedfile.drv
%    (b) Without embedfile.drv:
%           latex embedfile.dtx; ...
%    The class ltxdoc loads the configuration file ltxdoc.cfg
%    if available. Here you can specify further options, e.g.
%    use A4 as paper format:
%       \PassOptionsToClass{a4paper}{article}
%
%    Programm calls to get the documentation (example):
%       pdflatex embedfile.dtx
%       makeindex -s gind.ist embedfile.idx
%       pdflatex embedfile.dtx
%       makeindex -s gind.ist embedfile.idx
%       pdflatex embedfile.dtx
%
% Installation:
%    TDS:tex/generic/oberdiek/embedfile.sty
%    TDS:tex/generic/oberdiek/dtx-attach.sty
%    TDS:doc/latex/oberdiek/embedfile.pdf
%    TDS:doc/latex/oberdiek/embedfile-example-plain.tex
%    TDS:doc/latex/oberdiek/embedfile-example-collection.tex
%    TDS:doc/latex/oberdiek/test/embedfile-test1.tex
%    TDS:doc/latex/oberdiek/test/embedfile-test2.tex
%    TDS:doc/latex/oberdiek/test/embedfile-test3.tex
%    TDS:doc/latex/oberdiek/test/embedfile-test4.tex
%    TDS:source/latex/oberdiek/embedfile.dtx
%
%<*ignore>
\begingroup
  \catcode123=1 %
  \catcode125=2 %
  \def\x{LaTeX2e}%
\expandafter\endgroup
\ifcase 0\ifx\install y1\fi\expandafter
         \ifx\csname processbatchFile\endcsname\relax\else1\fi
         \ifx\fmtname\x\else 1\fi\relax
\else\csname fi\endcsname
%</ignore>
%<*install>
\input docstrip.tex
\Msg{************************************************************************}
\Msg{* Installation}
\Msg{* Package: embedfile 2011/04/13 v2.6 Embed files into PDF (HO)}
\Msg{************************************************************************}

\keepsilent
\askforoverwritefalse

\let\MetaPrefix\relax
\preamble

This is a generated file.

Project: embedfile
Version: 2011/04/13 v2.6

Copyright (C) 2006-2011 by
   Heiko Oberdiek <heiko.oberdiek at googlemail.com>

This work may be distributed and/or modified under the
conditions of the LaTeX Project Public License, either
version 1.3c of this license or (at your option) any later
version. This version of this license is in
   http://www.latex-project.org/lppl/lppl-1-3c.txt
and the latest version of this license is in
   http://www.latex-project.org/lppl.txt
and version 1.3 or later is part of all distributions of
LaTeX version 2005/12/01 or later.

This work has the LPPL maintenance status "maintained".

This Current Maintainer of this work is Heiko Oberdiek.

The Base Interpreter refers to any `TeX-Format',
because some files are installed in TDS:tex/generic//.

This work consists of the main source file embedfile.dtx
and the derived files
   embedfile.sty, embedfile.pdf, embedfile.ins, embedfile.drv,
   dtx-attach.sty, embedfile-example-plain.tex,
   embedfile-example-collection.tex, embedfile-test1.tex,
   embedfile-test2.tex, embedfile-test3.tex,
   embedfile-test4.tex.

\endpreamble
\let\MetaPrefix\DoubleperCent

\generate{%
  \file{embedfile.ins}{\from{embedfile.dtx}{install}}%
  \file{embedfile.drv}{\from{embedfile.dtx}{driver}}%
  \usedir{tex/generic/oberdiek}%
  \file{embedfile.sty}{\from{embedfile.dtx}{package}}%
  \usedir{tex/latex/oberdiek}%
  \file{dtx-attach.sty}{\from{embedfile.dtx}{dtxattach}}%
  \usedir{doc/latex/oberdiek}%
  \file{embedfile-example-plain.tex}{\from{embedfile.dtx}{exampleplain}}%
  \file{embedfile-example-collection.tex}{\from{embedfile.dtx}{examplecollection}}%
  \usedir{doc/latex/oberdiek/test}%
  \file{embedfile-test1.tex}{\from{embedfile.dtx}{test1}}%
  \file{embedfile-test2.tex}{\from{embedfile.dtx}{test2}}%
  \file{embedfile-test3.tex}{\from{embedfile.dtx}{test3}}%
  \file{embedfile-test4.tex}{\from{embedfile.dtx}{test4}}%
  \nopreamble
  \nopostamble
  \usedir{source/latex/oberdiek/catalogue}%
  \file{embedfile.xml}{\from{embedfile.dtx}{catalogue}}%
}

\catcode32=13\relax% active space
\let =\space%
\Msg{************************************************************************}
\Msg{*}
\Msg{* To finish the installation you have to move the following}
\Msg{* files into a directory searched by TeX:}
\Msg{*}
\Msg{*     embedfile.sty, dtx-attach.sty}
\Msg{*}
\Msg{* To produce the documentation run the file `embedfile.drv'}
\Msg{* through LaTeX.}
\Msg{*}
\Msg{* Happy TeXing!}
\Msg{*}
\Msg{************************************************************************}

\endbatchfile
%</install>
%<*ignore>
\fi
%</ignore>
%<*driver>
\NeedsTeXFormat{LaTeX2e}
\ProvidesFile{embedfile.drv}%
  [2011/04/13 v2.6 Embed files into PDF (HO)]%
\documentclass{ltxdoc}
\usepackage{holtxdoc}[2011/11/22]
\begin{document}
  \DocInput{embedfile.dtx}%
\end{document}
%</driver>
% \fi
%
% \CheckSum{1363}
%
% \CharacterTable
%  {Upper-case    \A\B\C\D\E\F\G\H\I\J\K\L\M\N\O\P\Q\R\S\T\U\V\W\X\Y\Z
%   Lower-case    \a\b\c\d\e\f\g\h\i\j\k\l\m\n\o\p\q\r\s\t\u\v\w\x\y\z
%   Digits        \0\1\2\3\4\5\6\7\8\9
%   Exclamation   \!     Double quote  \"     Hash (number) \#
%   Dollar        \$     Percent       \%     Ampersand     \&
%   Acute accent  \'     Left paren    \(     Right paren   \)
%   Asterisk      \*     Plus          \+     Comma         \,
%   Minus         \-     Point         \.     Solidus       \/
%   Colon         \:     Semicolon     \;     Less than     \<
%   Equals        \=     Greater than  \>     Question mark \?
%   Commercial at \@     Left bracket  \[     Backslash     \\
%   Right bracket \]     Circumflex    \^     Underscore    \_
%   Grave accent  \`     Left brace    \{     Vertical bar  \|
%   Right brace   \}     Tilde         \~}
%
% \GetFileInfo{embedfile.drv}
%
% \title{The \xpackage{embedfile} package}
% \date{2011/04/13 v2.6}
% \author{Heiko Oberdiek\\\xemail{heiko.oberdiek at googlemail.com}}
%
% \maketitle
%
% \begin{abstract}
% This package embeds files to a PDF document.
% Currently the only supported driver is \pdfTeX\ $>=$ 1.30 in PDF mode.
% \end{abstract}
%
% \tableofcontents
%
% \section{Documentation}
%
% \subsection{Introduction}
%
%    The PDF format (\cite{pdfspec}) allows the inclusion of files
%    inside the PDF document. The included files can be bound to an
%    annotation on a page. Or they can be recorded in a sorted
%    list of embedded files. The packages \xpackage{attachfile}
%    or \xpackage{attachfile2} follow the first approach, this
%    package uses the latter method.
%
% \subsubsection{Future development}
%
%    My dream is a large package that merges the features of
%    all these packages meantioned before:
%    \begin{itemize}
%    \item Files can be attached to a page.
%    \item Files can be attached to the document.
%    \item An easy user interface for simple, common tasks and
%          beginners.
%    \item An interface for the advanced users that want to setup
%          every detail.
%    \item Support of many drivers (pdftex, dvips, dvipdfm, \dots).
%    \item \dots
%    \end{itemize}
%    However, I have not managed to take the time for this project.
%    Instead:
%    \begin{itemize}
%    \item First I experimented with package \xpackage{attachfile},
%          adding driver support, fixing bugs, \dots. The result is
%          currently named as \xpackage{attachfile2}. It uses an external
%          script to get file properties (size, date, checksum, \dots).
%    \item In order to avoid an external program for getting basic
%          file properties I provided a patch ``EscapeAndOther'' for
%          pdfTeX that was accepted for version 1.30.
%    \item This package closes a gap left by the packages for attaching
%          files and allows the embedding of files to the document.
%          Also it makes use of the new primitives of \pdfTeX.
%    \end{itemize}
%
% \subsection{User interface}
%
%    This package \xpackage{embedfile} can be used with both
%    \LaTeX\ and \plainTeX. See \autoref{sec:plain} that
%    explains the use with \plainTeX\ by an example.
%    In \LaTeX\ the package is loaded as usually. There are no options.
%    \begin{quote}
%      \verb|\usepackage{embedfile}|
%    \end{quote}
%
%    \begin{declcs}{embedfile}
%      |[|\meta{options}|]| \M{file}
%    \end{declcs}
%    The macro \cs{embedfile} includes file \meta{file} and
%    attaches it to the PDF document. At the end of the document
%    the sorted list of embedded files are written. Thus you can
%    safely use \cs{embedfile} before \verb|\end{document}|.
%    Embedding files using \cs{AtEndDocument} will only work,
%    if \cs{AtEndDocument} is called before loading the package
%    \xpackage{embedfile}.
%
%    The \meta{options} are give as key value pairs. The following
%    keys are supported:
%    \begin{description}
%    \item[\xoption{filespec}]
%      This allows to override the file name
%      that appears in the PDF file. If you are using other than
%      simple file names (8-bit, path separators, \dots), look into
%      the PDF specification (\cite{pdfspec}).
%      There are rules how these file names must be written/encoded.
%      Avoid 8-bit characters
%      and other special characters, the behaviour is currently undefined.
%      Use option \xoption{ucfilespec} for more funny file names.
%      The string method, see below, is |escape| since version 2.4.
%
%      This name is also used as entry in a name tree
%      (see PDF specification: |/EmbeddedFiles|). Therefore the
%      value for \xoption{filespec} must be unique among all
%      embedded files. Also key \xoption{initialfiles} refers
%      to this name, if the file name and the value of \xoption{filespec}
%      are different.
%    \item[\xoption{ucfilespec}]
%      Since PDF 1.7 the file name may be provided in Unicode.
%      The conversion of the option value into a PDF string
%      is controlled by option \xoption{stringmethod}.
%    \item[\xoption{filesystem}]
%      This sets the entry \verb|/FS| in
%      the file specification dictionary, see PDF specification
%      (\cite{pdfspec}).
%      Example: \verb|filesystem=URL|.
%    \item[\xoption{mimetype}]
%      This sets the mime type (\cite{mime}) of the file,
%      see \autoref{sec:plain} for examples and \cite{types}
%      for a list of officially registered types.
%    \item[\xoption{desc}]
%      The description for the file.
%    \item[\xoption{stringmethod}]
%      The package must convert the values of the keys \xoption{ucfilespec}
%      and \xoption{desc} into a PDF string (before version 2.4: \xoption{filespec}
%      and \xoption{desc}). If \xpackage{hyperref}
%      is found, then its \cs{pdfstringdef} will be used, otherwise
%      \pdfTeX's \cs{pdfescapestring} is used. Value \verb|psd|
%      forces the use of \cs{pdfstringdef}, value \verb|escape|
%      the use of \cs{pdfescapestring}.
%    \item[\xoption{\meta{key}.value}] Sets the value of a collection
%      item property, see section \ref{sec:collection}.
%    \item[\xoption{\meta{key}.prefix}]
%      Sets the prefix of a collection item property,
%      see section \ref{sec:collection}.
%    \item[\xoption{id}]
%      The value must be an unique name. Macros
%      \cs{embedfileifobjectexists} and \cs{embedfilegetobject}
%      are using this name later.
%    \end{description}
%
%    \begin{declcs}{embedfilefinish}
%    \end{declcs}
%    The list of all embedded files must be added as data structure
%    in the PDF file. In case of \LaTeX\ this is automatically done.
%    The package uses \cs{AtEndDocument}. Then the list of all
%    files should be known. However, \plainTeX\ does not know
%    about \cs{AtEndDocument}. Thus the user must call
%    \cs{embedfilefinish} at the end of the document
%    after the last file is embedded.
%
%    \begin{declcs}{embedfilesetup} \M{options}
%    \end{declcs}
%    Options for \cs{embedfile} and collection support
%    can be set in \cs{embedfilesetup}.
%
% \subsection{Collection support (PDF 1.7)}
% \label{sec:collection}
%
%    Since PDF 1.7 the embedded files can form a \emph{collection}
%    (sometimes referred as \emph{package}), the main document
%    is called \emph{cover sheet}. See PDF specification
%    8.2.4 ``Collections'' and 3.10.5 ``Collection items'' \cite{pdfspec}.
%
%    Usually Acrobat Reader 7 or 8 shows the
%    embedded files in a table at the bottom with the following columns:
%    \begin{quote}
%    \begin{tabular}{@{}l|l|l|l@{}}
%       Name & Description & Modified & Size\\
%       \hline
%       \dots & \dots & \dots & \dots\\
%    \end{tabular}
%    \end{quote}
%    Acrobat Reader 10 shows the embeded files in the left panel
%    and adds a new column for the compressed size.
%
%    If the files form a collection, then they are displayed
%    in a table left or top (depending on option \xoption{view},
%    see \cs{embedfilesetup}).
%
%
%    Collection support is enabled automatically, if
%    it is used.
%
%    \begin{declcs}{embedfilesetup} \M{options}
%    \end{declcs}
%    The following options are supported in addition
%    to options for \cs{embedfile}:
%    \begin{description}
%    \item[\xoption{view}]
%      If the PDF file contains a collection, then Acrobat Reader 8
%      shows a line at the top below the menu bar and the toolbar.
%      It shows the current selected file, icons for changing the
%      view mode, an options menu.
%      The initial mode how the collection is presented is
%      set by this option \xoption{view}.
%      The following modes/values are supported, the default is |details|:
%        \begin{description}
%        \item[\texttt{details}] The full collection table is displayed
%          at the top below the collection bar.
%        \item[\texttt{tile}] The files of the collection are shown
%          in tile mode on the left.
%        \item[\texttt{hidden}] The collection table is not shown.
%        \end{description}
%    \item[\xoption{initialfile}]
%      Selects the file that is initially presented. Especially
%      useful for an embedded PDF file that is then shown instead
%      of the cover document. There must be an \cs{embedfile} command
%      somewhere whose value for key \xoption{filespec} is used here.
%      The \cs{embedfile} command can drop option \xoption{filespec}
%      if the file name is not different.
%    \end{description}
%
%    \begin{declcs}{embedfilefield} \M{key} \M{options}
%    \end{declcs}
%    Macro \cs{embedfilefield} defines a column/field in the collection table.
%    The name of the field is \meta{key}.
%    \begin{description}
%    \item[\xoption{type}]
%      sets the type of the field. The supported values are:
%      \begin{description}
%      \item[\texttt{text}] A text field. Its value is set in
%        \cs{embedfile} by option \xoption{\meta{key}.value}.
%      \item[\texttt{date}] A date field. Its value is set in
%        \cs{embedfile} by option \xoption{\meta{key}.value}.
%        A special format is required, see ``3.8.3 Dates'' \cite{pdfspec}.
%      \item[\texttt{number}] A field with an integer or float number.
%        Its value is set in \cs{embedfile} by option
%        \xoption{\meta{key}.value}.
%      \item[\texttt{file}] The file name of the embedded file.
%      \item[\texttt{desc}] The description text of the embedded file. It is
%       set in \cs{embedfile} by option \xoption{desc}.
%      \item[\texttt{moddate}] The modification date of the embedded file.
%      \item[\texttt{size}] The size of the embedded file.
%      \end{description}
%      All types allow the use of a prefix that is disregarded by
%      sorting. The prefix for this field is set in \cs{embedfile} by
%      option \xoption{\meta{key}.prefix}.
%    \item[\xoption{title}] sets the column title.
%    \item[\xoption{visible}] controls whether the column is presented:
%      \begin{description}
%      \item[\texttt{true}] shows the column.
%      \item[\texttt{false}] hides the column.
%      \end{description}
%      Default: \texttt{true}
%    \item[\xoption{edit}]
%      Allows the editing of field values. Does not seem to have an
%      effect for Acrobat Reader.
%      \begin{description}
%      \item[\texttt{true}] enables the feature, if available (depends
%        on the PDF viewer).
%      \item[\texttt{false}] disables the feature.
%      \end{description}
%      Default: \texttt{false}
%    \end{description}
%    The order of \cs{embedfilefield} statements defines the order
%    of the columns.
%
%    \begin{declcs}{embedfilesort} \M{key-sort-list}
%    \end{declcs}
%    The sort order of the embedded files are controlled by
%    macro \cs{embedfilesort}. \meta{key-sort-list} defines
%    the sort order. The key is a field name defined by \cs{embedfilefield}.
%    Its value is either \texttt{ascending} or \texttt{descending}.
%    The default is \texttt{ascending}.
%
% \subsection{Export of object references}
%
%    Caution: This feature is still experimental. It may be even removed
%    in future versions. Therefore feedback would be nice, if someone
%    has a useful application for this feature.
%
%    Object numbers are saved, if \xoption{id} is given in \cs{embedfile}.
%    The following objects are supported:
%    \begin{itemize}
%    \item \texttt{EmbeddedFile}
%    \item \texttt{Filespec}
%    \end{itemize}
%
%    \begin{declcs}{embedfileifobjectexists} \M{id} \M{type}
%      \M{then} \M{else}
%    \end{declcs}
%    Macro \cs{embedfileifobjectexists} tests whether object of \meta{type}
%    is available for the embedded file identified by \meta{id}.
%
%    \begin{declcs}{embedfilegetobject} \M{id} \M{type}
%    \end{declcs}
%    Macro \cs{embedfilegetobject} expands to the full object reference
%    object of \meta{type} for the embedded file identified by \meta{id}.
%
% \subsubsection{Example}
%
%\begin{quote}
%\begin{verbatim}
%\embedfile[id={foo}]{foo.pdf}
%\embedfileifobjectexists{foo}{Filespec}{%
%  \typeout{%
%    FileSpec object for `foo': %
%    \embedfilegetobject{foo}{Filespec}%
%  }%
%}{%
%  \typeout{No Filespec object for `foo'}%
%}
%\end{verbatim}
%\end{quote}
%
% \subsection{Examples}
% \label{sec:examples}
%
% \subsubsection{\plainTeX}
% \label{sec:plain}
%
%    The package can be used with \plainTeX. It can be used
%    with or without help from \xfile{miniltx.tex}.
%
%    If additionally package \xpackage{keyval} (\xpackage{graphicx})
%    is needed, load it first. Then package \xpackage{embedfile} avoids
%    a duplicate loading of package \xpackage{keyval}.
%
%    Because \plainTeX\ does not provide a hook at end of the document,
%    you have to call \cs{embedfilefinish} manually at the end
%    after the last embedded file.
%    \begin{macrocode}
%<*exampleplain>
%<<END
% Load packages
\input miniltx
% \def\Gin@driver{pdftex.def}
% \input graphicx.sty
\input embedfile.sty
\resetatcatcode

% default setting
\embedfilesetup{
  mimetype=text/plain
}

% Embed files
\embedfile[
  filespec=example.tex,
  desc={Source code (plain-TeX) of this example}
]{embedfile-example-plain.tex}

\embedfile[
  desc={Source of package `embedfile'}
]{embedfile.dtx}

\embedfile[
  mimetype=application/pdf,
  desc={Documentation of package `embedfile'}
]{embedfile.pdf}

% Some text
This example document contains three embedded files.

% End of document
\embedfilefinish % don't forget
\bye
%END
%</exampleplain>
%    \end{macrocode}
%
% \subsubsection{Collection example}
%
%    \begin{macrocode}
%<*examplecollection>
%<<END
\NeedsTeXFormat{LaTeX2e}
\documentclass{article}
\usepackage[bookmarks=false]{hyperref}
% provides \pdfstringdef that is then used by `title' and
% other keys.
\usepackage{embedfile}[2011/04/13]
\embedfilesetup{
  view=details,
  initialfile=embedfile.pdf
}
\embedfilefield{file}{
  type=file,
  title={File name}
}
\embedfilefield{description}{
  type=desc,
  title={Description}
}
\embedfilefield{date}{
  type=moddate,
  title={Date}
}
\embedfilefield{size}{
  type=size,
  title={Size}
}
\embedfilefield{type}{
  type=text,
  title={Type},
  visible=false
}
\embedfilesort{
  type,
  date=descending
}
\begin{document}
An example for embedded files as collection.
You need Acrobat Reader 8 or higher.

\embedfile[
  desc={Source file of package `embedfile'},
  description.prefix={Package: },
  type.value={DTX}
]{embedfile.dtx}

\embedfile[
  desc={Documentation of package `embedfile'},
  description.prefix={Package: },
  type.value={PDF}
]{embedfile.pdf}

\embedfile[
  desc={The source for this example},
  description.prefix={Example: },
  type.value={TEX}
]{\jobname.tex}

\end{document}
%END
%</examplecollection>
%    \end{macrocode}
%
% \subsection{Package \xpackage{dtx-attach}}
%
% Package \xpackage{dtx-attach} is just a small application of
% package \xpackage{embedfile}. I am using it for the CTAN
% documentation of my packages in \CTAN{macros/latex/contrib/oberdiek/}.
% It also serves as small example for the use of the package with \LaTeX.
%
%    \begin{macrocode}
%<*dtxattach>
\NeedsTeXFormat{LaTeX2e}
\ProvidesPackage{dtx-attach}
  [2011/04/13 v2.6 Embed \string\jobname.dtx (HO)]%
\RequirePackage{embedfile}[2011/04/13]
\embedfile[%
  stringmethod=escape,%
  mimetype=plain/text,%
  desc={LaTeX docstrip source archive for package `\jobname'}%
]{\jobname.dtx}
%</dtxattach>
%    \end{macrocode}
%
% \StopEventually{
% }
%
% \section{Implementation}
%
%    \begin{macrocode}
%<*package>
%    \end{macrocode}
%
% \subsection{Reload check and package identification}
%    Reload check, especially if the package is not used with \LaTeX.
%    \begin{macrocode}
\begingroup\catcode61\catcode48\catcode32=10\relax%
  \catcode13=5 % ^^M
  \endlinechar=13 %
  \catcode35=6 % #
  \catcode39=12 % '
  \catcode44=12 % ,
  \catcode45=12 % -
  \catcode46=12 % .
  \catcode58=12 % :
  \catcode64=11 % @
  \catcode123=1 % {
  \catcode125=2 % }
  \expandafter\let\expandafter\x\csname ver@embedfile.sty\endcsname
  \ifx\x\relax % plain-TeX, first loading
  \else
    \def\empty{}%
    \ifx\x\empty % LaTeX, first loading,
      % variable is initialized, but \ProvidesPackage not yet seen
    \else
      \expandafter\ifx\csname PackageInfo\endcsname\relax
        \def\x#1#2{%
          \immediate\write-1{Package #1 Info: #2.}%
        }%
      \else
        \def\x#1#2{\PackageInfo{#1}{#2, stopped}}%
      \fi
      \x{embedfile}{The package is already loaded}%
      \aftergroup\endinput
    \fi
  \fi
\endgroup%
%    \end{macrocode}
%    Package identification:
%    \begin{macrocode}
\begingroup\catcode61\catcode48\catcode32=10\relax%
  \catcode13=5 % ^^M
  \endlinechar=13 %
  \catcode35=6 % #
  \catcode39=12 % '
  \catcode40=12 % (
  \catcode41=12 % )
  \catcode44=12 % ,
  \catcode45=12 % -
  \catcode46=12 % .
  \catcode47=12 % /
  \catcode58=12 % :
  \catcode64=11 % @
  \catcode91=12 % [
  \catcode93=12 % ]
  \catcode123=1 % {
  \catcode125=2 % }
  \expandafter\ifx\csname ProvidesPackage\endcsname\relax
    \def\x#1#2#3[#4]{\endgroup
      \immediate\write-1{Package: #3 #4}%
      \xdef#1{#4}%
    }%
  \else
    \def\x#1#2[#3]{\endgroup
      #2[{#3}]%
      \ifx#1\@undefined
        \xdef#1{#3}%
      \fi
      \ifx#1\relax
        \xdef#1{#3}%
      \fi
    }%
  \fi
\expandafter\x\csname ver@embedfile.sty\endcsname
\ProvidesPackage{embedfile}%
  [2011/04/13 v2.6 Embed files into PDF (HO)]%
%    \end{macrocode}
%
% \subsection{Catcodes}
%
%    \begin{macrocode}
\begingroup\catcode61\catcode48\catcode32=10\relax%
  \catcode13=5 % ^^M
  \endlinechar=13 %
  \catcode123=1 % {
  \catcode125=2 % }
  \catcode64=11 % @
  \def\x{\endgroup
    \expandafter\edef\csname EmFi@AtEnd\endcsname{%
      \endlinechar=\the\endlinechar\relax
      \catcode13=\the\catcode13\relax
      \catcode32=\the\catcode32\relax
      \catcode35=\the\catcode35\relax
      \catcode61=\the\catcode61\relax
      \catcode64=\the\catcode64\relax
      \catcode123=\the\catcode123\relax
      \catcode125=\the\catcode125\relax
    }%
  }%
\x\catcode61\catcode48\catcode32=10\relax%
\catcode13=5 % ^^M
\endlinechar=13 %
\catcode35=6 % #
\catcode64=11 % @
\catcode123=1 % {
\catcode125=2 % }
\def\TMP@EnsureCode#1#2{%
  \edef\EmFi@AtEnd{%
    \EmFi@AtEnd
    \catcode#1=\the\catcode#1\relax
  }%
  \catcode#1=#2\relax
}
\TMP@EnsureCode{39}{12}% '
\TMP@EnsureCode{40}{12}% (
\TMP@EnsureCode{41}{12}% )
\TMP@EnsureCode{44}{12}% ,
\TMP@EnsureCode{46}{12}% .
\TMP@EnsureCode{47}{12}% /
\TMP@EnsureCode{58}{12}% :
\TMP@EnsureCode{60}{12}% <
\TMP@EnsureCode{62}{12}% >
\TMP@EnsureCode{91}{12}% [
\TMP@EnsureCode{93}{12}% ]
\TMP@EnsureCode{96}{12}% `
\edef\EmFi@AtEnd{\EmFi@AtEnd\noexpand\endinput}
%    \end{macrocode}
%
% \subsection{Tools}
%
%    \begin{macro}{\EmFi@RequirePackage}
%    \begin{macrocode}
\begingroup\expandafter\expandafter\expandafter\endgroup
\expandafter\ifx\csname RequirePackage\endcsname\relax
  \def\EmFi@RequirePackage#1[#2]{%
    \input #1.sty\relax
  }%
\else
  \let\EmFi@RequirePackage\RequirePackage
\fi
%    \end{macrocode}
%    \end{macro}
%
%    \begin{macro}{\EmFi@Error}
%    \begin{macrocode}
\EmFi@RequirePackage{infwarerr}[2007/09/09]%
\def\EmFi@Error{%
  \@PackageError{embedfile}%
}
%    \end{macrocode}
%    \end{macro}
%
% \subsection{Check for recent \pdfTeX\ in PDF mode}
%
%    Load package \xpackage{ifpdf} and check mode.
%    \begin{macrocode}
\EmFi@RequirePackage{ifpdf}[2007/09/09]
\ifpdf
\else
  \EmFi@Error{%
    Missing pdfTeX in PDF mode%
  }{%
    Currently other drivers are not supported. %
    Package loading is aborted.%
  }%
  \expandafter\EmFi@AtEnd
\fi%
%    \end{macrocode}
%    \begin{macrocode}
\EmFi@RequirePackage{pdftexcmds}[2007/11/11]
\EmFi@RequirePackage{ltxcmds}[2010/03/01]
\EmFi@RequirePackage{kvsetkeys}[2010/03/01]
\EmFi@RequirePackage{kvdefinekeys}[2010/03/01]
%    \end{macrocode}
%    Check version.
%    \begin{macrocode}
\begingroup\expandafter\expandafter\expandafter\endgroup
\expandafter\ifx\csname pdf@filesize\endcsname\relax
  \EmFi@Error{%
    Unsupported pdfTeX version%
  }{%
    At least version 1.30 is necessary. Package loading is aborted.%
  }%
  \expandafter\EmFi@AtEnd
\fi%
%    \end{macrocode}
%
% \subsection{Strings}
%
%    Minimal version of package \xpackage{pdfescape} is 2007/08/27 v1.5
%    because of \cs{EdefSanitize}.
%    \begin{macrocode}
\EmFi@RequirePackage{pdfescape}[2007/11/11]
%    \end{macrocode}
%
%    \begin{macrocode}
\def\EmFi@temp#1{%
  \expandafter\EdefSanitize\csname EmFi@S@#1\endcsname{#1}%
}
%    \end{macrocode}
%    \begin{macro}{\EmFi@details}
%    \begin{macrocode}
\EmFi@temp{details}%
%    \end{macrocode}
%    \end{macro}
%    \begin{macro}{\EmFi@tile}
%    \begin{macrocode}
\EmFi@temp{tile}%
%    \end{macrocode}
%    \end{macro}
%    \begin{macro}{\EmFi@hidden}
%    \begin{macrocode}
\EmFi@temp{hidden}%
%    \end{macrocode}
%    \end{macro}
%    \begin{macro}{\EmFi@S@text}
%    \begin{macrocode}
\EmFi@temp{text}
%    \end{macrocode}
%    \end{macro}
%    \begin{macro}{\EmFi@S@date}
%    \begin{macrocode}
\EmFi@temp{date}
%    \end{macrocode}
%    \end{macro}
%    \begin{macro}{\EmFi@S@number}
%    \begin{macrocode}
\EmFi@temp{number}
%    \end{macrocode}
%    \end{macro}
%    \begin{macro}{\EmFi@S@file}
%    \begin{macrocode}
\EmFi@temp{file}
%    \end{macrocode}
%    \end{macro}
%    \begin{macro}{\EmFi@S@desc}
%    \begin{macrocode}
\EmFi@temp{desc}
%    \end{macrocode}
%    \end{macro}
%    \begin{macro}{\EmFi@S@moddate}
%    \begin{macrocode}
\EmFi@temp{moddate}
%    \end{macrocode}
%    \end{macro}
%    \begin{macro}{\EmFi@S@creationdate}
%    \begin{macrocode}
\EmFi@temp{creationdate}
%    \end{macrocode}
%    \end{macro}
%    \begin{macro}{\EmFi@S@size}
%    \begin{macrocode}
\EmFi@temp{size}
%    \end{macrocode}
%    \end{macro}
%    \begin{macro}{\EmFi@S@ascending}
%    \begin{macrocode}
\EmFi@temp{ascending}
%    \end{macrocode}
%    \end{macro}
%    \begin{macro}{\EmFi@S@descending}
%    \begin{macrocode}
\EmFi@temp{descending}
%    \end{macrocode}
%    \end{macro}
%    \begin{macro}{\EmFi@S@true}
%    \begin{macrocode}
\EmFi@temp{true}
%    \end{macrocode}
%    \end{macro}
%    \begin{macro}{\EmFi@S@false}
%    \begin{macrocode}
\EmFi@temp{false}
%    \end{macrocode}
%    \end{macro}
%
% \subsection{Switches}
%
%    \begin{macro}{\ifEmFi@collection}
%    \begin{macrocode}
\ltx@newif\ifEmFi@collection
%    \end{macrocode}
%    \end{macro}
%    \begin{macro}{\ifEmFi@sort}
%    \begin{macrocode}
\ltx@newif\ifEmFi@sort
%    \end{macrocode}
%    \end{macro}
%    \begin{macro}{\ifEmFi@visible}
%    \begin{macrocode}
\ltx@newif\ifEmFi@visible
%    \end{macrocode}
%    \end{macro}
%    \begin{macro}{\ifEmFi@edit}
%    \begin{macrocode}
\ltx@newif\ifEmFi@edit
%    \end{macrocode}
%    \end{macro}
%    \begin{macro}{\ifEmFi@item}
%    \begin{macrocode}
\ltx@newif\ifEmFi@item
%    \end{macrocode}
%    \end{macro}
%    \begin{macro}{\ifEmFi@finished}
%    \begin{macrocode}
\ltx@newif\ifEmFi@finished
%    \end{macrocode}
%    \end{macro}
%    \begin{macro}{\ifEmFi@id}
%    \begin{macrocode}
\ltx@newif\ifEmFi@id
%    \end{macrocode}
%    \end{macro}
%
% \subsection{Key value definitions}
%
%    \begin{macro}{\EmFi@GlobalKey}
%    \begin{macrocode}
\def\EmFi@GlobalKey#1#2{%
  \global\expandafter\let\csname KV@#1@#2\expandafter\endcsname
                         \csname KV@#1@#2\endcsname
}
%    \end{macrocode}
%    \end{macro}
%    \begin{macro}{\EmFi@GlobalDefaultKey}
%    \begin{macrocode}
\def\EmFi@GlobalDefaultKey#1#2{%
  \EmFi@GlobalKey{#1}{#2}%
  \global\expandafter\let
      \csname KV@#1@#2@default\expandafter\endcsname
      \csname KV@#1@#2@default\endcsname
}
%    \end{macrocode}
%    \end{macro}
%
%    \begin{macro}{\EmFi@DefineKey}
%    \begin{macrocode}
\def\EmFi@DefineKey#1#2{%
  \kv@define@key{EmFi}{#1}{%
    \expandafter\def\csname EmFi@#1\endcsname{##1}%
  }%
  \expandafter\def\csname EmFi@#1\endcsname{#2}%
}
%    \end{macrocode}
%    \end{macro}
%    Subtype of the embedded file (optional).
%    \begin{macrocode}
\EmFi@DefineKey{mimetype}{}
%    \end{macrocode}
%    File specification string.
%    \begin{macrocode}
\EmFi@DefineKey{filespec}{\EmFi@file}
%    \end{macrocode}
%    File specification string in Unicode.
%    \begin{macrocode}
\EmFi@DefineKey{ucfilespec}{}
%    \end{macrocode}
%    File system (optional).
%    \begin{macrocode}
\EmFi@DefineKey{filesystem}{}
%    \end{macrocode}
%    Description (optional).
%    \begin{macrocode}
\EmFi@DefineKey{desc}{}
%    \end{macrocode}
%    Method for converting text to PDF strings.
%    \begin{macrocode}
\EmFi@DefineKey{stringmethod}{%
  \ifx\pdfstringdef\@undefined
    escape%
  \else
    \ifx\pdfstringdef\relax
      escape%
    \else
      psd%
    \fi
  \fi
}
%    \end{macrocode}
%
%    Option \xoption{id} as key for object numbers.
%    \begin{macrocode}
\kv@define@key{EmFi}{id}{%
  \def\EmFi@id{#1}%
  \EmFi@idtrue
}
%    \end{macrocode}
%    \begin{macro}{\EmFi@defobj}
%    \begin{macrocode}
\def\EmFi@defobj#1{%
  \ifEmFi@id
    \expandafter\xdef\csname EmFi@#1@\EmFi@id\endcsname{%
      \the\pdflastobj\ltx@space 0 R%
    }%
  \fi
}
%    \end{macrocode}
%    \end{macro}
%    \begin{macro}{\embedfileifobjectexists}
%    \begin{macrocode}
\def\embedfileifobjectexists#1#2{%
  \expandafter\ifx\csname EmFi@#2@#1\endcsname\relax
    \expandafter\ltx@secondoftwo
  \else
    \expandafter\ltx@firstoftwo
  \fi
}
%    \end{macrocode}
%    \end{macro}
%    \begin{macro}{\embedfilegetobject}
%    \begin{macrocode}
\def\embedfilegetobject#1#2{%
  \embedfileifobjectexists{#1}{#2}{%
    \csname EmFi@#2@#1\endcsname
  }{%
    0 0 R%
  }%
}
%    \end{macrocode}
%    \end{macro}
%
%    Initial view of the collection.
%    \begin{macrocode}
\kv@define@key{EmFi}{view}[]{%
  \EdefSanitize\EmFi@temp{#1}%
  \def\EmFi@next{%
    \global\EmFi@collectiontrue
  }%
  \ifx\EmFi@temp\ltx@empty
    \let\EmFi@view\EmFi@S@details
  \else\ifx\EmFi@temp\EmFi@S@details
    \let\EmFi@view\EmFi@S@details
  \else\ifx\EmFi@temp\EmFi@S@tile
    \let\EmFi@view\EmFi@S@tile
  \else\ifx\EmFi@temp\EmFi@S@hidden
    \let\EmFi@view\EmFi@S@hidden
  \else
    \let\EmFi@next\relax
    \EmFi@Error{%
      Unknown value `\EmFi@temp' for key `view'.\MessageBreak
      Supported values: `details', `tile', `hidden'.%
    }\@ehc
  \fi\fi\fi\fi
  \EmFi@next
}
%    \end{macrocode}
%    \begin{macrocode}
\EmFi@DefineKey{initialfile}{}
%    \end{macrocode}
%
%    \begin{macro}{\embedfilesetup}
%    \begin{macrocode}
\def\embedfilesetup{%
  \ifEmFi@finished
    \def\EmFi@next##1{}%
    \EmFi@Error{%
      \string\embedfilefield\ltx@space after \string\embedfilefinish
    }{%
      The list of embedded files is already written.%
    }%
  \else
    \def\EmFi@next{%
      \kvsetkeys{EmFi}%
    }%
  \fi
  \EmFi@next
}
%    \end{macrocode}
%    \end{macro}
%
%    \begin{macro}{\EmFi@schema}
%    \begin{macrocode}
\def\EmFi@schema{}
%    \end{macrocode}
%    \end{macro}
%    \begin{macro}{\EmFi@order}
%    \begin{macrocode}
\gdef\EmFi@order{0}
%    \end{macrocode}
%    \end{macro}
%    \begin{macro}{\EmFi@@order}
%    \begin{macrocode}
\let\EmFi@@order\relax
%    \end{macrocode}
%    \end{macro}
%    \begin{macro}{\EmFi@fieldlist}
%    \begin{macrocode}
\def\EmFi@fieldlist{}
%    \end{macrocode}
%    \end{macro}
%    \begin{macro}{\EmFi@sortcase}
%    \begin{macrocode}
\def\EmFi@sortcase{0}%
%    \end{macrocode}
%    \end{macro}
%    \begin{macro}{\embedfilefield}
%    \begin{macrocode}
\def\embedfilefield#1#2{%
  \ifEmFi@finished
    \EmFi@Error{%
      \string\embedfilefield\ltx@space after \string\embedfilefinish
    }{%
      The list of embedded files is already written.%
    }%
  \else
    \global\EmFi@collectiontrue
    \EdefSanitize\EmFi@key{#1}%
    \expandafter\ifx\csname KV@EmFi@\EmFi@key.prefix\endcsname\relax
      \begingroup
        \count@=\EmFi@order
        \advance\count@ 1 %
        \xdef\EmFi@order{\the\count@}%
        \let\EmFi@title\EmFi@key
        \let\EmFi@type\EmFi@S@text
        \EmFi@visibletrue
        \EmFi@editfalse
        \kvsetkeys{EmFiFi}{#2}%
        \EmFi@convert\EmFi@title\EmFi@title
        \xdef\EmFi@schema{%
          \EmFi@schema
          /\pdf@escapename{\EmFi@key}<<%
            /Subtype/%
            \ifx\EmFi@type\EmFi@S@date D%
            \else\ifx\EmFi@type\EmFi@S@number N%
            \else\ifx\EmFi@type\EmFi@S@file F%
            \else\ifx\EmFi@type\EmFi@S@desc Desc%
            \else\ifx\EmFi@type\EmFi@S@moddate ModDate%
            \else\ifx\EmFi@type\EmFi@S@creationdate CreationDate%
            \else\ifx\EmFi@type\EmFi@S@size Size%
            \else S%
            \fi\fi\fi\fi\fi\fi\fi
            /N(\EmFi@title)%
            \EmFi@@order{\EmFi@order}%
            \ifEmFi@visible
            \else
              /V false%
            \fi
            \ifEmFi@edit
              /E true%
            \fi
          >>%
        }%
        \let\do\relax
        \xdef\EmFi@fieldlist{%
          \EmFi@fieldlist
          \do{\EmFi@key}%
        }%
        \ifx\EmFi@type\EmFi@S@text
          \kv@define@key{EmFi}{\EmFi@key.value}{%
            \EmFi@itemtrue
            \def\EmFi@temp{##1}%
            \EmFi@convert\EmFi@temp\EmFi@temp
            \expandafter\def\csname EmFi@V@#1%
            \expandafter\endcsname\expandafter{%
              \expandafter(\EmFi@temp)%
            }%
          }%
          \EmFi@GlobalKey{EmFi}{\EmFi@key.value}%
        \else\ifx\EmFi@type\EmFi@S@date
          \kv@define@key{EmFi}{\EmFi@key.value}{%
            \EmFi@itemtrue
            \def\EmFi@temp{##1}%
            \EmFi@convert\EmFi@temp\EmFi@temp
            \expandafter\def\csname EmFi@V@#1%
            \expandafter\endcsname\expandafter{%
              \expandafter(\EmFi@temp)%
            }%
          }%
          \EmFi@GlobalKey{EmFi}{\EmFi@key.value}%
        \else\ifx\EmFi@type\EmFi@S@number
          \kv@define@key{EmFi}{\EmFi@key.value}{%
            \EmFi@itemtrue
            \expandafter\EdefSanitize\csname EmFi@V@#1\endcsname{ ##1}%
          }%
          \EmFi@GlobalKey{EmFi}{\EmFi@key.value}%
        \fi\fi\fi
        \kv@define@key{EmFi}{\EmFi@key.prefix}{%
          \EmFi@itemtrue
          \expandafter\def\csname EmFi@P@#1\endcsname{##1}%
        }%
        \EmFi@GlobalKey{EmFi}{\EmFi@key.prefix}%
        \kv@define@key{EmFiSo}{\EmFi@key}[ascending]{%
          \EdefSanitize\EmFi@temp{##1}%
          \ifx\EmFi@temp\EmFi@S@ascending
            \def\EmFi@temp{true}%
          \else\ifx\EmFi@temp\EmFi@S@descending
            \def\EmFi@temp{false}%
          \else
            \def\EmFi@temp{}%
            \EmFi@Error{%
              Unknown sort order `\EmFi@temp'.\MessageBreak
              Supported values: `\EmFi@S@ascending', %
              `\EmFi@S@descending
            }\@ehc
          \fi\fi
          \ifx\EmFi@temp\ltx@empty
          \else
            \xdef\EmFi@sortkeys{%
              \EmFi@sortkeys
              /\pdf@escapename{#1}%
            }%
            \ifx\EmFi@sortorders\ltx@empty
              \global\let\EmFi@sortorders\EmFi@temp
              \gdef\EmFi@sortcase{1}%
            \else
              \xdef\EmFi@sortorders{%
                \EmFi@sortorders
                \ltx@space
                \EmFi@temp
              }%
              \xdef\EmFi@sortcase{2}%
            \fi
          \fi
        }%
        \EmFi@GlobalDefaultKey{EmFiSo}\EmFi@key
      \endgroup
    \else
      \EmFi@Error{%
        Field `\EmFi@key' is already defined%
      }\@ehc
    \fi
  \fi
}
%    \end{macrocode}
%    \end{macro}
%    \begin{macrocode}
\kv@define@key{EmFiFi}{type}{%
  \EdefSanitize\EmFi@temp{#1}%
  \ifx\EmFi@temp\EmFi@S@text
    \let\EmFi@type\EmFi@temp
  \else\ifx\EmFi@temp\EmFi@S@date
    \let\EmFi@type\EmFi@temp
  \else\ifx\EmFi@temp\EmFi@S@number
    \let\EmFi@type\EmFi@temp
  \else\ifx\EmFi@temp\EmFi@S@file
    \let\EmFi@type\EmFi@temp
  \else\ifx\EmFi@temp\EmFi@S@desc
    \let\EmFi@type\EmFi@temp
  \else\ifx\EmFi@temp\EmFi@S@moddate
    \let\EmFi@type\EmFi@temp
  \else\ifx\EmFi@temp\EmFi@S@creationdate
    \let\EmFi@type\EmFi@temp
  \else\ifx\EmFi@temp\EmFi@S@size
    \let\EmFi@type\EmFi@temp
  \else
    \EmFi@Error{%
      Unknown type `\EmFi@temp'.\MessageBreak
      Supported types: `text', `date', `number', `file',\MessageBreak
      `desc', `moddate', `creationdate', `size'%
    }%
  \fi\fi\fi\fi\fi\fi\fi\fi
}
%    \end{macrocode}
%    \begin{macrocode}
\kv@define@key{EmFiFi}{title}{%
  \def\EmFi@title{#1}%
}
%    \end{macrocode}
%    \begin{macro}{\EmFi@setboolean}
%    \begin{macrocode}
\def\EmFi@setboolean#1#2{%
  \EdefSanitize\EmFi@temp{#2}%
  \ifx\EmFi@temp\EmFi@S@true
    \csname EmFi@#1true\endcsname
  \else
    \ifx\EmFi@temp\EmFi@S@false
      \csname EmFi@#1false\endcsname
    \else
      \EmFi@Error{%
        Unknown value `\EmFi@temp' for key `#1'.\MessageBreak
        Supported values: `true', `false'%
      }\@ehc
    \fi
  \fi
}
%    \end{macrocode}
%    \end{macro}
%    \begin{macrocode}
\kv@define@key{EmFiFi}{visible}[true]{%
  \EmFi@setboolean{visible}{#1}%
}
%    \end{macrocode}
%    \begin{macrocode}
\kv@define@key{EmFiFi}{edit}[true]{%
  \EmFi@setboolean{edit}{#1}%
}
%    \end{macrocode}
%
%    \begin{macro}{\EmFi@sortkeys}
%    \begin{macrocode}
\def\EmFi@sortkeys{}
%    \end{macrocode}
%    \end{macro}
%    \begin{macro}{\EmFi@sortorders}
%    \begin{macrocode}
\def\EmFi@sortorders{}
%    \end{macrocode}
%    \end{macro}
%    \begin{macro}{\embedfilesort}
%    \begin{macrocode}
\def\embedfilesort{%
  \kvsetkeys{EmFiSo}%
}
%    \end{macrocode}
%    \end{macro}
%
% \subsection{Embed the file}
%
%    \begin{macro}{\embedfile}
%    \begin{macrocode}
\def\embedfile{%
  \ltx@ifnextchar[\EmFi@embedfile{\EmFi@embedfile[]}%
}
%    \end{macrocode}
%    \end{macro}
%    \begin{macro}{\EmFi@embedfile}
%    \begin{macrocode}
\def\EmFi@embedfile[#1]#2{%
  \ifEmFi@finished
    \EmFi@Error{%
      \string\embedfile\ltx@space after \string\embedfilefinish
    }{%
      The list of embedded files is already written.%
    }%
  \else
    \begingroup
      \def\EmFi@file{#2}%
      \kvsetkeys{EmFi}{#1}%
      \expandafter\expandafter\expandafter
      \ifx\expandafter\expandafter\expandafter
          \\\pdf@filesize{\EmFi@file}\\%
        \EmFi@Error{%
          File `\EmFi@file' not found%
        }{%
          The unknown file is not embedded.%
        }%
      \else
        \edef\EmFi@@filespec{%
          \pdf@escapestring{\EmFi@filespec}%
        }%
        \ifx\EmFi@ucfilespec\ltx@empty
          \let\EmFi@@ucfilespec\ltx@empty
        \else
          \EmFi@convert\EmFi@ucfilespec\EmFi@@ucfilespec
        \fi
        \ifx\EmFi@desc\ltx@empty
          \let\EmFi@@desc\ltx@empty
        \else
          \EmFi@convert\EmFi@desc\EmFi@@desc
        \fi
        \ifEmFi@item
          \let\do\EmFi@do
          \immediate\pdfobj{%
            <<%
              \EmFi@fieldlist
            >>%
          }%
          \edef\EmFi@ci{\the\pdflastobj}%
        \fi
        \immediate\pdfobj stream attr{%
          /Type/EmbeddedFile%
          \ifx\EmFi@mimetype\ltx@empty
          \else
            /Subtype/\pdf@escapename{\EmFi@mimetype}%
          \fi
          /Params<<%
            /ModDate(\pdf@filemoddate{\EmFi@file})%
            /Size \pdf@filesize{\EmFi@file}%
            /CheckSum<\pdf@filemdfivesum{\EmFi@file}>%
          >>%
        }file{\EmFi@file}\relax
        \EmFi@defobj{EmbeddedFile}%
        \immediate\pdfobj{%
          <<%
            /Type/Filespec%
            \ifx\EmFi@filesystem\ltx@empty
            \else
            /FS/\pdf@escapename{\EmFi@filesystem}%
            \fi
            /F(\EmFi@@filespec)%
            \ifx\EmFi@@ucfilespec\ltx@empty
            \else
              /UF(\EmFi@@ucfilespec)%
            \fi
            \ifx\EmFi@@desc\ltx@empty
            \else
              /Desc(\EmFi@@desc)%
            \fi
            /EF<<%
              /F \the\pdflastobj\ltx@space 0 R%
            >>%
            \ifEmFi@item
              /CI \EmFi@ci\ltx@space 0 R%
            \fi
          >>%
        }%
        \EmFi@defobj{Filespec}%
        \EmFi@add{%
          \EmFi@@filespec
        }{\the\pdflastobj\ltx@space 0 R}%
      \fi
    \endgroup
  \fi
}
%    \end{macrocode}
%    \end{macro}
%    \begin{macro}{\EmFi@do}
%    \begin{macrocode}
\def\EmFi@do#1{%
  \expandafter\ifx\csname EmFi@P@#1\endcsname\relax
    \expandafter\ifx\csname EmFi@V@#1\endcsname\relax
    \else
      /\pdf@escapename{#1}\csname EmFi@V@#1\endcsname
    \fi
  \else
    /\pdf@escapename{#1}<<%
      \expandafter\ifx\csname EmFi@V@#1\endcsname\relax
      \else
        /D\csname EmFi@V@#1\endcsname
      \fi
      /P(\csname EmFi@P@#1\endcsname)%
    >>%
  \fi
}
%    \end{macrocode}
%    \end{macro}
%    \begin{macro}{\EmFi@convert}
%    \begin{macrocode}
\def\EmFi@convert#1#2{%
  \ifnum\pdf@strcmp{\EmFi@stringmethod}{psd}=0 %
    \pdfstringdef\EmFi@temp{#1}%
    \let#2\EmFi@temp
  \else
    \edef#2{\pdf@escapestring{#1}}%
  \fi
}
%    \end{macrocode}
%    \end{macro}
%    \begin{macrocode}
\global\let\EmFi@list\ltx@empty
%    \end{macrocode}
%    \begin{macro}{\EmFi@add}
%    Sorting is done by the insertion sort algorithm.
%    Probably the sorting could be done more reliable.
%    However, the PDF specification is not too clear to me
%    regarding precise sorting rules (how to deal with
%    different encodings, escaped characters, \dots).
%    \begin{macrocode}
\def\EmFi@add#1#2{%
  \begingroup
    \ifx\EmFi@list\ltx@empty
      \xdef\EmFi@list{\noexpand\do{#1}{#2}}%
    \else
      \def\do##1##2{%
        \ifnum\pdf@strcmp{##1}{#1}>0 %
          \edef\x{%
            \toks@{%
              \the\toks@%
              \noexpand\do{#1}{#2}%
              \noexpand\do{##1}{##2}%
            }%
          }%
          \x
          \def\do####1####2{%
            \toks@\expandafter{\the\toks@\do{####1}{####2}}%
          }%
          \def\stop{%
            \xdef\EmFi@list{\the\toks@}%
          }%
        \else
          \toks@\expandafter{\the\toks@\do{##1}{##2}}%
        \fi
      }%
      \def\stop{%
        \xdef\EmFi@list{\the\toks@\noexpand\do{#1}{#2}}%
      }%
      \toks@{}%
      \EmFi@list\stop
    \fi
  \endgroup
}
%    \end{macrocode}
%    \end{macro}
%    \begin{macro}{\embedfilefinish}
%    \begin{macrocode}
\def\embedfilefinish{%
  \ifEmFi@finished
    \EmFi@Error{%
      Too many invocations of \string\embedfilefinish
    }{%
      The list of embedded files is already written.%
    }%
  \else
    \ifx\EmFi@list\ltx@empty
    \else
%    \end{macrocode}
%    Write |/EmbeddedFiles| entry.
%    \begin{macrocode}
      \global\EmFi@finishedtrue
      \begingroup
        \def\do##1##2{%
          (##1)##2%
        }%
        \immediate\pdfobj{%
          <<%
            /Names[\EmFi@list]%
          >>%
        }%
        \pdfnames{%
          /EmbeddedFiles \the\pdflastobj\ltx@space 0 R%
        }%
      \endgroup
%    \end{macrocode}
%    Write collection objects.
%    \begin{macrocode}
      \ifx\EmFi@initialfile\ltx@empty
      \else
        \EmFi@collectiontrue
      \fi
      \ifEmFi@collection
        \ifx\EmFi@initialfile\ltx@empty
          \let\EmFi@@initialfile\ltx@empty
        \else
          \edef\EmFi@@initialfile{%
            \pdf@escapestring{\EmFi@initialfile}%
          }%
        \fi
%    \end{macrocode}
%    Look for initial file among the embedded files.
%    \begin{macrocode}
        \begingroup
          \let\f=N%
          \def\do##1##2{%
            \def\x{##1}%
            \ifx\x\EmFi@@initialfile
              \let\f=Y%
              \let\do\ltx@gobbletwo
            \fi
          }%
          \EmFi@list
        \expandafter\endgroup
        \ifx\f Y%
        \else
          \@PackageWarningNoLine{embedfile}{%
            Missing initial file `\EmFi@initialfile'\MessageBreak
            among the embedded files%
          }%
          \let\EmFi@initialfile\ltx@empty
          \let\EmFi@@initialfile\ltx@empty
        \fi
        \ifcase\EmFi@sortcase
          \def\EmFi@temp{}%
        \or
          \def\EmFi@temp{%
            /S\EmFi@sortkeys
            /A \EmFi@sortorders
          }%
        \else
          \def\EmFi@temp{%
            /S[\EmFi@sortkeys]%
            /A[\EmFi@sortorders]%
          }%
        \fi
        \def\EmFi@@order##1{%
          \ifnum\EmFi@order>1 %
            /O ##1%
          \fi
        }%
        \immediate\pdfobj{%
          <<%
            \ifx\EmFi@schema\ltx@empty
            \else
              /Schema<<\EmFi@schema>>%
            \fi
            \ifx\EmFi@@initialfile\ltx@empty
            \else
              /D(\EmFi@@initialfile)%
            \fi
            \ifx\EmFi@view\EmFi@S@tile
              /View/T%
            \else\ifx\EmFi@view\EmFi@S@hidden
              /View/H%
            \fi\fi
            \ifx\EmFi@temp\ltx@empty
              \EmFi@temp
            \else
              /Sort<<\EmFi@temp>>%
            \fi
          >>%
        }%
        \pdfcatalog{%
          /Collection \the\pdflastobj\ltx@space0 R%
        }%
      \fi
    \fi
  \fi
}
%    \end{macrocode}
%    \end{macro}
%    \begin{macrocode}
\begingroup\expandafter\expandafter\expandafter\endgroup
\expandafter\ifx\csname AtEndDocument\endcsname\relax
\else
  \AtEndDocument{\embedfilefinish}%
\fi
%    \end{macrocode}
%
%    \begin{macrocode}
\EmFi@AtEnd%
%</package>
%    \end{macrocode}
%
% \section{Test}
%
% \subsection{Catcode checks for loading}
%
%    \begin{macrocode}
%<*test1>
%    \end{macrocode}
%    \begin{macrocode}
\catcode`\{=1 %
\catcode`\}=2 %
\catcode`\#=6 %
\catcode`\@=11 %
\expandafter\ifx\csname count@\endcsname\relax
  \countdef\count@=255 %
\fi
\expandafter\ifx\csname @gobble\endcsname\relax
  \long\def\@gobble#1{}%
\fi
\expandafter\ifx\csname @firstofone\endcsname\relax
  \long\def\@firstofone#1{#1}%
\fi
\expandafter\ifx\csname loop\endcsname\relax
  \expandafter\@firstofone
\else
  \expandafter\@gobble
\fi
{%
  \def\loop#1\repeat{%
    \def\body{#1}%
    \iterate
  }%
  \def\iterate{%
    \body
      \let\next\iterate
    \else
      \let\next\relax
    \fi
    \next
  }%
  \let\repeat=\fi
}%
\def\RestoreCatcodes{}
\count@=0 %
\loop
  \edef\RestoreCatcodes{%
    \RestoreCatcodes
    \catcode\the\count@=\the\catcode\count@\relax
  }%
\ifnum\count@<255 %
  \advance\count@ 1 %
\repeat

\def\RangeCatcodeInvalid#1#2{%
  \count@=#1\relax
  \loop
    \catcode\count@=15 %
  \ifnum\count@<#2\relax
    \advance\count@ 1 %
  \repeat
}
\def\RangeCatcodeCheck#1#2#3{%
  \count@=#1\relax
  \loop
    \ifnum#3=\catcode\count@
    \else
      \errmessage{%
        Character \the\count@\space
        with wrong catcode \the\catcode\count@\space
        instead of \number#3%
      }%
    \fi
  \ifnum\count@<#2\relax
    \advance\count@ 1 %
  \repeat
}
\def\space{ }
\expandafter\ifx\csname LoadCommand\endcsname\relax
  \def\LoadCommand{\input embedfile.sty\relax}%
\fi
\def\Test{%
  \RangeCatcodeInvalid{0}{47}%
  \RangeCatcodeInvalid{58}{64}%
  \RangeCatcodeInvalid{91}{96}%
  \RangeCatcodeInvalid{123}{255}%
  \catcode`\@=12 %
  \catcode`\\=0 %
  \catcode`\%=14 %
  \LoadCommand
  \RangeCatcodeCheck{0}{36}{15}%
  \RangeCatcodeCheck{37}{37}{14}%
  \RangeCatcodeCheck{38}{47}{15}%
  \RangeCatcodeCheck{48}{57}{12}%
  \RangeCatcodeCheck{58}{63}{15}%
  \RangeCatcodeCheck{64}{64}{12}%
  \RangeCatcodeCheck{65}{90}{11}%
  \RangeCatcodeCheck{91}{91}{15}%
  \RangeCatcodeCheck{92}{92}{0}%
  \RangeCatcodeCheck{93}{96}{15}%
  \RangeCatcodeCheck{97}{122}{11}%
  \RangeCatcodeCheck{123}{255}{15}%
  \RestoreCatcodes
}
\Test
\csname @@end\endcsname
\end
%    \end{macrocode}
%    \begin{macrocode}
%</test1>
%    \end{macrocode}
%
% \subsection{Simple test}
%
%    \begin{macrocode}
%<*test2>
\input embedfile.sty\relax
\embedfile[%
  stringmethod=escape,%
  mimetype=plain/text,%
  desc={LaTeX docstrip source archive for package `embedfile'},%
  id={embedfile.dtx}%
]{embedfile.dtx}
\nopagenumbers
Test (plain-\TeX): {\tt embedfile.dtx} should be embedded.%

\def\Test#1{%
  \par
  \embedfileifobjectexists{embedfile.dtx}{#1}{%
    Object #1 (embedfile.dtx): %
    \embedfilegetobject{embedfile.dtx}{#1}%
  }{%
    \errmessage{Missing object #1 (embedfile.dtx)}%
  }%
}
\Test{EmbeddedFile}
\Test{Filespec}
\embedfilefinish
\bye
%</test2>
%    \end{macrocode}
%    \begin{macrocode}
%<*test3>
\NeedsTeXFormat{LaTeX2e}
\let\SavedJobname\jobname
\def\jobname{embedfile}
\RequirePackage{dtx-attach}[2011/04/13]
\let\jobname\SavedJobname
\documentclass{minimal}
\begin{document}
  Test (\LaTeX): \texttt{embedfile.dtx} should be embedded.%
\end{document}
%</test3>
%    \end{macrocode}
%
% \subsection{Test for \iniTeX}
%
%    \begin{macrocode}
%<*test4>
\catcode`\{=1 %
\catcode`\}=2 %
\input ifluatex.sty %
\ifluatex
  \directlua{%
    tex.enableprimitives('', {%
      'pdflastobj',%
      'pdfnames',%
      'pdfobj',%
      'pdfoutput'%
    })%
  }%
\fi
\pdfoutput=1 %
\input embedfile.sty %
\shipout\hbox{}
\embedfile[%
  stringmethod=escape,%
  mimetype=plain/text,%
  desc={iniTeX source},%
]{\jobname.tex}
\embedfilefinish
\end
%</test4>
%    \end{macrocode}
%
% \section{Installation}
%
% \subsection{Download}
%
% \paragraph{Package.} This package is available on
% CTAN\footnote{\url{ftp://ftp.ctan.org/tex-archive/}}:
% \begin{description}
% \item[\CTAN{macros/latex/contrib/oberdiek/embedfile.dtx}] The source file.
% \item[\CTAN{macros/latex/contrib/oberdiek/embedfile.pdf}] Documentation.
% \end{description}
%
%
% \paragraph{Bundle.} All the packages of the bundle `oberdiek'
% are also available in a TDS compliant ZIP archive. There
% the packages are already unpacked and the documentation files
% are generated. The files and directories obey the TDS standard.
% \begin{description}
% \item[\CTAN{install/macros/latex/contrib/oberdiek.tds.zip}]
% \end{description}
% \emph{TDS} refers to the standard ``A Directory Structure
% for \TeX\ Files'' (\CTAN{tds/tds.pdf}). Directories
% with \xfile{texmf} in their name are usually organized this way.
%
% \subsection{Bundle installation}
%
% \paragraph{Unpacking.} Unpack the \xfile{oberdiek.tds.zip} in the
% TDS tree (also known as \xfile{texmf} tree) of your choice.
% Example (linux):
% \begin{quote}
%   |unzip oberdiek.tds.zip -d ~/texmf|
% \end{quote}
%
% \paragraph{Script installation.}
% Check the directory \xfile{TDS:scripts/oberdiek/} for
% scripts that need further installation steps.
% Package \xpackage{attachfile2} comes with the Perl script
% \xfile{pdfatfi.pl} that should be installed in such a way
% that it can be called as \texttt{pdfatfi}.
% Example (linux):
% \begin{quote}
%   |chmod +x scripts/oberdiek/pdfatfi.pl|\\
%   |cp scripts/oberdiek/pdfatfi.pl /usr/local/bin/|
% \end{quote}
%
% \subsection{Package installation}
%
% \paragraph{Unpacking.} The \xfile{.dtx} file is a self-extracting
% \docstrip\ archive. The files are extracted by running the
% \xfile{.dtx} through \plainTeX:
% \begin{quote}
%   \verb|tex embedfile.dtx|
% \end{quote}
%
% \paragraph{TDS.} Now the different files must be moved into
% the different directories in your installation TDS tree
% (also known as \xfile{texmf} tree):
% \begin{quote}
% \def\t{^^A
% \begin{tabular}{@{}>{\ttfamily}l@{ $\rightarrow$ }>{\ttfamily}l@{}}
%   embedfile.sty & tex/generic/oberdiek/embedfile.sty\\
%   dtx-attach.sty & tex/generic/oberdiek/dtx-attach.sty\\
%   embedfile.pdf & doc/latex/oberdiek/embedfile.pdf\\
%   embedfile-example-plain.tex & doc/latex/oberdiek/embedfile-example-plain.tex\\
%   embedfile-example-collection.tex & doc/latex/oberdiek/embedfile-example-collection.tex\\
%   test/embedfile-test1.tex & doc/latex/oberdiek/test/embedfile-test1.tex\\
%   test/embedfile-test2.tex & doc/latex/oberdiek/test/embedfile-test2.tex\\
%   test/embedfile-test3.tex & doc/latex/oberdiek/test/embedfile-test3.tex\\
%   test/embedfile-test4.tex & doc/latex/oberdiek/test/embedfile-test4.tex\\
%   embedfile.dtx & source/latex/oberdiek/embedfile.dtx\\
% \end{tabular}^^A
% }^^A
% \sbox0{\t}^^A
% \ifdim\wd0>\linewidth
%   \begingroup
%     \advance\linewidth by\leftmargin
%     \advance\linewidth by\rightmargin
%   \edef\x{\endgroup
%     \def\noexpand\lw{\the\linewidth}^^A
%   }\x
%   \def\lwbox{^^A
%     \leavevmode
%     \hbox to \linewidth{^^A
%       \kern-\leftmargin\relax
%       \hss
%       \usebox0
%       \hss
%       \kern-\rightmargin\relax
%     }^^A
%   }^^A
%   \ifdim\wd0>\lw
%     \sbox0{\small\t}^^A
%     \ifdim\wd0>\linewidth
%       \ifdim\wd0>\lw
%         \sbox0{\footnotesize\t}^^A
%         \ifdim\wd0>\linewidth
%           \ifdim\wd0>\lw
%             \sbox0{\scriptsize\t}^^A
%             \ifdim\wd0>\linewidth
%               \ifdim\wd0>\lw
%                 \sbox0{\tiny\t}^^A
%                 \ifdim\wd0>\linewidth
%                   \lwbox
%                 \else
%                   \usebox0
%                 \fi
%               \else
%                 \lwbox
%               \fi
%             \else
%               \usebox0
%             \fi
%           \else
%             \lwbox
%           \fi
%         \else
%           \usebox0
%         \fi
%       \else
%         \lwbox
%       \fi
%     \else
%       \usebox0
%     \fi
%   \else
%     \lwbox
%   \fi
% \else
%   \usebox0
% \fi
% \end{quote}
% If you have a \xfile{docstrip.cfg} that configures and enables \docstrip's
% TDS installing feature, then some files can already be in the right
% place, see the documentation of \docstrip.
%
% \subsection{Refresh file name databases}
%
% If your \TeX~distribution
% (\teTeX, \mikTeX, \dots) relies on file name databases, you must refresh
% these. For example, \teTeX\ users run \verb|texhash| or
% \verb|mktexlsr|.
%
% \subsection{Some details for the interested}
%
% \paragraph{Attached source.}
%
% The PDF documentation on CTAN also includes the
% \xfile{.dtx} source file. It can be extracted by
% AcrobatReader 6 or higher. Another option is \textsf{pdftk},
% e.g. unpack the file into the current directory:
% \begin{quote}
%   \verb|pdftk embedfile.pdf unpack_files output .|
% \end{quote}
%
% \paragraph{Unpacking with \LaTeX.}
% The \xfile{.dtx} chooses its action depending on the format:
% \begin{description}
% \item[\plainTeX:] Run \docstrip\ and extract the files.
% \item[\LaTeX:] Generate the documentation.
% \end{description}
% If you insist on using \LaTeX\ for \docstrip\ (really,
% \docstrip\ does not need \LaTeX), then inform the autodetect routine
% about your intention:
% \begin{quote}
%   \verb|latex \let\install=y% \iffalse meta-comment
%
% File: embedfile.dtx
% Version: 2011/04/13 v2.6
% Info: Embed files into PDF
%
% Copyright (C) 2006-2011 by
%    Heiko Oberdiek <heiko.oberdiek at googlemail.com>
%
% This work may be distributed and/or modified under the
% conditions of the LaTeX Project Public License, either
% version 1.3c of this license or (at your option) any later
% version. This version of this license is in
%    http://www.latex-project.org/lppl/lppl-1-3c.txt
% and the latest version of this license is in
%    http://www.latex-project.org/lppl.txt
% and version 1.3 or later is part of all distributions of
% LaTeX version 2005/12/01 or later.
%
% This work has the LPPL maintenance status "maintained".
%
% This Current Maintainer of this work is Heiko Oberdiek.
%
% The Base Interpreter refers to any `TeX-Format',
% because some files are installed in TDS:tex/generic//.
%
% This work consists of the main source file embedfile.dtx
% and the derived files
%    embedfile.sty, embedfile.pdf, embedfile.ins, embedfile.drv,
%    dtx-attach.sty, embedfile-example-plain.tex,
%    embedfile-example-collection.tex, embedfile-test1.tex,
%    embedfile-test2.tex, embedfile-test3.tex,
%    embedfile-test4.tex.
%
% Distribution:
%    CTAN:macros/latex/contrib/oberdiek/embedfile.dtx
%    CTAN:macros/latex/contrib/oberdiek/embedfile.pdf
%
% Unpacking:
%    (a) If embedfile.ins is present:
%           tex embedfile.ins
%    (b) Without embedfile.ins:
%           tex embedfile.dtx
%    (c) If you insist on using LaTeX
%           latex \let\install=y% \iffalse meta-comment
%
% File: embedfile.dtx
% Version: 2011/04/13 v2.6
% Info: Embed files into PDF
%
% Copyright (C) 2006-2011 by
%    Heiko Oberdiek <heiko.oberdiek at googlemail.com>
%
% This work may be distributed and/or modified under the
% conditions of the LaTeX Project Public License, either
% version 1.3c of this license or (at your option) any later
% version. This version of this license is in
%    http://www.latex-project.org/lppl/lppl-1-3c.txt
% and the latest version of this license is in
%    http://www.latex-project.org/lppl.txt
% and version 1.3 or later is part of all distributions of
% LaTeX version 2005/12/01 or later.
%
% This work has the LPPL maintenance status "maintained".
%
% This Current Maintainer of this work is Heiko Oberdiek.
%
% The Base Interpreter refers to any `TeX-Format',
% because some files are installed in TDS:tex/generic//.
%
% This work consists of the main source file embedfile.dtx
% and the derived files
%    embedfile.sty, embedfile.pdf, embedfile.ins, embedfile.drv,
%    dtx-attach.sty, embedfile-example-plain.tex,
%    embedfile-example-collection.tex, embedfile-test1.tex,
%    embedfile-test2.tex, embedfile-test3.tex,
%    embedfile-test4.tex.
%
% Distribution:
%    CTAN:macros/latex/contrib/oberdiek/embedfile.dtx
%    CTAN:macros/latex/contrib/oberdiek/embedfile.pdf
%
% Unpacking:
%    (a) If embedfile.ins is present:
%           tex embedfile.ins
%    (b) Without embedfile.ins:
%           tex embedfile.dtx
%    (c) If you insist on using LaTeX
%           latex \let\install=y\input{embedfile.dtx}
%        (quote the arguments according to the demands of your shell)
%
% Documentation:
%    (a) If embedfile.drv is present:
%           latex embedfile.drv
%    (b) Without embedfile.drv:
%           latex embedfile.dtx; ...
%    The class ltxdoc loads the configuration file ltxdoc.cfg
%    if available. Here you can specify further options, e.g.
%    use A4 as paper format:
%       \PassOptionsToClass{a4paper}{article}
%
%    Programm calls to get the documentation (example):
%       pdflatex embedfile.dtx
%       makeindex -s gind.ist embedfile.idx
%       pdflatex embedfile.dtx
%       makeindex -s gind.ist embedfile.idx
%       pdflatex embedfile.dtx
%
% Installation:
%    TDS:tex/generic/oberdiek/embedfile.sty
%    TDS:tex/generic/oberdiek/dtx-attach.sty
%    TDS:doc/latex/oberdiek/embedfile.pdf
%    TDS:doc/latex/oberdiek/embedfile-example-plain.tex
%    TDS:doc/latex/oberdiek/embedfile-example-collection.tex
%    TDS:doc/latex/oberdiek/test/embedfile-test1.tex
%    TDS:doc/latex/oberdiek/test/embedfile-test2.tex
%    TDS:doc/latex/oberdiek/test/embedfile-test3.tex
%    TDS:doc/latex/oberdiek/test/embedfile-test4.tex
%    TDS:source/latex/oberdiek/embedfile.dtx
%
%<*ignore>
\begingroup
  \catcode123=1 %
  \catcode125=2 %
  \def\x{LaTeX2e}%
\expandafter\endgroup
\ifcase 0\ifx\install y1\fi\expandafter
         \ifx\csname processbatchFile\endcsname\relax\else1\fi
         \ifx\fmtname\x\else 1\fi\relax
\else\csname fi\endcsname
%</ignore>
%<*install>
\input docstrip.tex
\Msg{************************************************************************}
\Msg{* Installation}
\Msg{* Package: embedfile 2011/04/13 v2.6 Embed files into PDF (HO)}
\Msg{************************************************************************}

\keepsilent
\askforoverwritefalse

\let\MetaPrefix\relax
\preamble

This is a generated file.

Project: embedfile
Version: 2011/04/13 v2.6

Copyright (C) 2006-2011 by
   Heiko Oberdiek <heiko.oberdiek at googlemail.com>

This work may be distributed and/or modified under the
conditions of the LaTeX Project Public License, either
version 1.3c of this license or (at your option) any later
version. This version of this license is in
   http://www.latex-project.org/lppl/lppl-1-3c.txt
and the latest version of this license is in
   http://www.latex-project.org/lppl.txt
and version 1.3 or later is part of all distributions of
LaTeX version 2005/12/01 or later.

This work has the LPPL maintenance status "maintained".

This Current Maintainer of this work is Heiko Oberdiek.

The Base Interpreter refers to any `TeX-Format',
because some files are installed in TDS:tex/generic//.

This work consists of the main source file embedfile.dtx
and the derived files
   embedfile.sty, embedfile.pdf, embedfile.ins, embedfile.drv,
   dtx-attach.sty, embedfile-example-plain.tex,
   embedfile-example-collection.tex, embedfile-test1.tex,
   embedfile-test2.tex, embedfile-test3.tex,
   embedfile-test4.tex.

\endpreamble
\let\MetaPrefix\DoubleperCent

\generate{%
  \file{embedfile.ins}{\from{embedfile.dtx}{install}}%
  \file{embedfile.drv}{\from{embedfile.dtx}{driver}}%
  \usedir{tex/generic/oberdiek}%
  \file{embedfile.sty}{\from{embedfile.dtx}{package}}%
  \usedir{tex/latex/oberdiek}%
  \file{dtx-attach.sty}{\from{embedfile.dtx}{dtxattach}}%
  \usedir{doc/latex/oberdiek}%
  \file{embedfile-example-plain.tex}{\from{embedfile.dtx}{exampleplain}}%
  \file{embedfile-example-collection.tex}{\from{embedfile.dtx}{examplecollection}}%
  \usedir{doc/latex/oberdiek/test}%
  \file{embedfile-test1.tex}{\from{embedfile.dtx}{test1}}%
  \file{embedfile-test2.tex}{\from{embedfile.dtx}{test2}}%
  \file{embedfile-test3.tex}{\from{embedfile.dtx}{test3}}%
  \file{embedfile-test4.tex}{\from{embedfile.dtx}{test4}}%
  \nopreamble
  \nopostamble
  \usedir{source/latex/oberdiek/catalogue}%
  \file{embedfile.xml}{\from{embedfile.dtx}{catalogue}}%
}

\catcode32=13\relax% active space
\let =\space%
\Msg{************************************************************************}
\Msg{*}
\Msg{* To finish the installation you have to move the following}
\Msg{* files into a directory searched by TeX:}
\Msg{*}
\Msg{*     embedfile.sty, dtx-attach.sty}
\Msg{*}
\Msg{* To produce the documentation run the file `embedfile.drv'}
\Msg{* through LaTeX.}
\Msg{*}
\Msg{* Happy TeXing!}
\Msg{*}
\Msg{************************************************************************}

\endbatchfile
%</install>
%<*ignore>
\fi
%</ignore>
%<*driver>
\NeedsTeXFormat{LaTeX2e}
\ProvidesFile{embedfile.drv}%
  [2011/04/13 v2.6 Embed files into PDF (HO)]%
\documentclass{ltxdoc}
\usepackage{holtxdoc}[2011/11/22]
\begin{document}
  \DocInput{embedfile.dtx}%
\end{document}
%</driver>
% \fi
%
% \CheckSum{1363}
%
% \CharacterTable
%  {Upper-case    \A\B\C\D\E\F\G\H\I\J\K\L\M\N\O\P\Q\R\S\T\U\V\W\X\Y\Z
%   Lower-case    \a\b\c\d\e\f\g\h\i\j\k\l\m\n\o\p\q\r\s\t\u\v\w\x\y\z
%   Digits        \0\1\2\3\4\5\6\7\8\9
%   Exclamation   \!     Double quote  \"     Hash (number) \#
%   Dollar        \$     Percent       \%     Ampersand     \&
%   Acute accent  \'     Left paren    \(     Right paren   \)
%   Asterisk      \*     Plus          \+     Comma         \,
%   Minus         \-     Point         \.     Solidus       \/
%   Colon         \:     Semicolon     \;     Less than     \<
%   Equals        \=     Greater than  \>     Question mark \?
%   Commercial at \@     Left bracket  \[     Backslash     \\
%   Right bracket \]     Circumflex    \^     Underscore    \_
%   Grave accent  \`     Left brace    \{     Vertical bar  \|
%   Right brace   \}     Tilde         \~}
%
% \GetFileInfo{embedfile.drv}
%
% \title{The \xpackage{embedfile} package}
% \date{2011/04/13 v2.6}
% \author{Heiko Oberdiek\\\xemail{heiko.oberdiek at googlemail.com}}
%
% \maketitle
%
% \begin{abstract}
% This package embeds files to a PDF document.
% Currently the only supported driver is \pdfTeX\ $>=$ 1.30 in PDF mode.
% \end{abstract}
%
% \tableofcontents
%
% \section{Documentation}
%
% \subsection{Introduction}
%
%    The PDF format (\cite{pdfspec}) allows the inclusion of files
%    inside the PDF document. The included files can be bound to an
%    annotation on a page. Or they can be recorded in a sorted
%    list of embedded files. The packages \xpackage{attachfile}
%    or \xpackage{attachfile2} follow the first approach, this
%    package uses the latter method.
%
% \subsubsection{Future development}
%
%    My dream is a large package that merges the features of
%    all these packages meantioned before:
%    \begin{itemize}
%    \item Files can be attached to a page.
%    \item Files can be attached to the document.
%    \item An easy user interface for simple, common tasks and
%          beginners.
%    \item An interface for the advanced users that want to setup
%          every detail.
%    \item Support of many drivers (pdftex, dvips, dvipdfm, \dots).
%    \item \dots
%    \end{itemize}
%    However, I have not managed to take the time for this project.
%    Instead:
%    \begin{itemize}
%    \item First I experimented with package \xpackage{attachfile},
%          adding driver support, fixing bugs, \dots. The result is
%          currently named as \xpackage{attachfile2}. It uses an external
%          script to get file properties (size, date, checksum, \dots).
%    \item In order to avoid an external program for getting basic
%          file properties I provided a patch ``EscapeAndOther'' for
%          pdfTeX that was accepted for version 1.30.
%    \item This package closes a gap left by the packages for attaching
%          files and allows the embedding of files to the document.
%          Also it makes use of the new primitives of \pdfTeX.
%    \end{itemize}
%
% \subsection{User interface}
%
%    This package \xpackage{embedfile} can be used with both
%    \LaTeX\ and \plainTeX. See \autoref{sec:plain} that
%    explains the use with \plainTeX\ by an example.
%    In \LaTeX\ the package is loaded as usually. There are no options.
%    \begin{quote}
%      \verb|\usepackage{embedfile}|
%    \end{quote}
%
%    \begin{declcs}{embedfile}
%      |[|\meta{options}|]| \M{file}
%    \end{declcs}
%    The macro \cs{embedfile} includes file \meta{file} and
%    attaches it to the PDF document. At the end of the document
%    the sorted list of embedded files are written. Thus you can
%    safely use \cs{embedfile} before \verb|\end{document}|.
%    Embedding files using \cs{AtEndDocument} will only work,
%    if \cs{AtEndDocument} is called before loading the package
%    \xpackage{embedfile}.
%
%    The \meta{options} are give as key value pairs. The following
%    keys are supported:
%    \begin{description}
%    \item[\xoption{filespec}]
%      This allows to override the file name
%      that appears in the PDF file. If you are using other than
%      simple file names (8-bit, path separators, \dots), look into
%      the PDF specification (\cite{pdfspec}).
%      There are rules how these file names must be written/encoded.
%      Avoid 8-bit characters
%      and other special characters, the behaviour is currently undefined.
%      Use option \xoption{ucfilespec} for more funny file names.
%      The string method, see below, is |escape| since version 2.4.
%
%      This name is also used as entry in a name tree
%      (see PDF specification: |/EmbeddedFiles|). Therefore the
%      value for \xoption{filespec} must be unique among all
%      embedded files. Also key \xoption{initialfiles} refers
%      to this name, if the file name and the value of \xoption{filespec}
%      are different.
%    \item[\xoption{ucfilespec}]
%      Since PDF 1.7 the file name may be provided in Unicode.
%      The conversion of the option value into a PDF string
%      is controlled by option \xoption{stringmethod}.
%    \item[\xoption{filesystem}]
%      This sets the entry \verb|/FS| in
%      the file specification dictionary, see PDF specification
%      (\cite{pdfspec}).
%      Example: \verb|filesystem=URL|.
%    \item[\xoption{mimetype}]
%      This sets the mime type (\cite{mime}) of the file,
%      see \autoref{sec:plain} for examples and \cite{types}
%      for a list of officially registered types.
%    \item[\xoption{desc}]
%      The description for the file.
%    \item[\xoption{stringmethod}]
%      The package must convert the values of the keys \xoption{ucfilespec}
%      and \xoption{desc} into a PDF string (before version 2.4: \xoption{filespec}
%      and \xoption{desc}). If \xpackage{hyperref}
%      is found, then its \cs{pdfstringdef} will be used, otherwise
%      \pdfTeX's \cs{pdfescapestring} is used. Value \verb|psd|
%      forces the use of \cs{pdfstringdef}, value \verb|escape|
%      the use of \cs{pdfescapestring}.
%    \item[\xoption{\meta{key}.value}] Sets the value of a collection
%      item property, see section \ref{sec:collection}.
%    \item[\xoption{\meta{key}.prefix}]
%      Sets the prefix of a collection item property,
%      see section \ref{sec:collection}.
%    \item[\xoption{id}]
%      The value must be an unique name. Macros
%      \cs{embedfileifobjectexists} and \cs{embedfilegetobject}
%      are using this name later.
%    \end{description}
%
%    \begin{declcs}{embedfilefinish}
%    \end{declcs}
%    The list of all embedded files must be added as data structure
%    in the PDF file. In case of \LaTeX\ this is automatically done.
%    The package uses \cs{AtEndDocument}. Then the list of all
%    files should be known. However, \plainTeX\ does not know
%    about \cs{AtEndDocument}. Thus the user must call
%    \cs{embedfilefinish} at the end of the document
%    after the last file is embedded.
%
%    \begin{declcs}{embedfilesetup} \M{options}
%    \end{declcs}
%    Options for \cs{embedfile} and collection support
%    can be set in \cs{embedfilesetup}.
%
% \subsection{Collection support (PDF 1.7)}
% \label{sec:collection}
%
%    Since PDF 1.7 the embedded files can form a \emph{collection}
%    (sometimes referred as \emph{package}), the main document
%    is called \emph{cover sheet}. See PDF specification
%    8.2.4 ``Collections'' and 3.10.5 ``Collection items'' \cite{pdfspec}.
%
%    Usually Acrobat Reader 7 or 8 shows the
%    embedded files in a table at the bottom with the following columns:
%    \begin{quote}
%    \begin{tabular}{@{}l|l|l|l@{}}
%       Name & Description & Modified & Size\\
%       \hline
%       \dots & \dots & \dots & \dots\\
%    \end{tabular}
%    \end{quote}
%    Acrobat Reader 10 shows the embeded files in the left panel
%    and adds a new column for the compressed size.
%
%    If the files form a collection, then they are displayed
%    in a table left or top (depending on option \xoption{view},
%    see \cs{embedfilesetup}).
%
%
%    Collection support is enabled automatically, if
%    it is used.
%
%    \begin{declcs}{embedfilesetup} \M{options}
%    \end{declcs}
%    The following options are supported in addition
%    to options for \cs{embedfile}:
%    \begin{description}
%    \item[\xoption{view}]
%      If the PDF file contains a collection, then Acrobat Reader 8
%      shows a line at the top below the menu bar and the toolbar.
%      It shows the current selected file, icons for changing the
%      view mode, an options menu.
%      The initial mode how the collection is presented is
%      set by this option \xoption{view}.
%      The following modes/values are supported, the default is |details|:
%        \begin{description}
%        \item[\texttt{details}] The full collection table is displayed
%          at the top below the collection bar.
%        \item[\texttt{tile}] The files of the collection are shown
%          in tile mode on the left.
%        \item[\texttt{hidden}] The collection table is not shown.
%        \end{description}
%    \item[\xoption{initialfile}]
%      Selects the file that is initially presented. Especially
%      useful for an embedded PDF file that is then shown instead
%      of the cover document. There must be an \cs{embedfile} command
%      somewhere whose value for key \xoption{filespec} is used here.
%      The \cs{embedfile} command can drop option \xoption{filespec}
%      if the file name is not different.
%    \end{description}
%
%    \begin{declcs}{embedfilefield} \M{key} \M{options}
%    \end{declcs}
%    Macro \cs{embedfilefield} defines a column/field in the collection table.
%    The name of the field is \meta{key}.
%    \begin{description}
%    \item[\xoption{type}]
%      sets the type of the field. The supported values are:
%      \begin{description}
%      \item[\texttt{text}] A text field. Its value is set in
%        \cs{embedfile} by option \xoption{\meta{key}.value}.
%      \item[\texttt{date}] A date field. Its value is set in
%        \cs{embedfile} by option \xoption{\meta{key}.value}.
%        A special format is required, see ``3.8.3 Dates'' \cite{pdfspec}.
%      \item[\texttt{number}] A field with an integer or float number.
%        Its value is set in \cs{embedfile} by option
%        \xoption{\meta{key}.value}.
%      \item[\texttt{file}] The file name of the embedded file.
%      \item[\texttt{desc}] The description text of the embedded file. It is
%       set in \cs{embedfile} by option \xoption{desc}.
%      \item[\texttt{moddate}] The modification date of the embedded file.
%      \item[\texttt{size}] The size of the embedded file.
%      \end{description}
%      All types allow the use of a prefix that is disregarded by
%      sorting. The prefix for this field is set in \cs{embedfile} by
%      option \xoption{\meta{key}.prefix}.
%    \item[\xoption{title}] sets the column title.
%    \item[\xoption{visible}] controls whether the column is presented:
%      \begin{description}
%      \item[\texttt{true}] shows the column.
%      \item[\texttt{false}] hides the column.
%      \end{description}
%      Default: \texttt{true}
%    \item[\xoption{edit}]
%      Allows the editing of field values. Does not seem to have an
%      effect for Acrobat Reader.
%      \begin{description}
%      \item[\texttt{true}] enables the feature, if available (depends
%        on the PDF viewer).
%      \item[\texttt{false}] disables the feature.
%      \end{description}
%      Default: \texttt{false}
%    \end{description}
%    The order of \cs{embedfilefield} statements defines the order
%    of the columns.
%
%    \begin{declcs}{embedfilesort} \M{key-sort-list}
%    \end{declcs}
%    The sort order of the embedded files are controlled by
%    macro \cs{embedfilesort}. \meta{key-sort-list} defines
%    the sort order. The key is a field name defined by \cs{embedfilefield}.
%    Its value is either \texttt{ascending} or \texttt{descending}.
%    The default is \texttt{ascending}.
%
% \subsection{Export of object references}
%
%    Caution: This feature is still experimental. It may be even removed
%    in future versions. Therefore feedback would be nice, if someone
%    has a useful application for this feature.
%
%    Object numbers are saved, if \xoption{id} is given in \cs{embedfile}.
%    The following objects are supported:
%    \begin{itemize}
%    \item \texttt{EmbeddedFile}
%    \item \texttt{Filespec}
%    \end{itemize}
%
%    \begin{declcs}{embedfileifobjectexists} \M{id} \M{type}
%      \M{then} \M{else}
%    \end{declcs}
%    Macro \cs{embedfileifobjectexists} tests whether object of \meta{type}
%    is available for the embedded file identified by \meta{id}.
%
%    \begin{declcs}{embedfilegetobject} \M{id} \M{type}
%    \end{declcs}
%    Macro \cs{embedfilegetobject} expands to the full object reference
%    object of \meta{type} for the embedded file identified by \meta{id}.
%
% \subsubsection{Example}
%
%\begin{quote}
%\begin{verbatim}
%\embedfile[id={foo}]{foo.pdf}
%\embedfileifobjectexists{foo}{Filespec}{%
%  \typeout{%
%    FileSpec object for `foo': %
%    \embedfilegetobject{foo}{Filespec}%
%  }%
%}{%
%  \typeout{No Filespec object for `foo'}%
%}
%\end{verbatim}
%\end{quote}
%
% \subsection{Examples}
% \label{sec:examples}
%
% \subsubsection{\plainTeX}
% \label{sec:plain}
%
%    The package can be used with \plainTeX. It can be used
%    with or without help from \xfile{miniltx.tex}.
%
%    If additionally package \xpackage{keyval} (\xpackage{graphicx})
%    is needed, load it first. Then package \xpackage{embedfile} avoids
%    a duplicate loading of package \xpackage{keyval}.
%
%    Because \plainTeX\ does not provide a hook at end of the document,
%    you have to call \cs{embedfilefinish} manually at the end
%    after the last embedded file.
%    \begin{macrocode}
%<*exampleplain>
%<<END
% Load packages
\input miniltx
% \def\Gin@driver{pdftex.def}
% \input graphicx.sty
\input embedfile.sty
\resetatcatcode

% default setting
\embedfilesetup{
  mimetype=text/plain
}

% Embed files
\embedfile[
  filespec=example.tex,
  desc={Source code (plain-TeX) of this example}
]{embedfile-example-plain.tex}

\embedfile[
  desc={Source of package `embedfile'}
]{embedfile.dtx}

\embedfile[
  mimetype=application/pdf,
  desc={Documentation of package `embedfile'}
]{embedfile.pdf}

% Some text
This example document contains three embedded files.

% End of document
\embedfilefinish % don't forget
\bye
%END
%</exampleplain>
%    \end{macrocode}
%
% \subsubsection{Collection example}
%
%    \begin{macrocode}
%<*examplecollection>
%<<END
\NeedsTeXFormat{LaTeX2e}
\documentclass{article}
\usepackage[bookmarks=false]{hyperref}
% provides \pdfstringdef that is then used by `title' and
% other keys.
\usepackage{embedfile}[2011/04/13]
\embedfilesetup{
  view=details,
  initialfile=embedfile.pdf
}
\embedfilefield{file}{
  type=file,
  title={File name}
}
\embedfilefield{description}{
  type=desc,
  title={Description}
}
\embedfilefield{date}{
  type=moddate,
  title={Date}
}
\embedfilefield{size}{
  type=size,
  title={Size}
}
\embedfilefield{type}{
  type=text,
  title={Type},
  visible=false
}
\embedfilesort{
  type,
  date=descending
}
\begin{document}
An example for embedded files as collection.
You need Acrobat Reader 8 or higher.

\embedfile[
  desc={Source file of package `embedfile'},
  description.prefix={Package: },
  type.value={DTX}
]{embedfile.dtx}

\embedfile[
  desc={Documentation of package `embedfile'},
  description.prefix={Package: },
  type.value={PDF}
]{embedfile.pdf}

\embedfile[
  desc={The source for this example},
  description.prefix={Example: },
  type.value={TEX}
]{\jobname.tex}

\end{document}
%END
%</examplecollection>
%    \end{macrocode}
%
% \subsection{Package \xpackage{dtx-attach}}
%
% Package \xpackage{dtx-attach} is just a small application of
% package \xpackage{embedfile}. I am using it for the CTAN
% documentation of my packages in \CTAN{macros/latex/contrib/oberdiek/}.
% It also serves as small example for the use of the package with \LaTeX.
%
%    \begin{macrocode}
%<*dtxattach>
\NeedsTeXFormat{LaTeX2e}
\ProvidesPackage{dtx-attach}
  [2011/04/13 v2.6 Embed \string\jobname.dtx (HO)]%
\RequirePackage{embedfile}[2011/04/13]
\embedfile[%
  stringmethod=escape,%
  mimetype=plain/text,%
  desc={LaTeX docstrip source archive for package `\jobname'}%
]{\jobname.dtx}
%</dtxattach>
%    \end{macrocode}
%
% \StopEventually{
% }
%
% \section{Implementation}
%
%    \begin{macrocode}
%<*package>
%    \end{macrocode}
%
% \subsection{Reload check and package identification}
%    Reload check, especially if the package is not used with \LaTeX.
%    \begin{macrocode}
\begingroup\catcode61\catcode48\catcode32=10\relax%
  \catcode13=5 % ^^M
  \endlinechar=13 %
  \catcode35=6 % #
  \catcode39=12 % '
  \catcode44=12 % ,
  \catcode45=12 % -
  \catcode46=12 % .
  \catcode58=12 % :
  \catcode64=11 % @
  \catcode123=1 % {
  \catcode125=2 % }
  \expandafter\let\expandafter\x\csname ver@embedfile.sty\endcsname
  \ifx\x\relax % plain-TeX, first loading
  \else
    \def\empty{}%
    \ifx\x\empty % LaTeX, first loading,
      % variable is initialized, but \ProvidesPackage not yet seen
    \else
      \expandafter\ifx\csname PackageInfo\endcsname\relax
        \def\x#1#2{%
          \immediate\write-1{Package #1 Info: #2.}%
        }%
      \else
        \def\x#1#2{\PackageInfo{#1}{#2, stopped}}%
      \fi
      \x{embedfile}{The package is already loaded}%
      \aftergroup\endinput
    \fi
  \fi
\endgroup%
%    \end{macrocode}
%    Package identification:
%    \begin{macrocode}
\begingroup\catcode61\catcode48\catcode32=10\relax%
  \catcode13=5 % ^^M
  \endlinechar=13 %
  \catcode35=6 % #
  \catcode39=12 % '
  \catcode40=12 % (
  \catcode41=12 % )
  \catcode44=12 % ,
  \catcode45=12 % -
  \catcode46=12 % .
  \catcode47=12 % /
  \catcode58=12 % :
  \catcode64=11 % @
  \catcode91=12 % [
  \catcode93=12 % ]
  \catcode123=1 % {
  \catcode125=2 % }
  \expandafter\ifx\csname ProvidesPackage\endcsname\relax
    \def\x#1#2#3[#4]{\endgroup
      \immediate\write-1{Package: #3 #4}%
      \xdef#1{#4}%
    }%
  \else
    \def\x#1#2[#3]{\endgroup
      #2[{#3}]%
      \ifx#1\@undefined
        \xdef#1{#3}%
      \fi
      \ifx#1\relax
        \xdef#1{#3}%
      \fi
    }%
  \fi
\expandafter\x\csname ver@embedfile.sty\endcsname
\ProvidesPackage{embedfile}%
  [2011/04/13 v2.6 Embed files into PDF (HO)]%
%    \end{macrocode}
%
% \subsection{Catcodes}
%
%    \begin{macrocode}
\begingroup\catcode61\catcode48\catcode32=10\relax%
  \catcode13=5 % ^^M
  \endlinechar=13 %
  \catcode123=1 % {
  \catcode125=2 % }
  \catcode64=11 % @
  \def\x{\endgroup
    \expandafter\edef\csname EmFi@AtEnd\endcsname{%
      \endlinechar=\the\endlinechar\relax
      \catcode13=\the\catcode13\relax
      \catcode32=\the\catcode32\relax
      \catcode35=\the\catcode35\relax
      \catcode61=\the\catcode61\relax
      \catcode64=\the\catcode64\relax
      \catcode123=\the\catcode123\relax
      \catcode125=\the\catcode125\relax
    }%
  }%
\x\catcode61\catcode48\catcode32=10\relax%
\catcode13=5 % ^^M
\endlinechar=13 %
\catcode35=6 % #
\catcode64=11 % @
\catcode123=1 % {
\catcode125=2 % }
\def\TMP@EnsureCode#1#2{%
  \edef\EmFi@AtEnd{%
    \EmFi@AtEnd
    \catcode#1=\the\catcode#1\relax
  }%
  \catcode#1=#2\relax
}
\TMP@EnsureCode{39}{12}% '
\TMP@EnsureCode{40}{12}% (
\TMP@EnsureCode{41}{12}% )
\TMP@EnsureCode{44}{12}% ,
\TMP@EnsureCode{46}{12}% .
\TMP@EnsureCode{47}{12}% /
\TMP@EnsureCode{58}{12}% :
\TMP@EnsureCode{60}{12}% <
\TMP@EnsureCode{62}{12}% >
\TMP@EnsureCode{91}{12}% [
\TMP@EnsureCode{93}{12}% ]
\TMP@EnsureCode{96}{12}% `
\edef\EmFi@AtEnd{\EmFi@AtEnd\noexpand\endinput}
%    \end{macrocode}
%
% \subsection{Tools}
%
%    \begin{macro}{\EmFi@RequirePackage}
%    \begin{macrocode}
\begingroup\expandafter\expandafter\expandafter\endgroup
\expandafter\ifx\csname RequirePackage\endcsname\relax
  \def\EmFi@RequirePackage#1[#2]{%
    \input #1.sty\relax
  }%
\else
  \let\EmFi@RequirePackage\RequirePackage
\fi
%    \end{macrocode}
%    \end{macro}
%
%    \begin{macro}{\EmFi@Error}
%    \begin{macrocode}
\EmFi@RequirePackage{infwarerr}[2007/09/09]%
\def\EmFi@Error{%
  \@PackageError{embedfile}%
}
%    \end{macrocode}
%    \end{macro}
%
% \subsection{Check for recent \pdfTeX\ in PDF mode}
%
%    Load package \xpackage{ifpdf} and check mode.
%    \begin{macrocode}
\EmFi@RequirePackage{ifpdf}[2007/09/09]
\ifpdf
\else
  \EmFi@Error{%
    Missing pdfTeX in PDF mode%
  }{%
    Currently other drivers are not supported. %
    Package loading is aborted.%
  }%
  \expandafter\EmFi@AtEnd
\fi%
%    \end{macrocode}
%    \begin{macrocode}
\EmFi@RequirePackage{pdftexcmds}[2007/11/11]
\EmFi@RequirePackage{ltxcmds}[2010/03/01]
\EmFi@RequirePackage{kvsetkeys}[2010/03/01]
\EmFi@RequirePackage{kvdefinekeys}[2010/03/01]
%    \end{macrocode}
%    Check version.
%    \begin{macrocode}
\begingroup\expandafter\expandafter\expandafter\endgroup
\expandafter\ifx\csname pdf@filesize\endcsname\relax
  \EmFi@Error{%
    Unsupported pdfTeX version%
  }{%
    At least version 1.30 is necessary. Package loading is aborted.%
  }%
  \expandafter\EmFi@AtEnd
\fi%
%    \end{macrocode}
%
% \subsection{Strings}
%
%    Minimal version of package \xpackage{pdfescape} is 2007/08/27 v1.5
%    because of \cs{EdefSanitize}.
%    \begin{macrocode}
\EmFi@RequirePackage{pdfescape}[2007/11/11]
%    \end{macrocode}
%
%    \begin{macrocode}
\def\EmFi@temp#1{%
  \expandafter\EdefSanitize\csname EmFi@S@#1\endcsname{#1}%
}
%    \end{macrocode}
%    \begin{macro}{\EmFi@details}
%    \begin{macrocode}
\EmFi@temp{details}%
%    \end{macrocode}
%    \end{macro}
%    \begin{macro}{\EmFi@tile}
%    \begin{macrocode}
\EmFi@temp{tile}%
%    \end{macrocode}
%    \end{macro}
%    \begin{macro}{\EmFi@hidden}
%    \begin{macrocode}
\EmFi@temp{hidden}%
%    \end{macrocode}
%    \end{macro}
%    \begin{macro}{\EmFi@S@text}
%    \begin{macrocode}
\EmFi@temp{text}
%    \end{macrocode}
%    \end{macro}
%    \begin{macro}{\EmFi@S@date}
%    \begin{macrocode}
\EmFi@temp{date}
%    \end{macrocode}
%    \end{macro}
%    \begin{macro}{\EmFi@S@number}
%    \begin{macrocode}
\EmFi@temp{number}
%    \end{macrocode}
%    \end{macro}
%    \begin{macro}{\EmFi@S@file}
%    \begin{macrocode}
\EmFi@temp{file}
%    \end{macrocode}
%    \end{macro}
%    \begin{macro}{\EmFi@S@desc}
%    \begin{macrocode}
\EmFi@temp{desc}
%    \end{macrocode}
%    \end{macro}
%    \begin{macro}{\EmFi@S@moddate}
%    \begin{macrocode}
\EmFi@temp{moddate}
%    \end{macrocode}
%    \end{macro}
%    \begin{macro}{\EmFi@S@creationdate}
%    \begin{macrocode}
\EmFi@temp{creationdate}
%    \end{macrocode}
%    \end{macro}
%    \begin{macro}{\EmFi@S@size}
%    \begin{macrocode}
\EmFi@temp{size}
%    \end{macrocode}
%    \end{macro}
%    \begin{macro}{\EmFi@S@ascending}
%    \begin{macrocode}
\EmFi@temp{ascending}
%    \end{macrocode}
%    \end{macro}
%    \begin{macro}{\EmFi@S@descending}
%    \begin{macrocode}
\EmFi@temp{descending}
%    \end{macrocode}
%    \end{macro}
%    \begin{macro}{\EmFi@S@true}
%    \begin{macrocode}
\EmFi@temp{true}
%    \end{macrocode}
%    \end{macro}
%    \begin{macro}{\EmFi@S@false}
%    \begin{macrocode}
\EmFi@temp{false}
%    \end{macrocode}
%    \end{macro}
%
% \subsection{Switches}
%
%    \begin{macro}{\ifEmFi@collection}
%    \begin{macrocode}
\ltx@newif\ifEmFi@collection
%    \end{macrocode}
%    \end{macro}
%    \begin{macro}{\ifEmFi@sort}
%    \begin{macrocode}
\ltx@newif\ifEmFi@sort
%    \end{macrocode}
%    \end{macro}
%    \begin{macro}{\ifEmFi@visible}
%    \begin{macrocode}
\ltx@newif\ifEmFi@visible
%    \end{macrocode}
%    \end{macro}
%    \begin{macro}{\ifEmFi@edit}
%    \begin{macrocode}
\ltx@newif\ifEmFi@edit
%    \end{macrocode}
%    \end{macro}
%    \begin{macro}{\ifEmFi@item}
%    \begin{macrocode}
\ltx@newif\ifEmFi@item
%    \end{macrocode}
%    \end{macro}
%    \begin{macro}{\ifEmFi@finished}
%    \begin{macrocode}
\ltx@newif\ifEmFi@finished
%    \end{macrocode}
%    \end{macro}
%    \begin{macro}{\ifEmFi@id}
%    \begin{macrocode}
\ltx@newif\ifEmFi@id
%    \end{macrocode}
%    \end{macro}
%
% \subsection{Key value definitions}
%
%    \begin{macro}{\EmFi@GlobalKey}
%    \begin{macrocode}
\def\EmFi@GlobalKey#1#2{%
  \global\expandafter\let\csname KV@#1@#2\expandafter\endcsname
                         \csname KV@#1@#2\endcsname
}
%    \end{macrocode}
%    \end{macro}
%    \begin{macro}{\EmFi@GlobalDefaultKey}
%    \begin{macrocode}
\def\EmFi@GlobalDefaultKey#1#2{%
  \EmFi@GlobalKey{#1}{#2}%
  \global\expandafter\let
      \csname KV@#1@#2@default\expandafter\endcsname
      \csname KV@#1@#2@default\endcsname
}
%    \end{macrocode}
%    \end{macro}
%
%    \begin{macro}{\EmFi@DefineKey}
%    \begin{macrocode}
\def\EmFi@DefineKey#1#2{%
  \kv@define@key{EmFi}{#1}{%
    \expandafter\def\csname EmFi@#1\endcsname{##1}%
  }%
  \expandafter\def\csname EmFi@#1\endcsname{#2}%
}
%    \end{macrocode}
%    \end{macro}
%    Subtype of the embedded file (optional).
%    \begin{macrocode}
\EmFi@DefineKey{mimetype}{}
%    \end{macrocode}
%    File specification string.
%    \begin{macrocode}
\EmFi@DefineKey{filespec}{\EmFi@file}
%    \end{macrocode}
%    File specification string in Unicode.
%    \begin{macrocode}
\EmFi@DefineKey{ucfilespec}{}
%    \end{macrocode}
%    File system (optional).
%    \begin{macrocode}
\EmFi@DefineKey{filesystem}{}
%    \end{macrocode}
%    Description (optional).
%    \begin{macrocode}
\EmFi@DefineKey{desc}{}
%    \end{macrocode}
%    Method for converting text to PDF strings.
%    \begin{macrocode}
\EmFi@DefineKey{stringmethod}{%
  \ifx\pdfstringdef\@undefined
    escape%
  \else
    \ifx\pdfstringdef\relax
      escape%
    \else
      psd%
    \fi
  \fi
}
%    \end{macrocode}
%
%    Option \xoption{id} as key for object numbers.
%    \begin{macrocode}
\kv@define@key{EmFi}{id}{%
  \def\EmFi@id{#1}%
  \EmFi@idtrue
}
%    \end{macrocode}
%    \begin{macro}{\EmFi@defobj}
%    \begin{macrocode}
\def\EmFi@defobj#1{%
  \ifEmFi@id
    \expandafter\xdef\csname EmFi@#1@\EmFi@id\endcsname{%
      \the\pdflastobj\ltx@space 0 R%
    }%
  \fi
}
%    \end{macrocode}
%    \end{macro}
%    \begin{macro}{\embedfileifobjectexists}
%    \begin{macrocode}
\def\embedfileifobjectexists#1#2{%
  \expandafter\ifx\csname EmFi@#2@#1\endcsname\relax
    \expandafter\ltx@secondoftwo
  \else
    \expandafter\ltx@firstoftwo
  \fi
}
%    \end{macrocode}
%    \end{macro}
%    \begin{macro}{\embedfilegetobject}
%    \begin{macrocode}
\def\embedfilegetobject#1#2{%
  \embedfileifobjectexists{#1}{#2}{%
    \csname EmFi@#2@#1\endcsname
  }{%
    0 0 R%
  }%
}
%    \end{macrocode}
%    \end{macro}
%
%    Initial view of the collection.
%    \begin{macrocode}
\kv@define@key{EmFi}{view}[]{%
  \EdefSanitize\EmFi@temp{#1}%
  \def\EmFi@next{%
    \global\EmFi@collectiontrue
  }%
  \ifx\EmFi@temp\ltx@empty
    \let\EmFi@view\EmFi@S@details
  \else\ifx\EmFi@temp\EmFi@S@details
    \let\EmFi@view\EmFi@S@details
  \else\ifx\EmFi@temp\EmFi@S@tile
    \let\EmFi@view\EmFi@S@tile
  \else\ifx\EmFi@temp\EmFi@S@hidden
    \let\EmFi@view\EmFi@S@hidden
  \else
    \let\EmFi@next\relax
    \EmFi@Error{%
      Unknown value `\EmFi@temp' for key `view'.\MessageBreak
      Supported values: `details', `tile', `hidden'.%
    }\@ehc
  \fi\fi\fi\fi
  \EmFi@next
}
%    \end{macrocode}
%    \begin{macrocode}
\EmFi@DefineKey{initialfile}{}
%    \end{macrocode}
%
%    \begin{macro}{\embedfilesetup}
%    \begin{macrocode}
\def\embedfilesetup{%
  \ifEmFi@finished
    \def\EmFi@next##1{}%
    \EmFi@Error{%
      \string\embedfilefield\ltx@space after \string\embedfilefinish
    }{%
      The list of embedded files is already written.%
    }%
  \else
    \def\EmFi@next{%
      \kvsetkeys{EmFi}%
    }%
  \fi
  \EmFi@next
}
%    \end{macrocode}
%    \end{macro}
%
%    \begin{macro}{\EmFi@schema}
%    \begin{macrocode}
\def\EmFi@schema{}
%    \end{macrocode}
%    \end{macro}
%    \begin{macro}{\EmFi@order}
%    \begin{macrocode}
\gdef\EmFi@order{0}
%    \end{macrocode}
%    \end{macro}
%    \begin{macro}{\EmFi@@order}
%    \begin{macrocode}
\let\EmFi@@order\relax
%    \end{macrocode}
%    \end{macro}
%    \begin{macro}{\EmFi@fieldlist}
%    \begin{macrocode}
\def\EmFi@fieldlist{}
%    \end{macrocode}
%    \end{macro}
%    \begin{macro}{\EmFi@sortcase}
%    \begin{macrocode}
\def\EmFi@sortcase{0}%
%    \end{macrocode}
%    \end{macro}
%    \begin{macro}{\embedfilefield}
%    \begin{macrocode}
\def\embedfilefield#1#2{%
  \ifEmFi@finished
    \EmFi@Error{%
      \string\embedfilefield\ltx@space after \string\embedfilefinish
    }{%
      The list of embedded files is already written.%
    }%
  \else
    \global\EmFi@collectiontrue
    \EdefSanitize\EmFi@key{#1}%
    \expandafter\ifx\csname KV@EmFi@\EmFi@key.prefix\endcsname\relax
      \begingroup
        \count@=\EmFi@order
        \advance\count@ 1 %
        \xdef\EmFi@order{\the\count@}%
        \let\EmFi@title\EmFi@key
        \let\EmFi@type\EmFi@S@text
        \EmFi@visibletrue
        \EmFi@editfalse
        \kvsetkeys{EmFiFi}{#2}%
        \EmFi@convert\EmFi@title\EmFi@title
        \xdef\EmFi@schema{%
          \EmFi@schema
          /\pdf@escapename{\EmFi@key}<<%
            /Subtype/%
            \ifx\EmFi@type\EmFi@S@date D%
            \else\ifx\EmFi@type\EmFi@S@number N%
            \else\ifx\EmFi@type\EmFi@S@file F%
            \else\ifx\EmFi@type\EmFi@S@desc Desc%
            \else\ifx\EmFi@type\EmFi@S@moddate ModDate%
            \else\ifx\EmFi@type\EmFi@S@creationdate CreationDate%
            \else\ifx\EmFi@type\EmFi@S@size Size%
            \else S%
            \fi\fi\fi\fi\fi\fi\fi
            /N(\EmFi@title)%
            \EmFi@@order{\EmFi@order}%
            \ifEmFi@visible
            \else
              /V false%
            \fi
            \ifEmFi@edit
              /E true%
            \fi
          >>%
        }%
        \let\do\relax
        \xdef\EmFi@fieldlist{%
          \EmFi@fieldlist
          \do{\EmFi@key}%
        }%
        \ifx\EmFi@type\EmFi@S@text
          \kv@define@key{EmFi}{\EmFi@key.value}{%
            \EmFi@itemtrue
            \def\EmFi@temp{##1}%
            \EmFi@convert\EmFi@temp\EmFi@temp
            \expandafter\def\csname EmFi@V@#1%
            \expandafter\endcsname\expandafter{%
              \expandafter(\EmFi@temp)%
            }%
          }%
          \EmFi@GlobalKey{EmFi}{\EmFi@key.value}%
        \else\ifx\EmFi@type\EmFi@S@date
          \kv@define@key{EmFi}{\EmFi@key.value}{%
            \EmFi@itemtrue
            \def\EmFi@temp{##1}%
            \EmFi@convert\EmFi@temp\EmFi@temp
            \expandafter\def\csname EmFi@V@#1%
            \expandafter\endcsname\expandafter{%
              \expandafter(\EmFi@temp)%
            }%
          }%
          \EmFi@GlobalKey{EmFi}{\EmFi@key.value}%
        \else\ifx\EmFi@type\EmFi@S@number
          \kv@define@key{EmFi}{\EmFi@key.value}{%
            \EmFi@itemtrue
            \expandafter\EdefSanitize\csname EmFi@V@#1\endcsname{ ##1}%
          }%
          \EmFi@GlobalKey{EmFi}{\EmFi@key.value}%
        \fi\fi\fi
        \kv@define@key{EmFi}{\EmFi@key.prefix}{%
          \EmFi@itemtrue
          \expandafter\def\csname EmFi@P@#1\endcsname{##1}%
        }%
        \EmFi@GlobalKey{EmFi}{\EmFi@key.prefix}%
        \kv@define@key{EmFiSo}{\EmFi@key}[ascending]{%
          \EdefSanitize\EmFi@temp{##1}%
          \ifx\EmFi@temp\EmFi@S@ascending
            \def\EmFi@temp{true}%
          \else\ifx\EmFi@temp\EmFi@S@descending
            \def\EmFi@temp{false}%
          \else
            \def\EmFi@temp{}%
            \EmFi@Error{%
              Unknown sort order `\EmFi@temp'.\MessageBreak
              Supported values: `\EmFi@S@ascending', %
              `\EmFi@S@descending
            }\@ehc
          \fi\fi
          \ifx\EmFi@temp\ltx@empty
          \else
            \xdef\EmFi@sortkeys{%
              \EmFi@sortkeys
              /\pdf@escapename{#1}%
            }%
            \ifx\EmFi@sortorders\ltx@empty
              \global\let\EmFi@sortorders\EmFi@temp
              \gdef\EmFi@sortcase{1}%
            \else
              \xdef\EmFi@sortorders{%
                \EmFi@sortorders
                \ltx@space
                \EmFi@temp
              }%
              \xdef\EmFi@sortcase{2}%
            \fi
          \fi
        }%
        \EmFi@GlobalDefaultKey{EmFiSo}\EmFi@key
      \endgroup
    \else
      \EmFi@Error{%
        Field `\EmFi@key' is already defined%
      }\@ehc
    \fi
  \fi
}
%    \end{macrocode}
%    \end{macro}
%    \begin{macrocode}
\kv@define@key{EmFiFi}{type}{%
  \EdefSanitize\EmFi@temp{#1}%
  \ifx\EmFi@temp\EmFi@S@text
    \let\EmFi@type\EmFi@temp
  \else\ifx\EmFi@temp\EmFi@S@date
    \let\EmFi@type\EmFi@temp
  \else\ifx\EmFi@temp\EmFi@S@number
    \let\EmFi@type\EmFi@temp
  \else\ifx\EmFi@temp\EmFi@S@file
    \let\EmFi@type\EmFi@temp
  \else\ifx\EmFi@temp\EmFi@S@desc
    \let\EmFi@type\EmFi@temp
  \else\ifx\EmFi@temp\EmFi@S@moddate
    \let\EmFi@type\EmFi@temp
  \else\ifx\EmFi@temp\EmFi@S@creationdate
    \let\EmFi@type\EmFi@temp
  \else\ifx\EmFi@temp\EmFi@S@size
    \let\EmFi@type\EmFi@temp
  \else
    \EmFi@Error{%
      Unknown type `\EmFi@temp'.\MessageBreak
      Supported types: `text', `date', `number', `file',\MessageBreak
      `desc', `moddate', `creationdate', `size'%
    }%
  \fi\fi\fi\fi\fi\fi\fi\fi
}
%    \end{macrocode}
%    \begin{macrocode}
\kv@define@key{EmFiFi}{title}{%
  \def\EmFi@title{#1}%
}
%    \end{macrocode}
%    \begin{macro}{\EmFi@setboolean}
%    \begin{macrocode}
\def\EmFi@setboolean#1#2{%
  \EdefSanitize\EmFi@temp{#2}%
  \ifx\EmFi@temp\EmFi@S@true
    \csname EmFi@#1true\endcsname
  \else
    \ifx\EmFi@temp\EmFi@S@false
      \csname EmFi@#1false\endcsname
    \else
      \EmFi@Error{%
        Unknown value `\EmFi@temp' for key `#1'.\MessageBreak
        Supported values: `true', `false'%
      }\@ehc
    \fi
  \fi
}
%    \end{macrocode}
%    \end{macro}
%    \begin{macrocode}
\kv@define@key{EmFiFi}{visible}[true]{%
  \EmFi@setboolean{visible}{#1}%
}
%    \end{macrocode}
%    \begin{macrocode}
\kv@define@key{EmFiFi}{edit}[true]{%
  \EmFi@setboolean{edit}{#1}%
}
%    \end{macrocode}
%
%    \begin{macro}{\EmFi@sortkeys}
%    \begin{macrocode}
\def\EmFi@sortkeys{}
%    \end{macrocode}
%    \end{macro}
%    \begin{macro}{\EmFi@sortorders}
%    \begin{macrocode}
\def\EmFi@sortorders{}
%    \end{macrocode}
%    \end{macro}
%    \begin{macro}{\embedfilesort}
%    \begin{macrocode}
\def\embedfilesort{%
  \kvsetkeys{EmFiSo}%
}
%    \end{macrocode}
%    \end{macro}
%
% \subsection{Embed the file}
%
%    \begin{macro}{\embedfile}
%    \begin{macrocode}
\def\embedfile{%
  \ltx@ifnextchar[\EmFi@embedfile{\EmFi@embedfile[]}%
}
%    \end{macrocode}
%    \end{macro}
%    \begin{macro}{\EmFi@embedfile}
%    \begin{macrocode}
\def\EmFi@embedfile[#1]#2{%
  \ifEmFi@finished
    \EmFi@Error{%
      \string\embedfile\ltx@space after \string\embedfilefinish
    }{%
      The list of embedded files is already written.%
    }%
  \else
    \begingroup
      \def\EmFi@file{#2}%
      \kvsetkeys{EmFi}{#1}%
      \expandafter\expandafter\expandafter
      \ifx\expandafter\expandafter\expandafter
          \\\pdf@filesize{\EmFi@file}\\%
        \EmFi@Error{%
          File `\EmFi@file' not found%
        }{%
          The unknown file is not embedded.%
        }%
      \else
        \edef\EmFi@@filespec{%
          \pdf@escapestring{\EmFi@filespec}%
        }%
        \ifx\EmFi@ucfilespec\ltx@empty
          \let\EmFi@@ucfilespec\ltx@empty
        \else
          \EmFi@convert\EmFi@ucfilespec\EmFi@@ucfilespec
        \fi
        \ifx\EmFi@desc\ltx@empty
          \let\EmFi@@desc\ltx@empty
        \else
          \EmFi@convert\EmFi@desc\EmFi@@desc
        \fi
        \ifEmFi@item
          \let\do\EmFi@do
          \immediate\pdfobj{%
            <<%
              \EmFi@fieldlist
            >>%
          }%
          \edef\EmFi@ci{\the\pdflastobj}%
        \fi
        \immediate\pdfobj stream attr{%
          /Type/EmbeddedFile%
          \ifx\EmFi@mimetype\ltx@empty
          \else
            /Subtype/\pdf@escapename{\EmFi@mimetype}%
          \fi
          /Params<<%
            /ModDate(\pdf@filemoddate{\EmFi@file})%
            /Size \pdf@filesize{\EmFi@file}%
            /CheckSum<\pdf@filemdfivesum{\EmFi@file}>%
          >>%
        }file{\EmFi@file}\relax
        \EmFi@defobj{EmbeddedFile}%
        \immediate\pdfobj{%
          <<%
            /Type/Filespec%
            \ifx\EmFi@filesystem\ltx@empty
            \else
            /FS/\pdf@escapename{\EmFi@filesystem}%
            \fi
            /F(\EmFi@@filespec)%
            \ifx\EmFi@@ucfilespec\ltx@empty
            \else
              /UF(\EmFi@@ucfilespec)%
            \fi
            \ifx\EmFi@@desc\ltx@empty
            \else
              /Desc(\EmFi@@desc)%
            \fi
            /EF<<%
              /F \the\pdflastobj\ltx@space 0 R%
            >>%
            \ifEmFi@item
              /CI \EmFi@ci\ltx@space 0 R%
            \fi
          >>%
        }%
        \EmFi@defobj{Filespec}%
        \EmFi@add{%
          \EmFi@@filespec
        }{\the\pdflastobj\ltx@space 0 R}%
      \fi
    \endgroup
  \fi
}
%    \end{macrocode}
%    \end{macro}
%    \begin{macro}{\EmFi@do}
%    \begin{macrocode}
\def\EmFi@do#1{%
  \expandafter\ifx\csname EmFi@P@#1\endcsname\relax
    \expandafter\ifx\csname EmFi@V@#1\endcsname\relax
    \else
      /\pdf@escapename{#1}\csname EmFi@V@#1\endcsname
    \fi
  \else
    /\pdf@escapename{#1}<<%
      \expandafter\ifx\csname EmFi@V@#1\endcsname\relax
      \else
        /D\csname EmFi@V@#1\endcsname
      \fi
      /P(\csname EmFi@P@#1\endcsname)%
    >>%
  \fi
}
%    \end{macrocode}
%    \end{macro}
%    \begin{macro}{\EmFi@convert}
%    \begin{macrocode}
\def\EmFi@convert#1#2{%
  \ifnum\pdf@strcmp{\EmFi@stringmethod}{psd}=0 %
    \pdfstringdef\EmFi@temp{#1}%
    \let#2\EmFi@temp
  \else
    \edef#2{\pdf@escapestring{#1}}%
  \fi
}
%    \end{macrocode}
%    \end{macro}
%    \begin{macrocode}
\global\let\EmFi@list\ltx@empty
%    \end{macrocode}
%    \begin{macro}{\EmFi@add}
%    Sorting is done by the insertion sort algorithm.
%    Probably the sorting could be done more reliable.
%    However, the PDF specification is not too clear to me
%    regarding precise sorting rules (how to deal with
%    different encodings, escaped characters, \dots).
%    \begin{macrocode}
\def\EmFi@add#1#2{%
  \begingroup
    \ifx\EmFi@list\ltx@empty
      \xdef\EmFi@list{\noexpand\do{#1}{#2}}%
    \else
      \def\do##1##2{%
        \ifnum\pdf@strcmp{##1}{#1}>0 %
          \edef\x{%
            \toks@{%
              \the\toks@%
              \noexpand\do{#1}{#2}%
              \noexpand\do{##1}{##2}%
            }%
          }%
          \x
          \def\do####1####2{%
            \toks@\expandafter{\the\toks@\do{####1}{####2}}%
          }%
          \def\stop{%
            \xdef\EmFi@list{\the\toks@}%
          }%
        \else
          \toks@\expandafter{\the\toks@\do{##1}{##2}}%
        \fi
      }%
      \def\stop{%
        \xdef\EmFi@list{\the\toks@\noexpand\do{#1}{#2}}%
      }%
      \toks@{}%
      \EmFi@list\stop
    \fi
  \endgroup
}
%    \end{macrocode}
%    \end{macro}
%    \begin{macro}{\embedfilefinish}
%    \begin{macrocode}
\def\embedfilefinish{%
  \ifEmFi@finished
    \EmFi@Error{%
      Too many invocations of \string\embedfilefinish
    }{%
      The list of embedded files is already written.%
    }%
  \else
    \ifx\EmFi@list\ltx@empty
    \else
%    \end{macrocode}
%    Write |/EmbeddedFiles| entry.
%    \begin{macrocode}
      \global\EmFi@finishedtrue
      \begingroup
        \def\do##1##2{%
          (##1)##2%
        }%
        \immediate\pdfobj{%
          <<%
            /Names[\EmFi@list]%
          >>%
        }%
        \pdfnames{%
          /EmbeddedFiles \the\pdflastobj\ltx@space 0 R%
        }%
      \endgroup
%    \end{macrocode}
%    Write collection objects.
%    \begin{macrocode}
      \ifx\EmFi@initialfile\ltx@empty
      \else
        \EmFi@collectiontrue
      \fi
      \ifEmFi@collection
        \ifx\EmFi@initialfile\ltx@empty
          \let\EmFi@@initialfile\ltx@empty
        \else
          \edef\EmFi@@initialfile{%
            \pdf@escapestring{\EmFi@initialfile}%
          }%
        \fi
%    \end{macrocode}
%    Look for initial file among the embedded files.
%    \begin{macrocode}
        \begingroup
          \let\f=N%
          \def\do##1##2{%
            \def\x{##1}%
            \ifx\x\EmFi@@initialfile
              \let\f=Y%
              \let\do\ltx@gobbletwo
            \fi
          }%
          \EmFi@list
        \expandafter\endgroup
        \ifx\f Y%
        \else
          \@PackageWarningNoLine{embedfile}{%
            Missing initial file `\EmFi@initialfile'\MessageBreak
            among the embedded files%
          }%
          \let\EmFi@initialfile\ltx@empty
          \let\EmFi@@initialfile\ltx@empty
        \fi
        \ifcase\EmFi@sortcase
          \def\EmFi@temp{}%
        \or
          \def\EmFi@temp{%
            /S\EmFi@sortkeys
            /A \EmFi@sortorders
          }%
        \else
          \def\EmFi@temp{%
            /S[\EmFi@sortkeys]%
            /A[\EmFi@sortorders]%
          }%
        \fi
        \def\EmFi@@order##1{%
          \ifnum\EmFi@order>1 %
            /O ##1%
          \fi
        }%
        \immediate\pdfobj{%
          <<%
            \ifx\EmFi@schema\ltx@empty
            \else
              /Schema<<\EmFi@schema>>%
            \fi
            \ifx\EmFi@@initialfile\ltx@empty
            \else
              /D(\EmFi@@initialfile)%
            \fi
            \ifx\EmFi@view\EmFi@S@tile
              /View/T%
            \else\ifx\EmFi@view\EmFi@S@hidden
              /View/H%
            \fi\fi
            \ifx\EmFi@temp\ltx@empty
              \EmFi@temp
            \else
              /Sort<<\EmFi@temp>>%
            \fi
          >>%
        }%
        \pdfcatalog{%
          /Collection \the\pdflastobj\ltx@space0 R%
        }%
      \fi
    \fi
  \fi
}
%    \end{macrocode}
%    \end{macro}
%    \begin{macrocode}
\begingroup\expandafter\expandafter\expandafter\endgroup
\expandafter\ifx\csname AtEndDocument\endcsname\relax
\else
  \AtEndDocument{\embedfilefinish}%
\fi
%    \end{macrocode}
%
%    \begin{macrocode}
\EmFi@AtEnd%
%</package>
%    \end{macrocode}
%
% \section{Test}
%
% \subsection{Catcode checks for loading}
%
%    \begin{macrocode}
%<*test1>
%    \end{macrocode}
%    \begin{macrocode}
\catcode`\{=1 %
\catcode`\}=2 %
\catcode`\#=6 %
\catcode`\@=11 %
\expandafter\ifx\csname count@\endcsname\relax
  \countdef\count@=255 %
\fi
\expandafter\ifx\csname @gobble\endcsname\relax
  \long\def\@gobble#1{}%
\fi
\expandafter\ifx\csname @firstofone\endcsname\relax
  \long\def\@firstofone#1{#1}%
\fi
\expandafter\ifx\csname loop\endcsname\relax
  \expandafter\@firstofone
\else
  \expandafter\@gobble
\fi
{%
  \def\loop#1\repeat{%
    \def\body{#1}%
    \iterate
  }%
  \def\iterate{%
    \body
      \let\next\iterate
    \else
      \let\next\relax
    \fi
    \next
  }%
  \let\repeat=\fi
}%
\def\RestoreCatcodes{}
\count@=0 %
\loop
  \edef\RestoreCatcodes{%
    \RestoreCatcodes
    \catcode\the\count@=\the\catcode\count@\relax
  }%
\ifnum\count@<255 %
  \advance\count@ 1 %
\repeat

\def\RangeCatcodeInvalid#1#2{%
  \count@=#1\relax
  \loop
    \catcode\count@=15 %
  \ifnum\count@<#2\relax
    \advance\count@ 1 %
  \repeat
}
\def\RangeCatcodeCheck#1#2#3{%
  \count@=#1\relax
  \loop
    \ifnum#3=\catcode\count@
    \else
      \errmessage{%
        Character \the\count@\space
        with wrong catcode \the\catcode\count@\space
        instead of \number#3%
      }%
    \fi
  \ifnum\count@<#2\relax
    \advance\count@ 1 %
  \repeat
}
\def\space{ }
\expandafter\ifx\csname LoadCommand\endcsname\relax
  \def\LoadCommand{\input embedfile.sty\relax}%
\fi
\def\Test{%
  \RangeCatcodeInvalid{0}{47}%
  \RangeCatcodeInvalid{58}{64}%
  \RangeCatcodeInvalid{91}{96}%
  \RangeCatcodeInvalid{123}{255}%
  \catcode`\@=12 %
  \catcode`\\=0 %
  \catcode`\%=14 %
  \LoadCommand
  \RangeCatcodeCheck{0}{36}{15}%
  \RangeCatcodeCheck{37}{37}{14}%
  \RangeCatcodeCheck{38}{47}{15}%
  \RangeCatcodeCheck{48}{57}{12}%
  \RangeCatcodeCheck{58}{63}{15}%
  \RangeCatcodeCheck{64}{64}{12}%
  \RangeCatcodeCheck{65}{90}{11}%
  \RangeCatcodeCheck{91}{91}{15}%
  \RangeCatcodeCheck{92}{92}{0}%
  \RangeCatcodeCheck{93}{96}{15}%
  \RangeCatcodeCheck{97}{122}{11}%
  \RangeCatcodeCheck{123}{255}{15}%
  \RestoreCatcodes
}
\Test
\csname @@end\endcsname
\end
%    \end{macrocode}
%    \begin{macrocode}
%</test1>
%    \end{macrocode}
%
% \subsection{Simple test}
%
%    \begin{macrocode}
%<*test2>
\input embedfile.sty\relax
\embedfile[%
  stringmethod=escape,%
  mimetype=plain/text,%
  desc={LaTeX docstrip source archive for package `embedfile'},%
  id={embedfile.dtx}%
]{embedfile.dtx}
\nopagenumbers
Test (plain-\TeX): {\tt embedfile.dtx} should be embedded.%

\def\Test#1{%
  \par
  \embedfileifobjectexists{embedfile.dtx}{#1}{%
    Object #1 (embedfile.dtx): %
    \embedfilegetobject{embedfile.dtx}{#1}%
  }{%
    \errmessage{Missing object #1 (embedfile.dtx)}%
  }%
}
\Test{EmbeddedFile}
\Test{Filespec}
\embedfilefinish
\bye
%</test2>
%    \end{macrocode}
%    \begin{macrocode}
%<*test3>
\NeedsTeXFormat{LaTeX2e}
\let\SavedJobname\jobname
\def\jobname{embedfile}
\RequirePackage{dtx-attach}[2011/04/13]
\let\jobname\SavedJobname
\documentclass{minimal}
\begin{document}
  Test (\LaTeX): \texttt{embedfile.dtx} should be embedded.%
\end{document}
%</test3>
%    \end{macrocode}
%
% \subsection{Test for \iniTeX}
%
%    \begin{macrocode}
%<*test4>
\catcode`\{=1 %
\catcode`\}=2 %
\input ifluatex.sty %
\ifluatex
  \directlua{%
    tex.enableprimitives('', {%
      'pdflastobj',%
      'pdfnames',%
      'pdfobj',%
      'pdfoutput'%
    })%
  }%
\fi
\pdfoutput=1 %
\input embedfile.sty %
\shipout\hbox{}
\embedfile[%
  stringmethod=escape,%
  mimetype=plain/text,%
  desc={iniTeX source},%
]{\jobname.tex}
\embedfilefinish
\end
%</test4>
%    \end{macrocode}
%
% \section{Installation}
%
% \subsection{Download}
%
% \paragraph{Package.} This package is available on
% CTAN\footnote{\url{ftp://ftp.ctan.org/tex-archive/}}:
% \begin{description}
% \item[\CTAN{macros/latex/contrib/oberdiek/embedfile.dtx}] The source file.
% \item[\CTAN{macros/latex/contrib/oberdiek/embedfile.pdf}] Documentation.
% \end{description}
%
%
% \paragraph{Bundle.} All the packages of the bundle `oberdiek'
% are also available in a TDS compliant ZIP archive. There
% the packages are already unpacked and the documentation files
% are generated. The files and directories obey the TDS standard.
% \begin{description}
% \item[\CTAN{install/macros/latex/contrib/oberdiek.tds.zip}]
% \end{description}
% \emph{TDS} refers to the standard ``A Directory Structure
% for \TeX\ Files'' (\CTAN{tds/tds.pdf}). Directories
% with \xfile{texmf} in their name are usually organized this way.
%
% \subsection{Bundle installation}
%
% \paragraph{Unpacking.} Unpack the \xfile{oberdiek.tds.zip} in the
% TDS tree (also known as \xfile{texmf} tree) of your choice.
% Example (linux):
% \begin{quote}
%   |unzip oberdiek.tds.zip -d ~/texmf|
% \end{quote}
%
% \paragraph{Script installation.}
% Check the directory \xfile{TDS:scripts/oberdiek/} for
% scripts that need further installation steps.
% Package \xpackage{attachfile2} comes with the Perl script
% \xfile{pdfatfi.pl} that should be installed in such a way
% that it can be called as \texttt{pdfatfi}.
% Example (linux):
% \begin{quote}
%   |chmod +x scripts/oberdiek/pdfatfi.pl|\\
%   |cp scripts/oberdiek/pdfatfi.pl /usr/local/bin/|
% \end{quote}
%
% \subsection{Package installation}
%
% \paragraph{Unpacking.} The \xfile{.dtx} file is a self-extracting
% \docstrip\ archive. The files are extracted by running the
% \xfile{.dtx} through \plainTeX:
% \begin{quote}
%   \verb|tex embedfile.dtx|
% \end{quote}
%
% \paragraph{TDS.} Now the different files must be moved into
% the different directories in your installation TDS tree
% (also known as \xfile{texmf} tree):
% \begin{quote}
% \def\t{^^A
% \begin{tabular}{@{}>{\ttfamily}l@{ $\rightarrow$ }>{\ttfamily}l@{}}
%   embedfile.sty & tex/generic/oberdiek/embedfile.sty\\
%   dtx-attach.sty & tex/generic/oberdiek/dtx-attach.sty\\
%   embedfile.pdf & doc/latex/oberdiek/embedfile.pdf\\
%   embedfile-example-plain.tex & doc/latex/oberdiek/embedfile-example-plain.tex\\
%   embedfile-example-collection.tex & doc/latex/oberdiek/embedfile-example-collection.tex\\
%   test/embedfile-test1.tex & doc/latex/oberdiek/test/embedfile-test1.tex\\
%   test/embedfile-test2.tex & doc/latex/oberdiek/test/embedfile-test2.tex\\
%   test/embedfile-test3.tex & doc/latex/oberdiek/test/embedfile-test3.tex\\
%   test/embedfile-test4.tex & doc/latex/oberdiek/test/embedfile-test4.tex\\
%   embedfile.dtx & source/latex/oberdiek/embedfile.dtx\\
% \end{tabular}^^A
% }^^A
% \sbox0{\t}^^A
% \ifdim\wd0>\linewidth
%   \begingroup
%     \advance\linewidth by\leftmargin
%     \advance\linewidth by\rightmargin
%   \edef\x{\endgroup
%     \def\noexpand\lw{\the\linewidth}^^A
%   }\x
%   \def\lwbox{^^A
%     \leavevmode
%     \hbox to \linewidth{^^A
%       \kern-\leftmargin\relax
%       \hss
%       \usebox0
%       \hss
%       \kern-\rightmargin\relax
%     }^^A
%   }^^A
%   \ifdim\wd0>\lw
%     \sbox0{\small\t}^^A
%     \ifdim\wd0>\linewidth
%       \ifdim\wd0>\lw
%         \sbox0{\footnotesize\t}^^A
%         \ifdim\wd0>\linewidth
%           \ifdim\wd0>\lw
%             \sbox0{\scriptsize\t}^^A
%             \ifdim\wd0>\linewidth
%               \ifdim\wd0>\lw
%                 \sbox0{\tiny\t}^^A
%                 \ifdim\wd0>\linewidth
%                   \lwbox
%                 \else
%                   \usebox0
%                 \fi
%               \else
%                 \lwbox
%               \fi
%             \else
%               \usebox0
%             \fi
%           \else
%             \lwbox
%           \fi
%         \else
%           \usebox0
%         \fi
%       \else
%         \lwbox
%       \fi
%     \else
%       \usebox0
%     \fi
%   \else
%     \lwbox
%   \fi
% \else
%   \usebox0
% \fi
% \end{quote}
% If you have a \xfile{docstrip.cfg} that configures and enables \docstrip's
% TDS installing feature, then some files can already be in the right
% place, see the documentation of \docstrip.
%
% \subsection{Refresh file name databases}
%
% If your \TeX~distribution
% (\teTeX, \mikTeX, \dots) relies on file name databases, you must refresh
% these. For example, \teTeX\ users run \verb|texhash| or
% \verb|mktexlsr|.
%
% \subsection{Some details for the interested}
%
% \paragraph{Attached source.}
%
% The PDF documentation on CTAN also includes the
% \xfile{.dtx} source file. It can be extracted by
% AcrobatReader 6 or higher. Another option is \textsf{pdftk},
% e.g. unpack the file into the current directory:
% \begin{quote}
%   \verb|pdftk embedfile.pdf unpack_files output .|
% \end{quote}
%
% \paragraph{Unpacking with \LaTeX.}
% The \xfile{.dtx} chooses its action depending on the format:
% \begin{description}
% \item[\plainTeX:] Run \docstrip\ and extract the files.
% \item[\LaTeX:] Generate the documentation.
% \end{description}
% If you insist on using \LaTeX\ for \docstrip\ (really,
% \docstrip\ does not need \LaTeX), then inform the autodetect routine
% about your intention:
% \begin{quote}
%   \verb|latex \let\install=y\input{embedfile.dtx}|
% \end{quote}
% Do not forget to quote the argument according to the demands
% of your shell.
%
% \paragraph{Generating the documentation.}
% You can use both the \xfile{.dtx} or the \xfile{.drv} to generate
% the documentation. The process can be configured by the
% configuration file \xfile{ltxdoc.cfg}. For instance, put this
% line into this file, if you want to have A4 as paper format:
% \begin{quote}
%   \verb|\PassOptionsToClass{a4paper}{article}|
% \end{quote}
% An example follows how to generate the
% documentation with pdf\LaTeX:
% \begin{quote}
%\begin{verbatim}
%pdflatex embedfile.dtx
%makeindex -s gind.ist embedfile.idx
%pdflatex embedfile.dtx
%makeindex -s gind.ist embedfile.idx
%pdflatex embedfile.dtx
%\end{verbatim}
% \end{quote}
%
% \section{Catalogue}
%
% The following XML file can be used as source for the
% \href{http://mirror.ctan.org/help/Catalogue/catalogue.html}{\TeX\ Catalogue}.
% The elements \texttt{caption} and \texttt{description} are imported
% from the original XML file from the Catalogue.
% The name of the XML file in the Catalogue is \xfile{embedfile.xml}.
%    \begin{macrocode}
%<*catalogue>
<?xml version='1.0' encoding='us-ascii'?>
<!DOCTYPE entry SYSTEM 'catalogue.dtd'>
<entry datestamp='$Date$' modifier='$Author$' id='embedfile'>
  <name>embedfile</name>
  <caption>Embed files into PDF.</caption>
  <authorref id='auth:oberdiek'/>
  <copyright owner='Heiko Oberdiek' year='2006-2011'/>
  <license type='lppl1.3'/>
  <version number='2.6'/>
  <description>
    This package embeds files in a PDF document, using the PDF
    format's embedding operation (note the contrast with the attach
    operation used by the <xref refid='attachfile'>attachfile</xref>
    and <xref refid='attachfile2'>attachfile2</xref> packages).
    Currently only <xref refid='pdftex'>pdfTeX</xref> &gt;=1.30, in
    PDF mode, is supported.

    <p/>
    The package is part of the <xref refid='oberdiek'>oberdiek</xref>
    bundle.
  </description>
  <documentation details='Package documentation'
      href='ctan:/macros/latex/contrib/oberdiek/embedfile.pdf'/>
  <ctan file='true' path='/macros/latex/contrib/oberdiek/embedfile.dtx'/>
  <miktex location='oberdiek'/>
  <texlive location='oberdiek'/>
  <install path='/macros/latex/contrib/oberdiek/oberdiek.tds.zip'/>
</entry>
%</catalogue>
%    \end{macrocode}
%
% \begin{thebibliography}{9}
% \bibitem{attachfile}
%   Scott Pakin: \textit{The \xpackage{attachfile} package};
%   2005/02/20 v1.2;
%   \CTAN{macros/latex/contrib/attachfile/}.
%
% \bibitem{attachfile2}
%   Heiko Oberdiek: \textit{The \xpackage{attachfile2} package};
%   2006/08/16 v2.2;
%   \CTAN{macros/latex/contrib/oberdiek/attachfile2.pdf}.
%
% \bibitem{pdfspec}
%   Adobe Systems Incorporated:
%   \href{http://www.adobe.com/devnet/acrobat/pdfs/pdf_reference.pdf}%
%       {\textit{PDF Reference, Sixth Edition, Version 1.7}},%
%   Oktober 2006;
%   \url{http://www.adobe.com/devnet/pdf/pdf_reference.html}.
%
% \bibitem{mime}
%   Network Working Group: RFC 2046, \textit{Multipurpose Internet Mail Extensions (MIME) Part Two:
%     Media Types}, November 1996; \url{http://www.rfc-editor.org/}.
%
% \bibitem{types}
%   IANA (Internet Assigned Numbers Authority):
%   \textit{MIME Media Types}, May 2006;
%   \url{http://www.iana.org/assignments/media-types/}.
%
% \end{thebibliography}
%
% \begin{History}
%   \begin{Version}{2006/08/16 v1.0}
%   \item
%     First public version.
%   \end{Version}
%   \begin{Version}{2007/04/11 v1.1}
%   \item
%     Line ends sanitized.
%   \end{Version}
%   \begin{Version}{2007/09/09 v1.2}
%   \item
%     Fixes for plain-TeX, wrapper for package \xpackage{keyval} added.
%   \item
%     Catcode section rewritten.
%   \end{Version}
%   \begin{Version}{2007/10/28 v2.0}
%   \item
%     Collection support added (PDF 1.7).
%   \end{Version}
%   \begin{Version}{2007/10/29 v2.1}
%   \item
%     Export of object references by adding new option \xoption{id} and
%     new macros \cs{embedfileifobjectexists} and \cs{embedfilegetobject}.
%   \end{Version}
%   \begin{Version}{2007/11/11 v2.2}
%   \item
%     Use of package \xpackage{pdftexcmds} for \LuaTeX\ support.
%   \end{Version}
%   \begin{Version}{2007/11/25 v2.3}
%   \item
%     Fix in use of \cs{pdf@filesize}, bug introduced in previous version.
%   \end{Version}
%   \begin{Version}{2009/09/25 v2.4}
%   \item
%     Bug fix: If \xpackage{hyperref} is used with option \xoption{unicode},
%     the Unicode encoded file name causes trouble. Therefore \cs{pdfstringdef}
%     is now never used for option \xoption{filespec}, always method
%     \texttt{escape} is applied (Peter Cibulka).
%   \item
%     Bug fix for \xoption{initialfile}.
%   \item
%     Bug fix for file names in \texttt{/EmbeddedFiles}.
%   \item
%     New option \xoption{ucfilespec} for file name support in Unicode
%     (since PDF 1.7).
%   \end{Version}
%   \begin{Version}{2010/03/01 v2.5}
%   \item
%     Compatibility for \iniTeX.
%   \item
%     Package \xpackage{keyval} replaced by packages \xpackage{kvsetkeys}
%     and \xpackage{kvdefinekeys} because of compatibility for \iniTeX.
%   \item
%     TDS location moved from TDS:tex/latex/oberdiek/embedfile.sty to
%     TDS:tex/generic/oberdiek/embedfile.sty.
%   \end{Version}
%   \begin{Version}{2011/04/13 v2.6}
%   \item
%     Docu fixes (thanks Hans-Martin M\"unch).
%   \end{Version}
% \end{History}
%
% \PrintIndex
%
% \Finale
\endinput

%        (quote the arguments according to the demands of your shell)
%
% Documentation:
%    (a) If embedfile.drv is present:
%           latex embedfile.drv
%    (b) Without embedfile.drv:
%           latex embedfile.dtx; ...
%    The class ltxdoc loads the configuration file ltxdoc.cfg
%    if available. Here you can specify further options, e.g.
%    use A4 as paper format:
%       \PassOptionsToClass{a4paper}{article}
%
%    Programm calls to get the documentation (example):
%       pdflatex embedfile.dtx
%       makeindex -s gind.ist embedfile.idx
%       pdflatex embedfile.dtx
%       makeindex -s gind.ist embedfile.idx
%       pdflatex embedfile.dtx
%
% Installation:
%    TDS:tex/generic/oberdiek/embedfile.sty
%    TDS:tex/generic/oberdiek/dtx-attach.sty
%    TDS:doc/latex/oberdiek/embedfile.pdf
%    TDS:doc/latex/oberdiek/embedfile-example-plain.tex
%    TDS:doc/latex/oberdiek/embedfile-example-collection.tex
%    TDS:doc/latex/oberdiek/test/embedfile-test1.tex
%    TDS:doc/latex/oberdiek/test/embedfile-test2.tex
%    TDS:doc/latex/oberdiek/test/embedfile-test3.tex
%    TDS:doc/latex/oberdiek/test/embedfile-test4.tex
%    TDS:source/latex/oberdiek/embedfile.dtx
%
%<*ignore>
\begingroup
  \catcode123=1 %
  \catcode125=2 %
  \def\x{LaTeX2e}%
\expandafter\endgroup
\ifcase 0\ifx\install y1\fi\expandafter
         \ifx\csname processbatchFile\endcsname\relax\else1\fi
         \ifx\fmtname\x\else 1\fi\relax
\else\csname fi\endcsname
%</ignore>
%<*install>
\input docstrip.tex
\Msg{************************************************************************}
\Msg{* Installation}
\Msg{* Package: embedfile 2011/04/13 v2.6 Embed files into PDF (HO)}
\Msg{************************************************************************}

\keepsilent
\askforoverwritefalse

\let\MetaPrefix\relax
\preamble

This is a generated file.

Project: embedfile
Version: 2011/04/13 v2.6

Copyright (C) 2006-2011 by
   Heiko Oberdiek <heiko.oberdiek at googlemail.com>

This work may be distributed and/or modified under the
conditions of the LaTeX Project Public License, either
version 1.3c of this license or (at your option) any later
version. This version of this license is in
   http://www.latex-project.org/lppl/lppl-1-3c.txt
and the latest version of this license is in
   http://www.latex-project.org/lppl.txt
and version 1.3 or later is part of all distributions of
LaTeX version 2005/12/01 or later.

This work has the LPPL maintenance status "maintained".

This Current Maintainer of this work is Heiko Oberdiek.

The Base Interpreter refers to any `TeX-Format',
because some files are installed in TDS:tex/generic//.

This work consists of the main source file embedfile.dtx
and the derived files
   embedfile.sty, embedfile.pdf, embedfile.ins, embedfile.drv,
   dtx-attach.sty, embedfile-example-plain.tex,
   embedfile-example-collection.tex, embedfile-test1.tex,
   embedfile-test2.tex, embedfile-test3.tex,
   embedfile-test4.tex.

\endpreamble
\let\MetaPrefix\DoubleperCent

\generate{%
  \file{embedfile.ins}{\from{embedfile.dtx}{install}}%
  \file{embedfile.drv}{\from{embedfile.dtx}{driver}}%
  \usedir{tex/generic/oberdiek}%
  \file{embedfile.sty}{\from{embedfile.dtx}{package}}%
  \usedir{tex/latex/oberdiek}%
  \file{dtx-attach.sty}{\from{embedfile.dtx}{dtxattach}}%
  \usedir{doc/latex/oberdiek}%
  \file{embedfile-example-plain.tex}{\from{embedfile.dtx}{exampleplain}}%
  \file{embedfile-example-collection.tex}{\from{embedfile.dtx}{examplecollection}}%
  \usedir{doc/latex/oberdiek/test}%
  \file{embedfile-test1.tex}{\from{embedfile.dtx}{test1}}%
  \file{embedfile-test2.tex}{\from{embedfile.dtx}{test2}}%
  \file{embedfile-test3.tex}{\from{embedfile.dtx}{test3}}%
  \file{embedfile-test4.tex}{\from{embedfile.dtx}{test4}}%
  \nopreamble
  \nopostamble
  \usedir{source/latex/oberdiek/catalogue}%
  \file{embedfile.xml}{\from{embedfile.dtx}{catalogue}}%
}

\catcode32=13\relax% active space
\let =\space%
\Msg{************************************************************************}
\Msg{*}
\Msg{* To finish the installation you have to move the following}
\Msg{* files into a directory searched by TeX:}
\Msg{*}
\Msg{*     embedfile.sty, dtx-attach.sty}
\Msg{*}
\Msg{* To produce the documentation run the file `embedfile.drv'}
\Msg{* through LaTeX.}
\Msg{*}
\Msg{* Happy TeXing!}
\Msg{*}
\Msg{************************************************************************}

\endbatchfile
%</install>
%<*ignore>
\fi
%</ignore>
%<*driver>
\NeedsTeXFormat{LaTeX2e}
\ProvidesFile{embedfile.drv}%
  [2011/04/13 v2.6 Embed files into PDF (HO)]%
\documentclass{ltxdoc}
\usepackage{holtxdoc}[2011/11/22]
\begin{document}
  \DocInput{embedfile.dtx}%
\end{document}
%</driver>
% \fi
%
% \CheckSum{1363}
%
% \CharacterTable
%  {Upper-case    \A\B\C\D\E\F\G\H\I\J\K\L\M\N\O\P\Q\R\S\T\U\V\W\X\Y\Z
%   Lower-case    \a\b\c\d\e\f\g\h\i\j\k\l\m\n\o\p\q\r\s\t\u\v\w\x\y\z
%   Digits        \0\1\2\3\4\5\6\7\8\9
%   Exclamation   \!     Double quote  \"     Hash (number) \#
%   Dollar        \$     Percent       \%     Ampersand     \&
%   Acute accent  \'     Left paren    \(     Right paren   \)
%   Asterisk      \*     Plus          \+     Comma         \,
%   Minus         \-     Point         \.     Solidus       \/
%   Colon         \:     Semicolon     \;     Less than     \<
%   Equals        \=     Greater than  \>     Question mark \?
%   Commercial at \@     Left bracket  \[     Backslash     \\
%   Right bracket \]     Circumflex    \^     Underscore    \_
%   Grave accent  \`     Left brace    \{     Vertical bar  \|
%   Right brace   \}     Tilde         \~}
%
% \GetFileInfo{embedfile.drv}
%
% \title{The \xpackage{embedfile} package}
% \date{2011/04/13 v2.6}
% \author{Heiko Oberdiek\\\xemail{heiko.oberdiek at googlemail.com}}
%
% \maketitle
%
% \begin{abstract}
% This package embeds files to a PDF document.
% Currently the only supported driver is \pdfTeX\ $>=$ 1.30 in PDF mode.
% \end{abstract}
%
% \tableofcontents
%
% \section{Documentation}
%
% \subsection{Introduction}
%
%    The PDF format (\cite{pdfspec}) allows the inclusion of files
%    inside the PDF document. The included files can be bound to an
%    annotation on a page. Or they can be recorded in a sorted
%    list of embedded files. The packages \xpackage{attachfile}
%    or \xpackage{attachfile2} follow the first approach, this
%    package uses the latter method.
%
% \subsubsection{Future development}
%
%    My dream is a large package that merges the features of
%    all these packages meantioned before:
%    \begin{itemize}
%    \item Files can be attached to a page.
%    \item Files can be attached to the document.
%    \item An easy user interface for simple, common tasks and
%          beginners.
%    \item An interface for the advanced users that want to setup
%          every detail.
%    \item Support of many drivers (pdftex, dvips, dvipdfm, \dots).
%    \item \dots
%    \end{itemize}
%    However, I have not managed to take the time for this project.
%    Instead:
%    \begin{itemize}
%    \item First I experimented with package \xpackage{attachfile},
%          adding driver support, fixing bugs, \dots. The result is
%          currently named as \xpackage{attachfile2}. It uses an external
%          script to get file properties (size, date, checksum, \dots).
%    \item In order to avoid an external program for getting basic
%          file properties I provided a patch ``EscapeAndOther'' for
%          pdfTeX that was accepted for version 1.30.
%    \item This package closes a gap left by the packages for attaching
%          files and allows the embedding of files to the document.
%          Also it makes use of the new primitives of \pdfTeX.
%    \end{itemize}
%
% \subsection{User interface}
%
%    This package \xpackage{embedfile} can be used with both
%    \LaTeX\ and \plainTeX. See \autoref{sec:plain} that
%    explains the use with \plainTeX\ by an example.
%    In \LaTeX\ the package is loaded as usually. There are no options.
%    \begin{quote}
%      \verb|\usepackage{embedfile}|
%    \end{quote}
%
%    \begin{declcs}{embedfile}
%      |[|\meta{options}|]| \M{file}
%    \end{declcs}
%    The macro \cs{embedfile} includes file \meta{file} and
%    attaches it to the PDF document. At the end of the document
%    the sorted list of embedded files are written. Thus you can
%    safely use \cs{embedfile} before \verb|\end{document}|.
%    Embedding files using \cs{AtEndDocument} will only work,
%    if \cs{AtEndDocument} is called before loading the package
%    \xpackage{embedfile}.
%
%    The \meta{options} are give as key value pairs. The following
%    keys are supported:
%    \begin{description}
%    \item[\xoption{filespec}]
%      This allows to override the file name
%      that appears in the PDF file. If you are using other than
%      simple file names (8-bit, path separators, \dots), look into
%      the PDF specification (\cite{pdfspec}).
%      There are rules how these file names must be written/encoded.
%      Avoid 8-bit characters
%      and other special characters, the behaviour is currently undefined.
%      Use option \xoption{ucfilespec} for more funny file names.
%      The string method, see below, is |escape| since version 2.4.
%
%      This name is also used as entry in a name tree
%      (see PDF specification: |/EmbeddedFiles|). Therefore the
%      value for \xoption{filespec} must be unique among all
%      embedded files. Also key \xoption{initialfiles} refers
%      to this name, if the file name and the value of \xoption{filespec}
%      are different.
%    \item[\xoption{ucfilespec}]
%      Since PDF 1.7 the file name may be provided in Unicode.
%      The conversion of the option value into a PDF string
%      is controlled by option \xoption{stringmethod}.
%    \item[\xoption{filesystem}]
%      This sets the entry \verb|/FS| in
%      the file specification dictionary, see PDF specification
%      (\cite{pdfspec}).
%      Example: \verb|filesystem=URL|.
%    \item[\xoption{mimetype}]
%      This sets the mime type (\cite{mime}) of the file,
%      see \autoref{sec:plain} for examples and \cite{types}
%      for a list of officially registered types.
%    \item[\xoption{desc}]
%      The description for the file.
%    \item[\xoption{stringmethod}]
%      The package must convert the values of the keys \xoption{ucfilespec}
%      and \xoption{desc} into a PDF string (before version 2.4: \xoption{filespec}
%      and \xoption{desc}). If \xpackage{hyperref}
%      is found, then its \cs{pdfstringdef} will be used, otherwise
%      \pdfTeX's \cs{pdfescapestring} is used. Value \verb|psd|
%      forces the use of \cs{pdfstringdef}, value \verb|escape|
%      the use of \cs{pdfescapestring}.
%    \item[\xoption{\meta{key}.value}] Sets the value of a collection
%      item property, see section \ref{sec:collection}.
%    \item[\xoption{\meta{key}.prefix}]
%      Sets the prefix of a collection item property,
%      see section \ref{sec:collection}.
%    \item[\xoption{id}]
%      The value must be an unique name. Macros
%      \cs{embedfileifobjectexists} and \cs{embedfilegetobject}
%      are using this name later.
%    \end{description}
%
%    \begin{declcs}{embedfilefinish}
%    \end{declcs}
%    The list of all embedded files must be added as data structure
%    in the PDF file. In case of \LaTeX\ this is automatically done.
%    The package uses \cs{AtEndDocument}. Then the list of all
%    files should be known. However, \plainTeX\ does not know
%    about \cs{AtEndDocument}. Thus the user must call
%    \cs{embedfilefinish} at the end of the document
%    after the last file is embedded.
%
%    \begin{declcs}{embedfilesetup} \M{options}
%    \end{declcs}
%    Options for \cs{embedfile} and collection support
%    can be set in \cs{embedfilesetup}.
%
% \subsection{Collection support (PDF 1.7)}
% \label{sec:collection}
%
%    Since PDF 1.7 the embedded files can form a \emph{collection}
%    (sometimes referred as \emph{package}), the main document
%    is called \emph{cover sheet}. See PDF specification
%    8.2.4 ``Collections'' and 3.10.5 ``Collection items'' \cite{pdfspec}.
%
%    Usually Acrobat Reader 7 or 8 shows the
%    embedded files in a table at the bottom with the following columns:
%    \begin{quote}
%    \begin{tabular}{@{}l|l|l|l@{}}
%       Name & Description & Modified & Size\\
%       \hline
%       \dots & \dots & \dots & \dots\\
%    \end{tabular}
%    \end{quote}
%    Acrobat Reader 10 shows the embeded files in the left panel
%    and adds a new column for the compressed size.
%
%    If the files form a collection, then they are displayed
%    in a table left or top (depending on option \xoption{view},
%    see \cs{embedfilesetup}).
%
%
%    Collection support is enabled automatically, if
%    it is used.
%
%    \begin{declcs}{embedfilesetup} \M{options}
%    \end{declcs}
%    The following options are supported in addition
%    to options for \cs{embedfile}:
%    \begin{description}
%    \item[\xoption{view}]
%      If the PDF file contains a collection, then Acrobat Reader 8
%      shows a line at the top below the menu bar and the toolbar.
%      It shows the current selected file, icons for changing the
%      view mode, an options menu.
%      The initial mode how the collection is presented is
%      set by this option \xoption{view}.
%      The following modes/values are supported, the default is |details|:
%        \begin{description}
%        \item[\texttt{details}] The full collection table is displayed
%          at the top below the collection bar.
%        \item[\texttt{tile}] The files of the collection are shown
%          in tile mode on the left.
%        \item[\texttt{hidden}] The collection table is not shown.
%        \end{description}
%    \item[\xoption{initialfile}]
%      Selects the file that is initially presented. Especially
%      useful for an embedded PDF file that is then shown instead
%      of the cover document. There must be an \cs{embedfile} command
%      somewhere whose value for key \xoption{filespec} is used here.
%      The \cs{embedfile} command can drop option \xoption{filespec}
%      if the file name is not different.
%    \end{description}
%
%    \begin{declcs}{embedfilefield} \M{key} \M{options}
%    \end{declcs}
%    Macro \cs{embedfilefield} defines a column/field in the collection table.
%    The name of the field is \meta{key}.
%    \begin{description}
%    \item[\xoption{type}]
%      sets the type of the field. The supported values are:
%      \begin{description}
%      \item[\texttt{text}] A text field. Its value is set in
%        \cs{embedfile} by option \xoption{\meta{key}.value}.
%      \item[\texttt{date}] A date field. Its value is set in
%        \cs{embedfile} by option \xoption{\meta{key}.value}.
%        A special format is required, see ``3.8.3 Dates'' \cite{pdfspec}.
%      \item[\texttt{number}] A field with an integer or float number.
%        Its value is set in \cs{embedfile} by option
%        \xoption{\meta{key}.value}.
%      \item[\texttt{file}] The file name of the embedded file.
%      \item[\texttt{desc}] The description text of the embedded file. It is
%       set in \cs{embedfile} by option \xoption{desc}.
%      \item[\texttt{moddate}] The modification date of the embedded file.
%      \item[\texttt{size}] The size of the embedded file.
%      \end{description}
%      All types allow the use of a prefix that is disregarded by
%      sorting. The prefix for this field is set in \cs{embedfile} by
%      option \xoption{\meta{key}.prefix}.
%    \item[\xoption{title}] sets the column title.
%    \item[\xoption{visible}] controls whether the column is presented:
%      \begin{description}
%      \item[\texttt{true}] shows the column.
%      \item[\texttt{false}] hides the column.
%      \end{description}
%      Default: \texttt{true}
%    \item[\xoption{edit}]
%      Allows the editing of field values. Does not seem to have an
%      effect for Acrobat Reader.
%      \begin{description}
%      \item[\texttt{true}] enables the feature, if available (depends
%        on the PDF viewer).
%      \item[\texttt{false}] disables the feature.
%      \end{description}
%      Default: \texttt{false}
%    \end{description}
%    The order of \cs{embedfilefield} statements defines the order
%    of the columns.
%
%    \begin{declcs}{embedfilesort} \M{key-sort-list}
%    \end{declcs}
%    The sort order of the embedded files are controlled by
%    macro \cs{embedfilesort}. \meta{key-sort-list} defines
%    the sort order. The key is a field name defined by \cs{embedfilefield}.
%    Its value is either \texttt{ascending} or \texttt{descending}.
%    The default is \texttt{ascending}.
%
% \subsection{Export of object references}
%
%    Caution: This feature is still experimental. It may be even removed
%    in future versions. Therefore feedback would be nice, if someone
%    has a useful application for this feature.
%
%    Object numbers are saved, if \xoption{id} is given in \cs{embedfile}.
%    The following objects are supported:
%    \begin{itemize}
%    \item \texttt{EmbeddedFile}
%    \item \texttt{Filespec}
%    \end{itemize}
%
%    \begin{declcs}{embedfileifobjectexists} \M{id} \M{type}
%      \M{then} \M{else}
%    \end{declcs}
%    Macro \cs{embedfileifobjectexists} tests whether object of \meta{type}
%    is available for the embedded file identified by \meta{id}.
%
%    \begin{declcs}{embedfilegetobject} \M{id} \M{type}
%    \end{declcs}
%    Macro \cs{embedfilegetobject} expands to the full object reference
%    object of \meta{type} for the embedded file identified by \meta{id}.
%
% \subsubsection{Example}
%
%\begin{quote}
%\begin{verbatim}
%\embedfile[id={foo}]{foo.pdf}
%\embedfileifobjectexists{foo}{Filespec}{%
%  \typeout{%
%    FileSpec object for `foo': %
%    \embedfilegetobject{foo}{Filespec}%
%  }%
%}{%
%  \typeout{No Filespec object for `foo'}%
%}
%\end{verbatim}
%\end{quote}
%
% \subsection{Examples}
% \label{sec:examples}
%
% \subsubsection{\plainTeX}
% \label{sec:plain}
%
%    The package can be used with \plainTeX. It can be used
%    with or without help from \xfile{miniltx.tex}.
%
%    If additionally package \xpackage{keyval} (\xpackage{graphicx})
%    is needed, load it first. Then package \xpackage{embedfile} avoids
%    a duplicate loading of package \xpackage{keyval}.
%
%    Because \plainTeX\ does not provide a hook at end of the document,
%    you have to call \cs{embedfilefinish} manually at the end
%    after the last embedded file.
%    \begin{macrocode}
%<*exampleplain>
%<<END
% Load packages
\input miniltx
% \def\Gin@driver{pdftex.def}
% \input graphicx.sty
\input embedfile.sty
\resetatcatcode

% default setting
\embedfilesetup{
  mimetype=text/plain
}

% Embed files
\embedfile[
  filespec=example.tex,
  desc={Source code (plain-TeX) of this example}
]{embedfile-example-plain.tex}

\embedfile[
  desc={Source of package `embedfile'}
]{embedfile.dtx}

\embedfile[
  mimetype=application/pdf,
  desc={Documentation of package `embedfile'}
]{embedfile.pdf}

% Some text
This example document contains three embedded files.

% End of document
\embedfilefinish % don't forget
\bye
%END
%</exampleplain>
%    \end{macrocode}
%
% \subsubsection{Collection example}
%
%    \begin{macrocode}
%<*examplecollection>
%<<END
\NeedsTeXFormat{LaTeX2e}
\documentclass{article}
\usepackage[bookmarks=false]{hyperref}
% provides \pdfstringdef that is then used by `title' and
% other keys.
\usepackage{embedfile}[2011/04/13]
\embedfilesetup{
  view=details,
  initialfile=embedfile.pdf
}
\embedfilefield{file}{
  type=file,
  title={File name}
}
\embedfilefield{description}{
  type=desc,
  title={Description}
}
\embedfilefield{date}{
  type=moddate,
  title={Date}
}
\embedfilefield{size}{
  type=size,
  title={Size}
}
\embedfilefield{type}{
  type=text,
  title={Type},
  visible=false
}
\embedfilesort{
  type,
  date=descending
}
\begin{document}
An example for embedded files as collection.
You need Acrobat Reader 8 or higher.

\embedfile[
  desc={Source file of package `embedfile'},
  description.prefix={Package: },
  type.value={DTX}
]{embedfile.dtx}

\embedfile[
  desc={Documentation of package `embedfile'},
  description.prefix={Package: },
  type.value={PDF}
]{embedfile.pdf}

\embedfile[
  desc={The source for this example},
  description.prefix={Example: },
  type.value={TEX}
]{\jobname.tex}

\end{document}
%END
%</examplecollection>
%    \end{macrocode}
%
% \subsection{Package \xpackage{dtx-attach}}
%
% Package \xpackage{dtx-attach} is just a small application of
% package \xpackage{embedfile}. I am using it for the CTAN
% documentation of my packages in \CTAN{macros/latex/contrib/oberdiek/}.
% It also serves as small example for the use of the package with \LaTeX.
%
%    \begin{macrocode}
%<*dtxattach>
\NeedsTeXFormat{LaTeX2e}
\ProvidesPackage{dtx-attach}
  [2011/04/13 v2.6 Embed \string\jobname.dtx (HO)]%
\RequirePackage{embedfile}[2011/04/13]
\embedfile[%
  stringmethod=escape,%
  mimetype=plain/text,%
  desc={LaTeX docstrip source archive for package `\jobname'}%
]{\jobname.dtx}
%</dtxattach>
%    \end{macrocode}
%
% \StopEventually{
% }
%
% \section{Implementation}
%
%    \begin{macrocode}
%<*package>
%    \end{macrocode}
%
% \subsection{Reload check and package identification}
%    Reload check, especially if the package is not used with \LaTeX.
%    \begin{macrocode}
\begingroup\catcode61\catcode48\catcode32=10\relax%
  \catcode13=5 % ^^M
  \endlinechar=13 %
  \catcode35=6 % #
  \catcode39=12 % '
  \catcode44=12 % ,
  \catcode45=12 % -
  \catcode46=12 % .
  \catcode58=12 % :
  \catcode64=11 % @
  \catcode123=1 % {
  \catcode125=2 % }
  \expandafter\let\expandafter\x\csname ver@embedfile.sty\endcsname
  \ifx\x\relax % plain-TeX, first loading
  \else
    \def\empty{}%
    \ifx\x\empty % LaTeX, first loading,
      % variable is initialized, but \ProvidesPackage not yet seen
    \else
      \expandafter\ifx\csname PackageInfo\endcsname\relax
        \def\x#1#2{%
          \immediate\write-1{Package #1 Info: #2.}%
        }%
      \else
        \def\x#1#2{\PackageInfo{#1}{#2, stopped}}%
      \fi
      \x{embedfile}{The package is already loaded}%
      \aftergroup\endinput
    \fi
  \fi
\endgroup%
%    \end{macrocode}
%    Package identification:
%    \begin{macrocode}
\begingroup\catcode61\catcode48\catcode32=10\relax%
  \catcode13=5 % ^^M
  \endlinechar=13 %
  \catcode35=6 % #
  \catcode39=12 % '
  \catcode40=12 % (
  \catcode41=12 % )
  \catcode44=12 % ,
  \catcode45=12 % -
  \catcode46=12 % .
  \catcode47=12 % /
  \catcode58=12 % :
  \catcode64=11 % @
  \catcode91=12 % [
  \catcode93=12 % ]
  \catcode123=1 % {
  \catcode125=2 % }
  \expandafter\ifx\csname ProvidesPackage\endcsname\relax
    \def\x#1#2#3[#4]{\endgroup
      \immediate\write-1{Package: #3 #4}%
      \xdef#1{#4}%
    }%
  \else
    \def\x#1#2[#3]{\endgroup
      #2[{#3}]%
      \ifx#1\@undefined
        \xdef#1{#3}%
      \fi
      \ifx#1\relax
        \xdef#1{#3}%
      \fi
    }%
  \fi
\expandafter\x\csname ver@embedfile.sty\endcsname
\ProvidesPackage{embedfile}%
  [2011/04/13 v2.6 Embed files into PDF (HO)]%
%    \end{macrocode}
%
% \subsection{Catcodes}
%
%    \begin{macrocode}
\begingroup\catcode61\catcode48\catcode32=10\relax%
  \catcode13=5 % ^^M
  \endlinechar=13 %
  \catcode123=1 % {
  \catcode125=2 % }
  \catcode64=11 % @
  \def\x{\endgroup
    \expandafter\edef\csname EmFi@AtEnd\endcsname{%
      \endlinechar=\the\endlinechar\relax
      \catcode13=\the\catcode13\relax
      \catcode32=\the\catcode32\relax
      \catcode35=\the\catcode35\relax
      \catcode61=\the\catcode61\relax
      \catcode64=\the\catcode64\relax
      \catcode123=\the\catcode123\relax
      \catcode125=\the\catcode125\relax
    }%
  }%
\x\catcode61\catcode48\catcode32=10\relax%
\catcode13=5 % ^^M
\endlinechar=13 %
\catcode35=6 % #
\catcode64=11 % @
\catcode123=1 % {
\catcode125=2 % }
\def\TMP@EnsureCode#1#2{%
  \edef\EmFi@AtEnd{%
    \EmFi@AtEnd
    \catcode#1=\the\catcode#1\relax
  }%
  \catcode#1=#2\relax
}
\TMP@EnsureCode{39}{12}% '
\TMP@EnsureCode{40}{12}% (
\TMP@EnsureCode{41}{12}% )
\TMP@EnsureCode{44}{12}% ,
\TMP@EnsureCode{46}{12}% .
\TMP@EnsureCode{47}{12}% /
\TMP@EnsureCode{58}{12}% :
\TMP@EnsureCode{60}{12}% <
\TMP@EnsureCode{62}{12}% >
\TMP@EnsureCode{91}{12}% [
\TMP@EnsureCode{93}{12}% ]
\TMP@EnsureCode{96}{12}% `
\edef\EmFi@AtEnd{\EmFi@AtEnd\noexpand\endinput}
%    \end{macrocode}
%
% \subsection{Tools}
%
%    \begin{macro}{\EmFi@RequirePackage}
%    \begin{macrocode}
\begingroup\expandafter\expandafter\expandafter\endgroup
\expandafter\ifx\csname RequirePackage\endcsname\relax
  \def\EmFi@RequirePackage#1[#2]{%
    \input #1.sty\relax
  }%
\else
  \let\EmFi@RequirePackage\RequirePackage
\fi
%    \end{macrocode}
%    \end{macro}
%
%    \begin{macro}{\EmFi@Error}
%    \begin{macrocode}
\EmFi@RequirePackage{infwarerr}[2007/09/09]%
\def\EmFi@Error{%
  \@PackageError{embedfile}%
}
%    \end{macrocode}
%    \end{macro}
%
% \subsection{Check for recent \pdfTeX\ in PDF mode}
%
%    Load package \xpackage{ifpdf} and check mode.
%    \begin{macrocode}
\EmFi@RequirePackage{ifpdf}[2007/09/09]
\ifpdf
\else
  \EmFi@Error{%
    Missing pdfTeX in PDF mode%
  }{%
    Currently other drivers are not supported. %
    Package loading is aborted.%
  }%
  \expandafter\EmFi@AtEnd
\fi%
%    \end{macrocode}
%    \begin{macrocode}
\EmFi@RequirePackage{pdftexcmds}[2007/11/11]
\EmFi@RequirePackage{ltxcmds}[2010/03/01]
\EmFi@RequirePackage{kvsetkeys}[2010/03/01]
\EmFi@RequirePackage{kvdefinekeys}[2010/03/01]
%    \end{macrocode}
%    Check version.
%    \begin{macrocode}
\begingroup\expandafter\expandafter\expandafter\endgroup
\expandafter\ifx\csname pdf@filesize\endcsname\relax
  \EmFi@Error{%
    Unsupported pdfTeX version%
  }{%
    At least version 1.30 is necessary. Package loading is aborted.%
  }%
  \expandafter\EmFi@AtEnd
\fi%
%    \end{macrocode}
%
% \subsection{Strings}
%
%    Minimal version of package \xpackage{pdfescape} is 2007/08/27 v1.5
%    because of \cs{EdefSanitize}.
%    \begin{macrocode}
\EmFi@RequirePackage{pdfescape}[2007/11/11]
%    \end{macrocode}
%
%    \begin{macrocode}
\def\EmFi@temp#1{%
  \expandafter\EdefSanitize\csname EmFi@S@#1\endcsname{#1}%
}
%    \end{macrocode}
%    \begin{macro}{\EmFi@details}
%    \begin{macrocode}
\EmFi@temp{details}%
%    \end{macrocode}
%    \end{macro}
%    \begin{macro}{\EmFi@tile}
%    \begin{macrocode}
\EmFi@temp{tile}%
%    \end{macrocode}
%    \end{macro}
%    \begin{macro}{\EmFi@hidden}
%    \begin{macrocode}
\EmFi@temp{hidden}%
%    \end{macrocode}
%    \end{macro}
%    \begin{macro}{\EmFi@S@text}
%    \begin{macrocode}
\EmFi@temp{text}
%    \end{macrocode}
%    \end{macro}
%    \begin{macro}{\EmFi@S@date}
%    \begin{macrocode}
\EmFi@temp{date}
%    \end{macrocode}
%    \end{macro}
%    \begin{macro}{\EmFi@S@number}
%    \begin{macrocode}
\EmFi@temp{number}
%    \end{macrocode}
%    \end{macro}
%    \begin{macro}{\EmFi@S@file}
%    \begin{macrocode}
\EmFi@temp{file}
%    \end{macrocode}
%    \end{macro}
%    \begin{macro}{\EmFi@S@desc}
%    \begin{macrocode}
\EmFi@temp{desc}
%    \end{macrocode}
%    \end{macro}
%    \begin{macro}{\EmFi@S@moddate}
%    \begin{macrocode}
\EmFi@temp{moddate}
%    \end{macrocode}
%    \end{macro}
%    \begin{macro}{\EmFi@S@creationdate}
%    \begin{macrocode}
\EmFi@temp{creationdate}
%    \end{macrocode}
%    \end{macro}
%    \begin{macro}{\EmFi@S@size}
%    \begin{macrocode}
\EmFi@temp{size}
%    \end{macrocode}
%    \end{macro}
%    \begin{macro}{\EmFi@S@ascending}
%    \begin{macrocode}
\EmFi@temp{ascending}
%    \end{macrocode}
%    \end{macro}
%    \begin{macro}{\EmFi@S@descending}
%    \begin{macrocode}
\EmFi@temp{descending}
%    \end{macrocode}
%    \end{macro}
%    \begin{macro}{\EmFi@S@true}
%    \begin{macrocode}
\EmFi@temp{true}
%    \end{macrocode}
%    \end{macro}
%    \begin{macro}{\EmFi@S@false}
%    \begin{macrocode}
\EmFi@temp{false}
%    \end{macrocode}
%    \end{macro}
%
% \subsection{Switches}
%
%    \begin{macro}{\ifEmFi@collection}
%    \begin{macrocode}
\ltx@newif\ifEmFi@collection
%    \end{macrocode}
%    \end{macro}
%    \begin{macro}{\ifEmFi@sort}
%    \begin{macrocode}
\ltx@newif\ifEmFi@sort
%    \end{macrocode}
%    \end{macro}
%    \begin{macro}{\ifEmFi@visible}
%    \begin{macrocode}
\ltx@newif\ifEmFi@visible
%    \end{macrocode}
%    \end{macro}
%    \begin{macro}{\ifEmFi@edit}
%    \begin{macrocode}
\ltx@newif\ifEmFi@edit
%    \end{macrocode}
%    \end{macro}
%    \begin{macro}{\ifEmFi@item}
%    \begin{macrocode}
\ltx@newif\ifEmFi@item
%    \end{macrocode}
%    \end{macro}
%    \begin{macro}{\ifEmFi@finished}
%    \begin{macrocode}
\ltx@newif\ifEmFi@finished
%    \end{macrocode}
%    \end{macro}
%    \begin{macro}{\ifEmFi@id}
%    \begin{macrocode}
\ltx@newif\ifEmFi@id
%    \end{macrocode}
%    \end{macro}
%
% \subsection{Key value definitions}
%
%    \begin{macro}{\EmFi@GlobalKey}
%    \begin{macrocode}
\def\EmFi@GlobalKey#1#2{%
  \global\expandafter\let\csname KV@#1@#2\expandafter\endcsname
                         \csname KV@#1@#2\endcsname
}
%    \end{macrocode}
%    \end{macro}
%    \begin{macro}{\EmFi@GlobalDefaultKey}
%    \begin{macrocode}
\def\EmFi@GlobalDefaultKey#1#2{%
  \EmFi@GlobalKey{#1}{#2}%
  \global\expandafter\let
      \csname KV@#1@#2@default\expandafter\endcsname
      \csname KV@#1@#2@default\endcsname
}
%    \end{macrocode}
%    \end{macro}
%
%    \begin{macro}{\EmFi@DefineKey}
%    \begin{macrocode}
\def\EmFi@DefineKey#1#2{%
  \kv@define@key{EmFi}{#1}{%
    \expandafter\def\csname EmFi@#1\endcsname{##1}%
  }%
  \expandafter\def\csname EmFi@#1\endcsname{#2}%
}
%    \end{macrocode}
%    \end{macro}
%    Subtype of the embedded file (optional).
%    \begin{macrocode}
\EmFi@DefineKey{mimetype}{}
%    \end{macrocode}
%    File specification string.
%    \begin{macrocode}
\EmFi@DefineKey{filespec}{\EmFi@file}
%    \end{macrocode}
%    File specification string in Unicode.
%    \begin{macrocode}
\EmFi@DefineKey{ucfilespec}{}
%    \end{macrocode}
%    File system (optional).
%    \begin{macrocode}
\EmFi@DefineKey{filesystem}{}
%    \end{macrocode}
%    Description (optional).
%    \begin{macrocode}
\EmFi@DefineKey{desc}{}
%    \end{macrocode}
%    Method for converting text to PDF strings.
%    \begin{macrocode}
\EmFi@DefineKey{stringmethod}{%
  \ifx\pdfstringdef\@undefined
    escape%
  \else
    \ifx\pdfstringdef\relax
      escape%
    \else
      psd%
    \fi
  \fi
}
%    \end{macrocode}
%
%    Option \xoption{id} as key for object numbers.
%    \begin{macrocode}
\kv@define@key{EmFi}{id}{%
  \def\EmFi@id{#1}%
  \EmFi@idtrue
}
%    \end{macrocode}
%    \begin{macro}{\EmFi@defobj}
%    \begin{macrocode}
\def\EmFi@defobj#1{%
  \ifEmFi@id
    \expandafter\xdef\csname EmFi@#1@\EmFi@id\endcsname{%
      \the\pdflastobj\ltx@space 0 R%
    }%
  \fi
}
%    \end{macrocode}
%    \end{macro}
%    \begin{macro}{\embedfileifobjectexists}
%    \begin{macrocode}
\def\embedfileifobjectexists#1#2{%
  \expandafter\ifx\csname EmFi@#2@#1\endcsname\relax
    \expandafter\ltx@secondoftwo
  \else
    \expandafter\ltx@firstoftwo
  \fi
}
%    \end{macrocode}
%    \end{macro}
%    \begin{macro}{\embedfilegetobject}
%    \begin{macrocode}
\def\embedfilegetobject#1#2{%
  \embedfileifobjectexists{#1}{#2}{%
    \csname EmFi@#2@#1\endcsname
  }{%
    0 0 R%
  }%
}
%    \end{macrocode}
%    \end{macro}
%
%    Initial view of the collection.
%    \begin{macrocode}
\kv@define@key{EmFi}{view}[]{%
  \EdefSanitize\EmFi@temp{#1}%
  \def\EmFi@next{%
    \global\EmFi@collectiontrue
  }%
  \ifx\EmFi@temp\ltx@empty
    \let\EmFi@view\EmFi@S@details
  \else\ifx\EmFi@temp\EmFi@S@details
    \let\EmFi@view\EmFi@S@details
  \else\ifx\EmFi@temp\EmFi@S@tile
    \let\EmFi@view\EmFi@S@tile
  \else\ifx\EmFi@temp\EmFi@S@hidden
    \let\EmFi@view\EmFi@S@hidden
  \else
    \let\EmFi@next\relax
    \EmFi@Error{%
      Unknown value `\EmFi@temp' for key `view'.\MessageBreak
      Supported values: `details', `tile', `hidden'.%
    }\@ehc
  \fi\fi\fi\fi
  \EmFi@next
}
%    \end{macrocode}
%    \begin{macrocode}
\EmFi@DefineKey{initialfile}{}
%    \end{macrocode}
%
%    \begin{macro}{\embedfilesetup}
%    \begin{macrocode}
\def\embedfilesetup{%
  \ifEmFi@finished
    \def\EmFi@next##1{}%
    \EmFi@Error{%
      \string\embedfilefield\ltx@space after \string\embedfilefinish
    }{%
      The list of embedded files is already written.%
    }%
  \else
    \def\EmFi@next{%
      \kvsetkeys{EmFi}%
    }%
  \fi
  \EmFi@next
}
%    \end{macrocode}
%    \end{macro}
%
%    \begin{macro}{\EmFi@schema}
%    \begin{macrocode}
\def\EmFi@schema{}
%    \end{macrocode}
%    \end{macro}
%    \begin{macro}{\EmFi@order}
%    \begin{macrocode}
\gdef\EmFi@order{0}
%    \end{macrocode}
%    \end{macro}
%    \begin{macro}{\EmFi@@order}
%    \begin{macrocode}
\let\EmFi@@order\relax
%    \end{macrocode}
%    \end{macro}
%    \begin{macro}{\EmFi@fieldlist}
%    \begin{macrocode}
\def\EmFi@fieldlist{}
%    \end{macrocode}
%    \end{macro}
%    \begin{macro}{\EmFi@sortcase}
%    \begin{macrocode}
\def\EmFi@sortcase{0}%
%    \end{macrocode}
%    \end{macro}
%    \begin{macro}{\embedfilefield}
%    \begin{macrocode}
\def\embedfilefield#1#2{%
  \ifEmFi@finished
    \EmFi@Error{%
      \string\embedfilefield\ltx@space after \string\embedfilefinish
    }{%
      The list of embedded files is already written.%
    }%
  \else
    \global\EmFi@collectiontrue
    \EdefSanitize\EmFi@key{#1}%
    \expandafter\ifx\csname KV@EmFi@\EmFi@key.prefix\endcsname\relax
      \begingroup
        \count@=\EmFi@order
        \advance\count@ 1 %
        \xdef\EmFi@order{\the\count@}%
        \let\EmFi@title\EmFi@key
        \let\EmFi@type\EmFi@S@text
        \EmFi@visibletrue
        \EmFi@editfalse
        \kvsetkeys{EmFiFi}{#2}%
        \EmFi@convert\EmFi@title\EmFi@title
        \xdef\EmFi@schema{%
          \EmFi@schema
          /\pdf@escapename{\EmFi@key}<<%
            /Subtype/%
            \ifx\EmFi@type\EmFi@S@date D%
            \else\ifx\EmFi@type\EmFi@S@number N%
            \else\ifx\EmFi@type\EmFi@S@file F%
            \else\ifx\EmFi@type\EmFi@S@desc Desc%
            \else\ifx\EmFi@type\EmFi@S@moddate ModDate%
            \else\ifx\EmFi@type\EmFi@S@creationdate CreationDate%
            \else\ifx\EmFi@type\EmFi@S@size Size%
            \else S%
            \fi\fi\fi\fi\fi\fi\fi
            /N(\EmFi@title)%
            \EmFi@@order{\EmFi@order}%
            \ifEmFi@visible
            \else
              /V false%
            \fi
            \ifEmFi@edit
              /E true%
            \fi
          >>%
        }%
        \let\do\relax
        \xdef\EmFi@fieldlist{%
          \EmFi@fieldlist
          \do{\EmFi@key}%
        }%
        \ifx\EmFi@type\EmFi@S@text
          \kv@define@key{EmFi}{\EmFi@key.value}{%
            \EmFi@itemtrue
            \def\EmFi@temp{##1}%
            \EmFi@convert\EmFi@temp\EmFi@temp
            \expandafter\def\csname EmFi@V@#1%
            \expandafter\endcsname\expandafter{%
              \expandafter(\EmFi@temp)%
            }%
          }%
          \EmFi@GlobalKey{EmFi}{\EmFi@key.value}%
        \else\ifx\EmFi@type\EmFi@S@date
          \kv@define@key{EmFi}{\EmFi@key.value}{%
            \EmFi@itemtrue
            \def\EmFi@temp{##1}%
            \EmFi@convert\EmFi@temp\EmFi@temp
            \expandafter\def\csname EmFi@V@#1%
            \expandafter\endcsname\expandafter{%
              \expandafter(\EmFi@temp)%
            }%
          }%
          \EmFi@GlobalKey{EmFi}{\EmFi@key.value}%
        \else\ifx\EmFi@type\EmFi@S@number
          \kv@define@key{EmFi}{\EmFi@key.value}{%
            \EmFi@itemtrue
            \expandafter\EdefSanitize\csname EmFi@V@#1\endcsname{ ##1}%
          }%
          \EmFi@GlobalKey{EmFi}{\EmFi@key.value}%
        \fi\fi\fi
        \kv@define@key{EmFi}{\EmFi@key.prefix}{%
          \EmFi@itemtrue
          \expandafter\def\csname EmFi@P@#1\endcsname{##1}%
        }%
        \EmFi@GlobalKey{EmFi}{\EmFi@key.prefix}%
        \kv@define@key{EmFiSo}{\EmFi@key}[ascending]{%
          \EdefSanitize\EmFi@temp{##1}%
          \ifx\EmFi@temp\EmFi@S@ascending
            \def\EmFi@temp{true}%
          \else\ifx\EmFi@temp\EmFi@S@descending
            \def\EmFi@temp{false}%
          \else
            \def\EmFi@temp{}%
            \EmFi@Error{%
              Unknown sort order `\EmFi@temp'.\MessageBreak
              Supported values: `\EmFi@S@ascending', %
              `\EmFi@S@descending
            }\@ehc
          \fi\fi
          \ifx\EmFi@temp\ltx@empty
          \else
            \xdef\EmFi@sortkeys{%
              \EmFi@sortkeys
              /\pdf@escapename{#1}%
            }%
            \ifx\EmFi@sortorders\ltx@empty
              \global\let\EmFi@sortorders\EmFi@temp
              \gdef\EmFi@sortcase{1}%
            \else
              \xdef\EmFi@sortorders{%
                \EmFi@sortorders
                \ltx@space
                \EmFi@temp
              }%
              \xdef\EmFi@sortcase{2}%
            \fi
          \fi
        }%
        \EmFi@GlobalDefaultKey{EmFiSo}\EmFi@key
      \endgroup
    \else
      \EmFi@Error{%
        Field `\EmFi@key' is already defined%
      }\@ehc
    \fi
  \fi
}
%    \end{macrocode}
%    \end{macro}
%    \begin{macrocode}
\kv@define@key{EmFiFi}{type}{%
  \EdefSanitize\EmFi@temp{#1}%
  \ifx\EmFi@temp\EmFi@S@text
    \let\EmFi@type\EmFi@temp
  \else\ifx\EmFi@temp\EmFi@S@date
    \let\EmFi@type\EmFi@temp
  \else\ifx\EmFi@temp\EmFi@S@number
    \let\EmFi@type\EmFi@temp
  \else\ifx\EmFi@temp\EmFi@S@file
    \let\EmFi@type\EmFi@temp
  \else\ifx\EmFi@temp\EmFi@S@desc
    \let\EmFi@type\EmFi@temp
  \else\ifx\EmFi@temp\EmFi@S@moddate
    \let\EmFi@type\EmFi@temp
  \else\ifx\EmFi@temp\EmFi@S@creationdate
    \let\EmFi@type\EmFi@temp
  \else\ifx\EmFi@temp\EmFi@S@size
    \let\EmFi@type\EmFi@temp
  \else
    \EmFi@Error{%
      Unknown type `\EmFi@temp'.\MessageBreak
      Supported types: `text', `date', `number', `file',\MessageBreak
      `desc', `moddate', `creationdate', `size'%
    }%
  \fi\fi\fi\fi\fi\fi\fi\fi
}
%    \end{macrocode}
%    \begin{macrocode}
\kv@define@key{EmFiFi}{title}{%
  \def\EmFi@title{#1}%
}
%    \end{macrocode}
%    \begin{macro}{\EmFi@setboolean}
%    \begin{macrocode}
\def\EmFi@setboolean#1#2{%
  \EdefSanitize\EmFi@temp{#2}%
  \ifx\EmFi@temp\EmFi@S@true
    \csname EmFi@#1true\endcsname
  \else
    \ifx\EmFi@temp\EmFi@S@false
      \csname EmFi@#1false\endcsname
    \else
      \EmFi@Error{%
        Unknown value `\EmFi@temp' for key `#1'.\MessageBreak
        Supported values: `true', `false'%
      }\@ehc
    \fi
  \fi
}
%    \end{macrocode}
%    \end{macro}
%    \begin{macrocode}
\kv@define@key{EmFiFi}{visible}[true]{%
  \EmFi@setboolean{visible}{#1}%
}
%    \end{macrocode}
%    \begin{macrocode}
\kv@define@key{EmFiFi}{edit}[true]{%
  \EmFi@setboolean{edit}{#1}%
}
%    \end{macrocode}
%
%    \begin{macro}{\EmFi@sortkeys}
%    \begin{macrocode}
\def\EmFi@sortkeys{}
%    \end{macrocode}
%    \end{macro}
%    \begin{macro}{\EmFi@sortorders}
%    \begin{macrocode}
\def\EmFi@sortorders{}
%    \end{macrocode}
%    \end{macro}
%    \begin{macro}{\embedfilesort}
%    \begin{macrocode}
\def\embedfilesort{%
  \kvsetkeys{EmFiSo}%
}
%    \end{macrocode}
%    \end{macro}
%
% \subsection{Embed the file}
%
%    \begin{macro}{\embedfile}
%    \begin{macrocode}
\def\embedfile{%
  \ltx@ifnextchar[\EmFi@embedfile{\EmFi@embedfile[]}%
}
%    \end{macrocode}
%    \end{macro}
%    \begin{macro}{\EmFi@embedfile}
%    \begin{macrocode}
\def\EmFi@embedfile[#1]#2{%
  \ifEmFi@finished
    \EmFi@Error{%
      \string\embedfile\ltx@space after \string\embedfilefinish
    }{%
      The list of embedded files is already written.%
    }%
  \else
    \begingroup
      \def\EmFi@file{#2}%
      \kvsetkeys{EmFi}{#1}%
      \expandafter\expandafter\expandafter
      \ifx\expandafter\expandafter\expandafter
          \\\pdf@filesize{\EmFi@file}\\%
        \EmFi@Error{%
          File `\EmFi@file' not found%
        }{%
          The unknown file is not embedded.%
        }%
      \else
        \edef\EmFi@@filespec{%
          \pdf@escapestring{\EmFi@filespec}%
        }%
        \ifx\EmFi@ucfilespec\ltx@empty
          \let\EmFi@@ucfilespec\ltx@empty
        \else
          \EmFi@convert\EmFi@ucfilespec\EmFi@@ucfilespec
        \fi
        \ifx\EmFi@desc\ltx@empty
          \let\EmFi@@desc\ltx@empty
        \else
          \EmFi@convert\EmFi@desc\EmFi@@desc
        \fi
        \ifEmFi@item
          \let\do\EmFi@do
          \immediate\pdfobj{%
            <<%
              \EmFi@fieldlist
            >>%
          }%
          \edef\EmFi@ci{\the\pdflastobj}%
        \fi
        \immediate\pdfobj stream attr{%
          /Type/EmbeddedFile%
          \ifx\EmFi@mimetype\ltx@empty
          \else
            /Subtype/\pdf@escapename{\EmFi@mimetype}%
          \fi
          /Params<<%
            /ModDate(\pdf@filemoddate{\EmFi@file})%
            /Size \pdf@filesize{\EmFi@file}%
            /CheckSum<\pdf@filemdfivesum{\EmFi@file}>%
          >>%
        }file{\EmFi@file}\relax
        \EmFi@defobj{EmbeddedFile}%
        \immediate\pdfobj{%
          <<%
            /Type/Filespec%
            \ifx\EmFi@filesystem\ltx@empty
            \else
            /FS/\pdf@escapename{\EmFi@filesystem}%
            \fi
            /F(\EmFi@@filespec)%
            \ifx\EmFi@@ucfilespec\ltx@empty
            \else
              /UF(\EmFi@@ucfilespec)%
            \fi
            \ifx\EmFi@@desc\ltx@empty
            \else
              /Desc(\EmFi@@desc)%
            \fi
            /EF<<%
              /F \the\pdflastobj\ltx@space 0 R%
            >>%
            \ifEmFi@item
              /CI \EmFi@ci\ltx@space 0 R%
            \fi
          >>%
        }%
        \EmFi@defobj{Filespec}%
        \EmFi@add{%
          \EmFi@@filespec
        }{\the\pdflastobj\ltx@space 0 R}%
      \fi
    \endgroup
  \fi
}
%    \end{macrocode}
%    \end{macro}
%    \begin{macro}{\EmFi@do}
%    \begin{macrocode}
\def\EmFi@do#1{%
  \expandafter\ifx\csname EmFi@P@#1\endcsname\relax
    \expandafter\ifx\csname EmFi@V@#1\endcsname\relax
    \else
      /\pdf@escapename{#1}\csname EmFi@V@#1\endcsname
    \fi
  \else
    /\pdf@escapename{#1}<<%
      \expandafter\ifx\csname EmFi@V@#1\endcsname\relax
      \else
        /D\csname EmFi@V@#1\endcsname
      \fi
      /P(\csname EmFi@P@#1\endcsname)%
    >>%
  \fi
}
%    \end{macrocode}
%    \end{macro}
%    \begin{macro}{\EmFi@convert}
%    \begin{macrocode}
\def\EmFi@convert#1#2{%
  \ifnum\pdf@strcmp{\EmFi@stringmethod}{psd}=0 %
    \pdfstringdef\EmFi@temp{#1}%
    \let#2\EmFi@temp
  \else
    \edef#2{\pdf@escapestring{#1}}%
  \fi
}
%    \end{macrocode}
%    \end{macro}
%    \begin{macrocode}
\global\let\EmFi@list\ltx@empty
%    \end{macrocode}
%    \begin{macro}{\EmFi@add}
%    Sorting is done by the insertion sort algorithm.
%    Probably the sorting could be done more reliable.
%    However, the PDF specification is not too clear to me
%    regarding precise sorting rules (how to deal with
%    different encodings, escaped characters, \dots).
%    \begin{macrocode}
\def\EmFi@add#1#2{%
  \begingroup
    \ifx\EmFi@list\ltx@empty
      \xdef\EmFi@list{\noexpand\do{#1}{#2}}%
    \else
      \def\do##1##2{%
        \ifnum\pdf@strcmp{##1}{#1}>0 %
          \edef\x{%
            \toks@{%
              \the\toks@%
              \noexpand\do{#1}{#2}%
              \noexpand\do{##1}{##2}%
            }%
          }%
          \x
          \def\do####1####2{%
            \toks@\expandafter{\the\toks@\do{####1}{####2}}%
          }%
          \def\stop{%
            \xdef\EmFi@list{\the\toks@}%
          }%
        \else
          \toks@\expandafter{\the\toks@\do{##1}{##2}}%
        \fi
      }%
      \def\stop{%
        \xdef\EmFi@list{\the\toks@\noexpand\do{#1}{#2}}%
      }%
      \toks@{}%
      \EmFi@list\stop
    \fi
  \endgroup
}
%    \end{macrocode}
%    \end{macro}
%    \begin{macro}{\embedfilefinish}
%    \begin{macrocode}
\def\embedfilefinish{%
  \ifEmFi@finished
    \EmFi@Error{%
      Too many invocations of \string\embedfilefinish
    }{%
      The list of embedded files is already written.%
    }%
  \else
    \ifx\EmFi@list\ltx@empty
    \else
%    \end{macrocode}
%    Write |/EmbeddedFiles| entry.
%    \begin{macrocode}
      \global\EmFi@finishedtrue
      \begingroup
        \def\do##1##2{%
          (##1)##2%
        }%
        \immediate\pdfobj{%
          <<%
            /Names[\EmFi@list]%
          >>%
        }%
        \pdfnames{%
          /EmbeddedFiles \the\pdflastobj\ltx@space 0 R%
        }%
      \endgroup
%    \end{macrocode}
%    Write collection objects.
%    \begin{macrocode}
      \ifx\EmFi@initialfile\ltx@empty
      \else
        \EmFi@collectiontrue
      \fi
      \ifEmFi@collection
        \ifx\EmFi@initialfile\ltx@empty
          \let\EmFi@@initialfile\ltx@empty
        \else
          \edef\EmFi@@initialfile{%
            \pdf@escapestring{\EmFi@initialfile}%
          }%
        \fi
%    \end{macrocode}
%    Look for initial file among the embedded files.
%    \begin{macrocode}
        \begingroup
          \let\f=N%
          \def\do##1##2{%
            \def\x{##1}%
            \ifx\x\EmFi@@initialfile
              \let\f=Y%
              \let\do\ltx@gobbletwo
            \fi
          }%
          \EmFi@list
        \expandafter\endgroup
        \ifx\f Y%
        \else
          \@PackageWarningNoLine{embedfile}{%
            Missing initial file `\EmFi@initialfile'\MessageBreak
            among the embedded files%
          }%
          \let\EmFi@initialfile\ltx@empty
          \let\EmFi@@initialfile\ltx@empty
        \fi
        \ifcase\EmFi@sortcase
          \def\EmFi@temp{}%
        \or
          \def\EmFi@temp{%
            /S\EmFi@sortkeys
            /A \EmFi@sortorders
          }%
        \else
          \def\EmFi@temp{%
            /S[\EmFi@sortkeys]%
            /A[\EmFi@sortorders]%
          }%
        \fi
        \def\EmFi@@order##1{%
          \ifnum\EmFi@order>1 %
            /O ##1%
          \fi
        }%
        \immediate\pdfobj{%
          <<%
            \ifx\EmFi@schema\ltx@empty
            \else
              /Schema<<\EmFi@schema>>%
            \fi
            \ifx\EmFi@@initialfile\ltx@empty
            \else
              /D(\EmFi@@initialfile)%
            \fi
            \ifx\EmFi@view\EmFi@S@tile
              /View/T%
            \else\ifx\EmFi@view\EmFi@S@hidden
              /View/H%
            \fi\fi
            \ifx\EmFi@temp\ltx@empty
              \EmFi@temp
            \else
              /Sort<<\EmFi@temp>>%
            \fi
          >>%
        }%
        \pdfcatalog{%
          /Collection \the\pdflastobj\ltx@space0 R%
        }%
      \fi
    \fi
  \fi
}
%    \end{macrocode}
%    \end{macro}
%    \begin{macrocode}
\begingroup\expandafter\expandafter\expandafter\endgroup
\expandafter\ifx\csname AtEndDocument\endcsname\relax
\else
  \AtEndDocument{\embedfilefinish}%
\fi
%    \end{macrocode}
%
%    \begin{macrocode}
\EmFi@AtEnd%
%</package>
%    \end{macrocode}
%
% \section{Test}
%
% \subsection{Catcode checks for loading}
%
%    \begin{macrocode}
%<*test1>
%    \end{macrocode}
%    \begin{macrocode}
\catcode`\{=1 %
\catcode`\}=2 %
\catcode`\#=6 %
\catcode`\@=11 %
\expandafter\ifx\csname count@\endcsname\relax
  \countdef\count@=255 %
\fi
\expandafter\ifx\csname @gobble\endcsname\relax
  \long\def\@gobble#1{}%
\fi
\expandafter\ifx\csname @firstofone\endcsname\relax
  \long\def\@firstofone#1{#1}%
\fi
\expandafter\ifx\csname loop\endcsname\relax
  \expandafter\@firstofone
\else
  \expandafter\@gobble
\fi
{%
  \def\loop#1\repeat{%
    \def\body{#1}%
    \iterate
  }%
  \def\iterate{%
    \body
      \let\next\iterate
    \else
      \let\next\relax
    \fi
    \next
  }%
  \let\repeat=\fi
}%
\def\RestoreCatcodes{}
\count@=0 %
\loop
  \edef\RestoreCatcodes{%
    \RestoreCatcodes
    \catcode\the\count@=\the\catcode\count@\relax
  }%
\ifnum\count@<255 %
  \advance\count@ 1 %
\repeat

\def\RangeCatcodeInvalid#1#2{%
  \count@=#1\relax
  \loop
    \catcode\count@=15 %
  \ifnum\count@<#2\relax
    \advance\count@ 1 %
  \repeat
}
\def\RangeCatcodeCheck#1#2#3{%
  \count@=#1\relax
  \loop
    \ifnum#3=\catcode\count@
    \else
      \errmessage{%
        Character \the\count@\space
        with wrong catcode \the\catcode\count@\space
        instead of \number#3%
      }%
    \fi
  \ifnum\count@<#2\relax
    \advance\count@ 1 %
  \repeat
}
\def\space{ }
\expandafter\ifx\csname LoadCommand\endcsname\relax
  \def\LoadCommand{\input embedfile.sty\relax}%
\fi
\def\Test{%
  \RangeCatcodeInvalid{0}{47}%
  \RangeCatcodeInvalid{58}{64}%
  \RangeCatcodeInvalid{91}{96}%
  \RangeCatcodeInvalid{123}{255}%
  \catcode`\@=12 %
  \catcode`\\=0 %
  \catcode`\%=14 %
  \LoadCommand
  \RangeCatcodeCheck{0}{36}{15}%
  \RangeCatcodeCheck{37}{37}{14}%
  \RangeCatcodeCheck{38}{47}{15}%
  \RangeCatcodeCheck{48}{57}{12}%
  \RangeCatcodeCheck{58}{63}{15}%
  \RangeCatcodeCheck{64}{64}{12}%
  \RangeCatcodeCheck{65}{90}{11}%
  \RangeCatcodeCheck{91}{91}{15}%
  \RangeCatcodeCheck{92}{92}{0}%
  \RangeCatcodeCheck{93}{96}{15}%
  \RangeCatcodeCheck{97}{122}{11}%
  \RangeCatcodeCheck{123}{255}{15}%
  \RestoreCatcodes
}
\Test
\csname @@end\endcsname
\end
%    \end{macrocode}
%    \begin{macrocode}
%</test1>
%    \end{macrocode}
%
% \subsection{Simple test}
%
%    \begin{macrocode}
%<*test2>
\input embedfile.sty\relax
\embedfile[%
  stringmethod=escape,%
  mimetype=plain/text,%
  desc={LaTeX docstrip source archive for package `embedfile'},%
  id={embedfile.dtx}%
]{embedfile.dtx}
\nopagenumbers
Test (plain-\TeX): {\tt embedfile.dtx} should be embedded.%

\def\Test#1{%
  \par
  \embedfileifobjectexists{embedfile.dtx}{#1}{%
    Object #1 (embedfile.dtx): %
    \embedfilegetobject{embedfile.dtx}{#1}%
  }{%
    \errmessage{Missing object #1 (embedfile.dtx)}%
  }%
}
\Test{EmbeddedFile}
\Test{Filespec}
\embedfilefinish
\bye
%</test2>
%    \end{macrocode}
%    \begin{macrocode}
%<*test3>
\NeedsTeXFormat{LaTeX2e}
\let\SavedJobname\jobname
\def\jobname{embedfile}
\RequirePackage{dtx-attach}[2011/04/13]
\let\jobname\SavedJobname
\documentclass{minimal}
\begin{document}
  Test (\LaTeX): \texttt{embedfile.dtx} should be embedded.%
\end{document}
%</test3>
%    \end{macrocode}
%
% \subsection{Test for \iniTeX}
%
%    \begin{macrocode}
%<*test4>
\catcode`\{=1 %
\catcode`\}=2 %
\input ifluatex.sty %
\ifluatex
  \directlua{%
    tex.enableprimitives('', {%
      'pdflastobj',%
      'pdfnames',%
      'pdfobj',%
      'pdfoutput'%
    })%
  }%
\fi
\pdfoutput=1 %
\input embedfile.sty %
\shipout\hbox{}
\embedfile[%
  stringmethod=escape,%
  mimetype=plain/text,%
  desc={iniTeX source},%
]{\jobname.tex}
\embedfilefinish
\end
%</test4>
%    \end{macrocode}
%
% \section{Installation}
%
% \subsection{Download}
%
% \paragraph{Package.} This package is available on
% CTAN\footnote{\url{ftp://ftp.ctan.org/tex-archive/}}:
% \begin{description}
% \item[\CTAN{macros/latex/contrib/oberdiek/embedfile.dtx}] The source file.
% \item[\CTAN{macros/latex/contrib/oberdiek/embedfile.pdf}] Documentation.
% \end{description}
%
%
% \paragraph{Bundle.} All the packages of the bundle `oberdiek'
% are also available in a TDS compliant ZIP archive. There
% the packages are already unpacked and the documentation files
% are generated. The files and directories obey the TDS standard.
% \begin{description}
% \item[\CTAN{install/macros/latex/contrib/oberdiek.tds.zip}]
% \end{description}
% \emph{TDS} refers to the standard ``A Directory Structure
% for \TeX\ Files'' (\CTAN{tds/tds.pdf}). Directories
% with \xfile{texmf} in their name are usually organized this way.
%
% \subsection{Bundle installation}
%
% \paragraph{Unpacking.} Unpack the \xfile{oberdiek.tds.zip} in the
% TDS tree (also known as \xfile{texmf} tree) of your choice.
% Example (linux):
% \begin{quote}
%   |unzip oberdiek.tds.zip -d ~/texmf|
% \end{quote}
%
% \paragraph{Script installation.}
% Check the directory \xfile{TDS:scripts/oberdiek/} for
% scripts that need further installation steps.
% Package \xpackage{attachfile2} comes with the Perl script
% \xfile{pdfatfi.pl} that should be installed in such a way
% that it can be called as \texttt{pdfatfi}.
% Example (linux):
% \begin{quote}
%   |chmod +x scripts/oberdiek/pdfatfi.pl|\\
%   |cp scripts/oberdiek/pdfatfi.pl /usr/local/bin/|
% \end{quote}
%
% \subsection{Package installation}
%
% \paragraph{Unpacking.} The \xfile{.dtx} file is a self-extracting
% \docstrip\ archive. The files are extracted by running the
% \xfile{.dtx} through \plainTeX:
% \begin{quote}
%   \verb|tex embedfile.dtx|
% \end{quote}
%
% \paragraph{TDS.} Now the different files must be moved into
% the different directories in your installation TDS tree
% (also known as \xfile{texmf} tree):
% \begin{quote}
% \def\t{^^A
% \begin{tabular}{@{}>{\ttfamily}l@{ $\rightarrow$ }>{\ttfamily}l@{}}
%   embedfile.sty & tex/generic/oberdiek/embedfile.sty\\
%   dtx-attach.sty & tex/generic/oberdiek/dtx-attach.sty\\
%   embedfile.pdf & doc/latex/oberdiek/embedfile.pdf\\
%   embedfile-example-plain.tex & doc/latex/oberdiek/embedfile-example-plain.tex\\
%   embedfile-example-collection.tex & doc/latex/oberdiek/embedfile-example-collection.tex\\
%   test/embedfile-test1.tex & doc/latex/oberdiek/test/embedfile-test1.tex\\
%   test/embedfile-test2.tex & doc/latex/oberdiek/test/embedfile-test2.tex\\
%   test/embedfile-test3.tex & doc/latex/oberdiek/test/embedfile-test3.tex\\
%   test/embedfile-test4.tex & doc/latex/oberdiek/test/embedfile-test4.tex\\
%   embedfile.dtx & source/latex/oberdiek/embedfile.dtx\\
% \end{tabular}^^A
% }^^A
% \sbox0{\t}^^A
% \ifdim\wd0>\linewidth
%   \begingroup
%     \advance\linewidth by\leftmargin
%     \advance\linewidth by\rightmargin
%   \edef\x{\endgroup
%     \def\noexpand\lw{\the\linewidth}^^A
%   }\x
%   \def\lwbox{^^A
%     \leavevmode
%     \hbox to \linewidth{^^A
%       \kern-\leftmargin\relax
%       \hss
%       \usebox0
%       \hss
%       \kern-\rightmargin\relax
%     }^^A
%   }^^A
%   \ifdim\wd0>\lw
%     \sbox0{\small\t}^^A
%     \ifdim\wd0>\linewidth
%       \ifdim\wd0>\lw
%         \sbox0{\footnotesize\t}^^A
%         \ifdim\wd0>\linewidth
%           \ifdim\wd0>\lw
%             \sbox0{\scriptsize\t}^^A
%             \ifdim\wd0>\linewidth
%               \ifdim\wd0>\lw
%                 \sbox0{\tiny\t}^^A
%                 \ifdim\wd0>\linewidth
%                   \lwbox
%                 \else
%                   \usebox0
%                 \fi
%               \else
%                 \lwbox
%               \fi
%             \else
%               \usebox0
%             \fi
%           \else
%             \lwbox
%           \fi
%         \else
%           \usebox0
%         \fi
%       \else
%         \lwbox
%       \fi
%     \else
%       \usebox0
%     \fi
%   \else
%     \lwbox
%   \fi
% \else
%   \usebox0
% \fi
% \end{quote}
% If you have a \xfile{docstrip.cfg} that configures and enables \docstrip's
% TDS installing feature, then some files can already be in the right
% place, see the documentation of \docstrip.
%
% \subsection{Refresh file name databases}
%
% If your \TeX~distribution
% (\teTeX, \mikTeX, \dots) relies on file name databases, you must refresh
% these. For example, \teTeX\ users run \verb|texhash| or
% \verb|mktexlsr|.
%
% \subsection{Some details for the interested}
%
% \paragraph{Attached source.}
%
% The PDF documentation on CTAN also includes the
% \xfile{.dtx} source file. It can be extracted by
% AcrobatReader 6 or higher. Another option is \textsf{pdftk},
% e.g. unpack the file into the current directory:
% \begin{quote}
%   \verb|pdftk embedfile.pdf unpack_files output .|
% \end{quote}
%
% \paragraph{Unpacking with \LaTeX.}
% The \xfile{.dtx} chooses its action depending on the format:
% \begin{description}
% \item[\plainTeX:] Run \docstrip\ and extract the files.
% \item[\LaTeX:] Generate the documentation.
% \end{description}
% If you insist on using \LaTeX\ for \docstrip\ (really,
% \docstrip\ does not need \LaTeX), then inform the autodetect routine
% about your intention:
% \begin{quote}
%   \verb|latex \let\install=y% \iffalse meta-comment
%
% File: embedfile.dtx
% Version: 2011/04/13 v2.6
% Info: Embed files into PDF
%
% Copyright (C) 2006-2011 by
%    Heiko Oberdiek <heiko.oberdiek at googlemail.com>
%
% This work may be distributed and/or modified under the
% conditions of the LaTeX Project Public License, either
% version 1.3c of this license or (at your option) any later
% version. This version of this license is in
%    http://www.latex-project.org/lppl/lppl-1-3c.txt
% and the latest version of this license is in
%    http://www.latex-project.org/lppl.txt
% and version 1.3 or later is part of all distributions of
% LaTeX version 2005/12/01 or later.
%
% This work has the LPPL maintenance status "maintained".
%
% This Current Maintainer of this work is Heiko Oberdiek.
%
% The Base Interpreter refers to any `TeX-Format',
% because some files are installed in TDS:tex/generic//.
%
% This work consists of the main source file embedfile.dtx
% and the derived files
%    embedfile.sty, embedfile.pdf, embedfile.ins, embedfile.drv,
%    dtx-attach.sty, embedfile-example-plain.tex,
%    embedfile-example-collection.tex, embedfile-test1.tex,
%    embedfile-test2.tex, embedfile-test3.tex,
%    embedfile-test4.tex.
%
% Distribution:
%    CTAN:macros/latex/contrib/oberdiek/embedfile.dtx
%    CTAN:macros/latex/contrib/oberdiek/embedfile.pdf
%
% Unpacking:
%    (a) If embedfile.ins is present:
%           tex embedfile.ins
%    (b) Without embedfile.ins:
%           tex embedfile.dtx
%    (c) If you insist on using LaTeX
%           latex \let\install=y\input{embedfile.dtx}
%        (quote the arguments according to the demands of your shell)
%
% Documentation:
%    (a) If embedfile.drv is present:
%           latex embedfile.drv
%    (b) Without embedfile.drv:
%           latex embedfile.dtx; ...
%    The class ltxdoc loads the configuration file ltxdoc.cfg
%    if available. Here you can specify further options, e.g.
%    use A4 as paper format:
%       \PassOptionsToClass{a4paper}{article}
%
%    Programm calls to get the documentation (example):
%       pdflatex embedfile.dtx
%       makeindex -s gind.ist embedfile.idx
%       pdflatex embedfile.dtx
%       makeindex -s gind.ist embedfile.idx
%       pdflatex embedfile.dtx
%
% Installation:
%    TDS:tex/generic/oberdiek/embedfile.sty
%    TDS:tex/generic/oberdiek/dtx-attach.sty
%    TDS:doc/latex/oberdiek/embedfile.pdf
%    TDS:doc/latex/oberdiek/embedfile-example-plain.tex
%    TDS:doc/latex/oberdiek/embedfile-example-collection.tex
%    TDS:doc/latex/oberdiek/test/embedfile-test1.tex
%    TDS:doc/latex/oberdiek/test/embedfile-test2.tex
%    TDS:doc/latex/oberdiek/test/embedfile-test3.tex
%    TDS:doc/latex/oberdiek/test/embedfile-test4.tex
%    TDS:source/latex/oberdiek/embedfile.dtx
%
%<*ignore>
\begingroup
  \catcode123=1 %
  \catcode125=2 %
  \def\x{LaTeX2e}%
\expandafter\endgroup
\ifcase 0\ifx\install y1\fi\expandafter
         \ifx\csname processbatchFile\endcsname\relax\else1\fi
         \ifx\fmtname\x\else 1\fi\relax
\else\csname fi\endcsname
%</ignore>
%<*install>
\input docstrip.tex
\Msg{************************************************************************}
\Msg{* Installation}
\Msg{* Package: embedfile 2011/04/13 v2.6 Embed files into PDF (HO)}
\Msg{************************************************************************}

\keepsilent
\askforoverwritefalse

\let\MetaPrefix\relax
\preamble

This is a generated file.

Project: embedfile
Version: 2011/04/13 v2.6

Copyright (C) 2006-2011 by
   Heiko Oberdiek <heiko.oberdiek at googlemail.com>

This work may be distributed and/or modified under the
conditions of the LaTeX Project Public License, either
version 1.3c of this license or (at your option) any later
version. This version of this license is in
   http://www.latex-project.org/lppl/lppl-1-3c.txt
and the latest version of this license is in
   http://www.latex-project.org/lppl.txt
and version 1.3 or later is part of all distributions of
LaTeX version 2005/12/01 or later.

This work has the LPPL maintenance status "maintained".

This Current Maintainer of this work is Heiko Oberdiek.

The Base Interpreter refers to any `TeX-Format',
because some files are installed in TDS:tex/generic//.

This work consists of the main source file embedfile.dtx
and the derived files
   embedfile.sty, embedfile.pdf, embedfile.ins, embedfile.drv,
   dtx-attach.sty, embedfile-example-plain.tex,
   embedfile-example-collection.tex, embedfile-test1.tex,
   embedfile-test2.tex, embedfile-test3.tex,
   embedfile-test4.tex.

\endpreamble
\let\MetaPrefix\DoubleperCent

\generate{%
  \file{embedfile.ins}{\from{embedfile.dtx}{install}}%
  \file{embedfile.drv}{\from{embedfile.dtx}{driver}}%
  \usedir{tex/generic/oberdiek}%
  \file{embedfile.sty}{\from{embedfile.dtx}{package}}%
  \usedir{tex/latex/oberdiek}%
  \file{dtx-attach.sty}{\from{embedfile.dtx}{dtxattach}}%
  \usedir{doc/latex/oberdiek}%
  \file{embedfile-example-plain.tex}{\from{embedfile.dtx}{exampleplain}}%
  \file{embedfile-example-collection.tex}{\from{embedfile.dtx}{examplecollection}}%
  \usedir{doc/latex/oberdiek/test}%
  \file{embedfile-test1.tex}{\from{embedfile.dtx}{test1}}%
  \file{embedfile-test2.tex}{\from{embedfile.dtx}{test2}}%
  \file{embedfile-test3.tex}{\from{embedfile.dtx}{test3}}%
  \file{embedfile-test4.tex}{\from{embedfile.dtx}{test4}}%
  \nopreamble
  \nopostamble
  \usedir{source/latex/oberdiek/catalogue}%
  \file{embedfile.xml}{\from{embedfile.dtx}{catalogue}}%
}

\catcode32=13\relax% active space
\let =\space%
\Msg{************************************************************************}
\Msg{*}
\Msg{* To finish the installation you have to move the following}
\Msg{* files into a directory searched by TeX:}
\Msg{*}
\Msg{*     embedfile.sty, dtx-attach.sty}
\Msg{*}
\Msg{* To produce the documentation run the file `embedfile.drv'}
\Msg{* through LaTeX.}
\Msg{*}
\Msg{* Happy TeXing!}
\Msg{*}
\Msg{************************************************************************}

\endbatchfile
%</install>
%<*ignore>
\fi
%</ignore>
%<*driver>
\NeedsTeXFormat{LaTeX2e}
\ProvidesFile{embedfile.drv}%
  [2011/04/13 v2.6 Embed files into PDF (HO)]%
\documentclass{ltxdoc}
\usepackage{holtxdoc}[2011/11/22]
\begin{document}
  \DocInput{embedfile.dtx}%
\end{document}
%</driver>
% \fi
%
% \CheckSum{1363}
%
% \CharacterTable
%  {Upper-case    \A\B\C\D\E\F\G\H\I\J\K\L\M\N\O\P\Q\R\S\T\U\V\W\X\Y\Z
%   Lower-case    \a\b\c\d\e\f\g\h\i\j\k\l\m\n\o\p\q\r\s\t\u\v\w\x\y\z
%   Digits        \0\1\2\3\4\5\6\7\8\9
%   Exclamation   \!     Double quote  \"     Hash (number) \#
%   Dollar        \$     Percent       \%     Ampersand     \&
%   Acute accent  \'     Left paren    \(     Right paren   \)
%   Asterisk      \*     Plus          \+     Comma         \,
%   Minus         \-     Point         \.     Solidus       \/
%   Colon         \:     Semicolon     \;     Less than     \<
%   Equals        \=     Greater than  \>     Question mark \?
%   Commercial at \@     Left bracket  \[     Backslash     \\
%   Right bracket \]     Circumflex    \^     Underscore    \_
%   Grave accent  \`     Left brace    \{     Vertical bar  \|
%   Right brace   \}     Tilde         \~}
%
% \GetFileInfo{embedfile.drv}
%
% \title{The \xpackage{embedfile} package}
% \date{2011/04/13 v2.6}
% \author{Heiko Oberdiek\\\xemail{heiko.oberdiek at googlemail.com}}
%
% \maketitle
%
% \begin{abstract}
% This package embeds files to a PDF document.
% Currently the only supported driver is \pdfTeX\ $>=$ 1.30 in PDF mode.
% \end{abstract}
%
% \tableofcontents
%
% \section{Documentation}
%
% \subsection{Introduction}
%
%    The PDF format (\cite{pdfspec}) allows the inclusion of files
%    inside the PDF document. The included files can be bound to an
%    annotation on a page. Or they can be recorded in a sorted
%    list of embedded files. The packages \xpackage{attachfile}
%    or \xpackage{attachfile2} follow the first approach, this
%    package uses the latter method.
%
% \subsubsection{Future development}
%
%    My dream is a large package that merges the features of
%    all these packages meantioned before:
%    \begin{itemize}
%    \item Files can be attached to a page.
%    \item Files can be attached to the document.
%    \item An easy user interface for simple, common tasks and
%          beginners.
%    \item An interface for the advanced users that want to setup
%          every detail.
%    \item Support of many drivers (pdftex, dvips, dvipdfm, \dots).
%    \item \dots
%    \end{itemize}
%    However, I have not managed to take the time for this project.
%    Instead:
%    \begin{itemize}
%    \item First I experimented with package \xpackage{attachfile},
%          adding driver support, fixing bugs, \dots. The result is
%          currently named as \xpackage{attachfile2}. It uses an external
%          script to get file properties (size, date, checksum, \dots).
%    \item In order to avoid an external program for getting basic
%          file properties I provided a patch ``EscapeAndOther'' for
%          pdfTeX that was accepted for version 1.30.
%    \item This package closes a gap left by the packages for attaching
%          files and allows the embedding of files to the document.
%          Also it makes use of the new primitives of \pdfTeX.
%    \end{itemize}
%
% \subsection{User interface}
%
%    This package \xpackage{embedfile} can be used with both
%    \LaTeX\ and \plainTeX. See \autoref{sec:plain} that
%    explains the use with \plainTeX\ by an example.
%    In \LaTeX\ the package is loaded as usually. There are no options.
%    \begin{quote}
%      \verb|\usepackage{embedfile}|
%    \end{quote}
%
%    \begin{declcs}{embedfile}
%      |[|\meta{options}|]| \M{file}
%    \end{declcs}
%    The macro \cs{embedfile} includes file \meta{file} and
%    attaches it to the PDF document. At the end of the document
%    the sorted list of embedded files are written. Thus you can
%    safely use \cs{embedfile} before \verb|\end{document}|.
%    Embedding files using \cs{AtEndDocument} will only work,
%    if \cs{AtEndDocument} is called before loading the package
%    \xpackage{embedfile}.
%
%    The \meta{options} are give as key value pairs. The following
%    keys are supported:
%    \begin{description}
%    \item[\xoption{filespec}]
%      This allows to override the file name
%      that appears in the PDF file. If you are using other than
%      simple file names (8-bit, path separators, \dots), look into
%      the PDF specification (\cite{pdfspec}).
%      There are rules how these file names must be written/encoded.
%      Avoid 8-bit characters
%      and other special characters, the behaviour is currently undefined.
%      Use option \xoption{ucfilespec} for more funny file names.
%      The string method, see below, is |escape| since version 2.4.
%
%      This name is also used as entry in a name tree
%      (see PDF specification: |/EmbeddedFiles|). Therefore the
%      value for \xoption{filespec} must be unique among all
%      embedded files. Also key \xoption{initialfiles} refers
%      to this name, if the file name and the value of \xoption{filespec}
%      are different.
%    \item[\xoption{ucfilespec}]
%      Since PDF 1.7 the file name may be provided in Unicode.
%      The conversion of the option value into a PDF string
%      is controlled by option \xoption{stringmethod}.
%    \item[\xoption{filesystem}]
%      This sets the entry \verb|/FS| in
%      the file specification dictionary, see PDF specification
%      (\cite{pdfspec}).
%      Example: \verb|filesystem=URL|.
%    \item[\xoption{mimetype}]
%      This sets the mime type (\cite{mime}) of the file,
%      see \autoref{sec:plain} for examples and \cite{types}
%      for a list of officially registered types.
%    \item[\xoption{desc}]
%      The description for the file.
%    \item[\xoption{stringmethod}]
%      The package must convert the values of the keys \xoption{ucfilespec}
%      and \xoption{desc} into a PDF string (before version 2.4: \xoption{filespec}
%      and \xoption{desc}). If \xpackage{hyperref}
%      is found, then its \cs{pdfstringdef} will be used, otherwise
%      \pdfTeX's \cs{pdfescapestring} is used. Value \verb|psd|
%      forces the use of \cs{pdfstringdef}, value \verb|escape|
%      the use of \cs{pdfescapestring}.
%    \item[\xoption{\meta{key}.value}] Sets the value of a collection
%      item property, see section \ref{sec:collection}.
%    \item[\xoption{\meta{key}.prefix}]
%      Sets the prefix of a collection item property,
%      see section \ref{sec:collection}.
%    \item[\xoption{id}]
%      The value must be an unique name. Macros
%      \cs{embedfileifobjectexists} and \cs{embedfilegetobject}
%      are using this name later.
%    \end{description}
%
%    \begin{declcs}{embedfilefinish}
%    \end{declcs}
%    The list of all embedded files must be added as data structure
%    in the PDF file. In case of \LaTeX\ this is automatically done.
%    The package uses \cs{AtEndDocument}. Then the list of all
%    files should be known. However, \plainTeX\ does not know
%    about \cs{AtEndDocument}. Thus the user must call
%    \cs{embedfilefinish} at the end of the document
%    after the last file is embedded.
%
%    \begin{declcs}{embedfilesetup} \M{options}
%    \end{declcs}
%    Options for \cs{embedfile} and collection support
%    can be set in \cs{embedfilesetup}.
%
% \subsection{Collection support (PDF 1.7)}
% \label{sec:collection}
%
%    Since PDF 1.7 the embedded files can form a \emph{collection}
%    (sometimes referred as \emph{package}), the main document
%    is called \emph{cover sheet}. See PDF specification
%    8.2.4 ``Collections'' and 3.10.5 ``Collection items'' \cite{pdfspec}.
%
%    Usually Acrobat Reader 7 or 8 shows the
%    embedded files in a table at the bottom with the following columns:
%    \begin{quote}
%    \begin{tabular}{@{}l|l|l|l@{}}
%       Name & Description & Modified & Size\\
%       \hline
%       \dots & \dots & \dots & \dots\\
%    \end{tabular}
%    \end{quote}
%    Acrobat Reader 10 shows the embeded files in the left panel
%    and adds a new column for the compressed size.
%
%    If the files form a collection, then they are displayed
%    in a table left or top (depending on option \xoption{view},
%    see \cs{embedfilesetup}).
%
%
%    Collection support is enabled automatically, if
%    it is used.
%
%    \begin{declcs}{embedfilesetup} \M{options}
%    \end{declcs}
%    The following options are supported in addition
%    to options for \cs{embedfile}:
%    \begin{description}
%    \item[\xoption{view}]
%      If the PDF file contains a collection, then Acrobat Reader 8
%      shows a line at the top below the menu bar and the toolbar.
%      It shows the current selected file, icons for changing the
%      view mode, an options menu.
%      The initial mode how the collection is presented is
%      set by this option \xoption{view}.
%      The following modes/values are supported, the default is |details|:
%        \begin{description}
%        \item[\texttt{details}] The full collection table is displayed
%          at the top below the collection bar.
%        \item[\texttt{tile}] The files of the collection are shown
%          in tile mode on the left.
%        \item[\texttt{hidden}] The collection table is not shown.
%        \end{description}
%    \item[\xoption{initialfile}]
%      Selects the file that is initially presented. Especially
%      useful for an embedded PDF file that is then shown instead
%      of the cover document. There must be an \cs{embedfile} command
%      somewhere whose value for key \xoption{filespec} is used here.
%      The \cs{embedfile} command can drop option \xoption{filespec}
%      if the file name is not different.
%    \end{description}
%
%    \begin{declcs}{embedfilefield} \M{key} \M{options}
%    \end{declcs}
%    Macro \cs{embedfilefield} defines a column/field in the collection table.
%    The name of the field is \meta{key}.
%    \begin{description}
%    \item[\xoption{type}]
%      sets the type of the field. The supported values are:
%      \begin{description}
%      \item[\texttt{text}] A text field. Its value is set in
%        \cs{embedfile} by option \xoption{\meta{key}.value}.
%      \item[\texttt{date}] A date field. Its value is set in
%        \cs{embedfile} by option \xoption{\meta{key}.value}.
%        A special format is required, see ``3.8.3 Dates'' \cite{pdfspec}.
%      \item[\texttt{number}] A field with an integer or float number.
%        Its value is set in \cs{embedfile} by option
%        \xoption{\meta{key}.value}.
%      \item[\texttt{file}] The file name of the embedded file.
%      \item[\texttt{desc}] The description text of the embedded file. It is
%       set in \cs{embedfile} by option \xoption{desc}.
%      \item[\texttt{moddate}] The modification date of the embedded file.
%      \item[\texttt{size}] The size of the embedded file.
%      \end{description}
%      All types allow the use of a prefix that is disregarded by
%      sorting. The prefix for this field is set in \cs{embedfile} by
%      option \xoption{\meta{key}.prefix}.
%    \item[\xoption{title}] sets the column title.
%    \item[\xoption{visible}] controls whether the column is presented:
%      \begin{description}
%      \item[\texttt{true}] shows the column.
%      \item[\texttt{false}] hides the column.
%      \end{description}
%      Default: \texttt{true}
%    \item[\xoption{edit}]
%      Allows the editing of field values. Does not seem to have an
%      effect for Acrobat Reader.
%      \begin{description}
%      \item[\texttt{true}] enables the feature, if available (depends
%        on the PDF viewer).
%      \item[\texttt{false}] disables the feature.
%      \end{description}
%      Default: \texttt{false}
%    \end{description}
%    The order of \cs{embedfilefield} statements defines the order
%    of the columns.
%
%    \begin{declcs}{embedfilesort} \M{key-sort-list}
%    \end{declcs}
%    The sort order of the embedded files are controlled by
%    macro \cs{embedfilesort}. \meta{key-sort-list} defines
%    the sort order. The key is a field name defined by \cs{embedfilefield}.
%    Its value is either \texttt{ascending} or \texttt{descending}.
%    The default is \texttt{ascending}.
%
% \subsection{Export of object references}
%
%    Caution: This feature is still experimental. It may be even removed
%    in future versions. Therefore feedback would be nice, if someone
%    has a useful application for this feature.
%
%    Object numbers are saved, if \xoption{id} is given in \cs{embedfile}.
%    The following objects are supported:
%    \begin{itemize}
%    \item \texttt{EmbeddedFile}
%    \item \texttt{Filespec}
%    \end{itemize}
%
%    \begin{declcs}{embedfileifobjectexists} \M{id} \M{type}
%      \M{then} \M{else}
%    \end{declcs}
%    Macro \cs{embedfileifobjectexists} tests whether object of \meta{type}
%    is available for the embedded file identified by \meta{id}.
%
%    \begin{declcs}{embedfilegetobject} \M{id} \M{type}
%    \end{declcs}
%    Macro \cs{embedfilegetobject} expands to the full object reference
%    object of \meta{type} for the embedded file identified by \meta{id}.
%
% \subsubsection{Example}
%
%\begin{quote}
%\begin{verbatim}
%\embedfile[id={foo}]{foo.pdf}
%\embedfileifobjectexists{foo}{Filespec}{%
%  \typeout{%
%    FileSpec object for `foo': %
%    \embedfilegetobject{foo}{Filespec}%
%  }%
%}{%
%  \typeout{No Filespec object for `foo'}%
%}
%\end{verbatim}
%\end{quote}
%
% \subsection{Examples}
% \label{sec:examples}
%
% \subsubsection{\plainTeX}
% \label{sec:plain}
%
%    The package can be used with \plainTeX. It can be used
%    with or without help from \xfile{miniltx.tex}.
%
%    If additionally package \xpackage{keyval} (\xpackage{graphicx})
%    is needed, load it first. Then package \xpackage{embedfile} avoids
%    a duplicate loading of package \xpackage{keyval}.
%
%    Because \plainTeX\ does not provide a hook at end of the document,
%    you have to call \cs{embedfilefinish} manually at the end
%    after the last embedded file.
%    \begin{macrocode}
%<*exampleplain>
%<<END
% Load packages
\input miniltx
% \def\Gin@driver{pdftex.def}
% \input graphicx.sty
\input embedfile.sty
\resetatcatcode

% default setting
\embedfilesetup{
  mimetype=text/plain
}

% Embed files
\embedfile[
  filespec=example.tex,
  desc={Source code (plain-TeX) of this example}
]{embedfile-example-plain.tex}

\embedfile[
  desc={Source of package `embedfile'}
]{embedfile.dtx}

\embedfile[
  mimetype=application/pdf,
  desc={Documentation of package `embedfile'}
]{embedfile.pdf}

% Some text
This example document contains three embedded files.

% End of document
\embedfilefinish % don't forget
\bye
%END
%</exampleplain>
%    \end{macrocode}
%
% \subsubsection{Collection example}
%
%    \begin{macrocode}
%<*examplecollection>
%<<END
\NeedsTeXFormat{LaTeX2e}
\documentclass{article}
\usepackage[bookmarks=false]{hyperref}
% provides \pdfstringdef that is then used by `title' and
% other keys.
\usepackage{embedfile}[2011/04/13]
\embedfilesetup{
  view=details,
  initialfile=embedfile.pdf
}
\embedfilefield{file}{
  type=file,
  title={File name}
}
\embedfilefield{description}{
  type=desc,
  title={Description}
}
\embedfilefield{date}{
  type=moddate,
  title={Date}
}
\embedfilefield{size}{
  type=size,
  title={Size}
}
\embedfilefield{type}{
  type=text,
  title={Type},
  visible=false
}
\embedfilesort{
  type,
  date=descending
}
\begin{document}
An example for embedded files as collection.
You need Acrobat Reader 8 or higher.

\embedfile[
  desc={Source file of package `embedfile'},
  description.prefix={Package: },
  type.value={DTX}
]{embedfile.dtx}

\embedfile[
  desc={Documentation of package `embedfile'},
  description.prefix={Package: },
  type.value={PDF}
]{embedfile.pdf}

\embedfile[
  desc={The source for this example},
  description.prefix={Example: },
  type.value={TEX}
]{\jobname.tex}

\end{document}
%END
%</examplecollection>
%    \end{macrocode}
%
% \subsection{Package \xpackage{dtx-attach}}
%
% Package \xpackage{dtx-attach} is just a small application of
% package \xpackage{embedfile}. I am using it for the CTAN
% documentation of my packages in \CTAN{macros/latex/contrib/oberdiek/}.
% It also serves as small example for the use of the package with \LaTeX.
%
%    \begin{macrocode}
%<*dtxattach>
\NeedsTeXFormat{LaTeX2e}
\ProvidesPackage{dtx-attach}
  [2011/04/13 v2.6 Embed \string\jobname.dtx (HO)]%
\RequirePackage{embedfile}[2011/04/13]
\embedfile[%
  stringmethod=escape,%
  mimetype=plain/text,%
  desc={LaTeX docstrip source archive for package `\jobname'}%
]{\jobname.dtx}
%</dtxattach>
%    \end{macrocode}
%
% \StopEventually{
% }
%
% \section{Implementation}
%
%    \begin{macrocode}
%<*package>
%    \end{macrocode}
%
% \subsection{Reload check and package identification}
%    Reload check, especially if the package is not used with \LaTeX.
%    \begin{macrocode}
\begingroup\catcode61\catcode48\catcode32=10\relax%
  \catcode13=5 % ^^M
  \endlinechar=13 %
  \catcode35=6 % #
  \catcode39=12 % '
  \catcode44=12 % ,
  \catcode45=12 % -
  \catcode46=12 % .
  \catcode58=12 % :
  \catcode64=11 % @
  \catcode123=1 % {
  \catcode125=2 % }
  \expandafter\let\expandafter\x\csname ver@embedfile.sty\endcsname
  \ifx\x\relax % plain-TeX, first loading
  \else
    \def\empty{}%
    \ifx\x\empty % LaTeX, first loading,
      % variable is initialized, but \ProvidesPackage not yet seen
    \else
      \expandafter\ifx\csname PackageInfo\endcsname\relax
        \def\x#1#2{%
          \immediate\write-1{Package #1 Info: #2.}%
        }%
      \else
        \def\x#1#2{\PackageInfo{#1}{#2, stopped}}%
      \fi
      \x{embedfile}{The package is already loaded}%
      \aftergroup\endinput
    \fi
  \fi
\endgroup%
%    \end{macrocode}
%    Package identification:
%    \begin{macrocode}
\begingroup\catcode61\catcode48\catcode32=10\relax%
  \catcode13=5 % ^^M
  \endlinechar=13 %
  \catcode35=6 % #
  \catcode39=12 % '
  \catcode40=12 % (
  \catcode41=12 % )
  \catcode44=12 % ,
  \catcode45=12 % -
  \catcode46=12 % .
  \catcode47=12 % /
  \catcode58=12 % :
  \catcode64=11 % @
  \catcode91=12 % [
  \catcode93=12 % ]
  \catcode123=1 % {
  \catcode125=2 % }
  \expandafter\ifx\csname ProvidesPackage\endcsname\relax
    \def\x#1#2#3[#4]{\endgroup
      \immediate\write-1{Package: #3 #4}%
      \xdef#1{#4}%
    }%
  \else
    \def\x#1#2[#3]{\endgroup
      #2[{#3}]%
      \ifx#1\@undefined
        \xdef#1{#3}%
      \fi
      \ifx#1\relax
        \xdef#1{#3}%
      \fi
    }%
  \fi
\expandafter\x\csname ver@embedfile.sty\endcsname
\ProvidesPackage{embedfile}%
  [2011/04/13 v2.6 Embed files into PDF (HO)]%
%    \end{macrocode}
%
% \subsection{Catcodes}
%
%    \begin{macrocode}
\begingroup\catcode61\catcode48\catcode32=10\relax%
  \catcode13=5 % ^^M
  \endlinechar=13 %
  \catcode123=1 % {
  \catcode125=2 % }
  \catcode64=11 % @
  \def\x{\endgroup
    \expandafter\edef\csname EmFi@AtEnd\endcsname{%
      \endlinechar=\the\endlinechar\relax
      \catcode13=\the\catcode13\relax
      \catcode32=\the\catcode32\relax
      \catcode35=\the\catcode35\relax
      \catcode61=\the\catcode61\relax
      \catcode64=\the\catcode64\relax
      \catcode123=\the\catcode123\relax
      \catcode125=\the\catcode125\relax
    }%
  }%
\x\catcode61\catcode48\catcode32=10\relax%
\catcode13=5 % ^^M
\endlinechar=13 %
\catcode35=6 % #
\catcode64=11 % @
\catcode123=1 % {
\catcode125=2 % }
\def\TMP@EnsureCode#1#2{%
  \edef\EmFi@AtEnd{%
    \EmFi@AtEnd
    \catcode#1=\the\catcode#1\relax
  }%
  \catcode#1=#2\relax
}
\TMP@EnsureCode{39}{12}% '
\TMP@EnsureCode{40}{12}% (
\TMP@EnsureCode{41}{12}% )
\TMP@EnsureCode{44}{12}% ,
\TMP@EnsureCode{46}{12}% .
\TMP@EnsureCode{47}{12}% /
\TMP@EnsureCode{58}{12}% :
\TMP@EnsureCode{60}{12}% <
\TMP@EnsureCode{62}{12}% >
\TMP@EnsureCode{91}{12}% [
\TMP@EnsureCode{93}{12}% ]
\TMP@EnsureCode{96}{12}% `
\edef\EmFi@AtEnd{\EmFi@AtEnd\noexpand\endinput}
%    \end{macrocode}
%
% \subsection{Tools}
%
%    \begin{macro}{\EmFi@RequirePackage}
%    \begin{macrocode}
\begingroup\expandafter\expandafter\expandafter\endgroup
\expandafter\ifx\csname RequirePackage\endcsname\relax
  \def\EmFi@RequirePackage#1[#2]{%
    \input #1.sty\relax
  }%
\else
  \let\EmFi@RequirePackage\RequirePackage
\fi
%    \end{macrocode}
%    \end{macro}
%
%    \begin{macro}{\EmFi@Error}
%    \begin{macrocode}
\EmFi@RequirePackage{infwarerr}[2007/09/09]%
\def\EmFi@Error{%
  \@PackageError{embedfile}%
}
%    \end{macrocode}
%    \end{macro}
%
% \subsection{Check for recent \pdfTeX\ in PDF mode}
%
%    Load package \xpackage{ifpdf} and check mode.
%    \begin{macrocode}
\EmFi@RequirePackage{ifpdf}[2007/09/09]
\ifpdf
\else
  \EmFi@Error{%
    Missing pdfTeX in PDF mode%
  }{%
    Currently other drivers are not supported. %
    Package loading is aborted.%
  }%
  \expandafter\EmFi@AtEnd
\fi%
%    \end{macrocode}
%    \begin{macrocode}
\EmFi@RequirePackage{pdftexcmds}[2007/11/11]
\EmFi@RequirePackage{ltxcmds}[2010/03/01]
\EmFi@RequirePackage{kvsetkeys}[2010/03/01]
\EmFi@RequirePackage{kvdefinekeys}[2010/03/01]
%    \end{macrocode}
%    Check version.
%    \begin{macrocode}
\begingroup\expandafter\expandafter\expandafter\endgroup
\expandafter\ifx\csname pdf@filesize\endcsname\relax
  \EmFi@Error{%
    Unsupported pdfTeX version%
  }{%
    At least version 1.30 is necessary. Package loading is aborted.%
  }%
  \expandafter\EmFi@AtEnd
\fi%
%    \end{macrocode}
%
% \subsection{Strings}
%
%    Minimal version of package \xpackage{pdfescape} is 2007/08/27 v1.5
%    because of \cs{EdefSanitize}.
%    \begin{macrocode}
\EmFi@RequirePackage{pdfescape}[2007/11/11]
%    \end{macrocode}
%
%    \begin{macrocode}
\def\EmFi@temp#1{%
  \expandafter\EdefSanitize\csname EmFi@S@#1\endcsname{#1}%
}
%    \end{macrocode}
%    \begin{macro}{\EmFi@details}
%    \begin{macrocode}
\EmFi@temp{details}%
%    \end{macrocode}
%    \end{macro}
%    \begin{macro}{\EmFi@tile}
%    \begin{macrocode}
\EmFi@temp{tile}%
%    \end{macrocode}
%    \end{macro}
%    \begin{macro}{\EmFi@hidden}
%    \begin{macrocode}
\EmFi@temp{hidden}%
%    \end{macrocode}
%    \end{macro}
%    \begin{macro}{\EmFi@S@text}
%    \begin{macrocode}
\EmFi@temp{text}
%    \end{macrocode}
%    \end{macro}
%    \begin{macro}{\EmFi@S@date}
%    \begin{macrocode}
\EmFi@temp{date}
%    \end{macrocode}
%    \end{macro}
%    \begin{macro}{\EmFi@S@number}
%    \begin{macrocode}
\EmFi@temp{number}
%    \end{macrocode}
%    \end{macro}
%    \begin{macro}{\EmFi@S@file}
%    \begin{macrocode}
\EmFi@temp{file}
%    \end{macrocode}
%    \end{macro}
%    \begin{macro}{\EmFi@S@desc}
%    \begin{macrocode}
\EmFi@temp{desc}
%    \end{macrocode}
%    \end{macro}
%    \begin{macro}{\EmFi@S@moddate}
%    \begin{macrocode}
\EmFi@temp{moddate}
%    \end{macrocode}
%    \end{macro}
%    \begin{macro}{\EmFi@S@creationdate}
%    \begin{macrocode}
\EmFi@temp{creationdate}
%    \end{macrocode}
%    \end{macro}
%    \begin{macro}{\EmFi@S@size}
%    \begin{macrocode}
\EmFi@temp{size}
%    \end{macrocode}
%    \end{macro}
%    \begin{macro}{\EmFi@S@ascending}
%    \begin{macrocode}
\EmFi@temp{ascending}
%    \end{macrocode}
%    \end{macro}
%    \begin{macro}{\EmFi@S@descending}
%    \begin{macrocode}
\EmFi@temp{descending}
%    \end{macrocode}
%    \end{macro}
%    \begin{macro}{\EmFi@S@true}
%    \begin{macrocode}
\EmFi@temp{true}
%    \end{macrocode}
%    \end{macro}
%    \begin{macro}{\EmFi@S@false}
%    \begin{macrocode}
\EmFi@temp{false}
%    \end{macrocode}
%    \end{macro}
%
% \subsection{Switches}
%
%    \begin{macro}{\ifEmFi@collection}
%    \begin{macrocode}
\ltx@newif\ifEmFi@collection
%    \end{macrocode}
%    \end{macro}
%    \begin{macro}{\ifEmFi@sort}
%    \begin{macrocode}
\ltx@newif\ifEmFi@sort
%    \end{macrocode}
%    \end{macro}
%    \begin{macro}{\ifEmFi@visible}
%    \begin{macrocode}
\ltx@newif\ifEmFi@visible
%    \end{macrocode}
%    \end{macro}
%    \begin{macro}{\ifEmFi@edit}
%    \begin{macrocode}
\ltx@newif\ifEmFi@edit
%    \end{macrocode}
%    \end{macro}
%    \begin{macro}{\ifEmFi@item}
%    \begin{macrocode}
\ltx@newif\ifEmFi@item
%    \end{macrocode}
%    \end{macro}
%    \begin{macro}{\ifEmFi@finished}
%    \begin{macrocode}
\ltx@newif\ifEmFi@finished
%    \end{macrocode}
%    \end{macro}
%    \begin{macro}{\ifEmFi@id}
%    \begin{macrocode}
\ltx@newif\ifEmFi@id
%    \end{macrocode}
%    \end{macro}
%
% \subsection{Key value definitions}
%
%    \begin{macro}{\EmFi@GlobalKey}
%    \begin{macrocode}
\def\EmFi@GlobalKey#1#2{%
  \global\expandafter\let\csname KV@#1@#2\expandafter\endcsname
                         \csname KV@#1@#2\endcsname
}
%    \end{macrocode}
%    \end{macro}
%    \begin{macro}{\EmFi@GlobalDefaultKey}
%    \begin{macrocode}
\def\EmFi@GlobalDefaultKey#1#2{%
  \EmFi@GlobalKey{#1}{#2}%
  \global\expandafter\let
      \csname KV@#1@#2@default\expandafter\endcsname
      \csname KV@#1@#2@default\endcsname
}
%    \end{macrocode}
%    \end{macro}
%
%    \begin{macro}{\EmFi@DefineKey}
%    \begin{macrocode}
\def\EmFi@DefineKey#1#2{%
  \kv@define@key{EmFi}{#1}{%
    \expandafter\def\csname EmFi@#1\endcsname{##1}%
  }%
  \expandafter\def\csname EmFi@#1\endcsname{#2}%
}
%    \end{macrocode}
%    \end{macro}
%    Subtype of the embedded file (optional).
%    \begin{macrocode}
\EmFi@DefineKey{mimetype}{}
%    \end{macrocode}
%    File specification string.
%    \begin{macrocode}
\EmFi@DefineKey{filespec}{\EmFi@file}
%    \end{macrocode}
%    File specification string in Unicode.
%    \begin{macrocode}
\EmFi@DefineKey{ucfilespec}{}
%    \end{macrocode}
%    File system (optional).
%    \begin{macrocode}
\EmFi@DefineKey{filesystem}{}
%    \end{macrocode}
%    Description (optional).
%    \begin{macrocode}
\EmFi@DefineKey{desc}{}
%    \end{macrocode}
%    Method for converting text to PDF strings.
%    \begin{macrocode}
\EmFi@DefineKey{stringmethod}{%
  \ifx\pdfstringdef\@undefined
    escape%
  \else
    \ifx\pdfstringdef\relax
      escape%
    \else
      psd%
    \fi
  \fi
}
%    \end{macrocode}
%
%    Option \xoption{id} as key for object numbers.
%    \begin{macrocode}
\kv@define@key{EmFi}{id}{%
  \def\EmFi@id{#1}%
  \EmFi@idtrue
}
%    \end{macrocode}
%    \begin{macro}{\EmFi@defobj}
%    \begin{macrocode}
\def\EmFi@defobj#1{%
  \ifEmFi@id
    \expandafter\xdef\csname EmFi@#1@\EmFi@id\endcsname{%
      \the\pdflastobj\ltx@space 0 R%
    }%
  \fi
}
%    \end{macrocode}
%    \end{macro}
%    \begin{macro}{\embedfileifobjectexists}
%    \begin{macrocode}
\def\embedfileifobjectexists#1#2{%
  \expandafter\ifx\csname EmFi@#2@#1\endcsname\relax
    \expandafter\ltx@secondoftwo
  \else
    \expandafter\ltx@firstoftwo
  \fi
}
%    \end{macrocode}
%    \end{macro}
%    \begin{macro}{\embedfilegetobject}
%    \begin{macrocode}
\def\embedfilegetobject#1#2{%
  \embedfileifobjectexists{#1}{#2}{%
    \csname EmFi@#2@#1\endcsname
  }{%
    0 0 R%
  }%
}
%    \end{macrocode}
%    \end{macro}
%
%    Initial view of the collection.
%    \begin{macrocode}
\kv@define@key{EmFi}{view}[]{%
  \EdefSanitize\EmFi@temp{#1}%
  \def\EmFi@next{%
    \global\EmFi@collectiontrue
  }%
  \ifx\EmFi@temp\ltx@empty
    \let\EmFi@view\EmFi@S@details
  \else\ifx\EmFi@temp\EmFi@S@details
    \let\EmFi@view\EmFi@S@details
  \else\ifx\EmFi@temp\EmFi@S@tile
    \let\EmFi@view\EmFi@S@tile
  \else\ifx\EmFi@temp\EmFi@S@hidden
    \let\EmFi@view\EmFi@S@hidden
  \else
    \let\EmFi@next\relax
    \EmFi@Error{%
      Unknown value `\EmFi@temp' for key `view'.\MessageBreak
      Supported values: `details', `tile', `hidden'.%
    }\@ehc
  \fi\fi\fi\fi
  \EmFi@next
}
%    \end{macrocode}
%    \begin{macrocode}
\EmFi@DefineKey{initialfile}{}
%    \end{macrocode}
%
%    \begin{macro}{\embedfilesetup}
%    \begin{macrocode}
\def\embedfilesetup{%
  \ifEmFi@finished
    \def\EmFi@next##1{}%
    \EmFi@Error{%
      \string\embedfilefield\ltx@space after \string\embedfilefinish
    }{%
      The list of embedded files is already written.%
    }%
  \else
    \def\EmFi@next{%
      \kvsetkeys{EmFi}%
    }%
  \fi
  \EmFi@next
}
%    \end{macrocode}
%    \end{macro}
%
%    \begin{macro}{\EmFi@schema}
%    \begin{macrocode}
\def\EmFi@schema{}
%    \end{macrocode}
%    \end{macro}
%    \begin{macro}{\EmFi@order}
%    \begin{macrocode}
\gdef\EmFi@order{0}
%    \end{macrocode}
%    \end{macro}
%    \begin{macro}{\EmFi@@order}
%    \begin{macrocode}
\let\EmFi@@order\relax
%    \end{macrocode}
%    \end{macro}
%    \begin{macro}{\EmFi@fieldlist}
%    \begin{macrocode}
\def\EmFi@fieldlist{}
%    \end{macrocode}
%    \end{macro}
%    \begin{macro}{\EmFi@sortcase}
%    \begin{macrocode}
\def\EmFi@sortcase{0}%
%    \end{macrocode}
%    \end{macro}
%    \begin{macro}{\embedfilefield}
%    \begin{macrocode}
\def\embedfilefield#1#2{%
  \ifEmFi@finished
    \EmFi@Error{%
      \string\embedfilefield\ltx@space after \string\embedfilefinish
    }{%
      The list of embedded files is already written.%
    }%
  \else
    \global\EmFi@collectiontrue
    \EdefSanitize\EmFi@key{#1}%
    \expandafter\ifx\csname KV@EmFi@\EmFi@key.prefix\endcsname\relax
      \begingroup
        \count@=\EmFi@order
        \advance\count@ 1 %
        \xdef\EmFi@order{\the\count@}%
        \let\EmFi@title\EmFi@key
        \let\EmFi@type\EmFi@S@text
        \EmFi@visibletrue
        \EmFi@editfalse
        \kvsetkeys{EmFiFi}{#2}%
        \EmFi@convert\EmFi@title\EmFi@title
        \xdef\EmFi@schema{%
          \EmFi@schema
          /\pdf@escapename{\EmFi@key}<<%
            /Subtype/%
            \ifx\EmFi@type\EmFi@S@date D%
            \else\ifx\EmFi@type\EmFi@S@number N%
            \else\ifx\EmFi@type\EmFi@S@file F%
            \else\ifx\EmFi@type\EmFi@S@desc Desc%
            \else\ifx\EmFi@type\EmFi@S@moddate ModDate%
            \else\ifx\EmFi@type\EmFi@S@creationdate CreationDate%
            \else\ifx\EmFi@type\EmFi@S@size Size%
            \else S%
            \fi\fi\fi\fi\fi\fi\fi
            /N(\EmFi@title)%
            \EmFi@@order{\EmFi@order}%
            \ifEmFi@visible
            \else
              /V false%
            \fi
            \ifEmFi@edit
              /E true%
            \fi
          >>%
        }%
        \let\do\relax
        \xdef\EmFi@fieldlist{%
          \EmFi@fieldlist
          \do{\EmFi@key}%
        }%
        \ifx\EmFi@type\EmFi@S@text
          \kv@define@key{EmFi}{\EmFi@key.value}{%
            \EmFi@itemtrue
            \def\EmFi@temp{##1}%
            \EmFi@convert\EmFi@temp\EmFi@temp
            \expandafter\def\csname EmFi@V@#1%
            \expandafter\endcsname\expandafter{%
              \expandafter(\EmFi@temp)%
            }%
          }%
          \EmFi@GlobalKey{EmFi}{\EmFi@key.value}%
        \else\ifx\EmFi@type\EmFi@S@date
          \kv@define@key{EmFi}{\EmFi@key.value}{%
            \EmFi@itemtrue
            \def\EmFi@temp{##1}%
            \EmFi@convert\EmFi@temp\EmFi@temp
            \expandafter\def\csname EmFi@V@#1%
            \expandafter\endcsname\expandafter{%
              \expandafter(\EmFi@temp)%
            }%
          }%
          \EmFi@GlobalKey{EmFi}{\EmFi@key.value}%
        \else\ifx\EmFi@type\EmFi@S@number
          \kv@define@key{EmFi}{\EmFi@key.value}{%
            \EmFi@itemtrue
            \expandafter\EdefSanitize\csname EmFi@V@#1\endcsname{ ##1}%
          }%
          \EmFi@GlobalKey{EmFi}{\EmFi@key.value}%
        \fi\fi\fi
        \kv@define@key{EmFi}{\EmFi@key.prefix}{%
          \EmFi@itemtrue
          \expandafter\def\csname EmFi@P@#1\endcsname{##1}%
        }%
        \EmFi@GlobalKey{EmFi}{\EmFi@key.prefix}%
        \kv@define@key{EmFiSo}{\EmFi@key}[ascending]{%
          \EdefSanitize\EmFi@temp{##1}%
          \ifx\EmFi@temp\EmFi@S@ascending
            \def\EmFi@temp{true}%
          \else\ifx\EmFi@temp\EmFi@S@descending
            \def\EmFi@temp{false}%
          \else
            \def\EmFi@temp{}%
            \EmFi@Error{%
              Unknown sort order `\EmFi@temp'.\MessageBreak
              Supported values: `\EmFi@S@ascending', %
              `\EmFi@S@descending
            }\@ehc
          \fi\fi
          \ifx\EmFi@temp\ltx@empty
          \else
            \xdef\EmFi@sortkeys{%
              \EmFi@sortkeys
              /\pdf@escapename{#1}%
            }%
            \ifx\EmFi@sortorders\ltx@empty
              \global\let\EmFi@sortorders\EmFi@temp
              \gdef\EmFi@sortcase{1}%
            \else
              \xdef\EmFi@sortorders{%
                \EmFi@sortorders
                \ltx@space
                \EmFi@temp
              }%
              \xdef\EmFi@sortcase{2}%
            \fi
          \fi
        }%
        \EmFi@GlobalDefaultKey{EmFiSo}\EmFi@key
      \endgroup
    \else
      \EmFi@Error{%
        Field `\EmFi@key' is already defined%
      }\@ehc
    \fi
  \fi
}
%    \end{macrocode}
%    \end{macro}
%    \begin{macrocode}
\kv@define@key{EmFiFi}{type}{%
  \EdefSanitize\EmFi@temp{#1}%
  \ifx\EmFi@temp\EmFi@S@text
    \let\EmFi@type\EmFi@temp
  \else\ifx\EmFi@temp\EmFi@S@date
    \let\EmFi@type\EmFi@temp
  \else\ifx\EmFi@temp\EmFi@S@number
    \let\EmFi@type\EmFi@temp
  \else\ifx\EmFi@temp\EmFi@S@file
    \let\EmFi@type\EmFi@temp
  \else\ifx\EmFi@temp\EmFi@S@desc
    \let\EmFi@type\EmFi@temp
  \else\ifx\EmFi@temp\EmFi@S@moddate
    \let\EmFi@type\EmFi@temp
  \else\ifx\EmFi@temp\EmFi@S@creationdate
    \let\EmFi@type\EmFi@temp
  \else\ifx\EmFi@temp\EmFi@S@size
    \let\EmFi@type\EmFi@temp
  \else
    \EmFi@Error{%
      Unknown type `\EmFi@temp'.\MessageBreak
      Supported types: `text', `date', `number', `file',\MessageBreak
      `desc', `moddate', `creationdate', `size'%
    }%
  \fi\fi\fi\fi\fi\fi\fi\fi
}
%    \end{macrocode}
%    \begin{macrocode}
\kv@define@key{EmFiFi}{title}{%
  \def\EmFi@title{#1}%
}
%    \end{macrocode}
%    \begin{macro}{\EmFi@setboolean}
%    \begin{macrocode}
\def\EmFi@setboolean#1#2{%
  \EdefSanitize\EmFi@temp{#2}%
  \ifx\EmFi@temp\EmFi@S@true
    \csname EmFi@#1true\endcsname
  \else
    \ifx\EmFi@temp\EmFi@S@false
      \csname EmFi@#1false\endcsname
    \else
      \EmFi@Error{%
        Unknown value `\EmFi@temp' for key `#1'.\MessageBreak
        Supported values: `true', `false'%
      }\@ehc
    \fi
  \fi
}
%    \end{macrocode}
%    \end{macro}
%    \begin{macrocode}
\kv@define@key{EmFiFi}{visible}[true]{%
  \EmFi@setboolean{visible}{#1}%
}
%    \end{macrocode}
%    \begin{macrocode}
\kv@define@key{EmFiFi}{edit}[true]{%
  \EmFi@setboolean{edit}{#1}%
}
%    \end{macrocode}
%
%    \begin{macro}{\EmFi@sortkeys}
%    \begin{macrocode}
\def\EmFi@sortkeys{}
%    \end{macrocode}
%    \end{macro}
%    \begin{macro}{\EmFi@sortorders}
%    \begin{macrocode}
\def\EmFi@sortorders{}
%    \end{macrocode}
%    \end{macro}
%    \begin{macro}{\embedfilesort}
%    \begin{macrocode}
\def\embedfilesort{%
  \kvsetkeys{EmFiSo}%
}
%    \end{macrocode}
%    \end{macro}
%
% \subsection{Embed the file}
%
%    \begin{macro}{\embedfile}
%    \begin{macrocode}
\def\embedfile{%
  \ltx@ifnextchar[\EmFi@embedfile{\EmFi@embedfile[]}%
}
%    \end{macrocode}
%    \end{macro}
%    \begin{macro}{\EmFi@embedfile}
%    \begin{macrocode}
\def\EmFi@embedfile[#1]#2{%
  \ifEmFi@finished
    \EmFi@Error{%
      \string\embedfile\ltx@space after \string\embedfilefinish
    }{%
      The list of embedded files is already written.%
    }%
  \else
    \begingroup
      \def\EmFi@file{#2}%
      \kvsetkeys{EmFi}{#1}%
      \expandafter\expandafter\expandafter
      \ifx\expandafter\expandafter\expandafter
          \\\pdf@filesize{\EmFi@file}\\%
        \EmFi@Error{%
          File `\EmFi@file' not found%
        }{%
          The unknown file is not embedded.%
        }%
      \else
        \edef\EmFi@@filespec{%
          \pdf@escapestring{\EmFi@filespec}%
        }%
        \ifx\EmFi@ucfilespec\ltx@empty
          \let\EmFi@@ucfilespec\ltx@empty
        \else
          \EmFi@convert\EmFi@ucfilespec\EmFi@@ucfilespec
        \fi
        \ifx\EmFi@desc\ltx@empty
          \let\EmFi@@desc\ltx@empty
        \else
          \EmFi@convert\EmFi@desc\EmFi@@desc
        \fi
        \ifEmFi@item
          \let\do\EmFi@do
          \immediate\pdfobj{%
            <<%
              \EmFi@fieldlist
            >>%
          }%
          \edef\EmFi@ci{\the\pdflastobj}%
        \fi
        \immediate\pdfobj stream attr{%
          /Type/EmbeddedFile%
          \ifx\EmFi@mimetype\ltx@empty
          \else
            /Subtype/\pdf@escapename{\EmFi@mimetype}%
          \fi
          /Params<<%
            /ModDate(\pdf@filemoddate{\EmFi@file})%
            /Size \pdf@filesize{\EmFi@file}%
            /CheckSum<\pdf@filemdfivesum{\EmFi@file}>%
          >>%
        }file{\EmFi@file}\relax
        \EmFi@defobj{EmbeddedFile}%
        \immediate\pdfobj{%
          <<%
            /Type/Filespec%
            \ifx\EmFi@filesystem\ltx@empty
            \else
            /FS/\pdf@escapename{\EmFi@filesystem}%
            \fi
            /F(\EmFi@@filespec)%
            \ifx\EmFi@@ucfilespec\ltx@empty
            \else
              /UF(\EmFi@@ucfilespec)%
            \fi
            \ifx\EmFi@@desc\ltx@empty
            \else
              /Desc(\EmFi@@desc)%
            \fi
            /EF<<%
              /F \the\pdflastobj\ltx@space 0 R%
            >>%
            \ifEmFi@item
              /CI \EmFi@ci\ltx@space 0 R%
            \fi
          >>%
        }%
        \EmFi@defobj{Filespec}%
        \EmFi@add{%
          \EmFi@@filespec
        }{\the\pdflastobj\ltx@space 0 R}%
      \fi
    \endgroup
  \fi
}
%    \end{macrocode}
%    \end{macro}
%    \begin{macro}{\EmFi@do}
%    \begin{macrocode}
\def\EmFi@do#1{%
  \expandafter\ifx\csname EmFi@P@#1\endcsname\relax
    \expandafter\ifx\csname EmFi@V@#1\endcsname\relax
    \else
      /\pdf@escapename{#1}\csname EmFi@V@#1\endcsname
    \fi
  \else
    /\pdf@escapename{#1}<<%
      \expandafter\ifx\csname EmFi@V@#1\endcsname\relax
      \else
        /D\csname EmFi@V@#1\endcsname
      \fi
      /P(\csname EmFi@P@#1\endcsname)%
    >>%
  \fi
}
%    \end{macrocode}
%    \end{macro}
%    \begin{macro}{\EmFi@convert}
%    \begin{macrocode}
\def\EmFi@convert#1#2{%
  \ifnum\pdf@strcmp{\EmFi@stringmethod}{psd}=0 %
    \pdfstringdef\EmFi@temp{#1}%
    \let#2\EmFi@temp
  \else
    \edef#2{\pdf@escapestring{#1}}%
  \fi
}
%    \end{macrocode}
%    \end{macro}
%    \begin{macrocode}
\global\let\EmFi@list\ltx@empty
%    \end{macrocode}
%    \begin{macro}{\EmFi@add}
%    Sorting is done by the insertion sort algorithm.
%    Probably the sorting could be done more reliable.
%    However, the PDF specification is not too clear to me
%    regarding precise sorting rules (how to deal with
%    different encodings, escaped characters, \dots).
%    \begin{macrocode}
\def\EmFi@add#1#2{%
  \begingroup
    \ifx\EmFi@list\ltx@empty
      \xdef\EmFi@list{\noexpand\do{#1}{#2}}%
    \else
      \def\do##1##2{%
        \ifnum\pdf@strcmp{##1}{#1}>0 %
          \edef\x{%
            \toks@{%
              \the\toks@%
              \noexpand\do{#1}{#2}%
              \noexpand\do{##1}{##2}%
            }%
          }%
          \x
          \def\do####1####2{%
            \toks@\expandafter{\the\toks@\do{####1}{####2}}%
          }%
          \def\stop{%
            \xdef\EmFi@list{\the\toks@}%
          }%
        \else
          \toks@\expandafter{\the\toks@\do{##1}{##2}}%
        \fi
      }%
      \def\stop{%
        \xdef\EmFi@list{\the\toks@\noexpand\do{#1}{#2}}%
      }%
      \toks@{}%
      \EmFi@list\stop
    \fi
  \endgroup
}
%    \end{macrocode}
%    \end{macro}
%    \begin{macro}{\embedfilefinish}
%    \begin{macrocode}
\def\embedfilefinish{%
  \ifEmFi@finished
    \EmFi@Error{%
      Too many invocations of \string\embedfilefinish
    }{%
      The list of embedded files is already written.%
    }%
  \else
    \ifx\EmFi@list\ltx@empty
    \else
%    \end{macrocode}
%    Write |/EmbeddedFiles| entry.
%    \begin{macrocode}
      \global\EmFi@finishedtrue
      \begingroup
        \def\do##1##2{%
          (##1)##2%
        }%
        \immediate\pdfobj{%
          <<%
            /Names[\EmFi@list]%
          >>%
        }%
        \pdfnames{%
          /EmbeddedFiles \the\pdflastobj\ltx@space 0 R%
        }%
      \endgroup
%    \end{macrocode}
%    Write collection objects.
%    \begin{macrocode}
      \ifx\EmFi@initialfile\ltx@empty
      \else
        \EmFi@collectiontrue
      \fi
      \ifEmFi@collection
        \ifx\EmFi@initialfile\ltx@empty
          \let\EmFi@@initialfile\ltx@empty
        \else
          \edef\EmFi@@initialfile{%
            \pdf@escapestring{\EmFi@initialfile}%
          }%
        \fi
%    \end{macrocode}
%    Look for initial file among the embedded files.
%    \begin{macrocode}
        \begingroup
          \let\f=N%
          \def\do##1##2{%
            \def\x{##1}%
            \ifx\x\EmFi@@initialfile
              \let\f=Y%
              \let\do\ltx@gobbletwo
            \fi
          }%
          \EmFi@list
        \expandafter\endgroup
        \ifx\f Y%
        \else
          \@PackageWarningNoLine{embedfile}{%
            Missing initial file `\EmFi@initialfile'\MessageBreak
            among the embedded files%
          }%
          \let\EmFi@initialfile\ltx@empty
          \let\EmFi@@initialfile\ltx@empty
        \fi
        \ifcase\EmFi@sortcase
          \def\EmFi@temp{}%
        \or
          \def\EmFi@temp{%
            /S\EmFi@sortkeys
            /A \EmFi@sortorders
          }%
        \else
          \def\EmFi@temp{%
            /S[\EmFi@sortkeys]%
            /A[\EmFi@sortorders]%
          }%
        \fi
        \def\EmFi@@order##1{%
          \ifnum\EmFi@order>1 %
            /O ##1%
          \fi
        }%
        \immediate\pdfobj{%
          <<%
            \ifx\EmFi@schema\ltx@empty
            \else
              /Schema<<\EmFi@schema>>%
            \fi
            \ifx\EmFi@@initialfile\ltx@empty
            \else
              /D(\EmFi@@initialfile)%
            \fi
            \ifx\EmFi@view\EmFi@S@tile
              /View/T%
            \else\ifx\EmFi@view\EmFi@S@hidden
              /View/H%
            \fi\fi
            \ifx\EmFi@temp\ltx@empty
              \EmFi@temp
            \else
              /Sort<<\EmFi@temp>>%
            \fi
          >>%
        }%
        \pdfcatalog{%
          /Collection \the\pdflastobj\ltx@space0 R%
        }%
      \fi
    \fi
  \fi
}
%    \end{macrocode}
%    \end{macro}
%    \begin{macrocode}
\begingroup\expandafter\expandafter\expandafter\endgroup
\expandafter\ifx\csname AtEndDocument\endcsname\relax
\else
  \AtEndDocument{\embedfilefinish}%
\fi
%    \end{macrocode}
%
%    \begin{macrocode}
\EmFi@AtEnd%
%</package>
%    \end{macrocode}
%
% \section{Test}
%
% \subsection{Catcode checks for loading}
%
%    \begin{macrocode}
%<*test1>
%    \end{macrocode}
%    \begin{macrocode}
\catcode`\{=1 %
\catcode`\}=2 %
\catcode`\#=6 %
\catcode`\@=11 %
\expandafter\ifx\csname count@\endcsname\relax
  \countdef\count@=255 %
\fi
\expandafter\ifx\csname @gobble\endcsname\relax
  \long\def\@gobble#1{}%
\fi
\expandafter\ifx\csname @firstofone\endcsname\relax
  \long\def\@firstofone#1{#1}%
\fi
\expandafter\ifx\csname loop\endcsname\relax
  \expandafter\@firstofone
\else
  \expandafter\@gobble
\fi
{%
  \def\loop#1\repeat{%
    \def\body{#1}%
    \iterate
  }%
  \def\iterate{%
    \body
      \let\next\iterate
    \else
      \let\next\relax
    \fi
    \next
  }%
  \let\repeat=\fi
}%
\def\RestoreCatcodes{}
\count@=0 %
\loop
  \edef\RestoreCatcodes{%
    \RestoreCatcodes
    \catcode\the\count@=\the\catcode\count@\relax
  }%
\ifnum\count@<255 %
  \advance\count@ 1 %
\repeat

\def\RangeCatcodeInvalid#1#2{%
  \count@=#1\relax
  \loop
    \catcode\count@=15 %
  \ifnum\count@<#2\relax
    \advance\count@ 1 %
  \repeat
}
\def\RangeCatcodeCheck#1#2#3{%
  \count@=#1\relax
  \loop
    \ifnum#3=\catcode\count@
    \else
      \errmessage{%
        Character \the\count@\space
        with wrong catcode \the\catcode\count@\space
        instead of \number#3%
      }%
    \fi
  \ifnum\count@<#2\relax
    \advance\count@ 1 %
  \repeat
}
\def\space{ }
\expandafter\ifx\csname LoadCommand\endcsname\relax
  \def\LoadCommand{\input embedfile.sty\relax}%
\fi
\def\Test{%
  \RangeCatcodeInvalid{0}{47}%
  \RangeCatcodeInvalid{58}{64}%
  \RangeCatcodeInvalid{91}{96}%
  \RangeCatcodeInvalid{123}{255}%
  \catcode`\@=12 %
  \catcode`\\=0 %
  \catcode`\%=14 %
  \LoadCommand
  \RangeCatcodeCheck{0}{36}{15}%
  \RangeCatcodeCheck{37}{37}{14}%
  \RangeCatcodeCheck{38}{47}{15}%
  \RangeCatcodeCheck{48}{57}{12}%
  \RangeCatcodeCheck{58}{63}{15}%
  \RangeCatcodeCheck{64}{64}{12}%
  \RangeCatcodeCheck{65}{90}{11}%
  \RangeCatcodeCheck{91}{91}{15}%
  \RangeCatcodeCheck{92}{92}{0}%
  \RangeCatcodeCheck{93}{96}{15}%
  \RangeCatcodeCheck{97}{122}{11}%
  \RangeCatcodeCheck{123}{255}{15}%
  \RestoreCatcodes
}
\Test
\csname @@end\endcsname
\end
%    \end{macrocode}
%    \begin{macrocode}
%</test1>
%    \end{macrocode}
%
% \subsection{Simple test}
%
%    \begin{macrocode}
%<*test2>
\input embedfile.sty\relax
\embedfile[%
  stringmethod=escape,%
  mimetype=plain/text,%
  desc={LaTeX docstrip source archive for package `embedfile'},%
  id={embedfile.dtx}%
]{embedfile.dtx}
\nopagenumbers
Test (plain-\TeX): {\tt embedfile.dtx} should be embedded.%

\def\Test#1{%
  \par
  \embedfileifobjectexists{embedfile.dtx}{#1}{%
    Object #1 (embedfile.dtx): %
    \embedfilegetobject{embedfile.dtx}{#1}%
  }{%
    \errmessage{Missing object #1 (embedfile.dtx)}%
  }%
}
\Test{EmbeddedFile}
\Test{Filespec}
\embedfilefinish
\bye
%</test2>
%    \end{macrocode}
%    \begin{macrocode}
%<*test3>
\NeedsTeXFormat{LaTeX2e}
\let\SavedJobname\jobname
\def\jobname{embedfile}
\RequirePackage{dtx-attach}[2011/04/13]
\let\jobname\SavedJobname
\documentclass{minimal}
\begin{document}
  Test (\LaTeX): \texttt{embedfile.dtx} should be embedded.%
\end{document}
%</test3>
%    \end{macrocode}
%
% \subsection{Test for \iniTeX}
%
%    \begin{macrocode}
%<*test4>
\catcode`\{=1 %
\catcode`\}=2 %
\input ifluatex.sty %
\ifluatex
  \directlua{%
    tex.enableprimitives('', {%
      'pdflastobj',%
      'pdfnames',%
      'pdfobj',%
      'pdfoutput'%
    })%
  }%
\fi
\pdfoutput=1 %
\input embedfile.sty %
\shipout\hbox{}
\embedfile[%
  stringmethod=escape,%
  mimetype=plain/text,%
  desc={iniTeX source},%
]{\jobname.tex}
\embedfilefinish
\end
%</test4>
%    \end{macrocode}
%
% \section{Installation}
%
% \subsection{Download}
%
% \paragraph{Package.} This package is available on
% CTAN\footnote{\url{ftp://ftp.ctan.org/tex-archive/}}:
% \begin{description}
% \item[\CTAN{macros/latex/contrib/oberdiek/embedfile.dtx}] The source file.
% \item[\CTAN{macros/latex/contrib/oberdiek/embedfile.pdf}] Documentation.
% \end{description}
%
%
% \paragraph{Bundle.} All the packages of the bundle `oberdiek'
% are also available in a TDS compliant ZIP archive. There
% the packages are already unpacked and the documentation files
% are generated. The files and directories obey the TDS standard.
% \begin{description}
% \item[\CTAN{install/macros/latex/contrib/oberdiek.tds.zip}]
% \end{description}
% \emph{TDS} refers to the standard ``A Directory Structure
% for \TeX\ Files'' (\CTAN{tds/tds.pdf}). Directories
% with \xfile{texmf} in their name are usually organized this way.
%
% \subsection{Bundle installation}
%
% \paragraph{Unpacking.} Unpack the \xfile{oberdiek.tds.zip} in the
% TDS tree (also known as \xfile{texmf} tree) of your choice.
% Example (linux):
% \begin{quote}
%   |unzip oberdiek.tds.zip -d ~/texmf|
% \end{quote}
%
% \paragraph{Script installation.}
% Check the directory \xfile{TDS:scripts/oberdiek/} for
% scripts that need further installation steps.
% Package \xpackage{attachfile2} comes with the Perl script
% \xfile{pdfatfi.pl} that should be installed in such a way
% that it can be called as \texttt{pdfatfi}.
% Example (linux):
% \begin{quote}
%   |chmod +x scripts/oberdiek/pdfatfi.pl|\\
%   |cp scripts/oberdiek/pdfatfi.pl /usr/local/bin/|
% \end{quote}
%
% \subsection{Package installation}
%
% \paragraph{Unpacking.} The \xfile{.dtx} file is a self-extracting
% \docstrip\ archive. The files are extracted by running the
% \xfile{.dtx} through \plainTeX:
% \begin{quote}
%   \verb|tex embedfile.dtx|
% \end{quote}
%
% \paragraph{TDS.} Now the different files must be moved into
% the different directories in your installation TDS tree
% (also known as \xfile{texmf} tree):
% \begin{quote}
% \def\t{^^A
% \begin{tabular}{@{}>{\ttfamily}l@{ $\rightarrow$ }>{\ttfamily}l@{}}
%   embedfile.sty & tex/generic/oberdiek/embedfile.sty\\
%   dtx-attach.sty & tex/generic/oberdiek/dtx-attach.sty\\
%   embedfile.pdf & doc/latex/oberdiek/embedfile.pdf\\
%   embedfile-example-plain.tex & doc/latex/oberdiek/embedfile-example-plain.tex\\
%   embedfile-example-collection.tex & doc/latex/oberdiek/embedfile-example-collection.tex\\
%   test/embedfile-test1.tex & doc/latex/oberdiek/test/embedfile-test1.tex\\
%   test/embedfile-test2.tex & doc/latex/oberdiek/test/embedfile-test2.tex\\
%   test/embedfile-test3.tex & doc/latex/oberdiek/test/embedfile-test3.tex\\
%   test/embedfile-test4.tex & doc/latex/oberdiek/test/embedfile-test4.tex\\
%   embedfile.dtx & source/latex/oberdiek/embedfile.dtx\\
% \end{tabular}^^A
% }^^A
% \sbox0{\t}^^A
% \ifdim\wd0>\linewidth
%   \begingroup
%     \advance\linewidth by\leftmargin
%     \advance\linewidth by\rightmargin
%   \edef\x{\endgroup
%     \def\noexpand\lw{\the\linewidth}^^A
%   }\x
%   \def\lwbox{^^A
%     \leavevmode
%     \hbox to \linewidth{^^A
%       \kern-\leftmargin\relax
%       \hss
%       \usebox0
%       \hss
%       \kern-\rightmargin\relax
%     }^^A
%   }^^A
%   \ifdim\wd0>\lw
%     \sbox0{\small\t}^^A
%     \ifdim\wd0>\linewidth
%       \ifdim\wd0>\lw
%         \sbox0{\footnotesize\t}^^A
%         \ifdim\wd0>\linewidth
%           \ifdim\wd0>\lw
%             \sbox0{\scriptsize\t}^^A
%             \ifdim\wd0>\linewidth
%               \ifdim\wd0>\lw
%                 \sbox0{\tiny\t}^^A
%                 \ifdim\wd0>\linewidth
%                   \lwbox
%                 \else
%                   \usebox0
%                 \fi
%               \else
%                 \lwbox
%               \fi
%             \else
%               \usebox0
%             \fi
%           \else
%             \lwbox
%           \fi
%         \else
%           \usebox0
%         \fi
%       \else
%         \lwbox
%       \fi
%     \else
%       \usebox0
%     \fi
%   \else
%     \lwbox
%   \fi
% \else
%   \usebox0
% \fi
% \end{quote}
% If you have a \xfile{docstrip.cfg} that configures and enables \docstrip's
% TDS installing feature, then some files can already be in the right
% place, see the documentation of \docstrip.
%
% \subsection{Refresh file name databases}
%
% If your \TeX~distribution
% (\teTeX, \mikTeX, \dots) relies on file name databases, you must refresh
% these. For example, \teTeX\ users run \verb|texhash| or
% \verb|mktexlsr|.
%
% \subsection{Some details for the interested}
%
% \paragraph{Attached source.}
%
% The PDF documentation on CTAN also includes the
% \xfile{.dtx} source file. It can be extracted by
% AcrobatReader 6 or higher. Another option is \textsf{pdftk},
% e.g. unpack the file into the current directory:
% \begin{quote}
%   \verb|pdftk embedfile.pdf unpack_files output .|
% \end{quote}
%
% \paragraph{Unpacking with \LaTeX.}
% The \xfile{.dtx} chooses its action depending on the format:
% \begin{description}
% \item[\plainTeX:] Run \docstrip\ and extract the files.
% \item[\LaTeX:] Generate the documentation.
% \end{description}
% If you insist on using \LaTeX\ for \docstrip\ (really,
% \docstrip\ does not need \LaTeX), then inform the autodetect routine
% about your intention:
% \begin{quote}
%   \verb|latex \let\install=y\input{embedfile.dtx}|
% \end{quote}
% Do not forget to quote the argument according to the demands
% of your shell.
%
% \paragraph{Generating the documentation.}
% You can use both the \xfile{.dtx} or the \xfile{.drv} to generate
% the documentation. The process can be configured by the
% configuration file \xfile{ltxdoc.cfg}. For instance, put this
% line into this file, if you want to have A4 as paper format:
% \begin{quote}
%   \verb|\PassOptionsToClass{a4paper}{article}|
% \end{quote}
% An example follows how to generate the
% documentation with pdf\LaTeX:
% \begin{quote}
%\begin{verbatim}
%pdflatex embedfile.dtx
%makeindex -s gind.ist embedfile.idx
%pdflatex embedfile.dtx
%makeindex -s gind.ist embedfile.idx
%pdflatex embedfile.dtx
%\end{verbatim}
% \end{quote}
%
% \section{Catalogue}
%
% The following XML file can be used as source for the
% \href{http://mirror.ctan.org/help/Catalogue/catalogue.html}{\TeX\ Catalogue}.
% The elements \texttt{caption} and \texttt{description} are imported
% from the original XML file from the Catalogue.
% The name of the XML file in the Catalogue is \xfile{embedfile.xml}.
%    \begin{macrocode}
%<*catalogue>
<?xml version='1.0' encoding='us-ascii'?>
<!DOCTYPE entry SYSTEM 'catalogue.dtd'>
<entry datestamp='$Date$' modifier='$Author$' id='embedfile'>
  <name>embedfile</name>
  <caption>Embed files into PDF.</caption>
  <authorref id='auth:oberdiek'/>
  <copyright owner='Heiko Oberdiek' year='2006-2011'/>
  <license type='lppl1.3'/>
  <version number='2.6'/>
  <description>
    This package embeds files in a PDF document, using the PDF
    format's embedding operation (note the contrast with the attach
    operation used by the <xref refid='attachfile'>attachfile</xref>
    and <xref refid='attachfile2'>attachfile2</xref> packages).
    Currently only <xref refid='pdftex'>pdfTeX</xref> &gt;=1.30, in
    PDF mode, is supported.

    <p/>
    The package is part of the <xref refid='oberdiek'>oberdiek</xref>
    bundle.
  </description>
  <documentation details='Package documentation'
      href='ctan:/macros/latex/contrib/oberdiek/embedfile.pdf'/>
  <ctan file='true' path='/macros/latex/contrib/oberdiek/embedfile.dtx'/>
  <miktex location='oberdiek'/>
  <texlive location='oberdiek'/>
  <install path='/macros/latex/contrib/oberdiek/oberdiek.tds.zip'/>
</entry>
%</catalogue>
%    \end{macrocode}
%
% \begin{thebibliography}{9}
% \bibitem{attachfile}
%   Scott Pakin: \textit{The \xpackage{attachfile} package};
%   2005/02/20 v1.2;
%   \CTAN{macros/latex/contrib/attachfile/}.
%
% \bibitem{attachfile2}
%   Heiko Oberdiek: \textit{The \xpackage{attachfile2} package};
%   2006/08/16 v2.2;
%   \CTAN{macros/latex/contrib/oberdiek/attachfile2.pdf}.
%
% \bibitem{pdfspec}
%   Adobe Systems Incorporated:
%   \href{http://www.adobe.com/devnet/acrobat/pdfs/pdf_reference.pdf}%
%       {\textit{PDF Reference, Sixth Edition, Version 1.7}},%
%   Oktober 2006;
%   \url{http://www.adobe.com/devnet/pdf/pdf_reference.html}.
%
% \bibitem{mime}
%   Network Working Group: RFC 2046, \textit{Multipurpose Internet Mail Extensions (MIME) Part Two:
%     Media Types}, November 1996; \url{http://www.rfc-editor.org/}.
%
% \bibitem{types}
%   IANA (Internet Assigned Numbers Authority):
%   \textit{MIME Media Types}, May 2006;
%   \url{http://www.iana.org/assignments/media-types/}.
%
% \end{thebibliography}
%
% \begin{History}
%   \begin{Version}{2006/08/16 v1.0}
%   \item
%     First public version.
%   \end{Version}
%   \begin{Version}{2007/04/11 v1.1}
%   \item
%     Line ends sanitized.
%   \end{Version}
%   \begin{Version}{2007/09/09 v1.2}
%   \item
%     Fixes for plain-TeX, wrapper for package \xpackage{keyval} added.
%   \item
%     Catcode section rewritten.
%   \end{Version}
%   \begin{Version}{2007/10/28 v2.0}
%   \item
%     Collection support added (PDF 1.7).
%   \end{Version}
%   \begin{Version}{2007/10/29 v2.1}
%   \item
%     Export of object references by adding new option \xoption{id} and
%     new macros \cs{embedfileifobjectexists} and \cs{embedfilegetobject}.
%   \end{Version}
%   \begin{Version}{2007/11/11 v2.2}
%   \item
%     Use of package \xpackage{pdftexcmds} for \LuaTeX\ support.
%   \end{Version}
%   \begin{Version}{2007/11/25 v2.3}
%   \item
%     Fix in use of \cs{pdf@filesize}, bug introduced in previous version.
%   \end{Version}
%   \begin{Version}{2009/09/25 v2.4}
%   \item
%     Bug fix: If \xpackage{hyperref} is used with option \xoption{unicode},
%     the Unicode encoded file name causes trouble. Therefore \cs{pdfstringdef}
%     is now never used for option \xoption{filespec}, always method
%     \texttt{escape} is applied (Peter Cibulka).
%   \item
%     Bug fix for \xoption{initialfile}.
%   \item
%     Bug fix for file names in \texttt{/EmbeddedFiles}.
%   \item
%     New option \xoption{ucfilespec} for file name support in Unicode
%     (since PDF 1.7).
%   \end{Version}
%   \begin{Version}{2010/03/01 v2.5}
%   \item
%     Compatibility for \iniTeX.
%   \item
%     Package \xpackage{keyval} replaced by packages \xpackage{kvsetkeys}
%     and \xpackage{kvdefinekeys} because of compatibility for \iniTeX.
%   \item
%     TDS location moved from TDS:tex/latex/oberdiek/embedfile.sty to
%     TDS:tex/generic/oberdiek/embedfile.sty.
%   \end{Version}
%   \begin{Version}{2011/04/13 v2.6}
%   \item
%     Docu fixes (thanks Hans-Martin M\"unch).
%   \end{Version}
% \end{History}
%
% \PrintIndex
%
% \Finale
\endinput
|
% \end{quote}
% Do not forget to quote the argument according to the demands
% of your shell.
%
% \paragraph{Generating the documentation.}
% You can use both the \xfile{.dtx} or the \xfile{.drv} to generate
% the documentation. The process can be configured by the
% configuration file \xfile{ltxdoc.cfg}. For instance, put this
% line into this file, if you want to have A4 as paper format:
% \begin{quote}
%   \verb|\PassOptionsToClass{a4paper}{article}|
% \end{quote}
% An example follows how to generate the
% documentation with pdf\LaTeX:
% \begin{quote}
%\begin{verbatim}
%pdflatex embedfile.dtx
%makeindex -s gind.ist embedfile.idx
%pdflatex embedfile.dtx
%makeindex -s gind.ist embedfile.idx
%pdflatex embedfile.dtx
%\end{verbatim}
% \end{quote}
%
% \section{Catalogue}
%
% The following XML file can be used as source for the
% \href{http://mirror.ctan.org/help/Catalogue/catalogue.html}{\TeX\ Catalogue}.
% The elements \texttt{caption} and \texttt{description} are imported
% from the original XML file from the Catalogue.
% The name of the XML file in the Catalogue is \xfile{embedfile.xml}.
%    \begin{macrocode}
%<*catalogue>
<?xml version='1.0' encoding='us-ascii'?>
<!DOCTYPE entry SYSTEM 'catalogue.dtd'>
<entry datestamp='$Date$' modifier='$Author$' id='embedfile'>
  <name>embedfile</name>
  <caption>Embed files into PDF.</caption>
  <authorref id='auth:oberdiek'/>
  <copyright owner='Heiko Oberdiek' year='2006-2011'/>
  <license type='lppl1.3'/>
  <version number='2.6'/>
  <description>
    This package embeds files in a PDF document, using the PDF
    format's embedding operation (note the contrast with the attach
    operation used by the <xref refid='attachfile'>attachfile</xref>
    and <xref refid='attachfile2'>attachfile2</xref> packages).
    Currently only <xref refid='pdftex'>pdfTeX</xref> &gt;=1.30, in
    PDF mode, is supported.

    <p/>
    The package is part of the <xref refid='oberdiek'>oberdiek</xref>
    bundle.
  </description>
  <documentation details='Package documentation'
      href='ctan:/macros/latex/contrib/oberdiek/embedfile.pdf'/>
  <ctan file='true' path='/macros/latex/contrib/oberdiek/embedfile.dtx'/>
  <miktex location='oberdiek'/>
  <texlive location='oberdiek'/>
  <install path='/macros/latex/contrib/oberdiek/oberdiek.tds.zip'/>
</entry>
%</catalogue>
%    \end{macrocode}
%
% \begin{thebibliography}{9}
% \bibitem{attachfile}
%   Scott Pakin: \textit{The \xpackage{attachfile} package};
%   2005/02/20 v1.2;
%   \CTAN{macros/latex/contrib/attachfile/}.
%
% \bibitem{attachfile2}
%   Heiko Oberdiek: \textit{The \xpackage{attachfile2} package};
%   2006/08/16 v2.2;
%   \CTAN{macros/latex/contrib/oberdiek/attachfile2.pdf}.
%
% \bibitem{pdfspec}
%   Adobe Systems Incorporated:
%   \href{http://www.adobe.com/devnet/acrobat/pdfs/pdf_reference.pdf}%
%       {\textit{PDF Reference, Sixth Edition, Version 1.7}},%
%   Oktober 2006;
%   \url{http://www.adobe.com/devnet/pdf/pdf_reference.html}.
%
% \bibitem{mime}
%   Network Working Group: RFC 2046, \textit{Multipurpose Internet Mail Extensions (MIME) Part Two:
%     Media Types}, November 1996; \url{http://www.rfc-editor.org/}.
%
% \bibitem{types}
%   IANA (Internet Assigned Numbers Authority):
%   \textit{MIME Media Types}, May 2006;
%   \url{http://www.iana.org/assignments/media-types/}.
%
% \end{thebibliography}
%
% \begin{History}
%   \begin{Version}{2006/08/16 v1.0}
%   \item
%     First public version.
%   \end{Version}
%   \begin{Version}{2007/04/11 v1.1}
%   \item
%     Line ends sanitized.
%   \end{Version}
%   \begin{Version}{2007/09/09 v1.2}
%   \item
%     Fixes for plain-TeX, wrapper for package \xpackage{keyval} added.
%   \item
%     Catcode section rewritten.
%   \end{Version}
%   \begin{Version}{2007/10/28 v2.0}
%   \item
%     Collection support added (PDF 1.7).
%   \end{Version}
%   \begin{Version}{2007/10/29 v2.1}
%   \item
%     Export of object references by adding new option \xoption{id} and
%     new macros \cs{embedfileifobjectexists} and \cs{embedfilegetobject}.
%   \end{Version}
%   \begin{Version}{2007/11/11 v2.2}
%   \item
%     Use of package \xpackage{pdftexcmds} for \LuaTeX\ support.
%   \end{Version}
%   \begin{Version}{2007/11/25 v2.3}
%   \item
%     Fix in use of \cs{pdf@filesize}, bug introduced in previous version.
%   \end{Version}
%   \begin{Version}{2009/09/25 v2.4}
%   \item
%     Bug fix: If \xpackage{hyperref} is used with option \xoption{unicode},
%     the Unicode encoded file name causes trouble. Therefore \cs{pdfstringdef}
%     is now never used for option \xoption{filespec}, always method
%     \texttt{escape} is applied (Peter Cibulka).
%   \item
%     Bug fix for \xoption{initialfile}.
%   \item
%     Bug fix for file names in \texttt{/EmbeddedFiles}.
%   \item
%     New option \xoption{ucfilespec} for file name support in Unicode
%     (since PDF 1.7).
%   \end{Version}
%   \begin{Version}{2010/03/01 v2.5}
%   \item
%     Compatibility for \iniTeX.
%   \item
%     Package \xpackage{keyval} replaced by packages \xpackage{kvsetkeys}
%     and \xpackage{kvdefinekeys} because of compatibility for \iniTeX.
%   \item
%     TDS location moved from TDS:tex/latex/oberdiek/embedfile.sty to
%     TDS:tex/generic/oberdiek/embedfile.sty.
%   \end{Version}
%   \begin{Version}{2011/04/13 v2.6}
%   \item
%     Docu fixes (thanks Hans-Martin M\"unch).
%   \end{Version}
% \end{History}
%
% \PrintIndex
%
% \Finale
\endinput
|
% \end{quote}
% Do not forget to quote the argument according to the demands
% of your shell.
%
% \paragraph{Generating the documentation.}
% You can use both the \xfile{.dtx} or the \xfile{.drv} to generate
% the documentation. The process can be configured by the
% configuration file \xfile{ltxdoc.cfg}. For instance, put this
% line into this file, if you want to have A4 as paper format:
% \begin{quote}
%   \verb|\PassOptionsToClass{a4paper}{article}|
% \end{quote}
% An example follows how to generate the
% documentation with pdf\LaTeX:
% \begin{quote}
%\begin{verbatim}
%pdflatex embedfile.dtx
%makeindex -s gind.ist embedfile.idx
%pdflatex embedfile.dtx
%makeindex -s gind.ist embedfile.idx
%pdflatex embedfile.dtx
%\end{verbatim}
% \end{quote}
%
% \section{Catalogue}
%
% The following XML file can be used as source for the
% \href{http://mirror.ctan.org/help/Catalogue/catalogue.html}{\TeX\ Catalogue}.
% The elements \texttt{caption} and \texttt{description} are imported
% from the original XML file from the Catalogue.
% The name of the XML file in the Catalogue is \xfile{embedfile.xml}.
%    \begin{macrocode}
%<*catalogue>
<?xml version='1.0' encoding='us-ascii'?>
<!DOCTYPE entry SYSTEM 'catalogue.dtd'>
<entry datestamp='$Date$' modifier='$Author$' id='embedfile'>
  <name>embedfile</name>
  <caption>Embed files into PDF.</caption>
  <authorref id='auth:oberdiek'/>
  <copyright owner='Heiko Oberdiek' year='2006-2011'/>
  <license type='lppl1.3'/>
  <version number='2.6'/>
  <description>
    This package embeds files in a PDF document, using the PDF
    format's embedding operation (note the contrast with the attach
    operation used by the <xref refid='attachfile'>attachfile</xref>
    and <xref refid='attachfile2'>attachfile2</xref> packages).
    Currently only <xref refid='pdftex'>pdfTeX</xref> &gt;=1.30, in
    PDF mode, is supported.

    <p/>
    The package is part of the <xref refid='oberdiek'>oberdiek</xref>
    bundle.
  </description>
  <documentation details='Package documentation'
      href='ctan:/macros/latex/contrib/oberdiek/embedfile.pdf'/>
  <ctan file='true' path='/macros/latex/contrib/oberdiek/embedfile.dtx'/>
  <miktex location='oberdiek'/>
  <texlive location='oberdiek'/>
  <install path='/macros/latex/contrib/oberdiek/oberdiek.tds.zip'/>
</entry>
%</catalogue>
%    \end{macrocode}
%
% \begin{thebibliography}{9}
% \bibitem{attachfile}
%   Scott Pakin: \textit{The \xpackage{attachfile} package};
%   2005/02/20 v1.2;
%   \CTAN{macros/latex/contrib/attachfile/}.
%
% \bibitem{attachfile2}
%   Heiko Oberdiek: \textit{The \xpackage{attachfile2} package};
%   2006/08/16 v2.2;
%   \CTAN{macros/latex/contrib/oberdiek/attachfile2.pdf}.
%
% \bibitem{pdfspec}
%   Adobe Systems Incorporated:
%   \href{http://www.adobe.com/devnet/acrobat/pdfs/pdf_reference.pdf}%
%       {\textit{PDF Reference, Sixth Edition, Version 1.7}},%
%   Oktober 2006;
%   \url{http://www.adobe.com/devnet/pdf/pdf_reference.html}.
%
% \bibitem{mime}
%   Network Working Group: RFC 2046, \textit{Multipurpose Internet Mail Extensions (MIME) Part Two:
%     Media Types}, November 1996; \url{http://www.rfc-editor.org/}.
%
% \bibitem{types}
%   IANA (Internet Assigned Numbers Authority):
%   \textit{MIME Media Types}, May 2006;
%   \url{http://www.iana.org/assignments/media-types/}.
%
% \end{thebibliography}
%
% \begin{History}
%   \begin{Version}{2006/08/16 v1.0}
%   \item
%     First public version.
%   \end{Version}
%   \begin{Version}{2007/04/11 v1.1}
%   \item
%     Line ends sanitized.
%   \end{Version}
%   \begin{Version}{2007/09/09 v1.2}
%   \item
%     Fixes for plain-TeX, wrapper for package \xpackage{keyval} added.
%   \item
%     Catcode section rewritten.
%   \end{Version}
%   \begin{Version}{2007/10/28 v2.0}
%   \item
%     Collection support added (PDF 1.7).
%   \end{Version}
%   \begin{Version}{2007/10/29 v2.1}
%   \item
%     Export of object references by adding new option \xoption{id} and
%     new macros \cs{embedfileifobjectexists} and \cs{embedfilegetobject}.
%   \end{Version}
%   \begin{Version}{2007/11/11 v2.2}
%   \item
%     Use of package \xpackage{pdftexcmds} for \LuaTeX\ support.
%   \end{Version}
%   \begin{Version}{2007/11/25 v2.3}
%   \item
%     Fix in use of \cs{pdf@filesize}, bug introduced in previous version.
%   \end{Version}
%   \begin{Version}{2009/09/25 v2.4}
%   \item
%     Bug fix: If \xpackage{hyperref} is used with option \xoption{unicode},
%     the Unicode encoded file name causes trouble. Therefore \cs{pdfstringdef}
%     is now never used for option \xoption{filespec}, always method
%     \texttt{escape} is applied (Peter Cibulka).
%   \item
%     Bug fix for \xoption{initialfile}.
%   \item
%     Bug fix for file names in \texttt{/EmbeddedFiles}.
%   \item
%     New option \xoption{ucfilespec} for file name support in Unicode
%     (since PDF 1.7).
%   \end{Version}
%   \begin{Version}{2010/03/01 v2.5}
%   \item
%     Compatibility for \iniTeX.
%   \item
%     Package \xpackage{keyval} replaced by packages \xpackage{kvsetkeys}
%     and \xpackage{kvdefinekeys} because of compatibility for \iniTeX.
%   \item
%     TDS location moved from TDS:tex/latex/oberdiek/embedfile.sty to
%     TDS:tex/generic/oberdiek/embedfile.sty.
%   \end{Version}
%   \begin{Version}{2011/04/13 v2.6}
%   \item
%     Docu fixes (thanks Hans-Martin M\"unch).
%   \end{Version}
% \end{History}
%
% \PrintIndex
%
% \Finale
\endinput
|
% \end{quote}
% Do not forget to quote the argument according to the demands
% of your shell.
%
% \paragraph{Generating the documentation.}
% You can use both the \xfile{.dtx} or the \xfile{.drv} to generate
% the documentation. The process can be configured by the
% configuration file \xfile{ltxdoc.cfg}. For instance, put this
% line into this file, if you want to have A4 as paper format:
% \begin{quote}
%   \verb|\PassOptionsToClass{a4paper}{article}|
% \end{quote}
% An example follows how to generate the
% documentation with pdf\LaTeX:
% \begin{quote}
%\begin{verbatim}
%pdflatex embedfile.dtx
%makeindex -s gind.ist embedfile.idx
%pdflatex embedfile.dtx
%makeindex -s gind.ist embedfile.idx
%pdflatex embedfile.dtx
%\end{verbatim}
% \end{quote}
%
% \section{Catalogue}
%
% The following XML file can be used as source for the
% \href{http://mirror.ctan.org/help/Catalogue/catalogue.html}{\TeX\ Catalogue}.
% The elements \texttt{caption} and \texttt{description} are imported
% from the original XML file from the Catalogue.
% The name of the XML file in the Catalogue is \xfile{embedfile.xml}.
%    \begin{macrocode}
%<*catalogue>
<?xml version='1.0' encoding='us-ascii'?>
<!DOCTYPE entry SYSTEM 'catalogue.dtd'>
<entry datestamp='$Date$' modifier='$Author$' id='embedfile'>
  <name>embedfile</name>
  <caption>Embed files into PDF.</caption>
  <authorref id='auth:oberdiek'/>
  <copyright owner='Heiko Oberdiek' year='2006-2011'/>
  <license type='lppl1.3'/>
  <version number='2.6'/>
  <description>
    This package embeds files in a PDF document, using the PDF
    format's embedding operation (note the contrast with the attach
    operation used by the <xref refid='attachfile'>attachfile</xref>
    and <xref refid='attachfile2'>attachfile2</xref> packages).
    Currently only <xref refid='pdftex'>pdfTeX</xref> &gt;=1.30, in
    PDF mode, is supported.

    <p/>
    The package is part of the <xref refid='oberdiek'>oberdiek</xref>
    bundle.
  </description>
  <documentation details='Package documentation'
      href='ctan:/macros/latex/contrib/oberdiek/embedfile.pdf'/>
  <ctan file='true' path='/macros/latex/contrib/oberdiek/embedfile.dtx'/>
  <miktex location='oberdiek'/>
  <texlive location='oberdiek'/>
  <install path='/macros/latex/contrib/oberdiek/oberdiek.tds.zip'/>
</entry>
%</catalogue>
%    \end{macrocode}
%
% \begin{thebibliography}{9}
% \bibitem{attachfile}
%   Scott Pakin: \textit{The \xpackage{attachfile} package};
%   2005/02/20 v1.2;
%   \CTAN{macros/latex/contrib/attachfile/}.
%
% \bibitem{attachfile2}
%   Heiko Oberdiek: \textit{The \xpackage{attachfile2} package};
%   2006/08/16 v2.2;
%   \CTAN{macros/latex/contrib/oberdiek/attachfile2.pdf}.
%
% \bibitem{pdfspec}
%   Adobe Systems Incorporated:
%   \href{http://www.adobe.com/devnet/acrobat/pdfs/pdf_reference.pdf}%
%       {\textit{PDF Reference, Sixth Edition, Version 1.7}},%
%   Oktober 2006;
%   \url{http://www.adobe.com/devnet/pdf/pdf_reference.html}.
%
% \bibitem{mime}
%   Network Working Group: RFC 2046, \textit{Multipurpose Internet Mail Extensions (MIME) Part Two:
%     Media Types}, November 1996; \url{http://www.rfc-editor.org/}.
%
% \bibitem{types}
%   IANA (Internet Assigned Numbers Authority):
%   \textit{MIME Media Types}, May 2006;
%   \url{http://www.iana.org/assignments/media-types/}.
%
% \end{thebibliography}
%
% \begin{History}
%   \begin{Version}{2006/08/16 v1.0}
%   \item
%     First public version.
%   \end{Version}
%   \begin{Version}{2007/04/11 v1.1}
%   \item
%     Line ends sanitized.
%   \end{Version}
%   \begin{Version}{2007/09/09 v1.2}
%   \item
%     Fixes for plain-TeX, wrapper for package \xpackage{keyval} added.
%   \item
%     Catcode section rewritten.
%   \end{Version}
%   \begin{Version}{2007/10/28 v2.0}
%   \item
%     Collection support added (PDF 1.7).
%   \end{Version}
%   \begin{Version}{2007/10/29 v2.1}
%   \item
%     Export of object references by adding new option \xoption{id} and
%     new macros \cs{embedfileifobjectexists} and \cs{embedfilegetobject}.
%   \end{Version}
%   \begin{Version}{2007/11/11 v2.2}
%   \item
%     Use of package \xpackage{pdftexcmds} for \LuaTeX\ support.
%   \end{Version}
%   \begin{Version}{2007/11/25 v2.3}
%   \item
%     Fix in use of \cs{pdf@filesize}, bug introduced in previous version.
%   \end{Version}
%   \begin{Version}{2009/09/25 v2.4}
%   \item
%     Bug fix: If \xpackage{hyperref} is used with option \xoption{unicode},
%     the Unicode encoded file name causes trouble. Therefore \cs{pdfstringdef}
%     is now never used for option \xoption{filespec}, always method
%     \texttt{escape} is applied (Peter Cibulka).
%   \item
%     Bug fix for \xoption{initialfile}.
%   \item
%     Bug fix for file names in \texttt{/EmbeddedFiles}.
%   \item
%     New option \xoption{ucfilespec} for file name support in Unicode
%     (since PDF 1.7).
%   \end{Version}
%   \begin{Version}{2010/03/01 v2.5}
%   \item
%     Compatibility for \iniTeX.
%   \item
%     Package \xpackage{keyval} replaced by packages \xpackage{kvsetkeys}
%     and \xpackage{kvdefinekeys} because of compatibility for \iniTeX.
%   \item
%     TDS location moved from TDS:tex/latex/oberdiek/embedfile.sty to
%     TDS:tex/generic/oberdiek/embedfile.sty.
%   \end{Version}
%   \begin{Version}{2011/04/13 v2.6}
%   \item
%     Docu fixes (thanks Hans-Martin M\"unch).
%   \end{Version}
% \end{History}
%
% \PrintIndex
%
% \Finale
\endinput
